\newpage
\def\thoigian{90}%--Thời gian
\de{Đề số 3}{Chương II. Vectơ và hệ tọa độ trong không gian}

\begin{center}
	\textbf{PHẦN 1 - Câu trắc nghiệm nhiều phương án lựa chọn.}
\end{center}
\setcounter{ex}{0}
\Opensolutionfile{ans}[ans-ABCD]
%Câu 1
\begin{ex}%[2H2N1-2]%[Dự án D - đợt 3 NH 24-25 - Xuan Vy Pham]
	\immini[thm]{Cho hình hộp $ABCD.A'B'C'D'$. Khẳng định nào sau đây là đúng?
	\choice
	{$\overrightarrow{AB}+\overrightarrow{AC}+\overrightarrow{AD}=\overrightarrow{AC'}$}
	{\True $\overrightarrow{AB}+\overrightarrow{AA'}+\overrightarrow{AD}=\overrightarrow{AC'}$}
	{$\overrightarrow{AB}+\overrightarrow{AA'}+\overrightarrow{AD}=\overrightarrow{AC}$}
	{$\overrightarrow{AB}+\overrightarrow{AA'}+\overrightarrow{AD}=\overrightarrow{0}$}
}{
	\begin{tikzpicture}[line join=round, line cap=round,>=stealth,thick,scale=1.2]
		\coordinate (A) at (0,0);
		\coordinate (B) at (-120:1);
		\coordinate (D) at (0:2);
		\coordinate (C) at ($(B)+(D)-(A)$);
		\coordinate (A') at (110:1.5);
		\coordinate (B') at ($(A')+(-120:1)$);
		\coordinate (D') at ($(A')+(0:2)$);
		\coordinate (C') at ($(B')+(D')-(A')$);
		\foreach \i/\j in {A/B,A/D,A/A'}{\draw[dashed] (\i)--(\j);}
		\foreach \i/\j in {B/C,C/D,B/B',C/C',D/D',A'/B',B'/C',C'/D',D'/A'}{\draw (\i)--(\j);}
		\foreach \i/\g in {A/180,B/-90,C/-90,D/0,A'/90,B'/180,C'/0,D'/90}{\draw[fill=black](\i) circle (1.0pt) ($(\i)+(\g:3mm)$) node{$\i$};}
	\end{tikzpicture}
}
\loigiai{
	Theo quy tắc hình hộp ta có $\overrightarrow{AB}+\overrightarrow{AA'}+\overrightarrow{AD}=\overrightarrow{AC'}$.
}
\end{ex}
%Câu 2
\begin{ex}%[2H2N1-3]%[Dự án D - đợt 3 NH 24-25 - Xuan Vy Pham]
	\immini[thm]{Cho hình lập phương $ABCD.A'B'C'D'$ có cạnh bằng $2$. Độ dài của véctơ $\overrightarrow{u}=\overrightarrow{A'C'}-\overrightarrow{A'A}$ bằng
	\choice
	{$\sqrt{3}$}
	{$2\sqrt{2}$}
	{$2\sqrt{6}$}
	{\True $2\sqrt{3}$}}{\begin{tikzpicture}[scale=0.6, font=\footnotesize,line join=round, line cap=round, >=stealth]
		\path 
		(0,0) coordinate (A)
		++(-130:3) coordinate (B)
		++(0:4) coordinate (C)
		($(A)+(C)-(B)$) coordinate (D)
		($(A)!1/2!(C)$) coordinate (O)
		;
		\foreach \i in {A,B,C,D}{
			\coordinate (\i') at ($(\i)+(0,4)$);
		}
		\draw (A')--(B')--(C')--(D')--cycle;
		\draw (B)--(B') (C)--(C') (D)--(D')  (B)--(C)--(D);
		\draw[dashed,thin](B)--(A)--(A') (A)--(D);
		\foreach \i/\g in {A'/90,B'/90,C'/90,D'/90,A/-90,B/-90,C/-90,D/-90}
		\fill[black] (\i) circle(1pt)+(\g:5mm)node[scale=1]{$\i$};
\end{tikzpicture}}
\loigiai{
	\begin{center}
		\begin{tikzpicture}[scale=0.6, font=\footnotesize,line join=round, line cap=round, >=stealth]
			\path 
			(0,0) coordinate (A)
			++(-130:3) coordinate (B)
			++(0:4) coordinate (C)
			($(A)+(C)-(B)$) coordinate (D)
			($(A)!1/2!(C)$) coordinate (O)
			;
			\foreach \i in {A,B,C,D}{
				\coordinate (\i') at ($(\i)+(0,4)$);
			}
			\draw (A')--(B')--(C')--(D')--cycle;
			\draw (B)--(B') (C)--(C') (D)--(D')  (B)--(C)--(D) (A')--(C');
			\draw[dashed,thin](B)--(A)--(A') (A)--(D) (A)--(C');
			\foreach \i/\g in {A'/90,B'/90,C'/90,D'/90,A/-90,B/-90,C/-90,D/-90}
			\fill[black] (\i) circle(1pt)+(\g:5mm)node[scale=1]{$\i$};
		\end{tikzpicture}
	\end{center}
	Vì $A'C'$ là đường chéo của hình vuông $A'B'C'D'$ có cạnh bằng $2$ nên $A'C'=2\sqrt{2}$.\\
	Ta có $\left|\overrightarrow{u}\right|=\left|\overrightarrow{A'C'}-\overrightarrow{A'A}\right|=\left|\overrightarrow{AC'} \right|=AC'=\sqrt{AA'^2+A'C'^2}=\sqrt{2^2+\left(2\sqrt{2}\right)^2}=2\sqrt{3}$.}
\end{ex}
%Câu 3
\begin{ex}%[2H2H1-3]%[Dự án D - đợt 3 NH 24-25 - Xuan Vy Pham]
	Trong không gian, cho hai vectơ $\overrightarrow{a}, \overrightarrow{b}$ thỏa mãn $\left|\overrightarrow{a}\right|=4$, $\left|\overrightarrow{b}\right|=5$ và $\overrightarrow{a} \cdot \overrightarrow{b}=6$. Gọi $\alpha$ là góc giữa hai vectơ $2\overrightarrow{a}$ và $\overrightarrow{b}$. Khẳng định nào dưới đây đúng?
	\choice
	{$\cos \alpha=-\dfrac{3}{5}$}
	{$\cos \alpha=\dfrac{3}{20}$}
	{$\cos \alpha=\dfrac{3}{5}$}
	{\True $\cos \alpha=\dfrac{3}{10}$}
\loigiai{
	Ta có 
	\allowdisplaybreaks
	\begin{eqnarray*}
		\cos\alpha&=&\cos  \left(2\overrightarrow{a},\overrightarrow{b}\right)
		\\
		&=&\dfrac{2\overrightarrow{a}\cdot\overrightarrow{b}}{\left|2\overrightarrow{a}\right|\cdot\left|\overrightarrow{b}\right|}
		\\
		&=&\dfrac{2\cdot 6}{2\cdot 4\cdot 5}
		\\
		&=&\dfrac{3}{10}.
	\end{eqnarray*}	
}
\end{ex}
%Câu 4
\begin{ex}%[2H2H1-3]%[Dự án D - đợt 3 NH 24-25 - Xuan Vy Pham]
	\immini[thm]{Cho hình lập phương $A B C D \cdot A' B' C' D'$ có cạnh $a$. Tích vô hướng $\overrightarrow{A' C} \cdot \overrightarrow{B D}$ bằng
	\choice
	{$6a^2$}
	{$\sqrt{6}a^2$}
	{\True $0$}
	{$\sqrt{3}a^2$}}{
	\begin{tikzpicture}[scale=1, font=\footnotesize, line join=round, line cap=round, >=stealth,scale=0.7]
		\def\a{3}
		\def\b{2}
		\def\h{3}
		\path 	(0:0) coordinate (A)
		++(0:\a) coordinate (D)
		++(-130:\b) coordinate (C)
		($(A)+(C)-(D)$) coordinate (B)
		($(A)+(90:\h)$) coordinate (A')
		($(B)+(90:\h)$) coordinate (B')
		($(C)+(90:\h)$) coordinate (C')
		($(D)+(90:\h)$) coordinate (D');
		\draw[dashed,thick] 	(B)--(A)--(D)	(A)--(A');
		\draw[thick] 	(C)--(C') 	(D)--(D') 	(B)--(B')	(C)--(C')
		(B)--(C)--(D) 
		(A')--(B')--(C')--(D')--cycle;
		\foreach \x/\g in {A/180,B/180,C/0,D/0,A'/180,B'/180,C'/0,D'/0}
		\fill[black] 	(\x) circle (1pt)
		($(\g:4mm)+(\x)$) node {$\x$};	
\end{tikzpicture}}
\loigiai{
		\begin{center}
		\begin{tikzpicture}[scale=1, font=\footnotesize, line join=round, line cap=round, >=stealth,scale=0.7]
			\def\a{3}
			\def\b{2}
			\def\h{3}
			\path 	(0:0) coordinate (A)
			++(0:\a) coordinate (D)
			++(-130:\b) coordinate (C)
			($(A)+(C)-(D)$) coordinate (B)
			($(A)+(90:\h)$) coordinate (A')
			($(B)+(90:\h)$) coordinate (B')
			($(C)+(90:\h)$) coordinate (C')
			($(D)+(90:\h)$) coordinate (D');
			\draw[dashed,thick] 	(B)--(A)--(D)	(A)--(A') (A)--(C) (B)--(D);
			\draw[thick] 	(C)--(C') 	(D)--(D') 	(B)--(B')	(C)--(C')
			(B)--(C)--(D) 
			(A')--(B')--(C')--(D')--cycle;
			\foreach \x/\g in {A/180,B/180,C/0,D/0,A'/180,B'/180,C'/0,D'/0}
			\fill[black] 	(\x) circle (1pt)
			($(\g:4mm)+(\x)$) node {$\x$};	
		\end{tikzpicture}
	\end{center}
	Vì $ABCD$ là hình vuông nên $AC \perp BD$ hay $\overrightarrow{AC} \cdot \overrightarrow{BD}=0$.\\
	Vì $A'A \perp (ABCD)$ nên $A'A \perp BD$ hay $\overrightarrow{A'A} \cdot \overrightarrow{BD}=0$.\\
	Ta có 
	\[\overrightarrow{A' C} \cdot \overrightarrow{B D}=\left(\overrightarrow{A'A}+\overrightarrow{AC} \right)\cdot \overrightarrow{BD}=\overrightarrow{A'A}\cdot \overrightarrow{BD}+\overrightarrow{AC} \cdot \overrightarrow{BD}=0.  \]
	
}
\end{ex}
%Câu 5
\begin{ex}%[2H2N2-3]%[Dự án D - đợt 3 NH 24-25 - Xuan Vy Pham]
	Trong không gian $Oxyz$, cho vectơ $\overrightarrow{a}$ thỏa mãn $\overrightarrow{a}=2\overrightarrow{i}+\overrightarrow{k}-3\overrightarrow{j}$. Tọa độ của vectơ $\overrightarrow{a}$ là
\choice
{$(2;1;-3)$}
{\True $(2;-3;1)$}
{$(1;2;-3)$}
{$(1;-3;-2)$}
\loigiai{Ta có $\overrightarrow{a}=2\overrightarrow{i}+\overrightarrow{k}-3\overrightarrow{j}=2\overrightarrow{i}-3\overrightarrow{j}+\overrightarrow{k}\Rightarrow \overrightarrow{a}=(2;-3;1)$.}
\end{ex}
%Câu 6
\begin{ex}%[2H2H2-2]%[Dự án D - đợt 3 NH 24-25 - Xuan Vy Pham]
Trong không gian $Oxyz$, cho hai điểm $A(1;2;-4)$ và $\overrightarrow{AB}=(-4;0;6)$. Tọa độ của điểm $B$ là
\choice
{$(-1;6;-6)$}
{$(5;2;-10)$}
{\True $(-3;2;2)$}
{$(0;4;-5)$}
\loigiai{ Gọi $B(x_B; y_B; z_B)$. Khi đó $\overrightarrow{AB}=(x_B-1; y_B-2; z_B+4)$.\\
Mà $\overrightarrow{AB}= (-4;0;6)$ nên
$\heva{&x_B-1=-4\\&y_B-2=0\\&z_B+4=6}\Leftrightarrow\heva{&x_B=-3\\&y_B=2\\&z_B=2.}$\\
Vậy $B(-3;2;2)$.
}
\end{ex}
%Câu 7
\begin{ex}%[2H2N2-2]%[Dự án D - đợt 3 NH 24-25 - Xuan Vy Pham]
Trong không gian $Oxyz$, hình chiếu vuông góc của điểm $A(-5;2;8)$ trên trục $Oy$ là
\choice
{$(-5;0;0)$}
{\True $(0;2;0)$}
{$(-5;0;8)$}
{$(0;0;8)$}
\loigiai{ Hình chiếu vuông góc của điểm $A(-5;2;8)$ trên trục $Oy$ là $(0;2;0)$.
}
\end{ex}
%Câu 8
\begin{ex}%[2H5H2-3]%[Dự án D - đợt 3 NH 24-25 - Xuan Vy Pham]
	Trong không gian $Oxyz$, cho hai điểm $A\left(1;-1;2\right)$ và $B\left(2;0;-1\right)$. Tọa độ của điểm $C$ để tứ giác $OABC$ là hình bình hành là
	\choice
	{\True $(1; 1; -3)$}
	{$(1; 1; 3)$}
	{$(-1; -1; 3)$}
	{$(1; -1; 3)$}
	\loigiai{
		\begin{center}
			\begin{tikzpicture}[scale=1, font=\footnotesize, line join=round, line cap=round, >=stealth,scale=0.7]
				\def\a{4.5}
				\def\b{3}
				\path 	(0:0) coordinate (O)
				++(0:\a) coordinate (A)
				(O)++(-130:\b) coordinate (C)
				($(C)+(A)-(O)$) coordinate (B)
				;
				\draw[thick] 	(O)--(A)--(B)--(C)--(O);
				\foreach \x/\g in {A/0,B/0,C/180,O/180}
				\fill[black] 	(\x) circle (1pt)
				($(\g:4mm)+(\x)$) node {$\x$};	
			\end{tikzpicture}
		\end{center}
		Gọi $C(x;y;z)$. 
		\\
		Ta có $\overrightarrow{OC}=(x;y;z)$ và $\overrightarrow{AB}=(1;1;-3)$. \\
		Do $OABC$ là hình bình hành nên $\overrightarrow{OC}=\overrightarrow{AB}$ khi đó $x=1, y=1$ và $z=-3$.\\
		 Vậy $C(1; 1; -3)$.
	}
\end{ex}
%Câu 9
\begin{ex}%[2H2N2-3]%[Dự án D - đợt 3 NH 24-25 - Xuan Vy Pham]
	Trong không gian $Oxyz$, cho $A\left(-2;3;-4\right)$ và $B\left(4;-3;3\right)$. Khoảng cách giữa $A$ và $B$ bằng
\choice
{\True $11$}
{$8$}
{$7$}
{$9$}
\loigiai{
	Ta có $AB = \left|\overrightarrow{AB}\right| = \sqrt{(4+2)^2 + (-3-3)^2 + (3+4)^2} = 11$.
}
\end{ex}
%Câu 10
\begin{ex}%[2H2N2-2]%[Dự án D - đợt 3 NH 24-25 - Xuan Vy Pham]
Trong không gian $Oxyz$, cho hai điểm $A\left(3;-2;3\right)$ và $B\left(-1;2;5\right)$. Tọa độ trung điểm $I$ của đoạn thẳng $AB$ là
\choice
{$\left(-2;2;1\right)$}
{\True $\left(1;0;4\right)$}
{$\left(2,0,8\right)$}
{$\left(2;-2;-1\right)$}
\loigiai{
	Toạ độ trung điểm $I$ của $AB$ là
	$I\left(\dfrac{x_A + x_B}{2}; \dfrac{y_A + y_B}{2} ; \dfrac{z_A + z_B}{2}\right)$ hay $I\left(1;0;4\right)$.
}
\end{ex}
%Câu 11
\begin{ex}%[2H2H2-2]%[Dự án D - đợt 3 NH 24-25 - Xuan Vy Pham]
Trong không gian $Oxyz$, cho điểm $M\left(-2;3;4\right)$. Khoảng cách từ điểm $M$ đến trục $Ox$ là
\choice
{$2$}
{$3$}
{$4$}
{\True $5$}
\loigiai{
	Hình chiếu vuông góc của $M\left(-2;3;4\right)$ trên trục $Ox$ là $N\left(-2;0;0\right)$.\\
	Khoảng cách từ điểm $M$ đến trục $Ox$ là độ dài đoạn
	$$MN = \left|\overrightarrow{MN}\right| =  \sqrt{(-2+2)^2 + (0-3)^2 + (0-4)^2} = 5.$$
	}
\end{ex}
%Câu 12
\begin{ex}%[2H2H2-4]%[Dự án D - đợt 3 NH 24-25 - Xuan Vy Pham]
	Trong không gian $O xyz$, cho $A\left (3 ; 2 ; 1\right )$, $B\left (-1 ; 3 ; 2\right )$, $C\left (2 ; 4 ;-3\right )$. Tích vô hướng $\overrightarrow{A B} \cdot \overrightarrow{A C}$ bằng
\choice
{\True $2$}
{$-2$}
{$6$}
{$-6$}
\loigiai{
	Ta có $\overrightarrow{AB} = \left (-4;1;1 \right )$, $\overrightarrow{AC} = \left (-1;2;-4\right )$. \\
	Suy ra $\overrightarrow{AB} \cdot \overrightarrow{AC} = 4+2-4 = 2$.
}
\end{ex}
\Closesolutionfile{ans}
%\indapan{6}{ans-ABCD}
%\cauds
\begin{center}
	\textbf{PHẦN 2 - Câu trắc nghiệm đúng sai. Trong mỗi ý a, b, c, d ở mỗi câu, thí sinh chọn đúng hoặc sai}
\end{center}
\setcounter{ex}{0}
\Opensolutionfile{ans}[ans-DS]
%Câu 1
\begin{ex}%[2H2H2-3]%[Dự án D - đợt 3 NH 24-25 - Xuan Vy Pham]
	\immini{
	Cho hình chóp $S.ABCD$ có đáy $ABCD$ là hình vuông cạnh $2$, $SA$ vuông góc với đáy và $SA=5$. \\
	Chọn hệ trục $Oxyz$ sao cho gốc $O$ trùng với $A$; các điểm $B$, $D$, $S$ lần lượt thuộc các tia $Ox$, $Oy$, $Oz$ (tham khảo hình minh họa bên).
	\choiceTF
	{$\overrightarrow{AD}=2\overrightarrow{i}$}
	{\True $\overrightarrow{SA}=-5\overrightarrow{k}$}
	{\True $C(2; 2; 0)$}
	{$\overrightarrow{SC}=2\overrightarrow{i}+2\overrightarrow{j}+5\overrightarrow{k}$}
}
{
	\begin{tikzpicture}[scale=0.85, font=\footnotesize, line join=round, line cap=round, >=stealth]
		\def\a{4}
		\path 
		(0:0) coordinate (B)
		(0:3) coordinate (C)
		($(C)+(45:\a/2)$) coordinate (D)
		($(B)+(45:\a/2)$) coordinate (A)
		($(B)+(70:\a)$) coordinate (S)
		;
		\draw[dashed] 	(A)--(S);
		\draw[dashed] 	(A)--(B);
		\draw[dashed] 	(A)--(D);
		\draw (S)--(B)--(C)--(D)--(S)--(C)	
		pic[draw,angle radius=1.5mm]{right angle=D--A--S}
		pic[draw,angle radius=1.5mm]{right angle=B--A--D}
		pic[draw,angle radius=1.5mm]{right angle=S--A--B}
		;	
		\draw[->] (S)--++(0,1) node[above]{$z$};
		\draw[->] (D)--++(1,0) node[right]{$y$};
		\draw[->] (B)--($(A)!1.4!(B)$) node[left]{$x$};
		\foreach \x/\g in {A/180,B/160,C/-70,D/60,S/30}
		\draw[fill=black] 	(\x) circle (.5pt)
		($(\g:.4)+(\x)$) node {$\x$};	
	\end{tikzpicture}
}
\loigiai{Với hệ trục $Oxyz$ như hình vẽ, ta có $A(0;0;0)$, $B(2;0;0)$, $D(0;2;0)$, $S(0;0;5)$. 
	\begin{itemchoice}
		\itemch 
		Ta có $\overrightarrow{AD}=(0;2;0)$ nên $\overrightarrow{AD}=2\overrightarrow{j}$.
		\itemch 
		Ta có $\overrightarrow{SA}=(0;0;-5)$ nên $\overrightarrow{SA}=-5\overrightarrow{k}$.
		\itemch 
		Do tứ giác $ABCD$ là hình bình hành nên $\overrightarrow{BC}=\overrightarrow{AD}$, suy ra $C(2;2;0)$.
		\itemch 
		Ta có $\overrightarrow{SC}=(2;2;-5)$ nên $\overrightarrow{SC}=2\overrightarrow{i}+2\overrightarrow{j}-5\overrightarrow{k}$.
	\end{itemchoice}
}
\end{ex}
%Câu 2
\begin{ex}%[2H2V2-6]%[Dự án D - đợt 3 NH 24-25 - Xuan Vy Pham]
	Trên sân thể dục, thầy giáo dựng hai chiếc cột $AB$, $CD$ vuông góc với mặt sân, chiều cao của mỗi cột lần lượt là $3$ m; $2$ m và căng sợi dây $BD$. Xét hệ trục tọa độ $Oxyz$ sao cho mặt phẳng $(Oxy)$ trùng với mặt sân, trục $Oz$ hướng thẳng đứng lên trời.
\begin{center}
	\begin{tikzpicture}[line join = round, line cap = round,>=stealth,font=\footnotesize,scale=1,declare function={a=5.5;b=2.5;goc=70;}]
		\path (0,0) coordinate (E)
		(a,0) coordinate (F)
		(goc:b) coordinate (H)
		($(F)-(E)+(H)$) coordinate (G)
		(40:2.5) coordinate (A)
		(10:5) coordinate (C)
		(10:2.8) coordinate (M)
		($(A)+(0,3)$) coordinate (B)
		($(C)+(0,2)$) coordinate (D)
		;
		\fill[green!20] (E)--(F)--(G)--(H)--cycle;
		\draw[double] (E)--(F)--(G)--(H)--cycle;
		\draw[thick] (A)--(B) (C)--(D);
		\draw (B)--(M)--(D)--cycle
		(A)node[left]{$A(8;5;0)$} (C)node[below]{$C(3;2;0)$}
		;
		\fill (A) circle (1.5pt) (C) circle (1.5pt);
		\foreach \x/\gm in {M/-90,B/90,D/90} \fill (\x) circle (1pt) ($(\x)+(\gm:3.5mm)$)node{$\x$};
	\end{tikzpicture}	
\end{center}
\choiceTF
{\True Tọa độ hai đầu cột là $B(8;5;3)$, $D(3;2;2)$}
{\True Sợi dây $BD$ có chiều dài là $\sqrt{35}$ m}
{\True Diện tích hình thang $ABCD$ bằng $ \dfrac{5\sqrt{34}}{2}\ \mathrm{~m}^2$}
{Một học sinh đang ở vị trí $M\left(\dfrac{11}{2};\dfrac{7}{2};0\right)$ thì tổng khoảng cách từ điểm $M$ đến hai đầu cột là ngắn nhất}
\loigiai{
	\begin{itemchoice}
		\itemch \textbf{Đúng}. Chiều cao cột $AB=3$ m nên tọa độ điểm $B(8;5;3)$, chiều cao cột $CD=2$ m nên tọa độ điểm $D(3;2;2)$. 
		\itemch \textbf{Đúng}. Ta có $\overrightarrow{BD}=(-5;-3;-1)\Rightarrow BD=\sqrt{(-5)^2+(-3)^2+(-1)^2}=\sqrt{35}$ m. 
		\itemch \textbf{Đúng}. Ta có $AB=3$ m, $CD=2$ m, $\overrightarrow{AC}=(-5;-3;0)\Rightarrow AC=\sqrt{(-5)^2+(-3)^2}=\sqrt{34}$ m.\\
		Vậy diện tích hình thang $ABCD$ là $$S_{ABCD}=\dfrac{(AB+CD)\cdot AC}{2}=\dfrac{(3+2)\cdot\sqrt{34}}{2}=\dfrac{5}{2}\sqrt{34} \; (\text{m}^2).$$
		\itemch \textbf{Sai}.
		{
		\begin{center}
				\begin{tikzpicture}[line join = round, line cap = round,>=stealth,font=\footnotesize,scale=1,declare function={a=5.5;b=2.5;goc=70;}]
				\path (0,0) coordinate (E)
				(a,0) coordinate (F)
				(goc:b) coordinate (H)
				($(F)-(E)+(H)$) coordinate (G)
				(40:2.5) coordinate (A)
				(10:5) coordinate (C)
				(10:2.8) coordinate (M)
				($(A)+(0,3)$) coordinate (B)
				($(C)+(0,2)$) coordinate (D)
				($(C)+(0,-2)$) coordinate (D')
				(intersection of B--D' and M--D) coordinate (M')
				(intersection of E--F and M'--D') coordinate (M'')
				(intersection of E--F and C--D') coordinate (D'')
				;
				%			\fill[green!20] (E)--(F)--(G)--(H)--cycle;
				\draw[double] (E)--(F)--(G)--(H)--cycle;
				\draw[thick] (A)--(B) (C)--(D)node[pos=0.5,rotate=-20]{$/$};
				\draw (B)--(D)--cycle
				(A) (C)
				(B)--(M') (M'')--(D') (D'')--(D')node[pos=0.3,rotate=-20]{$/$}
				;
				\draw[dashed] (M')node[shift={(-120:4mm)}]{$M$}--(M'')
				(C)--(D'')
				;
				\fill (A) circle (1.5pt) (C) circle (1.5pt)
				(M') circle (1.5pt)
				;
				\foreach \x/\gm in {B/90,D/90,D'/-90,A/180,C/20} \fill (\x) circle (1pt) ($(\x)+(\gm:3.5mm)$)node{$\x$};
			\end{tikzpicture}	
		\end{center}	
		}
		Ta xét $M\in (Oxy)$.\\
		Vì $z_B>0$, $z_D>0$ nên $B$ và $D$ nằm về cùng một phía so với mặt phẳng $(Oxy)$.\\
		Gọi $D'(3;2;-2)$ là điểm đối xứng của $D$ qua $(Oxy)$. Suy ra $MD=MD'$\\
		Ta có $MB+MD=MB+MD'\geqslant BD'$,\\
		Dấu \lq\lq$=$\rq\rq\ xảy ra khi và chỉ khi $M$, $B$, $D'$ thẳng hàng.\\
		Gọi $M\left(x_M;y_M;0\right)\in (Oxy)$\\
		$\Rightarrow\overrightarrow{BM}=(x_M-8;y_M-5;-3)$, $\overrightarrow{D'B}=\left(5;3;5\right)$.	\\
		Vì $M$, $B$, $D'$ thẳng hàng nên $\overrightarrow{BM}$ và $\overrightarrow{D'B}$ cùng phương suy ra \[\dfrac{x_M-8}{5}=\dfrac{y_M-5}{3}=\dfrac{-3}{5}\Rightarrow\heva{& x_M=5 \\ & y_M=\dfrac{16}{5}}\Rightarrow M\left(5;\dfrac{16}{5};0\right).\]
	\end{itemchoice}
}
\end{ex}
\Closesolutionfile{ans}
\begin{center}
	\textbf{PHẦN 3 - Câu trắc nghiệm trả lời ngắn}
\end{center}
\setcounter{ex}{0}
%Câu 1
\begin{ex}%[2H2H2-3]%[Dự án D - đợt 3 NH 24-25 - Xuan Vy Pham]
	Trong không gian $Oxyz$, cho hai vectơ $\overrightarrow{a}=(-4 ; 2 ; 5)$ và $\overrightarrow{b}=(3 m+2 ; 2 ; 6-n)$. Biết rằng hai vectơ $\overrightarrow{a}, \overrightarrow{b}$ bằng nhau. Tính giá trị biểu thức $Q=6 m+2 n$.
\shortans{$ -10$}
\loigiai
{
Vì $\overrightarrow{a}=\overrightarrow{b}$ nên $\heva{& 3m+2=-4 \\ &6-n=5 }\Leftrightarrow \heva{& m=-2 \\ &n=1. }$\\
	Vậy $Q=-10$.
}
\end{ex}
%Câu 2
\begin{ex}%[2H2V2-3]%[Dự án D - đợt 3 NH 24-25 - Xuan Vy Pham]
	\immini[thm]
{
	Ở một sân bay, vị trí của máy bay được xác định bởi điểm $M$ trong không gian $Oxyz$ như hình vẽ. Gọi $H$ là hình chiếu của $M$ lên mặt phẳng $(Oxy)$. Biết $OM=79$, $(\overrightarrow{OH}, \overrightarrow{OM})=50^{\circ}$ và $(\overrightarrow{i},\overrightarrow{OH})=68^{\circ}$. Nếu $M(a;b;c)$ thì tổng $a+b+c$ có phần nguyên bằng bao nhiêu?
	\shortans{$136$}	
}
{
	\begin{tikzpicture}[scale=0.9, font=\footnotesize, line join=round, line cap=round, >=stealth]
		\path 
		(0,0) coordinate (O)++(0:4) coordinate (B)
		(O)++(90:2) coordinate (C)
		(O)++(-140:2) coordinate (A)		
		($(A)+(B)-(O)$) coordinate (H)
		($(H)+(C)-(O)$) coordinate (M)
		($(O)!1.3!(A)$) coordinate (x) node[below]{$x$}
		($(O)!1.2!(B)$) coordinate (y) node[below]{$y$}
		($(O)!1.3!(C)$) coordinate (z) node[right]{$z$}
		($(O)!0.3!(A)$) coordinate (i) node[below,scale =0.7]{$\overrightarrow{i}$}
		($(O)!0.3!(B)$) coordinate (j) node[below,scale =0.7]{$\overrightarrow{j}$}
		($(O)!0.3!(C)$) coordinate (k) node[right,scale =0.7]{$\overrightarrow{k}$}
		;
		\draw (O)--(A) (O)--(B) (O)--(C) (O)--(M);
		\draw[dashed] (C)--(M)--(H)--(O) (A)--(H)--(B)
		;
		\pic[draw,thin,angle radius=2mm] {right angle = M--C--O}
		pic[draw,thin,angle radius=2mm] {right angle = H--B--O}
		pic[draw,thin,angle radius=2mm] {right angle = H--A--O}
		;
		\draw[->] (A)--(x); 
		\draw[->] (B)--(y);
		\draw[->] (C)--(z);
		\draw[->,red] (O)--(i); 
		\draw[->,red] (O)--(j);
		\draw[->,red] (O)--(k);
		\foreach \x/\g in {A/130,H/-90,B/80,C/170,O/-90,M/45}
		\fill (\x) circle (1pt)
		+(\g:3mm) node[scale = 0.8]{$\x$};
	\end{tikzpicture}	
}
\loigiai{
	Tam giác $OMH$ vuông tại $H$, $OM = 79$; $\widehat{MOH} = 50^\circ$ nên ta có 
	$$OH = OM\cdot \cos 50^\circ=70 \cdot\cos 50^\circ ; \; OC = MH = OM \cdot \sin 50^\circ=70 \cdot \sin 50^\circ.$$
	Tam giác $OAH$ vuông tại $A$, $\widehat{AOH} = 68^\circ$ nên ta có 
	$$OA = OH\cdot \cos 68^\circ=70 \cdot\cos 50^\circ\cdot\cos 68^\circ.$$
	$$OB = AH = OH\cdot \sin 68^\circ=70 \cdot\cos 50^\circ\cdot\sin 68^\circ.$$
	Ta có
	\begin{eqnarray*}
		\overrightarrow{OM} & = & \overrightarrow{OC} + \overrightarrow{OH} \\
		& = & \overrightarrow{OC} + \overrightarrow{OA}+\overrightarrow{OB} \\
		& = & \left| OA\right|\overrightarrow{i}+\left| OB\right|\overrightarrow{j}+\left| OC\right|\overrightarrow{k}.
	\end{eqnarray*}
	Suy ra
	\begin{align*}
	a+b+c&=\left| OC\right|+\left| OA\right|+\left| OB\right|\\
	&=70 \cdot \sin 50^\circ+ 70 \cdot\cos 50^\circ\cdot\cos 68^\circ+70 \cdot\cos 50^\circ\cdot\sin 68^\circ\\
	&\approx 136.
	\end{align*}
}
\end{ex}
%Câu 3...........................
\begin{ex}%[2H2H2-2]%[Dự án D - đợt 3 NH 24-25 - Xuan Vy Pham]
	Trong không gian $Oxyz$, cho tam giác $ABC$ biết $A(3;1;5)$, $B(-1;-5;2)$ và $C(0;1;0)$. Gọi $M$ là điểm trên đoạn $BC$ sao cho $MC=2MB$. Tính độ dài đoạn thẳng $AM$ (kết quả làm tròn đến hàng phần mười).
\shortans{$6{,}5$}
\loigiai{
	\immini{Gọi $M(a;b;c)$ là điểm nằm trên $BC$ thỏa $MC=2MB$.\\
		Theo hình vẽ, ta thấy  $\overrightarrow{MC}=-2\overrightarrow{MB}$.
		\[\heva{&0-a=-2\cdot (-1-a)\\&1-b=-2\cdot (-5-b)\\&0-c=-2\cdot (2-c)}\Leftrightarrow \heva{&a=-\dfrac{2}{3}\\&b=-3\\&c=\dfrac{4}{3}.}\]
		Suy ra, tọa độ điểm $M\left(-\dfrac{2}{3};-3;\dfrac{4}{3}\right)$.
	}{\begin{tikzpicture}[scale=.8]
			\coordinate (A) at (0,3);
			\coordinate (B) at (-2,0);
			\coordinate (C) at (3,0);
			\draw(A)--(B)--(C)--cycle;
			%6) ĐIỂM M THUỘC ĐƯỜNG AB TỈ SỐ BẰNG 3
			\coordinate (M) at ($(B)!1/3!(C)$);
			\draw(A)--(M);
			\foreach \i/\g in {A/90,B/-90,C/-90,M/-90}{\draw[fill=black](\i) circle (1pt) ($(\i)+(\g:3mm)$) node[scale=1]{$\i$};}
	\end{tikzpicture}
}
\noindent Độ dài đoạn $AM=\sqrt{\left(-\dfrac{2}{3}-3\right)^2+(-3-1)^2+\left(\dfrac{4}{3}-5\right)^2}\approx 6{,}5$.	
}
\end{ex}
%Câu 4
\begin{ex}%[2H2V2-6]%[Dự án D - đợt 3 NH 24-25 - Xuan Vy Pham]
	\immini[thm]
{
	Ông Năm muốn xây một cái bể chứa nước mưa, không có nắp dạng hình hộp chữ nhật $ABCD.A'B'C'D'$ với đáy là hình vuông có thể tích là $32\mathrm{~m}^3$. Bể được gắn vào hệ trục tọa độ $Oxyz$ (\textit{được mô hình hóa như hình vẽ bên}) với điểm $A$ trùng gốc tọa độ $O$ và điểm $D'(a;b;c)$. Với chi phí xây dựng là $600\ 000$ đồng$/\mathrm{m}^2$, hãy tính giá trị biểu thức $\mathscr{D}=a+b+c$ để bể được xây dựng với chi phí tiết kiệm nhất.
	\shortans{$6$}	
}
{
	\begin{tikzpicture}[scale=1, font=\footnotesize, line join=round, line cap=round, >=stealth,declare function={ab=1.5;ad=4;cao=2.2;gocB=35;}]
		\path (0,0) coordinate (B)
		(ad,0) coordinate (C)
		(gocB:ab) coordinate (A)
		($(C)-(B)+(A)$) coordinate (D)
		($(A)+(90:cao)$) coordinate (A')
		($(B)+(90:cao)$) coordinate (B')
		($(C)+(90:cao)$) coordinate (C')
		($(D)+(90:cao)$) coordinate (D')
		($(A)!1.45!(B)$) coordinate (x)
		($(A)!1.3!(D)$) coordinate (y)
		($(A)!1.4!(A')$) coordinate (z)
		;
		\draw[dashed] (A')--(A)--(B) (A)--(D)
		;
		\draw (A')--(B')--(C')--(D')--cycle 
		(D)--(D')--(C')--(C) (C')--(B')--(B) (D)--(C)--(B)
		;
		\draw[->] (D)--(y)node[shift={(90:3.5mm)}]{$y$};
		\draw[->] (A')--(z)node[shift={(180:3.5mm)}]{$z$};
		\draw[->] (B)--(x)node[shift={(130:3.5mm)}]{$x$};
		\fill (D')node[shift={(90:3.5mm)}]{$D'(a;b;c)$} circle (1.5pt);
		\foreach \x/\gm in {A'/50,B'/90,C'/90,A/160,B/-90,D/40,C/-90} \fill (\x) circle (1.5pt) ($(\x)+(\gm:3.5mm)$)node{$\x$};
	\end{tikzpicture}		
}
\loigiai{Gắn hệ trục như hình vẽ nên $A\equiv O(0;0;0)$.\\
	Vì $D'\in (yOz)\Rightarrow a=0\Rightarrow D'(0;b;c)$.\\
	Vì đáy $ABCD$ là hình vuông nên $AB=AD=b$. Thể tích khối hộp là $32$ nên ta có \[V_{ABCD.A'B'C'D'}=AB\cdot AD\cdot AA'=b\cdot b\cdot c=b^2\cdot c\]
	Suy ra $b^2c=32\Rightarrow c=\dfrac{32}{b^2}$.\\
	Diện tích toàn phần là \allowdisplaybreaks
	\begin{eqnarray*}
		S_{\text{tp bể}}&=&b^2+4bc\\
		&=&b^2+4b\cdot\dfrac{32}{b^2}\\
		&=&b^2+\dfrac{64}{b}+\dfrac{64}{b}\\
		&\geqslant&3\sqrt[3]{b^2\cdot\dfrac{64}{b}\cdot\dfrac{64}{b}}\\
		&=&48.
	\end{eqnarray*}
	Chi phí ít nhất khi $S_{\text{tp bể}}$ min $\Leftrightarrow b^2=\dfrac{64}{b}\Leftrightarrow b=4\Rightarrow c=2$.\\
	Vậy $a+b+c=6$.
}
\end{ex}	
\Closesolutionfile{ansKQ}
\begin{center}
	\textbf{PHẦN 4 - Tự luận.}
\end{center}
\setcounter{ex}{0}
%Câu 1
\begin{ex}%[2H2V1-4]%[Dự án D - đợt 3 NH 24-25 - Xuan Vy Pham]
	Cho ba lực $\overrightarrow{F_1}=\overrightarrow{MA}$, $\overrightarrow{F_2}=\overrightarrow{MB}$, $\overrightarrow{F_3}=\overrightarrow{MC}$ cùng tác động vào một xe hàng tại điểm $M$ và xe hàng đứng yên. Cho biết cường độ hai lực $\overrightarrow{F}_1$, $\overrightarrow{F}_2$ đều bằng $25N$ và góc $\widehat{AMB}=60^{\circ}$ (tham khảo hình bên dưới). Tính cường độ lực $\overrightarrow{F}_3$.
\begin{center}
	\begin{tikzpicture}[>=stealth,line join=round,line cap=round,font=\footnotesize,scale=1]
		\filldraw[fill=orange!60, rounded corners] (0,1)--(0,0)--(3,0)--(3,1)--cycle (-0.1,1)--(3.1,1)--(3.1,1.2)--(-0.1,1.2)--cycle;
		\foreach \x/\y in {2.5/-0.2,0.5/-0.2}{
			\begin{scope}[shift={(\x,\y)}]
				% Banh xe
				\path
				(0,0) coordinate (O)
				($(O) + (0:0.3)$) coordinate (A)
				($(O) + (60:0.3)$) coordinate (B)
				($(O) + (120:0.3)$) coordinate (C)
				($(O) + (180:0.3)$) coordinate (A')
				($(O) + (240:0.3)$) coordinate (B')
				($(O) + (300:0.3)$) coordinate (C')
				;
				\filldraw[fill = white, draw = orange!40!brown,line width=0.5mm] (O) circle (0.3);
				\draw[draw = orange!40!brown,line width=0.5mm] (A)--(A') (B)--(B') (C)--(C');
		\end{scope}}
		\path
		(-3,0.5) coordinate (C)
		(6,0.5) coordinate (D)
		(4,2) coordinate (A)
		(4,-1.0) coordinate (B);
		\coordinate (M) at ($(C)!0.5!(D)$);
		\draw[->] (M)--(C);
		\draw[->] (M)--(A);
		\draw[->] (M)--(B);
		\foreach \x/\g in{C/90,M/90,A/90,B/-90}
		\fill[black](\x)circle(1pt) ($(\x)+(\g:3mm)$)node{$\x$};
		\pic["$60^\circ$"{shift={(1pt,3pt)}},draw,angle radius=4mm,angle eccentricity=1.79] {angle = B--M--A};
	\end{tikzpicture}
\end{center}
\loigiai{
	Vì ba lực $\overrightarrow{F_1}=\overrightarrow{MA}$, $\overrightarrow{F_2}=\overrightarrow{MB}$, $\overrightarrow{F_3}=\overrightarrow{MC}$ cùng tác động vào một xe hàng tại điểm $M$ và xe hàng đứng yên nên $\left|\overrightarrow{F}_3\right|=\left|\overrightarrow{F}_1+\overrightarrow{F}_2\right|$.\\
	Ta có $\left|\overrightarrow{F}_1+\overrightarrow{F}_2\right|^2=\left|\overrightarrow{F}_1\right|^2+\left|\overrightarrow{F}_2\right|^2+2\overrightarrow{F}_1\cdot \overrightarrow{F}_2=25^2+25^2+2\cdot 25\cdot 25\cdot \cos 60^\circ=1875$.\\
	Vậy $\left|\overrightarrow{F}_1+\overrightarrow{F}_2\right|=\sqrt{1875}=25\sqrt{3}$N.
}
\end{ex}
%Câu 2
\begin{ex}%[2H2V2-2]%[Dự án D - đợt 3 NH 24-25 - Xuan Vy Pham]
	\immini
{
	Cho hình chóp $S\cdot ABCD$ có đáy $ABCD$ là hình chữ nhật với $AB=3$, $AD=4$, $SA=5$ và $SA\perp(ABCD)$. Xét hệ tọa độ $Oxyz$ với $O$ trùng với $A$ và các tia $Ox$, $Oy$, $Oz$ lần lượt trùng với các tia $AB$, $AD$, $AS$. Biết tọa độ véctơ $\overrightarrow{AM}=\left(2;\dfrac{a}{3}; \dfrac{b}{3}\right)$, với $M$ là điểm thuộc $SC$ sao cho $\overrightarrow{SM}=2\overrightarrow{MC}$. Tính giá trị của $3a+b-3$.
}
{
	\begin{tikzpicture}[line join = round, line cap = round,>=stealth,font=\footnotesize,scale=0.8,declare function={ab=2.5;ad=4.5;gs=90;sa=3;gocA=-130;}]
		\path 
		(0,0) coordinate (A)
		(ad,0) coordinate (D)
		(gocA:ab) coordinate (B)
		($(D)-(A)+(B)$) coordinate (C)
		(gs:sa) coordinate (S)
		($(S)!2/3!(C)$) coordinate (M)
		($(A)!1.2!(B)$) coordinate (x)
		($(A)!1.2!(D)$) coordinate (y)
		($(A)!1.2!(S)$) coordinate (z)
		;
		\draw (S)--(B)--(C)--(D)--(S)--(C);
		\draw[dashed] (S)--(A)--(D) (B)--(A);
		\draw[->,>=stealth,dashed] (A)--(M);
		\draw[->,>=stealth] 
		(B)--(x)node[shift={(-20:3.5mm)}]{$x$};
		\draw[->,>=stealth] (D)--(y)node[shift={(90:3.5mm)}]{$y$};
		\draw[->,>=stealth] (S)--(z)node[shift={(0:3.5mm)}]{$z$}
		;
		\foreach \x/\gm in {A/150,B/180,C/-20,D/60,S/180,M/-90} \fill (\x) circle (1pt) ($(\x)+(\gm:3.5mm)$)node{$\x$};
		\foreach \x/\y/\z in {B/A/D/,S/A/D,A/D/C,A/B/C,D/C/B} \draw pic[draw,angle radius=2.5mm]{right angle=\x--\y--\z};
	\end{tikzpicture}	
}
\loigiai{
	Ta có tọa độ $A(0;0;0)$, $B(3;0;0)$, $C(3;4;0)$, $D(0;4;0)$, $S(0;0;5)$.\\
	Gọi $M(x_M;y_M;z_M)\Rightarrow\overrightarrow{SC}=\left(3;4;-5\right)$, $\overrightarrow{SM}=\left(x_M;y_M;z_M-5\right)$.\\
	Vì $\overrightarrow{SM}=2\overrightarrow{MC}\Rightarrow\overrightarrow{SM}=\dfrac{2}{3}\overrightarrow{SC}\Leftrightarrow\heva{& x_M=\dfrac{2}{3}\cdot 3 \\ & y_M=\dfrac{2}{3}\cdot 4 \\ & z_M-5=\dfrac{2}{3}\cdot(-5)}\Leftrightarrow\heva{& x_M=2 \\ & y_M=\dfrac{8}{3} \\ & z_M=\dfrac{5}{3}}\Rightarrow M\left(2;\dfrac{8}{3};\dfrac{5}{3}\right)$.\\
	Vậy $\overrightarrow{AM}=\left(2;\dfrac{8}{3};\dfrac{5}{3}\right)=\left(2;\dfrac{a}{3};\dfrac{b}{3}\right)\Rightarrow\heva{& a=8 \\ & b=5.}$\\
	Vậy $3a+b-3=3\cdot 8+5-3=26$.
}
\end{ex}
%Câu 3
\begin{ex}%[2H2V2-2]%[Dự án D - đợt 3 NH 24-25 - Xuan Vy Pham]
	Trong không gian $Oxy$z, cho bốn điểm $A\left(-6;4;-1\right)$, $B\left(1;1;2\right)$, $C\left(-3;2;4\right)$ và $D\left(-1;-1;0\right)$. Biết tọa độ điểm $M\left(a;b;c\right)$ thì biểu thức $MA^2+MB^2+MC^2+2MD^2$ đạt giá trị nhỏ nhất. Tính giá trị của biểu thức $100a+10b+c$.
\loigiai{
	Xét điểm $I\left(a;b;c\right)$ thỏa mãn $\overrightarrow{IA}+\overrightarrow{IB}+\overrightarrow{IC}+2\overrightarrow{ID}=\overrightarrow{0}$.\\
	Ta có $\heva{&-6-a+1-a-3-a-1-a=0\\
		&4-b+1-b+2-b-1-b=0\\
		&-1-c+2-c+4-c-c=0
	}$
	$\Rightarrow \heva{&a=-\dfrac{9}{4}\\
		&b=\dfrac{3}{2}\\
		&c=\dfrac{5}{4}.
	}$\\
	Vậy $I\left(-\dfrac{9}{4};\dfrac{3}{2};\dfrac{5}{4}\right)$.\\
	Ta có:
	$MA^2+MB^2+MC^2+MD^2=\left(\overrightarrow{MI}+\overrightarrow{IA} \right)^2+\left(\overrightarrow{MI}+\overrightarrow{IB} \right)^2+\left(\overrightarrow{MI}+\overrightarrow{IC} \right)^2+2\left(\overrightarrow{MI}+\overrightarrow{ID}\right)^2$.\\
	Suy ra $MA^2+MB^2+MC^2+MD^2=5\overrightarrow{MI}^2+\overrightarrow{IA}^2+\overrightarrow{IB}^2+\overrightarrow{IC}^2+2\overrightarrow{ID}^2$.\\
	Biểu thức trên đạt GTNN khi $M\equiv I\Rightarrow M\left( -\dfrac{9}{4};\dfrac{3}{2};\dfrac{5}{4} \right)$.\\
	Vậy $100a+10b+c=100\cdot\left(-\dfrac{9}{4}\right)+10\cdot\dfrac{3}{2}+\dfrac{5}{4}=-\dfrac{835}{4}$.
}
\end{ex}
