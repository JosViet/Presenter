\newpage
\section{Ôn tập chương 6}

\def\thoigian{90}%--Thời gian
\de{Đề số 1}{Chương VI. Xác suất có điều kiện}



\begin{center}
	\textbf{PHẦN 1 - CÂU TRẮC NGHIỆM BỐN PHƯƠNG ÁN}
\end{center}
\Opensolutionfile{ans}[ans/ans-TN-ONTAPCHUONG-DE1]
%%%=============EX_1=============%%%
\begin{ex}%[2D6N1-1]%[Dự án D - đợt 3. NH24-25-Dương Công Tạo]
	Cho hai biến cố $A$ và $B$ bất kì, với $\mathrm{P}\left(B\right) > 0$. Khẳng định nào sau đây đúng?
	\choice
	{$\mathrm{P}\left(B\mid A\right)=\dfrac{\mathrm{P}\left(AB\right)}{\mathrm{P}\left(B\right)}$}
	{$\mathrm{P}\left(A\mid B\right)=\dfrac{\mathrm{P}\left(AB\right)}{\mathrm{P}\left(A\right)}$}
	{\True $\mathrm{P}\left(A\mid B\right)=\dfrac{\mathrm{P}\left(AB\right)}{\mathrm{P}\left(B\right)}$}
	{$\mathrm{P}\left(A\mid B\right)=\dfrac{\mathrm{P}\left(A\cup B\right)}{\mathrm{P}\left(B\right)}$}
	\loigiai{
		Công thức xác suất có điều kiện là $\mathrm{P}\left(A\mid B\right)=\dfrac{\mathrm{P}\left(AB\right)}{\mathrm{P}\left(B\right)}$.
	}
\end{ex}

%%%=============EX_2=============%%%
\begin{ex}%[2D6H1-2]%[Dự án D - đợt 3. NH24-25-Dương Công Tạo]
	Cho hai biến cố $A$ và $B$ với $\mathrm{P}\left(B\right)=\dfrac{2}{3}$ và $\mathrm{P}\left(A\mid B\right)=\dfrac{19}{29}$. Khi đó $\mathrm{P}\left(AB\right)$ bằng
	\choice
	{$\dfrac{49}{87}$}
	{$\dfrac{57}{58}$}
	{\True $\dfrac{38}{87}$}
	{$\dfrac{1}{87}$}
	\loigiai{
		Áp dụng công thức tính xác suất có điều kiện ta có \[\mathrm{P}\left(AB\right)=\mathrm{P}\left(A\mid B\right)\cdot\mathrm{P}\left(B\right)=\dfrac{19}{29}\cdot\dfrac{2}{3}=\dfrac{38}{87}.\]
	}
\end{ex}

%%%=============EX_3=============%%%
\begin{ex}%[2D6N1-2]%[Dự án D - đợt 3. NH24-25-Dương Công Tạo]
	Một hộp kín có $10$ thẻ màu đỏ và $15$ thẻ màu xanh. Lấy ngẫu nhiên lần lượt $2$ thẻ, không trả lại. Xác suất để lần thứ hai lấy được thẻ màu xanh, biết rằng lần thứ nhất đã lấy được thẻ màu đỏ.
	\choice
	{$\dfrac{3}{5}$}
	{$\dfrac{5}{12}$}
	{$\dfrac{7}{12}$}
	{\True $\dfrac{15}{24}$}
	\loigiai{
		Gọi $A$ là biến cố \lq\lq  lần thứ nhất lấy được thẻ màu đỏ\rq\rq;\\ 
		Gọi $B$ là biến cố \lq\lq  lần thứ hai lấy được thẻ màu xanh\rq\rq.\\
		Do đó xác suất để lần thứ hai lấy được thẻ màu xanh, biết rằng lần thứ nhất đã lấy được thẻ màu đỏ là \[\mathrm{P}\left(B\mid A\right)=\dfrac{n(AB)}{n(A)}=\dfrac{10\cdot 15}{10\cdot9+10\cdot15}=\dfrac{5}{8}=\dfrac{15}{24}.\]
	}
\end{ex}

%%%=============EX_4=============%%%
\begin{ex}%[2D6H1-2]%[Dự án D - đợt 3. NH24-25-Dương Công Tạo]
	Một công ty bất động sản đấu giá quyền sử dụng hai mảnh đất độc lập. Khả năng trúng đấu giá cao nhất của mảnh đất số $1$ là $0{,}7$ và mảnh đất số $2$ là $0{,}8$. Xác suất để công ty trúng giá cao nhất mảnh đất số $2$, biết công ty trúng giá cao nhất mảnh đất số $1$ là
	\choice
	{\True $0{,}8$}
	{$0{,}7$}
	{$0{,}75$}
	{$0{,}6$}
	\loigiai{
		Gọi $A$ là biến cố \lq\lq  công ty trúng giá cao nhất mảnh đất số $1$\rq\rq;\\ 
		Gọi $B$ là biến cố \lq\lq  công ty trúng giá cao nhất mảnh đất số $2$\rq\rq.\\
		Ta có  \[\mathrm{P}\left(B\mid A\right)=\mathrm{P}\left(B\right)=0{,}8.\]
	}
\end{ex}

%%%=============EX_5=============%%%
\begin{ex}%[2D6H1-2]%[Dự án D - đợt 3. NH24-25-Dương Công Tạo]
	Gieo lần lượt hai con xúc xắc cân đối và đồng chất. Tính xác suất để tổng số chấm xuất hiện trên hai con xúc xắc bằng $5$, biết rằng con xúc xắc thứ nhất xuất hiện mặt $3$ chấm.
	\choice
	{$\dfrac{1}{5}$}
	{$\dfrac{3}{4}$}
	{$\dfrac{2}{5}$}
	{\True $\dfrac{1}{6}$}
	\loigiai{
		Gọi $A$ là biến cố \lq\lq  con xúc xắc thứ nhất xuất hiện mặt $3$ chấm\rq\rq;\\
		Gọi $B$ là biến cố \lq\lq  tổng số chấm xuất hiện trên hai con xúc xắc bằng $5$\rq\rq.\\
		Khi con xúc xắc thứ nhất đã xuất hiện mặt $3$, để tổng hai lần gieo bằng $5$ thì con súc sắc thứ $2$ phải xuất hiện mặt $2$ chấm.\\
		Khi đó \[\mathrm{P}\left(B\mid A\right)=\dfrac{n(AB)}{n(A)}=\dfrac{1}{6}.\]
	}
\end{ex}

%%%=============EX_6=============%%%
\begin{ex}%[2D6N1-2]%[Dự án D - đợt 3. NH24-25-Dương Công Tạo]
	Một cửa hàng thời trang ước lượng rằng có $86\%$ khách hàng đến cửa hàng mua quần áo là phụ nữ, và có $25\%$ số khách mua hàng là phụ nữ cần nhân viên tư vấn. Biết một người mua quần áo là phụ nữ, tính xác suất người đó cần nhân viên tư vấn.
	\choice
	{$\dfrac{1}{4}$}
	{$0{,}86$}
	{$\dfrac{30}{43}$}
	{\True $\dfrac{25}{86}$}
	\loigiai{
		Gọi $A$ là biến cố \lq\lq  người mua hàng là phụ nữ\rq\rq;\;
		$B$ là biến cố \lq\lq  người mua hàng là phụ nữ cần nhân viên tư vấn\rq\rq.\\
		Ta có $\mathrm{P}\left(A\right)=0{,}86$; $\mathrm{P}\left(AB\right)=0{,}25$.\\
		Vậy $\mathrm{P}\left(B\mid A\right)=\dfrac{0{,}25}{0{,}86}=\dfrac{25}{86}$.
	}
\end{ex}

%%%=============EX_7=============%%%
\begin{ex}%[2D6H2-2]%[Dự án D - đợt 3. NH24-25-Dương Công Tạo]
	Trong quân sự, một máy bay chiến đấu của đối phương có thể xuất hiện ở vị trí $X$ với xác suất $0{,}55$. Nếu máy bay đó không xuất hiện ở vị trí $X$ thì nó xuất hiện ở vị trí $Y$. Để phòng thủ, các bệ phóng tên lửa được bố trí tại các vị trí $X$ và $Y$. Khi máy bay đối phương xuất hiện ở vị trí $X$ hoặc $Y$ thì tên lửa sẽ được phóng để hạ máy bay đó.
	Xét phương án tác chiến sau: Nếu máy bay xuất hiện tại $X$ thì bắn $2$ quả tên lửa và nếu máy bay xuất hiện tại $Y$ thì bắn $1$ quả tên lửa.
	Biết rằng, xác suất bắn trúng máy bay của mỗi quả tên lửa là $0{,}8$ và các bệ phóng tên lửa hoạt động độc lập. Máy bay bị bắn hạ nếu nó trúng ít nhất $1$ quả tên lửa. Tính xác suất bắn hạ máy bay đối phương trong phương án tác chiến nêu trên.
	\choice
	{\True $0{,}888$}
	{$0{,}777$}
	{$0{,}666$}
	{$0{,}555$}
	\loigiai{
		Gọi $A$ là biến cố \lq\lq Máy bay xuất hiện ở vị trí $X$\rq\rq;\; $B$ là biến cố \lq\lq Máy bay bị bắn rơi\rq\rq.\\
		Ta có $\mathrm{P}(A)=0{,}55$ và $\mathrm{P}\left(\overline{A}\right)=0{,}45$.\\
		Nếu máy bay xuất hiện tại $X$ thì có hai quả tên lửa bắn lên.\\ $\mathrm{P}\left(B\mid A\right)$ là xác suất để máy bay rơi khi có hai quả tên lửa bắn lên. \\
		Khi đó, $\mathrm{P}\left(\overline{B}\mid A\right)=\left(1-0{,}8\right)^2=0{,}04$. \\
		Vậy $\mathrm{P}\left(B\mid A\right)=1-\mathrm{P}\left(\overline{B}\mid A\right)=1-0{,}04=0{,}96$.\\
		Theo công thức xác suất toàn phần
		\[\mathrm{P}\left(B\right)=\mathrm{P}\left(A\right)\cdot\mathrm{P}\left(B\mid A\right)+\mathrm{P}\left(\overline{A}\right)\cdot\mathrm{P}\left(B\mid \overline{A}\right)=0{,}55\cdot 0{,}96+0{,}45\cdot 0{,}8=0{,}888.\]
	}
\end{ex}

%%%=============EX_8=============%%%
\begin{ex}%[2D6H2-2]%[Dự án D - đợt 3. NH24-25-Dương Công Tạo]
	Có hai đội thi đấu môn Bắn súng. Đội $1$ có $5$ vận động viên, đội $2$ có $7$ vận động viên. Xác suất đạt huy chương vàng của mỗi vận động viên đội $1$ và đội $2$ tương ứng là $0{,}65$ và $0{,}55$. Chọn ngẫu nhiên một vận động viên. Giả sử vận động viên được chọn đạt huy chương vàng. Tính xác suất để vận động viên này thuộc đội $1$.
	\choice
	{\True $0{,}577$}
	{$0{,}677$}
	{$0{,}777$}
	{$0{,}877$}
	\loigiai{
		Gọi $A$ là biến cố \lq\lq VĐV thuộc đội $1$\rq\rq;\; $\overline{A}$ là biến cố \lq\lq VĐV thuộc đội $2$\rq\rq;\; $E$ là biến cố \lq\lq VĐV đạt HCV\rq\rq.\\
		Ta có $ \mathrm{P}(A)=\dfrac{5}{12}$;  $\mathrm{P}\left(\overline{A}\right)=\dfrac{7}{12}$;  $\mathrm{P}\left(E\mid A\right)=0{,}65$; $\mathrm{P}\left(E\mid \overline{A}\right)=0{,}55$.\\
		Theo công thức xác suất toàn phần $\mathrm{P}(E)=\mathrm{P}\left(A\right)\mathrm{P}\left(E\mid A\right)+\mathrm{P}\left(\overline{A}\right)\mathrm{P}\left(E\mid \overline{A}\right)=0{,}5917$.\\
		Vậy $\mathrm{P}\left(E\mid A\right)=\dfrac{\mathrm{P}(A)\cdot\mathrm{P}\left(A\mid E\right)}{\mathrm{P}(E)}=0{,}577$.
	}
\end{ex}

%%%=============EX_9=============%%%
\begin{ex}%[2D6H2-3]%[Dự án D - đợt 3. NH24-25-Dương Công Tạo]
	Một bộ lọc được sử dụng để chặn thư rác trong các tài khoản thư điện tử. Tuy nhiên, vì bộ lọc không tuyệt đối hoàn hảo nên một thư rác bị chặn với xác suất $0{,}95$ và một thư đúng (không phải là thư rác) bị chặn với xác suất $0{,}01$. Thống kê cho thấy tỉ lệ thư rác là $3\%$. Chọn ngẫu nhiên một thư bị chặn. Tính xác suất để đó là thư rác.
	\choice
	{\True $0{,}746$}
	{$0{,}657$}
	{$0{,}764$}
	{$0{,}846$}
	\loigiai{
		Gọi $A$ là biến cố \lq\lq Thư đó là thư rác\rq\rq;\; $B$ là biến cố \lq\lq Thư đó là bị chặn\rq\rq.\\
		Ta có $\mathrm{P}(A)=0{,}03$; $\mathrm{P}\left(\overline{A}\right)=0{,}97$; $\mathrm{P}\left(B\mid A\right)=0{,}95$; $\mathrm{P}\left(B\mid \overline{A}\right)=0{,}01$.\\
		Ta phải tính cần tính xác suất để đó là thư rác biết nó là thư bị chặn tức là ta cần tính $\mathrm{P}\left(A\mid B\right)$.\\
		Áp dụng công thức Bayes ta có
		\begin{align*}
			\mathrm{P}\left(A\mid B\right)=&\dfrac{\mathrm{P}(A)\cdot\mathrm{P}\left(B\mid A\right)}{\mathrm{P}(A)\cdot\mathrm{P}\left(B\mid A\right)+\mathrm{P}\left(\overline{A}\right)\cdot\mathrm{P}\left(B\mid \overline{A}\right)}\\
			=&\dfrac{0{,}3\cdot 0{,}95}{0{,}3\cdot 0{,}97+0{,}01\cdot 0{,}95}\\
			\approx& 0{,}746.
		\end{align*}
	}
\end{ex}

%%%=============EX_10=============%%%
\begin{ex}%[2D6H2-3]%[Dự án D - đợt 3. NH24-25-Dương Công Tạo]
	Một loại vaccine được tiêm ở địa phương $X$. Người có bệnh nền thì với xác suất $0{,}35$ có phản ứng phụ sau tiêm; người không có bệnh nền thì chỉ có phản ứng phụ sau tiêm với xác suất $0{,}16$. Chọn ngẫu nhiên một người được tiêm vaccine và người này có phản ứng phụ. Tính xác suất để người này có bệnh nền, biết rằng tỉ lệ người có bệnh nền ở địa phương $X$ là $18\%$.
	\choice
	{\True $0{,}324$}
	{$0{,}342$}
	{$0{,}424$}
	{$0{,}524$}
	\loigiai{
		Gọi $A$ là biến cố \lq\lq Người bị bệnh nền\rq\rq\; $B$ là biến cố \lq\lq Người có phản ứng phụ sau tiêm\rq\rq.\\
		Ta có $\mathrm{P}\left(A\right)=0{,}18$; $\mathrm{P}\left(\overline{A}\right)=0{,}82$; $\mathrm{P}\left(B\mid A\right)=0{,}35$;  $\mathrm{P}\left(B\mid \overline{A}\right)=0{,}16$.\\
		Áp dụng công thức Bayes ta có
		\begin{align*}
			\mathrm{P}\left(B\mid A\right)=&\dfrac{\mathrm{P}\left(A\right)\cdot\mathrm{P}\left(B\mid A\right)}{\mathrm{P}\left(A\right)\cdot\mathrm{P}\left(B\mid A\right)+\mathrm{P}\left(\overline{A}\right)\cdot\mathrm{P}\left(B\mid \overline{A}\right)}\\
			=&\dfrac{315}{971} \\
			\approx& 0{,}324.
		\end{align*}
	}
\end{ex}

%%%=============EX_11=============%%%
\begin{ex}%[2D6H2-2]%[Dự án D - đợt 3. NH24-25-Dương Công Tạo]
	Để thử nghiệm tác dụng điều trị thuốc mất ngủ của hai loại thuốc X và Y, người ta tiến hành thử nghiệm trên $4\,000$ người bệnh tình nguyện. Kết quả được cho trong bảng thống kê $2\times 2$ sau
	\begin{center}
		\begin{tikzpicture}
			\begin{scope}[xscale=4.4]
				\path
				(0,0) foreach \i[count=\k] in {X,Y} {++(1,0)node(1\k){\i}}
				(0,-1) node {Khỏi bệnh} foreach \i[count=\k] in {$1\,600$,$1\,200$} {++(1,0)node(2\k){\i}}
				(0,-2) node{Không khỏi bệnh} foreach \i[count=\k] in {$800$,$400$} {++(1,0)node(3\k){\i}}
				;
				\draw[shift={(-0.5,.5)}] (0,0) grid (3.,-3)
				(0,0)--(1.,-1)
				(0,-1) node[above right]{Kết quả}
				(1,0) node[below left]{Dùng thuốc}
				;
			\end{scope}
		\end{tikzpicture}
	\end{center}
	Chọn ngẫu nhiên $1$ người bệnh tham gia tình nguyện thử nghiệm thuốc. Tính xác suất để người bệnh đó khỏi bệnh.
	\choice
	{\True $0{,}36$}
	{$0{,}46$}
	{$0{,}56$}
	{$0{,}66$}
	\loigiai{
		Gọi $A$ là biến cố \lq\lq  người đó uống thuốc X\rq\rq;\;
		Gọi $B$ là biến cố \lq\lq  người đó khỏi bệnh\rq\rq.\\
		Ta có $\mathrm{P}(A)=\dfrac{1\,600+800}{4\,000}=\dfrac{3}{5}$; $\mathrm{P}\left(B\mid A\right)= \dfrac{1\,600}{4\,000}=\dfrac{2}{5}$.\\ Khi đó $\mathrm{P}\left(\overline{A}\right)=\dfrac{1\,200+400}{4\,000}=\dfrac{2}{5};\mathrm{P}\left(B\mid \overline{A}\right)= \dfrac{1\,200}{4\,000}=\dfrac{3}{10}$.\\
		Áp dụng công thức các suất toàn phần $\mathrm{P}\left(B\right)=\mathrm{P}\left(A\right)\mathrm{P}\left(B\mid A\right)+\mathrm{P}\left(\overline{A}\right)\mathrm{P}\left(B\mid \overline{A}\right)=\dfrac{3}{5} \cdot \dfrac{2}{5}+\dfrac{2}{5} \cdot \dfrac{3}{10}=0{,}36$.
	}
\end{ex}

%%%=============EX_12=============%%%
\begin{ex}%[2D6H2-3]%[Dự án D - đợt 3. NH24-25-Dương Công Tạo]
	Cho hai biến cố $A$,\ $B$ thoả mãn $\mathrm{P}\left(A\right)=0{,}4$; $\mathrm{P}\left(B\right)=0{,}3$; $\mathrm{P}\left(AB\right)=0{,}25$. Khi đó, $\mathrm{P}\left(BA\right)$ bằng
	\choice
	{\True $0{,}1875$}
	{$0{,}48$}
	{$0{,}333$}
	{$0{,}95$}
	\loigiai{
		Theo công thức Bayes, ta có $\mathrm{P}\left(BA\right)=\dfrac{\mathrm{P}\left(B\right)\cdot\mathrm{P}\left(AB\right)}{\mathrm{P}\left(A\right)}=\dfrac{0{,}3\cdot 0{,}25}{0{,}4}=0{,}1875$.
	}
\end{ex}
\Closesolutionfile{ans}


\begin{center}
	\textbf{PHẦN 2 - CÂU TRẮC NGHIỆM ĐÚNG SAI}
\end{center}

\Opensolutionfile{ans}[ans/answer-DS-ONTAPCHUONG-DE1]
\setcounter{ex}{0}
%%%=============EX_1=============%%%
\begin{ex}%[2D6V1-4]%[Dự án D - đợt 3. NH24-25-Dương Công Tạo]
	Một hộp có $10$ bi xanh và $8$ bi đen, các viên bi đều có cùng hình dáng, kích thước và khối lượng. Bạn Nam lấy ngẫu nhiên một viên trong hộp, không trả lại. Sau đó Bạn Lan lấy ngẫu nhiên một trong $17$ viên bi còn lại. Gọi $A$ là biến cố bạn Nam lấy được một viên bi xanh và $B$ là biến cố bạn Lan lấy được một viên bi đen.
	\choiceTF
	{\True $n\left(A\right)=10$}
	{\True $\mathrm{P}\left(A\right)=\dfrac{5}{9}$}
	{$\mathrm{P}\left(B\mid A\right)=\dfrac{4}{9}$}
	{$\mathrm{P}\left(AB\right)=0,8$}
	\loigiai{
		\begin{itemchoice}
			\itemch Vì hộp có $10$ bi xanh nên số phần tử của biến cố $A$ là $n\left(A\right)=10$.
			\itemch Vì bạn Nam lấy ngẫu nhiên 1 viên bi từ hộp chứa $10$ bi xanh và $8$ bi đen nên $n\left(\Omega \right)=10+8=18$.\\
			Do đó, $\mathrm{P}\left(A\right)=\dfrac{n\left(A\right)}{n\left(\Omega \right)}=\dfrac{10}{18}=\dfrac{5}{9}$.
			\itemch Nếu $A$ xảy ra tức là bạn Nam lấy được bi xanh thì trong hộp có $17$ viên bi với $8$ bi đen.\\
			Do đó, $\mathrm{P}\left(B\mid A\right)=\dfrac{8}{17} \ne \dfrac{4}{9}$.
			\itemch Áp dụng công thức xác suất có điều kiện, ta có $\mathrm{P}\left(AB\right)=\mathrm{P}\left(A\right)\cdot\mathrm{P}\left(B\mid A\right)=\dfrac{5}{9}\cdot\dfrac{8}{17}=\dfrac{40}{153} \approx 0{,}3\ne 0{,}8$.
		\end{itemchoice}
	}
\end{ex}

%%%=============EX_2=============%%%
\begin{ex}%[2D6V2-4]%[Dự án D - đợt 3. NH24-25-Dương Công Tạo]
	[Câu 4--MĐ 0101-Đề TNTHPT. NH 24--25] 
	Một phần mềm nhận dạng tin nhắn quảng cáo trên điện thoại bằng cách dựa theo từ khóa để đánh dấu một số tin nhắn được gửi đến. Qua một thời gian dài sử dụng, người ta thấy rằng trong số tất cả các tin nhắn gửi đến, có $15\%$ số tin nhắn bị đánh dấu. Trong số các tin nhắn bị đánh dấu, có $10\%$ số tin nhắn không phải là quảng cáo. Trong số các tin nhắn không bị đánh dấu, có $5\%$ số tin nhắn là quảng cáo.
	Chọn ngẫu nhiên một tin nhắn được gửi đến điện thoại.
	\choiceTF
	{\True Xác suất để tin nhắn đó không bị đánh dấu bằng $0{,}85$}
	{\True Xác suất để tin nhắn đó không phải là quảng cáo, biết rằng nó không bị đánh dấu, bằng $0{,}95$}
	{Xác suất để tin nhắn đó không phải là quảng cáo bằng $0{,}85$}
	{\True Xác suất để tin nhắn đó không bị đánh dấu, biết rằng nó không phải là quảng cáo, lớn hơn $0{,}95$}
	\loigiai{
		Gọi biến cố $A\colon$ \lq\lq Chọn được tin nhắn bị đánh dấu\rq\rq;\\
		Biến cố $B\colon$ \lq\lq Chọn được tin nhắn là tin nhắn quảng cáo\rq\rq.
		\begin{itemchoice}
			\itemch Theo đề bài, ta có
			\begin{itemize}
				\item $\mathrm{P}(A)=15\%=0{,}15$;
				\item $\mathrm{P}\left(\overline{B}\mid A\right)=10\%=0{,}1$;
				\item $\mathrm{P}\left(B\mid \overline{A}\right)=5\%=0{,}05$.
			\end{itemize}
			\itemch Xác suất để tin nhắn đó không bị đánh dấu là $\mathrm{P}\left(\overline{A}\right)=1-\mathrm{P}\left(A\right)=1-0{,}15=0{,}85$.
			\itemch Xác suất để tin nhắn đó không phải là quảng cáo, biết rằng nó không bị đánh dấu là \[\mathrm{P}\left(\overline{B}\mid \overline{A} \right)=1-\mathrm{P}\left(B\mid \overline{A}\right)=1-0{,}05=0{,}95.\]
			\itemch Xác suất để tin nhắn đó không phải là quảng cáo là
			\begin{align*}
				\mathrm{P}\left(\overline{B}\right)=&\mathrm{P}\left(A\right)\cdot \mathrm{P}\left(\overline{B}\mid A\right)+\mathrm{P}\left(\overline{A}\right)\cdot \mathrm{P}\left(\overline{B}\mid \overline{A}\right)\text{ (theo công thức xác suất toàn phần)}\\
				=&0{,}15\cdot 0{,}1+0{,}85\cdot 0{,}95\\
				\approx&0{,}8225.
			\end{align*}
			\itemch Xác suất để tin nhắn đó không bị đánh dấu, biết rằng nó không phải là quảng cáo là 
			\begin{align*}
				\mathrm{P}\left( \overline{A}\mid \overline{B}\right)=&\dfrac{\mathrm{P}\left(\overline{A}\right)\cdot \mathrm{P}\left(\overline{B}\mid \overline{A}\right)}{\mathrm{P}\left(\overline{B}\right)}\text{ (theo công thức Bayes)}\\
				=&\dfrac{0{,}85\cdot 0{,}95}{0{,}8225}\\
				=&\dfrac{323}{329} \\
				\approx& 0{,}98.
			\end{align*}
		\end{itemchoice}
	}
\end{ex}
\Closesolutionfile{ans}
%\inputansbox[2]{2}{ans/answer.tex}



\begin{center}
	\textbf{PHẦN 3 - CÂU TRẮC NGHIỆM TRẢ LỜI NGẮN}
\end{center}
\setcounter{ex}{0}
\Opensolutionfile{ans}[ans-KQ-ONTAPCHUONG-DE1]
%%%=============EX_1=============%%%
\begin{ex}%[2D6H1-2]%[Dự án D - đợt 3. NH24-25-Dương Công Tạo]
	Cho hai biến cố $A$, $B$ thỏa mãn $\mathrm{P}\left(A\right)=0{,}21$; $\mathrm{P}\left(B\right)=0{,}52$; $\mathrm{P}\left(B\mid A\right)=0{,}6$. Khi đó $\mathrm{P}\left(A\mid B\right)=\dfrac{a}{b}$ với $\dfrac{a}{b}$ là phân số tối giản, giá trị của $D=a+b$ là bao nhiêu?
	\shortans{323}
	\loigiai{
		Ta có $\mathrm{P}\left(AB\right)=\mathrm{P}\left(A\right).\mathrm{P}\left(B\mid A\right)=0{,}21.0{,}6=0{,}126$.\\
		Khi đó $\mathrm{P}\left(AB\right)=\mathrm{P}\left(B\right)\cdot\mathrm{P}\left(A\mid B\right)\Rightarrow \mathrm{P}\left(A\mid B\right)=\dfrac{\mathrm{P}\left(AB\right)}{\mathrm{P}\left(B\right)}=\dfrac{0{,}126}{0{,}52}=\dfrac{63}{260}$.\\
		Suy ra $a=63$ và $b=260$.\\
		Vậy $D=a+b=63+260=323$.
	}
\end{ex}

%%%=============EX_2=============%%%
\begin{ex}%[2D6V1-4]%[Dự án D - đợt 3. NH24-25-Dương Công Tạo]
	[Thi thử TN Sở Thanh Hoá. NH 24--25] 
	Một nhóm sinh viên y khoa thực hiện khảo sát những bệnh nhân bị tai nạn xe máy về mối liên hệ giữa việc đội mũ bảo hiểm và khả năng bị chấn thương vùng đầu cho thấy: Tỉ lệ bệnh nhân bị chấn thương vùng đầu là $60\%$; tỉ lệ bệnh nhân đội mũ bảo hiểm đúng cách là $40\%$; tỉ lệ bệnh nhân đội mũ bảo hiểm đúng cách và bị chấn thương vùng đầu là $10\%$. Hỏi theo kết quả khảo sát trên, việc đội mũ bảo hiểm đúng cách sẽ làm giảm khả năng bị chấn thương vùng đầu bao nhiêu lần?
	\shortans{3}
	\loigiai{
		Chọn $1$ bệnh nhân trong những bệnh nhân bị tai nạn xe máy.\\
		Gọi $A$ là biến cố \lq\lq Bệnh nhân bị chấn thương vùng đầu\rq\rq; $B$ là biến cố  \lq\lq Bệnh nhân đội mũ bảo hiểm đúng cách\rq\rq.\\
		$AB$ là biến cố  \lq\lq Bệnh nhân đội mũ bảo hiểm đúng cách và bị chấn thương vùng đầu\rq\rq.\\
		Theo đề bài, ta có $\mathrm{P}\left(A\right)=60\%=0{,}6$; $\mathrm{P}\left(B\right)=40\%=0{,}4$ và $\mathrm{P}\left(AB\right)=10\%=0{,}1$.\\
		Ta cần tính xác suất bệnh nhân bị chấn thương đầu khi đội mũ bảo hiểm đúng cách, tức $\mathrm{P}\left(A\mid B\right)$ và xác xuất bệnh nhận không chấn thương đầu khi đội mũ bảo hiểm đúng cách, tức $\mathrm{P}\left(\overline{A}\mid B\right)$.\\
		Ta có $\mathrm{P}\left(A\mid B\right)=\dfrac{\mathrm{P}\left(AB\right)}{\mathrm{P}\left(B\right)}=\dfrac{1}{4}=0{,}25\Rightarrow \mathrm{P}\left(\overline{A}\mid B\right)=\dfrac{3}{4}=0{,}75$.\\
		Suy ra việc đội mũ bảo hiểm đúng cách sẽ làm giảm khả năng chấn thương vùng đầu đi $\dfrac{0{,}75}{0{,}25}=3$ lần.
	}
\end{ex}

%%%=============EX_3=============%%%
\begin{ex}%[2D6V2-4]%[Dự án D - đợt 3. NH24-25-Dương Công Tạo]
	[Thi thử TN Chuyên KHTN TPHCM. NH 24--25] 
	Một nhà máy có hai phân xưởng I và II tương ứng làm ra $40\%$ và $60\%$ sản phẩm của nhà máy. Biết rằng tỉ lệ phế phẩm của hai phân xưởng I và II tương ứng là $1\%$ và $2\%$. Chọn ngẫu nhiên một sản phẩm của nhà máy thì thấy nó là phế phẩm. Tính xác suất để sản phẩm đó thuộc phân xưởng I.
	\shortans{0{,}25}
	\loigiai{
		Gọi $A_1$ là biến cố \lq\lq sản phẩm thuộc phân xưởng I\rq\rq,
		$A_2$ là biến cố \lq\lq sản phẩm thuộc phân xưởng II\rq\rq,\\
		$B$ là biến cố \lq\lq sản phẩm được chọn là phế phẩm\rq\rq.\\
		Từ giả thiết ta có $\mathrm{P}\left(A_1 \right)=0{,}4$; $\mathrm{P}\left(A_2 \right)=0{,}6$; $\mathrm{P}\left(B\mid A_1 \right)=0{,}01$; $\mathrm{P}\left(B\mid A_2 \right)=0{,}02$.\\
		Áp dụng công thức xác suất toàn phần ta có $\mathrm{P}\left(B\right)=\mathrm{P}\left(B\mid A_1 \right) \mathrm{P}\left(A_1 \right)+\mathrm{P}\left(B\mid A_2 \right) \mathrm{P}\left(A_2 \right)=0{,}016$.\\
		Xác suất phế phẩm được chọn thuộc phân xưởng I là \[\mathrm{P}\left(A_1\mid B\right)=\dfrac{\mathrm{P}\left(B\mid A_1 \right)\cdot \mathrm{P}\left(A_1 \right)}{\mathrm{P}\left(B\right)}=0{,}25.\]
	}
\end{ex}

%%%=============EX_4=============%%%
\begin{ex}%[2D6V2-4]%[Dự án D - đợt 3. NH24-25-Dương Công Tạo]
	[Thi thử TN Sở ĐăKLăk. NH 24--25] 
	Có hai chuồng thỏ. Chuồng thứ nhất có $6$ con thỏ đực và $4$ con thỏ cái. Chuồng thứ hai có $4$ con thỏ đực và $5$ con thỏ cái. Từ chuồng thứ nhất lấy ngẫu nhiên ra một con thỏ bỏ vào chuồng thứ hai. Rồi sau đó từ chuồng thứ hai lấy ngẫu nhiên ra $3$ con thỏ. Biết trong $3$ con thỏ lấy ra ở chuồng thứ hai có số thỏ đực nhiều hơn số thỏ cái. Tính xác suất con thỏ lấy ra ở chuồng thứ nhất là thỏ đực (làm tròn kết quả đến hàng phần trăm).
	\shortans{0{,}69}
	\loigiai{
		Gọi $A\colon$ \lq\lq Con thỏ lấy ra ở chuồng thứ nhất là thỏ đực\rq\rq;\\
		$B\colon$ \lq\lq Trong 3 con thỏ lấy ra ở chuồng thứ hai có số thỏ đực nhiều hơn số thỏ cái\rq\rq.\\
		Xác suất để con thỏ lấy ra ở chuồng thứ nhất là thỏ đực là $\mathrm{P}\left(A\right)=\dfrac{6}{10}=\dfrac{3}{5}$.\\
		Xác suất để con thỏ lấy ra ở chuồng thứ nhất là thỏ cái là $\mathrm{P}\left(\overline{A}\right)=1-\mathrm{P}\left(A\right)=1-\dfrac{3}{5}=\dfrac{2}{5}$.
		\begin{itemize}
			\item Nếu con thỏ lấy ra ở chuồng thứ nhất là thỏ đực thì chuồng thứ hai lúc đó có $5$ con thỏ đực và $5$ con thỏ cái.\\
			Để ba con thỏ lấy ra ở chuồng thứ hai có số thỏ đực nhiều hơn số thỏ cái ta có hai trường hợp
			\begin{enumerate}[\it TH 1:]
				\item Ba con thỏ lấy ra ở chuồng thứ hai có $2$ con thỏ đực và $1$ con thỏ cái có $\mathrm{C}_5^2\cdot \mathrm{C}_5^1$ (cách).
				\item Cả ba con thỏ lấy ra ở chuồng thứ hai đều là thỏ đực, có $\mathrm{C}_5^3$ (cách).
			\end{enumerate}
			Khi đó, $\mathrm{P}\left(B\mid A\right)=\dfrac{\mathrm{C}_5^2\cdot \mathrm{C}_5^1+\mathrm{C}_5^3}{\mathrm{C}_{10}^3}=\dfrac{1}{2}$
			\item Nếu con thỏ lấy ra ở chuồng thứ nhất là thỏ cái thì chuồng thứ hai lúc đó có $4$ con thỏ đực và $6$ con thỏ cái.\\
			Để ba con thỏ lấy ra ở chuồng thứ hai có số thỏ đực nhiều hơn số thỏ cái ta có hai trường hợp
			\begin{enumerate}[\it TH 1:]
				\item Ba con thỏ lấy ra ở chuồng thứ hai có $2$ con thỏ đực và $1$ con thỏ cái có $\mathrm{C}_4^2\cdot \mathrm{C}_6^1$ (cách).
				\item Cả ba con thỏ lấy ra ở chuồng thứ hai đều là thỏ đực, có $\mathrm{C}_4^3$ (cách).
			\end{enumerate}
			Khi đó, $\mathrm{P}\left(B\mid \overline{A}\right)=\dfrac{\mathrm{C}_4^2\cdot C_6^1+\mathrm{C}_4^3}{\mathrm{C}_{10}^3}=\dfrac{1}{3}$.\\
			Theo công thức Bayes, ta có xác suất con thỏ lấy ra ở chuồng thứ nhất là thỏ đực biết trong $3$ con thỏ lấy ra ở chuồng thứ hai có số thỏ đực nhiều hơn số thỏ cái là
			\[\mathrm{P}\left(A\mid B\right)=\dfrac{\mathrm{P}\left(A\right)\cdot \mathrm{P}\left(B\mid A\right)}{\mathrm{P}\left(A\right)\cdot \mathrm{P}\left(B\mid A\right)+\mathrm{P}\left(\overline{A}\right)\cdot \mathrm{P}\left(B\mid \overline{A}\right)}=\dfrac{\dfrac{3}{5}\cdot\dfrac{1}{2}}{\dfrac{3}{5}\cdot\dfrac{1}{2}+\dfrac{2}{5}\cdot\dfrac{1}{3}}=\dfrac{9}{13} \approx 0{,}69.\]
		\end{itemize}
	}
\end{ex}

\Closesolutionfile{ans}


\begin{center}
	\textbf{PHẦN 4 - TỰ LUẬN}
\end{center}

\setcounter{ex}{0}
%%%=============EX_1=============%%%
\begin{ex}%[2D6V1-4]%[Dự án D - đợt 3. NH24-25-Dương Công Tạo]
	[Thi thử TN Sở Vũng Tàu. NH 24--25] 
	Điều tra tình hình mắc bệch ung thư phổi của một vùng thấy tỉ lệ người hút thuốc lá và mắc bệnh là $15\%$. Tỉ lệ người hút thuốc lá và không mắc bệnh là $25\%$, tỉ lệ người không hút thuốc lá và không mắc bệnh là $50\%$ và $10\%$ là người không hút thuốc nhưng mắc bệnh. Tỉ lệ mắc bệnh ung thư phổi giữa người hút thuốc lá và không hút thuốc lá là bao nhiêu?
	\loigiai{
		Gọi biến cố $A\colon$ \lq\lq Người hút thuốc\rq\rq;\; $B$ \lq\lq Bị mắc bệnh ung thư phổi\rq\rq.\\
		Theo đề bài ta có
		\[\mathrm{P}\left(A\cap B\right)=0{,}15;\mathrm{P}\left(A\cap \overline{B}\right)=0{,}25;\mathrm{P}\left(\overline{A}\cap \overline{B}\right)=0{,}5;\mathrm{P}\left(\overline{A}\cap B\right)=0{,}1.\]
		Suy ra $\mathrm{P}(B)=\mathrm{P}\left(A\cap B\right)+\mathrm{P}\left(\overline{A}\cap B\right)=0{,}15+0{,}1=0{,}25$.\\
		Xác suất người đó hút thuốc lá biết họ mắc bệnh ung thư phổi là
		\[\mathrm{P}\left(A\mid B\right)=\dfrac{\mathrm{P}\left(A\cap B\right)}{\mathrm{P}(B)}=\dfrac{0{,}15}{0{,}25}=0{,}6.\]
		Xác suất người đó không hút thuốc lá biết họ mắc bệnh ung thư phổi là
		\[\mathrm{P}\left(\overline{A}\mid B\right)=\dfrac{\mathrm{P}\left(\overline{A}\cap B\right)}{\mathrm{P}(B)}=\dfrac{0{,}1}{0{,}25}=0{,}4.\]
		Vậy tỉ lệ mắc bệnh ung thư phổi giữa người hút thuốc lá và người không hút thuốc lá là $\dfrac{0{,}6}{0{,}4}=1{,}5$.
	}
\end{ex}

%%%=============EX_2=============%%%
\begin{ex}%[2D6V2-4]%[Dự án D - đợt 3. NH24-25-Dương Công Tạo]
	Trong một đợt kiểm tra sức khoẻ, có một loại bệnh X mà tỉ lệ người mắc bệnh là $0{,}2\%$ và một loại xét nghiệm Y mà ai mắc bệnh X khi xét nghiệm Y cũng có phản ứng dương tính. Tuy nhiên, có $6\%$ những người không bị bệnh X lại có phản ứng dương tính với xét nghiệm Y. Chọn ngẫu nhiên 1 người trong đợt kiểm tra sức khoẻ đó. Giả sử người đó có phản ứng dương tính với xét nghiệm Y. Tính xác suất người đó bị mắc bệnh X là bao nhiêu?
	\loigiai{
		Xét các biến cố
		$A\colon$ \lq\lq Người được chọn mắc bệnh X\rq\rq;\\
		$B\colon$ \lq\lq Người được chọn có phản ứng dương tính với xét nghiệm Y\rq\rq.\\
		Theo giả thiết ta có \[\mathrm{P}\left(A\right)=0{,}002;\;\mathrm{P}\left(\overline{A}\right)=1-0{,}002=0{,}998;\;\mathrm{P}\left(B\mid A\right)=1;\;\mathrm{P}\left(B\mid \overline{A}\right)=0{,}06.\]
		Theo công thức Bayes, ta có
		\begin{align*}
			\mathrm{P}\left(A\mid B\right)=&\dfrac{\mathrm{P}\left(A\right)\cdot \mathrm{P}\left(B\mid A\right)}{\mathrm{P}\left(A\right)\cdot \mathrm{P}\left(B\mid A\right)+\mathrm{P}\left(\overline{A}\right)\cdot \mathrm{P}\left(B\mid \overline{A}\right)}\\
			=&\dfrac{0{,}002\cdot1}{0{,}002\cdot 1+0{,}998\cdot 0{,}06}\\
			=&\dfrac{500}{1997}\\
			\approx& 0{,}25.
		\end{align*}
		Vậy xác suất người đó bị mắc bệnh X chiếm khoảng $25\%$.
	}
\end{ex}

%%%=============EX_3=============%%%
\begin{ex}%[2D6V2-4]%[Dự án D - đợt 3. NH24-25-Dương Công Tạo]
	Một hộp bút bi Thiên Long có $15$ chiếc bút trong đó có $9$ chiếc bút mới. Người ta lấy ngẫu nhiên $1$ chiếc bút để sử dụng sau đó trả lại vào hộp. Lần thứ hai lấy ngẫu nhiên $2$ chiếc bút, tính xác suất cả hai chiếc bút lấy ra đều là chiếc mới.
	\loigiai{
		Gọi $A\colon$ \lq\lq Hai chiếc bút lấy ra đều là chiếc mới\rq\rq;\\
		$B\colon$ \lq\lq Lấy ra một chiếc bút cũ\rq\rq\; và $\overline{B}\colon$ \lq\lq Lấy ra một chiếc bút mới\rq\rq.\\
		Từ $15$ chiếc bút có $9$ chiếc bút mới và $6$ chiếc bút cũ.\\
		Ta có
		\begin{itemize}
			\item $\mathrm{P}\left(B\right)=\dfrac{\mathrm{C}_6^1}{\mathrm{C}_{15}^1}=\dfrac{2}{5}$; $\mathrm{P}\left(\overline{B}\right)=\dfrac{\mathrm{C}_9^1}{\mathrm{C}_{15}^1}=\dfrac{3}{5}$.
			\item$\mathrm{P}\left(A\mid B\right)=\dfrac{\mathrm{C}_9^2}{\mathrm{C}_{15}^2}=\dfrac{12}{35}$ và $\mathrm{P}\left(A\mid \overline{B}\right)=\dfrac{\mathrm{C}_8^2}{\mathrm{C}_{15}^2}=\dfrac{4}{15}$.
		\end{itemize}
		Áp dụng công thức xác suất toàn phần
		\begin{align*}
			\mathrm{P}\left(A\right)=&\mathrm{P}\left(A\mid B\right)\cdot \mathrm{P}\left(B\right)+\mathrm{P}\left(A\mid \overline{B}\right)\cdot \mathrm{P}\left(\overline{B}\right)\\
			=&\dfrac{12}{35}\cdot \dfrac{2}{5}+\dfrac{4}{15}\cdot \dfrac{3}{5}\\
			=&\dfrac{52}{175}.
		\end{align*}
		Vậy xác suất cả hai chiếc bút lấy ra đều là chiếc mới là $\dfrac{52}{175}$.
	}
\end{ex}