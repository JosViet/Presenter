\newpage
\def\thoigian{90}%--Thời gian
\de{Đề số 2}{Chương III. Các số đặc trưng đo mức độ phân tán}


\begin{center}
	\textbf{PHẦN 1 - Câu trắc nghiệm nhiều phương án lựa chọn.}
\end{center}
\setcounter{ex}{0}
\Opensolutionfile{ans}[ans-ABCD]
%Câu 1
\begin{ex}%[2D3N1-2]%[Dự án D - đợt 2 NH 24-25 - Xuan Vy Pham]
Cho bảng số liệu  sau đây
	\begin{center}
		\begin{tabular}{|l|c|c|c|c|c|}
			\hline 
			{Nhóm} & $[1{,}5; 2{,}5)$ & $ [2{,}5; 3{,}5)$ & $ [3{,}5; 4{,}5)$ & $ [4{,}5; 5{,}5)$ & $ [5{,}5; 6{,}5)$   \\
			\hline
			{Tần số}&$2$ & $3$ & $7$ & $2$ & $1$ \\
			\hline 
		\end{tabular}
	\end{center}\noindent
	Khoảng biến thiên của mẫu số liệu cho bởi bảng trên là
	\choice
	{$2 $}
	{$ 3$}
	{$ 4$}
	{\True $ 5$}
	\loigiai{
		Khoảng biến thiên của mẫu số liệu là $6{,}5 - 1{,}5 = 5$. }
\end{ex}
%Câu 2
\begin{ex}%[2D3N1-1]%[Dự án D - đợt 2 NH 24-25 - Xuan Vy Pham]
	Bảng sau thống kê khối lượng một số quả măng cụt được lựa chọn ngẫu nhiên trong một thùng hàng
	\begin{center}
		\begin{tabular}{|c|c|c|c|c|c|}
			\hline
			Khối lượng (gam)& $[ 80;82 )$& $[ 82;84 )$& $[ 84;86 )$& $[ 86;88 )$ & $[ 88;90 )$\\
			\hline
			Số quả & $17$ & $20$ & $25$ & $16$ & $12$ \\
			\hline
		\end{tabular}
	\end{center}
	Khoảng biến thiên của mẫu số liệu ghép nhóm trên là
	\choice
	{\True $10$ gam}
	{ $12$ gam}
	{ $2$ gam}
	{ $20$ gam}
	\loigiai{
		Khoảng biến thiên của mẫu số liệu ghép nhóm trên là $90-80=10$ gam.
	}
\end{ex}
%Câu 3
\begin{ex}%[2D3H1-2]%[Dự án D - đợt 2 NH 24-25 - Xuan Vy Pham]
	Thâm niên công tác của các công nhân hai nhà máy $A$ và $B$ được cho trong bảng sau
	
	\begin{center}
		\begin{tabular}{|c|c|c|c|c|c|}
			\hline
			Thâm niên công tác (năm) &$[ 0;5 )$&$[ 5;10 )$&$[ 10;15 )$&$[ 15;20 )$&$[ 20;25 )$\\
			\hline
			Số công nhân nhà máy $A$&$35$&$13$&$12$&$12$&$8$\\
			\hline
			Số công nhân nhà máy $B$&$19$&$26$&$24$&$11$&$0$\\
			\hline
		\end{tabular}
	\end{center}
	Trong các mệnh đề sau, mệnh đề nào đúng?
	\choice
	{Nhà máy $A$ có thâm niên công tác của các công nhân phân tán bằng nhà máy $B$}
	{\True Nhà máy $A$ có thâm niên công tác của các công nhân phân tán lớn hơn nhà máy $B$}
	{ Nhà máy $A$ có thâm niên công tác của các công nhân phân tán nhỏ hơn nhà máy $B$}
	{ Nhà máy $A$ có thâm niên công tác của các công nhân phân tán bằng nhà máy $B$}
	\loigiai{
		Khoảng biến thiên của mẫu số liệu ghép nhóm về thâm niên công tác của các công nhân của nhà máy $A$ là $25-0=25$ năm.\\
		Khoảng biến thiên của mẫu số liệu ghép nhóm về thâm niên công tác của các công nhân của nhà máy $B$ là $20-0=20$ năm. \\
		Do vậy, nhà máy $A$ có thâm niên công tác của các công nhân phân tán lớn hơn nhà máy $B$.
	}
\end{ex}
%Câu 4
\begin{ex}%[2D3H1-2]%[Dự án D - đợt 2 NH 24-25 - Xuan Vy Pham]
	Một hãng xe ôtô thống kê lại số lần gặp sự cố về động cơ của $100$ chiếc xe cùng loại sau $2$ năm sử dụng đầu tiên ở bảng sau. Hãy tìm khoảng tứ phân vị của mẫu số liệu ghép nhóm này? (Làm tròn các kết quả đến hàng phần trăm).
	
	\begin{center}
		\begin{tabular}{|c|c|c|c|c|c|}
			\hline
			Số lần gặp sự cố &$[ 0{,}5;2{,}5 )$&$[ 2{,}5;4{,}5 )$&$[ 4{,}5;6{,}5 )$&$[ 6{,}5;8{,}5 )$&$[ 8{,}5;10{,}5 )$\\
			\hline
			Số xe&$17$&$33$&$25$&$20$&$5$\\
			\hline
		\end{tabular}
	\end{center}
	\choice
	{ $5{,}32$}
	{\True $3{,}52$}
	{ $2{,}53$}
	{ $5{,}23$}
	\loigiai{
		Do cỡ mẫu $ n=100$.\\
		Gọi $x_1$; $x_2$; $\ldots$; ${x_{100}}$ là mẫu số liệu gốc gồm số lần gặp sự cố của $100$ chiếc xe cùng loại sau $2$ năm sử dụng.\\
		Ta có $x_1$,$\ldots$, ${x_{17}}$ $\in $ $[ 0{,}5;2{,}5 )$; ${x_{18}}$, $\ldots$, ${x_{50}}$ $\in $ $[ 2{,}5;4{,}5 )$; ${x_{51}}$, $\ldots$, ${x_{75}}$ $\in $ $[ 4{,}5;6{,}5 )$; ${x_{76}}$, $\ldots$, ${x_{95}}$ $\in $ $[ 6{,}5;8{,}5 )$; ${x_{96}}$, $\ldots$, ${x_{100}}$ $\in $ $[ 8{,}5;10{,}5 )$.\\
		Nên tứ phân vị thứ nhất của mẫu số liệu gốc là $\dfrac{1}{2}( {x_{25}}+{x_{26}} )\in [ 2{,}5;4{,}5 )$.\\ Do đó tứ phân vị thứ nhất của mẫu số liệu ghép nhóm là\\
		$$Q_1=2{,}5+\dfrac{\dfrac{100}{4}-17}{33}\cdot ( 4{,}5-2{,}5 ) \approx 2{,}98$$
		Tứ phân vị thứ ba của mẫu số liệu gốc là $\dfrac{1}{2}( {x_{75}}+{x_{76}} )$.\\
		Mà ${x_{75}}$ $\in $ $[ 4{,}5;6{,}5 )$; ${x_{76}}$ $\in $ $[ 6{,}5;8{,}5 )$ nên $Q_3=6{,}5$.\\
		Vậy khoảng tứ phân vị của mẫu số liệu ghép nhóm là $\Delta_Q=Q_3-Q_1 \approx 6{,}5-2{,}98=3{,}52$.
	}
\end{ex}
%Câu 5
\begin{ex}%[2D3H1-3]%[Dự án D - đợt 2 NH 24-25 - Xuan Vy Pham]
	Kiểm tra điện lượng của một số viên pin tiểu do một hãng sản xuất thu được kết quả sau. Hãy tìm khoảng tứ phân vị của mẫu số liệu ghép nhóm này? (Làm tròn các kết quả đến hàng phần trăm).
	
	\begin{center}
		\begin{tabular}{|c|c|c|c|c|c|}
			\hline
			Điện lượng 
			(nghìn mAh)&$[ 0{,}9;0{,}95 )$&$[ 0{,}95;1{,}0 )$&$[ 1{,}0;1{,}05 )$&$[ 1{,}05;1{,}1 )$&$[ 1{,}1;1{,}15 )$\\
			\hline
			Số viên pin&$10$ &$20$&$35$&$15$&$5$\\
			\hline
		\end{tabular}
	\end{center}
	\choice
	{ $0{,}06$}
	{ $0{,}08$}
	{\True $0{,}07$}
	{ $0{,}09$}
	\loigiai{
		Cỡ mẫu $ n=10+20+35+15+5=85$.\\
		Gọi $x_1$; $x_2$; $\ldots$; ${x_{85}}$ là mẫu số liệu gốc gồm điện lượng của $85$ viên pin tiểu.\\
		Ta có $x_1$, $\ldots$, ${x_{10}}$ $\in $ $[ 0{,}9;0{,}95 )$; ${x_{11}}$, $\ldots$, ${x_{30}}$ $\in $ $[ 0{,}95;1{,}0 )$; ${x_{31}}$, $\ldots$, ${x_{65}}$ $\in $ $[ 1{,}0;1{,}05 )$; ${x_{66}}$, $\ldots$, ${x_{80}}$ $\in $ $[ 1{,}05;1{,}1 )$; ${x_{81}}$, $\ldots$, ${x_{85}}$ $\in $ $[ 1{,}1;1{,}15 )$.\\
		Nên tứ phân vị thứ nhất của mẫu số liệu gốc là $\dfrac{1}{2}( {x_{21}}+{x_{22}} ) \in [ 0{,}95;1{,}0 )$. \\
		Do đó tứ phân vị thứ nhất của mẫu số liệu ghép nhóm là\\
		$$Q_1=0{,}95+\dfrac{\dfrac{85}{4}-10}{20}\cdot ( 1{,}0-0{,}95 )\approx 0{,}98$$
		Tứ phân vị thứ ba của mẫu số liệu gốc là $\dfrac{1}{2}( {x_{64}}+{x_{65}} )\in [ 1{,}0;1{,}05 )$. \\
		Do đó tứ phân vị thứ ba của mẫu số liệu ghép nhóm là\\
		$$Q_3=1{,}0+\dfrac{\dfrac{3\cdot 85}{4}-30}{35}\cdot ( 1{,}05-1{,}0 ) \approx 1{,}05$$
		Vậy khoảng tứ phân vị của mẫu số liệu ghép nhóm là $\Delta_Q=Q_3-Q_1 \approx 1{,}05-0{,}98=0{,}07$.}
\end{ex}
%Câu 6
\begin{ex}%[2D3H1-3]%[Dự án D - đợt 2 NH 24-25 - Xuan Vy Pham]
	Một mẫu số liệu ghép nhóm có khoảng tứ phân vị là $4{,}43$ và tứ phân vị thứ $3$ là $\dfrac{68}{3}$ thì giá trị ngoại lệ của mẫu số liệu ghép nhóm đó phải là bao nhiêu?
	\choice
	{\True $ x>29{,}31$}
	{ $ x>1{,}51$}
	{ $ x<51{,}23$}
	{ $ x\le 25{,}11$}
	\loigiai{
		Do tứ phân vị thứ $3$ là $\dfrac{68}{3}$ nên giá trị ngoại lệ $ x>Q_3+1{,}5\Delta_Q=\dfrac{68}{3}+1{,}5\cdot 4{,}43\approx 29{,}31$.
	}
\end{ex}
%Câu 7
\begin{ex}%[2D3H2-2]%[Dự án D - đợt 2 NH 24-25 - Xuan Vy Pham]
	Thời gian truy cập Internet mỗi buổi tối của một số học sinh được cho trong bảng sau
	\begin{center}
		\begin{tabular}{|c|c|c|c|c|c|}
			\hline
			Thời gian(phút) &$[9{,}5;12{,}5)$&$[12{,}5;15{,}5)$&$[15{,}5;18{,}5)$&$[18{,}5;21{,}5)$&$[21{,}5;24{,}5)$\\
			\hline
			Số học sinh &$3$&$12$&$15$&$24$&$2$\\
			\hline
		\end{tabular}
	\end{center}
	Độ lệch chuẩn của mẫu số liệu là (kết quả làm tròn đến hàng phần trăm)
	\choice
	{\True $2{,}93$}
	{$8{,}56$}
	{$8{,}59$}
	{$3{,}01$}
	\loigiai{
		\begin{center}
			\begin{tabular}{|c|c|c|c|c|c|}
				\hline
				Thời gian(phút) &$[9{,}5;12{,}5)$&$[12{,}5;15{,}5)$&$[15{,}5;18{,}5)$&$[18{,}5;21{,}5)$&$[21{,}5;24{,}5)$\\
				\hline
				Giá trị đại diện &$11$&$14$&$17$&$20$&$23$\\
				\hline
				Số học sinh &$3$&$12$&$15$&$24$&$2$\\
				\hline
			\end{tabular}
		\end{center}
		Số trung bình của mẫu số liệu là
		\[
		\overline{x} = \dfrac{1}{56} \cdot \left( 3\cdot 11 + 12\cdot 14 + 15\cdot17 + 24\cdot 20 + 2\cdot 23 \right) \approx 17{,}54.
		\]
		Phương sai của mẫu số liệu ghép nhóm là
		\[
		S^2 = \dfrac{1}{56} \left( 3\cdot 11^2 + 12\cdot 14^2 + 15\cdot 17^2 + 24\cdot 20^2 + 2\cdot 23^2 \right) - 17\cdot 54^2 \approx 8{,}56.
		\]
		Độ lệch chuẩn của mẫu số liệu ghép nhóm là 
		\[
		S = \sqrt{8{,}56} \approx 2{,}93.
		\]
	}
\end{ex}
%Câu 8
\begin{ex}%[2D3H2-2]%[Dự án D - đợt 2 NH 24-25 - Xuan Vy Pham]
	Một vận động viên luyện tập chạy cự li $100$ m đã ghi lại kết quả luyện tập như sau
	\begin{center}
		\begin{tabular}{|c|c|c|c|c|}
			\hline
			{Thời gian (giây)} & $[10{,}2; 10{,}4)$ & $[10{,}4; 10{,}6)$ & $[10{,}6; 10{,}8)$ & $[10{,}8; 11)$ \\
			\hline
			{Số vận động viên} & $3$ & $7$ & $8$ & $2$ \\
			\hline
		\end{tabular}
	\end{center}
	Tìm phương sai của mẫu số liệu ghép nhóm (làm tròn kết quả đến chữ số thập phân thứ $2$)
	\choice
	{$1{,}03$}
	{\True $0{,}03$}
	{$2{,}90$}
	{$1{,}86$}
	\loigiai{
		Ta có
		\begin{center}
			\begin{tabular}{|c|c|c|c|c|}
				\hline
				{Thời gian (giây)} & $[10{,}2; 10{,}4) $& $[10{,}4; 10{,}6)$ & $[10{,}6; 10{,}8)$ & $[10{,}8; 11)$ \\
				\hline
			{Giá trị đại diện} &$10{,}3$&$10{,}5$&$10{,}7$&$10{,}9$\\
				\hline
				{Số vận động viên} & $3$ & $7$ & $8$ & $2$ \\
				\hline
			\end{tabular}
		\end{center}
		Thời gian trung bình là
		\[
		\overline{x} = \dfrac{1}{20} \left( 10{,}3 \cdot 3 + 10{,}5 \cdot 7 + 10{,}7 \cdot 8 + 10{,}9 \cdot 2 \right) = 10{,}59.
		\]
		Phương sai
		\[
		s^2 = \dfrac{1}{20} \cdot \left( 10{,}3^2 \cdot 3 + 10{,}5^2 \cdot 7 + 10{,}7^2 \cdot 8 + 10{,}9^2 \cdot 2 \right) - 10{,}59^2 \approx 0{,}03.
		\]
	}
\end{ex}
%Câu 9
\begin{ex}%[2D3H2-2]%[Dự án D - đợt 2 NH 24-25 - Xuan Vy Pham]
	Bộ phận kiểm tra chất lượng sản phẩm dùng máy để đo (chính xác đến $0{,}001$ mm) độ dày của một chi tiết máy. Kết quả đo một số sản phẩm được thống kê trong bảng sau
	\begin{center}
		\begin{tabular}{|c|c|c|c|c|c|}
			\hline
			{Độ dày (mm)} & $[18; 19)$ & $[19; 20)$ & $[20; 21)$ & $[21; 22)$ & $[22; 23)$ \\
			\hline
			{Tần số} & $3$ & $7$ & $23$ & $25$ & $2$ \\
			\hline
		\end{tabular}
	\end{center}
	Nhận xét nào sau đây \textbf{sai}?
	\choice
	{\True Độ lệch chuẩn của mẫu lớn hơn $2$}
	{Số trung bình của mẫu số liệu gần bằng với $20{,}77$}
	{Độ dày của chi tiết máy không bị sai lệch nhiều}
	{Cỡ mẫu của mẫu số liệu là $60$}
	\loigiai{
		Ta có cỡ mẫu $n = 60$.\\
		Số trung bình của mẫu số liệu là
		\[
		\overline{x}= \dfrac{3 \cdot 18{,}5 + 7 \cdot 19{,}5 + 23 \cdot 20{,}5 + 25 \cdot 21{,}5 + 2 \cdot 22{,}5}{60} = \dfrac{623}{30} \approx 20{,}77.
		\]
		Phương sai của mẫu số liệu là
		\[
		S^2 = \dfrac{1}{60} \left( 3 \cdot 18{,}5^2 + 7 \cdot 19{,}5^2 + 23 \cdot 20{,}5^2 + 25 \cdot 21{,}5^2 + 2 \cdot 22{,}5^2 \right) - \left( \dfrac{623}{30} \right)^2 = \dfrac{179}{225}.
		\]
		Độ lệch chuẩn của mẫu số liệu là $S$
		\[
		S = \sqrt{\dfrac{179}{225}} = \dfrac{\sqrt{179}}{15} \approx 0{,}89.
		\]
	}
\end{ex}
%Câu 10
\begin{ex}%[2D3H2-1]%[Dự án D - đợt 2 NH 24-25 - Xuan Vy Pham]
Trong các khẳng định sau, khẳng định nào sai?
\choice
{Phương sai luôn luôn là số không âm}
{Phương sai là bình phương của độ lệch chuẩn}
{Phương sai càng lớn thì độ phân tán của các giá trị quanh số trung bình càng lớn}
{\True Phương sai luôn luôn lớn hơn độ lệch chuẩn}
\loigiai{
	Ta có khi $s \in (0;1)$ thì $s^2 < s$.\\
	 Do đó khẳng định phương sai luôn lớn hơn độ lệch chuẩn là sai.}
\end{ex}
%Câu 11
\begin{ex}%[2D3H2-2]%[Dự án D - đợt 2 NH 24-25 - Xuan Vy Pham]
	Điều tra $42$ học sinh của một lớp $12$ về số giờ tự học ở nhà, người ta có bảng thống kê sau
	\begin{longtable}{|>{\centering\arraybackslash}p{3cm}|>{\centering\arraybackslash}p{1.5cm}|>{\centering\arraybackslash}p{1.5cm}|>{\centering\arraybackslash}p{1.5cm}|>{\centering\arraybackslash}p{1.5cm}|>{\centering\arraybackslash}p{1.5cm}|}
		\hline
		Số giờ tự học & $[ 1;2 )$ & $[ 2;3 )$ & $[ 3;4 )$ & $[ 4;5 )$ & $[ 5;6 )$ \\ \hline
		Số học sinh & $8$ & $10$ & $12$ & $9$ & $3$ \\ \hline
	\end{longtable}
	Phương sai của mẫu số liệu ghép nhóm trên (làm tròn đến hàng phần trăm) có giá trị là
	\choice
	{\True $1{,}43$}
	{$1{,}42$}
	{$1{,}24$}
	{$1{,}23$}
	\loigiai{
		\begin{itemize}
			\item 	Chọn giá trị đại diện cho mẫu số liệu, ta có
			\begin{longtable}{|>{\centering\arraybackslash}p{3.5cm}|>{\centering\arraybackslash}p{1.5cm}|>{\centering\arraybackslash}p{1.5cm}|>{\centering\arraybackslash}p{1.5cm}|>{\centering\arraybackslash}p{1.5cm}|>{\centering\arraybackslash}p{1.5cm}|}
				\hline
				Số giờ tự học & $[ 1;2 )$ & $[ 2;3 )$ & $[ 3;4 )$ & $[ 4;5 )$ & $[ 5;6 )$ \\ \hline
				Giá trị đại diện & $1{,}5$ & $2{,}5$& $3{,5}$& $4{,}5$& $5{,}5$\\ \hline
				Số học sinh & $8$ & $10$ & $12$ & $9$ & $3$ \\ \hline
			\end{longtable}
			\item 	Số giờ học trung bình là
			\[\overline {x}=\dfrac{8\cdot 1{,}5+10\cdot 2{,}5+12\cdot 3{,}5+9\cdot 4{,}5+3\cdot 5{,}5}{42}=\dfrac{68}{21}\approx 3{,}238.\]
			\item 	Phương sai là
			\begin{eqnarray*}
				&S^2&=\dfrac{1}{42}\left[ 8\cdot ( 1{,}5 )^2+10\cdot ( 2{,}5 )^2+12\cdot ( 3{,}5 )^2+9\cdot ( 4{,}5 )^2+3\cdot ( 5{,}5 )^2 \right]-\left( \dfrac{68}{21} \right)^2\\
				&&=\dfrac{2525}{1764}\approx 1{,}43.
			\end{eqnarray*}
		\end{itemize}
	}
	
\end{ex}
%Câu 12
\begin{ex}%[2D3H2-2]%[Dự án D - đợt 2 NH 24-25 - Xuan Vy Pham]
	Khảo sát thời gian tự học bài ở nhà của học sinh khối $9$ ở trường $X$, ta thu được bảng sau
	\begin{center}
		\begin{tabular}{|c|c|c|c|c|c|}
			\hline
			Thời gian(phút) & $\left[0; 30\right)$ & $\left[30; 60\right)$ & $\left[60; 90\right)$ & $\left[90; 120\right)$ & $\left[120; 150\right)$ \\
			\hline
			Số học sinh & $9$ & $10$ & $9$ & $15$ & $7$ \\
			\hline
		\end{tabular}
	\end{center}	
	Phương sai của mẫu số liệu ghép nhóm bằng
	\choice
	{$1602$}
	{\True $1601{,}64$}
	{$1601{,}9$}
	{$1603$}
	\loigiai{
		\begin{center}
			\begin{tabular}{|c|c|c|c|c|c|}
				\hline
				Thời gian(phút) & $\left[0; 30\right)$ & $\left[30; 60\right)$ & $\left[60; 90\right)$ & $\left[90; 120\right)$ & $\left[120; 150\right)$ \\
				\hline
				Giá trị đại diện & $15$ & $45$ & $75$ & $105$ & $135$ \\
				\hline
				Số học sinh & $9$ & $10$ & $9$ & $15$ & $7$ \\
				\hline
			\end{tabular}
		\end{center}		
		Thời gian trung bình tự học ở nhà của các em học sinh đó là
		\[\bar{x}=\dfrac{9\cdot 15+10\cdot 45+9\cdot 75+15\cdot 105+7\cdot 135}{50}=75{,}6.\]
		Phương sai của mẫu số kiệu ghép nhóm là
		\[s^2=\dfrac{1}{50} \left(9\cdot 15^2+10\cdot 45^2+9\cdot 75^2+15\cdot 105^2+7\cdot 135^2\right)-75{,}6^2=1601{,}64.\]
	}
\end{ex}
\Closesolutionfile{ans}
%\indapan{6}{ans-ABCD}
%\cauds
\begin{center}
	\textbf{PHẦN 2 - Câu trắc nghiệm đúng sai. Trong mỗi ý a, b, c, d ở mỗi câu, thí sinh chọn đúng hoặc sai}
\end{center}
\setcounter{ex}{0}
\Opensolutionfile{ans}[ans-DS]
%Câu 1
\begin{ex}%[2D3H1-2]%[Dự án D - đợt 2 NH 24-25 - Xuan Vy Pham]
	Thời gian chờ khám bệnh của hai phòng phám $1$ và phòng khám $2$ ở thành phố $X$ được cho trong bảng sau:
	\begin{longtable}[c]
		{|c|c|c|c|c|} \hline
		Thời gian (phút)	&$[0;5)$	&$[5;10)$&	$[10;15)$&	$[15;20)$\\
		\hline
		Phòng khám số $1$
		(Số bệnh nhân)	&$3$&$12$&$15$&$18$\\
		\hline
		Phòng khám số $2$
		(Số bệnh nhân)	&$5$&$10$&$12$&$0$\\
		\hline
	\end{longtable}
	\choiceTF
	{\True Tổng số bệnh nhân chờ khám bệnh ở phòng khám số $1$ dưới $5$ phút là $3$}
	{Khoảng biến thiên của mẫu số liệu ghép nhóm về thời gian chờ khám bệnh của phòng khám số $1$ là ${R_1}=15$}
	{Khoảng biến thiên của mẫu số liệu ghép nhóm về thời gian chờ khám bệnh của phòng khám số $2$ là ${R_2}=20$}
	{Thời gian chờ khám bệnh ở phòng khám số $2$ phân tán hơn thời gian chờ khám bệnh ở phòng khám số $1$}
	\loigiai{
		\begin{itemchoice}
			\itemch  Tổng số bệnh nhân chờ khám bệnh ở phòng khám số $1$ dưới $5$ phút là $3$. 
			\itemch 
			Khoảng biến thiên của mẫu số liệu ghép nhóm về thời gian chờ khám bệnh của phòng khám số $1$ là ${R_1}=20-0=20$.
			\itemch 
			Khoảng biến thiên của mẫu số liệu ghép nhóm về thời gian chờ khám bệnh của phòng khám số $2$ là ${R_2}=15-0=15$.
			\itemch  
			Vì ${R_1} > {R_2}$ nên thời gian khám bệnh ở phòng khám số $1$ phân tán hơn thời gian chờ khám bệnh ở phòng khám số $2$.
		\end{itemchoice}
	}
\end{ex}
%Câu 2
\begin{ex}%[2D3V2-2]%[Dự án D - đợt 2 NH 24-25 - Xuan Vy Pham]
	Kết quả kiểm tra môn Tiếng Anh (cùng một đề) của học sinh hai lớp 12A và 12B được cho lần lượt bởi mẫu số liệu ghép nhóm ở Bảng A và Bảng B.	
	\begin{center}
		\begin{tabular}{cc}	
			
			\begin{tabular}{|c|c|c|}
				\hline
				{Nhóm} & {Giá trị đại diện} & {Tần số} \\
				\hline
				$[0; 2)$ & $1$ & $3$ \\
				\hline
				$[2; 4)$ & $3$ & $5$ \\
				\hline
				$[4; 6)$ & $5$ & $5$ \\
				\hline
				$[6; 8)$ & $7$ & $25$ \\
				\hline
				$[8; 10)$ & $9$ & $2$ \\
				\hline
				& & $n = 40$ \\
				\hline
			\end{tabular}&
			\begin{tabular}{|c|c|c|}
				\hline
				{Nhóm} & {Giá trị đại diện} & {Tần số} \\
				\hline
				$[0; 2)$ & $1$ & $1$ \\
				\hline
				$[2; 4)$ & $3$ & $4$ \\
				\hline
				$[4; 6)$ & $5$ & $15$ \\
				\hline
				$[6; 8)$ & $7$ & $16$ \\
				\hline
				$[8; 10)$ & $9$ & $4$ \\
				\hline
				& & $n = 40$ \\
				\hline
			\end{tabular}  \\
			& \\
			{\textit{Bảng A}}&{\textit{Bảng B}}  \\
		\end{tabular}
	\end{center}
	\choiceTF
	{Độ lệch chuẩn của mẫu số liệu lớp $12A$ nhỏ hơn $2$}
	{\True Phương sai của mẫu số liệu lớp $12B$ lớn hơn $3$}
	{\True Số trung bình cộng của hai mẫu số liệu trên bằng nhau}
	{\True Dựa vào độ lệch chuẩn ta thấy điểm thi của học sinh lớp $12B$ đồng đều hơn lớp $12A$}
	\loigiai{
		\begin{itemize}
			\item 
			{Số trung bình cộng của mẫu số liệu lớp $12A$}
			\[
			\overline{x}_A = \dfrac{3 \cdot 1 + 5 \cdot 3 + 5 \cdot 5 + 25 \cdot 7 + 2 \cdot 9}{40} = \dfrac{59}{10} = 5,9.
			\]
			\item {Phương sai và độ lệch chuẩn của mẫu số liệu lớp $12A$}
			\begin{eqnarray*}
				s_A^2 &=& \dfrac{1}{40} \left[3 \cdot (1 - 5,9)^2 + 5 \cdot (3 - 5,9)^2 + 5 \cdot (5 - 5,9)^2\right. \\
				&&\left.+ 25 \cdot (7 - 5,9)^2 + 2 \cdot (9 - 5,9)^2\right]\\
				&=&  4{,}19.
			\end{eqnarray*}
			Độ lệch chuẩn của mẫu số liệu lớp $12A$ là
			\[
			s_A = \sqrt{4{,}19} \approx 2{,}05.
			\]
			\item {Số trung bình cộng của mẫu số liệu lớp $12B$}
			\[
			\overline{x}_B = \dfrac{1 \cdot 1 + 4 \cdot 3 + 15 \cdot 5 + 16 \cdot 7 + 4 \cdot 9}{40} = \dfrac{59}{10} = 5{,}9.
			\]			
			{Phương sai và độ lệch chuẩn của mẫu số liệu lớp $12B$}
			\begin{eqnarray*}
				s_B^2 &=& \dfrac{1}{40} \left[1 \cdot (1 - 5,9)^2 + 4 \cdot (3 - 5,9)^2 + 15 \cdot (5 - 5,9)^2\right.\\
				&&\left.+ 16 \cdot (7 - 5,9)^2 + 4 \cdot (9 - 5,9)^2\right] \\
				&=&3{,}19.
			\end{eqnarray*}
			Độ lệch chuẩn của mẫu số liệu lớp $12B$ là
			\[
			s_B = \sqrt{3{,}19} \approx 1{,}79.
			\]
		\end{itemize}
		Khi đó 
		\begin{itemchoice} 
			\itemch 
			Độ lệch chuẩn của lớp $12A$ là $2{,}05$, lớn hơn $2$ 
			\itemch 
			Phương sai của lớp $12B$ là $3{,}19$, nhỏ hơn $3$
			\itemch 
			Số trung bình cộng của hai mẫu bằng nhau và bằng $5,9$.
			\itemch 
			Độ lệch chuẩn của lớp $12B$ nhỏ hơn lớp $12A$, do đó điểm thi của lớp $12B$ đồng đều hơn lớp $12A$.
		\end{itemchoice}
	}
\end{ex}
\Closesolutionfile{ans}
\begin{center}
	\textbf{PHẦN 3 - Câu trắc nghiệm trả lời ngắn}
\end{center}
\setcounter{ex}{0}
%Câu 1
\begin{ex}%[2D3H1-2]%[Dự án D - đợt 2 NH 24-25 - Xuan Vy Pham]
	Cho bảng thống kê thời gian tập thể dục buổi sáng mỗi ngày trong tháng 9/2022 của Bác An và Bác Bình. 
	\begin{longtable}[c]
		{|c|c|c|c|c|c|} \hline
		Thời gian (phút)	&$[15;20)$	&$[20;25)$	&$[25;30)$	&$[30;35)$	&$[25;40)$\\
		\hline
		Bác An&$5$&$12$&$8$&$3$&$2$\\
		\hline
		Bác Bình&$0$&$20$&$5$&$5$&$0$\\
		\hline
	\end{longtable}
	Gọi $R_A, R_B$ lần lượt là khoảng biến thiên của mẫu số liệu về thời gian tập thể dục của Bác An và Bác Bình, khi đó $R_A+R_B$ bằng
	\shortans{$40$}
	\loigiai{
		Ta có $R_A=40-15=25$; $R_B=35-20=15$, suy ra $R_A+R_B=25+15=40$.}
\end{ex}
%Câu 2
\begin{ex}%[2D3H1-2]%[Dự án D - đợt 2 NH 24-25 - Xuan Vy Pham]
	Thống kê số thẻ vàng của mỗi câu lạc bộ trong giải ngoại hạng Anh mùa giải $2021-2022$ cho kết quả như sau
	\begin{longtable}[c]
		{|c|c|c|c|c|c|c|c|c|c|} \hline
		$101$&$79$&$79$&$78$&$75$&$73$&$68$&$67$&$67$&$63$\\
		\hline
		$63$&$61$	&$60$	&$59$&$57$&$55$&$55$&	$50$&$47$&$42$\\
		\hline
	\end{longtable}
	\noindent	Tính khoảng tứ phân vị của mẫu số liệu ghép nhóm dãy số liệu trên thành các nhóm có độ dài bằng nhau với nhóm đầu tiên là $[40;50)$.
	
	\shortans{$16$}
	\loigiai{
		Bảng số liệu ghép nhóm có dạng
		\begin{longtable}[c]
			{|c|c|c|c|c|c|c|c|} \hline
			Số thẻ	&$[40;50)$	&$[50;60)$	&$[60;70)$	&$[70;80)$	&$[80;90)$	&$[90;100)$	&$[100;110)$\\
			\hline
			Tần số	&$2$	&$5$	&$7$	&$5$	&$0$	&$0$	&$1$\\
			\hline
		\end{longtable}
		\noindent Cỡ mẫu $n=20$.\\
		Gọi ${x_1};{x_2};\ldots;{x_{20}}$ là số thẻ vàng của mỗi câu lạc bộ trong giải ngoại hạng Anh mùa giải $2021-2022$, các giá trị này đã được xếp theo thứ tự không giảm.\\
		Tứ phân vị thứ nhất của mẫu số liệu gốc là $\dfrac{{x_5}+{x_6}}{2}\in [50;60)$.\\ 
		Do đó, tứ phân vị thứ nhất của mẫu số liệu ghép nhóm là
		\begin{align*}Q_1=50+\dfrac{\dfrac{20}{4}-2}{5}\cdot 10=56.
		\end{align*}
		Tứ phân vị thứ ba của mẫu số liệu gốc là $\dfrac{x_{15}+{x_{16}}}{2}\in [70;80)$. \\
		Do đó, tứ phân vị thứ ba của mẫu số liệu ghép nhóm là
		\begin{align*}Q_3=70+\dfrac{\dfrac{3\cdot 20}{4}-(2+5+7)}{5}\cdot 10=72.
		\end{align*}
		Khoảng tứ phân vị của mẫu số liệu ghép nhóm là ${{\Delta}_Q}={Q_3}-{Q_1}=16$.}
\end{ex}
%Câu 3...........................
\begin{ex}%[2D3V2-2]%[Dự án D - đợt 2 NH 24-25 - Xuan Vy Pham]
	Điều tra $42$ học sinh của một lớp $12$ về số giờ tự học ở nhà, người ta có bảng thống kê sau
	\begin{center}
		\begin{tabular}{|c|c|c|c|c|c|}
			\hline
		{Số giờ tự học} & $[1; 2)$ & $[2; 3)$ & $[3; 4)$ & $[4; 5)$ & $[5; 6)$ \\
			\hline
		{Số học sinh} & $8$ & $10$ & $12$ & $9$ & $3$ \\
			\hline
		\end{tabular}
	\end{center}
	Tính phương sai của mẫu số liệu ghép nhóm trên.
	\shortans{$1{,}431$}
	\loigiai{
		Chọn giá trị đại diện cho mẫu số liệu, ta có
		\begin{center}
			\begin{tabular}{|c|c|c|c|c|c|}
				\hline
			{Số giờ tự học} & $[1; 2)$ & $[2; 3)$ & $[3; 4)$ & $[4; 5)$ & $[5; 6) $\\
				\hline
			{Giá trị đại diện} & $1{,}5$ & $2{,}5$ & $3{,}5$ & $4{,}5$ & $5{,}5$ \\
				\hline
			{Số học sinh} & $8$ & $10$ & $12$ & $9$ & $3 $\\
				\hline
			\end{tabular}
		\end{center}
		Số giờ học trung bình là
		\[
		\overline{x} = \dfrac{8 \cdot 1{,}5 + 10 \cdot 2{,}5 + 12 \cdot 3{,}5 + 9 \cdot 4{,}5 + 3 \cdot 5{,}5}{42} = \dfrac{68}{21} \approx 3{,}238.
		\]
		Phương sai là
		\begin{align*}
		s^2 &= \dfrac{1}{42} \left[ 8 \cdot \left( 1{,}5 \right)^2 + 10 \cdot \left( 2{,}5 \right)^2 + 12 \cdot \left( 3{,}5 \right)^2 + 9 \cdot \left( 4{,}5 \right)^2 + 3 \cdot \left( 5{,}5 \right)^2 \right] - \left( \dfrac{68}{21} \right)^2 \\
		&= \dfrac{2525}{1764} \approx 1{,}431.
		\end{align*}
	}
\end{ex}
%Câu 4
\begin{ex}%[2D3V2-2]%[Dự án D - đợt 2 NH 24-25 - Xuan Vy Pham]
	Bạn Mai ghi lại thời gian sử dụng điện thoại di động mỗi ngày của mình trong $10$ ngày liên tiếp ở bảng sau (đơn vị: phút)
	\begin{center}
		\begin{tabular}{|l|l|l|l|l|l|l|l|l|l|}
			\hline $150$ & $251$ & $73$ & $188$ & $165$ & $225$ & $235$ & $144$ & $160$ & $244$ \\
			\hline
		\end{tabular}
	\end{center}
	Bạn Mai ghép số liệu trên thành $4$ nhóm có độ dài bằng nhau, với nhóm cuối cùng là $[220 ; 270$ ). Tính tỉ số giữa độ lệch chuẩn và trung bình mẫu của mẫu số liệu ghép nhóm (kết quả làm tròn đến hàng phần trăm).
	\shortans{$0{,}29$}
	\loigiai{
		Bảng số liệu ghép nhóm
		\begin{center}
			\begin{tabular}{|l|c|c|c|c|}
				\hline Thời gian sử dụng & {$[70 ; 120)$} & {$[120 ; 170)$} & {$[170 ; 220)$} & {$[220 ; 270)$} \\
				\hline Số ngày & 1 & 4 & 1 & 4 \\
				\hline
			\end{tabular}
		\end{center}
		Số trung bình $\overline{x}=\dfrac{1\cdot 145+4\cdot 245+1\cdot 345+4\cdot 445}{10}=185$.\\
		Phương sai là
		\begin{align*}
		s^2&=\dfrac{1\cdot (145-\overline{x})^2+4\cdot (245-\overline{x})^2+1\cdot (345-\overline{x})^2+4\cdot (445-\overline{x})^2}{10}\\
		&=\dfrac{1\cdot (145-185)^2+4\cdot (245-185)^2+1\cdot (345-185)^2+4\cdot (445-185)^2}{10}\\
		&=2\,900.	
		\end{align*}
		Độ lệch chuẩn $s=\sqrt{s^2}=\sqrt{2\,900}$.\\
		Do đó $\dfrac{s}{\overline{x}}\approx 0{,}29$.
	}
\end{ex}
\Closesolutionfile{ansKQ}
\begin{center}
	\textbf{PHẦN 4 - Tự luận.}
\end{center}
\setcounter{ex}{0}
%Câu 1
\begin{ex}%[2D3H1-2]%[Dự án D - đợt 2 NH 24-25 - Xuan Vy Pham]
	Bạn Trang thống kê lại chiều cao (đơn vị: cm) của các bạn học sinh nữ lớp $12C$ ở bảng sau
	\begin{longtable}{|c|c|c|c|c|c|c|}
		\hline
		Chiều cao &$ [155; 160) $&$ [160; 165)$ &$ [165; 170) $&$ [170; 175)$ &$ [175; 180) $&$ [180; 185]$ \\ \hline
		Tần số & $2$ & $7$ & $12$ & $3$ & $0$ & $1$ \\ \hline
	\end{longtable}
	\noindent	Bạn Trang nhận xét như sau: Chênh lệch chiều cao của các bạn trong lớp không vượt quá $a$(cm). Hãy xác định giá trị của $a$ để nhận xét của Trang là đúng.
	\loigiai{
		Ta có $R=185-155=30$.\\
		Vậy giá trị của $a=30$.}
\end{ex}
%Câu 2
\begin{ex} %[2D3V2-2]%[Dự án D - đợt 2 NH 24-25 - Xuan Vy Pham] 
	Thống kê điểm trắc nghiệm môn Tiếng Anh của $40$ học sinh, người ta có bảng sau
	\begin{longtable}{|>{\centering\arraybackslash}p{3cm}|>{\centering\arraybackslash}p{1.3cm}|>{\centering\arraybackslash}p{1.3cm}|>{\centering\arraybackslash}p{1.3cm}|>{\centering\arraybackslash}p{1.3cm}|>{\centering\arraybackslash}p{1.3cm}|>{\centering\arraybackslash}p{1.3cm}|>{\centering\arraybackslash}p{1.3cm}|>{\centering\arraybackslash}p{1.3cm}|}
		\hline
		Điểm & $[20;30)$ & $[30;40)$ & $[40;50)$ & $[50;60)$ & $[60;70)$ & $[70;80)$ & $[80;90)$ & $[90;100]$ \\ \hline
		Số học sinh & $3$ & $5$ & $5$ & $8$ & $7$ & $5$ & $3$ & $4$ \\ \hline
	\end{longtable}
	\noindent	Tính phương sai của mẫu số liệu ghép nhóm trên (kết quả làm tròn đến hàng đơn vị).
	\loigiai{
		\begin{itemize}
			\item 	Chọn giá trị đại diện cho mẫu số liệu, ta có	\begin{longtable}{|>{\centering\arraybackslash}p{3cm}|>{\centering\arraybackslash}p{1.3cm}|>{\centering\arraybackslash}p{1.3cm}|>{\centering\arraybackslash}p{1.3cm}|>{\centering\arraybackslash}p{1.3cm}|>{\centering\arraybackslash}p{1.3cm}|>{\centering\arraybackslash}p{1.3cm}|>{\centering\arraybackslash}p{1.3cm}|>{\centering\arraybackslash}p{1.3cm}|}
				\hline
				Điểm & $[20;30)$ & $[30;40)$ & $[40;50)$ & $[50;60)$ & $[60;70)$ & $[70;80)$ & $[80;90)$ & $[90;100]$ \\ \hline
				Giá trị đại diện & $25$&$35$&$45$& $55$& $65$& $75$&$85$&$95$ \\ \hline
				Số học sinh & $3$ & $5$ & $5 $& $8$ & $7$ & $5$ & $3$ & $4$ \\ \hline
			\end{longtable}
			\item	Điểm trung bình là
			\[\overline{x}=\dfrac{3\cdot 25+5\cdot 35+5\cdot 45+8\cdot 55+7\cdot 65+5\cdot 75+3\cdot 85+4\cdot 95}{40}=59{,}5.\]
			\item	Phương sai là
			\begin{eqnarray*}
				&s^2&=\dfrac{1}{40}\left[ 3\cdot  25 ^2+5\cdot  35^2+5\cdot  45^2+8\cdot  55 ^2+7\cdot  65^2+5\cdot  75 ^2+3\cdot  85 ^2+4\cdot 95 ^2 \right]- 59{,}5^2 \\ &&\approx 405. 
			\end{eqnarray*}
		\end{itemize} 
	}
\end{ex}
%Câu 3
\begin{ex} %[2D3V2-2]%[Dự án D - đợt 2 NH 24-25 - Xuan Vy Pham]
	Thống kê doanh thu (đơn vị: triệu đô la) của $20$ công ty sản xuất ô tô trong năm $2\,023$, người ta có bảng sau
	\begin{longtable}{|>{\centering\arraybackslash}p{2.5cm}|>{\centering\arraybackslash}p{2cm}|>{\centering\arraybackslash}p{2cm}|>{\centering\arraybackslash}p{2cm}|>{\centering\arraybackslash}p{2cm}|>{\centering\arraybackslash}p{2cm}|}
		\hline
		Doanh thu & $[0;20)$ & $[20;40)$ & $[40;60)$ & $[60;80)$ & $[80;100]$ \\ \hline
		Số công ty &  $5$ & $5$ & $6$ & $2$ & $2$ \\ \hline
	\end{longtable}
	\noindent Tính độ lệch chuẩn của mẫu số liệu ghép nhóm trên (kết quả làm tròn đến hàng phần chục).
	\loigiai{	
		\begin{itemize}
			\item Chọn giá trị đại diện cho mẫu số liệu, ta có
			\begin{longtable}{|>{\centering\arraybackslash}p{3cm}|>{\centering\arraybackslash}p{2cm}|>{\centering\arraybackslash}p{2cm}|>{\centering\arraybackslash}p{2cm}|>{\centering\arraybackslash}p{2cm}|>{\centering\arraybackslash}p{2cm}|}
				\hline
				Doanh thu & $[0;20)$ & $[20;40)$ & $[40;60)$ & $[60;80)$ & $[80;100]$ \\ \hline
				Giá trị đại diện & $10$ & $30$ &$50$ &$70$ &$90$\\ \hline
				Số công ty & $5$ & $5$ & $6$ & $2$ & $2$ \\ \hline
			\end{longtable}
			\item 	Điểm trung bình là
			\[\overline{x}=\dfrac{5\cdot 10+5\cdot 30+6\cdot 50+2\cdot 70+2\cdot 90}{20}=41.\]
			\item 	Phương sai là
			\[s^2=\dfrac{1}{20}\left[ 5\cdot  10^2+5\cdot 30^2+6\cdot 50 ^2+2\cdot  70^2+2\cdot  90^2 \right]- 41^2=619.\]
			\item	Độ lệch chuẩn là $s=\sqrt {619}\approx 24{,}9$.
	\end{itemize}}
\end{ex}
