\newpage
\def\thoigian{90}%--Thời gian
\de{Đề số 2}{Chương V. Phương trình mặt phẳng, đường thẳng, mặt cầu}


\begin{center}
	\textbf{PHẦN 1 - Câu trắc nghiệm nhiều phương án lựa chọn.}
\end{center}
\setcounter{ex}{0}
\Opensolutionfile{ans}[ans-ABCD]
%Câu 1
\begin{ex}%[2H5N1-2]%[Dự án D - đợt 2 NH 24-25 - Xuan Vy Pham]
	Trong không gian $Oxyz$, cho mặt phẳng $(\alpha)\colon 3x+2y-4z+1=0$. Vectơ nào dưới đây là một vectơ pháp tuyến của $(\alpha)$?
	\choice
	{$\overrightarrow{n}_2=(3;2;4)$}
	{$\overrightarrow{n}_3=(2;-4;1)$}
	{$\overrightarrow{n}_1=(3;-4;1)$}
	{\True $\overrightarrow{n}_4=(3;2;-4)$}
	\loigiai{
		Vectơ $\overrightarrow{n}_4=(3;2;-4)$ là một vectơ pháp tuyến của $(\alpha)$.
	}
\end{ex} 
%Câu 2
\begin{ex}%[2H5N1-3]%[Dự án D - đợt 2 NH 24-25 - Xuan Vy Pham]
	Trong không gian $Oxyz$, cho điểm $M(m;1;6)$ và mặt phẳng $(P)\colon x-2y+z-5=0$. Điểm 
	$M$ thuộc mặt phẳng $(P)$ khi giá trị của tham số $m$ bằng
	\choice
	{\True $m=1$}
	{$m=-1$}
	{$m=3$}
	{$m=2$}
	\loigiai{Điểm 
		$M$ thuộc mặt phẳng $(P)$ khi $m-2\cdot1+6-5=0\Leftrightarrow m=1$.}
\end{ex}
%Câu 3
\begin{ex}%[2H5N1-5]%[Dự án D - đợt 2 NH 24-25 - Xuan Vy Pham]
	Trong không gian $Oxyz$, cho phương trình mặt phẳng $(P)\colon 2x-2y+z+1=0$ và điểm $M(2;-1;3)$. Khoảng cách điểm $M$ đến mặt phẳng $(P)$ bằng 
	\choice
	{$2$}
	{$3$}
	{$\dfrac{5}{3}$}
	{\True $\dfrac{10}{3}$}
	\loigiai{Khoảng cách điểm $M$ đến mặt phẳng $(P)$ là 
		$$\mathrm{d}\left( M,(P)\right)=\dfrac{\left|2\cdot 2-2\cdot(-1)+3+1 \right| }{\sqrt{2^2+2^2+1^2}}=\dfrac{10}{3}.$$}
\end{ex}
%Câu 4
\begin{ex}%[2H5N1-3]%[Dự án D - đợt 2 NH 24-25 - Xuan Vy Pham]
	Trong không gian $Oxyz$, cho ba điểm $A(2;0;0)$, $B(0;-1;0)$, $C(0;0;3)$. Mặt phẳng $(ABC)$ có phương trình là
	\choice
	{$\dfrac{x}{-2}+\dfrac{y}{1}+\dfrac{z}{3}=1$}
	{$\dfrac{x}{2}+\dfrac{y}{1}+\dfrac{z}{-3}=1$}
	{$\dfrac{x}{2}+\dfrac{y}{1}+\dfrac{z}{3}=1$}
	{\True $\dfrac{x}{2}+\dfrac{y}{-1}+\dfrac{z}{3}=1$} 
	\loigiai{
		Phương trình mặt phẳng $(ABC)$ là $\dfrac{x}{2}+\dfrac{y}{-1}+\dfrac{z}{3}=1$. 
	}
\end{ex}
%Câu 5
\begin{ex}%[2H5N2-2]%[Dự án D - đợt 2 NH 24-25 - Xuan Vy Pham]
	Trong không gian $Oxyz$, đường thẳng $d\colon\heva{&x=2-t \\
		&y=1+2t \\
		&z=3+t
	}$ có một vectơ chỉ phương là
	\choice
	{$\overrightarrow{u}_3=(2;1;3)$}
	{\True $\overrightarrow{u}_4=(-1;2;1)$}
	{$\overrightarrow{u}_2=(2;1;1)$}
	{$\overrightarrow{u}_1=(-1;2;3)$}
	\loigiai{
		Đường thẳng $d\colon\heva{&x=2-t \\
			&y=1+2t \\
			&z=3+t
		}$ có một vectơ chỉ phương là $\overrightarrow{u}_4=(-1;2;1)$.
	}
\end{ex}
%Câu 6
\begin{ex}%[2H5N2-1]%[Dự án D - đợt 2 NH 24-25 - Xuan Vy Pham]
	Trong không gian $Oxyz$, cho đường thẳng $d\colon \dfrac{x-1}{2}=\dfrac{y-2}{3}=\dfrac{z+1}{-1}$. Điểm nào dưới đây thuộc $d$?
	\choice
	{\True $P(1;2;-1)$}
	{$M(-1;-2;1)$}
	{$N(2;3;-1)$}
	{$Q(-2;-3;1)$}
	\loigiai{Ta có $\dfrac{1-1}{2}=\dfrac{2-2}{3}=\dfrac{-1+1}{-1}$ nên $P\in d$.}
\end{ex}
%Câu 7
\begin{ex}%[2H5N2-7]%[Dự án D - đợt 2 NH 24-25 - Xuan Vy Pham]
	Góc giữa hai đường thẳng $d\colon \heva{&x=-3+t \\&y=-\sqrt{2}t\\&z=1+t}$ và $d'\colon \dfrac{x+1}{1}=\dfrac{y-1}{\sqrt{2}}=\dfrac{z-3}{-1}$ bằng
	\choice
	{$45^{\circ}$}
	{$30^{\circ}$}
	{\True $60^{\circ}$}
	{$90^{\circ}$}
	\loigiai{
		Ta có vectơ chỉ phương của đường thẳng $d$ là $\overrightarrow{u}=\left(1;-\sqrt{2};1\right)$.\\
		Vectơ chỉ phương của đường thẳng $d'$ là $\overrightarrow{u'}=\left(1;\sqrt{2};-1\right)$.\\
		$\cos \left(d;d'\right)=\dfrac{\left|\overrightarrow{u}\cdot \overrightarrow{u'}\right|}{\left|\overrightarrow{u}\right|\cdot\left|\overrightarrow{u'}\right|}=\dfrac{\left|1-2-1\right| }{\sqrt{1+2+1}\cdot \sqrt{1+2+1}}=\dfrac{1}{2}$.\\
		Suy ra $\left(d;d'\right)=60^{\circ}$.
	}
\end{ex}
%Câu 8
\begin{ex}%[2H5N2-3]%[Dự án D - đợt 2 NH 24-25 - Xuan Vy Pham]
	Trong không gian $Oxyz$, phương trình tham số của đường thẳng đi qua điểm $M(2; 0;-1)$ và
	có vectơ chỉ phương $\overrightarrow{a}=(2;-3; 1)$ là
	\choice
	{$\heva{&x=4+2t \\& y=-6\\& z=2-t}$}
	{$\heva{&x=-2+2t \\& y=-3t \\& z=1+t}$}
	{$\heva{&x=-2+4t \\& y=-6t \\& z=1+2t}$}
	{\True $\heva{&x=2+2t \\& y=-3t \\& z=-1+t}$}
	\loigiai{
		Phương trình tham số của đường thẳng đi qua điểm $M(2; 0;-1)$ và
		có vectơ chỉ phương $\overrightarrow{a}=(2;-3; 1)$ là $\heva{&x=2+2t \\& y=-3t \\& z=-1+t}$ $t\in\mathbb{R}$.
	}
\end{ex}
%Câu 9
\begin{ex}%[2H5H3-2]%[Dự án D - đợt 2 NH 24-25 - Xuan Vy Pham]
	Trong không gian với hệ tọa độ $Oxyz$, cho mặt cầu \\$(S) \colon (x+3)^2+(y+1)^2+(z-1)^2=2$. Tìm tọa độ tâm $I$ và bán kính $R$ của mặt cầu $(S)$.
	\choice
	{$I(3;1;-1)$ và $R=2$}
	{$I(3;1;-1)$ và $R=\sqrt{2}$}
	{$I(-3;-1;1)$ và $R=2$}
	{\True $I(-3;-1;1)$ và $R=\sqrt{2}$}
\loigiai{Mặt cầu $(S) \colon (x+3)^2+(y+1)^2+(z-1)^2=2$ có tâm là $I(-3;-1;1)$ và $R=\sqrt{2}$.}
\end{ex}
%Câu 10
\begin{ex}%[2H5N3-2]%[Dự án D - đợt 2 NH 24-25 - Xuan Vy Pham]
	Trong không gian $Oxyz$, cho mặt cầu $(S): x^2+y^2+z^2-2x-4y-6z+5=0$. Diện tích của mặt cầu $(S)$ bằng
	\choice
	{$42\pi$}
	{\True $36\pi$}
	{$9\pi$}
	{$12\pi$}
	\loigiai{
		Mặt cầu $(S)$ có tâm $I(1;2;3)$ và bán kính $R=\sqrt{1^2+2^2+3^2-5}=3$.\\
		Diện tích mặt cầu $4\pi R^2= 36\pi$.
	}
\end{ex}
%Câu 11
\begin{ex}%[2H5H3-3]%[Dự án D - đợt 2 NH 24-25 - Xuan Vy Pham]
		Lập phương trình mặt cầu $(S)$ biết mặt cầu $(S)$ có đường kính $MN$ với $M(0;-3; 4)$ và $N(2; 3; 0)$.
		\choice
		{\True $(x - 1)^2 + y^2 + (z - 2)^2 = 14$}
		{$(x - 1)^2 + y^2 + (z - 2)^2 = \sqrt{14}$}
		{$(x - 1)^2 + y^2 + (z - 2)^2 = 2\sqrt{3}$}
		{$(x - 1)^2 + y^2 + (z - 2)^2 = 12$}
		\loigiai{Gọi $I$ là trung điểm $MN$, khi đó $I= \left(\dfrac{0 + 2}{2}; \dfrac{-3 + 3}{2}; \dfrac{4 + 0}{2}\right)=(1; 0; 2)$.\\ 
			Bán kính $R= \sqrt{(1- 0)^2 + (0 + 3)^2 + (2 - 4)^2} = \sqrt{14}$. \\
			Phương trình mặt cầu là: $$(S)\colon (x - 1)^2 + y^2 + (z - 2)^2 = 14.$$}
\end{ex}
%Câu 12
\begin{ex}%[2H5H3-3]%[Dự án D - đợt 2 NH 24-25 - Xuan Vy Pham]
	Trong không gian $Oxyz$, phương trình mặt cầu $(S)$ có tâm $I(1;-1;3)$ và tiếp xúc với trục hoành $Ox$ là
	\choice
	{$(x+1)^2+(y-1)^2+(z+3)^2=10$}
	{$(x-1)^2+(y+1)^2+(z-3)^2=9$}
	{\True $(x-1)^2+(y+1)^2+(z-3)^2=10$}
	{$(x+1)^2+(y-1)^2+(z+3)^2=9$}
	\loigiai{
		Gọi $H$ là hình chiếu của $I$ lên $Ox$, ta có $H(1;0;0)$.\\
		Ta có $\overrightarrow{IH}=(0;1;-3) \Rightarrow R=IH=\sqrt{10}$.\\
		Phương trình mặt cầu $(S): (x-1)^2+(y+1)^2+(z-3)^2=10$.
	}
\end{ex}
\Closesolutionfile{ans}
%\indapan{6}{ans-ABCD}
%\cauds
\begin{center}
	\textbf{PHẦN 2 - Câu trắc nghiệm đúng sai. Trong mỗi ý a, b, c, d ở mỗi câu, thí sinh chọn đúng hoặc sai}
\end{center}
\setcounter{ex}{0}
\Opensolutionfile{ans}[ans-DS]
%Câu 1
\begin{ex}%[2H5H2-3]%[Dự án D - đợt 2 NH 24-25 - Xuan Vy Pham]
	Trong không gian $Oxyz$, cho tam giác $ABC$ có $A(2; 0; 1)$, $B(0; 2;-1)$ và $C(1; 2; 0)$.
	\choiceTF
	{\True Phương trình chính tắc của đường thẳng $AB$ là $\dfrac{x-2}{1}=\dfrac{y}{-1}=\dfrac{z-1}{1}$}
	{Đường trung tuyến $CM$ của tam giác $ABC$ có một vectơ chỉ phương là $\overrightarrow{u}=(0; 1; 1)$}
	{Đặt $\varphi=(AB,(Oxy))$, khi đó $\cos \varphi=\dfrac{\sqrt{3}}{3}$}
	{\True Đường trung trực cạnh $AB$ của $\triangle ABC$ có phương trình là $\dfrac{x}{1}=\dfrac{y+1}{2}=\dfrac{z+1}{1}$}
	\loigiai{
		\begin{itemchoice}
			\itemch  \textbf{Đúng}.\\
			Ta có $\overrightarrow{AB}=(-2;2;-2)=-2(1;-1;1)$.\\
			Phương trình chính tắc của đường thẳng $AB$ là $\dfrac{x-2}{1}=\dfrac{y}{-1}=\dfrac{z-1}{1}$.
			\itemch  \textbf{Sai}.\\
			Vì $CM$ là đường trung tuyến nên $M$ là trung điểm $AB$ suy ra $M(1;1;0)$.\\
			Khi đó đường trung tuyến $CM$ của tam giác $ABC$ có một vectơ chỉ phương là $\overrightarrow{u}=\overrightarrow{CM}=(0; -1; 0)$.
			\itemch  \textbf{Sai}.\\
			Ta có $\overrightarrow{AB}=(-2;2;-2)$ và $\overrightarrow{j}=(0;0;1)$.\\
			Khi đó $\sin \varphi=\left|\cos\left(\overrightarrow{AB},\overrightarrow{j}\right)\right|=\dfrac{|-2|}{\sqrt{4+4+4}\cdot 1}=\dfrac{\sqrt{3}}{3}$.\\
			Vậy $\cos \varphi=\sqrt{1-\sin^2 \varphi}=\dfrac{\sqrt{6}}{3}$.
			\itemch  \textbf{Đúng}.\\
			Gọi $M$ là trung điểm $AB$ suy ra $M(1;1;0)$.\\
			Đường trung trực cạnh $AB$ của $\triangle ABC$ qua $M(1;1;0)$ có $2$ vectơ pháp tuyến $\overrightarrow{AB}=(-2;2;-2)$ và $\overrightarrow{n}=\left[\overrightarrow{AB},\overrightarrow{AC}\right]=(2;0;-2)$ nên có vectơ là $\overrightarrow{u}=\left[\overrightarrow{AB},\overrightarrow{n}\right]=(-4;-8;-4)=-4(1;2;1)$.\\
			Đường trung trực $d$ cạnh $AB$ của $\triangle ABC$ có phương trình $\dfrac{x-1}{1}=\dfrac{y-1}{2}=\dfrac{z}{1}$.\\
			Ta thấy $(0;-1;-1)\in d$ nên phương trình $d$ cũng được viết là $\dfrac{x}{1}=\dfrac{y+1}{2}=\dfrac{z+1}{1}$.
		\end{itemchoice}
	}
\end{ex}
%Câu 2
\begin{ex}%[2H5V3-4]%[Dự án D - đợt 2 NH 24-25 - Xuan Vy Pham]
	Trong không gian $Oxyz$, cho đường thẳng $d \colon \dfrac{x+1}{1}=\dfrac{y+3}{2}=\dfrac{z+2}{2}$ và mặt cầu $(S)$ có tâm $I(3;2;0)$. Gọi $d$ là đường thẳng cắt $(S)$ tại hai điểm $A$ và $B$ sao cho $AB=8$.
	\choiceTF
	{\True Phương trình mặt phẳng $(P)$ đi qua $I$ và vuông góc với $d$ là $x+2y+2z-7=0$}
	{Gọi $H(a;b;c)$ là hình chiếu vuông góc của $I$ lên $d$. Khi đó $a+b+c=2$}
	{\True Mặt cầu $(S)$ có bán kính $R=5$}
	{\True Phương trình mặt cầu $(S)$ là $(x-3)^2+(y-2)^2+z^2=25$}
	\loigiai{
		\begin{center}
			\begin{tikzpicture}[>=stealth, line join=round, line cap=round, font=\footnotesize, scale=1.4]
				\def\r{1.5}
				\path
				(0:0) coordinate (I)
				(-130:\r) coordinate (A)
				(-20:\r) coordinate (B)
				($(A)!(I)!(B)$) coordinate (H)
				;
				\draw (I) circle (\r) 
				(I)--(A)node[midway,left]{$R$}--(B)--(I)--(H)
				;
				\foreach \p/\g in {I/90, A/200, B/-20, H/-90}
				\fill (\p) circle (1pt) node[shift=(\g:3mm)] {$\p$};
				\pic[draw,angle radius=2mm]{right angle=A--H--I};
			\end{tikzpicture} 
		\end{center}
		\begin{itemchoice}
			\itemch  \textbf{Đúng}. 
			Do $(P) \perp d$ nên $\overrightarrow{n}_{(P)}=\overrightarrow{u}_d=(1;2;2)$.\\
			Phương trình mặt phẳng $(P)$ là $(x-3)+2(y-2)+2(z-0)=0 \Leftrightarrow x+2y+2z-7=0$.						
			\itemch  \textbf{Sai}. 
			Do $H$ là hình chiếu vuông góc của	$I$ lên $d$ nên $H(-1+t;-3+2t;-2+2t)$.\\
			Ta có $\overrightarrow{IH}=(t-4;2t-5;2t-2)$.\\
			Mà $IH \perp d$ nên $\overrightarrow{IH} \cdot \overrightarrow{u}_{d}=0$. \\
			Hay $t-4+2(2t-5)+2(2t-2)=0 \Leftrightarrow t=2$ $\Rightarrow H(1;1;2)$.\\
			Vậy $a=1$, $b=1$, $c=2$ nên $a+b+c=4$.		
			\itemch  \textbf{Đúng}. 
			Theo b) ta có $IH=\sqrt{(1-3)^2+(1-2)^2+(2-0)^2}=3$. \\
			Bán kính mặt cầu $(S)$ là $R=\sqrt{IH^2+HB^2}=\sqrt{3^2+4^2}=5$.
			\itemch  \textbf{Đúng}. 
			Phương trình mặt cầu $(S)$ là $(x-3)^2+(y-2)^2+z^2=25$.
		\end{itemchoice}
	}
\end{ex}
\Closesolutionfile{ans}
\begin{center}
	\textbf{PHẦN 3 - Câu trắc nghiệm trả lời ngắn}
\end{center}
\setcounter{ex}{0}
%Câu 1
\begin{ex}%[2H5H2-7]%[Dự án D - đợt 2 NH 24-25 - Xuan Vy Pham]
	Trong không gian $Oxyz$, tính góc giữa hai đường thẳng $d \colon \heva{&x=2+t\\&y=-2+t\\&z=1}$ và $\Delta \colon \heva{&x=2+2t'\\&y=-1+2t'\\&z=1+t'}$ (đơn vị độ, làm tròn đến hàng đơn vị).
	\shortans{$20$}
	\loigiai{Một vectơ chỉ phương của hai đường thẳng $d$ và $d'$ lần lượt là $\overrightarrow{u}_1=(1;1;0)$ và $\overrightarrow{u}_2=(2;2;1)$.\\
		Ta có
		$$ \cos (d,d')=\dfrac{\left| \overrightarrow{u}_1 \cdot \overrightarrow{u}_2\right|}{\left| \overrightarrow{u}_1 \right| \cdot \left| \overrightarrow{u}_2 \right|}=\dfrac{ \left| 1 \cdot 2+1 \cdot 2 +0 \cdot 1 \right|}{\sqrt{1^2+1^2+0^2} \cdot \sqrt{2^2+2^2+1^2}}=\dfrac{2\sqrt{2}}{3}.$$
		Do đó $(d,d') \approx 20^\circ$.\\
		Vậy góc giữa đường thẳng $d$ và $d'$ xấp xỉ khoảng $20^\circ$.
	} 
\end{ex}
%Câu 2
\begin{ex}%[2H5V2-5]%[Dự án D - đợt 2 NH 24-25 - Xuan Vy Pham]
	Trong không gian $Oxyz$, một cabin cáp treo xuất phát từ điểm $A(11;4;0)$ và chuyển động đều theo đường cáp có véc-tơ chỉ phương $\overrightarrow{u}=(-3;-4;0)$ với tốc độ là $5$ m/s (đơn vị trên mỗi trục toạ độ là mét); giả sử sau $t$ (s) kể từ lúc xuất phát $(t\geq 0)$, cabin đến điểm $M$. Một người đứng tại điểm $O$ quan sát cabin chạy trên cáp treo, sau thời gian bao nhiêu thì khoảng cách giữa người quan sát và cabin gần nhau nhất? (\textit{làm tròn đến hai chữ số thập phân sau dấu phẩy}).
\shortans{$1{,}96$}
\loigiai{Gọi $\Delta$ là đường thẳng chuyển động của cabin.\\
	Đường thẳng $\Delta$ đi qua $A(11;4;0)$, có véc-tơ chỉ phương $\overrightarrow{u}=(-3;-4;0)$ có dạng
	\begin{align*}
		\heva{&x=11-3t\\ &y=4-4t\\ &z=0
		}\; (t\in\mathbb{R}).
	\end{align*}
	Khoảng cách giữa người quan sát và cabin gần nhau nhất khi và chỉ khi $M$ là hình chiếu của $O$ lên $\Delta$.\\
	Vì $M\in \Delta$ nên $M(11-3t;4-4t;0)$.\\
	Vì $M$ là hình chiếu của $O$ lên $\Delta$ nên
	\begin{eqnarray*}
		& &OM\perp \Delta\\
		&\Leftrightarrow &\overrightarrow{OM}\cdot \overrightarrow{u}=0\\
		&\Leftrightarrow & (11-3t)\cdot (-3) + (4-4t)\cdot (-4) + 0\cdot 0=0\\
		&\Leftrightarrow &t=\dfrac{49}{25}.
	\end{eqnarray*}
	Với $t=\dfrac{49}{25}$, ta có $M\left(\dfrac{128}{25};-\dfrac{96}{25}; 0 \right)$.\\
	Khoảng cách từ vị trí ban đầu đến $M$ là $AM=\sqrt{\left(\dfrac{128}{25}-11 \right)^2+ \left(-\dfrac{96}{25}-4 \right)^2 }=9{,}8$.\\
	Thời gian để cabin từ vị trí ban đầu di chuyển đến $M$ là $t=\dfrac{9{,}8}{5}=1{,}96$ (giây).
}
\end{ex}
%Câu 3...........................
\begin{ex}%[2H5H3-2]%[Dự án D - đợt 2 NH 24-25 - Xuan Vy Pham]
Ericsson Globe (Thụy Điển) là tòa nhà bán cầu lớn nhất trên thế giới (năm 2020), với hình dạng một quả cầu màu trắng có đường kính là $110$ m và chiều cao bên trong $85$ m, nó có đủ chỗ ngồi cho $16\,000$ khán giả của các buổi biểu diễn hòa nhạc hoặc $13\,850$ khán giả của các trận đấu khúc côn cầu trên băng. Giả sử ta biểu diễn mô phỏng của tòa nhà Ericsson Globe trong hệ trục tọa độ $Oxyz$ bởi một mặt cầu có tâm $I$, đường kính $110$ m và $OA=85$ m như hình vẽ (đơn vị trên trục là mét). Phương trình của mặt cầu này có dạng $x^2+y^2+\left(z-a\right)^2=b^2$. Tính $a+b$.
\begin{center}
	\begin{tikzpicture}[line join=round, line cap=round,>=stealth,font=\footnotesize,scale=1]
	\def\a{6.5}
	\def\b{2.3}
	\def\h{1.7}
	\path (0:0) coordinate (E)
	++(0:\a) coordinate (D)
	++(-135:\b) coordinate (C)
	($(E)+(C)-(D)$) coordinate (B)
	($(E)!1/2!(C)$) coordinate (O)
	($(O)+(90:\h*0.7)$) coordinate (I)
	($(I)+(90:\h)$) coordinate (A)
	($(I)+(-45:\h)$) coordinate (F)
	($(I)+(-135:\h)$) coordinate (G)
	($(O)+(-135:0.5)$) coordinate (H)
	($(E)!0.1!(D)$) coordinate (D1)
	($(E)!0.65!(D)$) coordinate (D2);
	\draw[dashed,thick] (O)--(A) (D1)--(D2);
	\draw[thick] (E)--(B)--(C)--(D) (E)--(D1) (D2)--(D);
	\draw[thick] (A) arc (90:-45:\h);
	\draw[thick] (A) arc (90:225:\h);
	\draw[dashed,thick] (F) arc
	(0:180:{0.7141*\h} and {0.7141*\h*0.25});
	\draw[thick] (G) arc
	(180:360:{0.7141*\h} and {0.7141*\h*0.25}); 
	\draw[thick,dashed] (O)--(H) (O)--(F);
	\foreach \x/\g in {O/180,I/30}
	\fill[black] (\x) circle (1pt) ($(\g:3mm)+(\x)$) node {$\x$};
	\draw (A) node[above right]{$A$};
	%Ve truc Ox,Oy, Oz
	\draw[thick,->](H)--($(O)!3.5!(H)$) node [pos=0.9,below left]{$x$};
	\draw[thick,->](F)--($(O)!1.9!(F)$) node [pos=0.9, above]{$y$};
	\draw[thick,->](A)--($(O)!1.3!(A)$) node [pos=0.9,above right]{$z$};	
	\draw[thick] ($(B)!0.1!(C)$) arc
	(0:100:0.3) node [pos=0.2, above right] {Mặt đất};
\end{tikzpicture}
\end{center}
\shortans{$85$}
\loigiai{Vì $I \in Oz$ nên $I(0;0;z_I)$ với $z_I=OI$.\\
Vì mặt cầu có đường kính $110$ m nên $IA=\dfrac{110}{2}=55$ (m).\\
Do đó $OI=OA-IA=85-55=30$ m nên $I(0;0;30)$.\\
Phương trình mặt cầu có tâm $I(0;0;30)$ và bán kính $IA=55$ có dạng 
$$x^2+y^2+(z-30)^2=55^2.$$
Vậy $a=30$, $b=55$ nên $a+b=85$.}
\end{ex}
%Câu 4
\begin{ex}%[2H5V3-4]%[Dự án D - đợt 2 NH 24-25 - Xuan Vy Pham]
	\immini{
		Trong không gian $O x y z$ (đơn vị trên mỗi trục là kilômét), đài kiểm soát không lưu sân bay Cam Ranh - Khánh Hòa ở vị trí $O(0 ; 0 ; 0)$ và được thiết kế phát hiện máy bay ở khoảng cách tối đa 600 km. Một máy bay của hãng Việt Nam Airlines đang ở vị trí $A(-800 ;-40 ; 10)$, chuyển động theo đường thẳng $d \colon  \heva{&x=-1000+100t \\&y=-200+80t \\& z=10}$ và hướng về đài kiểm soát không lưu (tham khảo hình vẽ bên cạnh).
	}
	{
		\begin{tikzpicture}[scale=0.7,>=stealth, font=\footnotesize, line join=round, line cap=round]
			\path
			(0,0) coordinate (O)
			(-2,1.2) coordinate (B)
			(2.16,0.61) coordinate (C) 
			(2.6069,0) coordinate (K) 
			($(B)!{-0.4}!(C)$) coordinate  (A) 
			($(B)!1.25!(C)$) coordinate  (D) 
			
			;
			\draw (A)--(B) (C)--(D) ;
			\draw[dashed](B)--(C) ;
			\draw[->] (-3.2,0)--(3.4,0) node[below]{$x$}; 
			\draw[->] (1.5,1.76)--(-2.2,-2.6) node[below]{$y$}; 
			\draw[->] (0,-2)--(0,3.5) node[left]{$z$}; 
			\draw[dashed] (K) arc (0:180:2.6069cm and 1.6721cm);
			\draw (K) arc (0:-180:2.6069cm and 1.6721cm);
			\draw (O) circle(2.6069) ;
			%\draw (C)--(A)--(B)--(B') (A)--(H)  (G)--(C);
			%\draw pic[draw,angle radius=8mm,angle eccentricity=0.6] {angle = B--A--C}; 
			\draw(D) node[above]{$d$} ;
			\foreach \x/\g in {A/90,B/60,C/-120,O/-40}
			\fill[black] (\x) circle(1.1pt) + (\g:3mm) node {$\x$};
		\end{tikzpicture}
	}
	\shortans{$749$}
	\loigiai{
		Ranh giới vùng phủ sóng của đài kiểm soát không lưu là mặt cầu $(S)$ tâm $O(0;0;0)$, bán kính $R=600$ km. Mặt cầu $(S)$ có phương trình là
		$$x^2+y^2+z^2=600^2.$$
		Quãng đường mà máy bay nhận được tín hiệu của đài kiểm soát không lưu là khoảng cách giữa hai giao điểm $A$, $B$ của mặt cầu $(S)$ và đường thẳng $d$.
		\\
		Thay $\heva{&x=-1000+100t \\&y=-200+80t \\& z=10}$ vào $(S)\colon x^2+y^2+z^2=600^2$.
		\\
		Ta có 
		\allowdisplaybreaks{\begin{eqnarray*}
				&& (100t-1000)^2+(80t-200)^2+10^2=600^2\\
				&\Leftrightarrow& 16 \,400t^2-232 \,000t+680\,100=0\\
				&\Leftrightarrow& \hoac{&t\approx 10\\& t\approx 4{,}15.}
		\end{eqnarray*} }
		Từ đó ta có hai giao điểm là $A\left(0;600;10\right)$ và $B\left(-585;132;10\right)$.
		$$AB=\sqrt{(-585-0)^2+(132-600)^2+(10-10)^2}\approx 749 ~ \text{(km)}$$
		Vậy quãng đường mà máy bay nhận được tín hiệu của đài kiểm soát không lưu khoảng $749$ (km).
		
	}
\end{ex}
\Closesolutionfile{ansKQ}
\begin{center}
	\textbf{PHẦN 4 - Tự luận.}
\end{center}
\setcounter{ex}{0}
%Câu 1
\begin{ex}%[2H5H1-3]%[Dự án D - đợt 2 NH 24-25 - Xuan Vy Pham]
	Trong không gian $Oxyz$, mặt phẳng $(P)\colon x+by+cz+d=0$ cắt ba trục $Ox$, $Oy$, $Oz$ tại $A$, $B$, $C$ (khác $O$) sao cho trực tâm tam giác $ABC$ là $H(1;2;3)$. Viết phương trình mặt phẳng $(P)$.
	\loigiai{Ta có $\heva{&BC \perp AH \text{ ($H$ là trực tâm $\triangle ABC$)} \\
			&BC \perp OA} \Rightarrow BC \perp (OAH) \Rightarrow BC \perp OH$.\\
		Ta có $\heva{&AC \perp BH \text{ ($H$ là trực tâm $\triangle ABC$)}\\
			&AC \perp OB} \Rightarrow AC \perp (OBH) \Rightarrow AC \perp OH$.\\
		Suy ra $\heva{&OH \perp BC\\
			&OH \perp AC} \Rightarrow OH \perp (ABC)$.\\
		Phương trình mặt phẳng $(P)$ đi qua $H(1;2;3)$, có vectơ pháp tuyến $\overrightarrow{n}=\overrightarrow{OH}=(1;2;3)$ có dạng
		\begin{eqnarray*}
			& & 1(x-1)+2(y-2)+3(z-3)=0\\
			&\Leftrightarrow & x+2y+3z-14=0.
		\end{eqnarray*}
	}
\end{ex}
%Câu 2
\begin{ex}%[2H5V2-8]%[Dự án D - đợt 2 NH 24-25 - Xuan Vy Pham]
	\immini{Trên mặt đất phẳng, người ta dựng một cây cột thẳng cao $6$m vuông góc với mặt đất, có chân cột đặt tại vị trí $O$ trên mặt đất. Tại một thời điểm, dưới ánh nắng mặt trời cách chân cột $3$m về hướng S$60^\circ$E (hướng tạo với hướng nam góc $60^\circ$ và tạo với hướng đông góc $30^\circ$) (tham khảo hình vẽ bên cạnh). Chọn hệ trục tọa độ $Oxyz$ có gốc tọa độ là $O$, tia $Ox$ chỉ hướng nam, tia $Oy$ chỉ hướng đông, tia $Oz$ chứa cây cột, đơn vị đo là mét. Biết phương trình đường thẳng chứa tia nắng mặt trời đi qua đỉnh cột tại thời điểm đang xét có dạng $\dfrac{x}{1}=\dfrac{y}{a}=\dfrac{z-6}{b}$. Tính giá trị của $a-b$ (\textit{làm tròn đến hai chữ số thập phân sau dấu phẩy}).}{\begin{tikzpicture}[scale=1,>=stealth, font=\footnotesize, line join=round, line cap=round,declare function={a=2;h=1;db=15;tn=-135;}]
		\def\a{3}
		\def\b{2.5}
		\def\h{1.8}
		\path (0:0) coordinate (O)
		++(0:\a) coordinate (E)
		(O)++(-120:\b) coordinate (S)
		(O)++(90:\h) coordinate (z)
		($(O)!0.5!(z)$) coordinate (A)
		(O)++(-30:\a*0.7) coordinate (A')
		(z)++(120:1.4*\h) coordinate (M)
		($(M)!0.23!(z)$) coordinate (M1)
		($(M)!0.8!(z)$) coordinate (M2)
		(M1)++(--40:0.3) coordinate (M11)
		(M11)++(-60:1.5) coordinate (M21)
		(M1)++(-160:0.3) coordinate (M12)
		(M12)++(-60:1.4) coordinate (M22);
		\draw[thick,->] (O)--(S) node[right]{$x$};
		\draw[thick,->] (O)--(z) node[right]{$z$};
		\draw[thick,->] (O)--(E) node[right]{$y$}
		;
		\draw[thick] (O)--(A');
		\draw[thick] ($(O)!0.2!(E)$) arc (0:-30:0.6) node[pos=0.7,right]{$30^\circ$};
		\draw[thick] ($(O)!0.2!(A')$) arc (-30:-100:0.6) node[pos=0.5,below]{$60^\circ$};
		\draw[thick,->] (M1)--(M2) ;
		\draw[thick,->] (M11)--(M21) ;
		\draw[thick,->] (M12)--(M22) ;
		\draw[thick] (M) circle (0.6);
		\draw (M) node[align=left] {Mặt \\trời};
		\draw (E) node [above left]{$E$};
		\draw (S) node [above left]{$S$};
		\draw (A') node [below right]{$A'$};
		\foreach \x/\g in {A/0,O/140}
		\fill[black] (\x) circle (1pt) ($(\g:4mm)+(\x)$) node {$\x$};	
		\newcommand{\gv}[4][black]{\draw[thick] ($(#3)!8pt!(#2)$)--($(#3)!2!($($(#3)!8pt!(#2)$)!.5!($(#3)!8pt!(#4)$)$)$)--($(#3)!8pt!(#4)$);}
		\gv{E}{O}{z}
\end{tikzpicture}}
\shortans{$5{,}73$}
\loigiai{Vì cây cột thẳng cao $6$m nên điểm $A \in Oz$ có tọa độ là $A(0;0;6)$.\\
	Vì vị trí $A'$ cách chân cột $3$m nên $OA'=3$.\\
	Gọi $A_1$, $A_2$ là hình chiếu vuông góc của $A'$ lên $Ox$, $Oy$.\\
	Xét tam giác $OA'A_1$ vuông tại $A_1$, ta có
	$$\heva{&\cos \widehat{A'OA_1}=\dfrac{OA_1}{OA'} \Rightarrow OA_1=\cos 60^\circ \cdot 3=\dfrac{3}{2}\\&\sin \widehat{A'OA_1}=\dfrac{A'A_1}{OA'} \Rightarrow A'A_1=\sin 60^\circ \cdot 3=\dfrac{3 \sqrt{3}}{2}.}$$
	Vì $A' \in (Oxy)$ nên $z_{A'}=0$.\\
	Ta có $x_{A'}=OA_1=\dfrac{3}{2}$ và $y_{A'}=OA_2=A'A_1=\dfrac{3\sqrt{3}}{2}$.\\
	Do đó $A' \left(\dfrac{3}{2};\dfrac{3\sqrt{3}}{2};0\right)$.\\
	Ta có $\overrightarrow{AA'}=\left(\dfrac{3}{2};\dfrac{3\sqrt{3}}{2};-6\right)=\dfrac{3}{2} \left(1;\sqrt{3};-4\right)$.\\
	Phương trình đường thẳng chứa tia nắng mặt trời đi qua đỉnh cột tại thời điểm đang xét là $AA'$ qua điểm $A(0;0;6)$ và có vectơ chỉ phương là $\overrightarrow{u}=\left(1;\sqrt{3};-4\right)$ có dạng
	$$\dfrac{x}{1}=\dfrac{y}{\sqrt{3}}=\dfrac{z-6}{-4}.$$
	Do đó $a=\sqrt{3}$, $b=-4$ nên $a-b=\sqrt{3}+4 \approx 5{,}73$.}
\end{ex}
%Câu 3
\begin{ex}%[2H5V3-2]%[Dự án D - đợt 2 NH 24-25 - Xuan Vy Pham]
	Trong không gian $Oxyz$, cho mặt cầu $(S) \colon  x^2+y^2+z^2-2 x+4 z-4=0$ và đường thẳng $d \colon  \dfrac{x-1}{1}=\dfrac{y}{-2}=\dfrac{z+1}{-5}$. Mặt phẳng $(P)$ vuông góc với $d$ và cắt mặt cầu $(S)$ theo giao tuyến là một đường tròn có bán kính bằng $3$. Tính khoảng cách từ gốc tọa độ $O$ đến mặt phẳng $(P)$ (làm tròn đến hai chữ số thập phân sau dấu phẩy).
	\loigiai{
		Mặt cầu $(S) \colon  x^2+y^2+z^2-2 x+4 z-4=0$ có tâm $I(1;0;-2)$, bán kính $R=3$.\\
		Mặt phẳng $(P)$ vuông góc với $d$ nên có một vectơ pháp tuyến là $\overrightarrow{n} = (1;-2;-5)$. \\
		Mặt phẳng $(P)$ vuông góc với $d$ và cắt mặt cầu $(S)$ theo giao tuyến là một đường tròn có bán kính bằng $3$ nên đi qua tâm $I$. \\
		Phương trình của $(P)$ là $x-1-2(y-0)-5(z+2) =0 \Leftrightarrow x-2y-5z -11=0$. \\
		Khoảng cách từ $O$ đến mặt phẳng $(P)$ là 
		\[
		\mathrm{d}(O,(P)) = \dfrac{|-11|}{\sqrt{1^2+2^2+5^2}} \approx 2{,}01.
		\]
	}
\end{ex}
