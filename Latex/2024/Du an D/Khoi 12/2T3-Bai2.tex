\newpage
\section{Phương sai và độ lêch chuẩn của mẫu số liệu ghép nhóm}
\subsection{LÝ THUYẾT CẦN NHỚ}
\subsubsection{Phương sai của mẫu số liệu ghép nhóm, kí hiệu $ s^2 $, được tính bởi công thức}
Cho mẫu số liệu ghép nhóm
\begin{center}
	\begin{tabular}{|c|c|c|c|c|c|}
	\hline
	Nhóm & $\left[a_1;a_2 \right)$ & $\ldots$ & $\left[a_{;};a_{i+1} \right)$ & $\ldots$ & $\left[a_k;a_{k+1} \right)$ \\
	\hline
	Tần số & $m_1$ & $\ldots$ & $m_i$ & $\ldots$ & $m_k$ \\
	\hline
\end{tabular}
\end{center}
\begin{itemize}
	\item Phương sai của mẫu số liệu ghép nhóm, kí hiệu là $s^2$, là một số được tính theo công thức sau
	\begin{eqnarray*}
		s^2&=&\dfrac{m_1 \left(x_1-\overline{x}\right)^2+\cdots+m_k \left(x_k-\overline{x}\right)^2}{n}.
	\end{eqnarray*}
Trong đó
\begin{itemize}
	\item $n=m_1+\cdots+m_k$;
	\item $x_i=\dfrac{a_i+a_{i+1}}{2}$ với $i=1,2,\ldots, k$ là giá trị đại diện cho nhóm $\left[a_i;a_{i+1} \right)$;
	\item $\overline{x}=\dfrac{m_1 x_1+\cdots+m_k x_k}{n}$ là số trung bình của mẫu số liệu ghép nhóm.
\end{itemize}
	\item Độ lệch chuẩn của mẫu số liệu ghép nhóm, kí hiệu là $s$, là căn bậc hai số học của phương sai của mẫu số liệu ghép nhóm, tức là $s=\sqrt{s^2}$.
\end{itemize}
\begin{nx}\hfill
	\begin{itemize}
		\item Ta có thể tính phương sai theo công thức \begin{eqnarray*}
			s^2&=&\dfrac{1}{n} \left(m_1 \cdot x_1^2+\cdots+m_k \cdot x_k^2\right)-(\overline{x})^2.
		\end{eqnarray*}
		\item Độ lệch chuẩn có cùng đơn vị với đơn vị của mẫu số liệu.
	\end{itemize}
\end{nx}
\begin{luuy}
	Người ta còn sử dụng các đại lượng sau để đo mức độ phân tán của mẫu số liệu ghép nhóm
	\begin{eqnarray*}
		\widehat{s}^2&=&\dfrac{1}{n-1}\left[n_1(c_1-\overline{x})^2+n_2(c_2-\overline{x})^2+\cdots+n_k(c_k-\overline{x})^2 \right].
	\end{eqnarray*}
	Trong đó \begin{itemize}
		\item $ n=n_1+n_2+\cdots+n_k $ là cỡ mẫu.
		\item $ \overline{x}=\dfrac{1}{n}(n_1c_1+n_2c_2+\ldots+n_kc_k) $ là số trung bình.
	\end{itemize}
\end{luuy}
\subsubsection{Ý nghĩa của phương sai và độ lệch chuẩn của mẫu số liệu ghép nhóm}
\begin{itemize}
	\item Phương sai của mẫu số liệu ghép nhóm là giá trị xấp xỉ cho phương sai của mẫu số liệu gốc.
	\item Độ lệch chuẩn của mẫu số liệu ghép nhóm cũng là giá trị xấp xỉ cho độ lệch chuẩn của mẫu
	số liệu gốc. Chúng được dùng để đo mức độ phân tán của mẫu số liệu ghép nhóm. 
	\item Phương sai và độ lệch chuẩn càng lớn thì dữ liệu càng phân tán.
	\item Độ lệch chuẩn có cùng đơn vị với đơn vị của mẫu số liệu.
\end{itemize}


%-------------------------------------------------------------------------------------------------------------
\subsection{PHÂN LOẠI VÀ PHƯƠNG PHÁP GIẢI TOÁN}
\begin{dang}{Tính phương sai, độ lệch chuẩn}
	$\star$ {\bf Bài toán}. Cho mẫu số liệu ghép nhóm
	\begin{center}
		\begin{tabular}{|c|c|c|c|c|c|}
			\hline
			Nhóm & $\left[a_1;a_2 \right)$ & $\ldots$ & $\left[a_{;};a_{i+1} \right)$ & $\ldots$ & $\left[a_k;a_{k+1} \right)$ \\
			\hline
			Tần số & $m_1$ & $\ldots$ & $m_i$ & $\ldots$ & $m_k$ \\
			\hline
		\end{tabular}
	\end{center}
	Tính phương sai, độ lệch chuẩn của mẫu số liệu ghép nhóm trên.
	\loigiai{
		Để tính phương sai, độ lệch chuẩn của mẫu số liệu ghép nhóm ta thực hiện các bước như sau
		\begin{itemize}
			\item Bổ sung vào Bảng số liệu hàng giá trị đại diện $c_i$ như sau
			\begin{center}
				\begin{tabular}{|c|c|c|c|c|c|}
					\hline
					Nhóm & $\left[a_1;a_2 \right)$ & $\ldots$ & $\left[a_{;};a_{i+1} \right)$ & $\ldots$ & $\left[a_k;a_{k+1} \right)$ \\
					\hline
					Nhóm & $c_1$ & $\ldots$ & $c_i$ & $\ldots$ & $c_k$ \\
					\hline
					Tần số & $m_1$ & $\ldots$ & $m_i$ & $\ldots$ & $m_k$ \\
					\hline
				\end{tabular}
			\end{center}
			Trong đó
			\begin{itemize}
				\item $n=m_1+\cdots+m_k$;
				\item $c_i=\dfrac{a_i+a_{i+1}}{2}$ với $i=1,2,\ldots, k$ là giá trị đại diện cho nhóm $\left[a_i;a_{i+1} \right)$.
			\end{itemize}
			\item Tính số trung bình của Bảng số liệu bằng công thức \[\overline{x}=\dfrac{1}{n}\left(m_1c_1+\cdots+ m_kc_k\right).\]
			\item Tính phương sai của Bảng số liệu bằng công thức  \[s^2=\dfrac{1}{n}\left[m_1 \left(c_1-\overline{x}\right)^2+\cdots+m_k \left(c_k-\overline{x}\right)^2\right].\]
			\item Độ lệch chuẩn của mẫu số liệu ghép nhóm là $s=\sqrt{s^2}$.
		\end{itemize}
	}
	
\end{dang}
%%%=============VD_1=============%%%
\begin{vd}
	Người ta theo dõi sự thay đổi cân nặng, được tính bằng hiệu cân nặng trước và sau ba tháng áp dụng chế độ ăn kiêng của một số người cho kết quả như sau
	\begin{center}
		\begin{tabular}{|c|c|c|c|c|c|} 
			\hline
			Thay đổi cân nặng (kg) & $[-1;0)$ & $[0;1)$ & $[1;2)$ & $[2;3)$ & $[3;4)$ \\ 
			\hline
			Số người nam & 2 & 3 & 5 & 3 & 2 \\ 
			\hline
			Số người nữ & 2 & 7 & 12 & 7 & 2 \\ 
			\hline
		\end{tabular}
	\end{center}
	Tính số trung bình, phương sai, độ lệch chuẩn và nhận xét về sự thay đổi cân nặng của người nam, người nữ sau ba tháng áp dụng chế độ ăn kiêng.
	\loigiai{
		Chọn giá trị đại diện cho các nhóm số liệu, ta có
		\begin{center}
			\begin{tabular}{|c|c|c|c|c|c|} 
				\hline
				Giá trị đại diện & $-0{,}5$ & $0{,}5$ & $1{,}5$ & $2{,}5$ & $3{,}5$ \\ 
				\hline
				Số người nam & 2 & 3 & 5 & 3 & 2 \\ 
				\hline
				Số người nữ & 2 & 7 & 12 & 7 & 2 \\ 
				\hline
			\end{tabular}
		\end{center}
		\begin{itemize}
			\item Tổng số người nam là $n_1=2+3+5+3+2=15$.
			\item Tổng số người nư là $n_2=2+7+12+7+2=30$.
		\end{itemize}
		Thay đổi cân nặng trung bình của người nam là
		\begin{eqnarray*}
			\overline{x}_1&=&\dfrac{1}{15} [2\cdot (-0{,}5)+3\cdot 0{,}5+5\cdot 1{,}5+3\cdot 2{,}5+2\cdot 3{,}5]=1{,}5\;\mathrm{( kg)}.
		\end{eqnarray*}
		Thay đổi cân nặng trung bình của người nữ là
		\begin{eqnarray*}
			\overline{x}_2&=&\dfrac{1}{30} [2\cdot (-0{,}5)+7\cdot 0{,}5+12\cdot 1{,}5+7\cdot 2{,}5+2\cdot 3{,}5]=1{,}5\;\mathrm{( kg)}.
		\end{eqnarray*}
		Phương sai và độ lệch chuẩn của mẫu số liệu về thay đổi cân nặng của người nam là
		\begin{eqnarray*}
			s_1^2&=&\dfrac{1}{15} \left[2\cdot (-0{,}5)^2+3\cdot 0{,}5^2+5\cdot 1{,}5^2+3\cdot 2{,}5^2+2\cdot 3{,}5^2\right]-1{,}5^2;\\
			\Rightarrow s_1&\approx& 1{,}21.
		\end{eqnarray*}
		Phương sai và độ lệch chuẩn của mẫu số liệu về thay đổi cân nặng của người nữ là
		\begin{eqnarray*}
			s_2^2&=&\dfrac{1}{30} \left[2\cdot (-0{,}5)^2+7\cdot 0{,}5^2+12\cdot 1{,}5^2+7\cdot 2{,}5^2+2\cdot 3{,}5^2\right]-1{,}5^2;\\
			\Rightarrow s_2 &\approx& 2{,}06.
		\end{eqnarray*}
		Như vậy, sau ba tháng áp dụng chế độ ăn kiêng này, về trung bình sự thay đổi cân nặng của nam và nữ là như nhau. Tuy nhiên, sự biến động về thay đổi cân nặng của nữ nhiều hơn so với của nam.
	}
\end{vd}
\begin{dang}{Sử dụng phương sai, độ lệch chuẩn đo độ rủi ro}
	Trong tài chính, người ta có nhiều cách để đo độ rủi ro của một phương án đầu tư. Một trong các cách đó là sử dụng độ lệch chuẩn của lợi nhuận thu được theo phương án đầu tư. Độ lệch chuẩn càng lớn thì phương án đầu tư càng rủi ro.
\end{dang}

%%%=============VD_2=============%%%
\begin{vd}
	Anh An đầu tư số tiền bằng nhau vào hai lĩnh vực kinh doanh A, B. Anh An thống kê số tiền thu được mỗi tháng trong vòng $60$ tháng theo mỗi lĩnh vực cho kết quả như sau
	\begin{center}
		\begin{tabular}{|c|c|c|c|c|c|} 
			\hline
			Số tiền (triệu đồng) & $[5;10)$ & $[10;15)$ & $[15;20)$ & $[20;25)$ & $[25;30)$ \\ 
			\hline
			Số tháng đầu tư vào lĩnh vực A & 5 & 10 & 30 & 10 & 5 \\ 
			\hline
			Số tháng đầu tư vào lĩnh vực B & 20 & 5 & 10 & 5 & 20 \\ 
			\hline
		\end{tabular}
	\end{center}
	So sánh giá trị trung bình và độ lệch chuẩn của số tiền thu được mỗi tháng khi đầu tư vào mỗi lĩnh vực A, B. Đầu tư vào lĩnh vực nào \lq\lq rủi ro\rq\rq\, hơn?
	\loigiai{
		Chọn giá trị đại diện cho các nhóm số liệu ta có
		\begin{center}
			\begin{tabular}{|c|c|c|c|c|c|} 
				\hline
				Giá trị đại diện & $7{,}5$ & $12{,}5$ & $17{,}5$ & $22{,}5$ & $27{,}5$ \\ 
				\hline
				Số tháng đầu tư vào lĩnh vực A & 5 & 10 & 30 & 10 & 5 \\ 
				\hline
				Số tháng đầu tư vào lĩnh vực B & 20 & 5 & 10 & 5 & 20 \\ 
				\hline
			\end{tabular}
		\end{center}
		Số tiền trung bình thu được khi đầu tư vào các lĩnh vực A, B tương ứng là
		\begin{itemize}
			\item $\overline{x}_A=\dfrac{1}{60} (5\cdot 7{,}5+\cdots+5\cdot 27{,}5)=17{,}5$.
			\item $\overline{x}_B=\dfrac{1}{60} (20\cdot 7{,}5+\cdots+20\cdot 27{,}5)=17{,}5$.
		\end{itemize}
		Như vậy, về trung bình đầu tư vào các lĩnh vực A, B số tiền thu được hàng tháng như nhau. \\
		Độ lệch chuẩn của số tiền thu được hàng tháng khi đầu tư vào các lĩnh vực A, B tương ứng là
		\begin{itemize}
			\item $s_A=\sqrt{\dfrac{1}{60} \left(5\cdot 7{,}5^2+\cdots+5\cdot 27{,}5^2\right)-(17{,}5)^2}=55$.
			\item $s_B=\sqrt{\dfrac{1}{60} \left(20\cdot 7{,}5^2+\cdots+20\cdot 27{,}5^2\right)-(17{,}5)^2} \approx 8{,}42$.
		\end{itemize}
		Như vậy, độ lệch chuẩn của mẫu số liệu về số tiền thu được hàng tháng khi đầu tư vào lĩnh vực B cao hơn khi đầu tư vào lĩnh vực A. Người ta nói rằng, đầu tư vào lĩnh vực B là \lq\lq rủi ro\rq\rq\, hơn.\\
		Ví dụ sau cho thấy không phải lúc nào ta cũng có thể dùng độ lệch chuẩn của lợi nhuận thu được để so sánh độ rủi ro của các phương án đầu tư.
	}
\end{vd}

%%%=============VD_3=============%%%
\begin{vd}
	Thống kê lợi nhuận hàng tháng (đơn vị: triệu đồng) trong 20 tháng của hai nhà đầu tư được cho như sau
	\begin{center}
		\begin{tabular}{|c|c|c|c|c|c|}
			\hline
			Lợi nhuận & $[10;20)$ & $[20;30)$ & $[30;40)$ & $[40;50)$ & $[50;60)$ \\ 
			\hline
			Số tháng & 2 & 4 & 8 & 4 & 2 \\ 
			\hline
		\end{tabular}\\
		Bảng 3.2. Lợi nhuận theo tháng của nhà đầu tư nhỏ.
	\end{center}
	\begin{center}
		\begin{tabular}{|c|c|c|c|c|c|} 
			\hline
			Lợi nhuận & $[510;520)$ & $[520;530)$ & $[530;540)$ & $[540;550)$ & $[550;560)$ \\ 
			\hline
			Số tháng & 4 & 3 & 6 & 3 & 4 \\ 
			\hline
		\end{tabular}\\
		Bảng 3.3. Lợi nhuận theo tháng của nhà đầu tư lớn.
	\end{center}
	Tính độ lệch chuẩn của hai mẫu số liệu ghép nhóm trên. Có nên dựa vào độ lệch chuẩn để so sánh độ rủi ro của hai nhà đầu tư này không?
	\loigiai{
		Chọn điểm đại diện cho các nhóm số liệu ta tính được các số đặc trưng như sau
		\begin{center}
			\begin{tabular}{|c|c|c|c|c|c|}
				\hline
				Lợi nhuận & $[10;20)$ & $[20;30)$ & $[30;40)$ & $[40;50)$ & $[50;60)$ \\ 
				\hline
				Giá trị đại diện & $15$ & $25$ & $35$ & $45$ & $55$ \\ 
				\hline
				Số tháng & 2 & 4 & 8 & 4 & 2 \\ 
				\hline
			\end{tabular}\\
			Bảng 3.2. Lợi nhuận theo tháng của nhà đầu tư nhỏ.
		\end{center}
		\begin{center}
			\begin{tabular}{|c|c|c|c|c|c|} 
				\hline
				Lợi nhuận & $[510;520)$ & $[520;530)$ & $[530;540)$ & $[540;550)$ & $[550;560)$ \\ 
				\hline
				Giá trị đại diện & $515$ & $525$ & $535$ & $545$ & $555$ \\ 
				\hline
				Số tháng & 4 & 3 & 6 & 3 & 4 \\ 
				\hline
			\end{tabular}\\
			Bảng 3.3. Lợi nhuận theo tháng của nhà đầu tư lớn.
		\end{center}
		Lợi nhuận trung bình một tháng của các nhà đầu tư tương ứng là
		\begin{itemize}
			\item $\overline{x}_A=\dfrac{1}{20} (2\cdot 15+\cdots+2\cdot 55)=35$ (triệu đồng).
			\item $\overline{x}_B=\dfrac{1}{20} (4\cdot 515+\cdots+4\cdot 555)=535$ (triệu đồng).
		\end{itemize}
		Độ lệch chuẩn của lợi nhuận hàng tháng của hai nhà đầu tư tương ứng là
		\begin{itemize}
			\item $s_A=\sqrt{\dfrac{1}{20} \left(2\cdot 15^2+\cdots+2\cdot 55^2\right)-(35)^2} \approx 10{,}95$.
			\item $s_B=\sqrt{\dfrac{1}{20} \left(4\cdot 515^2+\cdots+4\cdot 555^2\right)-(535)^2} \approx 13{,}78$.
		\end{itemize}
		Độ lệch chuẩn cho lợi nhuận hàng tháng của nhà đầu tư lớn cao hơn của nhà đầu tư nhỏ. Lợi nhuận trung bình của hai nhà đầu tư khác nhau rất nhiều, do đó ta không nên dùng độ lệch chuẩn để so sánh mức độ rủi ro của hai nhà đầu tư này.
		\begin{nx}
			Ta không nên dùng phương sai hay độ lệch chuẩn để so sánh độ rủi ro của hai phương án đầu tư khi lợi nhuận trung bình của hai phương án đầu tư này khác nhau rất nhiều.
		\end{nx}
	}
\end{vd}
%-----------------------------------------------------------------------------
\subsection{Bài tập rèn luyện}
\ind{PHẦN I.} \inden{Câu trắc nghiệm nhiều phương án lựa chọn. Mỗi câu hỏi học sinh chỉ chọn một phương án.}\\
\setcounter{ex}{0}
\Opensolutionfile{ans}[ans/2D1-Bai1-TN]%--Đặt tên 2D1-Bai1-Dang1-TN
%%%=============EX_1=============%%%
\begin{ex}
	Cho mẫu số liệu ghép nhóm như bảng dưới đây
	\begin{center}
		\begin{tabular}{|c|c|c|c|c|c|}
			\hline
			Nhóm & $\left[a_1;a_2 \right)$ & $\ldots$ & $\left[a_{;};a_{i+1} \right)$ & $\ldots$ & $\left[a_k;a_{k+1} \right)$ \\
			\hline
			Tần số & $m_1$ & $\ldots$ & $m_i$ & $\ldots$ & $m_k$ \\
			\hline
		\end{tabular}
	\end{center}
	Khi đó phương sai của mẫu số liệu được xác định theo công thức
	\choice
	{\True $s^2=\dfrac{1}{n^2} \left[n_1 \left(c_1-\overline{x}\right)^2+n_2 \left(c_2-\overline{x}\right)^2+\cdots+n_k \left(c_k-\overline{x}\right)^2\right]$}
	{$s^2=\dfrac{1}{n} \left[n_1 \left(c_1-\overline{x}\right)^2+n_2 \left(c_2-\overline{x}\right)^2+\cdots+n_k \left(c_k-\overline{x}\right)^2\right]$}
	{$s^2=\dfrac{1}{n} \left[n_1 \left(c_1-\overline{x}\right)+n_2 \left(c_2-\overline{x}\right)+\cdots+n_k \left(c_k-\overline{x}\right)\right]$}
	{$s^2=\dfrac{1}{n^2} \left[n_1 \left(c_1-\overline{x}\right)+n_2 \left(c_2-\overline{x}\right)+\cdots+n_k \left(c_k-\overline{x}\right)\right]$}
	\loigiai{
	}
\end{ex}

%%%=============EX_2=============%%%
\begin{ex}
	Trong mẫu số liệu ghép nhóm, phương sai bằng
	\choice
	{một nửa của độ lệch chuẩn}
	{hai lần của độ lệch chuẩn}
	{căn bậc hai của độ lệch chuẩn}
	{\True bình phương của độ lệch chuẩn}
	\loigiai{
		Theo lý thuyết, phương sai bằng bình phương của độ lệch chuẩn.
	}
\end{ex}

%%%=============EX_3=============%%%
\begin{ex}
	Số đặc trưng nào sau đây không sử dụng để đo mức độ phân tán của mẫu số liệu ghép nhóm?
	\choice
	{Phương sai}
	{\True Khoảng biến thiên}
	{Khoảng tứ phân vị}
	{Trung vị}
	\loigiai{
		\begin{itemize}
			\item Khoảng biến thiên của mẫu số liệu ghép nhóm xấp xỉ cho khoảng biến thiên của mẫu số liệu gốc. Khoảng biến thiên được dùng để đo mức độ phân tán của mẫu số liệu ghép nhóm. Khoảng biến thiên càng lớn thì mẫu số liệu càng phân tán.
			\item Trung vị của mẫu số liệu ghép nhóm xấp xỉ cho trung vị của mẫu số liệu gốc, nó chia mẫu số liệu thành hai phần, mỗi phần chứa $50\%$ giá trị.
			\item Phương sai của mẫu số liệu ghép nhóm xấp xỉ cho phương sai của mẫu số liệu gốc. Phương sai được dùng để đo mức độ phân tán của mẫu số liệu ghép nhóm xung quanh số trung bình của mẫu số liệu đó. Phương sai càng lớn thì mẫu số liệu càng phân tán.
			\item Khoảng tứ phân vị của mẫu số liệu ghép nhóm xấp xỉ cho khoảng tứ phân vị của mẫu số liệu gốc. Khoảng tứ phân vị được dùng để đo mức độ phân tán của mẫu số liệu ghép nhóm. Khoảng tứ phân vị càng lớn thì mẫu số liệu càng phân tán.
		\end{itemize}
	}
\end{ex}

%%%=============EX_4=============%%%
\begin{ex}
	Một mẫu số liệu ghép nhóm có phương sai bằng $25$ thì có độ lệch chuẩn bằng
	\choice
	{$5$}
	{$256$}
	{$4$}
	{$50$}
	\loigiai{
		Ta có độ lệch chuẩn bằng căn bậc hai số học của phương sai nên $s=5$.
	}
\end{ex}

%%%=============EX_5=============%%%
\begin{ex}
	Đo chiều cao các em học sinh khối $10$ ta thu được kết quả
	\begin{center}
		\begin{tabular}{|c|c|c|c|c|c|c|}
			\hline Chiều cao &$[150; 152)$ &$[152; 154)$ &$[154; 156)$ &$[156; 158)$ &$[158; 160)$ &$[160; 162)$ \\
			\hline Số học sinh  & $5$ & $18$ & $40$ & $ 26$ & $8$ & $3$  \\
			\hline
		\end{tabular}
	\end{center}
	Phương sai của mẫu số liệu ghép nhóm trên là $\overline{abcde}$, với $a$, $b$, $c$, $d$, $e$ là các số tự nhiên. Khi đó $a+b+c+d+e$ bằng
	\choice
	{\True $20$}
	{$21$}
	{$22$}
	{$23$}
	\loigiai{
		Tổng số học sinh $100$.\\
		Giá trị trung bình của mẫu số liệu
		\begin{align*}
			\overline{x}=\dfrac{5\cdot 151+18\cdot 153+40\cdot 155+26\cdot 157+8\cdot 159+3\cdot 161}{100}=155{,}46.
		\end{align*}
		Khi đó phương sai của mẫu số liệu là
		\begin{align*}
			s_x^2&=\dfrac{5\left(151-155{,}46\right)^2+18\left(153-155{,}46\right)^2+40\left(155-155{,}46\right)^2+26\left(157-155{,}46\right)^2}{100}\\
			&+\dfrac{8\left(159-155{,}46\right)^2+3\left(161-155{,}46\right)^2}{100}\\
			&=4{,}7084.
		\end{align*} 
		Vậy $a+b+c+d+e=4+7+0+8+4=23$.
	}
\end{ex}

%%%=============EX_6=============%%%
\begin{ex}
	Bộ phận kiểm tra chất lượng sản phẩm dùng máy để đo (chính xác đến $0{,}001$ mm) độ dày của một chi tiết máy. Kết quả đo một số sản phẩm được thống kê trong bảng sau
	\begin{center}
		\begin{tabular}{|c|c|c|c|c|c|}
			\hline
			Độ dày & $[18;19)$ & $[19;20)$ & $[20;21)$ & $[21;22)$ & $[22;23)$ \\
			\hline
			Tần số & 3 & 7 & 23 & 25 & 2 \\
			\hline
		\end{tabular}
	\end{center}
	Nhận xét nào sau đây {\bf sai}?
	\choice
	{Cỡ mẫu của mẫu số liệu là $60$}
	{Số trung bình của mẫu số liệu gần bằng với $20{,}77$}
	{Độ dày của chi tiết máy không bị sai lệch nhiều}
	{\True Độ lệch chuẩn của mẫu lớn hơn $2$}
	\loigiai{
		\begin{itemize}
			\item Cỡ mẫu $n=60$.
			\item Số trung bình của mẫu số liệu là
			\begin{align*}
				\overline{x}=\dfrac{3\cdot 18{,}5+7\cdot 19{,}5+23\cdot 20{,}5+25\cdot 21{,}5+2\cdot 22{,}5}{60}=\dfrac{623}{30} \approx 20{,}77.
			\end{align*}
			\item Phương sai của mẫu số liệu là
			\begin{align*}
				s^2&=\dfrac{1}{60} \left(3\cdot 18{,}5^2+7\cdot 19{,}5^2+23\cdot 20{,}5^2+25\cdot 21{,}5^2+2\cdot 22{,}5^2\right)-\left(\dfrac{623}{30} \right)^2=\dfrac{179}{225}.
			\end{align*}
			\item Độ lệch chuẩn của mẫu số liệu là $s=\sqrt{\dfrac{179}{225}}=\dfrac{\sqrt{179}}{15} \approx 0{,}89$.
		\end{itemize}
	}
\end{ex}

%%%=============EX_7=============%%%
\begin{ex}[Đề thi thử TN THPT ThạchThành NH 24-25] Bảng dưới đây thống kê cự li ném tạ của một vận động viên
	\begin{center}
		\begin{tabular}{|c|c|c|c|c|c|}
			\hline
			Cự li (m) & $\left[19;19{,}5\right)$ & $\left[19{,}5;20\right)$ & $\left[20;20{,}5\right)$ & $\left[20{,}5,5;21\right)$ & $\left[21;21{,}5\right)$ \\
			\hline
			Tần số & $13$ & $45$ & $24$ & $12$ & $6$ \\
			\hline
		\end{tabular}
	\end{center}
	Phương sai của mẫu số liệu ghép nhóm trên là (làm tròn kết quả đến hàng phần trăm).
	\choice
	{\True $0{,}28$}
	{$0{,}22$}
	{$0{,}24$}
	{$0{,}26$}
	\loigiai{
		Ta có
		\begin{itemize}
			\item Cỡ mẫu $n=13+45+24+12+6=100$.
			\item Cự li ném trung bình $\overline{x}=\dfrac{13\cdot 19{,}25+45\cdot 19{,}75+24\cdot 20{,}25+12\cdot 20{,}75+6\cdot 21{,}25}{100}=20{,}015$.
			\item Phương sai của mẫu số liệu
			\begin{align*}
				s^2=&\dfrac{13\cdot \left(19{,}25-20{,}015\right)^2+45\cdot \left(19{,}75-20{,}015\right)^2+24\cdot \left(20{,}25-20{,}015\right)^2}{100}\\
				&+\dfrac{12\cdot \left(20{,}75-20{,}015\right)^2+6\cdot \left(21{,}25-20{,}015\right)^2}{100} \approx 0{,}28.
			\end{align*}
		\end{itemize}
	}
\end{ex}

%%%=============EX_8=============%%%
\begin{ex}[Đề thi thử TN Sở Thái Nguyên lần 2 NH 24-25] Một bác tài xế thống kê lại độ dài quãng đường (đơn vị: km) bác đã lái xe mỗi ngày trong một tháng ở bảng sau
	\begin{center}
		\begin{tabular}{|c|c|c|c|c|c|}
			\hline
			Độ dài quãng đường (km) & $\left[50;100\right)$ & $\left[100;150\right)$ & $\left[150;200\right)$ & $\left[200;250\right)$ & $\left[250;300\right)$ \\
			\hline
			Số ngày & $5$ & $10$ & $9$ & $4$ & $2$ \\
			\hline
		\end{tabular}
	\end{center}
	Độ lệch chuẩn của mẫu số liệu ghép nhóm này có giá trị gần nhất với giá trị nào dưới đây?
	\choice
	{\True $55{,}68$}
	{$55{,}25$}
	{$53{,}15$}
	{$60{,}24$}
	\loigiai{
		Ta có
		\begin{itemize}
			\item Giá trị trung bình của mẫu số liệu \[\overline{x}=\dfrac{75\cdot 5+125\cdot 10+175\cdot 9+225\cdot 4+275\cdot 2}{30}=155.\]
			\item Phương sai của mẫu số liệu là
			\[s^2=\dfrac{5\cdot \left(75-155\right)^2+10\cdot \left(125-155\right)^2+9\cdot \left(175-155\right)^2+4\cdot \left(225-155\right)^2+2\cdot \left(275-155\right)^2}{30}=3\,100.\]
			\item Độ lệch chuẩn của mẫu số liệu là
			$s=\sqrt{s^2}=55{,}68$.
		\end{itemize}
	}
\end{ex}

%%%=============EX_9=============%%%
\begin{ex}[Đề thi thử TN Sở Nghệ An lần 3 NH 24-25] Anh A rất thích chạy bộ. Thời gian chạy bộ mỗi ngày trong thời gian gần đây của anh A được thống kê lại ở bảng sau
	\begin{center}
		\begin{tabular}{|c|c|c|c|c|c|}
			\hline
			Thời gian (phút) & $\left[20;25\right)$ & $\left[25;30\right)$ & $\left[30;35\right)$ & $\left[35;40\right)$ & $\left[40;45\right)$ \\
			\hline
			Số ngày & $6$ & $6$ & $4$ & $1$ & $1$ \\
			\hline
		\end{tabular}
	\end{center}
	Độ lệch chuẩn của mẫu số liệu ghép nhóm là
	\choice
	{$31{,}25$}
	{$31$}
	{$5{,}59$}
	{\True $\dfrac{5\sqrt{5}}{2}$}
	\loigiai{
		\begin{itemize}
			\item Cỡ mẫu $n=18$.
			\item Giá trị trung bình $\overline{x}=\dfrac{1}{18} \left(6\cdot 22{,}5+6\cdot 27{,}5+4\cdot 32{,}5+37{,}5+42{,}5\right)=\dfrac{85}{3}$.
			\item Phương sai \[s^2=\dfrac{1}{18}\cdot \left[6\cdot\left(22{,}5-\dfrac{85}{3} \right)^2+6\cdot\left(27{,}5-\dfrac{85}{3} \right)^3+4\cdot\left(32{,}5-\dfrac{85}{3} \right)^2+\left(37{,}5-\dfrac{85}{3} \right)^2+\left(42{,}5-\dfrac{85}{3} \right)^2\right]=31{,}25.\]
			\item Độ lệch chuẩn $\sqrt{s^2}=\dfrac{5\sqrt{2}}{2}$.
		\end{itemize}
	}
\end{ex}

%%%=============EX_10=============%%%
\begin{ex}[Đề thi thử TN ĐH Quy Nhơn NH 24-25] Thống kê điểm trung bình môn Toán của các học sinh lớp $12$A được cho ở bảng sau
	\begin{center}
		\begin{tabular}{|c|c|c|c|c|}
			\hline
			Nhóm & $\left[6;7\right)$ & $\left[7;8\right)$ & $\left[8;9\right)$ & $\left[9;10\right)$ \\
			\hline
			Tần số & 2 & 8 & 18 & 12 \\
			\hline
		\end{tabular}
	\end{center}
	Phương sai của mẫu số liệu là
	\choice
	{$8{,}5$}
	{$0{,}15$}
	{\True $0{,}7$}
	{$6$}
	\loigiai{
		Chọn giá trị đại diện cho các nhóm số liệu
		\begin{center}
			\begin{tabular}{|c|c|c|c|c|}
				\hline
				Giá trị đại diện & $6{,}5$ & $7{,}5$ & $8{,}5$ & $9{,}5$ \\
				\hline
				Tần số & 2 & 8 & 18 & 12 \\
				\hline
			\end{tabular}
		\end{center}
		\begin{itemize}
			\item Tổng số học sinh trong lớp $12$A là $n=2+8+18+12=40$.
			\item Điểm trung bình môn Toán của các học sinh lớp 12A là
			\[\overline{x}=\dfrac{1}{40} \left(2\cdot 6{,}5+8\cdot 7{,}5+18\cdot 8{,}5+12\cdot 9{,}5\right)=8{,}5.\]
			\item Phương sai của mẫu số liệu là
			\[S^2=\dfrac{1}{40} \left(2\cdot 6{,}5^2+8\cdot 7{,}5^2+18\cdot 8{,}5^2+12\cdot 9{,}5^2\right)-8{,}5^2=0{,}7.\]
		\end{itemize}
		
	}
\end{ex}

%%%=============EX_11=============%%%
\begin{ex}[Đề thi thử TN Chuyên Vinh lần 2 NH 24-25] Mỗi ngày bác Bình đều đi bộ để rèn luyện sức khỏe. Quãng đường đi bộ mỗi ngày (đơn vị: km) của bác Bình trong $20$ ngày được thống kê lại ở bảng sau: 
	\begin{center}
		\begin{tabular}{|c|c|c|c|c|c|}
			\hline
			Quãng đường & $\left[2{,}7; 3{,}0\right)$ & $\left[3{,}0; 3{,}3\right)$ & $\left[3{,}3; 3{,}6\right)$ & $\left[3{,}6; 3{,}9\right)$ & $\left[3{,}9; 4{,}2\right)$ \\
			\hline
			Số ngày & $3$ & $6$ & $5$ & $4$ & $2$ \\
			\hline
		\end{tabular}
	\end{center}
	Độ lệch chuẩn của mẫu số liệu ghép nhóm có giá trị gần nhất với giá trị nào dưới đây?
	\choice
	{$0{,}36$}
	{$0{,}13$}
	{$3{,}39$}
	{\True $0{,}37$}
	\loigiai{
		\begin{center}
			\begin{tabular}{|c|c|c|c|c|c|}
				\hline
				Quãng đường & $\left[2{,}7; 3{,}0\right)$ & $\left[3{,}0; 3{,}3\right)$ & $\left[3{,}3; 3{,}6\right)$ & $\left[3{,}6; 3{,}9\right)$ & $\left[3{,}9; 4{,}2\right)$ \\
				\hline
				Giá trị đại diện & $2{,}85$ & $3{,}15$ & $3{,}3$ & $3{,}75$ & $4{,}05$ \\
				\hline
				Số ngày & $3$ & $6$ & $5$ & $4$ & $2$ \\
				\hline
			\end{tabular}
		\end{center}
		Số trung bình của mẫu số liệu ghép nhóm là
		\[\overline{x}=\dfrac{1}{20}\cdot \left(3\cdot 2{,}85+6\cdot 3{,}15+5\cdot 3{,}3+4\cdot 3{,}75+2\cdot 4{,}05\right)=3{,}3525.\]
		Độ lệch chuẩn của mẫu số liệu ghép nhóm là
		\[s=\sqrt{\dfrac{1}{20}\cdot \left(3\cdot 2{,}85^2+6\cdot 3{,}15^2+5\cdot 3{,}3^2+4\cdot 3{,}75^2+2\cdot 4{,}05^2\right)-3{,}3525^2} \approx 0{,}37.\]
	}
\end{ex}

%%%=============EX_12=============%%%
\begin{ex}[Đề thi thử TN Sở Nghệ An NH 24-25] Hai mẫu số liệu ghép nhóm $M_1$, $M_2$ có bảng tần số ghép nhóm như sau
	\begin{center}
		\begin{tabular}{|c|c|c|c|c|c|c|}
			\hline
			$M_1$ & Nhóm & $\left[1;3\right)$ & $\left[3;5\right)$ & $\left[5;7\right)$ & $\left[7;9\right)$ & $\left[9;11\right)$ \\
			\hline
			& Tần số & 6 & 12 & 10 & 8 & 4 \\
			\hline
			$M_2$ & Nhóm & $\left[1;3\right)$ & $\left[3;5\right)$ & $\left[5;7\right)$ & $\left[7;9\right)$ & $\left[9;11\right)$ \\
			\hline
			& Tần số & 3 & 6 & 5 & 4 & 2 \\
			\hline
		\end{tabular}
	\end{center}
	Gọi $s_1^2$, $s_2^2$ lần lượt là phương sai của mẫu số liệu ghép nhóm $M_1$, $M_2$. Phát biểu nào sau đây đúng?
	\choice
	{$s_1^2=4s_2^2$}
	{\True $s_1^2=s_2^2$}
	{$s_1^2=2s_2^2$}
	{$2s_1^2=s_2^2$}
	\loigiai{
		\begin{enumerate}[\bf Cách 1.]
			\item Vì Các nhóm ở hai mẫu $M_1$, $M_2$ giống hệt nhau, tần số mỗi nhóm tương ứng ở mẫu $M_1$ gấp đôi tần số mỗi nhóm tương ứng ở mẫu $M_2$. Nên phương sai của 2 mẫu bằng nhau.
			\item 
			\begin{itemize}
				\item Giá trị trung bình của mẫu số liệu $M_1$ là
				\[\overline{x}_1=\dfrac{2\cdot 6+4\cdot 12+6\cdot 10+8\cdot 8+10\cdot 4}{40}=5{,}6.\]
				\item Phương sai của mẫu số liệu $M_1$ là
				\[s_1 ^2=\dfrac{2^2\cdot 6+4^2\cdot 12+6^2\cdot 10+8^2\cdot 8+10^2\cdot 4}{40}-5{,}6^2=5{,}84.\]
				\item Giá trị trung bình của mẫu số liệu $M_2$ là
				\[\overline{x}_2=\dfrac{2\cdot 3+4\cdot 6+6\cdot 5+8\cdot 4+10.2}{20}=5{,}6.\]
				\item Phương sai của mẫu số liệu $M_2$ là
				\[s_2 ^2=\dfrac{2^2\cdot 3+4^2\cdot 6+6^2\cdot 5+8^2\cdot 4+10^2\cdot 2}{20}-5{,}6^2=5{,}84.\]
			\end{itemize}
			Vậy $s_1^2=s_2^2$.
		\end{enumerate}
	}
\end{ex}

%%%=============EX_13=============%%%
\begin{ex}[Đề thi thử TN Sở Nam Định lần 1 NH 24-25] Cho mẫu số liệu ghép nhóm có bảng tần số ghép nhóm như sau:
	\begin{center}
		\begin{tabular}{|c|c|c|c|c|c|}
			\hline
			Nhóm & $\left[0;2\right)$ & $\left[2;4\right)$ & $\left[4;6\right)$ & $\left[6;8\right)$ & $\left[8;10\right)$ \\
			\hline
			Tần số & $1$ & $2$ & $10$ & $15$ & $2$ \\
			\hline
		\end{tabular}
	\end{center}
	Phương sai của mẫu số liệu ghép nhóm trên là
	\choice
	{\True $\dfrac{43}{15}$}
	{$\dfrac{344}{225}$}
	{$\dfrac{17}{30}$}
	{$\dfrac{4}{3}$}
	\loigiai{
		\begin{itemize}
			\item Cỡ mẫu $n=1+2+10+15+2=30$.
			\item Số trung bình của mẫu số liệu ghép nhóm trên là
			\[\overline{x}=\dfrac{1\cdot 1+3\cdot 2+5\cdot 10+7\cdot 15+9\cdot 2}{30}=6.\]
			\item Phương sai của mẫu số liệu ghép nhóm trên là
			\[s^2=\dfrac{1}{30} \left[\left(1-6\right)^2+2\left(3-6\right)^2+10\left(5-6\right)^2+15\left(7-6\right)^2+2\left(9-6\right)^2\right]=\dfrac{43}{15}.\]
		\end{itemize}
	}
\end{ex}

%%%=============EX_14=============%%%
\begin{ex}[Đề thi thử TN Sở Cà Mau NH 24-25] Bảng dưới đây thống kê cự li ném tạ của một vận động viên.
	\begin{center}
		\begin{tabular}{|c|c|c|c|c|c|}
			\hline
			Cự li (m) & $\left[19; \; 19{,}5\right)$ & $\left[19{,}5; \; 20\right)$ & $\left[20; \; 20{,}5\right)$ & $\left[20{,}5; \; 21\right)$ & $\left[21; \; 21{,}5\right)$ \\
			\hline
			Tần số & $13$ & $45$ & $24$ & $12$ & $6$ \\
			\hline
		\end{tabular}
	\end{center}
	Phương sai của mẫu số liệu ghép nhóm trên (làm tròn đến hàng phần trăm) là
	\choice
	{$0{,}29$}
	{$0{,}27$}
	{\True $0{,}28$}
	{$0{,}26$}
	\loigiai{
		\begin{center}
			\begin{tabular}{|c|c|c|c|c|c|}
				\hline
				Cự li & $\left[19;19{,}5\right)$ & $\left[19{,}5;20\right)$ & $\left[20;20{,}5\right)$ & $\left[20{,}5;21\right)$ & $\left[21;21{,}5\right)$ \\
				\hline
				Giá trị đại diện & $19{,}25$ & $19{,}75$ & $20{,}25$ & $20{,}75$ & $21{,}25$ \\
				\hline
				Tần số & $13$ & $45$ & $24$ & $12$ & $6$ \\
				\hline
			\end{tabular}
		\end{center}
		Ta có
		\begin{itemize}
			\item Số trung bình của mẫu số liệu \[\overline{x}=\dfrac{19{,}25\cdot 13+19{,}75\cdot 45+20{,}25\cdot 24+20{,}75\cdot 12+21{,}25.6}{100}=20{,}015\]
			\item Phương sai
			\[s^2=\dfrac{19{,}25^2\cdot 13+19{,}75^2\cdot 45+20{,}25^2\cdot 24+20{,}75^2\cdot 12+21{,}25^2\cdot 6}{100}-\left(20{,}015\right)^2\approx 0{,}28.\]
		\end{itemize}
	}
\end{ex}

%%%=============EX_15=============%%%
\begin{ex}[Đề thi thử TN Sở Khánh Hoà NH 24-25] Cho mẫu số liệu ghép nhóm có bảng tần số ghép nhóm như sau:
	\begin{center}
		\begin{tabular}{|c|c|c|c|c|c|c|}
			\hline
			Nhóm & $\left[40;45\right)$ & $\left[45;50\right)$ & $\left[50;55\right)$ & $\left[55;60\right)$ & $\left[60;65\right)$ & $\left[65;70\right)$ \\
			\hline
			Tần số & $4$ & $11$ & $7$ & $8$ & $8$ & $2$ \\
			\hline
		\end{tabular}
	\end{center}
	Độ lệch chuẩn của mẫu số liệu ghép nhóm trên (làm tròn kết quả đến hàng phần mười) là
	\choice
	{$53{,}9$}
	{$53{,}6$}
	{$51{,}2$}
	{\True $7{,}2$}
	\loigiai{
		\begin{itemize}
			\item Số trung bình $\overline{x}=\dfrac{42{,}5\cdot 4+47{,}5\cdot 11+52{,}5\cdot 7+57{,}5\cdot 8+62{,}5\cdot 8+67{,}5\cdot 2}{4+11+7+8+8+2}=53{,}875$
			\item Phương sai $s^2=\dfrac{1}{40} \left[4\cdot 42{,}5^2+11\cdot 47{,}5^2+7\cdot 52{,}5^2+8\cdot 57{,}5^2+8\cdot 62{,}5^2+2\cdot 67{,}5^2\right]-53{,}875^2=\dfrac{3279}{64}$.
			\item Độ lệch chuẩn $s=\sqrt{s^2}=\sqrt{\dfrac{3279}{64}} \approx 7{,}2$.
		\end{itemize}
	}
\end{ex}

%%%=============EX_16=============%%%
\begin{ex}[Sáng tác Strong đề thi thử TN NH 24-25] Cân nặng (kg) của một số quả mít trong một khu vườn được thống kê ở bảng sau 
	\begin{center}
		\begin{tabular}{|c|c|c|c|c|c|}
			\hline
			Cân nặng (kg) & $\left[4;6\right)$ & $\left[6;8\right)$ & $\left[8;10\right)$ & $\left[10;12\right)$ & $\left[12;14\right)$ \\
			\hline
			Số cây giống & $6$ & $12$ & $19$ & $9$ & $4$ \\
			\hline
		\end{tabular}
	\end{center}
	Hãy tính phương sai của mẫu số liệu ghép nhóm trên (kết quả làm tròn đến hàng phần mười).
	\choice
	{$4{,}7$}
	{$4{,}6$}
	{$1{,}9$}
	{\True $4{,}8$}
	\loigiai{
		Ta có bảng số liệu ghép nhóm
		\begin{center}
			\begin{tabular}{|c|c|c|c|c|c|}
				\hline
				Cân nặng (kg) & $\left[4;6\right)$ & $\left[6;8\right)$ & $\left[8;10\right)$ & $\left[10;12\right)$ & $\left[12;14\right)$ \\
				\hline
				Giá trị đại diện & $5$ & $7$ & $9$ & $11$ & $13$ \\
				\hline
				Số cây giống & $6$ & $12$ & $19$ & $9$ & $4$ \\
				\hline
			\end{tabular}
		\end{center}
		\begin{itemize}
			\item Số trung bình cộng của mẫu số liệu ghép nhóm trên là \[\overline{x}=\dfrac{5\cdot 6+7\cdot 12+9\cdot 19+11\cdot 9+13\cdot 4}{47}=8{,}72\]
			\item Phương sai của mẫu số liệu ghép nhóm trên là
			\[s^2=\dfrac{1}{50} \left[6\left(5-\dfrac{436}{47} \right)^2+12\left(7-\dfrac{436}{47} \right)^2+19\left(9-\dfrac{436}{47} \right)^2+9\left(11-\dfrac{436}{47} \right)^2+\left(13-\dfrac{436}{47} \right)^2\right]=4{,}8.\]
		\end{itemize}
	}
\end{ex}

%%%=============EX_17=============%%%
\begin{ex}[Đề thi thử TN Sở Trà Vinh NH 24-25] Xét mẫu số liệu ghép nhóm cho ở Bảng 1. Gọi $\overline{x}$ là số trung bình cộng của mẫu số liệu ghép nhóm \begin{center}
		\begin{tabular}{|c|c|c|}
			\hline
			Nhóm & Giá trị đại diện & Tần số \\
			\hline
			$\left[100;130\right)$ & $115$ & $7$ \\
			\hline
			$\left[130;160\right)$ & $145$ & $15$ \\
			\hline
			$\left[160;190\right)$ & $175$ & $12$ \\
			\hline
			$\left[190;220\right)$ & $205$ & $7$ \\
			\hline
			$\left[220;250\right)$ & $235$ & $9$ \\
			\hline
			& & $n=50$ \\
			\hline
		\end{tabular}
	\end{center}
	Bảng 1
	Phương sai của mẫu số liệu ghép nhóm, kí hiệu $s^2$, là một số được tính theo công thức nào dưới đây?
	\choice
	{$S^2=\sqrt{\dfrac{7\left(115-\overline{x}\right)^2+15\left(145-\overline{x}\right)^2+12\left(175-\overline{x}\right)^2+7\left(205-\overline{x}\right)^2+9\left(235-\overline{x}\right)^2}{50}}$}
	{$s^2=\dfrac{7\left(115-\overline{x}\right)+15\left(145-\overline{x}\right)+12\left(175-\overline{x}\right)+7\left(205-\overline{x}\right)+9\left(235-\overline{x}\right)}{50}$}
	{$s^2=\dfrac{7^2\left(115-\overline{x}\right)^2+15^2\left(145-\overline{x}\right)^2+12^2\left(175-\overline{x}\right)^2+7^2\left(205-\overline{x}\right)^2+9^2\left(235-\overline{x}\right)^2}{50}$}
	{\True $s^2=\dfrac{7\left(115-\overline{x}\right)^2+15\left(145-\overline{x}\right)^2+12\left(175-\overline{x}\right)^2+7\left(205-\overline{x}\right)^2+9\left(235-\overline{x}\right)^2}{50}$}
	\loigiai{
		Có công thức $s^2=\dfrac{n_1 \left(x_1-\overline{x}\right)^2+n_2 \left(x_2-\overline{x}\right)^2+\cdots+n_k \left(x_k-\overline{x}\right)^2}{n}$.\\
		Với $x_i$ là các giá trị đại diện của các nhóm, $n_i$ là tần số của các nhóm và $n=n_1+n_2+\cdots+n_k$.
	}
\end{ex}

%%%=============EX_18=============%%%
\begin{ex}[Đề thi thử TN Chuyên Lam Sơn NH 24-25] Một siêu thị thống kê số tiền (đơn vị: chục nghìn đồng) mà $44$ khách hàng mua hàng ở siêu thị đó trong một ngày. Số liệu được cho ở bảng sau
	\begin{center}
		\begin{tabular}{|c|c|c|c|c|c|c|}
			\hline
			Nhóm & $\left[40;45\right)$ & $\left[45;50\right)$ & $\left[50;55\right)$ & $\left[55;60\right)$ & $\left[60;65\right)$ & $\left[65;70\right)$ \\
			\hline
			Tần số & $4$ & $14$ & $8$ & $10$ & $6$ & $2$ \\
			\hline
		\end{tabular}
	\end{center}
	Biết trung bình cộng của mẫu số liệu đã cho là $\overline{x}=53{,}18$. Phương sai của mẫu số liệu ghép nhóm (kết quả làm tròn đến hàng phần mười) là
	\choice
	{$s^2=46{,}2$}
	{$s^2=46{,}12$}
	{$s^2=46{,}21$}
	{\True $S^2=46{,}1$}
	\loigiai{
		Sử dụng công thức tính phương sai cho mẫu số liệu ghép nhóm \[s^2=\dfrac{1}{n} \sum _{i=1}^kn_i (x_i-\overline{x})^2.\]
		Trong đó $n=44$, $\overline{x}=53{,}18$, $n_i$ là tần số và $x_i$ là giá trị đại diện của mỗi lớp.\\
		Tính $s^2=\dfrac{\sum n_i (x_i-\overline{x})^2}{N}$.
		\begin{align*}
			s^2&=\dfrac{4(42{,}5-53{,}18)^2+14(47{,}5-53{,}18)^2+8(52{,}5-53{,}18)^2}{44}\\
			&+\dfrac{10(57{,}5-53{,}18)^2+6(62{,}5-53{,}18)^2+2(67{,}5-53{,}18)^2}{44}\\
			\approx& 46{,}126.
		\end{align*}
		Làm tròn đến hàng phần mười, ta được $s^2\approx 46{,}1$.
	}
\end{ex}

%%%=============EX_19=============%%%
\begin{ex}[Sáng tác Strong đề thi thử TN NH 24-25] Tìm độ lệch chuẩn của mẫu số liệu ghép nhóm được cho ở bảng sau (làm tròn kết quả đến hàng phần mười)
	\begin{center}
		\begin{tabular}{|c|c|}
			\hline
			Nhóm & Tần số \\
			\hline
			$\left[25;35\right)$ & 10 \\
			\hline
			$\left[35;45\right)$ & 7 \\
			\hline
			$\left[45;55\right)$ & 5 \\
			\hline
			$\left[65;75\right)$ & 9 \\
			\hline
			$\left[75;85\right)$ & 9 \\
			\hline
			& $n=40$ \\
			\hline
		\end{tabular}
	\end{center}
	\choice
	{$15{,}1$}
	{$15{,}0$}
	{$14{,}8$}
	{\True $14{,}9$}
	\loigiai{
		Ta có bảng thống kê sau
		\begin{center}
			\begin{tabular}{|c|c|c|}
				\hline
				Nhóm & Giá trị đại diện & Tần số \\
				\hline
				$\left[25;35\right)$ & 30 & 9 \\
				\hline
				$\left[35;45\right)$ & 40 & 7 \\
				\hline
				$\left[45;55\right)$ & 50 & 5 \\
				\hline
				$\left[65;75\right)$ & 60 & 10 \\
				\hline
				$\left[75;85\right)$ & 70 & 9 \\
				\hline
				& & $n=40$ \\
				\hline
			\end{tabular}
		\end{center}
		\begin{itemize}
			\item Số trung bình cộng của mẫu số liệu ghép nhóm là:
			\[\overline{x}=\dfrac{30\cdot 9+40\cdot 7+50\cdot 5+60\cdot 10+70\cdot 9}{40}=50{,}75\]
			\item Phương sai của mẫu số liệu là:
			\begin{align*}
				s^2=&\dfrac{9\cdot \left(30-50{,}75\right)^2+7\cdot \left(40-50{,}75\right)^2+5\cdot \left(50-50{,}75\right)^2+10\cdot \left(60-50{,}75\right)^2+9\cdot \left(70-50{,}75\right)^2}{40}\\
				=&221{,}9375.
			\end{align*}
			\item Độ lệch chuẩn của mẫu số liệu trên là $s=\sqrt{221{,}9375} \approx 14{,}9$.
		\end{itemize}
	}
\end{ex}

%%%=============EX_20=============%%%
\begin{ex}[Sáng tác Strong đề thi thử TN NH 24-25] Khảo sát thời gian tự học bài ở nhà của học sinh khối 9 ở trường X, ta thu được bảng sau
	\begin{center}
		\begin{tabular}{|c|c|c|c|c|c|}
			\hline
			Thời gian (phút) & $\left[0; 30\right)$ & $\left[30; 60\right)$ & $\left[60; 90\right)$ & $\left[90; 120\right)$ & $\left[120; 150\right)$ \\
			\hline
			Số học sinh & 9 & 10 & 9 & 15 & 7 \\
			\hline
		\end{tabular}
	\end{center}
	Phương sai của mẫu số liệu ghép nhóm là
	\choice
	{$1602$}
	{\True $1601{,}64$}
	{$1601{,}9$}
	{$1603$}
	\loigiai{
		\begin{center}
			\begin{tabular}{|c|c|c|c|c|c|}
				\hline
				Thời gian (phút) & $\left[0; 30\right)$ & $\left[30; 60\right)$ & $\left[60; 90\right)$ & $\left[90; 120\right)$ & $\left[120; 150\right)$ \\
				\hline
				Giá trị đại diện & 15 & 45 & 75 & 105 & 135 \\
				\hline
				Số học sinh & 9 & 10 & 9 & 15 & 7 \\
				\hline
			\end{tabular}
		\end{center}
		\begin{itemize}
			\item Thời gian trung bình tự học ở nhà của các em học sinh đó là
			\[\overline{x}=\dfrac{9\cdot 15+10\cdot 45+9\cdot 75+15\cdot 105+7\cdot 135}{50}=75{,}6\text{ (phút)}.\]
			\item Phương sai của mẫu số kiệu ghép nhóm là
			\begin{align*}
				s^2=&\dfrac{1}{50} \left[9\cdot \left(15-75{,}6\right)^2+10\cdot \left(45-75{,}6\right)^2+9\cdot \left(75-75{,}6\right)^2+15\cdot \left(105-75{,}6\right)^2+7\cdot \left(135-75{,}6\right)^2\right]\\
				=&1\,601{,}64.
			\end{align*}
		\end{itemize}
	}
\end{ex}

\Closesolutionfile{ans}

\ind{PHẦN II.} \inden{Câu trắc nghiệm đúng sai. Trong mỗi ý a), b), c), d) ở mỗi câu, học sinh chọn đúng hoặc sai.}\\
\setcounter{ex}{0}
\Opensolutionfile{ans}[ans/2D1-Bai1-DS]%--Đặt tên 2D1-Bai1-DS
%%%=============EX_1=============%%%
\begin{ex}
	[Sáng tác Strong đề thi thử TN NH 24-25] Cho mẫu số liệu dưới dạng bảng sau
	\begin{center}
		\begin{tabular}{|c|c|c|c|c|c|}
			\hline
			Số câu trả lời đúng & $\left[16;21\right)$ & $\left[21;26\right)$ & $\left[26;31\right)$ & $\left[31;36\right)$ & $\left[36;41\right)$ \\
			\hline
			Tần số & $4$ & $6$ & $8$ & $18$ & $4$ \\
			\hline
		\end{tabular}
	\end{center}
	\choiceTF
	{\True Giá trị đại diện của lớp $\left[36;41\right)$ là $38{,}5$}
	{\True Nhóm có tần số lớn nhất là nhóm $\left[31;36\right)$}
	{\True Số trung bình là $30$}
	{Phương sai của mẫu số liệu là $S=\sqrt{32{,}75}$}
	\loigiai{
		\begin{itemchoice}
			\itemch  Giá trị đại diện của lớp $\left[36;41\right)$ là $38{,}5$.
			\itemch  Tần số lớn nhất là $18$, nên nhóm có tần số lớn nhất là nhóm $\left[31;36\right)$.
			\itemch  Công thức tính số trung bình là $\overline{x}=\dfrac{18{,}5\cdot 4+23{,}5\cdot 6+28{,}5\cdot 8+33{,}5\cdot 18+38{,}5\cdot 4}{40}=30$.
			\itemch  Phương sai của mẫu số liệu là \allowdisplaybreaks
			\begin{align*}
				s^2&=\dfrac{1}{40}\left[4\cdot(18{,}5-30)^2+6\cdot(23{,}5-30)^2+8\cdot(28{,}5-30)^2+18\cdot(33{,}5-30)^2+4\cdot(38{,}5-30)^2\right]\\
				&=32{,}75.
			\end{align*}
		\end{itemchoice}
	}
\end{ex}

%%%=============EX_2=============%%%
\begin{ex}
	[Sáng tác Strong đề thi thử TN NH 24-25] Một người đầu tư cùng một số tiền vào hai lĩnh vực A và B. Nhà đầu tư này ghi lại số tiền thu được hằng tháng trong hai năm theo mỗi lĩnh vực cho kết quả như sau
	\begin{center}
		\begin{tabular}{|c|c|c|c|c|c|}
			\hline
			Số tiền (triệu đồng) & $\left[5;10\right)$ & $\left[10;15\right)$ & $\left[15;20\right)$ & $\left[20;25\right)$ & $\left[25;30\right)$ \\
			\hline
			Số tháng theo lĩnh vực A & $5$ & $3$ & $4$ & $10$ & $2$ \\
			\hline
			Số tháng theo lĩnh vực B & $3$ & $5$ & $5$ & $9$ & $2$ \\
			\hline
		\end{tabular}
	\end{center}
	\choiceTF
	{\True Khoảng biến thiên số tiền của cả hai lĩnh vực A và B là như nhau}
	{Số tiền trung bình của lĩnh vực A trong $24$ tháng đạt dưới $15$ triệu đồng}
	{Nếu so sánh theo số trung bình thì lĩnh vực A tốt hơn lĩnh vực B}
	{\True Nếu so sánh theo độ lệch chuẩn thì lĩnh vực A rủi ro hơn lĩnh vực B}
	\loigiai{
		\begin{itemchoice}
			\itemch  Khoảng biến thiên số tiền của lĩnh vực A và B là bằng nhau, đều là $R=30-5=25$.
			\itemch  Số tiền trung bình của lĩnh vực A trong $24$ tháng là
			\[\overline{x}_A=\dfrac{7{,}5\cdot 5+12{,}5\cdot 3+17{,}5\cdot 4+22{,}5\cdot 10+27{,}5\cdot 2}{24} \approx 17{,}71> 15.
			\]
			\itemch  Số tiền trung bình của lĩnh vực B trong $24$ tháng là
			\[\overline{x}_B=\dfrac{7{,}5\cdot 3+12{,}5\cdot 5+17{,}5\cdot 5+22{,}5\cdot 9+27{,}5\cdot 2}{24} \approx 17{,}92.
			\]
			Vì giá trị trung bình của lĩnh vực B cao hơn giá trị trung bình của lĩnh vực A trong $24$ tháng nên đầu tư vào lĩnh vực B tốt hơn lĩnh vực A.
			\itemch  Độ lệch chuẩn của lĩnh vực A là
			\[S_A=\sqrt{\dfrac{5\left(7{,}5-\overline{x}_A\right)^2+3\left(12{,}5-\overline{x}_A\right)^2+4\left(17{,}5-\overline{x}_A\right)^2+10\left(22{,}5-\overline{x}_A\right)^2+2\left(27{,}5-\overline{x}_A\right)^2}{24}} \approx 6{,}53.
			\]
			Độ lệch chuẩn của lĩnh vực B là
			\[S_A=\sqrt{\dfrac{3\left(7{,}5-\overline{x}_B\right)^2+5\left(12{,}5-\overline{x}_B\right)^2+5\left(17{,}5-\overline{x_B}\right)^2+9\left(22{,}5-\overline{xx}_B\right)^2+2\left(27{,}5-\overline{x}_B\right)^2}{24}} \approx 5{,}94.
			\]
			Vì độ lệch chuẩn của lĩnh vực A cao hơn độ lệch chuẩn của lĩnh vực B nên lĩnh vực A rủi ro hơn lĩnh vực B.
		\end{itemchoice}
	}
\end{ex}

%%%=============EX_3=============%%%
\begin{ex}
	Biểu đồ sau mô tả kết quả điều tra về điểm trung bình năm học của học sinh hai trường A và B.
	\begin{center}
		\begin{tikzpicture}[line join=round, line cap=round,>=stealth,xscale=1,yscale=1]
			\def\a{1}
			\def\xmax{12}
			\def\ymax{8}
			\tikzset{label style/.style={font=\footnotesize}}
			\draw[->] (0,0)--(\xmax,0) node[below] {\text{Điểm trung bình}};
			\draw[->] (0,0)--(0,\ymax) node[below left] {\text{Số học sinh}};
			\draw (0,0) node [below left] {$0$};
			\draw[dashed,thin,gray!60]
			(0,1)--(\xmax,1)
			(0,2)--(\xmax,2)
			(0,3)--(\xmax,3)
			(0,4)--(\xmax,4)
			(0,5)--(\xmax,5)
			(0,6)--(\xmax,6)
			(0,7)--(\xmax,7)
			;
			\foreach \i in {1,2,3,4,5,6,7}{
				\draw (0,\i) node[left]{$\i$};
			}
			\foreach \i/\j in {1*\a/[5;6),3*\a/{[6;7)},5*\a/{[7;8)},7*\a/{[8;9)},9*\a/{[9;10)}}{
				\draw (\i,-.5) node[below]{\small $\j$};
			}
			\fill[pattern = north west lines,pattern color=blue]
			(0,0) rectangle (\a,4)
			(2*\a,0) rectangle (3*\a,5)
			(4*\a,0) rectangle (5*\a,3)
			(6*\a,0) rectangle (7*\a,4)
			(8*\a,0) rectangle (9*\a,2);
			\fill[pattern = dots,pattern color=violet]
			(\a,0) rectangle (2*\a,2)
			(3*\a,0) rectangle (4*\a,5)
			(5*\a,0) rectangle (6*\a,4)
			(7*\a,0) rectangle (8*\a,3)
			(9*\a,0) rectangle (10*\a,1);
			\begin{scope}
				\draw 
				(0,4)--(\a,4) 
				(\a,2)--(2*\a,2)
				(2*\a,5)--(3*\a,5)
				(3*\a,5)--(4*\a,5)
				(4*\a,3)--(5*\a,3)
				(5*\a,4)--(6*\a,4)
				(6*\a,4)--(7*\a,4)
				(7*\a,3)--(8*\a,3)
				(8*\a,2)--(9*\a,2)
				(9*\a,1)--(10*\a,1)
				(0,0)--(0,4)
				(\a,0)--(\a,4)
				(2*\a,0)--(2*\a,5)
				(3*\a,0)--(3*\a,5)
				(4*\a,0)--(4*\a,5)
				(5*\a,0)--(5*\a,4)
				(6*\a,0)--(6*\a,4)
				(7*\a,0)--(7*\a,4)
				(8*\a,0)--(8*\a,3)
				(9*\a,0)--(9*\a,2)
				(10*\a,0)--(10*\a,1);
				\draw (6,\ymax+1) node[above]{\textbf{Điểm trung bình năm học của học sinh hai trường A và B}};
			\end{scope}
			\fill[pattern = north west lines,pattern color=blue,draw] (10,6.5) rectangle(10.5,7)node[midway, right=2mm]{Học sinh trường A};
			\fill[pattern = dots,pattern color=violet,draw] (10,5.5) rectangle (10.5,6)node[midway, right=2mm]{Học sinh trường B};
		\end{tikzpicture}
	\end{center}
	Người ta lập được bảng tần số ghép nhóm cho mẫu số liệu trên như sau
	\begin{center}
		\begin{tabular}{|c|c|c|c|c|c|}
			\hline
			Điểm trung bình & $[5;6)$ & $[6;7)$ & $[7;8)$ & $[8;9)$ & $[9;10)$ \\
			\hline
			Học sinh trường A & $4$ & $5$ & $3$ & $4$ & $2$ \\
			\hline
			Học sinh trường B & $2$ & $5$ & $4$ & $3$ & $1$ \\
			\hline
		\end{tabular}
	\end{center}
	\choiceTF
	{\True Tứ phân vị thứ nhất của mẫu số liệu ghép nhóm của học sinh trường A là $6{,}1$}
	{Khoảng tứ phân vị của mẫu số liệu ghép nhóm của học sinh trường B là $1{,}23$}
	{\True Nếu so sánh theo khoảng tứ phân vị của mẫu số liệu ghép nhóm thì học sinh trường B có điểm trung bình đồng đều hơn}
	{Nếu so sánh theo độ lệch chuẩn của mẫu số liệu ghép nhóm thì học sinh trường A có điểm trung bình đồng đều hơn}
	\loigiai{
		\begin{itemchoice}
			\itemch Xét điểm trung bình của học sinh trường A, ta có\\
			Cỡ mẫu $n_A=18$.\\
			Gọi $x_1$; $x_2$; $\ldots$; $x_{18}$ là mẫu số liệu gốc về điểm trung bình năm học của học sinh hai trường A được xếp theo thứ tự không giảm.\\
			Ta có 
			\begin{itemize}
				\item $x_1;\ldots;x_4 \in [5;6)$; 
				\item $x_5;\ldots;x_9 \in [6;7)$; 
				\item $x_{10};\ldots;x_{12} \in [7;8)$; 
				\item $x_{13};\ldots;x_{16} \in [8;9)$;
				\item $x_{17};x_{18} \in [9;10)$.
			\end{itemize}
			Tứ phân vị thứ nhất của mẫu số liệu gốc là $x_5 \in [6;7)$. \\
			Do đó, tứ phân vị thứ nhất của mẫu số liệu ghép nhóm là $Q_1=6+\dfrac{\dfrac{18}{4}-4}{5} (7-6)=6{,}1$.\\
			Tứ phân vị thứ ba của mẫu số liệu gốc là $x_{14} \in [8;9)$. \\
			Do đó, tứ phân vị thứ ba của mẫu số liệu ghép nhóm là \[Q_3=8+\dfrac{\dfrac{3\cdot 18}{4}-(4+5+3)}{4} (9-8)=8{,}375.\]
			Khoảng tứ phân vị của mẫu số liệu ghép nhóm là $\Delta_Q=Q_3-Q_1=2{,}275$.
			\itemch Xét điểm trung bình của học sinh trường B, ta có\\
			Cỡ mẫu $n_B=15$.\\
			Gọi $y_1$; $y_2$; $\ldots$; $y_{15}$ là mẫu số liệu gốc về điểm trung bình năm học của học sinh hai trường B được xếp theo thứ tự không giảm.\\
			Ta có 
			\begin{itemize}
				\item $y_1;y_2 \in [5;6)$; 
				\item $y_3;\ldots;y_7 \in [6;7)$; 
				\item $y_8;\ldots;y_{11} \in [7;8)$; 
				\item $y_{12};\ldots;y_{14} \in [8;9)$; 
				\item $y_{15} \in [9;10)$.
			\end{itemize}
			Tứ phân vị thứ nhất của mẫu số liệu gốc là $y_4 \in [6;7)$. \\
			Do đó, tứ phân vị thứ nhất của mẫu số liệu ghép nhóm là $Q'_1=6+\dfrac{\dfrac{15}{4}-2}{5} (7-6)=6{,}35$.\\
			Tứ phân vị thứ ba của mẫu số liệu gốc là $y_{12} \in [8;9)$. \\
			Do đó, tứ phân vị thứ ba của mẫu số liệu ghép nhóm là $Q'_3 =8+\dfrac{\dfrac{3\cdot 15}{4}-(2+5+4)}{3} (9-8)=8{,}08$.\\
			Khoảng tứ phân vị của mẫu số liệu ghép nhóm là $\Delta_Q'=Q'_3 -Q_1=1{,}73$.
			\itemch Vì $\Delta_Q>\Delta_Q'$ nên học sinh trường B có điểm trung bình đồng đều hơn.
			\itemch \begin{itemize}
				\item Xét số liệu của trường A.
				\begin{itemize}
					\item Số trung bình $\overline{x}_A=\dfrac{4\cdot 5{,}5+5\cdot 6{,}5+3\cdot 7{,}5+4\cdot 8{,}5+2\cdot 9{,}5}{18}=7{,}22$.\\
					\item Độ lệch chuẩn $s_A=\sqrt{\dfrac{4\cdot 5{,}5^2+5\cdot 6{,}5^2+3\cdot 7{,}5^2+4\cdot 8{,}5^2+2\cdot 9{,}5^2}{18}-7{,}22^2} \approx 1{,}79$.
				\end{itemize}
				\item Xét số liệu của trường B.
				\begin{itemize}
					\item Số trung bình $\overline{x}_B=\dfrac{2\cdot 5{,}5+5\cdot 6{,}5+4\cdot 7{,}5+3\cdot 8{,}5+1\cdot 9{,}5}{15}=7{,}23$.\\
					\item Độ lệch chuẩn $s_B=\sqrt{\dfrac{2\cdot 5{,}5^2+5\cdot 6{,}5^2+4\cdot 7{,}5^2+3\cdot 8{,}5^2+1\cdot 9{,}5^2}{15}-7{,}23^2} \approx 1{,}31$.
				\end{itemize}
				Vì $s_B<s_A$ nên học sinh trường B có điểm trung bình đồng đều hơn.
			\end{itemize}
		\end{itemchoice}
	}
\end{ex}

%%%=============EX_4=============%%%
\begin{ex}
	Thầy Tuấn thống kê lại điểm trung bình cuối năm của các học sinh lớp $11$A và $11$B ở bảng sau
	\begin{center}
		\begin{tabular}{|c|c|c|c|c|c|}
			\hline
			\diagbox{Lớp}{Điểm trung bình} &{$[5; 6)$} &{$[6; 7)$} &{$[7; 8)$} &{$[8; 9)$} &{$[9; 10)$} \\
			\hline 11A & $ 1 $ & $ 0 $ & $ 11 $ & $ 22 $ & $ 6 $ \\
			\hline 11B & $ 0 $ & $ 6 $ & $ 8 $ & $ 14 $ & $ 12 $ \\
			\hline
		\end{tabular}
	\end{center}
	\choiceTF
	{\True Khoảng biến thiên của điểm số học sinh lớp $11$A là $5$}
	{\True Nếu so sánh theo khoảng biến thiên thì điểm trung bình của các học sinh lớp $11$B ít phân tán hơn điểm trung bình của các học sinh lớp $11$A}
	{Xét mẫu số liệu của lớp $11$A ta có độ lệch chuẩn của mẫu số liệu ghép nhóm là $\sqrt{0{,}51}$}
	{\True Nếu so sánh theo độ lệch chuẩn thì học sinh lớp $11$A có điểm trung bình ít phân tán hơn học sinh lớp $11$B}
	\loigiai{
		\begin{itemchoice}
			\itemch Khoảng biến thiên của điểm số học sinh lớp $11$A là $R_A=10-5=5$.
			\itemch Khoảng biến thiên của điểm số học sinh lớp $11$B là $R_B=10-6=4$.\\
			Vì $R_B<R_A$ nên học sinh lớp $11$B có điểm trung bình ít phân tán hơn học sinh lớp $11$A.
			\itemch Ta có bảng thống kê điểm trung bình theo giá trị đại diện
			\begin{center}
				\begin{tabular}{|c|c|c|c|c|c|}
					\hline
					\diagbox{Lớp}{Giá trị đại diện} &{$5{,}5$} &{$6{,}5$} &{$7{,}5$} &{$8{,}5$} &{$9{,}5$} \\
					\hline 11A & $ 1 $ & $ 0 $ & $ 11 $ & $ 22 $ & $ 6 $ \\
					\hline 11B & $ 0 $ & $ 6 $ & $ 8 $ & $ 14 $ & $ 12 $ \\
					\hline
				\end{tabular}
			\end{center}
			Xét mẫu số liệu của lớp $11$A, ta có
			\begin{itemize}
				\item Cỡ mẫu là $n_1=1+11+22+6=40$.
				\item Số trung bình của mẫu số liệu ghép nhóm là
				\begin{align*}
					\overline{x}_1=\dfrac{1\cdot 5{,}5+11\cdot 7{,}5+22\cdot 8{,}5+6\cdot 9{,}5}{40}=8{,}3.
				\end{align*}
				\item Phương sai của mẫu số liệu ghép nhóm là
				\begin{align*}
					s_1^2=\dfrac{1}{40} \left(1\cdot 5{,}5^2+11\cdot 7{,}5^2+22\cdot 8{,}5^2+6\cdot 9{,}5^2\right)-8{,}3^2=0{,}61.
				\end{align*}
				\item Độ lệch chuẩn của mẫu số liệu ghép nhóm là $s_1=\sqrt{0{,}61}$.
			\end{itemize}
			\itemch Xét mẫu số liệu của lớp $11$B, ta có
			\begin{itemize}
				\item Cỡ mẫu là $n_2=6+8+14+12=40$.
				\item Số trung bình của mẫu số liệu ghép nhóm là
				\begin{align*}
					\overline{x}_2=\dfrac{6\cdot 6{,}5+8\cdot 7{,}5+14\cdot 8{,}5+12\cdot 9{,}5}{40}=8{,}3.
				\end{align*}
				Phương sai của mẫu số liệu ghép nhóm là
				\begin{align*}
					s_2^2=\dfrac{1}{40} \left(6\cdot 6{,}5^2+8\cdot 7{,}5^2+14\cdot 8{,}5^2+12\cdot 9{,}5^2\right)-8{,}3^2=1{,}06.
				\end{align*}
				Độ lệch chuẩn của mẫu số liệu ghép nhóm là $s_2=\sqrt{1{,}06}$.
			\end{itemize}
			Do $s_1 < s_2$ nên học sinh lớp $11$A có điểm trung bình ít phân tán hơn học sinh lớp $11$B.
		\end{itemchoice}
	}
	\begin{luuy}
		Trong bài toán trên, kết quả so sánh độ phân tán theo giá trị trung bình và độ lệch chuẩn có sự khác biệt. Điều này là do mẫu số liệu của học sinh lớp $11$A có một giá trị ngoại lệ.
	\end{luuy}
\end{ex}

%%%=============EX_5=============%%%
\begin{ex}
	Biểu đồ dưới đây mô tả kết quả điều tra về mức lương khởi điểm (đơn vị: triệu đồng) của một số công nhân ở hai khu vực A và B.
	\begin{center}
		\begin{tikzpicture}[>=stealth,line join=round,line cap=round,font=\footnotesize,scale=0.85,line width=1pt]
			\draw[->] (0,0)--(0,9)node[left]{(\text{Số công nhân})};
			\foreach \y in {1,2,3,4,5,6,7,8}
			\draw[shift={(0,\y)}] (0,0)--(-2pt,0) node[left]{\scriptsize ${\y}$};
			\path (6,10.5) node {
				$\begin{array}{c}
					\normalsize{\textbf{Mức lương khởi điểm của công nhân ở hai khu vực $A$ và $B$}}
				\end{array}$
			};
			
			\foreach \y in {1,2,3,4,5,6,7,8}{
				\draw[dashed,thin,line width=0.01pt] (0,\y)--(13,\y);
			}
			\draw[line cap=round,pattern=north east lines] (10,6)--(10,7)--(9,7)--(9,6)--(10,6) node[above right]{\text{Khu vực $ B $}};
			\draw[line cap=round,pattern=dots] (10,8)--(10,9)--(9,9)--(9,8)--(10,8) node[above right]{\text{Khu vực $ A $}};
			%% cột
			%		\draw[fill=cyan,draw=none] (0,0)--(0,0.8)--(1,0.8)node[midway,above]{$ 8 $}--(1,0)--cycle;
			\draw[line cap=round,pattern=dots] (1,0)--(1,4)--(2,4)node[midway,above]{$  $}--(2,0)--cycle;
			\draw[line cap=round,pattern=north east lines] (2,0)--(2,3)--(3,3)node[midway,above]{$  $}--(3,0)--cycle;
			\draw[line cap=round,pattern=dots] (3,0)--(3,5)--(4,5)node[midway,above]{$  $}--(4,0)--cycle;
			\draw[line cap=round,pattern=north east lines] (4,0)--(4,6)--(5,6)node[midway,above]{$  $}--(5,0)--cycle;
			\draw[line cap=round,pattern=dots] (5,0)--(5,5)--(6,5)node[midway,above]{$  $}--(6,0)--cycle;
			\draw[line cap=round,pattern=north east lines] (6,0)--(6,5)--(7,5)node[midway,above]{$  $}--(7,0)--cycle;
			\draw[line cap=round,pattern=dots] (7,0)--(7,4)--(8,4)node[midway,above]{$  $}--(8,0)--cycle;
			\draw[line cap=round,pattern=north east lines] (8,0)--(8,5)--(9,5)node[midway,above]{$  $}--(9,0)--cycle;
			\draw[line cap=round,pattern=dots] (9,0)--(9,2)--(10,2)node[midway,above]{$  $}--(10,0)--cycle;
			\draw[line cap=round,pattern=north east lines] (10,0)--(10,1)--(11,1)node[midway,above]{$  $}--(11,0)--cycle;
			%		%% miền
			
			\node [below] at (2,0){$ [5;6)$};
			\node [below] at (4,0){$ [6;7)$};
			\node [below] at (6,0){$ [7;8)$};
			\node [below] at (8,0){$ [8;9)$};
			\node [below] at (10,0){$ [9;10)$};
			\draw[->] (0,0)node [below left=-2pt]{$ O $}--(13,0)node[below]{(\text{Mức lương})};
		\end{tikzpicture}
	\end{center}
	Người ta lập được bảng tần số ghép nhóm cho mẫu số liệu như sau
	\begin{center}
		\begin{tabular}{|c|c|c|c|c|c|}
			\hline Mức lương &{$[5; 6)$} &{$[6; 7)$} &{$[7; 8)$} &{$[8; 9)$} &{$[9; 10)$} \\
			\hline Khu vực A & $ 4 $ & $ 5 $ & $ 5 $ & $ 4 $ & $ 2 $ \\
			\hline Khu vực B & $ 3 $ & $ 6 $ & $ 5 $ & $ 5 $ & $ 1 $ \\
			\hline
		\end{tabular}
	\end{center}
	\choiceTF
	{Xét mẫu số liệu của khu vực A ta có số trung bình của mẫu số liệu ghép nhóm là $6{,}25$}
	{\True Xét mẫu số liệu của khu vực A ta có độ lệch chuẩn của mẫu số liệu ghép nhóm là $\sqrt{1{,}5875}$}
	{Xét mẫu số liệu của khu vực B ta có phương sai của mẫu số liệu ghép nhóm là $1{,}3875$}
	{\True Nếu so sánh theo độ lệch chuẩn của mẫu số liệu ghép nhóm thì mức lương khởi điểm của công nhân khu vực B đồng đều hơn của công nhân khu vực A}
	\loigiai{
		\begin{itemchoice}
			\itemch Xét mẫu số liệu của khu vực A, ta có
			\begin{itemize}
				\item Cỡ mẫu là $n_A=4+5+5+4+2=20$.
				\item Số trung bình của mẫu số liệu ghép nhóm là
				\begin{align*}
					\overline{x}_A=\dfrac{4\cdot 5{,}5+5\cdot 6{,}5+5\cdot 7{,}5+4\cdot 8{,}5+2\cdot 9{,}5}{20}=7{,}25.
				\end{align*}
				\item Phương sai của mẫu số liệu ghép nhóm là
				\begin{align*}
					s_A^2=\dfrac{1}{20} \left(4\cdot 5{,}5^2+5\cdot 6{,}5^2+5\cdot 7{,}5^2+4\cdot 8{,}5^2+2\cdot 9{,}5^2\right)-(7{,}25)^2=1{,}5875.
				\end{align*}
				\item Độ lệch chuẩn của mẫu số liệu ghép nhóm là $s_A=\sqrt{1{,}5875}$.
			\end{itemize}
			\itemch Xét mẫu số liệu của khu vực $B$, ta có
			\begin{itemize}
				\item Cỡ mẫu là $n_B=3+6+5+5+1=20$.
				\item Số trung bình của mẫu số liệu ghép nhóm là
				\begin{align*}
					\overline{x}_B=\dfrac{3\cdot 5{,}5+6\cdot 6{,}5+5\cdot 7{,}5+5\cdot 8{,}5+1\cdot 9{,}5}{20}=7{,}25.
				\end{align*}
				\item Phương sai của mẫu số liệu ghép nhóm là
				\begin{align*}
					s_B^2=\dfrac{1}{20} \left(3\cdot 5{,}5^2+6\cdot 6{,}5^2+5\cdot 7{,}5^2+5\cdot 8{,}5^2+1\cdot 9{,}5^2\right)-(7{,}25)^2=1{,}2875.
				\end{align*}
				\item Độ lệch chuẩn của mẫu số liệu ghép nhóm là $s_B=\sqrt{1{,}2875}$.
			\end{itemize}
			Do $s_A > s_B$ nên nếu so sánh theo độ lệch chuẩn của mẫu số liệu ghép nhóm thì mức lương khởi điểm của công nhân khu vực B đồng đều hơn của công nhân khu vực A.
			\begin{luuy}
				Với các mẫu số liệu ghép nhóm có cùng số trung bình (hoặc xấp xỉ nhau), ta thường sử dụng phương sai và độ lệch chuẩn để so sánh mức độ phân tán của các mẫu số liệu đó.
			\end{luuy}
		\end{itemchoice}
	}
\end{ex}

\Closesolutionfile{ans}


\ind{PHẦN III.} \inden{Câu trả lời ngắn. Thí sinh ghi kết quả.}\\
\setcounter{ex}{0}
\Opensolutionfile{ans}[ans/2D1-Bai1-DS]%--Đặt tên 2D1-Bai1-DS
%%%=============EX_1=============%%%
\begin{ex}[Đề thi thử TN Sở Trà Vinh NH 24-25] Khảo sát thời gian tập thể dục trong ngày của một số học sinh lớp 12 thu được mẫu số liệu ghép nhóm sau
	\begin{center}
		\begin{tabular}{|c|c|c|c|c|c|}
			\hline
			Thời gian (phút) & $\left[0;10\right)$ & $\left[10;20\right)$ & $\left[20;30\right)$ & $\left[30;40\right)$ & $\left[40;50\right)$ \\
			\hline
			Số học sinh & $11$ & $10$ & $13$ & $9$ & $7$ \\
			\hline
		\end{tabular}
	\end{center}
	Độ lệch chuẩn của mẫu số liệu ghép nhóm trên bằng bao nhiêu? (làm tròn kết quả đến hàng phần mười)
	\shortans{13{,}4}
	\loigiai{
		Ta viết lại bảng ở đề bài ra như sau
		\begin{center}
			\begin{tabular}{|c|c|c|c|c|c|c|}
				\hline
				Thời gian (phút) & $\left[0;10\right)$ & $\left[10;20\right)$ & $\left[20;30\right)$ & $\left[30;40\right)$ & $\left[40;50\right]$ & \\
				\hline
				Giá trị đại diện & 5 & 15 & 25 & 35 & 45 & \\
				\hline
				Số học sinh & 11 & 10 & 13 & 9 & 7 & $n=50$ \\
				\hline
			\end{tabular}
		\end{center}
		Số trung bình cộng của mẫu số liệu ghép nhóm biểu thị số phút tập thể dục trong ngày của một số học sinh lớp $12$ là
		\[\overline{x}=\dfrac{5\cdot 11+15\cdot 10+25\cdot 13+35\cdot 9+45\cdot 7}{50}=23{,}2.\]
		Vậy phương sai của mẫu số liệu ghép nhóm biểu thị số phút tập thể dục trong ngày của một số học sinh lớp 12 là
		\[s^2=\dfrac{11\cdot \left(5-23{,}2\right)^2+10\cdot (15-23{,}2)^2+13\cdot \left(25-23{,}2\right)^2+9\cdot \left(35-23{,}2\right)^2+7\cdot \left(45-23{,}2\right)^2}{50}=178{,}76.\]
		Vậy độ lệch chuẩn của mẫu số liệu là $s=\sqrt{s^2}=\sqrt{178{,}76} \approx 13{,}4$.
	}
\end{ex}

%%%=============EX_2=============%%%
\begin{ex}[Sáng tác Strong đề thi thử TN NH 24-25] Trong một đợt khám sức khoẻ của 50 học sinh nam lớp $12$, người ta được kết quả như Bảng $1$. Độ lệch chuẩn của mẫu số liệu ghép nhóm cho ở Bảng 1 bằng bao nhiêu centimét?
	\begin{center}
		\begin{tabular}{|c|c|c|c|c|c|}
			\hline
			Nhóm & $\left[160;164\right)$ & $\left[164;168\right)$ & $\left[168;172\right)$ & $\left[172;176\right)$ & $\left[176;180\right)$ \\
			\hline
			Tần số & $3$ & $8$ & $18$ & $12$ & $9$ \\
			\hline
		\end{tabular}
	\end{center}
	\shortans{4{,}5}
	\loigiai{
		Số trung bình cộng của mẫu số liệu đó là
		\[\overline{x}=\dfrac{3\cdot 162+8\cdot 166+18\cdot 170+12\cdot 174+9\cdot 178}{50}=171{,}28\text{ (cm)}.\]
		Phương sai của mẫu số liệu là
		\begin{align*}
			s^2=&\dfrac{1}{50} \left[3\cdot (171{,}28-162)^2+8\cdot (171{,}28-166)^2+18\cdot (171{,}28-170)^2\right.\\
			&\left.+12\cdot (171{,}28-174)^2+9\cdot (171{,}28-178)^2\right]=20{,}1216.
		\end{align*}
		Độ lệch chuẩn của mẫu số liệu là $s=\sqrt{s^2}=\sqrt{20{,}1216} \approx 4{,}5\text{ (cm)}$.
	}
\end{ex}

%%%=============EX_3=============%%%
\begin{ex}
	\immini[thm]{Kết quả khảo sát thời gian sử dụng liên tục (đơn vị: giờ) từ lúc sạc đầy cho đến khi hết của pin một số máy vi tính cùng loại được mô tả bằng biểu đồ bên. Hãy xác định độ lệch chuẩn của thời gian sử dụng pin (kết quả được làm tròn đến hàng phần trăm).}
	{\begin{tikzpicture}[>=stealth,line join=round,line cap=round,font=\footnotesize,scale=0.85,line width=1pt]
			\draw[->] (0,0)--(0,9)node[left]{\text{Số máy}};
			\draw[](0,8.5)node[left]{\text{vi tính}};
			\foreach \y in {1,2,3,4,5,6,7,8}
			\draw[shift={(0,\y)}] (0,0)--(-2pt,0) node[left]{\scriptsize ${\y}$};
			\path (4.5,9.5) node {
				$\begin{array}{c}
					\normalsize{\textbf{Thời gian sử dụng pin của một số máy vi tính}}
				\end{array}$
			};
			\foreach \y in {1,2,3,4,5,6,7,8}{
				\draw[dashed,thin,line width=0.01pt] (0,\y)--(7,\y);
			}
			%% cột
			%		\draw[fill=cyan,draw=none] (0,0)--(0,0.8)--(1,0.8)node[midway,above]{$ 8 $}--(1,0)--cycle;
			\draw[line cap=round,pattern=dots] (1,0)--(1,2)--(2,2)node[midway,above]{$  $}--(2,0)--cycle;
			\draw[line cap=round,pattern=dots] (2,0)--(2,4)--(3,4)node[midway,above]{$  $}--(3,0)--cycle;
			\draw[line cap=round,pattern=dots] (3,0)--(3,7)--(4,7)node[midway,above]{$  $}--(4,0)--cycle;
			\draw[line cap=round,pattern=dots] (4,0)--(4,5)--(5,5)node[midway,above]{$  $}--(5,0)--cycle;
			%		%% miền
			\node [below] at (1,0){$ 7{,}2$};
			\node [below] at (2,0){$ 7{,}4$};
			\node [below] at (3,0){$ 7{,}6$};
			\node [below] at (4,0){$ 7{,}8$};
			\node [below] at (5,0){$ 8{,}0$};
			\draw[->] (0,0)node [below left=-2pt]{$ O $}--(7,0)node[below]{(\text{thời gian (giờ)})};
			%		\draw[pattern=horizontal lines,bar width=8mm]plot coordinates{(1,10)};
	\end{tikzpicture}}
	\shortans{0,19}
	\loigiai{
		\begin{center}
			\begin{tabular}{|c|c|c|c|c|}
				\hline
				Giá trị đại diện & $7{,}3$ & $7{,}5$ & $7{,}7$ & $7{,}9$ \\
				\hline
				Số máy & 2 & 4 & 7 & 5 \\
				\hline
			\end{tabular}
		\end{center}
		Cỡ mẫu $n=18$.\\
		Số trung bình $\overline{x}=\dfrac{2\cdot 7{,}3+4\cdot 7{,}5+7\cdot 7{,}7+5\cdot 7{,}9}{18} \approx 7{,}67$.\\
		Phương sai $s^2=\dfrac{2\cdot 7{,}3^2+4\cdot 7{,}5^2+7\cdot 7{,}7^2+5\cdot 7{,}9^2}{18}-7{,}67^2\approx 0{,}04$.\\
		Độ lệch chuẩn $s=\sqrt{0{,}04} \approx 0{,}19$.
	}
\end{ex}

%%%=============EX_4=============%%%
\begin{ex}
	Người ta ghi lại tiền lãi (đơn vị: triệu đồng) của một số nhà đầu tư (với số tiền đầu tư như nhau), khi đầu tư vào hai lĩnh vực A, B cho kết quả như sau
	\begin{center}
		\begin{tabular}{|c|c|c|c|c|c|}
			\hline
			Tiền lãi & $[5;10)$ & $[10;15)$ & $[15;20)$ & $[20;25)$ & $[25;30)$ \\
			\hline
			Số nhà đầu tư vào lĩnh vực $A$ & $2$ & $5$ & $8$ & $6$ & $4$ \\
			\hline
			Số nhà đầu tư vào lĩnh vực $B$ & $8$ & $4$ & $2$ & $5$ & $6$ \\
			\hline
		\end{tabular}
	\end{center}
	Tính hiệu phương sai $s_B^2-s_A^2$ cho các mẫu số liệu về tiền lãi của các nhà đầu tư ở hai lĩnh vực này (làm tròn đến hàng phần chục).
	\shortans{47,7}
	\loigiai{
		Ta có mẫu số liệu ghép nhóm với giá trị đại diện là
		\begin{center}
			\begin{tabular}{|c|c|c|c|c|c|}
				\hline
				Tiền lãi & $[5;10)$ & $[10;15)$ & $[15;20)$ & $[20;25)$ & $[25;30)$ \\
				\hline
				Giá trị đại diện & $7{,}5$ & $12{,}5$ & $17{,}5$ & $22{,}5$ & $27{,}5$ \\
				\hline
				Số nhà đầu tư vào lĩnh vực $A$ & $2$ & $5$ & $8$ & $6$ & $4$ \\
				\hline
				Số nhà đầu tư vào lĩnh vực $B$ & $8$ & $4$ & $2$ & $5$ & $6$ \\
				\hline
			\end{tabular}
		\end{center}
		\begin{itemize}
			\item Tiền lãi trung bình khi đầu tư vào lĩnh vực $A$ là
			\begin{align*}
				\overline{x}_A=\dfrac{7{,}5\cdot 2+12{,}5\cdot 5+17{,}5\cdot 8+22{,}5\cdot 6+27{,}5\cdot 4}{2+5+8+6+4}=18{,}5.
			\end{align*}
			\item Tiền lãi trung bình khi đầu tư vào lĩnh vực $B$ là
			\begin{align*}
				\overline{x}_B=\dfrac{7{,}5.8+12{,}5.4+17{,}5.2+22{,}5.5+27{,}5.6}{8+4+2+5+6}=16{,}9.
			\end{align*}
			\item Phương sai của mẫu số liệu về tiền lãi khi đầu tư vào lĩnh vực $A$:
			\begin{align*}
				s_A^2=\dfrac{1}{25} \left(7{,}5^2\cdot 2+12{,}5^2\cdot 5+17{,}5^2\cdot 8+22{,}5^2\cdot 6+27{,}5^2\cdot 4\right)-18{,}5^2=34.
			\end{align*}
			\item Phương sai của mẫu số liệu về tiền lãi khi đầu tư vào lĩnh vực $B$:
			\begin{align*}
				s_B^2=\dfrac{1}{25} \left(7{,}5^2\cdot 8+12{,}5^2\cdot 4+17{,}5^2\cdot 2+22{,}5^2\cdot 5+27{,}5^2\cdot 6\right)-16{,}9^2=64{,}64.
			\end{align*}
			Do đó $s_B^2-s_A^2\approx 64{,}64-16{,}9=47{,}74$.
		\end{itemize}
	}
\end{ex}

%%%=============EX_5=============%%%
\begin{ex}
	\immini[thm]{Cho biểu đồ dưới đây thống kê cự li ném tạ của vận động viên A.
		Giả sử số lần ném tạ ở các cự li $[19; 19{,}5)$; $[19; 19{,}5)$; $[20; 20{,}5)$ là không đổi.
		Hỏi số lần ném tạ đạt được ở cự li $[20{,}5; 21)$ nhỏ nhất bằng bao nhiêu để độ lệch chuẩn lớn hơn độ lệch chuẩn ở bảng $1$?}
	{	\begin{tikzpicture}[>=stealth,line join=round,line cap=round,font=\footnotesize,scale=0.85,line width=1pt,yscale=.1,xscale=.7]
			\draw[->] (0,0)--(0,50)node[above right]{\text{Số lần đạt được}};
			\foreach \y in {10,20,30,40,50}
			\draw[shift={(0,\y)}] (0,0)--(-2pt,0) node[left]{\scriptsize ${\y}$};
			\foreach \y in {10,20,30,40,50}{
				\draw[dashed,thin,line width=0.01pt] (0,\y-.02)--(8,\y-.02);
			}
			\draw[line cap=round,pattern=dots] (1,0)--(1,13)--(2,13)node[midway,above]{$13$}--(2,0)--cycle;
			\draw[line cap=round,pattern=dots] (2,0)--(2,45)--(3,45)node[midway,above]{$45$}--(3,0)--cycle;
			\draw[line cap=round,pattern=dots] (3,0)--(3,24)--(4,24)node[midway,above]{$24$}--(4,0)--cycle;
			\draw[line cap=round,pattern=dots] (4,0)--(4,12)--(5,12)node[midway,above]{$12$}--(5,0)--cycle;
			%		%% miền
			\node [below] at (1,0){$ 19$};
			\node [below] at (2,0){$ 19{,}5$};
			\node [below] at (3,0){$ 20$};
			\node [below] at (4,0){$ 20{,}5$};
			\node [below] at (5,0){$21$};
			\draw[->] (0,0)node [below left=-2pt]{$ O $}--(8,0)node[below]{\text{Cự li (m)}};
	\end{tikzpicture}}
	\shortans{13}
	\loigiai{
		Từ biểu đồ, ta lập được bảng tần số ghép nhóm và tính được giá trị đại diện của mỗi nhóm như sau
		\begin{center}
			\begin{tabular}{|c|c|c|c|c|}
				\hline Cự li (m) & $\left[19;19{,}5\right)$ & $\left[19{,}5;20\right)$ & $\left[20;20{,}5\right)$ & $\left[20{,}5;21\right)$ \\ 
				\hline Giá trị đại diện & $19{,}25$ & $19{,}75$ & $20{,}25$ & $20{,}75$ \\
				\hline Tần số & $13$ & $45$ & $24$ &$12$ \\
				\hline
			\end{tabular}
		\end{center} 
		Đặt số lần vận động viên ném được ở cự li $\left[20{,}5;21\right)$ (m) là $x$. Ta có bảng sau
		\begin{center}
			\begin{tabular}{|c|c|c|c|c|} 
				\hline Cự li (m) & $\left[19;19{,}5\right)$ & $\left[19{,}5;20\right)$ & $\left[20;20{,}5\right)$ & $\left[20{,}5;21\right)$ \\
				\hline Giá trị đại diện & $19{,}25$ & $19{,}75$ & $20{,}25$ & $20{,}75$ \\
				\hline Tần số & $13$ & $45$ & $24$ & $x$ \\
				\hline
			\end{tabular}
		\end{center}
		\begin{itemize}
			\item Cỡ mẫu là $n=13+45+24+x=82+x$.
			\item Số trung bình của mẫu số liệu ghép nhóm là
			\begin{align*}
				\overline{x}=\dfrac{13\cdot 19{,}25+45\cdot 19{,}75+24\cdot 20{,}25+x\cdot 20{,}75}{82+x}=\dfrac{1625+x\cdot 20{,}75}{82+x}.
			\end{align*}
			\item Phương sai của mẫu số liệu ghép nhóm là
			\begin{align*}
				s^2&=\dfrac{1}{82+x} \left[13\cdot \left(19{,}25\right)^2+45\cdot \left(19{,}75\right)^2+24\cdot \left(20{,}25\right)^2+x\cdot \left(20{,}75\right)^2\right]-\left(\dfrac{1\,625+x\cdot 20{,}75}{82+x} \right)^2\\
				&=\left(\dfrac{515\,386+6\,889x}{16(82+x)} \right)-\left(\dfrac{1\,625+x\cdot 20{,}75}{82+x} \right)^2
			\end{align*}
			\item Độ lệch chuẩn của mẫu số liệu ghép nhóm là
			\begin{align*}
				s=f(x)=\sqrt{s^2}=\sqrt{\left(\dfrac{515\,386+6\,889x}{16(82+x)} \right)-\left(\dfrac{1\,625+x\cdot 20{,}75}{82+x} \right)^2}.
			\end{align*}
			Để độ lệch chuẩn lớn hơn độ lệch chuẩn của mẫu số liệu ban đầu
			thì $f(x)-f(12) > 0$\quad$(*)$.\\
			Sử dụng chức năng bảng giá trị của máy tính ta tìm được giá trị nguyên nhỏ nhất thoã mãn yêu cầu $(*)$ là $13$.
		\end{itemize}
	}
\end{ex}

\Closesolutionfile{ans}


\ind{PHẦN IV.} \inden{Tự luận.}\\
\setcounter{ex}{0}
%%%=============EX_1=============%%%
\begin{ex}%[2D4H2-2]
	Hãy tính phương sai và độ lệch chuẩn của mẫu số liệu ghép nhóm sau
	\begin{center}
		\begin{tabular}{|c|c|c|c|c|c|}
			\hline Chiều cao (cm) &{$[160;164)$} &{$[164;168)$} &{$[168;172)$} &{$[172;176)$} &{$[176;180)$} \\
			\hline Số học sinh & $ 3 $ & $ 5 $ & $ 8 $ & $ 4 $ & $ 1 $ \\
			\hline
		\end{tabular}
	\end{center}
	\loigiai{
		Ta có bảng sau
		\begin{center}
			\begin{tabular}{|c|c|c|c|c|c|}
				\hline Chiều cao (cm) &{$[160;164)$} &{$[164;168)$} &{$[168;172)$} &{$[172;176)$} &{$[176;180)$} \\
				\hline Giá trị đại diện & $ 162 $ & $ 166 $ & $ 170 $ & $ 174 $ & $ 178 $ \\
				\hline Số học sinh & $ 3 $ & $ 5 $ & $ 8 $ & $ 4 $ & $ 1 $ \\
				\hline
			\end{tabular}
		\end{center}
		Ta có cỡ mẫu $ n=21 $.\\
		Số trung bình của mẫu số liệu ghép nhóm là 
		$$ \overline{x}=\dfrac{3\cdot162+5\cdot166+8\cdot170+4\cdot174+1\cdot178}{21}=\dfrac{3550}{21}. $$
		Phương sai của mẫu số liệu ghép nhóm là  $$ s^2=\dfrac{1}{21}(3\cdot162^2+5\cdot166^2+8\cdot170^2+4\cdot174^2+1\cdot178^2)-\left(\dfrac{3550}{21} \right)^2\approx18{,}14.  $$
		Độ lệch chuẩn của mẫu số liệu ghép nhóm là $ s\approx\sqrt{18{,}14} $.
	}
\end{ex}

%%%=============EX_2=============%%%
\begin{ex}%[2D4H2-2]
	Mai và Ngọc cùng sử dụng vòng đeo tay thông minh để ghi lại số bước chân hai bạn đi mỗi ngày trong một tháng. Kết quả được ghi lại ở bảng sau
	\begin{center}
		\begin{tabular}{|c|c|c|c|c|c|}
			\hline Số bước (đơn vị: nghìn) &{$[3;5)$} &{$[5;7)$} &{$[7;9)$} &{$[9;11)$} &{$[11;13)$} \\
			\hline Mai & $ 6 $ & $ 7 $ & $ 6 $ & $ 6 $ & $ 5 $ \\
			\hline Ngọc & $ 2 $ & $ 5 $ & $ 13 $ & $ 8 $ & $ 2 $ \\
			\hline
		\end{tabular}
	\end{center}
	\begin{enumerate}
		\item Hãy tính số trung bình và độ lệch chuẩn của mẫu số liệu ghép nhóm trên.
		\item Nếu so sánh theo độ lệch chuẩn thì bạn nào có số lượng bước chân đi mỗi ngày đều đặn hơn?
	\end{enumerate}
	\loigiai{
		\begin{enumerate}
			\item Ta có bảng sau
			\begin{center}
				\begin{tabular}{|c|c|c|c|c|c|}
					\hline Số bước (đơn vị: nghìn) &{$[3;5)$} &{$[5;7)$} &{$[7;9)$} &{$[9;11)$} &{$[11;13)$} \\
					\hline Số bước đại diện & $ 4 $ & $ 6 $ & $ 8 $ & $ 10 $ & $ 12 $ \\
					\hline Mai & $ 6 $ & $ 7 $ & $ 6 $ & $ 6 $ & $ 5 $ \\
					\hline Ngọc & $ 2 $ & $ 5 $ & $ 13 $ & $ 8 $ & $ 2 $ \\
					\hline
				\end{tabular}
			\end{center}
			\begin{itemize}
				\item Xét mẫu số liệu của Mai.\\
				Cỡ mẫu là $ n_1=6+7+6+6+5=30 $.\\
				Số trung bình của mẫu số liệu ghép nhóm là\\
				$$ \overline{x}_1=\dfrac{6\cdot 4+7\cdot6+6\cdot8+6\cdot10+5\cdot12}{30}=7{,}8. $$
				Phương sai của mẫu số liệu ghép nhóm là
				$$ s_1^2=\dfrac{1}{30}\left(6\cdot 4^2+7\cdot6^2+6\cdot8^2+6\cdot10^2+5\cdot12^2 \right) -7{,}8^2=7{,}56. $$
				Độ lệch chuẩn của mẫu số liệu ghép nhóm là $ s_1=\sqrt{7{,}56}\approx 2{,}75.$
				
				\item Xét mẫu số liệu của Ngọc.\\
				Cỡ mẫu là $ n_2=2+5+13+8+2=30 $.\\
				Số trung bình của mẫu số liệu ghép nhóm là
				$$ \overline{x}_2=\dfrac{2\cdot 4+5\cdot6+13\cdot8+8\cdot10+2\cdot12}{30}=8{,}2. $$
				Phương sai của mẫu số liệu ghép nhóm là
				$$ s_2^2=\dfrac{1}{30}\left(2\cdot 4^2+5\cdot6^2+13\cdot8^2+8\cdot10^2+2\cdot12^2 \right) -8{,}2^2=\dfrac{287}{75}. $$
				Độ lệch chuẩn của mẫu số liệu ghép nhóm là $ s_2=\sqrt{\dfrac{287}{75}}\approx 1{,}96. $
			\end{itemize}
			\item Do $ s_1>s_2 $ nên nếu so sánh theo độ lệch chuẩn thì Ngọc có số lượng bước chân đi mỗi ngày đều đặn hơn Mai.
		\end{enumerate}
	}
\end{ex}	

%%%=============EX_3=============%%%
\begin{ex}%[2D4H2-2]
	Bảng dưới đây thống kê cự li ném tạ của một vận động viên.
	\begin{center}
		\begin{tabular}{|c|c|c|c|c|c|}
			\hline Cự li (m) &{$[19;19{,}5)$} &{$[19{,}5;20)$} &{$[20;20{,}5)$} &{$[20{,}5;21)$} &{$[21;21{,}5)$} \\
			\hline Tần số & $ 13 $ & $ 45 $ & $ 24 $ & $ 12 $ & $ 6 $ \\
			\hline
		\end{tabular}
	\end{center}
	Hãy tính phương sai và độ lệch chuẩn của mẫu số liệu ghép nhóm trên.
	\loigiai{
		Ta có bảng sau	
		\begin{center}
			\begin{tabular}{|c|c|c|c|c|c|}
				\hline Cự li (m) &{$[19;19{,}5)$} &{$[19{,}5;20)$} &{$[20;20{,}5)$} &{$[20{,}5;21)$} &{$[21;21{,}5)$} \\
				\hline Cự li đại diện & $ 19{,}25 $ & $ 19{,}75 $ & $ 20{,}25 $ & $ 20{,}75 $ & $ 21{,}25 $ \\
				\hline Tần số & $ 13 $ & $ 45 $ & $ 24 $ & $ 12 $ & $ 6 $ \\
				\hline
			\end{tabular}
		\end{center}
		Cỡ mẫu là $ n=13+45+24+12+6=100 $.\\
		Số trung bình của mẫu số liệu ghép nhóm là
		$$ \overline{x}=\dfrac{13\cdot19{,}25+45\cdot 19{,}75+24\cdot 20{,}25+12\cdot 20{,}75+6\cdot 21{,}25}{100}=20{,}015. $$
		Phương sai của mẫu số liệu ghép nhóm là
		$$ s^2=\dfrac{1}{100}\left(13\cdot19{,}25^2+45\cdot 19{,}75^2+24\cdot 20{,}25^2+12\cdot 20{,}75^2+6\cdot 21{,}25^2 \right) -20{,}015^2\approx 0{,}277275. $$
		Độ lệch chuẩn của mẫu số liệu ghép nhóm là $ s=\sqrt{0{,}277275}\approx 0{,}53 $.
	}
\end{ex}	

%%%=============EX_4=============%%%
\begin{ex}%[2D4H2-2]
	\immini{Kết quả khảo sát thời gian sử dụng liên tục
		(đơn vị: giờ) từ lúc sạc đầy cho đến khi hết
		của pin một số máy vi tính cùng loại được
		mô tả bằng biểu đồ bên.
		\begin{enumerate}
			\item Hãy cho biết có bao nhiêu máy vi tính có
			thời gian sử dụng pin từ $7{,}2$ đến dưới $ 7{,}4 $ giờ?
			\item Hãy xác định số trung bình và độ lệch chuẩn
			của thời gian sử dụng pin.
	\end{enumerate}}{	\begin{tikzpicture}[>=stealth,line join=round,line cap=round,font=\footnotesize,scale=0.85,line width=1pt]
			\draw[->] (0,0)--(0,9)node[left]{\text{Số máy}};
			\draw[](0,8.5)node[left]{\text{vi tính}};
			\foreach \y in {1,2,3,4,5,6,7,8}
			\draw[shift={(0,\y)}] (0,0)--(-2pt,0) node[left]{\scriptsize ${\y}$};
			\path (4.5,9.5) node {
				$\begin{array}{c}
					\normalsize{\textbf{Thời gian sử dụng pin của một số máy vi tính}}
				\end{array}$
			};
			\foreach \y in {1,2,3,4,5,6,7,8}{
				\draw[dashed,thin,line width=0.01pt] (0,\y)--(7,\y);
			}
			%% cột
			%		\draw[fill=cyan,draw=none] (0,0)--(0,0.8)--(1,0.8)node[midway,above]{$ 8 $}--(1,0)--cycle;
			\draw[line cap=round,pattern=dots] (1,0)--(1,2)--(2,2)node[midway,above]{$  $}--(2,0)--cycle;
			\draw[line cap=round,pattern=dots] (2,0)--(2,4)--(3,4)node[midway,above]{$  $}--(3,0)--cycle;
			\draw[line cap=round,pattern=dots] (3,0)--(3,7)--(4,7)node[midway,above]{$  $}--(4,0)--cycle;
			\draw[line cap=round,pattern=dots] (4,0)--(4,5)--(5,5)node[midway,above]{$  $}--(5,0)--cycle;
			%		%% miền
			\node [below] at (1,0){$ 7{,}2$};
			\node [below] at (2,0){$ 7{,}4$};
			\node [below] at (3,0){$ 7{,}6$};
			\node [below] at (4,0){$ 7{,}8$};
			\node [below] at (5,0){$ 8{,}0$};
			\draw[->] (0,0)node [below left=-2pt]{$ O $}--(7,0)node[below]{(\text{thời gian (giờ)})};
			%		\draw[pattern=horizontal lines,bar width=8mm]plot coordinates{(1,10)};
	\end{tikzpicture}}
	\loigiai{
		\begin{enumerate}
			\item Có $ 2 $ máy vi tính có
			thời gian sử dụng pin từ $7{,}2$ đến dưới $ 7{,}4 $ giờ.
			\item Ta có bảng sau
			\begin{center}
				\begin{tabular}{|c|c|c|c|c|}
					\hline Thời gian (giờ)  &{$[7{,}2;7{,}4)$} &{$[7{,}4;7{,}6)$} &{$[7{,}6;7{,}8)$} &{$[7{,}8;8{,}0)$}  \\
					\hline Giá trị đại diện & $ 7{,}3 $ & $ 7{,}5 $ & $ 7{,}7 $ & $ 7{,}9 $  \\
					\hline Số máy tính & $ 2 $ & $ 4 $ & $ 7 $ & $ 5 $  \\
					\hline
				\end{tabular}
			\end{center}
			Cỡ mẫu là $ n=2+4+7+5=18 $.\\
			Số trung bình của mẫu số liệu ghép nhóm là
			$$ \overline{x}=\dfrac{2\cdot7{,}3+4\cdot7{,}5+7\cdot7{,}7+5\cdot7{,}9}{18}=\dfrac{23}{3}. $$
			Phương sai của mẫu số liệu ghép nhóm là
			$$ s^2=\dfrac{1}{18}\left(2\cdot7{,}3^2+4\cdot7{,}5^2+7\cdot7{,}7^2+5\cdot7{,}9^2 \right)-\left(\dfrac{23}{3} \right)^2 \approx 0{,}037.  $$
			Độ lệch chuẩn của mẫu số liệu ghép nhóm là $ s\approx\sqrt{0{,}037} $.
		\end{enumerate}
	}
\end{ex}

%%%=============EX_5=============%%%
\begin{ex}%[2D4H2-3]
	Tốc độ của $20$ xe hơi khi đi qua một trạm kiểm tra tốc độ (đơn vị: km/h) được thống kê lại như sau
	\begin{center}
		\begin{tabular}{|c|c|c|c|c|c|c|c|c|c|c|c|c|c|c|c|c|c|c|c|}
			\hline 
			$ 42 $	& $ 43{,}4 $ & $ 43{,}4 $ & $ 46{,}5 $ & $ 46{,}7 $ & $ 46{,}8 $ & $ 47{,}5 $ & $ 47{,}7 $ & $ 48{,}1 $ & $ 48{,}4 $  \\ 
			\hline 
			$ 50{,}8 $	& $ 52{,}1 $ & $ 52{,}7 $ & $ 53{,}9 $ & $ 54{,}8 $ & $ 55{,}6 $ & $ 57{,}5 $ & $ 59{,}6 $ & $ 60{,}3 $ & $ 61{,}1 $ \\ 
			\hline 
		\end{tabular} 
	\end{center}
	\begin{enumerate}
		\item Hãy tính khoảng biến thiên, khoảng tứ phân vị và độ lệch chuẩn của mẫu số liệu trên.
		\item Hãy lập bảng tần số ghép nhóm với nhóm đầu tiên là $[42; 46)$ và độ dài mỗi nhóm bằng $4$.
		\item Hãy tính khoảng biến thiên, khoảng tứ phân vị và độ lệch chuẩn của mẫu số liệu ghép nhóm.
	\end{enumerate}
	\loigiai{
		\begin{enumerate}
			\item \begin{itemize}
				\item Khoảng biến thiên $ R_1=61{,}1-42=19{,}1 $.
				\item Khoảng tứ phân vị\\
				Ta có $ Q_1=\dfrac{46{,}7+46{,}8}{2}=46{,}75 $; $ Q_3=\dfrac{54{,}8+55{,}6}{2}=55{,}2 $.\\
				Từ đó suy ra khoảng tứ phân vị $ \Delta_Q=55{,}2-46{,}75=8{,}45 $.
				\item Cỡ mẫu $ n=20 $.\\
				Số trung bình của mẫu số liệu trên là $\overline{x}_1=\dfrac{42+43{,}4+\cdots+61{,}1}{20}=50{,}945$.\\
				Phương sai của mẫu số liệu trên là $s_1^2=\dfrac{1}{20}\left( 42^2+43{,}4^2+\ldots+61{,}1^2   \right)-50{,}945^2\approx 32{,}2$.\\
				Độ lệch chuẩn của mẫu số liệu trên là $ s_1\approx \sqrt{32{,}2} $.
			\end{itemize}
			\item Ta có bảng tần số ghép nhóm là
			\begin{center}
				\begin{tabular}{|c|c|c|c|c|c|}
					\hline Tốc độ (km/h) &{$[42;46)$} &{$[46;50)$} &{$[50;54)$} &{$[54;58)$} &{$[58;62)$} \\
					\hline Số xe & $ 3 $ & $ 7 $ & $ 4 $ & $ 3 $ & $ 3 $ \\
					\hline
				\end{tabular}
			\end{center}
			\item
			Ta có bảng sau
			\begin{center}
				\begin{tabular}{|c|c|c|c|c|c|}
					\hline Tốc độ (km/h) &{$[42;46)$} &{$[46;50)$} &{$[50;54)$} &{$[54;58)$} &{$[58;62)$} \\
					\hline Tốc độ đại diện & $ 44 $ & $ 48 $ & $ 52 $ & $ 56 $ & $ 60 $ \\
					\hline Số xe & $ 3 $ & $ 7 $ & $ 4 $ & $ 3 $ & $ 3 $ \\
					\hline
				\end{tabular}
			\end{center}
			\begin{itemize}
				\item Ta có khoảng biến thiên $ R_2=62-42=20 $.\\
				\item \begin{itemize}
					\item Ta có nhóm chứa trung vị là nhóm  $[46;50)$ từ đó ta có 
					$$ Q'_2=46+\dfrac{\dfrac{20}{2}-3}{7}(50-46)=50. $$
					\item Ta có nhóm chứa tứ phân vị thứ nhất là nhóm  $[46;50)$ từ đó ta có 
					$$ Q'_1=46+\dfrac{\dfrac{20}{4}-3}{7}(50-46)\approx47{,}14. $$
					\item Ta có nhóm chứa tứ phân vị thứ ba là nhóm  $[54;58)$ từ đó ta có 
					$$ Q'_3=54+\dfrac{\dfrac{3\cdot20}{4}-14}{7}(58-54)\approx54{,}57. $$
				\end{itemize}
				Từ đó suy ra khoảng tứ phân vị $ \Delta'_Q\approx 54{,}57-47{,}14\approx7{,}43 $.
				\item 
				Số trung bình của mẫu số liệu ghép nhóm là
				$$ \overline{x}_2=\dfrac{3\cdot 44+7\cdot48+4\cdot52+3\cdot56+3\cdot60}{20}=51{,}2. $$
				Phương sai của mẫu số liệu ghép nhóm là
				$$ s_2^2=\dfrac{1}{20}\left(3\cdot 44^2+7\cdot48^2+4\cdot52^2+3\cdot56^2+3\cdot60^2 \right) -51{,}2^2=26{,}56. $$
				Độ lệch chuẩn của mẫu số liệu ghép nhóm là $ s_2=\sqrt{26{,}56}. $
			\end{itemize}
			
		\end{enumerate}
	}
\end{ex}

%%%=============EX_6=============%%%
\begin{ex}%[2D4H2-2]
	Một giống cây xoan đào được trồng tại hai địa điểm A và B. Người ta thống kê đường kính thân của một số cây xoan đào $5$ năm tuổi ở bảng sau
	\begin{center}
		\begin{tabular}{|c|c|c|c|c|c|}
			\hline Đường kính (cm)  &{$[30;32)$} &{$[32;34)$} &{$[34;36)$} &{$[36;38)$} &{$[38;40)$} \\
			\hline Số cây trồng ở địa điểm A  & $ 25 $ & $ 38 $ & $ 20 $ & $ 10 $ & $ 7 $ \\
			\hline Số cây trồng ở địa điểm B  & $ 22 $ & $ 27 $ & $ 19 $ & $ 18 $ & $ 14 $ \\
			\hline
		\end{tabular}
	\end{center}
	\begin{enumerate}
		\item Hãy so sánh đường kính trung bình của thân cây xoan đào trồng tại địa điểm A và địa điểm B.
		\item Nếu so sánh theo độ lệch chuẩn thì cây trồng tại địa điểm nào có đường kính đồng đều hơn?
	\end{enumerate}
	\loigiai{
		\begin{enumerate}
			\item Ta có bảng sau
			\begin{center}
				\begin{tabular}{|c|c|c|c|c|c|}
					\hline Đường kính (cm)  &{$[30;32)$} &{$[32;34)$} &{$[34;36)$} &{$[36;38)$} &{$[38;40)$} \\
					\hline Đường kính đại diện  & $ 31 $ & $ 33 $ & $ 35 $ & $ 37 $ & $ 39 $ \\
					\hline Số cây trồng ở địa điểm $ A $  & $ 25 $ & $ 38 $ & $ 20 $ & $ 10 $ & $ 7 $ \\
					\hline Số cây trồng ở địa điểm $ B $  & $ 22 $ & $ 27 $ & $ 19 $ & $ 18 $ & $ 14 $ \\
					\hline
				\end{tabular}
			\end{center}
			\begin{itemize} 
				\item Cỡ mẫu $ n=100 $.
				\item Đường kính trung bình của cây xoan đào trồng ở địa điểm A 
				\begin{align*}
					\overline{x}_A=\dfrac{25\cdot31+38\cdot33+20\cdot35+10\cdot37+7\cdot39}{100}=33{,}72.
				\end{align*}
				\item Đường kính trung bình của cây xoan đào trồng ở địa điểm $B$ 
				\begin{align*}
					\overline{x}_B=\dfrac{22\cdot31+27\cdot33+19\cdot35+18\cdot37+14\cdot39}{100}=34{,}5.
				\end{align*}
			\end{itemize}
			Ta có $\overline{x}_B>\overline{x}_A$ nên đường kính trung bình cây xoan đào ở địa điểm B lớn hơn ở địa điểm A.
			\item \begin{itemize}
				\item Phương sai của mẫu số liệu cây xoan đào trồng ở địa điểm A
				\begin{align*}
					s_A^2=\dfrac{1}{100}\left(25\cdot31^2+38\cdot33^2+20\cdot35^2+10\cdot37^2+7\cdot39^2 \right)- 33{,}72^2=5{,}4016.
				\end{align*}
				Suy ra độ lệch chuẩn của mẫu số liệu cây xoan đào trồng ở địa điểm A là $ s_A=\sqrt{5{,}4016} $.
				\item Phương sai của mẫu số liệu cây xoan đào trồng ở địa điểm B
				\begin{align*}
					s_B^2=\dfrac{1}{100}\left(22\cdot31^2+27\cdot33^2+19\cdot35^2+18\cdot37^2+14\cdot39^2 \right)- 34{,}5^2=7{,}31.
				\end{align*}
				Suy ra độ lệch chuẩn của mẫu số liệu cây xoan đào trồng ở địa điểm B là $ s_B=\sqrt{7{,}31} $.
			\end{itemize}
			Vì $ s_B>s_A $ nên nếu so sánh theo độ lệch chuẩn thì cây trồng tại địa điểm A đồng đều hơn ở địa điểm B.
		\end{enumerate}
	}
\end{ex}

%%%=============EX_7=============%%%
\begin{ex}%[2D4H2-3]
	Một bác tài xế thống kê lại độ dài quãng đường (đơn vị km) bác đã lái xe mỗi ngày trong một tháng ở bảng sau
	\begin{center}
		\begin{tabular}{|c|c|c|c|c|c|}
			\hline
			Độ dài quãng đường (km) &$[50;100)$  & $[100;150)$ & $[150;200)$ & $[200;250)$ & $[250;300)$ \\
			\hline
			Số ngày	& $5$ & $10$ & $9$ & $4$ & $2$ \\
			\hline
		\end{tabular}
	\end{center}
	Hãy xác định khoảng biến thiên, khoảng tứ phân vị và độ lệch chuẩn của mẫu số liệu trên.
	\loigiai 
	{
		\begin{itemize}
			\item Khoảng biến thiên của mẫu số liệu là $300-50=250$.
			\item Khoảng tứ phân vị
			\begin{itemize}
				\item Cỡ mẫu $n=30$.\\
				Gọi $x_1$; $x_2$;$\ldots$; $x_{30}$ là mẫu số liệu gốc gồm thời gian của $30$ ngày mà bác tài xế đã lái xe mỗi ngày được sắp xếp theo thứ tự không giảm.\\
				Ta có 
				\begin{itemize}
					\item $x_1$, $\ldots$, $x_5\in [50;100)$; 
					\item $x_6$; $\ldots$; $x_{15}\in [100;150)$; 
					\item $x_{16}$; $\ldots$; $x_{24}\in [150;200)$; 
					\item $x_{25}$; $\ldots$; $x_{28}\in [200;250)$; 
					\item $x_{29}$; $x_{30}\in [250;300)$.
				\end{itemize}
				\item Tứ phân vị thứ nhất của mẫu số liệu gốc là $x_8\in [100;150)$. \\
				Do đó, tứ phân vị thứ nhất của mẫu số liệu ghép nhóm là 
				$$Q_1=100+\dfrac{\dfrac{30}{4}-5}{10}(150-100)=112{,}5.$$
				\item Tứ phân vị thứ ba của mẫu số liệu gốc là $x_{23}\in [150;200)$. \\
				Do đó, tứ phân vị thứ ba của mẫu số liệu ghép nhóm là 
				$$Q_3=150+\dfrac{\dfrac{3\cdot 30}{4}-(5+10)}{9}(200-150)=\dfrac{575}{3}.$$
				Vậy khoảng tứ phân vị của mẫu số liệu ghép nhóm là 
				$$\Delta_Q=\dfrac{575}{3}-112{,}5=\dfrac{475}{6}.$$
			\end{itemize}
			\item Xét mẫu số liệu ghép nhóm cho bởi bảng sau
			\begin{center}
				\begin{tabular}{|c|c|c|c|c|c|}
					\hline
					Nhóm &$[50;100)$  & $[100;150)$ & $[150;200)$ & $[200;250)$ & $[250;300)$ \\
					\hline
					Giá trị đại diện & $75$ & $125$ & $175$ & $225$ & $275$ \\
					\hline
					Tần số	& $5$ & $10$ & $9$ & $4$ & $2$ \\
					\hline
				\end{tabular}
			\end{center}
			\begin{itemize}
				\item Số trung bình của mẫu số liệu là 
				\begin{align*}
					\overline{x}=\dfrac{1}{30}\cdot (75\cdot 5+125\cdot 10+175\cdot 9+225\cdot 4+275\cdot 2)=155.
				\end{align*}
				\item Phương sai của mẫu số liệu ghép nhóm là 
				\begin{align*}
					s^2=\dfrac{1}{30}\left(5\cdot 75^2+10\cdot 125^2+9\cdot 175^2+4\cdot 225^2+2\cdot 275^2\right)-155^2=3\,100.
				\end{align*}
				\item Độ lệch chuẩn của mẫu số liệu ghép nhóm là $s=\sqrt{3\,100}=10\sqrt{31}\approx 55{,}68$.
			\end{itemize}
			
		\end{itemize}
	}
\end{ex}

%%%=============EX_8=============%%%
\begin{ex}%[2D4V2-3]
	Kết quả khảo sát năng suất (đơn vị tấn/ha) của một số thửa ruộng được minh họa ở biểu đồ sau
	\begin{center}
		\begin{tikzpicture}[line join=round, line cap=round,>=stealth,xscale=1.5,yscale=1]
			\def\a{1}
			\def\xmax{9}
			\def\ymax{7}
			\tikzset{label style/.style={font=\footnotesize}}
			\draw[->] (0,0)--(\xmax,0) node[below] {\text{Năng suất (tấn/ha)}};
			\draw[->] (0,0)--(0,\ymax) node[below left] {\text{Số thửa ruộng}};
			\draw (0,0) node [below left] {$O$};
			\draw[thin,gray!60]
			(0,1)--(\xmax,1)
			(0,2)--(\xmax,2)
			(0,3)--(\xmax,3)
			(0,4)--(\xmax,4)
			(0,5)--(\xmax,5)
			(0,6)--(\xmax,6)
			;
			\foreach \i in {1,2,3,4,5,6}{
				\draw (0,\i) node[left]{$\i$};
			}
			%		\foreach \i/\j in {0.5*\a/3,1.5*\a/4,2.5*\a/6,3.5*\a/5,4.5*\a/5,5.5*\a/2}{
			%			\draw (\i,\j) node[above]{$\j$};
			%		}
			\foreach \i/\j in {1.5*\a/{[5.5,5.7)},2.5*\a/{[5.7,5.9)},3.5*\a/{[5.9,6.1)},4.5*\a/{[6.1,6.3)},5.5*\a/{[6.3,6.5)},6.5*\a/{[6.5,6.7)}}{
				\draw (\i,-0.5) node[below,rotate=45]{\small $\j$};
			}
			\fill[blue!20]
			(\a,0) rectangle (2*\a,3)
			(2*\a,0) rectangle (3*\a,4)
			(3*\a,0) rectangle (4*\a,6)
			(4*\a,0) rectangle (5*\a,5)
			(5*\a,0) rectangle (6*\a,5)
			(6*\a,0) rectangle (7*\a,2)
			;
			\begin{scope}
				\draw 
				(\a,3)--(2*\a,3) 
				(2*\a,4)--(3*\a,4)
				(3*\a,6)--(4*\a,6)
				(4*\a,5)--(5*\a,5)
				(5*\a,5)--(6*\a,5)
				(6*\a,2)--(7*\a,2)
				(\a,0)--(\a,3)
				(2*\a,0)--(2*\a,4)
				(3*\a,0)--(3*\a,6)
				(4*\a,0)--(4*\a,6)
				(5*\a,0)--(5*\a,5)
				(6*\a,0)--(6*\a,5)
				(7*\a,0)--(7*\a,2)
				;
				\draw (3,\ymax+1) node[above]{\textbf{Năng suất lúa của một số thửa ruộng}};
			\end{scope}
		\end{tikzpicture}
	\end{center}
	\begin{enumerate}
		\item Có bao nhiêu thửa ruộng được khảo sát?
		\item Lập bảng tần số ghép nhóm và tần số tương đối ghép nhóm tương ứng của mẫu số liệu trên.
		\item Hãy xác định khoảng biến thiên, khoảng tứ phân vị và độ lệch chuẩn của mẫu số liệu trên.
	\end{enumerate}
	\loigiai 
	{
		\begin{enumerate}
			\item Có tất cả $n=3+4+6+5+5+2=25$ thửa ruộng được khảo sát.
			\item Bảng tần số ghép nhóm và tần số tương đối ghép nhóm tương ứng của mẫu số liệu trên.
			\begin{center}
				\begin{tabular}{|c|c|c|c|c|c|c|}
					\hline
					Năng suất (tấn/ha) &$[5{,}5;5{,}7)$  & $[5{,}7;5{,}9)$ & $[5{,}9;6{,}1)$ & $[6{,}1;6{,}3)$ & $[6{,}3;6{,}5)$ & $[6{,}5;6{,}7)$\\
					\hline
					Số thửa ruộng	& $3$ & $4$ & $6$ & $5$ & $5$ & $2$\\
					\hline
					Tần số tương đối & $\dfrac{3}{25}$ & $\dfrac{4}{25}$ & $\dfrac{6}{25}$ & $\dfrac{1}{5}$ & $\dfrac{1}{5}$ & $\dfrac{2}{25}$\\
					\hline
				\end{tabular}
			\end{center}
			\item 
			\begin{itemize}
				\item Khoảng biến thiên của mẫu số liệu trên là $6{,}7-5{,}5=1{,}2$.
				\item Khoảng tứ phân vị
				\begin{itemize}
					\item Cỡ mẫu $n=25$.\\
					Gọi $x_1$; $x_2$;$\ldots$; $x_{25}$ là mẫu số liệu gốc gồm năng suất của $25$ thửa ruộng được sắp xếp theo thứ tự không giảm.\\
					Ta có 
					\begin{itemize}
						\item $x_1$, $x_2$, $x_3\in [5{,}5;5{,}7)$; 
						\item $x_4$; $\ldots$; $x_7\in [5{,}7;5{,}9)$; 
						\item $x_8$; $\ldots$; $x_{13}\in [5{,}9;6{,}1)$; \item $x_{14}$; $\ldots$; $x_{18}\in [6{,}1;6{,}3)$; \item $x_{19}$; $\ldots$; $x_{23}\in [6{,}3;6{,}5)$; \item $x_{24}$; $x_{25}\in [6{,}5;6{,}7)$.
					\end{itemize}
					\item Tứ phân vị thứ nhất của mẫu số liệu gốc là $\dfrac{1}{2}(x_6+x_7)\in [5{,}7;5{,}9)$. \\
					Do đó, tứ phân vị thứ nhất của mẫu số liệu ghép nhóm là 
					\begin{align*}
						Q_1=5{,}7+\dfrac{\dfrac{25}{4}-3}{4}(5{,}9-5{,}7)=5{,}8625.
					\end{align*}
					\item Tứ phân vị thứ ba của mẫu số liệu gốc là $\dfrac{1}{2}(x_{19}+x_{20})\in [6{,}3;6{,}5)$. \\
					Do đó, tứ phân vị thứ ba của mẫu số liệu ghép nhóm là 
					\begin{align*}
						Q_3=6{,}3+\dfrac{\dfrac{3\cdot 25}{4}-(3+4+6+5)}{5}(6{,}5-6{,}3)=6{,}33.
					\end{align*}
					Vậy khoảng tứ phân vị của mẫu số liệu ghép nhóm là 
					$\Delta_Q=6{,}33-5{,}8625=0{,}4675$.	
				\end{itemize}
				Xét mẫu số liệu ghép nhóm cho bởi bảng sau
				\begin{center}
					\begin{tabular}{|c|c|c|c|c|c|c|}
						\hline
						Năng suất (tấn/ha) &$[5{,}5;5{,}7)$  & $[5{,}7;5{,}9)$ & $[5{,}9;6{,}1)$ & $[6{,}1;6{,}3)$ & $[6{,}3;6{,}5)$ & $[6{,}5;6{,}7)$\\
						\hline
						Giá trị đại diện	& $5{,}6$ & $5{,}8$ & $6{,}0$ & $6{,}2$ & $6{,}4$ & $6{,}6$\\
						\hline
						Số thửa ruộng	& $3$ & $4$ & $6$ & $5$ & $5$ & $2$\\
						\hline
					\end{tabular}
				\end{center}
			\begin{itemize}
				\item Số trung bình của mẫu số liệu là 
				\begin{align*}
					\overline{x}=\dfrac{1}{25}\cdot (5{,}6\cdot 3+5{,}8\cdot 4+6\cdot 6+6{,}2\cdot 5+6{,}4\cdot 5+6{,}6\cdot 2)=6{,}088.
				\end{align*}
				\item Phương sai của mẫu số liệu ghép nhóm là 
				\begin{align*}
					s^2=\dfrac{1}{25}\cdot (5{,}6^2\cdot 3+5{,}8^2\cdot 4+6^2\cdot 6+6{,}2^2\cdot 5+6{,}4^2\cdot 5+6{,}6^2\cdot 2)-6{,}088^2=0{,}086656.
				\end{align*}
				\item Độ lệch chuẩn của mẫu số liệu ghép nhóm là $s=\sqrt{0{,}086656}\approx 0{,}294$.
			\end{itemize}
				
			\end{itemize}
		\end{enumerate}
	}
\end{ex}

%%%=============EX_9=============%%%
\begin{ex}%[2D4V2-3]
	Thời gian hoàn thành một bài viết chính tả của một số học sinh lớp $4$ hai trường X và Y được ghi lại ở bảng sau
	\begin{center}
		\begin{tabular}{|c|c|c|c|c|c|}
			\hline
			Thời gian (phút) &$[6;7)$  & $[7;8)$ & $[8;9)$ & $[9;10)$ & $[10;11)$ \\
			\hline
			Học sinh trường $X$	& $8$ & $10$ & $13$ & $10$ & $9$ \\
			\hline
			Học sinh trường $Y$	& $4$ & $12$ & $17$ & $14$ & $3$ \\
			\hline
		\end{tabular}
	\end{center}
	\begin{enumerate}
		\item Nếu so sánh theo số trung bình thì học sinh trường nào viết nhanh hơn?
		\item Nếu so sánh theo khoảng tứ phân vị thì học sinh trường nào có tốc độ viết đồng đều hơn?
		\item Nếu so sánh theo độ lệch chuẩn thì học sinh trường nào có tốc độ viết đồng đều hơn?
	\end{enumerate}
	\loigiai 
	{
		\begin{enumerate}
			\item Ta có bảng thống kê thời gian ghi bài của hoc sinh lớp $4$ ở hai trường theo giá trị đại diện
			\begin{center}
				\begin{tabular}{|c|c|c|c|c|c|}
					\hline
					Thời gian (phút) &$[6;7)$  & $[7;8)$ & $[8;9)$ & $[9;10)$ & $[10;11)$ \\
					\hline
					Giá trị đại diện & $6{,}5$ & $7{,}5$ & $8{,}5$ & $9{,}5$ & $10{,}5$ \\
					\hline
					Học sinh trường $X$	& $8$ & $10$ & $13$ & $10$ & $9$ \\
					\hline
					Học sinh trường $Y$	& $4$ & $12$ & $17$ & $14$ & $3$ \\
					\hline
				\end{tabular}
			\end{center}
			\begin{itemize}
				\item Xét mẫu số liệu của học sinh trường X.\\
				\begin{itemize}
					\item Cỡ mẫu là $n_X=8+10+13+10+9=50$.
					\item Số trung bình của mẫu số liệu ghép nhóm là 
					\begin{align*}
						\overline{x}_X=\dfrac{1}{50}\left(8\cdot 6{,}5+10\cdot 7{,}5+13\cdot 8{,}5+10\cdot 9{,}5+9\cdot 10{,}5\right)=8{,}54.
					\end{align*}
				\end{itemize}
				\item Xét mẫu số liệu của học sinh trường Y.
				\begin{itemize}
					\item Cỡ mẫu là $n_Y=8+10+13+10+9=50$.
					\item Số trung bình của mẫu số liệu ghép nhóm là 
					\begin{align*}
						\overline{x}_Y=\dfrac{1}{50}\left(4\cdot 6{,}5+12\cdot 7{,}5+17\cdot 8{,}5+14\cdot 9{,}5+3\cdot 10{,}5\right)=8{,}5.
					\end{align*}
				\end{itemize}
			\end{itemize}
			Nếu so sánh theo số trung bình thì học sinh trường Y viết nhanh hơn.
			\item 
			\begin{itemize}
				\item Xét mẫu số liệu của học sinh trường X.\\
				\begin{itemize}
					\item Cỡ mẫu $n=50$.\\
					Gọi $x_1$; $x_2$;$\ldots$; $x_{50}$ là mẫu số liệu gốc gồm thời gian của $50$ học sinh ghi bài được sắp xếp theo thứ tự không giảm.\\
					Ta có 
					\begin{itemize}
						\item $x_1$, $\ldots$, $x_8\in [6;7)$; 
						\item $x_9$; $\ldots$; $x_{18}\in [7;8)$; 
						\item $x_{19}$; $\ldots$; $x_{31}\in [8;9)$; 
						\item $x_{32}$; $\ldots$; $x_{41}\in [9;10)$; 
						\item $x_{42}$; $\ldots$; $x_{50}\in [10;11)$.
					\end{itemize}
					\item Tứ phân vị thứ nhất của mẫu số liệu gốc là $x_{13}\in [7;8)$. \\
					Do đó, tứ phân vị thứ nhất của mẫu số liệu ghép nhóm là 
					\begin{align*}
						Q_1=7+\dfrac{\dfrac{50}{4}-8}{10}(8-7)=7{,}45.
					\end{align*}
					\item Tứ phân vị thứ ba của mẫu số liệu gốc là $x_{38}\in [9;10)$.\\ Do đó, tứ phân vị thứ ba của mẫu số liệu ghép nhóm là 
					\begin{align*}
						Q_3=9+\dfrac{\dfrac{3\cdot 50}{4}-(8+10+13)}{10}(10-9)=9{,}65.
					\end{align*}
					Vậy khoảng tứ phân vị của mẫu số liệu ghép nhóm là 
					$\Delta_Q=9{,}65-7{,}45=2{,}2$.
				\end{itemize}
				\item Xét mẫu số liệu của học sinh trường Y.
				\begin{itemize}
					\item Cỡ mẫu $n=50$.\\
					Gọi $x_1$; $x_2$;$\ldots$; $x_{50}$ là mẫu số liệu gốc gồm thời gian của $50$ học sinh ghi bài được sắp xếp theo thứ tự không giảm.\\
					Ta có 
					\begin{itemize}
						\item $x_1$, $\ldots$, $x_4\in [6;7)$; 
						\item $x_5$; $\ldots$; $x_{16}\in [7;8)$; 
						\item $x_{17}$; $\ldots$; $x_{33}\in [8;9)$; 
						\item $x_{34}$; $\ldots$; $x_{47}\in [9;10)$; 
						\item $x_{48}$; $\ldots$; $x_{50}\in [10;11)$.
					\end{itemize}
					\item Tứ phân vị thứ nhất của mẫu số liệu gốc là $x_{13}\in [7;8)$. \\
					Do đó, tứ phân vị thứ nhất của mẫu số liệu ghép nhóm là 
					\begin{align*}
						Q_1=7+\dfrac{\dfrac{50}{4}-4}{12}(8-7)=\dfrac{185}{24}.
					\end{align*}
					\item Tứ phân vị thứ ba của mẫu số liệu gốc là $x_{38}\in [9;10)$. \\
					Do đó, tứ phân vị thứ ba của mẫu số liệu ghép nhóm là 
					\begin{align*}
						Q_3=9+\dfrac{\dfrac{3\cdot 50}{4}-(4+12+17)}{14}(10-9)=\dfrac{261}{28}.
					\end{align*}
					Vậy khoảng tứ phân vị của mẫu số liệu ghép nhóm là 
					$\Delta_Q=\dfrac{261}{28}-\dfrac{185}{24}=\dfrac{271}{168}\approx 1{,}613$.
				\end{itemize}
			\end{itemize}
			Nếu so sánh theo khoảng tứ phân vị thì học sinh trường $Y$ có tốc độ viết đồng đều hơn.
			\item 
			\begin{itemize}
				\item Xét mẫu số liệu của học sinh trường X.
				\begin{itemize}
					\item Phương sai của mẫu số liệu ghép nhóm là
					\begin{align*}
						s^2_X=\dfrac{1}{50}\left(8\cdot 6{,}5^2+10\cdot 7{,}5^2+13\cdot 8{,}5^2+10\cdot 9{,}5^2+9\cdot 10{,}5^2\right)-8{,}54^2=1{,}7584.
					\end{align*}
					\item Độ lệch chuẩn của mẫu số liệu là 
					$s_X=\sqrt{1{,}7584}\approx 1{,}326$.
				\end{itemize}
				\item Xét mẫu số liệu của học sinh trường Y.
				\begin{itemize}
					\item Phương sai của mẫu số liệu ghép nhóm là
					\begin{align*}
						s^2_Y=\dfrac{1}{50}\left(4\cdot 6{,}5^2+12\cdot 7{,}5^2+17\cdot 8{,}5^2+14\cdot 9{,}5^2+3\cdot 10{,}5^2\right)-8{,}5^2=1{,}08.
					\end{align*}
					\item Độ lệch chuẩn của mẫu số liệu là 
					$s_Y=\sqrt{1{,}08}\approx 1{,}039$.
				\end{itemize}
			\end{itemize}
			Do $s_X>s_Y$ nên nếu so sánh theo độ lệch chuẩn thì học sinh trường Y có tốc độ viết đồng đều hơn.
		\end{enumerate}
	}
\end{ex}
%Câu 7

%%%=============EX_10=============%%%
\begin{ex}%[2D4V2-3]
	Bảng sau thống kê lại tổng số giờ nắng trong tháng $6$ của các năm từ $2\,002$ đến $2\,021$ tại hai trạm quan trắc đặt ở Nha Trang và Quy Nhơn
	\begin{center}
		\begin{tabular}{|c|c|c|c|c|c|c|}
			\hline
			Số giờ nắng &$[130;160)$  & $[160;190)$ & $[190;220)$ & $[220;250)$ & $[250;280)$ & $[280;310)$\\
			\hline
			Số năm ở Nha Trang	& $1$ & $1$ & $1$ & $8$ & $7$ & $2$\\
			\hline
			Số năm ở Quy Nhơn & $0$ & $1$ & $2$ & $4$ & $10$ & $3$\\
			\hline
		\end{tabular}
	\end{center}
	\begin{enumerate}
		\item Nếu so sánh theo khoảng tứ phân vị thì số giờ nắng trong tháng $6$ của địa phương nào đồng đều hơn?
		\item Nếu so sánh theo độ lệch chuẩn thì số giờ nắng trong tháng $6$ của địa phương nào đồng đều hơn?
	\end{enumerate}
	\loigiai 
	{
		\begin{enumerate}
			\item 
			\begin{itemize}
				\item Xét mẫu số liệu ở Nha Trang.\\
				Cỡ mẫu $n=20$.\\
				Gọi $x_1$; $x_2$;$\ldots$; $x_{20}$ là mẫu số liệu gốc gồm số giờ nắng trong tháng $6$ ở Nha Trang được sắp xếp theo thứ tự không giảm.\\
				Ta có 
				\begin{itemize}
					\item $x_1\in [130;160)$; 
					\item $x_2\in [160;190)$; 
					\item $x_3\in [190;220)$; 
					\item $x_4$; $\ldots$; $x_{11}\in [220;250)$;
					\item $x_{12}$; $\ldots$; $x_{18}\in [250;280)$; 
					\item $x_{19}$; $x_{20}\in [280;310)$.
				\end{itemize}
				Tứ phân vị thứ nhất của mẫu số liệu gốc là $\dfrac{1}{2}(x_5+x_6)\in [220;250)$. \\
				Do đó, tứ phân vị thứ nhất của mẫu số liệu ghép nhóm là 
				$$Q_1=220+\dfrac{\dfrac{20}{4}-(1+1+1)}{8}(250-220)=227{,}5.$$
				Tứ phân vị thứ ba của mẫu số liệu gốc là $\dfrac{1}{2}(x_{15}+x_{16})\in [250;280)$. \\
				Do đó, tứ phân vị thứ ba của mẫu số liệu ghép nhóm là 
				\[Q_3=250+\dfrac{\dfrac{3\cdot 20}{4}-(1+1+1+8)}{7}(280-250)=\dfrac{1870}{7}.\]
				Vậy khoảng tứ phân vị của mẫu số liệu ghép nhóm là 
				$$\Delta_Q=\dfrac{1870}{7}-227{,}5=\dfrac{555}{14}\approx 39{,}643.$$
				\item Xét mẫu số liệu ở Quy Nhơn.\\
				Cỡ mẫu $n=20$.\\
				Gọi $x_1$; $x_2$;$\ldots$; $x_{20}$ là mẫu số liệu gốc gồm số giờ năng trong tháng $6$ ở Quy Nhơn được sắp xếp theo thứ tự không giảm.\\
				Ta có 
				\begin{itemize}
					\item $x_1\in [160;190)$; 
					\item $x_2$; $x_3\in [190;220)$; 
					\item $x_4$; $\ldots$; $x_7\in [220;250)$; 
					\item $x_{8}$; $\ldots$; $x_{17}\in [250;280)$; 
					\item $x_{18}$; $x_{19}$; $x_{20}\in [280;310)$.
				\end{itemize}
				Tứ phân vị thứ nhất của mẫu số liệu gốc là $\dfrac{1}{2}(x_5+x_6)\in [220;250)$. \\
				Do đó, tứ phân vị thứ nhất của mẫu số liệu ghép nhóm là 
				$$Q_1=220+\dfrac{\dfrac{20}{4}-(1+2)}{4}(250-220)=235.$$
				Tứ phân vị thứ ba của mẫu số liệu gốc là $\dfrac{1}{2}(x_{15}+x_{16})\in [250;280)$. \\
				Do đó, tứ phân vị thứ ba của mẫu số liệu ghép nhóm là 
				$Q_3=250+\dfrac{\dfrac{3\cdot 20}{4}-(1+2+4)}{10}(280-250)=274$.\\
				Vậy khoảng tứ phân vị của mẫu số liệu ghép nhóm là 
				$\Delta_Q=274-235=39$.
			\end{itemize}
			Nếu so sánh theo khoảng tứ phân vị thì số giờ nắng trong tháng $6$ của Quy Nhơn đồng đều hơn.
			\item 
			Ta có bảng thống kê tổng số giờ nắng trong tháng $6$ ở hai địa phương theo giá trị đại diện.
			\begin{center}
				\begin{tabular}{|c|c|c|c|c|c|c|}
					\hline
					Số giờ nắng &$[130;160)$  & $[160;190)$ & $[190;220)$ & $[220;250)$ & $[250;280)$ & $[280;310)$\\
					\hline
					Giá trị đại diện &$145$  & $175$ & $205$ & $235$ & $265$ & $295$\\
					\hline
					Số năm ở Nha Trang	& $1$ & $1$ & $1$ & $8$ & $7$ & $2$ \\
					\hline
					Số năm ở Quy Nhơn & $0$ & $1$ & $2$ & $4$ & $10$ & $3$ \\
					\hline
				\end{tabular}
			\end{center}
			\begin{itemize}
				\item Xét mẫu số liệu ở Nha Trang.
				\begin{itemize}
					\item Cỡ mẫu là $n_X=20$.
					\item Số trung bình của mẫu số liệu ghép nhóm là 
					\begin{align*}
						\overline{x}_X=\dfrac{1}{20}\left(1\cdot 145+1\cdot 175+1\cdot 205+8\cdot 235+7\cdot 265+2\cdot 295\right)=242{,}5.
					\end{align*}
					\item Phương sai của mẫu số liệu ghép nhóm là
					\begin{align*}
						s^2_X=\dfrac{1}{20}\left(1\cdot 145^2+1\cdot 175^2+1\cdot 205^2+8\cdot 235^2+7\cdot 265^2+2\cdot 295^2\right)-242{,}5^2=1248{,}75.
					\end{align*}
					\item Độ lệch chuẩn của mẫu số liệu là 
					$s_X=\sqrt{1248{,}75}\approx 33{,}338$.
				\end{itemize}
				\item Xét mẫu số liệu ở Quy Nhơn.
				\begin{itemize}
					\item Cỡ mẫu là $n_Y=20$.
					\item Số trung bình của mẫu số liệu ghép nhóm là 
					\begin{align*}
						\overline{x}_Y=\dfrac{1}{20}\left(0\cdot 145+1\cdot 175+2\cdot 205+4\cdot 235+10\cdot 265+3\cdot 295\right)=253.
					\end{align*}
					\item Phương sai của mẫu số liệu ghép nhóm là
					\begin{align*}
						s^2_Y=\dfrac{1}{20}\left(0\cdot 145^2+1\cdot 175^2+2\cdot 205^2+4\cdot 235^2+107\cdot 265^2+3\cdot 295^2\right)-253^2=936.
					\end{align*}
					\item Độ lệch chuẩn của mẫu số liệu là 
					$s_Y=\sqrt{936}\approx 30{,}594$.
				\end{itemize}
			\end{itemize}
			Nếu so sánh theo độ lệch chuẩn thì số giờ nắng trong tháng $6$ của Quy Nhơn đồng đều hơn.
		\end{enumerate}
	}
\end{ex}



