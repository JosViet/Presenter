\newpage
\def\thoigian{90}%--Thời gian
\de{Đề số 3}{Chương III. Các số đặc trưng đo mức độ phân tán}


\begin{center}
	\textbf{PHẦN 1 - Câu trắc nghiệm nhiều phương án lựa chọn.}
\end{center}
\setcounter{ex}{0}
\Opensolutionfile{ans}[ans-ABCD]
%Câu 1
\begin{ex}%[2D3N1-2]%[Dự án D - đợt 2 NH 24-25 - Xuan Vy Pham]
Cho bảng số liệu  sau đây
\begin{center}
	\begin{tabular}{|l|c|c|c|c|c|}
		\hline
		{Nhóm} & $[1{,}5; 2{,}5)$ & $ [2{,}5; 3{,}5)$ & $ [3{,}5; 4{,}5)$ & $ [4{,}5; 5{,}5)$ & $ [5{,}5; 6{,}5)$   \\
		\hline
		{Tần số}&$2$ & $3$ & $7$ & $2$ & $1$ \\
		\hline 
	\end{tabular}
\end{center}\noindent
Khoảng biến thiên của mẫu số liệu cho bởi bảng trên là
\choice
{$2 $}
{$ 3$}
{$ 4$}
{\True $ 5$}
\loigiai{
	Khoảng biến thiên của mẫu số liệu là $6{,}5 - 1{,}5 = 5.$ }
\end{ex}
%Câu 2
\begin{ex}%[2D3N1-2]%[Dự án D - đợt 2 NH 24-25 - Xuan Vy Pham]
	Người ta thống kê tốc độ của một số xe ô tô di chuyển qua một trạm kiểm soát trên đường cao tốc trong một khoảng thời gian ở bảng sau.
	\begin{longtable}[c]
		{|c|c|c|c|c|c|}
		\hline
		Tốc độ (km/h)	&$[75;80)$&	$[80;85)$	&$[85;90)$	&$[90;95)$&	$[95;100)$\\
		\hline
		Số xe&	$15$&	$22$&	$28$	&$34$&	$19$\\
		\hline
	\end{longtable}
	\noindent Khoảng biến thiên của mẫu số liệu ghép nhóm trên là\\
	\choice
	{$75$ km/h}
	{\True $25$ km/h}
	{$100$ km/h}
	{$5$ km/h}
	\loigiai{
		Khoảng biến thiên của mẫu số liệu ghép nhóm trên là $100-75=25$  (km/h).}
\end{ex}
%Câu 3
\begin{ex}%[2D3H1-2]%[Dự án D - đợt 2 NH 24-25 - Xuan Vy Pham]
	Kết quả khảo sát cân nặng của $40$ quả cam Canh ở mỗi lô hàng $1$ và lô hàng $2$ được cho ở bảng sau
	\begin{longtable}[c]
		{|c|c|c|c|c|c|}
		\hline
		Cân nặng (gam)	&$[100;110)$	&$[110;120)$	&$[120;130)$	&$[130;140)$&	$[140;150)$\\
		\hline
		Số quả cam ở lô hàng $1$&	$0$&	$10$&	$11$&	$19$	&$0$\\
		\hline
		Số quả cam ở lô hàng $2$&$3$&	$15$&	$12$&	$7$&	$3$\\
		\hline
	\end{longtable}
	\noindent Sử dụng khoảng biến thiên, hãy cho biết cân nặng của $40$ quả cam Canh của lô hàng nào có độ phân tán lớn hơn.
	\choice
	{Không so sánh được}
	{\True Lô hàng $2$ có cân nặng của $40$ quả cam Canh phân tán lớn hơn lô hàng $1$}
	{Lô hàng $1$ có cân nặng của $40$ quả cam Canh phân tán lớn hơn lô hàng $2$}
	{Lô hàng $1$ có cân nặng của $40$ quả cam Canh phân tán bằng lô hàng $2$}
	\loigiai{
		Khoảng biến thiên của mẫu số liệu ghép nhóm về cân nặng của $40$ quả cam Canh của lô hàng $1$ là $140-110=30$ gam.\\
		Khoảng biến thiên của mẫu số liệu ghép nhóm về cân nặng của $40$ quả cam Canh của lô hàng $2$ là $150-100=50$ gam.\\
		Do vậy, lô hàng $2$ có cân nặng của $40$ quả cam Canh phân tán lớn hơn lô hàng $1$.}
\end{ex}
%Câu 4
\begin{ex}%[2D3H1-2]%[Dự án D - đợt 2 NH 24-25 - Xuan Vy Pham]
	Một người thống kê lại thời gian thực hiện các cuộc gọi điện thoại của người đó trong một tuần ở bảng sau. Khoảng tứ phân vị của mẫu số liệu ghép nhóm là
	\begin{longtable}[c]
		{|c|c|c|c|c|c|c|}
		\hline
		Thời gian (đơn vị: giây)&$[0;60)$&$[60;120)$&$[120;180)$&$[180;240)$&$[240;300)$	&$[300;360)$\\
		\hline
		Số cuộc gọi	&$8$&$10$&$7$&$5$&$2$&$1$\\
		\hline
	\end{longtable}
	\choice
	{$100$}
	{$110$}
	{\True $120$}
	{$130$}
	\loigiai{
		Do cỡ mẫu $n=8+10+7+5+2+1=33$.\\
		Nên tứ phân vị thứ nhất của mẫu số liệu gốc là $\dfrac{1}{2}({x_8}+{x_9})$.\\ 
		Mà ${x_8}\in [0;60)$; ${x_9}\in [60;120)$ nên ${Q_1}=60$\\
		Tứ phân vị thứ ba của mẫu số liệu gốc là $\dfrac{1}{2}({x_{25}}+{x_{26}})$.\\ 
		Mà ${x_{25}}\in [120;180)$; ${x_{26}}\in [180;240)$ nên ${Q_3}=180$.\\
		Vậy khoảng tứ phân vị của mẫu số liệu ghép nhóm là $\Delta_Q={Q_3}-{Q_1}=180-60=120$.}
\end{ex}
%Câu 5
\begin{ex}%[2D3H1-2]%[Dự án D - đợt 2 NH 24-25 - Xuan Vy Pham]
	Một phòng khám thống kê số bệnh nhân đến khám bệnh mỗi ngày trong tháng $4$ năm $2\,022$ ở bảng sau. Hãy tìm khoảng tứ phân vị của mẫu số liệu ghép nhóm này? (Làm tròn các kết quả đến hàng phần chục).
	\begin{longtable}[c]
		{|c|c|c|c|c|c|}
		\hline
		Số bệnh nhân	&$[0{,}5;10{,}5)$&$[10{,}5;20{,}5)$&$[20{,}5;30{,}5)$&$[30{,}5;40{,}5)$&$[40{,}5;50{,}5)$\\
		\hline
		Số ngày&$7$&$8$&$7$&$6$&$2$\\
		\hline
	\end{longtable}
	\choice
	{$20{,}3$}
	{\True $20{,}2$}
	{$20{,}4$}
	{$20{,}5$}
	\loigiai{
		Do cỡ mẫu $n=7+8+7+6+2=30$.\\
		Gọi $x_1$; $x_2$; $\ldots$; ${x_{85}}$ là mẫu số liệu gốc gồm điện lượng của $85$ viên pin tiểu.\\
		Ta có
		${x_1}$, $\ldots$, ${x_{10}}\in [0{,}9;0{,}95)$; ${x_{11}}$, $\ldots$, ${x_{30}}\in [0{,}95;1{,}0)$; ${x_{31}}$, $\ldots$, ${x_{65}}\in [1{,}0;1{,}05)$; ${x_{66}}$, $\ldots$, ${x_{80}}\in [1{,}05;1{,}1)$; ${x_{81}}$, $\ldots$, ${x_{85}}\in [1{,}1;1{,}15)$.\\
		Nên tứ phân vị thứ nhất của mẫu số liệu gốc là ${x_8}\in [10{,}5;20{,}5)$. \\
		Do đó tứ phân vị thứ nhất của mẫu số liệu ghép nhóm là
		$$Q_1=10{,}5+\dfrac{\dfrac{30}{4}-7}{8}\cdot (20{,}5-10{,}5)\approx 11{,}1.$$
		Tứ phân vị thứ ba của mẫu số liệu gốc là ${x_{23}}\in [30{,}5;40{,}5)$.\\ Do đó tứ phân vị thứ ba của mẫu số liệu ghép nhóm là\\
		$$Q_3=30{,}5+\dfrac{\dfrac{3\cdot 30}{4}-22}{6}\cdot (40{,}5-30{,}5)\approx 31{,}3.$$
		Vậy khoảng tứ phân vị của mẫu số liệu ghép nhóm là $\Delta_Q=Q_3-Q_1\approx 31{,}3-11{,}1=20{,}2$.}
\end{ex}
%Câu 6
\begin{ex}%[2D3H1-2]%[Dự án D - đợt 2 NH 24-25 - Xuan Vy Pham]
	Một mẫu số liệu ghép nhóm có tứ phân vị thứ $1$ là $254{,}9$ và tứ phân vị thứ $3$ là $417{,}25$ thì điều kiện giá trị ngoại lệ của mẫu số liệu ghép nhóm đó là
	\choice
	{$\hoac{& x\ge 12{,}1\\ & x\le 1{,}35}$}
	{$\hoac{& x > 11{,}2\\ & x < 0{,}375}$}
	{\True $\hoac{& x > 660{,}775\\ & x < 11{,}375}$}
	{$x > 11{,}375$}
	\loigiai{
		Gọi giá trị ngoại lệ của mẫu số liệu ghép nhóm là $x$.\\
		Ta có khoảng tứ phân vị ${{\Delta}_Q}=417{,}25-254{,}9=162{,}35$.\\
		Nên giá trị ngoại lệ
		$\hoac{& x > Q_3+1{,}5{{\Delta}_Q}=417{,}25+1{,}5\cdot 162{,}35=\dfrac{26431}{40}\approx 660{,}775 \\& x < Q_1-1{,}5{{\Delta}_Q}=254{,}25-1{,}5\cdot 162{,}35=\dfrac{91}{8}\approx 11{,}375.}$\\
		Vậy $\hoac{& x > 660{,}775 \\& x < 11{,}375.}$}
\end{ex}
%Câu 7
\begin{ex}%[2D3H2-2]%[Dự án D - đợt 2 NH 24-25 - Xuan Vy Pham]
	Chiều dài của $40$ bé trai sơ sinh $12$ ngày tuổi chọn ngẫu nhiên ở một bệnh viện được nhà nghiên cứu thống kê trong bảng dưới đây
\begin{center}
	\begin{tabular}{|l|c|c|c|c|c|c|}
		\hline {Chiều dài (cm)} & {$[44 ; 46)$} & {$[46 ; 48)$} & {$[48 ; 50)$} & {$[50 ; 52)$} & {$[52 ; 54)$} & {$[54 ; 56)$} \\
		\hline {Số trẻ }& 3 & 3 & 10 & 15 & 7 & 2 \\
		\hline
	\end{tabular}
\end{center}
Độ lệch chuẩn của chiều dài nhóm 40 bé trai sơ sinh (làm tròn kết quả đến hàng phần nghìn) là
	\choice
	{\True $2{,}43$}
	{$50{,}3$}
	{$7{,}09$}
	{$5{,}91$}
	\loigiai{
		Bổ sung thêm các giá trị đại diện vào bảng, ta lập được bảng 
	\begin{center}
		\begin{tabular}{|c|c|c|}
			\hline \textbf{Nhóm} & $\boldsymbol{c}_{\boldsymbol{i}}$ & $\boldsymbol{n}_{\boldsymbol{i}}$ \\
			\hline$[44 ; 46)$ & 45 & 3 \\
			\hline$[46 ; 48)$ & 47 & 3 \\
			\hline$[48 ; 50)$ & 49 & 10 \\
			\hline$[50 ; 52)$ & 51 & 15 \\
			\hline$[52 ; 54)$ & 53 & 7 \\
			\hline$[54 ; 56)$ & 55 & 2 \\
			\hline & & $N=40$ \\
			\hline
		\end{tabular}
	\end{center}
	Từ mẫu số liệu đã cho, ta tính được số trung bình là
	$$
	\overline{x}=\dfrac{3\cdot45+3\cdot47+10\cdot49+15\cdot51+7\cdot53+2\cdot55}{40}=\frac{2012}{40}=50{,}3.
	$$
 	Phương sai là
 	$$s^2=\dfrac{1}{40} \left(3 \cdot 45^2+3 \cdot 47^2+10\cdot 49^2+15 \cdot 51^2+7 \cdot 53^2+2 \cdot 55^2\right)-50{,}3^2=5{,}91.$$
	Vậy mẫu số liệu về chiều dài của $40$ trẻ sơ sinh có độ lệch chuẩn là $s=\sqrt{5{,}91} \approx 2{,}43$.
	}
\end{ex}
%Câu 8
\begin{ex}%[2D3H2-2]%[Dự án D - đợt 2 NH 24-25 - Xuan Vy Pham]
Một câu lạc bộ thể dục thể thao đã ghi lại số giờ các thành viên của mình sử dụng cơ sở vật chất của câu lạc bộ để tập luyện trong một tháng như sau
\begin{center}
	\begin{tabular}{|c|c|c|c|c|c|c|}
		\hline
		Thời gian (giờ) &$[1;5)$&$[5;9)$&$[9;13)$&$[13;17)$&$[17;21)$&$[21;25)$\\
		\hline
		Tần số (số người)&$10$&$14$&$31$&$2$&$5$&$23$\\
		\hline
	\end{tabular}
\end{center}
Độ lệch chuẩn của mẫu số liệu là (kết quả làm tròn đến hàng phần trăm).
\choice
{$6{,}9$}
{$9{,}6$}
{\True $6{,}96$}
{$7{,}96$}
\loigiai{
	\begin{center}
		\begin{tabular}{|c|c|c|c|c|c|c|}
			\hline
			Thời gian (giờ) &$[1;5)$&$[5;9)$&$[9;13)$&$[13;17)$&$[17;21)$&$[21;25)$\\
			\hline
			Giá trị đại diện &$3$&$7$&$11$&$15$&$19$& $23$\\
			\hline
			Tần số (số người)&$10$&$14$&$31$&$2$&$5$&$23$\\
			\hline
		\end{tabular}
	\end{center}
	Số trung bình của mẫu số liệu là
	\[
	\overline{x} = \dfrac{1}{85} \cdot \left( 10\cdot 3 + 14\cdot 7 + 31\cdot 11 + 2\cdot 15 + 5\cdot 19 + 23\cdot 23 \right) \approx 13{,}21.
	\]
	Phương sai của mẫu số liệu ghép nhóm là
	\[
	s^2 = \dfrac{1}{85} \cdot \left( 10\cdot 3^2 + 14\cdot 7^2 + 31\cdot 11^2 + 2\cdot 15^2 + 5\cdot 19^2 + 23\cdot 23^2 \right) - 13\cdot 21^2 \approx 48{,}43.
	\]
	Độ lệch chuẩn của mẫu số liệu ghép nhóm là
	\[
	s = \sqrt{48{,}43} \approx 6{,}96.
	\]
}
\end{ex}
%Câu 9
\begin{ex}%[2D3H2-2]%[Dự án D - đợt 2 NH 24-25 - Xuan Vy Pham]
	Thống kê tổng số giờ nắng trong tháng $9$ tại một trạm quan trắc đặt ở Cà Mau trong các năm từ $2002$  đến $2021$  được thống kê như sau
	\begin{center}
		\begin{tabular}{|c|c|c|c|c|c|}
			\hline
			Số giờ nắng & $[80;98)$ & $[98;116)$ & $[116;134)$ & $[134;152)$& $[152;170)$ \\
			\hline
			Số năm & $3$ & $6$ & $3$ & $5$ & $3$ \\
			\hline
		\end{tabular}
	\end{center}
	Độ lệch chuẩn của mẫu số liệu là (kết quả làm tròn đến hàng phần nghìn)
	\choice
	{\True $23{,}795$}
	{$24{,}795$}
	{$23{,}794$}
	{$23{,}796$}
	\loigiai{
		\begin{center}
			\begin{tabular}{|c|c|c|c|c|c|}
				\hline
				Số giờ nắng & $[80;98)$ & $[98;116)$ & $[116;134)$ & $[134;152)$& $[152;170)$ \\
				\hline
				Giá trị đại diện &$89$&$107$&$125$&$143$&$161$\\
				\hline
				Số năm & $3$ & $6$ & $3$ & $5$ & $3$ \\
				\hline
			\end{tabular}
		\end{center}
		Số trung bình của mẫu số liệu là
		\[
		\overline{x} = \dfrac{1}{20} \cdot \left( 3\cdot 89 + 6\cdot 107 + 3\cdot 125 + 5\cdot 143 + 3\cdot 161 \right) = 124{,}1.
		\]
		Phương sai của mẫu số liệu ghép nhóm là
		\[
		s^2 = \dfrac{1}{20} \cdot \left( 3\cdot 89^2 + 6\cdot 107^2 + 3\cdot 125^2 + 5\cdot 143^2 + 3\cdot 161^2 \right) - 124{,}1^2 = 566{,}19.
		\]
		Độ lệch chuẩn của mẫu số liệu ghép nhóm là
		\[
		s = \sqrt{566,19} \approx 23{,}795.
		\]
	}
\end{ex}
%Câu 10
\begin{ex}%[2D3H2-2]%[Dự án D - đợt 2 NH 24-25 - Xuan Vy Pham]
	Trong $30$ ngày, một nhà đầu tư đã theo dõi giá cổ phiếu của hai công ty $G$ và $H$  vào phiên mở cửa mỗi ngày. Thông tin được ghi lại ở hai bảng dưới đây
	\begin{center}
		{Giá cổ phiếu của công ty G}
		
		\begin{tabular}{|c|c|c|c|c|c|}
			\hline
			{Giá (nghìn đồng)} & $[50; 52]$ & $[52; 54]$ & $[54; 56]$ & $[56; 58]$ & $[58; 60]$ \\
			\hline
			{Tần số} & $3$ & $7$ & $9$ & $8$ & $3$ \\
			\hline
		\end{tabular}
	\end{center}
	
	\begin{center}
		{Giá cổ phiếu của công ty H}
		
		\begin{tabular}{|c|c|c|c|c|c|}
			\hline
			{Giá (nghìn đồng)} & $[40; 42]$ & $[42; 44]$ & $[44; 46]$ & $[46; 48] $& $[48; 50]$ \\
			\hline
			{Tần số} & $6 $&$ 7$ & $5$ & $7$ & $5 $\\
			\hline
		\end{tabular}
	\end{center}
	Biết độ lệch chuẩn càng cao thì tỷ lệ rủi ro càng lớn. Trong các mệnh đề sau, mệnh đề đúng là
	\choice
	{công ty G rủi ro hơn}
	{\True công ty H rủi ro hơn}
	{cả hai đều rủi ro như nhau}
	{cả hai công ty đều không rủi ro}
	\loigiai{
	Bổ sung thêm các giá trị đại diện bảng số liệu của công ty $G$, ta có bảng sau
		\begin{center}
			\begin{tabular}{|c|c|c|c|c|c|c|}
				\hline
				{Giá (nghìn đồng)} & $[50; 52]$ & $[52; 54]$ & $[54; 56]$ & $[56; 58]$ & $[58; 60]$& \\
				\hline
				Giá trị đại diện &$51$&$53$&$55$&$57$&$59$&\\
				\hline
				{Tần số} & $3$ & $7$ & $9 $& $8$ & $3$&$N=30$ \\
				\hline
			\end{tabular}
		\end{center}
		Giá trị trung bình của mẫu số liệu là
		\[
		\overline{x_1} = \dfrac{51 \cdot 3 + 53 \cdot 7 + 55 \cdot 9 + 57 \cdot 8 + 59 \cdot 3}{30}=\dfrac{826}{15}.
		\]
		Phương sai là
		\[
		s_1^2 = \dfrac{51^2 \cdot 3 + 53^2 \cdot 7 + 55^2 \cdot 9 + 57^2 \cdot 8 + 59^2 \cdot 3}{30} -\left(\dfrac{826}{15}\right)^2 \approx 5{,}2.
		\]
		Từ đó ta có độ lệch chuẩn của mẫu số liệu là
		\[
		s_1  \approx \sqrt{5{,}2} \approx 2{,}3.
		\]
		Bổ sung thêm các giá trị đại diện bảng số liệu của công ty $H$, ta có bảng sau
		\begin{center}
			\begin{tabular}{|c|c|c|c|c|c|c|}
				\hline
				{Giá (nghìn đồng)} & $[40; 42]$ & $[42; 44]$ & $[44; 46]$ & $[46; 48]$ & $[48; 50]$& \\
				\hline
				Giá trị đại diện &$41$&$43$&$45$&$47$&$49$&\\
				\hline
				{Tần số} & $6$ & $7$ & $5$ & $7$ & $5$ &$N=30$\\
				\hline
			\end{tabular}
		\end{center}
		Giá trị trung bình của mẫu số liệu là
		\[
		\overline{x_2} = \dfrac{41 \cdot 6 + 43 \cdot 7 + 45 \cdot 5 + 47 \cdot 7 + 49 \cdot 5}{30} =\dfrac{673}{15}.
		\]
	Phương sai là
		\[
	s_2^2= \dfrac{41^2 \cdot 6 + 43^2 \cdot 7 + 45^2 \cdot 5 + 47^2 \cdot 7 + 49^2 \cdot 5}{30}-\left(\dfrac{673}{15}\right)^2 \approx7{,}7.
		\]
		Từ đó ta có độ lệch chuẩn của mẫu số liệu là
		\[
		s_2 \approx \sqrt{7{,}7} \approx 2{,}8.
		\]
		Vì $s_2>s_1$ nên công ty $H$ rủi ro hơn.
	}
\end{ex}
%Câu 11
\begin{ex}%[2D3H2-2]%[Dự án D - đợt 2 NH 24-25 - Xuan Vy Pham]
Anh An đầu tư số tiền bằng nhau vào hai lĩnh vực kinh doanh $A$, $B$. Anh An thống kê số tiền thu được mỗi tháng trong vòng $60$ tháng theo mỗi lĩnh vực cho kết quả như sau
\begin{center}
	\begin{tabular}{|c|c|c|c|c|c|}
		\hline
		{Số tiền (triệu đồng)} & $[5; 10$) & $[10; 15)$ & $[15; 20)$ & $[20; 25)$ & $[25; 30) $\\
		\hline
		{Số tháng đầu tư vào lĩnh vực $A$} & $5$ & $10$ & $30$ & $10$ & $5$ \\
		\hline
		{Số tháng đầu tư vào lĩnh vực $B$} & $20$ & $5$ & $10$ & $5$ & $20$ \\
		\hline
	\end{tabular}
\end{center}
Khẳng định nào sau đây đúng?
\choice
{Đầu tư ở lĩnh vực $A$ rủi ro hơn}
{\True Đầu tư ở lĩnh vực $B$ rủi ro hơn}
{Độ lệch chuẩn ở lĩnh vực $A$ lớn hơn $10$.}
{Đầu tư ở hai lĩnh vực $A$ và $B$ rủi ro như nhau}
\loigiai{
	Ta có
	\begin{center}
		\begin{tabular}{|c|c|c|c|c|c|}
			\hline
			{Giá trị đại diện} & $7{,}5$ & $12{,}5$& $17{,}5$ & $22{,}5$& $27{,}5$ \\
			\hline
			{Số tháng đầu tư vào lĩnh vực $A$} & $5$ & $10$ & $30$ & $10$ & $5$ \\
			\hline
			{Số tháng đầu tư vào lĩnh vực $B$} & $20$ & $5$ & $10$ & $5$ & $20$ \\
			\hline
		\end{tabular}
	\end{center}
	Số tiền trung bình thu được khi đầu tư vào các lĩnh vực $A$, $B$ tương ứng là
	\[
	\overline{x_A} = \dfrac{1}{60} \left( 5 \cdot 7{,}5 + 10 \cdot 12{,}5 +30 \cdot 17{,}5+10 \cdot 22{,}5   + 5 \cdot 27{,}5 \right) = 17{,}5;
	\]
	\[
	\overline{x_B} = \dfrac{1}{60} \left( 20 \cdot 7{,}5 + 5 \cdot 12{,}5 +10 \cdot 17{,}5 +5 \cdot 22{,}5 + 20 \cdot 27{,}5 \right) = 17{,}5.
	\]
	Độ lệch chuẩn của số tiền thu được hàng tháng khi đầu tư vào các lĩnh vực $A$, $B$ tương ứng là
	\[
	s_A = \sqrt{\dfrac{1}{60} \left( 5 \cdot 7{,}5^2 + 10 \cdot 12{,}5^2 +30 \cdot 17{,}5^2 +10 \cdot 22{,}5^2  + 5 \cdot 27{,}5^2 \right) - \left( 17{,}5 \right)^2}= 5;
	\]
	\[
	s_B = \sqrt{\dfrac{1}{60} \left(20 \cdot 7{,}5^2 + 5 \cdot 12{,}5^2 +10 \cdot 17{,}5^2 +5 \cdot 22{,}5^2 + 20 \cdot 27{,}5^2 \right) - \left( 17{,}5 \right)^2} \approx 8.
	\]
	Như vậy, về trung bình đầu tư vào các lĩnh vực $A$, $B$ số tiền thu được hàng tháng như nhau, tuy nhiên độ lệch chuẩn của mẫu số liệu về số tiền thu được hàng tháng khi đầu tư vào lĩnh vực $B$ cao hơn khi đầu tư vào lĩnh vực $A$. \\
	Người ta nói rằng, đầu tư vào lĩnh vực $B$ là rủi ro hơn.}
\end{ex}
%Câu 12
\begin{ex}%[2D3H2-2]%[Dự án D - đợt 2 NH 24-25 - Xuan Vy Pham]
	Khảo sát thời gian tự học trong một tuần của một số học sinh lớp $12$, ta được bảng sau
\begin{center}
	\begin{tabular}{|c|c|c|c|c|c|}\hline
		Thời gian (giờ) & $[12,5;14,5)$& $[14,5;16,5)$& $[16,5;18,5)$& $[18,5;20,5)$&$[20,5;22,5)$ \\\hline
		Số học sinh	& $9$ & $13$ &$17$ & $9$& $4$ \\\hline
	\end{tabular}
\end{center}
Phương sai của mẫu số liệu ghép nhóm (làm tròn kết quả đến phần mười) đã khảo sát là	
\choice
{$4{,}3$}
{\True $5{,}4$}
{$2{,}3$}
{$6{,}1$}
\loigiai{
	Theo bài ra, ta có bảng sau
	\begin{center}
		\begin{tabular}{|c|c|c|c|c|c|}\hline
			Giá trị đại diện & $13{,}5$& $15{,}5$& $17{,}5$& $19{,}5$&$21{,}5)$ \\\hline
			Số học sinh	& $9$ & $13$ &$17$ & $9$& $4$ \\\hline
		\end{tabular}
	\end{center}
	Ta có $\overline{x}=\dfrac{9\cdot 13{,}5+13\cdot 15{,}5+17\cdot 17{,}5+9\cdot 19{,}5+4\cdot 21{,}5}{52}\approx 17$.\\
	Phương sai là
	\begin{align*}	s^2&=\dfrac{9\cdot(13{,}5-17)^2+13\cdot(15{,}5-17)^2+17\cdot(17{,}5-17)^2+9\cdot(19{,}5-17)^2+4\cdot(21{,}5-17)^2}{52}\\
	&\approx 5{,}4.
	\end{align*}}
\end{ex}
\Closesolutionfile{ans}
%\indapan{6}{ans-ABCD}
%\cauds
\begin{center}
	\textbf{PHẦN 2 - Câu trắc nghiệm đúng sai. Trong mỗi ý a, b, c, d ở mỗi câu, thí sinh chọn đúng hoặc sai}
\end{center}
\setcounter{ex}{0}
\Opensolutionfile{ans}[ans-DS]
%Câu 1
\begin{ex}%[2D3H1-2]%[Dự án D - đợt 2 NH 24-25 - Xuan Vy Pham]
	Thống kê thu nhập theo tháng (đơn vị: triệu đồng) của một nhóm người chạy Grab được cho trong bảng sau
	\begin{longtable}{|c|c|c|c|c|}
		\hline
		Thu nhập (triệu đồng) & $[3;5]$ &$[5;7]$ & $[7;9]$ &$[9;11]$ \\ \hline
		Số người & $5$ & $10$ & $5$ & $2$ \\ \hline
	\end{longtable}
	\choiceTF
	{\True Cỡ mẫu là $n=22$}
	{Tứ phân vị thứ nhất của mẫu số liệu ghép nhóm là $Q_1=10$}
	{Tứ phân vị thứ ba của mẫu số liệu ghép nhóm là $Q_3=5$}
	{Khoảng tứ phân vị của mẫu số liệu ghép nhóm là $\Delta Q=5$}
	\loigiai{
		\begin{itemchoice}
			\itemch  Cỡ mẫu là $n=22$.
			\itemch  Tứ phân vị thứ nhất của mẫu số liệu ghép nhóm là $Q_1=10$.\\
			Gọi $x_1, $\ldots$, {x_{22}}$ là thu nhập của 22 người đã được sắp xếp theo thứ tự tăng dần.\\
			Tứ phân vị thứ nhất của mẫu số liệu gốc là $x_6$ nên nhóm chứa tứ phân vị thứ nhất là nhóm $[5;7)$.\\
			Do đó $Q_1=5+\dfrac{\dfrac{1\cdot 22}{4}-5}{10}\cdot 2=5{,}1$.
			\itemch  Tứ phân vị thứ ba của mẫu số liệu ghép nhóm là $Q_3=5$.\\
			Tứ phân vị thứ ba của mẫu số liệu gốc là ${x_{17}}$ nên nhóm chứa tứ phân vị thứ nhất là nhóm $[7;9)$.\\
			Do đó $Q_3=7+\dfrac{\dfrac{3\cdot 22}{4}-15}{5}\cdot 2=7{,}6$.
			\itemch  Khoảng tứ phân vị của mẫu số liệu ghép nhóm là ${{\Delta}_Q}=5$.\\
			Khoảng tứ phân vị của mẫu số liệu ghép nhóm là ${{\Delta}_Q}=7{,}6-5{,}1=2{,}5$.
		\end{itemchoice}	
	}
\end{ex}
%Câu 2
\begin{ex}%[2D3V2-3]%[Dự án D - đợt 2 NH 24-25 - Xuan Vy Pham]
	Bác sĩ A điều trị $18$ bệnh nhân mỡ máu bằng cách xét nghiệm Cholesterol toàn phần trong buổi sáng điều trị như sau
	\[
	3{,}8 \quad 4{,}0 \quad 3{,}8 \quad 4{,}2 \quad 4{,}3 \quad 4{,}5 \quad 4{,}1 \quad 4{,}6 \quad 4{,}8 
	\]
	\[
	5{,}0 \quad 5{,}2 \quad 5{,}1 \quad 4{,}7 \quad 5{,}3 \quad 5{,}6 \quad 5{,}8 \quad 5{,}6 \quad 4{,}4
	\]
	\choiceTF
	{\True Khoảng tứ phân vị của mẫu số liệu trên bảng $1$, độ lệch chuẩn của mẫu số liệu trên bảng $0{,}61$ do bác sĩ $A$ điều trị}
	{Bảng tần số ghép nhóm với nhóm đầu tiên là $[3; 7{,}4; 14]$ và độ dài mỗi nhóm bằng $0{,}44$ do bác sĩ A điều trị được thống kê dưới đây\\
		\begin{tabular}{|c|c|c|c|c|}
			\hline
			{Chỉ số Cholesterol} & $[3;7{,}4;14]$ & $[4{,}14;4{,}58]$ & $[4{,}58;5{,}02]$ & $[5{,}02;5{,}46]$ \\
			{toàn phần do bác sĩ A điều trị} & & & & \\
			\hline
			Số bệnh nhân & $3$ & $5$ & $6$ & $4$ \\
			\hline
		\end{tabular}
	}
	{Giá trị độ lệch chuẩn của mẫu số liệu ghép nhóm đầu tiên là $[3; 7{,}4; 14]$ và độ dài mỗi nhóm bằng $0{,}44$ do bác sĩ $A$ điều trị là $0{,}58$}
	{\True Biết rằng bác sĩ $B$ cũng điều trị $18$ bệnh nhân trên với nhóm đầu tiên là $[3; 7{,}4; 14]$ và độ dài mỗi nhóm bằng $0{,}44$ được thống kê dưới đây:\\
		\begin{tabular}{|c|c|c|c|c|}
			\hline
			{Chỉ số Cholesterol} & $[3;7{,}4;14]$ & $[4{,}14;4{,}58]$ & $[4{,}58;5{,}02]$ & $[5{,}02;5{,}46] $\\
			{toàn phần do bác sĩ B điều trị} & & & & \\
			\hline
			Số bệnh nhân & $4$ & $5$ & $6$ & $3$ \\
			\hline
	\end{tabular}}
	\loigiai{
		\begin{itemchoice}
			\itemch
			Sắp xếp lại bảng số liệu theo thứ tự không giảm như sau
			\[3{,}8; 3{,}8; 4{,}0; 4{,}1; 4{,}2; 4{,}3; 4{,}4; 4{,}5; 4{,}6; 4{,}7; 4{,}8; 5{,}0; 5{,}1; 5{,}2; 5{,}3; 5{,}6; 5{,}8.\]
			Gọi $x_1, x_2, ..., x_{18}$ là mẫu số liệu gốc của $18$ bệnh nhân mỡ máu bằng cách xét nghiệm Cholesterol toàn phần trong một ngày theo thứ tự không giảm.\\
			Trung vị $Q_2$ là
			\[
			Q_2 = \dfrac{1}{2} \left( x_9 + x_{10} \right) = \dfrac{1}{2} \left( 4{,}6 + 4{,}7 \right) = 4{,}65.
			\]
			Tứ phân vị thứ nhất là $Q_1 = 4{,}2$.\\
			Tứ phân vị thứ ba là $Q_3 = 5{,}2$.\\
			Khoảng tứ phân vị của mẫu số liệu trên $\Delta_Q = Q_3 - Q_1 = 5{,}2 - 4{,}2 = 1$.\\
			Số trung bình của mẫu số liệu trên do bác sĩ $A$ điều trị bằng
			\begin{eqnarray*}
				\overline{x} = &&\dfrac{2{,}3 + 3{,}8 + 4{,}0 + 4{,}1 + 4{,}2 + 4{,}3 + 4{,}4 + 4{,}4 + 4{,}5 + 4{,}6 + 4{,}7 + 4{,}8}{18}\\
				&& \dfrac{ + 5{,}0 + 5{,}1 + 5{,}2 + 5{,}3 + 2{,}5 + 5{,}8}{18} = \dfrac{212}{45}.
			\end{eqnarray*}
			Phương sai của mẫu số liệu trên do bác sĩ $A$ điều trị bằng
			\[
			s^2 = \dfrac{x_1^2 + x_2^2 + x_3^2 + \cdots + x_{18}^2}{18} - \bar{x}^2 = \dfrac{3023}{8100}.
			\]
			Độ lệch chuẩn của mẫu số liệu trên do bác sĩ $A$ điều trị bằng $s=\sqrt{s^2}=0{,}61$. 
			\itemch 
			Bảng tần số ghép nhóm với nhóm đầu tiên là $[3; 7{,}4; 14]$ và độ dài mỗi nhóm bằng $0{,}44$ do bác sĩ $A$ điều trị được thống kê dưới đây
			\begin{center}
				\begin{tabular}{|p{5cm}|c|c|c|c|c|}
					\hline
					Chỉ số Cholesterol toàn phần do bác sĩ $A$  điều trị & \([3{,}7; 4{,}14]\) & \([4{,}14; 4{,}58]\) & \([4{,}58; 5{,}02]\) & \([5{,}02; 5{,}46]\) & \([5{,}46; 5{,}9]\) \\
					\hline
					\multicolumn{1}{|p{5cm}|}{\centering Số bệnh nhân} & $4$ & $4$ & $3$ & $3 $& $3$ \\
					\hline
				\end{tabular}
			\end{center}
			\itemch 
			Giá trị độ lệch chuẩn của mẫu số liệu ghép nhóm đầu tiên là $[3; 7{,}4; 14]$ và độ dài mỗi nhóm bằng $0{,}44$ do bác sĩ $A$ điều trị
			\begin{center}
				\begin{tabular}{|p{5cm}|c|c|c|c|c|}
					\hline
					Chỉ số Cholesterol toàn phần do bác sĩ  $A$ điều trị & \([3{,}7; 4{,}14]\) & \([4{,}14; 4{,}58]\) & \([4{,}58; 5{,}02]\) & \([5{,}02; 5{,}46]\) & \([5{,}46; 5{,}9]\) \\
					\hline
					\multicolumn{1}{|p{5cm}|}{\centering Giá trị đại diện} &$3{,}92$ & $4{,}36$ & $4{,}8$ & $5{,}24$ & $5{,}68$\\
					\hline
					\multicolumn{1}{|p{5cm}|}{\centering Số bệnh nhân} & $4$ & $4$ & $3$ & $3$ & $3$ \\
					\hline
				\end{tabular}
			\end{center}
			Số trung bình của mẫu số liệu trên do bác sĩ $A$ điều trị bằng
			\[
			\overline{x}_A = \dfrac{4 \cdot 3{,}92 + 4 \cdot 4{,}36 + 4 \cdot 4{,}8 + 3 \cdot 5{,}24 + 3 \cdot 5{,}68}{18} = \dfrac{709}{150}.
			\]
			Phương sai của mẫu số liệu trên do bác sĩ $A$ điều trị bằng
			\[
			s_A^2 = \dfrac{4 \cdot 3{,}92^2 + 4 \cdot 4{,}36^2 + 4 \cdot 4{,}8^2 + 3 \cdot 5{,}24^2 + 3 \cdot 5{,}68^2}{18} - \left( \dfrac{709}{150} \right)^2 = \dfrac{2783}{7500}.
			\]
			Suy ra $s_A = \sqrt{s_A^2} = 0{,}609$.
			\itemch 
			Ta có		
			\begin{center}
				\begin{tabular}{|>{\centering\arraybackslash}m{5cm}|>{\centering\arraybackslash}m{2cm}|>{\centering\arraybackslash}m{2cm}|>{\centering\arraybackslash}m{2cm}|>{\centering\arraybackslash}m{2cm}|>{\centering\arraybackslash}m{2cm}|}
					\hline
					\multicolumn{1}{|>{\centering\arraybackslash}m{5cm}|}{\centering Chỉ số Cholesterol toàn phần do bác sĩ $B$ điều trị} & $[3{,}7;4{,}14]$ & $[4{,}14;4{,}58]$ & $[4{,}58;5{,}02]$ & $[5{,}02;5{,}46]$ & $[5{,}46;5{,}9]$ \\
					\hline
					\multicolumn{1}{|>{\centering\arraybackslash}m{5cm}|}{\centering Giá trị đại diện} & $3{,}92$ & $4{,}36 $& $4{,}8$ & $5{,}24$ & $5{,}68$ \\
					\hline
					\multicolumn{1}{|>{\centering\arraybackslash}m{5cm}|}{\centering Số bệnh nhân} & $3$ & $4$ & $3$ & $4$ & $4 $\\
					\hline
				\end{tabular}
			\end{center}		
			Số trung bình của mẫu số liệu trên do bác sĩ $B$ điều trị bằng
			\[
			\overline{x_B} = \dfrac{3 \cdot 3{,}92 + 4 \cdot 4{,}36 + 3 \cdot 4{,}8 + 4 \cdot 5{,}24 + 4 \cdot 5{,}68}{18} = \dfrac{1091}{225}
			\]
			Phương sai của mẫu số liệu trên do bác sĩ $B$ điều trị bằng:
			\[
			s_B^2 = \dfrac{3 \cdot 3{,}92^2 + 4 \cdot 4{,}36^2 + 3 \cdot 4{,}8^2 + 4 \cdot 5{,}24^2 + 4 \cdot 5{,}68^2}{18} - \left( \dfrac{1091}{225} \right)^2 \approx 0{,}3848
			\]
			Độ lệch chuẩn của mẫu số liệu trên do bác sĩ $B$ điều trị bằng $ s_B = \sqrt{s_B^2} = 0{,}62$.\\
			Vì $ s_A <s_B $ nên so sánh về độ lệch chuẩn thì chỉ số Cholesterol toàn phần bác sĩ $A$ điều trị ít phân tán hơn bác sĩ $B$ điều trị.
		\end{itemchoice}
	}
\end{ex}
\Closesolutionfile{ans}
\begin{center}
	\textbf{PHẦN 3 - Câu trắc nghiệm trả lời ngắn}
\end{center}
\setcounter{ex}{0}
%Câu 1
\begin{ex}%[2D3H1-2]%[Dự án D - đợt 2 NH 24-25 - Xuan Vy Pham]
	Bảng tần số ghép nhóm dưới đây thể hiện kết quả điều tra về tuổi thọ trung bình của nam giới ở 50 quốc gia.
	\begin{longtable}{|c|c|c|c|c|c|c|c|c|}
		\hline
		Độ tuổi &$ [50; 55)$ & $[55; 60) $&$ [60; 65)$ &$ [65; 70) $&$ [70; 75)$ &$ [75; 80)$ &$ [80; 85) $&$ [85; 90]$ \\ \hline
		Tần số &$4$ &$7$& $4$ &$6$& $16$& $12$ & $2$ & $0$ \\ \hline
	\end{longtable}
	\noindent	Hãy xác định khoảng biến thiên của tuổi thọ trung bình của nam giới trong mẫu số liệu ghép nhóm trên.
	\shortans{$35$}
	\loigiai{
		Do nhóm số liệu $[85; 90)$ có tần số là 0 nên ta sẽ chỉ xét đến nhóm số liệu [$80$; $85$).\\
		Do đó $R=85-50=35$.}
\end{ex}
%Câu 2
\begin{ex}%[2D3V1-2]%[Dự án D - đợt 2 NH 24-25 - Xuan Vy Pham]
Bảng dưới đây cho ta bảng tần số ghép nhóm về số liệu thống kê tỉ lệ che phủ rừng (đơn vị $\%$) của $60$ tỉnh, thành phố ở Việt Nam (không bao gồm Hưng Yên, Vĩnh Long, Cần Thơ) tính đến ngày $31$/$12$/$2020$. \textit{(Nguồn: https://bandolamnghiep.com)}
\begin{center}
	\begin{tabular}{|c|c|c|c|c|c|c|c|c|}
		\hline
		Nhóm & $[0;10)$ & $[10;20)$ & $[20;30)$ & $[30;40)$ & $[40;50)$ & $[50;60)$ & $[60;70)$ & $[70;80)$ \\
		\hline
		Tần số & $17$ & $6$ & $3$ & $4$ & $9$ & $15$ & $5$ & $1$ \\
		\hline
	\end{tabular}
\end{center}
\noindent Tìm khoảng tứ phân vị của mẫu số liệu đã cho (kết quả làm tròn đến hàng phần chục).
\shortans{$45{,}2$}
\loigiai
{
	Cỡ mẫu $n=60$. \\
	Ta có $\dfrac{n}{4}=15$. \\
	Vì nhóm $[0;10)$ có tần số tích lũy bằng $17$ nên $Q_1\in(0;10]$. \\
	Ta có $Q_1=0+\dfrac{\dfrac{60}{4}-0}{17}\cdot(10-0)=\dfrac{150}{17}$. \\
	Ta có $\dfrac{3n}{4}=45$. \\
	Vì nhóm $[50;60)$ có tần số tích lũy bằng $17+6+3+4+9+15=54$ nên $Q_3\in(50;60]$. \\
	Ta có $Q_3=50+\dfrac{\dfrac{3\cdot60}{4}-39}{15}\cdot(60-50)=54$. \\
	Vậy khoảng tứ phân vị là $\Delta_Q=Q_3-Q_1=\dfrac{768}{17}\approx45{,}2$.
}
\end{ex}
%Câu 3...........................
\begin{ex}%[2D3H2-2]%[Dự án D - đợt 2 NH 24-25 - Xuan Vy Pham]
	Kết quả khảo sát năng suất (đơn vị: tấn/ha) của một số thửa ruộng được minh hoạ ở biểu đồ sau:
\begin{center}
	\begin{tikzpicture}[>=stealth,line join=round,line cap=round,font=\footnotesize,scale=0.78,line width=1pt]
		%Vẽ các trục tọa độ
		\draw[->] (0,0)--(0,7)node[left]{\text{Số thửa ruộng}};
		\draw[->] (0,0)node[shift={(-135:.4)}]{$O$}--(9.5,0)node[shift={(-40:.6)}]{\text{Năng suất (tấn/ha)}};
		%Vẽ trục gióng ngang và tọa độ trên trục Oy
		\foreach \y in {1,2,3,4,5,6}
		\draw[shift={(0,\y)}] (0,0)--(-2pt,0) node[left]{\scriptsize ${\y}$};
		\foreach \y in {1,2,3,4,5,6}{
			\draw[dashed,thin,line width=0.01pt] (0,\y)--(6,\y);
		}
		%Vẽ đồ thị cột với các tùy chọn
		\foreach \x/\y in {1/3,2/4,3/6,4/5,5/5,6/2}
		\draw[line cap=round,pattern=dots] (\x,0) rectangle (\x+1,\y);
		
		\foreach \x/\y in {1/5{,}5,2/5{,}7,3/5{,}9,4/6{,}1,5/6{,}3,6/6{,}5,7/6{,}7}
		\path (\x,0)node[shift={(-90:.4)}]{$\y$};
		\path (4,7.8)node[shift={(90:.2)},scale=1.2]{\textbf{Năng suất lúa của một số thửa ruộng}};
	\end{tikzpicture}
\end{center}
Tìm phương sai của mẫu số liệu trên (làm tròn đến hàng phần trăm).
\shortans{$0{,}09$}
\loigiai{
	Ta lập được bảng tần số ghép nhóm và tần số tương đối ghép nhóm tương ứng của mẫu số liệu trên như sau:
	\begin{center}
		\begin{tabular}{|c|c|c|c|c|c|c|}
			\hline
			{Năng suất} & $[5{,}5;5{,}7)$ & $[5{,}7;5{,}9)$ & $[5{,}9;6{,}1)$ & $[6{,}1;6{,}3)$ & $[6{,}3;6{,}5)$ & $[6{,}5;6{,}7)$\\ 
			\hline
			{Số thửa ruộng} & $3$ & $4$ & $6$ & $5$ & $5$ & $2$\\
			\hline
			{Giá trị đại diện} & $5{,}6$ & $5{,}8$ & $6{,}0$ & $6{,}2$ & $6{,}4$ & $6{,}6$\\ 
			\hline
			{Tần số} & $3$ & $4$ & $6$ & $5$ & $5$ & $2$\\
			\hline
		\end{tabular}
	\end{center}
	Cỡ mẫu $n = 25$.\\
	Năng suất lúa trung bình là
	\[
	\overline{x} = \dfrac{1}{25} (3 \cdot 5{,}6 + 4 \cdot 5{,}8 + 6 \cdot 6{,}0 + 5 \cdot 6{,}2 + 5 \cdot 6{,}4 + 2 \cdot 6{,}6) = 6{,}088.
	\]
	Phương sai của mẫu số liệu là
	\[
	s^2 = \dfrac{1}{25}\left(
	3 \cdot 5{,}6^2 + 4 \cdot 5{,}8^2 + 6 \cdot 6{,}0^2 + 5 \cdot 6{,}2^2 + 5 \cdot 6{,}4^2 + 2 \cdot 6{,}6^2\right) - 6{,}088^2
	= \dfrac{1354}{15625}\approx 0{,}09.
	\]
}
\end{ex}
%Câu 4
\begin{ex}%[2D3V2-2]%[Dự án D - đợt 2 NH 24-25 - Xuan Vy Pham]
So sánh giá trị trung bình và độ lệch chuẩn của số tiền thu được mỗi tháng khi bắt đầu tư vào từng lĩnh vực $A,B$. Kí hiệu $\Delta =s_A-s_B$ nếu $\Delta >0$ thì đầu tư vào lĩnh vực $A$ “rủi ro” hơn, ngược lại nếu $\Delta <0$ thì đầu tư vào lĩnh vực $B$ “rủi ro” hơn. Tính $\Delta =s_A-s_B$( làm tròn kết quả đến hàng phần trăm).
\begin{longtable}{|>{\centering\arraybackslash}p{4cm}|>{\centering\arraybackslash}p{1.6cm}|>{\centering\arraybackslash}p{1.6cm}|>{\centering\arraybackslash}p{1.6cm}|>{\centering\arraybackslash}p{1.6cm}|>{\centering\arraybackslash}p{1.6cm}|}
	\hline
	Giá đóng cửa & $[120;122)$ & $[122;124)$ & $[124;126)$ & $[126;128)$ & $[128;130)$ \\ \hline
	Cổ phiếu  $A$ & $8$ & $9$ & $12$ & $10$ & $11$ \\ \hline
	Cổ phiếu $B$ & $16$ & $4$ & $3$ & $6$ & $21$ \\ \hline
\end{longtable}
\shortans{$-0{,}78$}
\loigiai{
	\begin{itemize}
		\item Ta có bảng thống kê giá đóng cửa theo giá trị đại diện
		\begin{longtable}{|>{\centering\arraybackslash}p{4cm}|>{\centering\arraybackslash}p{1.5cm}|>{\centering\arraybackslash}p{1.5cm}|>{\centering\arraybackslash}p{1.5cm}|>{\centering\arraybackslash}p{1.5cm}|>{\centering\arraybackslash}p{1.5cm}|}
			\hline
			Giá trị đại diện & $121$ & $123$ & $125$ & $127$ & $129$ \\ \hline
			Cổ phiếu  $A$ & $8$ & $9$ & $12$ & $10$ & $11$ \\ \hline
			Cổ phiếu $B$ & $16$ & $4$ & $3$ & $6$ & $21$ \\ \hline
		\end{longtable}
		\item	 Xét mẫu số liệu của cổ phiếu $A$\\
		Số trung bình của mẫu số liệu ghép nhóm là
		\[\overline{x}_A=\dfrac{8\cdot 121+9\cdot 123+12\cdot 125+10\cdot 127+11\cdot 129}{50}=125{,}28.\]
		Phương sai của mẫu số liệu ghép nhóm là
		\[s_A^2=\dfrac{1}{50}\left( 8\cdot 121^2+9\cdot 123^2+12\cdot 125^2+10\cdot 127^2+11\cdot 129^2 \right)-125{,}28^2\approx 7{,}52.\]
		Độ lệch chuẩn của mẫu số liệu ghép nhóm là ${s_A}=\sqrt {s_A^2}=\sqrt {7{,}52}\approx 2{,}74$.
		\item	Xét mẫu số liệu của cổ phiếu $B$:\\
		Số trung bình của mẫu số liệu ghép nhóm là
		\[\overline{x}_B=\dfrac{16\cdot 121+4\cdot 123+3\cdot 125+6\cdot 127+21\cdot 129}{50}=125{,}28 .\]
		Phương sai của mẫu số liệu ghép nhóm là\\
		$s_B^2=\dfrac{1}{50}\left( 16\cdot 121^2+4\cdot 123^2+3\cdot 125^2+6\cdot 127^2+21\cdot 129^2 \right)-125{,}48^2\approx 12{,}4.$\\
		Độ lệch chuẩn của mẫu số liệu ghép nhóm là ${s_B}=\sqrt {s_B^2}=\sqrt {12{,}4}\approx 3{,}52$.
	\end{itemize}
	Vậy	$\Delta =s_A-s_B=2{,}74-3{,}52=-0{,}78$.
}
\end{ex}
\Closesolutionfile{ansKQ}
\begin{center}
	\textbf{PHẦN 4 - Tự luận.}
\end{center}
\setcounter{ex}{0}
%Câu 1
\begin{ex}%[2D3V1-2]%[Dự án D - đợt 2 NH 24-25 - Xuan Vy Pham]
	Kết quả đo chiều cao của $100$ cây keo $3$ năm tuổi tại một nông trường được cho ở bảng sau
	\begin{longtable}{|c|c|c|c|c|c|}
		\hline
		Chiều cao (m) & $[8, 4;8, 6] $&$ [8, 6;8, 8]$ &$ [8, 9;9, 0] $&$ [9, 0;9, 2] $& $[9, 2;9, 4] $\\
		\hline
		Tần số & $5$ & $12$ & $25$ & $44$ &$14$ \\
		\hline
	\end{longtable}
	\noindent	Hãy xác định khoảng tứ phân vị của mẫu số liệu ghép nhóm trên (làm tròn đến hàng phần trăm).
	\loigiai{
		Cỡ mẫu $n=100$.\\
		Gọi ${x_1};{x_2};\ldots;{x_{100}}$ là mẫu số liệu gốc gồm các chiều cao của $100$ cây keo $3$ năm tuổi được xếp theo thứ tự không giảm.\\
		Ta có ${x_1}, $\ldots$, {x_5}\in [8{,}4;8{,}6)$; ${x_6}, \ldots, {x_{17}}\in [8{,}6;8{,}8)$; ${x_{18}}, \ldots, {x_{42}}\in [8{,}9;9{,}0)$;${x_{43}}, \ldots, {x_{86}}\in [9{,}0;9{,}2)$; ${x_{87}}, \ldots, {x_{100}}\in [9{,}2;9{,}4)$\\
		Tứ phân vị thứ nhất của mẫu số liệu gốc là $\dfrac{1}{2}({x_{50}}+{x_{51}})\in [9{,}0;9{,}2)$. \\
		Do đó, tứ phân vị thứ nhất của mẫu số liệu ghép nhóm là
		\begin{align*}Q_1=9{,}0+\dfrac{\dfrac{100}{4}-42}{44}\cdot (9{,}2-9{,}0)=\dfrac{1963}{220}.
		\end{align*}
		Tứ phân vị thứ ba của mẫu số liệu gốc là ${x_{75}}\in [9{,}0;9{,}2)$. \\
		Do đó, tứ phân vị thứ ba của mẫu số liệu ghép nhóm là
		\begin{align*}Q_3=9{,}0+\dfrac{\dfrac{3\cdot 100}{4}-42}{44}\cdot (9{,}2-9{,}0)=\dfrac{183}{20}.
		\end{align*}
		Vậy khoảng tứ phân vị của mẫu số liệu ghép nhóm là ${{\Delta}_Q}=\dfrac{183}{20}-\dfrac{1963}{220}=\dfrac{5}{22}\approx 0{,}23$.}
\end{ex}
%Câu 2
\begin{ex}%[2D3V2-2]%[Dự án D - đợt 2 NH 24-25 - Xuan Vy Pham]
	Một giống cây xoan đào được trồng tại hai địa điểm $A$ và $B$. Người ta thống kê đường kính thân của một số cây xoan đào $5$ năm tuổi ở bảng sau.
	\begin{longtable}{|>{\centering\arraybackslash}p{5cm}|>{\centering\arraybackslash}p{1.6cm}|>{\centering\arraybackslash}p{1.6cm}|>{\centering\arraybackslash}p{1.6cm}|>{\centering\arraybackslash}p{1.6cm}|>{\centering\arraybackslash}p{1.6cm}|}
		\hline
		Đường kính (cm) & $[ 30;32 )$ & $[ 32;34 )$ & $[ 34;36 )$ & $[ 36;38 )$ & $[ 38;40 )$ \\ \hline
		Số cây trồng ở địa điểm $A$ & $25$ & $38$ & $20$ & $10$ & $7$ \\ \hline
		Số cây trồng ở địa điểm $B$ & $22$ & $27$ & $19$ & $18$ & $14$ \\ \hline
	\end{longtable}
	\noindent	Gọi phương sai đường kính thân của một số cây xoan đào $5$ năm tuổi ở địa điểm $A$ và $B$ lần lượt là $S_A^2$ và  $S_B^2$. Tính $\left|S_A^2-S_B^2\right|$ bằng bao nhiêu?			
	\loigiai{
		\begin{itemize}
			\item Ta lập bảng theo giá trị đại diện như sau
			\begin{longtable}{|>{\centering\arraybackslash}p{5cm}|>{\centering\arraybackslash}p{1.6cm}|>{\centering\arraybackslash}p{1.6cm}|>{\centering\arraybackslash}p{1.6cm}|>{\centering\arraybackslash}p{1.6cm}|>{\centering\arraybackslash}p{1.6cm}|}
				\hline
				Đường kính (cm) & $[ 30;32 )$ & $[ 32;34 )$ & $[ 34;36 )$ & $[ 36;38 )$ & $[ 38;40 )$ \\ \hline
				Giá trị đại diện & $31 $& $33$ & $35$ & $37$ & $39$ \\ \hline
				Số cây trồng ở địa điểm $A$ & $25$ & $38$ & $20 $& $10$ & $9 $\\ \hline
				Số cây trồng ở địa điểm $B$ & $22$ & $27$ & $19$ & $14$ & $14$ \\ \hline
			\end{longtable}
			\noindent
			\item 	Cỡ mẫu: ${n_A}=25+38+20+10+7=100;{n_B}=22+27+19+18+14=100$.
			\item 	Đường kính trung bình của thân cây xoan đào trồng tại địa điểm $A$ là			\[\overline{x}_A=\dfrac{25\cdot 31+38\cdot 33+20\cdot 35+10\cdot 37+7\cdot 39}{100}=33{,}72.\]
			\item Phương sai của mẫu số liệu ghép nhóm về đường kính của thân cây xoan đào trồng tại địa điểm $A$ là
			\[S_A^2=\dfrac{1}{100}\left( 25\cdot 31^2+38\cdot 33^2+20\cdot 35^2+10\cdot 37^2+7\cdot 39^2 \right)- 33{,}72^2\approx 5{,}40.\]
			\item Đường kính trung bình của thân cây xoan đào trồng tại địa điểm $B$ là
			\[\overline{x}_B=\dfrac{22\cdot 31+27\cdot 33+19\cdot 35+18\cdot 37+14\cdot 39}{100}=34{,}5.\]
			\item Phương sai của mẫu số liệu ghép nhóm về đường kính của thân cây xoan đào trồng tại địa điểm $B$ là
			\[S_B^2=\dfrac{1}{100}\left( 22\cdot 31^2+27\cdot 33^2+19\cdot 35^2+18\cdot 37^2+14\cdot 39^2 \right)- 34{,}5^2=7{,}31.\]
		\end{itemize}
		Vậy $| S_A^2-S_B^2 |=| 5{,}40-7{,}31 |=1{,}91$.}
\end{ex}
%Câu 3
\begin{ex}%[2D3H2-2]%[Dự án D - đợt 2 NH 24-25 - Xuan Vy Pham]
	Điều tra về số tiền mua sách (đơn vị: nghìn đồng) trong một năm của $50$ học sinh trong một trường THPT, người ta có bảng sau
	\begin{longtable}{|>{\centering\arraybackslash}p{4cm}|>{\centering\arraybackslash}p{2.cm}|>{\centering\arraybackslash}p{2.cm}|>{\centering\arraybackslash}p{2.cm}|>{\centering\arraybackslash}p{2.cm}|>{\centering\arraybackslash}p{2.cm}|}
		\hline
		{Số tiền mua sách} & $[0;200)$ & $[200;400)$ & $[400;600)$ & $[600;800)$ & $[800;1000)$ \\ \hline
		{Số học sinh} & $29$ & $11$ & $3$ & $4$ & $3$ \\ \hline
	\end{longtable}
	\noindent Tính độ lệch chuẩn của mẫu số liệu ghép nhóm trên (kết quả làm tròn đến hàng đơn vị).
	\loigiai{
		\begin{itemize}
			\item Chọn giá trị đại diện cho mẫu số liệu, ta có
			\begin{longtable}{|>{\centering\arraybackslash}p{4cm}|>{\centering\arraybackslash}p{2.cm}|>{\centering\arraybackslash}p{2.cm}|>{\centering\arraybackslash}p{2.cm}|>{\centering\arraybackslash}p{2.cm}|>{\centering\arraybackslash}p{2.cm}|}
				\hline
				\textbf{Số tiền mua sách} & $[0;200)$ & $[200;400)$ & $[400;600)$ & $[600;800)$ & $[800;1000)$ \\ \hline
				\textbf{Giá trị đại diện} & $100$ & $300$ &$500$ &$700$ &$900$\\ \hline
				\textbf{Số học sinh} & $29$ & $11$ & $3$ & $4$ & $3$ \\ \hline
			\end{longtable}
			\item 	Điểm trung bình là
			\[\overline{x}=\dfrac{29\cdot 100+11\cdot 300+3\cdot 500+4\cdot 700+3\cdot 900}{50}=264.\]
			\item 	Phương sai là
			\[S^2=\dfrac{1}{50}\left[ 29\cdot  100 ^2+11\cdot  300 ^2+3\cdot  500^2+4\cdot  700^2+3\cdot  900^2 \right]- 264 ^2=58704.\]
			\item	Độ lệch chuẩn: $S=\sqrt {58704}\approx 242$.
	\end{itemize}}
\end{ex}	

