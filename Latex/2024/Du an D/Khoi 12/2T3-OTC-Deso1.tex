\newpage
\section{Ôn tập chương 3}

\def\thoigian{90}%--Thời gian
\de{Đề số 1}{CHƯƠNG III. CÁC SỐ ĐẶC TRƯNG ĐO ĐỘ PHÂN TÁN}
\begin{center}
	\textbf{PHẦN 1 - CÂU TRẮC NGHIỆM BỐN PHƯƠNG ÁN}
\end{center}
\Opensolutionfile{ans}[ans/ans-TN-ONTAPCHUONG-DE1]
\begin{ex}%[Dự án đề cương 3 khối NH24-25 - Đợt 3 - Lê Phúc]%[2D3N1-1]
	Cho mẫu số liệu ghép nhóm có tứ phân vị thứ nhất, thứ hai, thứ ba lần lượt là $Q_1$, $Q_2$, $Q_3$. Khoảng tứ phân vị của mẫu số liệu ghép nhóm đó bằng
	\choice
	{\True $Q_3-Q_1$}
	{$Q_2-Q_1$}
	{$Q_2$}
	{$Q_3+Q_1$}
	\loigiai{
		Khoảng tứ phân vị của mẫu số liệu ghép nhóm là $\Delta_Q=Q_3-Q_1$.
	}
\end{ex}
\begin{ex}%[2D3N2-2]%[Dự án đề cương 3 khối NH24-25 - Đợt 3 - Lê Phúc]
	Khẳng định nào sau đây \textbf{sai}?
	\choice
	{\True Phương sai của mẫu số liệu ghép nhóm là căn bậc hai số học của độ lệch chuẩn}
	{Độ lệch chuẩn càng lớn thì mẫu số liệu càng phân tán}
	{Phương sai càng lớn thì mẫu số liệu càng phân tán}
	{Độ lệch chuẩn của mẫu số liệu ghép nhóm là căn bậc hai số học của phương sai}
	\loigiai{
		Khẳng định sai là \lq\lq Phương sai của mẫu số liệu ghép nhóm là căn bậc hai số học của độ lệch chuẩn \rq\rq \, vì độ lệch chuẩn của mẫu số liệu ghép nhóm là căn bậc hai số học của phương sai.
	}
\end{ex}
\begin{ex}%[2D3H1-3]%[Dự án đề cương 3 khối NH24-25 - Đợt 3 - Lê Phúc]
	Bảng thống kê cân nặng $50$ quả thanh long được lựa chọn ngẫu nhiên sau khi thu hoạch ở nông trường
	\begin{center}
		\begin{tabular}{|c|c|c|c|c|c|}
			\hline
			Cân nặng (gam) & $[250 ; 290)$ & $[290 ; 330)$ & $[330 ; 370)$ & $[370 ; 410)$ & $[410 ; 450)$ \\
			\hline
			Số quả thanh long & $3$ & $13$ & $18$ & $11$ & $5$ \\
			\hline
		\end{tabular}
	\end{center}
	Khoảng tứ phân vị của mẫu số liệu ghép nhóm là (làm tròn kết quả đến hàng phần mười)
	\choice
	{\True $63{,}5$}
	{$65{,}3$}
	{$382{,}7$}
	{$319{,}2$}
	\loigiai{
		Ta có 
		\begin{itemize}
			\item Vì $\dfrac{n}{4}=12{,}5$ nên $Q_1\in (290;330]$.\\
			Suy ra $Q_1=290+\dfrac{\dfrac{50}{4}-13}{3}\cdot(330-290)=319{,}32$.
			\item Vì $\dfrac{3n}{4}=37{,}5$ nên $Q_3\in (370;410]$.\\
			Suy ra $Q_3=370+\dfrac{\dfrac{3\cdot50}{4}-13}{3}\cdot(410-370)=382{,}73$
		\end{itemize}
		Vậy $\Delta_Q=Q_3-Q_1 \approx 63{,}5$.}
\end{ex}
\begin{ex}%[2D3N2-2]%[Dự án đề cương 3 khối NH24-25 - Đợt 3 - Lê Phúc]
	Cho mẫu số liệu ghép nhóm có phương sai bằng $32$. Độ lệch chuẩn của mẫu số liệu trên là
	\choice
	{$16$}
	{$4\sqrt{3}$}
	{$1024$}
	{\True $4\sqrt{2}$}
	\loigiai{
		Mẫu số liệu ghép nhóm có phương sai bằng $32$ suy ra độ lệch chuẩn của mẫu số liệu trên là $\sqrt{32}=4\sqrt{2}$.
	}
\end{ex}
\begin{ex}%[2D3N1-2]%[Dự án đề cương 3 khối NH24-25 - Đợt 3 - Lê Phúc]
	Một công ty cung cấp nước sạch thống kê lượng nước các hộ gia đình tổng một khu vực tiêu thụ trong một tháng ở bảng sau
	\setlength{\extrarowheight}{0.2cm}
	\begin{longtable}{|c|c|c|c|c|c|}
		\hline
		Lượng nước tiêu thụ (m$^3$)&$[3;6)$&$[6;9)$&$[9;12)$&$[12;15)$&$[15;18)$  \\
		\hline
		Số hộ gia đình&$20$&$60$&$40$&$32$&$7$  \\
		\hline
	\end{longtable}
	Khoảng biến thiên của mẫu số liệu ghép nhóm trên là
	\choice
	{$3$ m$^3$}
	{$20$ m$^3$}
	{\True $15$ m$^3$}
	{$18$ m$^3$}
	\loigiai{
		Khoảng biến thiên của mẫu số liệu là $R=18-3=15$ m$^3$.}
\end{ex}
\begin{ex}%[2D3H2-2]%[Dự án đề cương 3 khối NH24-25 - Đợt 3 - Lê Phúc]
	Mỗi ngày bác Hương đều đi bộ để rèn luyện sức khỏe. Quãng đường đi bộ mỗi ngày (đơn vị: km) của bác Hương trong $20$ ngày được thống kê lại ở bảng sau
	\begin{center}
		\begin{tabular}{|c|c|c|c|c|c|}
			\hline
			\textbf{Quãng đường (km)} & $[2,7;3,0)$ & $[3,0;3,3)$ & $[3,3;3,6)$ & $[3,6;3,9)$ & $[3,9;4,2)$ \\
			\hline
			\textbf{Số ngày} & $3$ & $6$ & $5$ & $4$ & $2$ \\
			\hline
		\end{tabular}
	\end{center}
	Độ lệch chuẩn của mẫu số liệu ghép nhóm có giá trị gần nhất với giá trị nào dưới đây?
	\choice
	{\True $0{,}36$}
	{$11{,}62$}
	{$0{,}017$}
	{$3{,}41$}
	\loigiai{
		Bảng giá trị đại diện
		\begin{center}
			\begin{tabular}{|c|c|c|c|c|c|}
				\hline
				Nhóm & $[2{,}7;3{,}0)$ & $[3{,}0;3{,}3)$ & $[3{,}3;3{,}6)$ & $[3{,}6;3{,}9)$ & $[3{,}9;4{,}2)$ \\
				\hline
				Giá trị đại diện & $2{,}85$ & $3{,}15$ & $3{,}45$ & $3{,}75$ & $4{,}05$ \\
				\hline
				Tần số & $3$ & $6$ & $5$ & $4$ & $2$ \\
				\hline
			\end{tabular}
		\end{center}
		Cỡ mẫu là $n = 20$.\\
		Số trung bình là $\overline{x} = \dfrac{3\cdot2{,}85 + 6\cdot3{,}15 + 5\cdot3{,}45 + 4\cdot3{,}75 + 2\cdot4{,}05}{20}=3{,}39$.\\
		Phương sai là $s^2= \dfrac{3\cdot\left(2{,}85-3{,}39\right)^2 + \cdots + 2\cdot\left(4{,}05-3{,}39\right)^2}{20}=0{,}1314$.\\
		Độ lệch chuẩn là $s = \sqrt{s^2} = \sqrt{0{,}1314} \approx 0{,}36$.
	}
\end{ex}
\begin{ex}%[2D3N1-3]%[Dự án đề cương 3 khối NH24-25 - Đợt 3 - Lê Phúc]
	Khi thống kê điểm kiểm tra học kì $1$ môn Toán khối $12$ ở một trường phổ thông, người ta tổng hợp kết quả bằng một mẫu số liệu ghép nhóm. Mẫu số liệu ghép nhóm đó có tứ phân vị thứ nhất, tứ phân vị thứ hai, tứ phân vị thứ ba lần lượt là $4{,}0$ ; $5{,}5$ và $7{,}0$. Khoảng tứ phân vị của mẫu số liệu ghép nhóm đó bằng bao nhiêu?
	\choice
	{$2{,}5$}
	{\True $3$}
	{$5{,}5$}
	{$7$}
	\loigiai{
		Khoảng tứ phân vị của mẫu số liệu ghép nhóm là $\Delta_Q = Q_3 - Q_1=7{,}0-4{,}0=3$.
	}
\end{ex}
\begin{ex}%[2D3H2-2]%[Dự án đề cương 3 khối NH24-25 - Đợt 3 - Lê Phúc]
	Bảng sau thống kê số lượt khách mỗi ngày của một hãng taxi trong $30$ ngày
	\begin{center}
		\begin{tabular}{|l|c|c|c|c|c|}
			\hline
			Số lượt & {$[5; 8)$} & $(8;11)$ & $11; 14)$ & $14; 17)$ & $(17;20)$ \\
			\hline
			Số ngày & 4 & 9 & 9 & 5 & 3\\
			\hline
		\end{tabular}
	\end{center}
	Độ lệch chuẩn $s$ của mẫu số liệu trên thỏa
	\choice
	{$4< s< 5$}
	{$2< s< 3$}
	{\True $3< s< 4$}
	{$1< s< 2$}
	\loigiai{
		\begin{center}
			\begin{tabular}{|l|c|c|c|c|c|}
				\hline
				Số lượt & $[5; 8)$ & $(8;11)$ & $11; 14)$ & $14; 17)$ & $(17;20)$ \\
				\hline
				Giá trị đại diện & $6{,}5$ & $9{,}5$ & $12{,}5$ & $15{,}5$ & $18{,}5$ \\
				\hline
				Số ngày & 4 & 9 & 9 & 5 & 3\\
				\hline
			\end{tabular}
		\end{center}
		Số khách trung bình trong $30$ ngày
		\begin{align*}
			\overline{x}=\dfrac{1}{30}\left(6{,}5\cdot 4+9{,}5\cdot 9+12{,}5\cdot 9+15{,}5\cdot5+18{,}5\cdot 3\right)=11{,}9\text{ (khách)}.
		\end{align*}
		Phương sai của mẫu số liệu trên là
		\begin{align*}
			s^2=&\dfrac{1}{30}[(6{,}5-11{,}9)^2\cdot 4+(9{,}5-11{,}9)^2\cdot 9+(12{,}5-11{,}9)^2\cdot 9+\\&(15{,}5-11{,}9)^2\cdot5+(18{,}5-11{,}9)^2\cdot 3]=12{,}24.
		\end{align*}
		Vậy độ lệch chuẩn $s$ của mẫu số liệu trên là $S\approx3{,}5$.
	}
\end{ex}
\begin{ex}%[2D3N1-2]%[Dự án đề cương 3 khối NH24-25 - Đợt 3 - Lê Phúc]
	(\textit{Trích đề thi thử cụm các trường THPT Tuyên Quang - Vĩnh Phúc})\\
	Một trường THPT tổ chức khám sức khỏe miễn phí cho học sinh khối $12$, kết quả đo chiều cao của các nam sinh được cho trong bảng sau
	\begin{center}
		\begin{tabular}{|c|c|c|c|c|c|c|}
			\hline
			Chiều cao (cm) & {$[155; 160)$} & {$[160; 165)$} & {$[165; 170)$} & {$[170; 175)$} & {$[175; 180)$} & {$[180;185)$}\\
			\hline
			Số học sinh nam & $45$ & $78$ & $120$ & $45$ & $12$ &$12$\\
			\hline
		\end{tabular}
	\end{center}
	Khoảng biến thiên $R$ của mẫu số liệu ghép nhóm trên là
	\choice
	{\True $R=30$}
	{$R=108$}
	{$R=40$}
	{$R=5$}
	\loigiai{
		Khoảng biến thiên của mẫu số liệu ghép nhóm trên là $R=185-155=30$.
	}
\end{ex}
\begin{ex}%[2D3H2-2]%[Dự án đề cương 3 khối NH24-25 - Đợt 3 - Lê Phúc]
	Một siêu thị thống kê số tiền (đơn vị: chục nghìn đồng) mà $44$ khách hàng mua hàng ở siêu thị đó trong một ngày. Số liệu được ghi lại trong Bảng sau.
	\begin{center}
		\begin{tabular}{|c|c|c|}
			\hline
			Nhóm & Giá trị đại diện & Tần số \\
			\hline
			$[40;45)$ & $42{,}5$ & $4$ \\
			\hline
			$[45;50)$ & $47{,}5$ & $14$ \\
			\hline
			$[50;55)$ & $52{,}5$ & $8$ \\
			\hline
			$[55;60)$ & $57{,}5$ & $10$ \\
			\hline
			$[60;65)$ & $62{,}5$ & $6$ \\
			\hline
			$[65;70)$ & $67{,}5$ & $2$ \\
			\hline
			& & $n=44$ \\
			\hline
		\end{tabular}
	\end{center}
	Phương sai của mẫu số liệu ghép nhóm trên là (kết quả làm tròn đến hàng phần mười)
	\choice
	{$6{,}8$}
	{$53{,}2$}
	{\True $46{,}1$}
	{$47{,}2$}
	\loigiai{
		Ta có \[\overline{x}=\dfrac{42{,}5\cdot 4+47{,}5\cdot 14+52{,}5\cdot 8+57{,}5\cdot 10+62{,}5\cdot 6+67{,}5\cdot 2}{44}=\dfrac{585}{11}.\]
	Suy ra
	\[s^2=\dfrac{(42{,}5-\overline{x})\cdot 4+(47{,}5-\overline{x})\cdot 14+(52{,}5-\overline{x})\cdot 8+(57{,}5-\overline{x})\cdot 10+(62{,}5-\overline{x})\cdot 6+(67{,}5-\overline{x})\cdot 2}{44}\approx 46{,}1.\]
	}
\end{ex}
\begin{ex}%[2D3H1-3]%[Dự án đề cương 3 khối NH24-25 - Đợt 3 - Lê Phúc]
	(\textit{Trích đề thi thử chuyên Đại học khoa học Huế})\\
	Mẫu số liệu ghép nhóm thống kê mức lương của một công ty (đơn vị: triệu đồng) được cho trong bảng dưới đây
	\begin{center}
		\begin{tabular}{|c|c|c|c|c|c|c|}
			\hline
			\begin{tabular}{c} Nhóm \\
				(đơn vị: triệu đồng)
			\end{tabular} & {$[6; 8)$} & {$[8;10)$} & {$[10;12)$} & {$[12; 14)$} & {$[14; 16)$}&  \\
			\hline
			Tần số & $6$ & $14$ & $16$ & $12$ & $2$& $n=50$ \\
			\hline
		\end{tabular}
	\end{center}
	Tìm khoảng tứ phân vị của mẫu số liệu ghép nhóm (\textit{kết quả làm tròn đến hàng phần trăm}).
	\choice
	{$3{,}15$}
	{$3{,}16$}
	{\True $3{,}32$}
	{$3{,}34$}
	\loigiai{
		Sắp xếp mẫu số liệu theo thứ tự không giảm $x_1$, $x_2$, $\ldots$, $x_{50}$.\\
		Tứ phân vị thứ nhất $Q_1=\dfrac{x_{12}+x_{13}}{2}\in[8;10]$.\\
		Ta có $Q_1=8+\dfrac{\dfrac{50}{4}-6}{14}\cdot(10-8)=\dfrac{125}{14}$.\\
		Tứ phân vị thứ ba $Q_3=\dfrac{x_{38}+x_{39}}{2}\in[12;14]$.\\
		Ta có $Q_3=12+\dfrac{3\cdot\dfrac{50}{4}-(6+14+16)}{12}\cdot(14-12)=\dfrac{49}{4}$.\\
		Khoảng tứ phân vị của mẫu số liệu ghép nhóm đã cho là $\Delta_Q=Q_3-Q_1=\dfrac{93}{28}\approx 3{,}32$.
	}
\end{ex}
\begin{ex}%[2D3H2-2]%[Dự án đề cương 3 khối NH24-25 - Đợt 2 - Lê Phúc]
	Thời gian chạy $100$ m (đơn vị: giây) của $40$ học sinh lớp $12$ được cho trong bảng sau
	\begin{center}
		\begin{tabular}{|c|c|}
			\hline
			Nhóm & Tần số \\
			\hline
			$[11;12)$ & $4$\\
			\hline
			$[12;13)$ & $12$\\
			\hline
			$[13;14)$ & $16$\\
			\hline
			$[14;15)$ & $5$\\
			\hline
			$[15;16)$ & $3$\\
			\hline
		\end{tabular}
	\end{center}
	Giá trị trung bình của mẫu số liệu trên (làm tròn đến chữ số thập phân thứ hai) là
	\choice
	{$13{,}76$}
	{$13{,}75$}
	{\True $13{,}28$}
	{$13{,}88$}
	\loigiai{
		Ta có bảng tần số ghép nhóm với giá trị đại diện của mẫu số liệu là
		\begin{center}
			\begin{tabular}{|c|c|c|}
				\hline
				Nhóm & Giá trị đại diện & Tần số \\
				\hline
				$[11;12)$ & $11{,}5$ & $4$\\
				\hline
				$[12;13)$ & $12{,}5$ & $12$\\
				\hline
				$[13;14)$ & $13{,}5$ & $16$\\
				\hline
				$[14;15)$ & $14{,}5$ & $5$\\
				\hline
				$[15;16)$ & $15{,}5$ & $3$\\
				\hline
			\end{tabular}
		\end{center}
		Giá trị trung bình của mẫu số liệu trên là
		$$ \overline{x} = \dfrac{4\cdot 11{,}5 + 12\cdot 12{,}5 + 16\cdot 13{,}5 + 5\cdot 14{,}5 + 3\cdot 15{,}5}{4 + 12 + 16 + 5 + 3} = \dfrac{531}{40} \approx 13{,}28. $$
	}
\end{ex}
\Closesolutionfile{ans}
%\begin{center}
%	\textbf{ĐÁP ÁN}
%	\inputansbox{10}{ans/ans}	
%\end{center}
\begin{center}
	\textbf{PHẦN 2 - CÂU TRẮC NGHIỆM ĐÚNG SAI}
\end{center}
\Opensolutionfile{ans}[ans/answer-DS-ONTAPCHUONG-DE1]
\setcounter{ex}{0}
\begin{ex}%[2D3H1-3]%[Dự án đề cương 3 khối NH24-25 - Đợt 2 - Lê Phúc]
	(\textit{Trích đề thi thử Chuyên Vinh lần 2 năm học 2024-2025})\\
	Khảo sát chiều cao của $20$ học sinh nam lớp $12$A của một trường THPT X, người ta được kết quả thống kê trong bảng sau
		\begin{center}
		\begin{tabular}{|c|c|c|c|c|c|}
			\hline
			Chiều cao (cm) & {$[160;165)$} & {$[165;170)$} & $[170;175)$ & $[175;180)$ & $[180;185)$ \\
			\hline
			Số học sinh & $3$ & $5$ & $7$ & $4$ & $1$ \\
			\hline
		\end{tabular}
	\end{center}
	\choiceTF
	{Gọi $x_1$, $x_2$,$\ldots$, $x_{20}$ là mẫu số liệu gốc gồm chiều cao của $20$ học sinh trên được xếp theo thứ tự không giảm. Khi đó $x_3\in [165;170)$ và $x_9\in [170;175)$}
	{\True Tứ phân vị thứ ba của mẫu số liệu ghép nhóm đã cho bằng $175$}
	{\True Khoảng tứ phân vị của mẫu số liệu ghép nhóm đã cho bằng $175$}
	{Bạn An là một học sinh nam của lớp $12$A, An có chiều cao $182$ cm. Chiều cao của An là một giá trị ngoại lệ (giá trị bất thường) trong mẫu số liệu đã cho}
	\loigiai{
		\begin{itemchoice}
			\itemch Ta có $x_3\in [160;165)$ và $x_9\in[170;175)$.
			\itemch Tứ phân vị thứ ba thuộc nhóm $[170;175)$.\\
			Suy ra $Q_3=170+\dfrac{\dfrac{3\cdot20}{4}-(3+5)}{7}\cdot 5=175$.
			\itemch Tứ phân vị thứ nhất thuộc nhóm $[165;170)$.\\
			Suy ra $Q_1=165+\dfrac{\dfrac{20}{4}-3}{5}\cdot 5=167$.\\
			Vậy khoảng tứ phân vị là $\Delta_Q=Q_3-Q_1=8$.
			\itemch Giả sử chiều cao $x$ có giá trị ngoại lệ.\\
			Suy ra $x\le Q_1-1,5\cdot\Delta_Q$ hoặc $x\ge Q_3+1,5\cdot\Delta_Q$.\\
			Do đó $\hoac{&x\le 155\\&x\ge 187.}$\\
			Vậy chiều cao của An là $x=182$ không là giá trị ngoại lệ.
		\end{itemchoice}
	}
\end{ex}
\begin{ex}%[2D3V2-2]%[Dự án đề cương 3 khối NH24-25 - Đợt 2 - Lê Phúc]
		(\textit{Trích đề thi học sinh giỏi Thanh Hóa năm học 2024-2025})\\
	Tại một trường THPT, để khảo sát năng lực học môn Toán của hai lớp $12$E và $12$F, giáo viên đã cho học sinh ở hai lớp làm bài kiểm tra khảo sát đầu năm, thống kê điểm của học sinh được cho trong bảng sau
	\begin{center}
		\begin{tabular}{|w{c}{4cm}|w{c}{1.5cm}|w{c}{1.5cm}|w{c}{1.5cm}|w{c}{1.5cm}|w{c}{1.5cm}|}
			\hline Lớp điểm thi & {$[0;2)$} & {$[2;4)$} & {$[4;6)$} & {$[6;8)$} & {$[8;10)$} \\
			\hline Số học sinh lớp 12E & $3$ & $6$ & $12$ & $24$ & $5$ \\
			\hline Số học sinh lớp 12F & $4$ & $5$ & $16$ & $18$ & $7$ \\
			\hline
		\end{tabular}
	\end{center}
	\choiceTF
	{\True Dựa vào điểm trung bình môn Toán ta đánh giá được lớp $12$E học tốt môn Toán hơn lớp $12$F}
	{Khoảng tứ phân vị của mẫu số liệu thống kê điểm của lớp $12$F nhỏ hơn $2{,}9$}
	{Điểm thi có số học sinh đạt được nhiều nhất ở lớp $12$E nhỏ hơn ở lớp $12$F}
	{\True Dựa vào độ lệch chuẩn của mẫu số liệu thống kê ghép nhóm, ta thấy rằng lớp $12$E học đều hơn lớp $12$F}
	\loigiai{
		Ta có bảng thống kê giá đóng cửa theo giá trị đại diện
		\begin{center}
		\begin{tabular}{|w{c}{4cm}|w{c}{1.5cm}|w{c}{1.5cm}|w{c}{1.5cm}|w{c}{1.5cm}|w{c}{1.5cm}|}
			\hline Giá trị đại diện & {$1$} & {$3$} & {$5$} & {$7$} & {$9$} \\
			\hline Số học sinh lớp $12$E & $3$ & $6$ & $12$ & $24$ & $5$ \\
			\hline Số học sinh lớp $12$F & $4$ & $5$ & $16$ & $18$ & $7$ \\
			\hline
		\end{tabular}
	\end{center}
		\begin{itemchoice}
			\itemch 
			\begin{itemize}
				\item Điểm trung bình của lớp $12$E
				\[\overline{x}_E=\dfrac{1}{50}(3\cdot 1+6\cdot 3+12\cdot 5+24\cdot 7+5\cdot 9)=5{,}88.\]
				\item Điểm trung bình của lớp $12$F
				\[\overline{x}_F=\dfrac{1}{50}(4\cdot 1+5\cdot 3+16\cdot 5+18\cdot 7+7\cdot 9)=5{,}76.\]
			\end{itemize}
			Suy ra $\overline{x}_E>\overline{x}_F$.\\
			Do đó dựa vào điểm trung bình ta thấy học sinh lớp $12$E học tốt hơn học sinh lớp $12$F.
			\itemch 
			\begin{itemize}
				\item Nhóm chứa tứ phân vị thứ nhất là nhóm $[4;6)$, do đó
				\[Q_1=4+\dfrac{\dfrac{50}{4}-9}{16}\cdot(6-4)=\dfrac{71}{16}.\]
				\item Nhóm chứa tứ phân vị thứ ba là nhóm $[6;8)$, do đó
				\[Q_3=6+\dfrac{\dfrac{3\cdot50}{4}-25}{18}\cdot(8-6)=\dfrac{133}{18}.\]
			\end{itemize}
			Khoảng tứ phân vị là $\Delta_Q=Q_3-Q_1=\dfrac{425}{144}\approx 2{,}95$.
			\itemch Điểm học sinh đạt được nhiều nhất ở mỗi lớp là mốt của mẫu số liệu ghép nhóm của mỗi lớp được cho bảng trên.
			\begin{itemize}
				\item Nhóm chứa mốt của mẫu số liệu thống kê ghép nhóm ở lớp $12$E là nhóm $[6;8)$  
			\[M_0=6+\dfrac{24-12}{24-12+24-5}\cdot(8-6)=\dfrac{21}{31}\approx 6{,}77.\]
			\item Nhóm chứa mốt của mẫu số liệu thống kê ghép nhóm ở lớp $12$F là nhóm $[6;8)$ 
			\[M_0=6+\dfrac{18-16}{18-16+18-7}\cdot(8-6)=\dfrac{82}{13}\approx 6{,}31.\]
			\end{itemize}
			\itemch
			\begin{itemize}
				\item Độ lệch chuẩn của mẫu số liệu ghép nhóm của lớp $12$E
				\[s_E=\sqrt{s_E^2}=\sqrt{\dfrac{1}{50}\left(3\cdot1^2+6\cdot3^2+12\cdot5^2+24\cdot7^2+5\cdot9^2\right)-5{,}88^2}\approx2{,}05.\]
			\item Độ lệch chuẩn của mẫu số liệu ghép nhóm của lớp $12$F
			\[s_F=\sqrt{s_F^2}=\sqrt{\dfrac{1}{50}\left(4\cdot1^2+5\cdot3^2+16\cdot5^2+18\cdot7^2+7\cdot9^2\right)-5{,}76^2}\approx2{,}19.\]
			\end{itemize}
		Ta thấy độ lệch chuẩn ở mẫu số liệu lớp $12$E nhỏ hơn độ lệch chuẩn của mẫu số liệu lớp $12$F. Do đó lớp $12$E học đều môn Toán hơn lớp $12$F.
		\end{itemchoice}
	}
\end{ex}
\Closesolutionfile{ans}
%\inputansbox[2]{2}{ans/answer.tex}
\begin{center}
\textbf{PHẦN 3 - CÂU TRẮC NGHIỆM TRẢ LỜI NGẮN}
\end{center}
\setcounter{ex}{0}
\Opensolutionfile{ans}[ans-KQ-ONTAPCHUONG-DE1]
\begin{ex}%[2D3H1-3]%[Dự án đề cương 3 khối NH24-25 - Đợt 2 - Lê Phúc]
	Thống kê số lượt xem mỗi video trong $35$ video của một bạn Youtuber mới lập kênh được một tháng
	\begin{center}
		\begin{tabular}{|c|c|c|c|c|c|}
			\hline
			Số lượt xem & {$[200;500)$} & {$[500;500)$} & $[500;1100)$ & $[1100;1400)$ & $[1400;1700)$ \\
			\hline
			Số video & 4 & 7 & 2 & 15 & 7 \\
			\hline
		\end{tabular}
	\end{center}
	Tứ phân vị thứ ba của mẫu số liệu ghép nhóm trên bằng bao nhiêu?
	\shortans{$1365$}
	\loigiai{
		Vì $\dfrac{3n}{4}=\dfrac{3\cdot35}{4}=26{,}25$ nên $Q_{3}=x_{27} \in [1100; 1400)$.\\
		Suy ra $Q_{3}=1100+\dfrac{\dfrac{3\cdot 35}{4}-13}{15}\cdot 300= 1365$.	
	}
\end{ex}
\begin{ex}%[2D3H2-2]%[Dự án đề cương 3 khối NH24-25 - Đợt 2 - Lê Phúc]
	Cự li cú nhảy $3$ bước của $40$ học sinh lớp $12$ được ghi lại ở bảng tần số ghép nhóm sau
	\begin{center}
		\begin{tabular}{|c|c|c|c|c|c|}
			\hline
			Độ dài (m) & {$[9; 10)$} & {$[10; 11)$} & {$[11; 12)$} & {$[12; 13)$} & {$[13; 14)$} \\
			\hline
			Tần số & 18 & 10 & 6 & 4 & 2 \\
			\hline
		\end{tabular}
	\end{center}
	Tính độ lệch chuẩn của mẫu số liệu ghép nhóm trên (kết quả làm tròn đến hàng phần mười).
	\shortans{$1,2$}
	\loigiai{
		Ta có bảng thống kê cự li cú nhảy của các học sinh theo giá trị đại diện
		\begin{center}
			\begin{tabular}{|c|c|c|c|c|c|}
				\hline Độ dài (m) & {$[9;10)$} & {$[10;11)$} & {$[11;12)$} & {$[12;13)$} & {$[13;14)$} \\
				\hline Độ dài đại diện & 9,5 & 10,5 & 11,5 & 12,5 & 13,5 \\
				\hline Tần số & 18 & 10 & 6 & 4 & 2 \\
				\hline
			\end{tabular}
		\end{center}
		Cỡ mẫu $n=40$.
		Số trung bình của mẫu số liệu ghép nhóm là
		\[\overline{x}=\dfrac{18 \cdot 9{,}5+10 \cdot 10{,}5+6 \cdot 11{,}5+4 \cdot 12{,}5+2 \cdot 13{,}5}{40}=10{,}55.\]
		Phương sai của mẫu số liệu ghép nhóm là
		\[s^2=\dfrac{1}{40}\left(18 \cdot 9{,}5^2+10 \cdot 10{,}5^2+6 \cdot 11{,}5^2+4 \cdot 12{,}5^2+2 \cdot 13{,}5^2\right)-10{,}55^2=1{,}4475.\]
		Độ lệch chuẩn của mẫu số liệu ghép nhóm là $s=\sqrt{1{,}4475} \approx 1{,}2$.	
	}
\end{ex}
\begin{ex}%[2D3V2-2]%[Dự án đề cương 3 khối NH24-25 - Đợt 2 - Lê Phúc]
	Biểu đồ sau mô tả kết quả điều tra về mức tiêu thụ nước hàng tháng của hai khu dân cư $A$ và $B$.
	\begin{center}
		\begin{tikzpicture}[scale=0.9,>=stealth]
			% Axes
			\draw [->] (0,0) -- (11.5,0) node [right] {Mức tiêu thụ ($m^3$)};
			\draw [->] (0,0) -- (0,6.5) node [above] {Số hộ dân};
			% Y ticks
			\foreach \y in {1,2,3,4,5,6} {
				\draw (0.1,\y) -- (-0.1,\y) node [left] {$\y$};
			}
			\node [left] at (0,0) {$0$};
			
			% Bars and labels
			% Group 1: [10; 20)
			\draw[fill=cyan] (0.6,0) rectangle (1.4,4); \node[above] at (1,4) {$4$};
			\draw[pattern=north east lines, pattern color=black] (1.4,0) rectangle (2.2,2); \node[above] at (1.8,2) {$2$};
			\node[below] at (1.4,-0.1) {$[10; 20)$};
			
			% Group 2: [20; 30)
			\draw[fill=cyan] (2.6,0) rectangle (3.4,5); \node[above] at (3,5) {$5$};
			\draw[pattern=north east lines, pattern color=black] (3.4,0) rectangle (4.2,5); \node[above] at (3.8,5) {$5$};
			\node[below] at (3.4,-0.1) {$[20; 30)$};
			
			% Group 3: [30; 40)
			\draw[fill=cyan] (4.6,0) rectangle (5.4,3); \node[above] at (5,3) {$3$};
			\draw[pattern=north east lines, pattern color=black] (5.4,0) rectangle (6.2,4); \node[above] at (5.8,4) {$4$};
			\node[below] at (5.4,-0.1) {$[30; 40)$};
			
			% Group 4: [40; 50)
			\draw[fill=cyan] (6.6,0) rectangle (7.4,4); \node[above] at (7,4) {$4$};
			\draw[pattern=north east lines, pattern color=black] (7.4,0) rectangle (8.2,3); \node[above] at (7.8,3) {$3$};
			\node[below] at (7.4,-0.1) {$[40; 50)$};
			
			% Group 5: [50; 60)
			\draw[fill=cyan] (8.6,0) rectangle (9.4,2); \node[above] at (9,2) {$2$};
			\draw[pattern=north east lines, pattern color=black] (9.4,0) rectangle (10.2,1); \node[above] at (9.8,1) {$1$};
			\node[below] at (9.4,-0.1) {$[50; 60)$};
			
			% Title
			\node[above,xshift=1cm] at (5.5, 6.5) {\textbf{MỨC TIÊU THỤ NƯỚC CỦA HAI KHU DÂN}};
			\node[above,xshift=1cm] at (5.5, 6.0) {\textbf{CƯ A VÀ B}};
			
			% Legend
			\draw[fill=cyan,yshift=-0.5cm,xshift=-2cm] (2,-1) rectangle (2.5,-0.7); \node[right,yshift=-0.5cm,xshift=-2cm] at (2.7,-0.85) {Số hộ dân khu dân cư A};
			\draw[pattern=north east lines, pattern color=black,yshift=-0.5cm] (6,-1) rectangle (6.5,-0.7); \node[right,yshift=-0.5cm] at (6.7,-0.85) {Số hộ dân khu dân cư B};
			
		\end{tikzpicture}
	\end{center}
	Gọi $S_A$; $S_B$ lần lượt là độ lệch chuẩn của mẫu số liệu ghép nhóm của khu dân cư $A$ và khu dân cư $B$. Để so sánh khu dân cư nào có mức tiêu thụ nước đồng đều hơn ta tính $\Delta_S = S_A - S_B$. Nếu $\Delta_S < 0$ thì mức tiêu thụ nước của khu dân cư $A$ đồng đều hơn, nếu $\Delta_S > 0$ thì mức tiêu thụ nước của khu dân cư $B$ đồng đều hơn. Tính $\Delta_S$ (kết quả làm tròn tới hàng phần trăm)\shortans{$2,02$}
	\loigiai{
		Ta có bảng mô tả mức tiêu thụ của khu dân cư A
		\begin{center}
			\begin{tabular}{|c|c|c|c|c|c|}
				\hline
				Mức tiêu thụ & [10;20) & [20;30) & [30;40) & [40;50) & [50;60) \\
				\hline
				Giá trị đại diện & 15 & 25 & 35 & 45 & 55 \\
				\hline
				Số hộ tiêu thụ & 4 & 5 & 3 & 4 & 2 \\
				\hline
			\end{tabular}
		\end{center}
		Giá trị trung bình $\overline{x}_A=\dfrac{290}{9}$.\\
		Phương sai $S^2_A=\dfrac{1}{18}\cdot (4\cdot 15^2+5\cdot 25^2+3\cdot 35^2+4\cdot 45^2+2\cdot 55^2)-\left(\dfrac{290}{9}\right)^2=\dfrac{14225}{81}$.\\
		Độ lệch chuẩn $S_A=\dfrac{5\sqrt{569}}{9}$.
		\begin{center}
			\begin{tabular}{|c|c|c|c|c|c|}
				\hline
				Mức tiêu thụ & [10;20) & [20;30) & [30;40) & [40;50) & [50;60) \\
				\hline
				Giá trị đại diện & 15 & 25 & 35 & 45 & 55 \\
				\hline
				Số hộ tiêu thụ & 2 & 5 & 4 & 3 & 1 \\
				\hline
			\end{tabular}
		\end{center}
		Giá trị trung bình $\overline{x}_B=\dfrac{97}{3}$.\\
		Phương sai $S^2_B=\dfrac{1}{15}\cdot (2\cdot 15^2+5\cdot 25^2+4\cdot 35^2+3\cdot 45^2+1\cdot 55^2)-\left(\dfrac{97}{3}\right)^2=\dfrac{1136}{9}$.\\
		Độ lệch chuẩn $S_B=\dfrac{4\sqrt{71}}{3}$.\\
		Vậy $\Delta_S =S_A-S_B=\dfrac{5\sqrt{569}}{9}-\dfrac{4\sqrt{71}}{3}\approx 2,02$.
	}
\end{ex}
\begin{ex}%[2D3V1-4]%[Dự án đề cương 3 khối NH24-25 - Đợt 2 - Lê Phúc]
	Một siêu thị thống kê số tiền (đơn vị: chục nghìn đồng) mà $44$ khách hàng mua hàng ở siêu thị đó trong một ngày. Số liệu được ghi lại trong bảng sau:
	\begin{center}
		\begin{tabular}{|c|c|c|c|c|c|c|c|}
			\hline 
			Nhóm & $[40;45)$ & $[45;50)$ & $[50;55)$ & $[55;60)$ & $[60;65)$ & $[65;70)$ &  \\ 
			\hline
			Tần số & $4$ & $14$ & $8$ & $10$ & $6$ & $2$ & $n=44$ \\
			\hline
		\end{tabular}
	\end{center}
	Hiệu giữa khoảng biến thiên và khoảng tứ phân vị của bảng số liệu trên bằng bao nhiêu?
	\shortans{19}
	\loigiai{
		Ta có $n=44$.\\
		Khoảng biến thiên của mẫu số liệu là $R=70-40=30$.\\
		Gọi $x_1$, $x_2$,$\cdots$, $x_{44}$ lần lượt là số tiền (đơn vị: chục nghìn đồng) của $44$ khách hàng theo thứ tự không giảm.\\
		Tứ phân vị thứ nhất của dãy số liệu là $\dfrac{1}{2}\left(x_{11}+x_{12}\right)$ thuộc nhóm $[45;50)$ nên tứ phân vị thứ nhất của mẫu số liệu là
		\[
		Q_1 = 45+\dfrac{\dfrac{44}{4}-4}{14}\cdot (50-45) = 47{,}5.
		\]
		Tứ phân vị thứ ba của dãy số liệu là $\dfrac{1}{2}\left(x_{33}+x_{34}\right)$ thuộc nhóm $[55;60)$ nên tứ phân vị thứ ba của mẫu số liệu là
		\[
		Q_3=55+\dfrac{\dfrac{3\cdot 44}{4}-(4+14+8)}{10}\cdot (65-60) = 58{,}5.
		\]
		Khi đó khoảng tứ phân vị là $\Delta_Q = Q_3 - Q_1 = 11$.\\
		Vậy, kết quả cần tìm là $R-\Delta_Q = 30-11=19$.
	}
\end{ex}
\Closesolutionfile{ans}
\begin{center}
	\textbf{PHẦN 4 - TỰ LUẬN}
\end{center}
\setcounter{ex}{0}
\begin{ex}%[2D3H2-2]%[Dự án đề cương 3 khối NH24-25 - Đợt 2 - Lê Phúc]
	Thống kê mật độ dân số (đơn vị: người/km$^2$) của $23$ tỉnh, thành phố thuộc vùng Trung du và miền núi phía Bắc, Đồng bằng sông Hồng (không kể thành phố Hà Nội và tỉnh Bắc Ninh) trong năm $2\,021$.
	\begin{flushright}
		(Nguồn: Niên giám Thống kê $2\,021$, NXB Thống kê, $2\,022$).
	\end{flushright}
	\begin{center}
		\begin{tabular}{|c|c|c|c|c|c|c|c|}
			\hline Mật độ& {$[0 ; 200)$} & {$[200 ; 400)$} & {$[400 ; 600)$} & {$[600 ; 800)$} & {$[800 ; 1\,000)$} & {$[1\,000 ; 1\,200)$} & {$[1\,200 ; 1\,400)$} \\
			\hline Tần số & $13$ & $2$ & $2$ & $0$ & $1$ & $3$ & $2$ \\
			\hline
		\end{tabular}	
	\end{center}
	Tính số trung bình của mẫu số liệu ghép nhóm (\textit{kết quả làm tròn đến hàng đơn vị}).
	\loigiai{
		Ta có\\
		\begin{tabular}{|c|c|c|c|c|c|c|c|}
			\hline Mật độ& {$[0 ; 200)$} & {$[200 ; 400)$} & {$[400 ; 600)$} & {$[600 ; 800)$} & {$[800 ; 1\,000)$} & {$[1\,000 ; 1\,200)$} & {$[1\,200 ; 1\,400)$} \\
			\hline Tần số & $13$ & $2$ & $2$ & $0$ & $1$ & $3$ & $2$ \\
			\hline GT đại diện & $100$ & $300$ & $500$ & $700$ & $900$ & $1\,100$ & $1\,300$ \\
			\hline
		\end{tabular}\\
		Số trung bình 
		\[\overline{x}=\dfrac{13\cdot 100+2\cdot 300+2\cdot 500+0\cdot 700+1\cdot 900+3\cdot 1\,100+2\cdot 1\,300}{23}=\dfrac{9\,700}{23} \approx 422.\]
	}
\end{ex}
\begin{ex}%[2D3V2-2]%[Dự án đề cương 3 khối NH24-25 - Đợt 2 - Lê Phúc]
	Bạn A và B cùng sử dụng vòng đeo tay thông minh để ghi lại số bước chân hai bạn đi mỗi ngày trong một tháng. Kết quả được ghi lại ở bảng sau
	\begin{center}
		\begin{tabular}{|l|c|c|c|c|c|}
			\hline Số bước (đơn vị: nghìn) & {$[3 ; 5)$} & {$[5 ; 7)$} & {$[7 ; 9)$} & {$[9 ; 11)$} & {$[11 ; 13)$} \\
			\hline Số ngày của A& $6$ & $7$ & $6$ & $6$ & $5$ \\
			\hline Số ngày của $B$ & $2$ & $5$ & $13$ & $8$ & $2$ \\
			\hline Giá trị đại diện & $4$ & $6$ & $8$ & $10$ & $12$ \\
			\hline
		\end{tabular}
	\end{center}
	Tổng độ lệch chuẩn của mẫu số liệu ghép nhóm của hai bạn A và B (làm tròn kết quả đến hàng phần trăm).
	\loigiai{
		Số bước chân trung bình của mẫu số liệu ghép nhóm của bạn A là
		$$\overline{x}_{A}=\dfrac{6\cdot4+7\cdot6+6\cdot8+6\cdot10+5\cdot12}{30}=\dfrac{39}{5}.$$
		Số bước chân trung bình của mẫu số liệu ghép nhóm của bạn B là
		$$\overline{x}_{A}=\dfrac{2\cdot4+5\cdot6+13\cdot8+8\cdot10+2\cdot12}{30}=\dfrac{41}{5}.$$
		Phương sai của mẫu số liệu ghép nhóm của bạn A là $$S^2_A=\dfrac{6\cdot4^2+7\cdot6^2+6\cdot8^2+6\cdot10^2+5\cdot12^2}{30}-\left(\dfrac{39}{5}\right)^2=\dfrac{767}{75}.$$
		Phương sai của mẫu số liệu ghép nhóm của bạn B là
		$$S^2_B=\dfrac{2\cdot4^2+5\cdot6^2+13\cdot8^2+8\cdot10^2+2\cdot12^2}{30}-\left(\dfrac{41}{5}\right)^2=\dfrac{287}{75}.$$
		Độ lệch chuẩn của mẫu số liệu ghép nhóm A là
		$$S_A=\sqrt{S^2_A}=\sqrt{\dfrac{767}{75}}.$$
		Độ lệch chuẩn của mẫu số liệu ghép nhóm B là
		$$S_B=\sqrt{S^2_B}=\sqrt{\dfrac{287}{75}}.$$
		Tổng độ lệch chuẩn của mẫu số liệu ghép nhóm của hai bạn A và B là $S_A+S_B \approx 5{,}15$.
	}
\end{ex}
\begin{ex}%[2D3V1-3]%[Dự án đề cương 3 khối NH24-25 - Đợt 2 - Lê Phúc]
	\immini{Biểu đồ hình bên thể hiện điểm trung bình môn Toán của $501$ học sinh khối $12$ một trường THPT. Khoảng tứ phân vị của mẫu số liệu ghép nhóm cho bởi biểu đồ này bằng bao nhiêu? (kết quả làm tròn đến hàng phần trăm).}{\begin{tikzpicture}[scale=0.7, font=\footnotesize, line join=round, line cap=round, >=stealth]
			% \draw[](6,9.6)node[scale=1.2]{Thời gian trong ngày của Nam };
			\coordinate [label= left: $0$](O) at (0,0) ;
			%\coordinate [label= left: (mm)](y) at (0,9.5) ;
			\coordinate [label= right: Số học sinh ](y) at (0,6) ;
			\coordinate [label= above: Điểm ](x) at (13.3,0) ;
			%\coordinate [label= below right: $M$](M) at ($(B)!0.5!(C)$) ;
			%		\draw[](0,8)node[left]{$40$};
			%		\draw[](0,7)node[left]{$35$};
			%		\draw[](0,6)node[left]{$30$};
			\draw[](0,5)node[left]{$250$};
			\draw[](0,4)node[left]{$200$};
			\draw[](0,3)node[left]{$150$};
			\draw[](0,2)node[left]{$100$};
			\draw[](0,1)node[left]{$50$};
			%
			\draw[-,color=gray!50] (0,1)--(12.5,1);
			\draw[-,color=gray!50] (0,2)--(12.5,2);
			\draw[-,color=gray!50] (0,3)--(12.5,3);
			\draw[-,color=gray!50] (0,4)--(12.5,4);
			\draw[-,color=gray!50] (0,5)--(12.5,5);
			%		\draw[-,color=gray!50] (0,6)--(12.5,6);
			%		\draw[-,color=gray!50] (0,7)--(11,7);
			%		\draw[-,color=gray!50] (0,8)--(11,8);
			%		\draw[-,color=gray!50] (0,9)--(11,9);
			%
			\draw[](1.5,0.5)node[above]{$25$};
			\draw[](3.5,1)node[above]{$50$};
			\draw[](5.5,2.03)node[above]{$102$};
			\draw[](7.5,4.03)node[above]{$202$};
			\draw[](9.5,2.07)node[above]{$112$};
			\draw[](11.5,0.2)node[above]{$10$};
			%
			\draw[](1.5,0)node[below]{[4;5)};
			\draw[](3.5,0)node[below]{[5;6)};
			\draw[](5.5,0)node[below]{[6;7)};
			\draw[](7.5,0)node[below]{[7;8)};
			\draw[](9.5,0)node[below]{[8;9)};
			\draw[](11.5,0)node[below]{[9;10)};
			
			\draw[fill=blue!80] (1,0) rectangle (2,0.5);
			\draw[fill=blue!80] (3,0) rectangle (4,1);
			\draw[fill=blue!80] (5,0) rectangle (6,2.03);
			\draw[fill=blue!80] (7,0) rectangle (8,4.03);
			\draw[fill=blue!80] (9,0) rectangle (10,2.07);
			\draw[fill=blue!80] (11,0) rectangle (12,0.2);
			\draw[->] (O)--(x);
			\draw[->] (O)--(y);
	\end{tikzpicture}}
	\loigiai{Ta có bảng tần số ghép nhóm
		\begin{center}
			\renewcommand{\arraystretch}{1.5}
			\begin{tabular}{|c|c|c|c|c|c|c|}
				\hline Số câu đúng &$[4; 5)$ &$[5; 6)$ &$[6; 7)$ &$[7; 8)$ &$[8; 9)$ &$[9; 10)$ \\
				\hline Số học sinh & 25 & 50 & 102 & 202 & 112 & 10 \\
				\hline
			\end{tabular}
		\end{center}
		Cỡ mẫu $n=501$.\\
		Gọi $x_1$; $x_2$; $x_3$; $ \cdots $; $x_{501}$ là mẫu số liệu gốc.\\
		Tứ phân vị thứ hai của mẫu số liệu gốc là $x_{251}\in \left[ 7;8\right)$.\\
		Tứ phân vị thứ nhất của mẫu số liệu gốc là $\dfrac{1}{2}\left( x_{125}+x_{126}\right) \in \left[ 6;7\right)$.\\
		Do đó tứ phân vị thứ nhất của mẫu số liệu ghép nhóm là \[Q_{1} = 6+ \dfrac{\dfrac{1\cdot 501}{4}-75}{102}\cdot\left( 7-6\right)=\dfrac{883}{136}\approx 6{,}49.\]\\
		Tứ phân vị thứ ba của mẫu số liệu gốc là $\dfrac{1}{2}\left( x_{376}+x_{377}\right) \in \left[ 7;8\right)$.\\
		Do đó tứ phân vị thứ ba của mẫu số liệu ghép nhóm là \[Q_{3} = 7+ \dfrac{\dfrac{3\cdot 501}{4}-177}{202}\cdot\left( 8-7\right)=\dfrac{6451}{808}\approx 7{,}98.\]\\
		Vậy khoảng tứ phân vị là $\Delta_ Q=Q_{3}-Q_{1}\approx 1{,}49$.
	}
\end{ex}
