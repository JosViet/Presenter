\newpage
\def\thoigian{90}%--Thời gian
\de{Đề số 2}{Chương II. Dãy số. Cấp số cộng. Cấp số nhân}



\begin{center}
	\textbf{PHẦN 1 - CÂU TRẮC NGHIỆM BỐN PHƯƠNG ÁN}
\end{center}
\Opensolutionfile{ans}[ans/ans-TN-ONTAPCHUONG7-DE2]


\begin{ex}%[1D2N1-3]%[Dự án D - đợt 4 NH24-25- Nguyễn Hoài Nam]
	Cho dãy số $(u_n)$ có số hạng tổng quát $u_n=\dfrac{n}{2^n-1}$. Số hạng $u_3$ bằng
	\choice
	{$\dfrac{3}{8}$}
	{$\dfrac{7}{3}$}
	{$\dfrac{3}{5}$}
	{\True $\dfrac{3}{7}$}
	\loigiai{
		Số hạng $u_3=\dfrac{3}{2^3-1}=\dfrac{3}{7}$.}
\end{ex}
\begin{ex}%[1D2H1-3]%[Dự án D - đợt 4 NH24-25- Nguyễn Hoài Nam]
	Cho dãy số $(u_n)=-n^2+n+1$. Số $-19$ là số hạng thứ mấy của dãy số?
	\choice
	{\True $5$}
	{$7$}
	{$6$}
	{$4$}
	\loigiai{
		Ta có $-n^2+n+1=-19 \Leftrightarrow -n^2+n+20=0 \Leftrightarrow \hoac{&n=5\\&n=-4.}$\\
		Do $n \in \mathbb{N^*}$ nên $n=5$.\\
		Vậy số $-19$ là số hạng thứ $5$ của dãy.
	}
\end{ex}
\begin{ex} %[1D2H1-4]%[Dự án D - đợt 4 NH24-25- Nguyễn Hoài Nam]
	Cho dãy số $( u_n )$ với $u_n=n^2+n+1$. Khẳng định nào sau đây đúng?
	\choice 
	{ ($u_n$) là dãy giảm} 
	{ \True ($u_n$) là dãy tăng} 
	{ ($u_n$) là dãy không tăng không giảm} 
	{ ($u_n$) là dãy không đổi}
	\loigiai {
		Ta có $u_{n+1}-u_{n}=\left(n+1\right)^2+n+1+1-n^2-n-1=2n+2>0,\ \forall n\in {{\mathbb{N}}^{*}}$.\\ 
		Vậy $u_n$ là dãy tăng.}
\end{ex}

\begin{ex}%[1D2H2-4]%[Dự án D - đợt 4 NH24-25- Nguyễn Hoài Nam]
	Cho cấp số cộng $(u_n)$ có $u_1=5$, $u_5=13$. Công sai của cấp số cộng $(u_n)$ bằng
	\choice
	{$1$}
	{\True $2$}
	{$3$}
	{$5$}
	\loigiai{Ta có $u_5=u_1+4d\Leftrightarrow d=\dfrac{u_5-u_1}{4} = \dfrac{13-5}{4}=2$.}
\end{ex}
\begin{ex}%[1D2H2-4]%[Dự án D - đợt 4 NH24-25- Nguyễn Hoài Nam]
	Cho cấp số cộng $(u_n)$ có số hạng đầu $u_1=2$ và công sai $d=5$. Giá trị của $u_4$ bằng
	\choice
	{$22$}
	{\True $17$}
	{$12$}
	{$250$}
	\loigiai{
		Ta có $u_4=u_1+3d=2+3 \cdot 5=17$.
	}
\end{ex}

\begin{ex}%[1D2N2-4]%[Dự án D - đợt 4 NH24-25- Nguyễn Hoài Nam]
	Cho cấp số cộng $\left(u_n \right)$ có $u_1=-0{,}1$; $d=0{,}1$. Số hạng thứ $7$ của cấp số cộng này là
	\choice
	{$1{,}6$}
	{$6$}
	{\True $0{,}5$}
	{$0{,}6$}
	\loigiai{
		Ta có
		\[u_7=u_1 + 6d =-0{,}1+6\cdot0{,}1=0{,}5.\]
	}
\end{ex}
\begin{ex}%[1D2H2-4]%[Dự án D - đợt 4 NH24-25- Nguyễn Hoài Nam]
	Cho cấp số cộng $\left(u_n \right)$ có số hạng đầu $u_1=-5$ và công sai $d=3$. Số $100$ là số hạng thứ mấy của cấp số cộng?
	\choice
	{$15$}
	{$20$}
	{$35$}
	{\True $36$}
	\loigiai{
		Ta có $u_n=u_1+\left(n-1\right)d\Leftrightarrow 100=-5+\left(n-1\right)\cdot3\Leftrightarrow 100=3n-8\Leftrightarrow n=36$.
	}
\end{ex}

\begin{ex}%[1D2N2-4]%[Dự án D - đợt 4 NH24-25- Nguyễn Hoài Nam]
	Cho cấp số cộng $\left(u_n \right)$ có $u_1=-3$, $d=4$. Chọn khẳng định đúng trong các khẳng định sau?
	\choice
	{$u_5=15$}
	{$u_4=8$}
	{\True $u_3=5$}
	{$u_2=2$}
	\loigiai{
		Ta có $u_n=u_1+\left(n-1\right)d\Rightarrow u_3=u_1+2d=-3+2\cdot4=5$.
	}
\end{ex}

\begin{ex}%[1D2N3-2]%[Dự án D - đợt 4 NH24-25- Nguyễn Hoài Nam]
	Dãy số nào sau đây \textbf{không} phải là cấp số nhân?
	\choice
	{$1$; $2$; $4$; $8$}
	{$3$; $3^2$; $3^3$; $3^4$}
	{$4$; $2$; $1$; $\dfrac{1}{2}$}
	{\True $\dfrac{1}{\pi}$; $\dfrac{1}{\pi^2}$; $\dfrac{1}{\pi^4}$; $\dfrac{1}{\pi^6}$}
	\loigiai{
		Xét $1$; $2$; $4$; $8$ là cấp số nhân công bội là $2$.\\
		Xét $3$; $3^2$; $3^3$; $3^4$ là cấp số nhân công bội là $3$.\\
		Xét $4$; $2$; $1$; $\dfrac{1}{2}$ là cấp số nhân công bội là $\dfrac{1}{2}$.\\
		Xét $\dfrac{1}{\pi}$; $\dfrac{1}{\pi^2}$; $\dfrac{1}{\pi^4}$; $\dfrac{1}{\pi^6}$ không là cấp số nhân vì $\dfrac{u_2}{u_1}=\dfrac{1}{\pi} \neq \dfrac{1}{\pi^2}=\dfrac{u_3}{u_2}$.\\
	}
\end{ex}

\begin{ex}%[1D2H3-4]%[Dự án D - đợt 4 NH24-25- Nguyễn Hoài Nam]
	Cho cấp số nhân $\left(u_n\right)$ có số hạng đầu $u_1=\dfrac{1}{2}$ và công bội $q=3$. Giá trị của $u_5$ bằng
	\choice
	{\True $\dfrac{81}{2}$}
	{$\dfrac{163}{2}$}
	{$\dfrac{27}{2}$}
	{$\dfrac{55}{2}$}
	\loigiai{
		Áp dụng công thức số hạng tổng quát $u_n=u_1 q^{n-1}(n \in \mathbb{N},\, n \geq 2)$.\\
		Do đó $u_5=u_1 q^4 \Rightarrow u_5=\dfrac{1}{2} \cdot 3^4=\dfrac{81}{2}$.
	}
\end{ex}
\begin{ex}%[1D2H3-4]%[Dự án D - đợt 4 NH24-25- Nguyễn Hoài Nam]
	Cho cấp số nhân $\left(u_n\right)$ có $u_1=-1$, công bội $q=3$. Giá trị của $u_3$ bằng
	\choice
	{$5$}
	{$27$ }
	{$8$}
	{\True $-9$}
	\loigiai{
		Cấp số nhân $\left(u_n\right)$ có $u_1=-1$, công bội $q=3$ thì $u_3 = u_1 \cdot q^2 = (-1)\cdot3^2 = -9$. 
	}
\end{ex}
\begin{ex}%[1D2N3-4]%[Dự án D - đợt 4 NH24-25- Nguyễn Hoài Nam]
	Cho cấp số nhân $\left(u_n\right)$ có số hạng đầu $u_1=-3$ và công bội $ q=\dfrac{2}{3}$. Số hạng thứ năm của $\left(u_n\right)$ là
	\choice
	{$\dfrac{27}{16}$}
	{$\dfrac{16}{27}$}
	{$-\dfrac{27}{16}$}
	{\True $-\dfrac{16}{27}$}
	\loigiai{
		Ta có $u_n=u_1\cdot q^{n-1}\Rightarrow u_5=-3\cdot \left(\dfrac{2}{3}\right)^4=-\dfrac{16}{27}$.}
\end{ex}



\Closesolutionfile{ans}
%\begin{center}
%	\textbf{ĐÁP ÁN}
%	\inputansbox{10}{ans/ans}	
%\end{center}


\begin{center}
	\textbf{PHẦN 2 - CÂU TRẮC NGHIỆM ĐÚNG SAI}
\end{center}
\setcounter{ex}{0}
\Opensolutionfile{ans}[ans/answer-DS-ONTAPCHUONG2-DE2]

\begin{ex}%[1D2V2-6]%[Dự án D - đợt 4 NH24-25- Nguyễn Hoài Nam]
	Cho cấp số cộng $(u_{n})$ có công sai $d$ và thỏa mãn $\heva{&u_{7} - u_{1} = 12\\&3u_{3} - 2u_{5} = 7.}$
	\choiceTF
	{$u_7=u_1+7d$}
	{$u_1=10$}
	{\True $d=2$}
	{\True Tổng $10$ số hạng đầu tiên của cấp số cộng này bằng $200$}
	\loigiai{
		Gọi $d$ là công sai của cấp số cộng thỏa mãn bài toán. \\
		Ta có $u_{n} =  u_{1} + \left(n - 1\right)d$ với $\forall n\geq 1, n\in\mathbb{N}$.\\
		\begin{itemchoice}
			\itemch Ta có $u_7=u_1+6d$.
			\itemch Ta có
			$$\heva{&u_{7} - u_{1} = 12\\&3u_{3} - 2u_{5} = 7}\Leftrightarrow \heva{&u_{1} + 6d  - u_{1} = 12\\&3u_{1} + 6d - 2u_{1} - 8d = 7}\Leftrightarrow \heva{&d = 2\\&u_{1} - 2d = 7}\Leftrightarrow \heva{&d = 2\\&u_{1} = 11.}$$
			\itemch Ta có $d=2$.
			\itemch Tổng $10$ số hạng đầu tiên của cấp số cộng này bằng $$S_{10}= n\cdot u_1 + \dfrac{n(n-1)d}{2}=10\cdot 11+\dfrac{10\cdot 9\cdot 2}{2}=200.$$
		\end{itemchoice}		
	}
\end{ex}
\begin{ex}%[1D2V1-5]%[Dự án D - đợt 4 NH24-25- Nguyễn Hoài Nam]
	Cho dãy số $(u_n)$ có số hạng tổng quát $u_n=\dfrac{2n+1}{n+2}$, $n\ge1$.
	\choiceTF
	{\True Số $\dfrac{167}{84}$ là số hạng thứ $250$ của dãy số $(u_n)$}
	{\True Tổng $5$ số hạng đầu tiên của dãy số $(u_n)$ là $\dfrac{914}{140}$}
	{\True Dãy số $(u_n)$ bị chặn}
	{Dãy số có $2$ số hạng là số nguyên}
	\loigiai{
		\begin{itemchoice}
			\itemch Ta có $ u_n=\dfrac{167}{84}\Rightarrow \dfrac{2n+1}{n+2}=\dfrac{167}{84}\Leftrightarrow 84(2n+1)=167(n+2)$
			$ \Leftrightarrow n=250 $. \\
			Vậy $ \dfrac{167}{84}$ là số hạng thứ $250$ của dãy số $(u_n)$. 
			\itemch Năm số hạng đầu tiên của dãy số là $ u_1=1$, $u_2=\dfrac{5}{4}$, $u_3=\dfrac{7}{5}$, $u_4=\dfrac{3}{2}$, $u_5=\dfrac{11}{7}$.\\
			Tổng $5$ số hạng đầu của dãy số $(u_n)$ là $u_1+u_2+u_3+u_4+u_5 = \dfrac{941}{140}$.
			\itemch Ta có $u_n=\dfrac{2n+1}{n+2} = \dfrac{2(n+2)-3}{n+2} = 2-\dfrac{3}{n+2} <2,\ \forall n \in \mathbb{N^*}$.\\
			Hơn nữa, $u_n=\dfrac{2n+1}{n+2}>0,\ \forall n \in \mathbb{N^*}$.\\
			Do đó, dãy số $(u_n)$ bị chặn.
			\itemch Ta có $ u_n=\dfrac{2(n+2)-3}{n+2}=2-\dfrac{3}{n+2}$
			$ \Rightarrow u_n\in \mathbb{Z}\Leftrightarrow \dfrac{3}{n+2}\in \mathbb{Z}\Leftrightarrow 3\;\vdots\; (n+2)\Leftrightarrow n=1 $.\\
			Vậy dãy số có duy nhất một số hạng là số nguyên.
		\end{itemchoice}
	}
\end{ex}

\Closesolutionfile{ans}
%\inputansbox[2]{2}{ans/answer.tex}



\begin{center}
	\textbf{PHẦN 3 - CÂU TRẮC NGHIỆM TRẢ LỜI NGẮN}
\end{center}
\setcounter{ex}{0}
\Opensolutionfile{ans}[ans-KQ-ONTAPCHUONG2-DE2]

\begin{ex}%[1D2V3-6]%[Dự án D - đợt 4 NH24-25- Nguyễn Hoài Nam]
	Cho cấp số nhân $(u_n)$ có $u_2 = 4$, $u_3 = 8$. Hỏi tổng $8$ số hạng đầu tiên của $(u_n)$ bằng bao nhiêu?
	\par
	\shortans[oly]{510}
	
	\loigiai{
		Gọi $q$ là công bội của cấp số nhân $(u_n)$. Ta có 
		$$q=\dfrac{u_3}{u_2} = \dfrac{8}{4} = 2 \Rightarrow u_1 = \dfrac{u_2}{q} = \dfrac{4}{2} = 2.$$ 
		Suy ra tổng $8$ số hạng đầu tiên của $(u_n)$ là
		$$S_8 = \dfrac{u_1\left( q^8-1 \right)}{q-1} = \dfrac{2\left( 2^8-1 \right)}{2-1} = 510.$$
	}
\end{ex}


\begin{ex}%[1D2V2-4]%[Dự án D - đợt 4 NH24-25- Nguyễn Hoài Nam]
	Cho cấp số cộng $(u_n)$ có số hạng đầu $u_1 = -8$ và công sai $d=\dfrac{2}{3}$. Hỏi số $\dfrac{14}{3}$ là số hạng thứ mấy của cấp số cộng trên?
	\par
	\shortans[oly]{20}
	\loigiai{
		Giả sử $u_n = \dfrac{14}{3}$ với $n\in\mathbb{N}^*$. Khi đó 
		$$u_1 + (n-1)d = \dfrac{14}{3}
		\Leftrightarrow -8 + (n-1)\cdot \dfrac{2}{3} = \dfrac{14}{3}
		\Leftrightarrow n=20.$$
		Vậy $\dfrac{14}{3}$ là số hạng thứ $20$ của cấp số cộng $(u_n)$.
	}
\end{ex}


\begin{ex}%[1D2V3-8]%[Dự án D - đợt 4 NH24-25- Nguyễn Hoài Nam]
	Một gia đình mua một chiếc ô tô giá $900$ triệu đồng. Trung bình sau mỗi năm sử dụng, giá trị còn lại của ô tô giảm đi $5\%$ so với năm trước đó. Sau $10$ năm, giá trị của ô tô ước tính còn bao nhiêu triệu đồng (làm tròn kết quả đến hàng đơn vị)?
	\par
	\shortans[oly]{539}
	\loigiai{
		Gọi $(u_n)$ (triệu đồng) là giá trị của ô tô sau $n+1$ năm.\\ 
		Theo đề, ta có $u_1=900$, $u_{n+1}=u_n(100\%-5\%)=u_n\cdot 0{,}95$.\\ 
		Vậy $(u_n)$ là cấp số nhân với $u_1=900$ và công bội $q=0{,}95$.\\ 
		Giá trị của ô tô sau $10$ năm là $u_{11}=u_1\cdot q^{10}=900\cdot 0{,}95^{10}\approx 539$ (triệu đồng).
	} 
\end{ex}


\begin{ex}%[1D2V2-7]%[Dự án D - đợt 4 NH24-25- Nguyễn Hoài Nam]
	Người ta trồng cây theo các hàng ngang với quy luật: ở hàng thứ nhất có $3$ cây, ở hàng thứ hai có $5$ cây, ở hàng thứ ba có $7$ cây,\ldots ở hàng thứ $n$ có $2n+1$ cây. Biết rằng người ta trồng tổng cộng $9 \, 603$ cây. Hỏi số hàng cây được trồng theo cách trên là bao nhiêu?
	\par
	\shortans[oly]{97}
	\loigiai{
		Với $n\in\mathbb{N}^*$ gọi $u_n$ là số cây được trồng ở hàng thứ $n$. Theo giả thiết ta có
		$$\heva{&u_1=3 \\ &u_{n+1} = u_n+2, \, \forall n\in\mathbb{N}^*.}$$
		Suy ra $(u_n)$ là cấp số cộng với $u_1=3$, công sai $d=2$.\\
		Giả sử có $n$ hàng cây được trồng. Tổng số cây được trồng là 
		\allowdisplaybreaks
		\begin{eqnarray*}
			& & u_1 + u_2 + \ldots + u_n = 9 \, 603\\
			&\Leftrightarrow & S_n = 9 \, 603 \\
			&\Leftrightarrow & n u_n + \dfrac{n(n-1)}{2}d = 9 \, 603 \\
			&\Leftrightarrow & 3n + \dfrac{n(n-1)}{2}\cdot 2 = 9 \, 603 \\
			&\Leftrightarrow & n^2 + 2n - 9 \, 603 = 0\\
			&\Leftrightarrow & \hoac{&n=97 \text{ (nhận)} \\ &n=-99 \text{ (loại)}.} 
		\end{eqnarray*}	
		Vậy có $97$ hàng cây. 
	} 
\end{ex}




\Closesolutionfile{ans}



\begin{center}
	\textbf{PHẦN 4 - TỰ LUẬN}
\end{center}


\begin{ex}%[1D2V2-3]%[Dự án D - đợt 4 NH24-25- Nguyễn Hoài Nam]
	Một cấp số cộng $(u_n)$ có tổng của sáu số hạng đầu bằng $18$ và tổng của mười số hạng đầu bằng $110$. Lập công thức số hạng tổng quát của cấp số cộng này.
	\loigiai{
		Vì $(u_n)$ là một cấp số cộng nên có số hạng tổng quát $u_n=u_1+(n-1)d$ $(n\in\mathbb{N}^*)$, trong đó $u_1$ là số hạng đầu và $d$ là công sai.\\
		Ta có $S_6=18\Leftrightarrow \dfrac{6\cdot (u_1+u_6)}{2}=18\Leftrightarrow 2u_1+5d=6$. $(1)$\\
		Và $S_{10}=110\Leftrightarrow \dfrac{10(u_1+u_{10})}{2}=110\Leftrightarrow 2u_1+9d=22$. $(2)$\\
		Từ $(1)$ và $(2)$ ta có hệ phương trình $\heva{& 2u_1+5d=6\\ & 2u_1+9d=22}\Leftrightarrow\heva{& u_1=-7\\ & d=4.}$\\
		Vậy số hạng tổng quát $u_n=-7+(n-1)\cdot 4=4n-11$ $(n\in\mathbb{N}^*)$.
	}
\end{ex}
\begin{ex}%[1D2V3-8]%[Dự án D - đợt 4 NH24-25- Nguyễn Hoài Nam]
	Đầu năm $2026$, một công ty dịch vụ taxi được thành lập. Ban đầu công ty thực hiện gói kinh doanh như sau. Công ty mua $5$ xe ô tô bảy chỗ để kinh doanh, mỗi chiếc ô tô có giá $700$ triệu đồng. Biết rằng, sau mỗi tháng sử dụng, giá trị mỗi ô tô giảm đi $0{,}4\%$ so với tháng ngay trước đó, và mỗi tháng một xe thu nhập được $16$ triệu đồng (giả sử số tiền làm ra mỗi tháng không đổi). Hỏi sau $3$ năm tổng số tiền (bao gồm giá tiền còn lại của $5$ xe ô tô và tổng số tiền thu được) của gói kinh doanh này là bao nhiêu?
	\loigiai{	
		Xét mỗi chiếc ô tô, ta có
		\begin{itemize}
			\item Sau $1$ tháng, giá trị của ô tô còn lại là $$u_1=700-700\cdot 0{,}4\%=700\cdot (1-0{,}4\%) \,\text{ triệu đồng.}$$ 
			\item Sau $2$ tháng, giá trị của ô tô còn lại là $$u_2=700\cdot (1-0{,}4\%)-700\cdot (1-0{,}4\%)\cdot 0{,}4\% =700\cdot (1-0{,}4\%)^2 \,\text{ triệu đồng.}$$
			\item Sau $3$ tháng, giá trị của ô tô còn lại là $$u_3=700\cdot (1-0{,}4\%)^2-700\cdot (1-0{,}4\%)^2\cdot 0{,}4\% =700\cdot (1-0{,}4\%)^3 \,\text{ triệu đồng.}$$
			\item $\ldots$
		\end{itemize}
		Gọi $u_n$ là giá trị của ô tô còn lại sau $n$ tháng sử dụng.\\
		Dãy số $(u_n)$ lập thành một cấp số nhân với số hạng đầu là $u_1=700\cdot (1-0{,}4\%)$ và $q=(1-0{,}4\%)$.\\
		Khi đó $u_n=700\cdot (1-0{,}4\%)^n$.\\	
		Sau $3$ năm ($36$ tháng) thì giá trị của một chiếc ô tô còn lại là $u_{36}=700\cdot (1-0{,}4\%)^{36}\approx 605{,}95$ triệu đồng.\\
		Sau $3$ năm thì thu nhập của một chiếc ô tô là $16\cdot 36 =576$ triệu đồng.\\
		Vậy sau $3$ năm, tổng số tiền (bao gồm giá tiền $5$ xe ô tô và tổng số tiền thu được) của gói kinh doanh này là $$S=5\cdot (605{,}95+576)=5\ 909{,}75\,\text{triệu đồng.} $$
	}
\end{ex}

\begin{ex}%[1D2V3-7]%[Dự án D - đợt 4 NH24-25- Nguyễn Hoài Nam]
	Một kiến trúc sư thiết kế một hội trường với $16$ ghế ngồi ở hàng thứ nhất, $20$ ghế ngồi ở hàng thứ hai, $24$ ghế ngồi ở hàng thứ ba và cứ như vậy, số ghế ở hàng sau nhiều hơn $4$ ghế so với số ghế ở hàng liền trước nó. Nếu muốn hội trường đó có sức chứa ít nhất $988$ ghế ngồi thì kiến trúc sư đó phải thiết kế tối thiểu bao nhiêu hàng ghế?
	\loigiai{
		Gọi $u_n$ là số ghế ở hàng thứ $n$ (với $n\in \mathbb{N}^*$).\\
		Theo giả thiết, ta có: $u_1=16$, $u_2=20$, $u_3=24$, $\ldots$\\
		Suy ra dãy số $(u_n)$ là một cấp số cộng với số hạng đầu $u_1=16$ và công sai $d=4$.\\
		Vì hội trường đó có sức chứa ít nhất $988$ ghế ngồi nên
		$$S_n\geqslant 988 \Leftrightarrow \dfrac{n}{2}\left[2\cdot 16+4(n-1)\right]\geqslant 988 \Leftrightarrow 2n^2+14n-988 \geqslant 0 \Leftrightarrow n\leqslant -26 \text{ hoặc } n\geqslant 19.$$
		Vì $n\in \mathbb{N}^*$ nên ta nhận $n\geqslant 19$.	\\
		Vậy cần tối thiểu $19$ hàng ghế.
	}
\end{ex}
