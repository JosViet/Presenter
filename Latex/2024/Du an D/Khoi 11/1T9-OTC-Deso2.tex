\newpage
\def\thoigian{90}%--Thời gian
\de{Đề số 2}{Đề ôn tập Chương IX - Xác suất}

\begin{center}
	\textbf{PHẦN 1 - CÂU TRẮC NGHIỆM BỐN PHƯƠNG ÁN}
\end{center}
\setcounter{ex}{0}
\Opensolutionfile{ans}[ans-ABCD]


\begin{ex}%[1D9H1-3]
	Cho $A$, $B$ là hai biến cố độc lập. Biết $\mathrm{P}(A)=\dfrac{1}{3}$, $\mathrm{P}(B)=\dfrac{1}{4}$. Giá trị của $\mathrm{P}(A \cap B)$ bằng
	\choice
	{\True $\dfrac{1}{12}$}
	{$\dfrac{1}{7}$}
	{$\dfrac{7}{12}$}
	{$\dfrac{1}{2}$}
	\loigiai
	{
		Do $A$, $B$ là hai biến cố độc lập nên
		\[\mathrm{P}(A \cap B)=\mathrm{P}(A)\cdot\mathrm{P}(B)=\dfrac{1}{3}\cdot\dfrac{1}{4}=\dfrac{1}{12}.\]
	}
\end{ex}

\begin{ex}%[1D9N2-1]
	Cho hai biến cố $A$ và $B$, biến cố giao của hai biến cố $A$ và $B$ kí hiệu là
	\choice
	{$A \backslash B$}
	{$B \backslash A$}
	{$A \cup B$}
	{\True $A \cap B$}
	\loigiai
	{
		Biến cố giao của hai biến cố $A$ và $B$ kí hiệu là $A \cap B$.
	}
\end{ex}
\begin{ex}%[1D9N2-3]
	Cho $A$ và $\overline{A}$ là hai biến cố đối nhau và $P (A)=0{,}4$. Khẳng định nào sau đây đúng?
	\choice
	{$\mathrm{P}(\overline{A})=0{,}4$}
	{\True $\mathrm{P}(\overline{A})=0{,}6$}
	{$\mathrm{P}(\overline{A})=1$}
	{$\mathrm{P}(\overline{A})=0$}
	\loigiai{
		Ta có $A$ và $\overline{A}$ là hai biến cố xung khắc nên $\mathrm{P}(\overline{A})=1-\mathrm{P}(A)=1-0{,}4=0{,}6$.	
	}
\end{ex}



\begin{ex}%[1D9N2-3]
	Nếu hai biến cố $A$ và $B$ xung khắc thì xác suất của biến cố $\mathrm{P}(A\cup B)$ bằng
	\choice
	{\True $\mathrm{P}(A)+\mathrm{P}(B)$}
	{$\mathrm{P}(A) \cdot \mathrm{P}(B)$}
	{$1-\mathrm{P}(A)-\mathrm{P}(B)$}
	{$\mathrm{P}(A) \cdot \mathrm{P}(B)-\mathrm{P}(A)-\mathrm{P}(B)$}
	\loigiai{
		Ta có hai biến cố $A$ và $B$ xung khắc nên $\mathrm{P}(A\cup B) = \mathrm{P}(A)+\mathrm{P}(B)$.
	}
\end{ex}


\begin{ex}%[1D9N1-3]
	Cho $A$ và $B$ là hai biến cố độc lập. Biết $\mathrm{P}(A)=0{,}8$ và $\mathrm{P}(B)=0{,}45$. Tính xác suất của biến cố $AB$.
	\choice
	{${0{,}09}$}
	{\True ${0{,}36}$}
	{${0{,}44}$}
	{${0{,}11}$}
	\loigiai{
		Vì $A$ và $B$ là hai biến cố độc lập nên $\mathrm{P}(AB)=\mathrm{P}(A)\cdot\mathrm{P}(B)=0{,}8\cdot0{,}45=0{,}36$.
	}
\end{ex}


\begin{ex}%[1D9H2-3]
	Trong một lớp học có $15$ học sinh nam và $10$ học sinh nữ. Giáo viên gọi $4$ học sinh lên bảng làm bài tập. Tính xác suất để $4$ học sinh lên bảng có cả nam và nữ.
	\choice
	{$3$}
	{$\dfrac{3}{8}$}
	{$2$}
	{\True $\dfrac{443}{506}$}
	\loigiai
	{Phép thử \lq\lq Chọn ngẫu nhiên $4$ học sinh\rq\rq\, có số phần tử của không gian mẫu là $n(\Omega) = \mathrm{C}_{25}^4$.\\
		Gọi $A$ là biến cố \lq\lq $4$ học sinh được chọn có cả nam và nữ\rq\rq.\\
		Các trường hợp có thể xảy ra như sau:
		\begin{itemize}
			\item Có $1$ nam và $3$ nữ được chọn, trường hợp này có $\mathrm{C}_{15}^1\cdot \mathrm{C}_{10}^3 = 1\,800$ khả năng.
			\item Có $2$ nam và $2$ nữ được chọn, trường hợp này có $\mathrm{C}_{15}^2\cdot \mathrm{C}_{10}^2 = 4\,725$ khả năng.
			\item Có $3$ nam và $1$ nữ được chọn, trường hợp này có $\mathrm{C}_{15}^3\cdot \mathrm{C}_{10}^1 = 4\,550$ khả năng.
		\end{itemize}
		Suy ra $n(A)=1\,800+ 4\,725+ 4\,550 = 11\,075$.\\
		Vậy xác suất của biến cố $A$ là $\mathrm{P}(A)=\dfrac{n(A)}{n(\Omega)} = \dfrac{11\,075}{\mathrm{C}_{25}^4} = \dfrac{443}{506}$.}
\end{ex}
\begin{ex}%[1D9H1-3]
	Hai vận động viên cùng bắn độc lập vào một bia. Xác suất để người thứ nhất, người thứ hai bắn trúng đích lần lượt là $0{,}4$ và $0{,}3$. Xác suất để cả hai người cùng bắn trúng bằng
	\choice
	{$0{,}7$}
	{\True $0{,}12$}
	{$0{,}1$}
	{$0{,}3$}
	\loigiai{
		Gọi $A\colon$ là biến cố \lq\lq người thứ nhất bắn trúng đích\rq\rq.\\
		Gọi $B\colon$ là biến cố \lq\lq người thứ hai bắn trúng đích\rq\rq.\\
		Ta có $A$, $B$ độc lập và $\mathrm{P}(A)=0{,}4$, $\mathrm{P}(B)=0{,}3$.\\
		Xác suất để cả hai người cùng bắn trúng là
		\allowdisplaybreaks
		\begin{eqnarray*}
			\mathrm{P}(A\cap B)&=&\mathrm{P}(A)\cdot \mathrm{P}(B)
			\\
			&=&0{,}4\cdot 0{,}3
			\\
			&=&0{,}12.
		\end{eqnarray*}
	}
\end{ex}


\begin{ex}%[1D9N1-1]
	Cho phép thử có không gian mẫu $\Omega=\{1;2;3;4;5;6\}$. Cặp biến cố không đối nhau là
	\choice
	{$A=\{1\}$ và $B=\{2;3;4;5;6\}$}
	{$C=\{1;4;5\}$ và $D=\{2;3;6\}$}
	{\True $E=\{1;4;6\}$ và $F=\{2;3\}$}
	{$\Omega$ và $\varnothing$}
	\loigiai{
		Cặp biến cố không đối nhau là $E=\{1;4;6\}$ và $F=\{2;3\}$.
	}
\end{ex}


\begin{ex}%[1D9N2-4]
	Cho hai biến cố xung khắc $A$ và $B$ biết $\mathrm{P}(A)=\dfrac{1}{3}$, $\mathrm{P}(B)=\dfrac{2}{5}$. Tính $\mathrm{P}(A\cup B)$.
	\choice
	{$\dfrac{13}{15}$}
	{\True $\dfrac{11}{15}$}
	{$\dfrac{1}{15}$}
	{$\dfrac{2}{15}$}
	\loigiai{
		Do $A$ và $B$ là hai biến cố xung khắc nên $\mathrm{P}(A\cup B)=\mathrm{P}(A)+\mathrm{P}(B)=\dfrac{1}{3}+\dfrac{2}{5}=\dfrac{11}{15}$.
	}
\end{ex}


\begin{ex}%[1D9V2-3]
	Một hộp đựng $4$ viên bi xanh, $3$ viên bi đỏ và $4$ viên bi vàng. Chọn ngẫu nhiên đồng thời $2$ viên bi từ hộp. Xác suất để chọn được hai viên bi cùng màu là
	\choice
	{$\dfrac{2}{9}$}
	{$\dfrac{2}{3}$}
	{$\dfrac{5}{18}$}
	{\True $\dfrac{3}{11}$}
	\loigiai{Chọn 2 viên bi từ hộp có $ \mathrm{C}_{11}^2 $ cách.\\
		Suy ra $n(\Omega)=\mathrm{C}_{11}^2=55$.\\
		Gọi $A$ là biến cố \lq\lq Chọn được hai viên bi cùng màu\rq\rq.\\
		Chọn $2$ bi màu xanh có $\mathrm{C}_{4}^2 $ cách.\\
		Chọn $2$ bi màu đỏ có $\mathrm{C}_{3}^2 $ cách.\\
		Chọn $2$ bi màu vàng có $\mathrm{C}_{4}^2 $ cách.\\
		Suy ra $n(A)=\mathrm{C}_{4}^2+\mathrm{C}_{3}^2+\mathrm{C}_{4}^2 =15$.\\
		Vậy xác suất chọn được hai viên bi cùng màu là $\mathrm{P}(A)=\dfrac{15}{55}=\dfrac{3}{11}$.
		
	}
\end{ex}



\begin{ex}%[1D9N2-2]
	Cho hai biến cố $A=\{1;2;3\}$, $B=\{3;4;5\}$. Xác định biến cố $C$ là biến cố hợp của hai biến cố $A$ và $B$.
	\choice
	{\True $C=\{1;2;3;4;5\}$}
	{$C=\{1;2;4;5\}$}
	{$C=\{3;4;5\}$}
	{$C=\{1;2;3\}$}
	\loigiai{
		Ta có $A\cup B=\{1;2;3;4;5\}$.
	}
\end{ex}


\begin{ex}%[1D9V1-3]
	Một bệnh truyền nhiễm có xác suất truyền bệnh là $0{,}7$ nếu tiếp xúc với người bệnh mà không đeo khẩu trang; là $0{,}2$ nếu tiếp xúc với người bệnh mà có đeo khẩu trang. Chị A tiếp xúc với một người bệnh ba lần, trong đó có hai lần đeo khẩu trang và một lần không đeo khẩu trang. Tính xác suất chị A bị lây bệnh từ người bệnh mà chị tiếp xúc.
	\choice
	{$0{,}028$}
	{$ 0{,}972$}
	{\True $0{,}808$}
	{$0{,}192$}
	\loigiai{
		Gọi $A$ là biến cố \lq\lq Chị A bị lây bệnh từ người bệnh mà chị tiếp xúc\rq\rq;\\
		$\overline{A}$ là biến cố \lq\lq Chị A không bị lây bệnh từ người bệnh mà chị tiếp xúc\rq\rq.\\
		Xác suất chị Dung không bị lây bệnh từ người bệnh mà chị tiếp xúc là
		$$\mathrm{P}(\overline{A})=(1-0{,}2)\cdot (1-0{,}2)\cdot (1-0{,}7)=0{,}192.$$
		Vậy xác suất chị A bị lây bệnh từ người bệnh mà chị tiếp xúc là
		\[ \mathrm{P}(A)=1-0{,}192=0{,}808.\]
	}
\end{ex}

\Closesolutionfile{ans}

%\indapan{6}{ans-ABCD}

%\cauds

\begin{center}
	\textbf{PHẦN 2 - CÂU TRẮC NGHIỆM ĐÚNG SAI}
\end{center}
\setcounter{ex}{0}
\Opensolutionfile{ans}[ans-DS]

\begin{ex}%[1D9H2-4]
	Ở một trường trung học phổ thông $X$, có $19\%$ học sinh học khá môn Ngữ văn, $32\%$ học sinh học khá môn Toán, $7\%$ học sinh học khá cả hai môn Ngữ văn và Toán. Chọn ngẫu nhiên một học sinh của trường $X$. Xét hai biến cố sau
	\begin{align*}
		& A: \text{\lq\lq Học sinh đó học khá môn Ngữ văn\rq\rq}. \\
		& B: \text{\lq\lq Học sinh đó học khá môn Toán\rq\rq}.
	\end{align*}
	Xác định khẳng định đúng trong các khẳng định dưới đây.
	\choiceTF
	{\True $\mathrm{P}(A)=19\%$}
	{$\mathrm{P}(AB)=32\%$}
	{$\mathrm{P}(B)=7\%$}
	{\True $\mathrm{P}(A\cup B)=44\%$}
	\loigiai{
		\begin{itemchoice}
			\itemch Ta có $\mathrm{P}(A)=0{,}19$.
			\itemch Ta có $\mathrm{P}(AB)=0{,}07$.
			\itemch Ta có $\mathrm{P}(B)=0{,}32$.
			\itemch Ta có $\mathrm{P}(A\cup B)=\mathrm{P}(A)+\mathrm{P}(B)-\mathrm{P}(AB)=0{,}19+0{,}32-0{,}07=0{,}44$.
		\end{itemchoice}
	}	
\end{ex}
\begin{ex}%[1D9H2-2]
	Một hộp có $20$ chiếc thẻ cùng loại, mỗi thẻ được ghi một số trong các số $1$, $2$, $\ldots$, $19$, $20$; hai thẻ khác nhau thì ghi hai số khác nhau. Rút ngẫu nhiên một chiếc thẻ trong hộp. Xét các biến cố\\
	\hspace*{1cm}$A$: ``Số trên thẻ được rút ra là số chia hết cho $5$''.\\
	\hspace*{1cm}$B$: ``Số trên thẻ được rút ra là số chia hết cho $3$''.\\
	\hspace*{1cm}$C$: ``Số trên thẻ được rút ra là số chia hết cho $5$ hoặc chia hết cho $3$''.\\
	\hspace*{1cm}$D$: ``Số trên thẻ được rút ra là số chia hết cho $15$''.\\
	Khi đó
	\choiceTF
	{Số phần tử của không gian mẫu là $190$}
	{\True Biến cố $D$ là biến cố giao của biến cố $A$ và biến cố $B$}
	{Biến cố $A$ và biến cố $B$ là hai biến cố xung khắc}
	{\True Xác suất của biến cố $C$ là $\dfrac{1}{2}$}
	\loigiai{
		Số phần tử của không gian mẫu là $n(\Omega)=20$.\\
		Ta có $A=\{5;10;15;20\}$; $B=\{3;6;9;12;15;18\}$; $D=\{15\}$.
		\begin{itemchoice}
			\itemch Số phần tử của không gian mẫu là $20$.
			\itemch Ta có $D=A\cap B=\{15\}$.
			\itemch Vì $A\cap B=\{15\}$ nên biến cố $A$ và biến cố $B$ \textbf{không} xung khắc.
			\itemch Ta có $\mathrm{P}(C)=\mathrm{P}(A\cup B)=\mathrm{P}(A)+\mathrm{P}(B)-\mathrm{P}(A\cap B)=\dfrac{5}{20}+\dfrac{6}{20}-\dfrac{1}{20}=\dfrac{1}{2}$.
		\end{itemchoice}
	}
\end{ex}


\Closesolutionfile{ans}


\begin{center}
	\textbf{PHẦN 3 - CÂU TRẮC NGHIỆM TRẢ LỜI NGẮN}
\end{center}
\setcounter{ex}{0}

\Opensolutionfile{ans}[ans-KQ]
\begin{ex}%[1D9H2-3]
	Một hộp có $20$ chiếc thẻ cùng loại, mỗi thẻ được ghi một trong các số $1$, $2$, $3$, \ldots, $20$; hai thẻ khác nhau thì ghi hai số khác nhau. Rút ngẫu nhiên 1 chiếc thẻ trong hộp. Xác suất để rút được thẻ ghi số chia hết cho $3$ hoặc chia hết cho $7$ bằng bao nhiêu?
	\shortans[]{0{,}4}
	\loigiai
	{
		Số phần tử của không gian mẫu là $n(\Omega)=20$.\\
		Xét biến cố $A$: ``Số trên thẻ được rút ra là chia hết cho $3$''.\\
		Xét biến cố $B$: ``Số trên thẻ được rút ra là chia hết cho $7$''.\\
		Biến cố $A\cup B$: ``Số trên thẻ được rút ra là chia hết cho $3$ hoặc chia hết cho $7$''.\\
		Ta có $A=\{3;6;9;12;15;18\}$, $B=\{7;14\}$.\\
		Do $A$ và $B$ xung khắc nên
		\[\mathrm{P}(A\cup B)=\mathrm{P}(A)+\mathrm{P}(B)=\dfrac{6}{20}+\dfrac{2}{20}=\dfrac{8}{20}=0{,}4.\]
	}
\end{ex}
\begin{ex}%[1D9V2-3]
	Trong một chiếc hộp có $15$ viên bi có cùng kích thước và khối lượng, trong đó có $4$ viên bi màu đỏ, $5$ viên bi màu xanh và $6$ viên bi màu vàng. Lấy ngẫu nhiên đồng thời $3$ viên bi. Gọi $\mathrm{P}$ là xác suất lấy được $3$ viên bi có đúng hai màu. Giá trị của $65\cdot \mathrm{P}$ bằng bao nhiêu?
	\shortans[]{43}
	\loigiai
	{
		Số phần tử của không gian mẫu là $n(\Omega)=\mathrm{C}_{15}^3=455$.\\
		Xét biến cố $A$: ``$3$ viên bi có đúng hai viên bi màu đỏ''. Suy ra $n(A)=\mathrm{C}_{4}^2\cdot\mathrm{C}_{11}^1=66$.\\
		Xét biến cố $B$: ``$3$ viên bi có đúng hai viên bi màu xanh''. Suy ra $n(B)=\mathrm{C}_{5}^2\cdot\mathrm{C}_{10}^1=100$.\\
		Xét biến cố $C$: ``$3$ viên bi có đúng hai viên bi màu vàng''. Suy ra $n(C)=\mathrm{C}_{6}^2\cdot\mathrm{C}_{9}^1=135$.\\
		Biến cố $A\cup B\cup C$: ``$3$ viên bi có đúng hai màu''.\\
		Do $A$, $B$ và $C$ xung khắc nên
		\[\mathrm{P}(A\cup B\cup C)=\mathrm{P}(A)+\mathrm{P}(B)+\mathrm{P}(C)=\dfrac{66}{455}+\dfrac{100}{455}+\dfrac{135}{455}=\dfrac{301}{455}=\dfrac{43}{65}.\]
		Khi đó $65\cdot\mathrm{P}(A\cup B\cup C)=43$.
	}
\end{ex}

\begin{ex}%[1D9V2-5]
	Một tổ học sinh có $12$ bạn, trong đó có $6$ bạn thích môn Bóng đá, $4$ bạn thích môn Cầu lông và $2$ bạn thích cả hai môn Bóng đá và Cầu lông. Chọn ngẫu nhiên một học sinh trong tổ. Tính xác suất để chọn được bạn đó không thích cả hai môn Bóng đá và Cầu lông (làm tròn đến hàng phần trăm).
	\shortans[]{0{,}33}
	\loigiai
	{
		Gọi $A$ là biến cố \lq\lq Chọn được bạn thích môn Bóng đá\rq\rq.\\
		$B$ là biến cố \lq\lq Chọn được bạn thích môn Cầu lông\rq\rq.
		$$\mathrm{P}(A \cup B)=\mathrm{P}(A)+\mathrm{P}(B)-\mathrm{P}(AB)=\dfrac{6}{12}+\dfrac{4}{12}-\dfrac{2}{12}=\dfrac{2}{3}.$$
		Xác suất để chọn được bạn đó không thích cả môn Bóng đá và Cầu lông
		$$1-\mathrm{P}(A \cup B)=1-\dfrac{2}{3}=\dfrac{1}{3}\approx 0{,}33.$$
	}
\end{ex}



\begin{ex}%[1D9H1-3]
	Hai chuyến bay của hai hãng hàng không $X$ và $Y$ hoạt động độc lập với nhau. Xác suất để chuyến bay của hãng $X$ và hãng $Y$ khởi hành đúng giờ tương ứng là $0{,}8$ và $0{,}9$. Tính xác suất để có ít nhất một trong hai chuyến bay khởi hành đúng giờ.
	\shortans[]{0{,}98}
	\loigiai{
		Gọi các biến cố
		\begin{itemize}
			\item $A$: \lq\lq Chuyến bay của hãng $X$ khởi hành đúng giờ\rq\rq.
			\item $B$: \lq\lq Chuyến bay của hãng $Y$ khởi hành đúng giờ\rq\rq.
		\end{itemize}
		Ta có $\mathrm{P}(A)=0{,}8$ và $\mathrm{P}(B)=0{,}9$.\\
		Gọi $C$ là biến cố \lq\lq Có ít nhất một trong hai chuyến bay khởi hành đúng giờ\rq\rq.\\
		Ta có $C=A\overline{B}\cup\overline{A}B\cup AB$. Do đó
		\begin{eqnarray*}
			\mathrm{P}(C)&=&\mathrm{P}(A)\cdot \mathrm{P}\left(\overline{B}\right)+\mathrm{P}\left(\overline{A}\right)\cdot \mathrm{P}(B)+\mathrm{P}(A)\cdot \mathrm{P}(B)\\
			&=&0{,}8\cdot 0{,}1+0{,}2\cdot 0{,}9+0{,}8\cdot 0{,}9\\
			&=& 0{,}98.
		\end{eqnarray*}
}
\end{ex}
\Closesolutionfile{ans}
\begin{center}
	\textbf{PHẦN 4 - TỰ LUẬN}
\end{center}
\setcounter{ex}{0}

\Opensolutionfile{ans}[ans-TL]


\begin{ex}%[1D9H1-3]
	Một chiếc máy có $2$ động cơ $I$ và $II$ hoạt động độc lập với nhau. Xác suất để động cơ $I$ chạy tốt và động cơ $II$ chạy tốt lần lượt là $0{,}8$ và $0{,}7$. Tính xác suất đề cả hai động cơ đều chạy không tốt.
	\loigiai{
		Xét các biến cố sau\\
		$A$ \text{\lq\lq}Động cơ $I$ chạy tốt\text{\rq\rq}. $\mathrm{P}(A)=0{,}8$; $\mathrm{P}(\overline{A})=0{,}2$.\\
		$B$ \text{\lq\lq}Động cơ $II$ chạy tốt\text{\rq\rq}. $\mathrm{P}(B)=0{,}7$, $\mathrm{P}(\overline{B})=0{,}3$.\\
		$C$ \text{\lq\lq}Cả hai động cơ đều chạy không tốt\text{\rq\rq}.\\		
		 Khi đó $C=\overline{A}\cap \overline{B}$, $\overline{A}$ và $\overline{B}$ là hai biến cố độc lập nên
		$$\mathrm{P}(C)=\mathrm{P}(\overline{A})\cdot \mathrm{P}(\overline{B})=0{,}06.$$
	}
\end{ex}
\begin{ex}%[1D9H2-4]
	Xác suất bắn trúng mục tiêu của một vận động viên khi bắn một viên đạn là
	$0,6$. Người đó bắn hai viên đạn một cách độc lập. Tìm xác suất của biến cố $A$: \lq\lq Một viên trúng
	mục tiêu và một viên trượt mục tiêu \rq\rq.
	\loigiai{Gọi $B_i$ là biến cố: \lq\lq Lần thứ $i$ người đó bắn trúng mục tiêu\rq\rq, $i=1;2$.\\
		Khi đó $A=\left( B_1\cap\overline{B_2}\right) \cup \left( \overline{B_1}\cap B_2\right) )$.\\
		Khi đó $\mathrm{P}(A)=\mathrm{P}(B_1)\cdot \mathrm{P}(\overline{B_2})+\mathrm{P}(\overline{B_1})\cdot \mathrm{P}(B_2)=2\cdot 0{,}6\cdot 0{,}4=0{,}48$.}
\end{ex}


\begin{ex}%[1D9H2-4]
	Có $2$ hộp bút chì màu. Hộp thứ nhất có $5$ bút chì màu đỏ và $7$ bút chì màu xanh. Hộp thứ hai có $8$ bút chì màu đỏ và $4$ bút chì màu xanh. Chọn ngẫu nhiên mỗi hộp một cây bút chì. Tính xác suất để chọn được $1$ cây bút chì màu đỏ và $1$ cây bút chì màu xanh.
	\loigiai{
		$n\left(\Omega\right)=\mathrm{C}^{1}_{12}\cdot \mathrm{C}^{1}_{12}=144.$\\
		Xét các biến cố\\
		$H$: \text{\lq\lq}Chọn được $1$ cây bút chì màu đỏ và $1$ cây bút chì màu xanh\text{\rq\rq}.\\
		$A$: \text{\lq\lq}Chọn được $1$ cây bút chì màu đỏ ở hộp $1$ và $1$ cây bút chì màu xanh ở hộp $2$\text{\rq\rq}, $n(A)=\mathrm{C}^{1}_{5}\cdot \mathrm{C}^{1}_{4}=20.$\\
		$B$: \text{\lq\lq}Chọn được $1$ cây bút chì màu đỏ ở hộp $2$ và $1$ cây bút chì màu xanh ở hộp $1$\text{\rq\rq}, $n(B)=\mathrm{C}^{1}_{8}\cdot \mathrm{C}^{1}_{7}=56.$\\
		Khi đó $H=A\cup B$. Do hai biến cố $A$ và $B$ xung khắc nên
		$$\mathrm{P}(H)=\mathrm{P}(A)+\mathrm{P}(B)=\dfrac{n(A)+n(B)}{n(\Omega)}=\dfrac{20+56}{144}=\dfrac{19}{36}.$$
	}
\end{ex}
\Closesolutionfile{ans}
