\newpage
\def\thoigian{90}%--Thời gian
\de{Đề số 3}{Đề ôn tập Chương IX - Xác suất}

\begin{center}
	\textbf{PHẦN 1 - CÂU TRẮC NGHIỆM BỐN PHƯƠNG ÁN}
\end{center}
\setcounter{ex}{0}
\Opensolutionfile{ans}[ans-ABCD]


\begin{ex}%[1D9H2-5]
	Trong một lần làm bài kiểm tra môn Toán bằng hình thức trắc nghiệm, mỗi câu hỏi có $4$ phương án lựa chọn, trong đó chỉ có $1$ phương án đúng. Bạn An vì không còn đủ thời gian nên bạn khoanh bừa $5$ câu. Xác suất để bạn An làm sai cả $5$ câu đó bằng
	\choice
	{$\dfrac{3}{4}$}
	{\True $\dfrac{243}{1024}$}
	{$\dfrac{15}{1024}$}
	{$\dfrac{1}{1024}$}
	\loigiai{
		Xác suất để An làm sai $1$ câu là $\dfrac{3}{4}$.\\
		Vì khả năng làm sai các câu là độc lập với nhau nên xác suất để làm sai cả $5$ câu là $\left(\dfrac{3}{4}\right)^5=\dfrac{243}{1024}$.	
	}
\end{ex}

\begin{ex}%[1D9H1-3]
	Hai xạ thủ A và B cùng bắn súng vào một tấm bia, mỗi người bắn một viên. Biết rằng xác suất bắn trúng của xạ thủ A là $0{,}5$ và của xạ thủ B là $0{,}7$. Khả năng bắn trúng của hai xạ thủ là độc lập. Xác suất của biến cố \lq\lq Cả hai xạ thủ đều bắn trúng\rq\rq ~là
	\choice
	{$0{,}65$}
	{\True $0{,}35$}
	{$0{,}85$}
	{$0{,}15$}
	\loigiai{
		Gọi $C$ là biến cố \lq\lq Xạ thủ A bắn trúng\rq\rq. Ta có $\mathrm{P}(C)=0{,}5 $.\\
		Gọi $D$ là biến cố \lq\lq Xạ thủ B bắn trúng\rq\rq. Ta có $\mathrm{P}(D)=0{,}7 $.\\
		Vì hai biến cố $C$ và $D$ độc lập nên xác suất để cả hai xạ thủ đều bắn trúng là
		$$\mathrm{P}(CD)=\mathrm{P}(C)\cdot\mathrm{P}(D)=0{,}5\cdot 0{,}7=0{,}35.$$
	}
\end{ex}



\begin{ex}%[1D9H1-3]
	Trong phòng làm việc có hai máy tính hoạt động độc lập với nhau, khả năng hoạt động tốt trong ngày của hai máy này tương ứng là $0{,}75$ và $0{,}85$. Xác suất để cả hai máy hoạt động không tốt trong ngày là
	\choice
	{$0{,}0675$}
	{\True $0{,}0375$}
	{$0{,}0575$}
	{$0{,}0475$}
	\loigiai{
		Gọi $A$ là biến cố \lq\lq Máy thứ nhất hoạt động tốt\rq\rq\,và  $B$ là biến cố \lq\lq Máy thứ hai hoạt động tốt\rq\rq.\\
		Theo giả thiết $\mathrm{P}(A)=0{,}75$, $\mathrm{P}(B)=0{,}85$.\\
		Suy ra $\mathrm{P}(\overline{A})=0{,}25$, $\mathrm{P}(\overline{B})=0{,}15$.\\
		Xác suất để cả hai máy hoạt động không tốt trong ngày là 
		$$\mathrm{P}(\overline{A}\cap \overline{B})=\mathrm{P}(\overline{A})\cdot \mathrm{P}(\overline{B})=0{,}25\cdot 0{,}15=0{,}0375.$$
	}
\end{ex}

\begin{ex}%[1D9N2-2]
	Một hộp đựng $12$ viên bi màu xanh và $20$ viên bi màu đỏ. Lấy ngẫu nhiên đồng thời $2$ viên bi từ hộp. Xét các biến cố $A:$ \lq\lq Lấy được hai viên bi màu đỏ\rq\rq, biến cố $B:$ \lq\lq Lấy được hai viên bi màu xanh\rq\rq. Biến cố hợp của hai biến cố $A$ và $B$ là biến cố nào sau đây?
	\choice
	{\lq\lq Lấy được ít nhất một viên bi màu xanh\rq\rq}
	{\lq\lq Lấy được hai viên bi khác màu\rq\rq}
	{\lq\lq Lấy được ít nhất một viên bi màu đỏ\rq\rq}
	{\True \lq\lq Lấy được hai viên bi cùng màu\rq\rq}
	\loigiai{
		Biến cố hợp của $A$ và $B$ là biến cố \lq\lq Lấy được hai viên bi cùng màu\rq\rq.
	}
\end{ex}
\begin{ex}%[1D9N1-3]
	Cho $A$, $B$ là hai biến cố độc lập và $\mathrm{P}(A)=0{,}4$; $\mathrm{P}(AB)=0{,}3$. Xác suất của biến cố $B$ là
	\choice
	{\True $\mathrm{P}(B)=0{,}75$}
	{$\mathrm{P}(B)=0{,}5$}
	{$\mathrm{P}(B)=0{,}12$}
	{$\mathrm{P}(B)=0{,}2$}
	\loigiai{
		Vì $A$ và $B$ là hai biến cố độc lập nên
		$$\mathrm{P}(AB)=\mathrm{P}(A)\cdot\mathrm{P}(B)\Rightarrow\mathrm{P}(B)=\dfrac{\mathrm{P(AB)}}{\mathrm{P}(A)}=\dfrac{0{,}3}{0{,}4}=0{,}75.$$
	}
\end{ex}

\begin{ex}%[1D9H2-5]
	Trên kệ có 4 cuốn sách toán và 5 cuốn sách văn. Lấy ngẫu nhiên 3 cuốn sách, xác suất của biến cố "ba cuốn sách lấy được cùng loại" bằng
	\choice
	{\True $\dfrac{1}{6}$}
	{$\dfrac{1}{3}$}
	{$\dfrac{1}{2}$}
	{$\dfrac{5}{6}$}
	\loigiai{Ta có $n(\Omega)=\mathrm{C}_9^3=84$.\\
		Gọi $A$ là biến cố "ba cuốn sách lấy được cùng loại", ta có
		$$n(A)=\mathrm{C}_4^3+\mathrm{C}_5^3=14.$$
		Vậy $P(A)=\dfrac{n(A)}{n(\Omega)}=\dfrac{1}{6}.$
	}
\end{ex}
\begin{ex}%[1D9N1-1]
	Cho $A$ và $B$ là hai biến cố độc lập, khi đó công thức tính xác suất nào dưới đây là đúng?
	\choice
	{\True $\mathrm{P}(AB)=\mathrm{P}(A)\cdot \mathrm{P}(B)$}
	{$\mathrm{P}(AB)=\mathrm{P}(A)+\mathrm{P}(B)$}
	{$\mathrm{P}(A\cup B)=\mathrm{P}(A)\cdot \mathrm{P}(B)$}
	{$\mathrm{P}(A\cup B)=\mathrm{P}(A)+\mathrm{P}(B)$}
	\loigiai{
		Với $A$ và $B$ là hai biến cố độc lập, ta có $\mathrm{P}(AB)=\mathrm{P}(A)\mathrm{P}(B)$.
	}
\end{ex}
\begin{ex}%[1D9H2-1]
	Có hai xạ thủ $X$ và $Y$, mỗi người bắn một viên đạn vào mục tiêu. Xét các biến cố $A\colon$ \lq\lq$\text{Xạ thủ
		$X$ bắn trúng}$\rq\rq; $B\colon$ \lq\lq Xạ thủ $Y$ bắn trúng\,\rq\rq; $C:$ \lq\lq\text{Cả hai xạ thủ bắn trượt}\rq\rq. Biểu diễn biễn cố $C$ theo hai biến cố $A$ và $B$ ta được kết quả là
	\choice
	{$C=A\cup B$}
	{$C=\overline{A}\cap\overline{B}$}
	{\True $C=\overline{A}\overline{B}$}
	{$C=AB$}
	\loigiai{
		Ta có $C=\overline{A}\overline{B}$.
	}
\end{ex}


\begin{ex}%[1D9H2-4]
	Một lớp học gồm $40$ học sinh trong đó có $15$ học sinh giỏi Toán, $10$ học sinh giỏi Lý, trong đó có $5$ học sinh giỏi Toán và Lý. Chọn ngẫu nhiên một học sinh. Tính xác suất đề học sinh đó giỏi Toán hay giỏi Lý.
	\choice
	{\True $\dfrac{1}{2}$}
	{$\dfrac{3}{8}$}
	{$\dfrac{1}{8}$}
	{$\dfrac{5}{8}$}
	\loigiai{
		Gọi $A,\ B$ lần lượt là tập các học sinh gỏi Toán, Lý. Khi đó
		$$n(A\cup B)=n(A)+n(B)-n(A\cap B)=15+10-5=20.$$
		Xác suất cần tính là $P(A\cup B)=\dfrac{20}{40}=\dfrac{1}{2}.$
	}
\end{ex}


\begin{ex}%[1D9H1-3]
	Cho $A$, $B$ là hai biến cố độc lập. Biết $\mathrm{P}(A)=\dfrac{1}{4}$, $\mathrm{P}(AB)=\dfrac{1}{9}$. Tính $\mathrm{P}(B)$.
	\choice
	{$\dfrac{1}{5}$}
	{$\dfrac{5}{36}$}
	{\True $\dfrac{4}{9}$}
	{$\dfrac{7}{36}$}		
	\loigiai{Ta có $A$, $B$ là hai biến cố độc lập nên $\mathrm{P}(AB)=\mathrm{P}(A) \cdot \mathrm{P}(B) \Leftrightarrow \dfrac{1}{9}=\dfrac{1}{4}\cdot\mathrm{P}(B)\Leftrightarrow \mathrm{P}(B)=\dfrac{4}{9}$.
	}
\end{ex}


\begin{ex}%[1D9H1-3]
	Cho $A$, $B$ là hai biến cố độc lập. Biết $\mathrm{P}(A)=0{,}5$ và $\mathrm{P}(AB)=0{,}2$. Tính $\mathrm{P}(A \cup B)$.
	\choice
	{\True $0{,}7$}
	{$0{,}6$ }
	{$0{,}5$}
	{$0{,}3$ }
	\loigiai{Ta có $A$, $B$ là hai biến cố độc lập nên $\mathrm{P}(AB)=\mathrm{P}(A) \cdot \mathrm{P}(B) \Leftrightarrow 0{,}2=0{,}5\cdot\mathrm{P}(B)\Leftrightarrow \mathrm{P}(B)=0{,}4$.
		$$
		\mathrm{P}(A \cup B)=\mathrm{P}(A)+\mathrm{P}(B)-\mathrm{P}(AB)=0{,}5+0{,}4-0{,}2=0{,}7.
		$$
	}
\end{ex}


\begin{ex}%[1D9H1-3]
	Ba người cùng bắn vào $1$ bia. Xác suất người thứ nhất, thứ hai, thứ ba bắn trúng đích lần lượt là  $0{,}8$; $0{,}6$; $0{,}5$. Xác suất để có đúng $1$ người bắn trúng đích bằng
	\choice
	{\True $0{,}26$}
	{$0{,}24$}
	{$0{,}96$}
	{$0{,}92$}	
	\loigiai{Gọi $X$ là biến cố \lq\lq có đúng 1 người bắn trúng đích\rq\rq.\\	
		Gọi $A$ là biến cố \lq\lq người thứ nhất bắn trúng đích\rq\rq. Khi đó  $\mathrm{P}(A)=0{,}8$; $\mathrm{P}(\overline{A})=0{,}2$.\\
		Gọi $B$ là biến cố \lq\lq người thứ hai bắn trúng đích\rq\rq. Khi đó $\mathrm{P}(B)=0{,}6$; $\mathrm{P}(\overline{B})=0{,}4$.\\
		Gọi $C$ là biến cố \lq\lq người thứ ba bắn trúng đích \rq\rq. Khi đó$\mathrm{P}(C)=0{,}5$; $\mathrm{P}(\overline{C})=0{,}5$.\\
		Ta thấy biến cố $A$, $B$, $C$ là $3$ biến cố độc lập nhau, do đó các biến cố  $A$, $B$, $C$, $\overline{A}$, $\overline{B}$, $\overline{A}$ độc lập nhau.\\ Theo công thức nhân xác suất ta có
		\begin{eqnarray*}
			\mathrm{P}(X)&=&\mathrm{P}(A\overline{B}\,\overline{C})+\mathrm{P}(\overline{A}  B \overline{C})+\mathrm{P}(\overline{A}\, \overline{B} C)\\
			&=& \mathrm{P}(A)\cdot \mathrm{P}(\overline{B})\cdot \mathrm{P}(\overline{C})+\mathrm{P}(\overline{A})\cdot \mathrm{P}(B)\cdot \mathrm{P}(\overline{C})+\mathrm{P}(\overline{A})\cdot \mathrm{P}(\overline{B})\cdot \mathrm{P}(C) \\
			&=& 0{,}8 \cdot 0{,}4 \cdot 0{,}5+0{,}2 \cdot 0{,}6 \cdot 0{,}5+0{,}2 \cdot 0{,}4 \cdot 0{,}5=0{,}26.
		\end{eqnarray*}	
	}
\end{ex}

\Closesolutionfile{ans}

%\indapan{6}{ans-ABCD}

%\cauds

\begin{center}
	\textbf{PHẦN 2 - CÂU TRẮC NGHIỆM ĐÚNG SAI}
\end{center}
\setcounter{ex}{0}
\Opensolutionfile{ans}[ans-DS]
\begin{ex}%[1D9H1-3]
	Cả hai xạ thủ cùng bán vào bia. Xác suất người thứ nhất bắn trúng bia là $0{,}8$; người thứ hai bắn trúng bia là $0{,}7$. Khi đó xác suất để
	\choiceTF
	{người thứ nhất bắn trúng và người thứ hai bắn không trúng bia bằng $0{,}14$}
	{\True người thứ nhất không bắn trúng và người thứ hai bắn trúng bia bằng $0{,}14$}
	{\True hai người đều bắn trúng bia bằng $0{,}56$}
	{\True có ít nhất một người bắn trúng bia bằng $0{,}94$}
	\loigiai{
		Gọi $A$, $B$ lần lượt là biến cố người thứ nhất và người thứ hai bắn trúng bia.\\
		Khi đó $A$, $B$ là hai biến cố độc lập và  $\mathrm{P}(A)=0{,}8$, $\mathrm{P}(B)=0{,}7$.\\
		Suy ra $\mathrm{P}\left(\overline{A}\right)=0{,}2$, $\mathrm{P}\left(\overline{B}\right)=0{,}3$.
		\begin{itemchoice}
			\itemch Xác suất người thứ nhất bắn trúng và người thứ hai bắn không trúng bia là
			$$\mathrm{P}(A\overline{B})=\mathrm{P}(A)\cdot \mathrm{P}(\overline{B})=0{,}8\cdot 0{,}3=0{,}24.$$
			\itemch Xác suất người thứ nhất không bắn trúng và người thứ hai bắn trúng bia là
			$$\mathrm{P}(\overline{A}B)=\mathrm{P}(\overline{A})\cdot \mathrm{P}(B)=0{,}2\cdot 0{,}7=0{,}14.$$
			\itemch Xác suất người thứ nhất bắn trúng và người thứ hai bắn trúng bia là
			$$\mathrm{P}(AB)=\mathrm{P}(A)\cdot \mathrm{P}(B)=0{,}8\cdot 0{,}7=0{,}56.$$
			\itemch Xác suất có ít nhất một người bắn trúng bia là
			$$\mathrm{P}(A\cup B)=\mathrm{P}(A)+ \mathrm{P}(B)-\mathrm{P}(AB)=0{,}8+0{,}7-0{,}56=0{,}94.$$
		\end{itemchoice}
	}
\end{ex}
\begin{ex}%[1D9V2-5]
	Một hộp đựng $20$ tấm thẻ cùng loại được đánh số từ $1$ đến $20$. Rút ngẫu nhiên một tấm thẻ trong hộp. Gọi $A$ là biến cố \lq\lq Rút được tấm thẻ ghi số chẵn lớn hơn $9$\rq\rq; $B$ là biến cố \lq\lq Rút được tấm thẻ ghi số từ $9$ đến $14$\rq\rq.
	\choiceTF
	{$A$ và $B$ là hai biến cố xung khắc}
	{\True $\mathrm{P}(A)=\dfrac{3}{10}$}
	{$\mathrm{P}(AB)=\dfrac{1}{5}$}
	{\True $\mathrm{P}(A\cup B)=\dfrac{9}{20}$}
	\loigiai{
		Số phần tử của không gian mẫu là $n(\Omega)=20$.\\
		Ta có $A=\{10;12;14;16;18;20\}$ và $B=\{9;10;11;12;13;14\}$.\\
		Khi đó, $A\cap B=\{10;12;14\}$.
		\begin{itemchoice}
			\itemch Do $A\cap B\ne \varnothing$ nên $A$ và $B$ không phải là hai biến cố xung khắc.
			\itemch Ta có $\mathrm{P}(A)=\dfrac{n(A)}{n(\Omega)}=\dfrac{6}{20}=\dfrac{3}{10}$.
			\itemch Ta có $\mathrm{P}(AB)=\dfrac{n(AB)}{n(\Omega)}=\dfrac{3}{20}$.
			\itemch Ta có $\mathrm{P}(B)=\dfrac{n(B)}{n(\Omega)}=\dfrac{6}{20}=\dfrac{3}{10}$.\\
			Khi đó, $\mathrm{P}(A\cup B)=\mathrm{P}(A)+\mathrm{P}(B)-\mathrm{P}(AB)=\dfrac{3}{10}+\dfrac{3}{10}-\dfrac{3}{20}=\dfrac{9}{20}$.
		\end{itemchoice}
	}
\end{ex}
\Closesolutionfile{ans}
\begin{center}
	\textbf{PHẦN 3 - CÂU TRẮC NGHIỆM TRẢ LỜI NGẮN}
\end{center}
\setcounter{ex}{0}
\Opensolutionfile{ans}[ans-KQ]
\begin{ex}%[1D9H2-5]
	Hai bạn Trang và Hà của lớp 11A tham gia giải bóng bàn đơn nữ do nhà trường tổ chức. Hai bạn đó nằm ở hai bảng đấu khác nhau, mỗi bảng đấu chỉ chọn một người vào chung kết. Xác suất vượt qua vòng bảng để vào chung kết của Trang và Hà lần lượt là $0{,}6$ và $0{,}7$. Tính xác suất để có ít nhất một trong hai bạn Trang, Hà lọt vào chung kết.
		\shortans[]{0{,}88}
	\loigiai{
		Gọi $A$ là biến cố Trang lọt vào chung kết, $B$ là biến cố Hà lọt vào chung kết.\\
		Ta thấy $A$, $B$ là hai biến cố độc lập.
		\\
		Khi đó biến cố \lq\lq ít nhất một trong hai bạn Trang, Hà lọt vào chung kết\rq\rq~ là $A\cup B$.\\
		Vì $A$, $B$ độc lập nên $P(AB)=\mathrm{P}(A)\cdot \mathrm{P}(B)=0{,6}\cdot0{,}7=0{,}42$.\\
		Khi đó $\mathrm{P}(A\cup B)=\mathrm{P}(A)+\mathrm{P}(B)-\mathrm{P}(AB)=0{,}6+0{,}7-0{,}42=0{,}88$.
	}
\end{ex}
\begin{ex}%[1D9H2-4]
	Một nhà sản xuất phát hành hai cuốn sách $A$ và $B$. Thống kê cho thấy có $50\%$ người mua sách $A$; $70\%$ người mua sách $B$ và $30\%$ người mua cả sách $A$ và sách $B$. Chọn ngẫu nhiên một người mua sách. Tính xác suất để người đó mua ít nhất một trong hai sách $A$ hoặc sách $B$.
	\shortans[]{0{,}9}
	\loigiai{
		\begin{itemize}
			\item Gọi $M$ là biến cố \lq\lq người mua sách $A$\rq\rq.\\
			Xác suất của người mua sách $A$ là $\mathrm{P}(M)=50\%$.
			\item Gọi $N$ là biến cố \lq\lq người mua sách $B$\rq\rq.\\
			Xác suất của người mua sách $B$ là $\mathrm{P}(N)=70\%$.
			\item Xác suất của người mua cả sách $A$ và sách $B$ là $\mathrm{P}(M\cap N)=30\%$.
		\end{itemize}
		Xác suất để người đó mua ít nhất một trong hai sách $A$ hoặc sách $B$ là
		$$\mathrm{P}(M\cup N)=\mathrm{P}(M)+\mathrm{P}(N)-\mathrm{P}(M\cap N)=50\%+70\%-30\%=90\%=0{,}9.$$
	}
\end{ex}


\begin{ex}%[1D9H1-3]
	Một cửa hàng máy photocopy có hai máy photo $X$ và $Y$ hoạt động độc lập với nhau. Xác suất của máy photo $X$ và $Y$ bị lỗi kĩ thuật khi hoạt động lần lượt là $0{,}1$ và $0{,}18$. Tính xác suất để ít nhất một trong hai máy photo của cửa hàng bị lỗi kĩ thuật khi hoạt động (kết quả làm tròn hàng phần trăm).
	\shortans[]{0{,}26}
	\loigiai{
		Gọi $A$ là biến cố \lq\lq máy photo $X$ bị lỗi kĩ thuật khi hoạt động\rq\rq.\\
		Gọi $B$ là biến cố \lq\lq máy photo $Y$ bị lỗi kĩ thuật khi hoạt động \rq\rq.\\
		Ta có $\mathrm{P}(A)=0{,}1$; $\mathrm{P}(B)=0{,}18$.\\
		Suy ra $\mathrm{P}\left(\overline{A}\right)=0{,}9$; $\mathrm{P}\left(\overline{B}\right)=0{,}82$.\\
		Vì $A$, $B$ là hai biến cố độc lập nên $\overline{A}$, $\overline{B}$ là hai biến cố độc lập.\\
		Gọi $E$ là biến cố \lq\lq ít nhất một trong hai máy photo của cửa hàng bị lỗi kĩ thuật khi hoạt động\rq\rq.\\
		Ta có $E$ là biến cố đối của biến cố $\overline{A}\cap\overline{B}$.
		Do đó ta có $$\mathrm{P}(E)=1-\mathrm{P}\left(\overline{A}\cap\overline{B}\right)
		=1-\mathrm{P}\left(\overline{A}\right)\mathrm{P}\left(\overline{B}\right)=1-0{,}9\cdot 0{,}82=0{,}262\approx 0{,}26.$$
	}
\end{ex}

\begin{ex}%[1D9V1-3]
	Ba xạ thủ cùng nổ súng bắn vào mục tiêu một cách độc lập. Biết rằng xác suất bắn trúng mục tiêu của ba xạ thủ lần lượt là $0{,}7$; $0{,}6$; $0{,}5$. Tính xác suất để có ít nhất một xạ thủ bắn trúng mục tiêu.
		\shortans[]{0{,}94}
	\loigiai{
		Gọi $A_i$ là biến cố người thứ $i$ bắn trúng mục tiêu.\\
		Ta có $\mathrm{P}(A_1)=0{,}7$; $\mathrm{P}(A_2)=0{,}6$; $\mathrm{P}(A_3)=0{,}5$.\\
		Gọi $B$ là biến cố có ít nhất một xa thủ bắn trúng mục tiêu. \\
		Suy ra $\overline{B}$ là biến cố không có xạ thủ nào bắn trúng mục tiêu.\\
		Do đó $\mathrm{P}(\overline{B})=\mathrm{P}(\overline{A_1}\, \overline{A_2}\, \overline{A_3})=\mathrm{P}(\overline{A_1})\, \mathrm{P}(\overline{A_2})\, \mathrm{P}(\overline{A_3})=0{,}3\cdot 0{,}4\cdot 0{,}5=0{,}06$.\\
		Vậy xác suất có ít nhất một xạ thủ bắn trúng mục tiêu là $\mathrm{P}(B)=0{,}94$.
	}
\end{ex}


\Closesolutionfile{ans}
\begin{center}
	\textbf{PHẦN 4 - TỰ LUẬN}
\end{center}
\setcounter{ex}{0}

\Opensolutionfile{ans}[ans-TL]
\begin{ex}%[1D9B1-3]
	Hai xạ thủ bắn mỗi người một viên đạn vào bia, biết xác suất bắn trúng vòng $10$ của xạ thủ thứ nhất là $0{,}75$ và của xạ thủ thứ hai là $0{,}85$. Tính xác suất để có ít nhất một xạ thủ bắn trúng vòng $10$.
	\loigiai{
		Gọi biến cố $A:$ \lq\lq Xạ thủ thứ nhất bắn trúng vòng $10$\rq\rq. Khi đó $\mathrm{P}(A)=0{,}75$.\\
		Biến cố $B:$ \lq\lq Xạ thủ thứ hai bắn trúng vòng $10$\rq\rq. Khi đó $\mathrm{P}(B)=0{,}85$.\\
		Gọi biến cố $C:$ \lq\lq Ít nhất một xạ thủ bắn trúng vòng $10$\rq\rq.\\
		Suy ra biến cố đối $\overline{C}:$ \lq\lq Không có xạ thủ nào bắn trúng vòng $10$\rq\rq.\\
		Khi đó $\overline{C}=\overline{A} \cap \overline{B}$.
		Do $A$ và $B$ là các biến cố độc lập nên $\overline{A}$ và $\overline{B}$ cũng độc lập.\\
		Do đó
		$$\mathrm{P}(\overline{C}) = \mathrm{P}(\overline{A} \cap \overline{B}) = \mathrm{P}(\overline{A})\cdot \mathrm{P}(\overline{B}) = (1-0{,}75)\cdot (1-0{,}85) = 0{,}0375.$$
		Vậy $\mathrm{P}(C) = 1 - \mathrm{P}(\overline{C}) = 1 - 0{,}0375 = 0{,}9625$.
	}
\end{ex}
\begin{ex}%[1D9C2-4]
	Khi khảo sát một lớp học có màn hình thông minh, người ta thấy có $65\%$ học sinh thích xem bóng đá và $48\%$ học sinh thích xem ca nhạc trong giờ nghỉ. Giả sử đặc điểm thích hay không thích xem bóng đá không ảnh hướng đến việc thích xem ca nhạc. Gặp ngẫu nhiên một học sinh của lớp, tính xác suất của biến cố học sinh đó không thích xem cả bóng đá và ca nhạc.
	\loigiai{
		Gọi $A$ là biến cố \lq\lq Gặp được học sinh thích xem bóng đá\rq\rq.\\
		$B$ là biến cố \lq\lq Gặp được học sinh thích xem ca nhạc\rq\rq.\\
		Theo giả thiết có $A$, $B$ là hai biến cố độc lập, $\mathrm{P}(A)=0{,}65$, $\mathrm{P}(B)=0{,}48$.\\
		Theo đó biến cố $C$: \lq\lq Học sinh đó thích xem bóng đá học ca nhạc\rq\rq\,  là $A\cup B$.\\
		Biến cố \lq\lq Học sinh đó không thích xem cả hai loại bóng đá hoặc ca nhạc\rq\rq\, là $\overline{C}$.\\
		Ta có
		$$\mathrm{\mathrm{P}}(A\cup B)=\mathrm{P}(A)+\mathrm{P}(B)-\mathrm{P}(AB)=\mathrm{P}(A)+\mathrm{P}(B)-\mathrm{P}(A)\cdot\mathrm{P}(B)=0{,}818.$$
		Suy ra $\mathrm{P}(\overline{C})=1-\mathrm{P}(C)=1-0{,}818=0{,}182$.
	}
\end{ex}
\begin{ex}%[1D9V1-2]
	Người ta dùng $28$ cuốn sách bao gồm $11$ cuốn sách toán, $9$ cuốn sách vật lí và $8$ cuốn sách hóa học (các cuốn sách cùng loại giống nhau) để làm phần thưởng cho $14$ học sinh (trong đó có $2$ học sinh An và Bình) mỗi học sinh nhận được $2$ cuốn sách khác thể loại (không tính thứ tự các cuốn sách). Tính xác suất để hai học sinh An và Bình nhận được phần thưởng giống nhau.
	\loigiai{
		Giả sử có $x$, $y$, $z$ lần lượt là số học sinh nhận phần thưởng là hai cuốn sách (Toán, Lý), (Toán, Hóa), (Lý, Hóa).\\
		Ta có $\heva{&x+y=11\\&x+z=9\\&y+z=8}\Leftrightarrow\heva{&x=6\\&y=5\\&z=3.}$\\
		Vậy có $6$ học sinh nhận sách Toán và Lý, $5$ học sinh nhận sách Toán và Hóa, $3$ học sinh nhận sách Lý và Hóa.\\
		Số cách trao thưởng cho $14$ học sinh là $\mathrm{C}_{14}^6\cdot \mathrm{C}_8^5\cdot \mathrm{C}_3^3=168\,168$.\\
		Ta có $n(\Omega)=168\,168$.\\
		Gọi $A$ là biến cố để hai học sinh An và Bình nhận được phần thưởng giống nhau.
		\begin{itemize}
			\item An và Bình nhận sách Toán và Lý có $\mathrm{C}_{12}^4\cdot \mathrm{C}_8^5\cdot \mathrm{C}_3^3=27\,720$.
			\item An và Bình nhận sách Toán và Hóa có $\mathrm{C}_{12}^3\cdot \mathrm{C}_9^6\cdot \mathrm{C}_3^3=18\,480$.
			\item An và Bình nhận sách Lý và Hóa có $\mathrm{C}_{12}^1\cdot \mathrm{C}_{11}^6\cdot \mathrm{C}_5^5=5\,544$.
		\end{itemize}
		Suy ra $n(A)=51\,744 $.\\
		Vậy xác suất để hai học sinh An và Bình nhận được phần thưởng giống nhau là \[\mathrm{P}=\dfrac{n(A)}{n(\Omega)}=\dfrac{51\,744}{168\,168}=\dfrac{4}{13}.\]
		}
\end{ex}
\Closesolutionfile{ans}
