\newpage
\section{Giới hạn hàm số}
\subsection{LÝ THUYẾT CẦN NHỚ}
\subsubsection{Giới hạn hàm số tại một điểm}
\begin{dn}
	Cho khoảng $K$ chứa $x_0$ và hàm số $y=f(x)$ xác định trên $K$ hoặc $K\setminus \left\{x_0 \right\}$.\\
	Ta nói hàm số $y=f(x)$ có giới hạn là số thực $L$ khi $x$ dần tới $x_0$ nếu với dãy số $\left(x_n \right)$ bất kì, $x_n \in K\setminus \left\{x_0 \right\}$ và $x_n \to x_0$ ta có $f(x)\to L$.\\
	Kí hiệu $\lim\limits_{x\to x_0} f(x)=L$.
\end{dn}
\begin{nx}
	$\lim\limits_{x\to x_0} x=x_0$; $\lim\limits_{x\to x_0} c=c$ (với $c$ là hằng số).
\end{nx}
\begin{dl}
	Giả sử $\lim\limits_{x\to x_0} f(x)=L$ và $\lim\limits_{x\to x_0} g(x)=M$ khi đó:
	\begin{enumEX}[$\bullet$]{1}
		\item $\lim\limits_{x\to x_0} f(x)+\lim\limits_{x\to x_0} g(x)=L+M$.
		\item $\lim\limits_{x\to x_0} \left[f(x).g(x)\right]=L.M$.
		\item $\lim\limits_{x\to x_0} f(x)-\lim\limits_{x\to x_0} g(x)=L-M$.
		\item $\lim\limits_{x\to x_0} \dfrac{f(x)}{g(x)}=\dfrac{L}{M}\, (\text{nếu } M\ne 0)$.
		\item Nếu $f(x)\ge 0$ và $\lim\limits_{x\to x_0} f(x)=L$ thì $L\ge 0$ và $\lim\limits_{x\to x_0}=\sqrt{f(x)}=\sqrt{L}$
		(dấu của $f(x)$ được xét trên khoảng đang tìm giới hạn, với $x\ne x_0$).
	\end{enumEX}
\end{dl}
\subsubsection{Giới hạn một bên}
\begin{itemize}
	\item Cho hàm số $y=f(x)$ xác định trên khoảng $\left(x_0;b\right)$. Số thực $L$ được gọi là giới hạn bên phải của hàm số $f(x)$ khi $x\to x_0$ nếu với mọi dãy số $\left(x_n \right)$ bất kì, $x_0 < x_n < b$ và $x_n \to x_0$ ta có $\lim f\left(x_n \right)=L$.
	Kí hiệu: ${\mathop{\lim}\limits_{x\to x_0 ^{+}}} f(x)=L$ hoặc $f(x)\to L$ khi $x\to x_0 ^{+}$.
	\item Cho hàm số $y=f(x)$ xác định trên khoảng $\left(a;x_0 \right)$. Số thực $L$ được gọi là giới hạn bên trái của hàm số $f(x)$ khi $x\to x_0$ nếu với mọi dãy số $\left(x_n \right)$ bất kì, $a < x_n < x_0$ và $x_n \to x_0$ ta có $\lim f\left(x_n \right)=L$.
	Kí hiệu: ${\mathop{\lim}\limits_{x\to x_0 ^{-}}} f(x)=L$ hoặc $f(x)\to L$ khi $x\to x_0 ^{-}$.
\end{itemize}
\begin{dl}
	$\lim\limits_{x\to x_0} f(x)=L\Leftrightarrow {\mathop{\lim}\limits_{x\to x_0 ^{-}}} f(x)={\mathop{\lim}\limits_{x\to x_0 ^{+}}} f(x)=L$.\\
	Nguyên lý kẹp: Cho ba hàm số $f(x)$, $g(x)$, $h(x)$ xác định trên $K$ chứa điểm $x_0$.
	Nếu $g(x)\le f(x)\le h(x)$, $\forall x\in K$ và $\lim\limits_{x\to x_0} g(x)=\lim\limits_{x\to x_0} h(x)=L$ thì $\lim\limits_{x\to x_0} f(x)=L$.
\end{dl}
\subsubsection{Giới hạn hữu hạn của hàm số tại vô cực}
\begin{dn}
	\item 
	\begin{itemize}
		\item Cho hàm số $y=f(x)$ xác định trên khoảng $\left(a;+\infty \right)$. Ta nói hàm số $y=f(x)$ có giới hạn là số thực $L$ khi $x\to+\infty$ nếu với dãy số $\left(x_n \right)$ bất kỳ, $x_n > a$ và $x_n \to+\infty$ ta có $f\left(x_n \right)\to L$.\\
		Kí hiệu: $\lim\limits_{x\to  +\infty}f(x)=L$ hay $f(x)\to L$ khi $x\to+\infty$.
		\item Cho hàm số $y=f(x)$ xác định trên khoảng $\left(-\infty;a\right)$. Ta nói hàm số $y=f(x)$ có giới hạn là số thực $L$ khi $x\to-\infty$ nếu với dãy số $\left(x_n \right)$ bất kỳ, $x_n < a$ và $x_n \to-\infty$ ta có $f\left(x_n \right)\to L$.\\
		Kí hiệu: $\lim\limits_{x\to  -\infty} f(x)=L$ hay $f(x)\to L$ khi $x\to-\infty$.
	\end{itemize}
\end{dn}
\begin{luuy}
	\begin{itemize}
		\item Với $c$, $k$ là các hằng số và $k$ nguyên dương, ta luôn có: ${\mathop{\lim}\limits_{x\to \pm \infty}} c=c$; ${\mathop{\lim}\limits_{x\to \pm \infty}} \dfrac{c}{x^k}=0$.
		\item Định lý về giới hạn hữu hạn của hàm số khi $x\to x_0$ vẫn còn đúng khi $x\to \pm \infty$.
	\end{itemize}
\end{luuy}
\subsubsection{Giới hạn vô cực của hàm số}
Các định nghĩa về giới hạn $+\infty$ (hoặc $-\infty$) của hàm số được phát biểu tương tự các định nghĩa về giới hạn hữu hạn.
Chẳng hạn, giới hạn $-\infty$ của hàm số $y=f(x)$ khi $x$ dần tới $+\infty$ được định nghĩa như sau:\\
\textbf{Định nghĩa giới hạn vô cực:}\\ Cho hàm số $y=f(x)$ xác định trên khoảng $\left(a;+\infty \right)$. Ta nói hàm số $y=f(x)$ có giới hạn $-\infty$ khi $x$ dần tới dương vô cực nếu với dãy số $\left(x_n \right)$ bất kì, $x_n > a$, $x_n \to+\infty$, ta có $\left(f\left(x_n \right)\right)\to-\infty$.\\ Kí hiệu: $\lim\limits_{x\to  +\infty}f(x)=-\infty$.
\begin{nx}
	$\lim\limits_{x\to  +\infty}f(x)=+\infty \Leftrightarrow \lim\limits_{x\to  +\infty}\left(-f(x)\right)=-\infty$.
\end{nx}
\begin{tc}
	\item 
	\begin{itemize}
		\item $\lim\limits_{x\to  +\infty}x^k=+\infty$ với $k$ nguyên dương.
		\item $\lim\limits_{x\to  -\infty} x^k=-\infty$ với $k$ là số lẻ.
		\item $\lim\limits_{x\to  -\infty} x^k=+\infty$ với $k$ là số chẵn.
		\item $\lim\limits_{x\to a^+}\dfrac{1}{x-a}=+\infty$ và $\lim\limits_{x\to a^-}\dfrac{1}{x-a}=-\infty$, $a\in\mathbb{R}$.
	\end{itemize}
\end{tc}
\subsubsection{Quy tắc tìm giới hạn vô cực của hàm số}
Các định lí sau vẫn đúng cho các trường hợp $x\to x_0^{+}$, $x\to x_0^{-}$, $x\to+\infty$, $x\to-\infty$.
\begin{enumerate}
	\item Quy tắc tìm giới hạn của tích $f(x)\cdot g(x)$:\\
	Nếu $\lim\limits_{x\to x_0} f(x)=L\ne 0$ và $\lim\limits_{x\to x_0} g(x)=\pm \infty$ thì $\lim\limits_{x\to x_0} \left[f(x)g(x)\right]$ được tính theo quy tắc trong bảng sau:
	\begin{center}
		\begin{tabular}{|c|p{0.6in}|c|}
		\hline
		Dấu của L & $\lim\limits_{x\to x_0} f(x)$ & $\lim\limits_{x\to x_0} \left[f(x).g(x)\right]$ \\
		\hline
		$+$ & $+\infty$ & $+\infty$ \\
		\hline
		$+$ & $-\infty$ & $-\infty$ \\
		\hline
		$-$ & $+\infty$ & $-\infty$ \\
		\hline
		$-$ & $-\infty$ & $+\infty$ \\
		\hline
	\end{tabular}
	\end{center}
	\item Quy tắc tìm giới hạn của thương $\dfrac{f(x)}{g(x)}$:\\
	Nếu $\lim\limits_{x\to x_0} f(x)=L$ và $\lim\limits_{x\to x_0} g(x)=\pm \infty$ thì $\lim\limits_{x\to x_0} \dfrac{f(x)}{g(x)}=0$.\\
	Nếu $\lim\limits_{x\to x_0} f(x)=L\ne 0$ và $\lim\limits_{x\to x_0} g(x)=0$ và $g(x)>0$ hoặc $g(x) < 0$ với mọi $x\ne x_0$ thì $\lim\limits_{x\to x_0} \dfrac{f(x)}{g(x)}$ được tính theo quy tắc trong bảng sau:
\begin{center}
		\begin{tabular}{|c|c|c|}
		\hline
		Dấu của L & Dấu của $g(x)$ & $\lim\limits_{x\to x_0} \dfrac{f(x)}{g(x)}$ \\
		\hline
		$+$ & $+\infty$ & $+\infty$ \\
		\hline
		$+$ & $-\infty$ & $-\infty$ \\
		\hline
		$-$ & $+\infty$ & $-\infty$ \\
		\hline
		$-$ & $-\infty$ & $+\infty$ \\
		\hline
	\end{tabular}
\end{center}
\end{enumerate}

%-------------------------------------------------------------------------------------------------------------
\subsection{PHÂN LOẠI VÀ PHƯƠNG PHÁP GIẢI TOÁN}
\begin{dang}{Tính giới hạn hàm số tại một điểm (không vô định)}
	Phương pháp: thay trực tiếp.
\end{dang}
\begin{vd}
	Tính các giới hạn sau đây
	\begin{listEX}[4]
		\item $\lim\limits_{x\to -3}\dfrac{x^2-9}{x-3}$.
		\item $\lim\limits_{x\to -2} \dfrac{x^2-5x+2}{2\left|x\right|+1}$.
		\item $\lim\limits_{x\to -1} \dfrac{\sqrt{3x^2+1}-x}{x-1}$.
		\item $\lim\limits_{x\to \tfrac{\pi}{2}} \dfrac{\sin \left(x-\dfrac{\pi}{4} \right)}{x}$.
	\end{listEX}
	\loigiai{
		\begin{listEX}[1]
			\item $\lim\limits_{x\to -3}\dfrac{x^2-9}{x-3}=\dfrac{\left(-3\right)^2-9}{-3-3}=0$.
			\item $\lim\limits_{x\to -2} \dfrac{x^2-5x+2}{2\left|x\right|+1}=\dfrac{\left(-2\right)^2-5\left(-2\right)+2}{2\left|-2\right|+1}=\dfrac{16}{5}$.
			\item $\lim\limits_{x\to -1} \dfrac{\sqrt{3x^2+1}-x}{x-1}=\dfrac{\sqrt{3\left(-1\right)^2+1}-\left(-1\right)}{\left(-1\right)-1}=\dfrac{-3}{2}$.
			\item $\lim\limits_{x\to \tfrac{\pi}{2}} \dfrac{\sin \left(x-\dfrac{\pi}{4} \right)}{x}=\dfrac{\sin \left(\dfrac{\pi}{2}-\dfrac{\pi}{4} \right)}{\dfrac{\pi}{2}}=\dfrac{\sqrt{2}}{\pi}$.
		\end{listEX}
	}
\end{vd}
\begin{dang}{Tính giới hạn hàm số tại một điểm (vô định $\dfrac{0}{0}$)}
		\begin{itemize}
			\item Khử dạng vô định $\dfrac{0}{0}$: phân tích tử và mẫu thành nhân tử với nhân tử chung là $x-x_0$.
		\item Giả sử $f(x)=\left(x-x_0 \right)\cdot f_1 (x)$ và $g(x)=\left(x-x_0 \right)\cdot g_1 (x)$.
		\item Khi đó $\lim\limits_{x\to x_0} \dfrac{f(x)}{g(x)}=\lim\limits_{x\to x_0} \dfrac{f_1 (x)}{g_1 (x)}$.
		\end{itemize}
	\begin{nx}
		Nếu giới hạn $\lim\limits_{x\to x_0} \dfrac{f_1 (x)}{g_1 (x)}$ vẫn ở dạng vô định $\dfrac00$ thì ta lặp lại quá trình trên cho đến khi không còn dạng vô định.
		Việc phân tích thành nhân tử ở trên được thực hiện bằng phương pháp chia Horner.
	\end{nx}
\end{dang}

\begin{vd}
	Tính các giới hạn sau đây
	\begin{listEX}[3]
		\item $\lim\limits_{x\to -4}\dfrac{x^2+2x-8}{x^2+4x}$.
		\item $\lim\limits_{x\to  \frac{1}{2}}\dfrac{2x^2-5x+2}{1-2x}$. \item $\lim\limits_{x\to  2}\dfrac{2x^2-5x+2}{x^2+x-6}$.
		\item $\lim\limits_{x\to -1} \dfrac{1+x^3}{1-x^2}$.
		\item $\lim\limits_{x\to -3}\dfrac{x+3}{x^2-9}$.
		\item $\lim\limits_{x\to  2}\dfrac{x^2+x-6}{2x^2-5x+2}$.
	\end{listEX}
	\loigiai{
		\begin{listEX}[2]
			\item \allowdisplaybreaks
			\begin{eqnarray*}
				\lim\limits_{x\to -4}\dfrac{x^2+2x-8}{x^2+4x}&=&\lim\limits_{x\to -4}\dfrac{\left(x+4\right)\left(x-2\right)}{x\left(x+4\right)}\\
				&=&\lim\limits_{x\to -4}\dfrac{x-2}{x}\\
				&=&\dfrac{-4-2}{-4}=\dfrac{3}{2}.
			\end{eqnarray*}
			\item \allowdisplaybreaks
			\begin{eqnarray*}
				\lim\limits_{x\to  \dfrac{1}{2}}\frac{2x^2-5x+2}{1-2x}&=&\lim\limits_{x\to  \frac{1}{2}}\dfrac{\left(2x-1\right)\left(x-2\right)}{1-2x}\\
				&=&\lim\limits_{x\to  \frac{1}{2}}\left(2-x\right)\\
				&=&2-\dfrac{1}{2}=\dfrac{3}{2}.
			\end{eqnarray*}
			\item \allowdisplaybreaks
			\begin{eqnarray*}
				\lim\limits_{x\to  2}\dfrac{2x^2-5x+2}{x^2+x-6}&=&\lim\limits_{x\to  2}\dfrac{\left(x-2\right)\left(2x-1\right)}{\left(x-2\right)\left(x+3\right)}\\
				&=&\lim\limits_{x\to  2}\dfrac{2x-1}{x+3}\\
				&=&\dfrac{2\cdot 2-1}{2+3}=\dfrac{3}{5}.
			\end{eqnarray*}
			\item \allowdisplaybreaks
			\begin{eqnarray*}
				\lim\limits_{x\to -1} \dfrac{1+x^3}{1-x^2}&=&\lim\limits_{x\to -1} \dfrac{\left(1+x\right)\left(1-x+x^2\right)}{\left(1+x\right)\left(1-x\right)}\\
				&=&\lim\limits_{x\to -1} \dfrac{1-x+x^2}{1-x}\\
				&=&\dfrac{1-\left(-1\right)+\left(-1\right)^2}{1-\left(-1\right)}=\dfrac{3}{2}.
			\end{eqnarray*}
			\item \allowdisplaybreaks
			\begin{eqnarray*}
				\lim\limits_{x\to -3}\dfrac{x+3}{x^2-9}&=&\lim\limits_{x\to -3}\dfrac{x+3}{\left(x+3\right)\left(x-3\right)}\\
				&=&\lim\limits_{x\to -3}\dfrac{1}{x-3}\\
				&=&-\dfrac{1}{6}.
			\end{eqnarray*}
			\item \allowdisplaybreaks
			\begin{eqnarray*}
				\lim\limits_{x\to  2}\dfrac{x^2+x-6}{2x^2-5x+2}&=&\lim\limits_{x\to  2}\dfrac{\left(x-2\right)\left(x+3\right)}{\left(x-2\right)\left(2x-1\right)}\\
				&=&\lim\limits_{x\to  2}\dfrac{x+3}{2x-1}\\
				&=&\dfrac{5}{3}.
			\end{eqnarray*}
		\end{listEX}
	}
\end{vd}


\begin{dang}{Tính giới hạn hàm số tại một điểm (vô định $\dfrac{0}{0}$) mà cả tử số và mẫu số có căn thức.}
		\begin{itemize}
			\item Biểu thức có dạng $\lim\limits_{x\to x_0} \dfrac{f(x)}{g(x)}$, trong đó $f(x)$, $g(x)$ là các căn thức và $f\left(x_0 \right)=g\left(x_0 \right)=0$.
			\item Khử dạng vô định $\dfrac00$: nhân cả tử và mẫu với biểu thức liên hợp tương ứng của biểu thức chứa căn để trục các nhân tử $x-x_0$ ra khỏi căn thức, nhằm khử các thành phần có giới hạn bằng $0$.
		\end{itemize}
		\begin{nx}
			Có thể nhân liên hợp một hoặc nhiều lần để khử dạng vô định.
		\end{nx}
		\begin{luuy}
			Các hằng đẳng thức
			\begin{itemize}
				\item $A^2-B^2=\left(A-B\right)\left(A+B\right)$.
				\item $A^3-B^3=\left(A-B\right)\left(A^2+AB+B^2\right)$.
				\item $A^3+B^3=\left(A+B\right)\left(A^2-AB+B^2\right)$.
			\end{itemize}
		\end{luuy}
\end{dang}
\begin{vd}
	Tính các giới hạn sau đây
	\begin{listEX}[4]
		\item $\lim\limits_{x\to 0}\dfrac{\sqrt{1+2x}-1}{2x}$.
		\item $\lim\limits_{x\to 0}\dfrac{4x}{\sqrt{9+x}-3}$.
		\item $\lim\limits_{x\to 1}\dfrac{\sqrt{2x+7}-3}{2-\sqrt{x+3}}$.
		\item $\lim\limits_{x\to 0}\dfrac{\sqrt{1+x^2}-1}{\sqrt{x^2+16}-4}$.
		\item $\lim\limits_{x\to  2}\dfrac{\sqrt{x+2}-\sqrt{2x}}{\sqrt{x-1}-\sqrt{3-x}}$.
		\item $\lim\limits_{x\to 5} \dfrac{2x-5\sqrt{x-1}}{3-\sqrt{x+4}}$.
		\item $\lim\limits_{x\to 0}\dfrac{1-\sqrt[3]{12x+1}}{4x}$.
	\end{listEX}
	\loigiai{
		\begin{listEX}[2]
			\item \allowdisplaybreaks
			\begin{eqnarray*}
				\lim\limits_{x\to 0}\dfrac{\sqrt{1+2x}-1}{2x}&=&\lim\limits_{x\to 0}\dfrac{2x}{2x\left(\sqrt{1+2x}+1\right)}\\
				&=&\lim\limits_{x\to 0}\dfrac{1}{\left(\sqrt{1+2x}+1\right)}\\
				&=&\dfrac{1}{2}.
			\end{eqnarray*}
			\item \allowdisplaybreaks
			\begin{eqnarray*}
				\lim\limits_{x\to 0}\dfrac{4x}{\sqrt{9+x}-3}&=&\lim\limits_{x\to 0}\dfrac{4x\left(\sqrt{9+x}+3\right)}{x}\\
				&=&\lim\limits_{x\to 0}\left[4\left(\sqrt{9+x}+3\right)\right]\\
				&=&24.
			\end{eqnarray*}
			\item \allowdisplaybreaks
			\begin{eqnarray*}
				\lim\limits_{x\to 1}\dfrac{\sqrt{2x+7}-3}{2-\sqrt{x+3}}&=&\lim\limits_{x\to 1}\dfrac{2\left(x-1\right)\left(2+\sqrt{x+3} \right)}{-\left(x-1\right)\left(\sqrt{2x+7}+3\right)}\\
				&=&\lim\limits_{x\to 1}\dfrac{2\left(2+\sqrt{x+3} \right)}{-\left(\sqrt{2x+7}+3\right)}\\
				&=&-\dfrac{4}{3}.
			\end{eqnarray*}
			\item \allowdisplaybreaks
			\begin{eqnarray*}
				\lim\limits_{x\to 0}\dfrac{\sqrt{1+x^2}-1}{\sqrt{x^2+16}-4}&=&\lim\limits_{x\to 0}\dfrac{x^2\left(\sqrt{x^2+16}+4\right)}{x^2\left(\sqrt{1+x^2}+1\right)}\\
				&=&\lim\limits_{x\to 0}\dfrac{\sqrt{x^2+16}+4}{\sqrt{1+x^2}+1}\\
				&=&4.
			\end{eqnarray*}
			\item \allowdisplaybreaks
			\begin{eqnarray*}
				\lim\limits_{x\to  2}\dfrac{\sqrt{x+2}-\sqrt{2x}}{\sqrt{x-1}-\sqrt{3-x}}&=&\lim\limits_{x\to  2}\dfrac{\left(2-x\right)\left(\sqrt{x-1}+\sqrt{3-x} \right)}{2\left(x-2\right)\left(\sqrt{x+2}+\sqrt{2x} \right)}\\
				&=&\lim\limits_{x\to  2}\dfrac{\sqrt{x-1}+\sqrt{3-x}}{2\left(\sqrt{x+2}+\sqrt{2x} \right)}\\
				&=&\dfrac{1}{4}.
			\end{eqnarray*}
			\item \allowdisplaybreaks
			\begin{eqnarray*}
				\lim\limits_{x\to 5} \dfrac{2x-5\sqrt{x-1}}{3-\sqrt{x+4}}&=&\lim\limits_{x\to 5} \dfrac{\left[4x^2-25\left(x-1\right)\right]\left(3+\sqrt{x+4} \right)}{\left(9-\left(x+4\right)\right)\left(2x+5\sqrt{x-1} \right)}\\
				&=&\lim\limits_{x\to 5} \dfrac{\left(4x^2-25x+25\right)\left(3+\sqrt{x+4} \right)}{\left(5-x\right)\left(2x+5\sqrt{x-1} \right)}\\
				&=&\lim\limits_{x\to 5} \dfrac{\left(x-5\right)\left(4x-5\right)\left(3+\sqrt{x+4} \right)}{\left(5-x\right)\left(2x+5\sqrt{x-1} \right)}\\
				&=&\lim\limits_{x\to 5} \dfrac{\left(5-4x\right)\left(3+\sqrt{x+4} \right)}{2x+5\sqrt{x-1}}\\
				&=&\dfrac{(5-4.5)\left(3+\sqrt{5+4} \right)}{2.5+5\sqrt{5-1}}\\
				&=&-\dfrac{9}{2}.
			\end{eqnarray*}
			\item \allowdisplaybreaks
			\begin{eqnarray*}
				\lim\limits_{x\to 0}\dfrac{1-\sqrt[3]{12x+1}}{4x}&=&\lim\limits_{x\to 0}\dfrac{1-\left(12x+1\right)}{4x\left[1+\sqrt[3]{12x+1}+\sqrt[3]{\left(12x+1\right)^2} \right]}\\
				&=&\lim\limits_{x\to 0}\dfrac{-12x}{4x\left[1+\sqrt[3]{12x+1}+\sqrt[3]{\left(12x+1\right)^2} \right]}\\
				&=&\lim\limits_{x\to 0}\dfrac{-3}{1+\sqrt[3]{12x+1}+\sqrt[3]{\left(12x+1\right)^2}}\\
				&=&\dfrac{-3}{1+\sqrt[3]{12\cdot 0+1}+\sqrt[2]{\left(12\cdot 0+1\right)^2}}\\
				&=&-1.
			\end{eqnarray*}
		\end{listEX}
	}
\end{vd}
\begin{dang}{Tính giới hạn của hàm số tại vô cực}
	\begin{enumerate}[\bf 1)]
		\item $I=\lim\limits_{x\to \infty} \dfrac{P(x)}{Q(x)}$ với $P(x)$, $Q(x)$ là các đa thức hoặc các hàm số.
		\begin{itemize}
			\item Gọi $p$, $q$ lần lượt là bậc của đa thức $P(x)$ ,$Q(x)$ và $m=\max \{p, q\}$.
			\item Chia cả tử và mẫu cho $x^m$ ta có kết luận.\\ Khi đó
			\begin{itemize}
				\item Nếu $p\le q$ thì tồn tại giới hạn.
				\item Nếu $p > q$ thì không tồn tại giới hạn.
			\end{itemize}
		\end{itemize}
	\item Giới hạn $\infty-\infty$.\newline Sử dụng các biểu thức liên hợp đưa về dạng $\dfrac{\infty}{\infty}$ hoặc đưa về dạng tích các đa thức.
	\item Giới hạn $0\cdot \infty$.\newline Sử dụng các biểu thức liên hợp đưa về dạng $\dfrac{\infty}{\infty}$ \newline Các công thức liên hợp thường gặp
	\begin{multicols}{2}
		\begin{itemize}
			\item $\sqrt{A}-\sqrt{B}=\dfrac{A-B}{\sqrt{A}+\sqrt{B}}$.
			\item $\sqrt{A}-B=\dfrac{A-B^2}{\sqrt{A}+B}$.
			\item $\sqrt[3]A-\sqrt[3]B=\dfrac{A-B}{\left(\sqrt[3]A\right)^2+\sqrt[3]A\cdot\sqrt[3]B+\left(\sqrt[3]B\right)^2}$.
			\item $ \sqrt[3]A-B=\dfrac{A-B^3}{\left(\sqrt[3]A\right)^2+\sqrt[3]A\cdot B+\left(B\right)^2}$
		\end{itemize}
	\end{multicols}
	\end{enumerate}
	
\end{dang}
\begin{vd}
	Tính các giới hạn sau đây
	\begin{listEX}[2]
		\item $A=\lim\limits_{x\to  +\infty}\dfrac{2x^3-3x^2+4x+1}{-5x^3+2x^2-x+3}$.
		\item $B=\lim\limits_{x\to  -\infty} \dfrac{x+\sqrt{x^2+2}}{\sqrt[3]{8x^3+x^2+1}}$.
		\item $D=\lim\limits_{x\to  +\infty}x\left(\sqrt{x^2+1}-x\right)$.
		\item $E=\lim\limits_{x\to  +\infty}\dfrac{3x^2-x+7}{2x^3-1}$.
		\item $G=\lim\limits_{x\to  +\infty}\dfrac{3x^2-x+3}{x-4}$.
		\item $H=\lim\limits_{x\to  -\infty} \dfrac{2x^3-2x+3}{5-x}$.
	\end{listEX}
\loigiai{
	\begin{listEX}[2]
		\item Ta có \allowdisplaybreaks
		\begin{eqnarray*}
			A&=&\lim\limits_{x\to  +\infty}\dfrac{2x^3-3x^2+4x+1}{-5x^3+2x^2-x+3}\\
			&=&\lim\limits_{x\to  +\infty}\dfrac{2-\dfrac{3}{x}+\dfrac{4}{x^2}+\dfrac{1}{x^3}}{-5+\dfrac{2}{x}-\dfrac{1}{x^2}+\dfrac{3}{x^3}}\\
			&=&-\dfrac{2}{5}.
		\end{eqnarray*}
		\item Ta có \allowdisplaybreaks
		\begin{eqnarray*}
			B&=&\lim\limits_{x\to  -\infty} \dfrac{x+\sqrt{x^2+2}}{\sqrt[3]{8x^3+x^2+1}}\\
			&=&\lim\limits_{x\to  -\infty} \dfrac{x+\left|x\right|\sqrt{1+\dfrac{2}{x^2}}}{x\sqrt[3]{8+\dfrac{1}{x}+\dfrac{1}{x^3}}}\\
			&=&\lim\limits_{x\to  -\infty} \dfrac{1-\sqrt{1+\dfrac{2}{x^2}}}{\sqrt[3]{8+\dfrac{1}{x}+\dfrac{1}{x^3}}}\\
			&=&\dfrac{0}{2}=0.
		\end{eqnarray*}
		\item Ta có \allowdisplaybreaks
		\begin{eqnarray*}
			D&=&\lim\limits_{x\to  +\infty}x\left(\sqrt{x^2+1}-x\right)\\
			&=&\lim\limits_{x\to  +\infty}\dfrac{x\left(x^2+1-x^2\right)}{\sqrt{x^2+1}+x}\\
			&=&\lim\limits_{x\to  +\infty}\dfrac{x}{x\left(\sqrt{1+\dfrac{1}{x^2}}+1\right)}\\
			&=&\lim\limits_{x\to  +\infty}\dfrac{1}{\sqrt{1+\dfrac{1}{x^2}}+1}\\
			&=&\dfrac{1}{2}.
		\end{eqnarray*}
		\item Ta có \allowdisplaybreaks
		\begin{eqnarray*}
			E&=&\lim\limits_{x\to  +\infty}\dfrac{x^3\left(\dfrac{3}{x}-\dfrac{1}{x^2}+\dfrac{7}{x^3} \right)}{x^3\left(2-\dfrac{1}{x^3} \right)}\\
			&=&\lim\limits_{x\to  +\infty}\dfrac{\dfrac{3}{x}-\dfrac{1}{x^2}+\dfrac{7}{x^3}}{2-\dfrac{1}{x^3}}\\
			&=&\dfrac{0-0+0}{2-0}=0.
		\end{eqnarray*}
		\item Ta có \allowdisplaybreaks
		\begin{eqnarray*}
			F&=&\lim\limits_{x\to  -\infty} \dfrac{x^3\left(2+\dfrac{3}{x^2}-\dfrac{4}{x^3} \right)}{x^3\left(-1-\dfrac{1}{x}+\dfrac{1}{x^3} \right)}\\
			&=&\lim\limits_{x\to  -\infty} \dfrac{\left(2+\dfrac{3}{x^2}-\dfrac{4}{x^3} \right)}{\left(-1-\dfrac{1}{x}+\dfrac{1}{x^3} \right)}\\
			&=&\dfrac{2+0-0}{-1-0+0}=-2.
		\end{eqnarray*}
		\item Ta có \allowdisplaybreaks
		\begin{eqnarray*}
			G&=&\lim\limits_{x\to  +\infty}\dfrac{x^2\left(3-\dfrac{1}{x}+\dfrac{3}{x^2} \right)}{x\left(1-\dfrac{4}{x} \right)}\\
			&=&\lim\limits_{x\to  +\infty}\left(x.\dfrac{3-\dfrac{1}{x}+\dfrac{3}{x^2}}{1-\dfrac{4}{x}} \right)\\
			&=&+\infty,\text{ vì }\lim\limits_{x\to  +\infty}x=+\infty\text{ và }\lim\limits_{x\to  +\infty}\dfrac{3-\dfrac{1}{x}+\dfrac{3}{x^2}}{1-\dfrac{4}{x}}=3.
		\end{eqnarray*}
		\item Ta có \allowdisplaybreaks
		\begin{eqnarray*}
			H&=&\lim\limits_{x\to  -\infty} \dfrac{x^3\left(2-\dfrac{2}{x^2}+\dfrac{3}{x^3} \right)}{x\left(\dfrac{5}{x}-1\right)}\\
			&=&\lim\limits_{x\to  -\infty} \left(x^2\cdot\dfrac{2-\dfrac{2}{x^2}+\dfrac{3}{x^3}}{\dfrac{5}{x}-1} \right)\\
			&=&-\infty,\text{ vì }\lim\limits_{x\to  -\infty} x^2=+\infty \text{ và }\lim\limits_{x\to  -\infty} \dfrac{2-\dfrac{2}{x^2}+\dfrac{3}{x^3}}{\dfrac{5}{x}-1}=-2.
		\end{eqnarray*}
	\end{listEX}
}
\end{vd}

\begin{vd}
	Tính các giới hạn sau đây
	\begin{listEX}[2]
		\item $\lim\limits_{x\to  +\infty}\left(\sqrt{x^2+3x}-\sqrt{x^2+4x} \right)$.
		\item $\lim\limits_{x\to  +\infty}\left(\sqrt[3]{2x-1}-\sqrt[3]{2x+1} \right)$.
		\item $\lim\limits_{x\to  -\infty} \left(\sqrt{x^2+4x}-x\right)$.
		\item $\lim\limits_{x\to  +\infty}\left(\sqrt{x^2+4x}-2x\right)$.
	\end{listEX}
	\loigiai{
		\begin{enumerate}
			\item \allowdisplaybreaks
			\begin{eqnarray*}
				\lim\limits_{x\to  +\infty}\left(\sqrt{x^2+3x}-\sqrt{x^2+4x} \right)&=&\lim\limits_{x\to  +\infty}\dfrac{\left(x^2+3x\right)-\left(x^2+4x\right)}{\sqrt{x^2+3x}+\sqrt{x^2+4x}}\\
				&=&\lim\limits_{x\to  +\infty}\dfrac{-x}{\left|x\right|\sqrt{1+\dfrac{3}{x}}+\left|x\right|\sqrt{1+\dfrac{4}{x}}}\\
				&=&\lim\limits_{x\to  +\infty}\dfrac{-1}{\sqrt{1+\dfrac{3}{x}}+\sqrt{1+\dfrac{4}{x}}}\\
				&=&-\dfrac{1}{2}.
			\end{eqnarray*}
			\item \allowdisplaybreaks
			\begin{eqnarray*}
				\lim\limits_{x\to  +\infty}\left(\sqrt[3]{2x-1}-\sqrt[3]{2x+1} \right)&=&\lim\limits_{x\to  +\infty}\dfrac{2x-1-2x-1}{\sqrt[3]{\left(2x-1\right)^2}+\sqrt[3]{\left(2x-1\right)\left(2x+1\right)}+\sqrt[3]{\left(2x+1\right)^2}}\\
				&=&\lim\limits_{x\to  +\infty}\dfrac{-2}{\sqrt[3]{\left(2x-1\right)^2}+\sqrt[3]{\left(2x-1\right)\left(2x+1\right)}+\sqrt[3]{\left(2x+1\right)^2}}\\
				&=&\lim\limits_{x\to  +\infty}\dfrac{-2}{\sqrt[3]{x^2\left(2-\dfrac{1}{x} \right)^2}+\sqrt[3]{x^2\left(4-\dfrac{1}{x^2} \right)}+\sqrt[3]{x^2\left(2+\dfrac{1}{x} \right)^2}}\\
				&=&\lim\limits_{x\to  +\infty}\dfrac{-2}{\sqrt[3]{x^2} \left[\sqrt[3]{\left(2-\dfrac{1}{x} \right)^2}+\sqrt[3]{4-\dfrac{1}{x^2}}+\sqrt[3]{\left(2+\dfrac{1}{x} \right)^2} \right]}\\
				&=&0.
			\end{eqnarray*}
			\item Ta có $\lim\limits_{x\to  -\infty} \left(\sqrt{x^2+4x}-x\right)=\lim\limits_{x\to  -\infty}\left[-x\cdot\left(\sqrt{1+\dfrac{4}{x}}+1\right)\right]$.\\
			Mà $\lim\limits_{x\to  -\infty}(-x)=+\infty$ và $\lim\limits_{x\to  -\infty}\left(\sqrt{1+\dfrac{4}{x^2}}+1\right)=2$.\\
			Do đó $\lim\limits_{x\to  +\infty}\left(2x+\sqrt{4x^2+1} \right)=+\infty$.
			\item  $\lim\limits_{x\to  +\infty}\left(\sqrt{x^2+4x}-2x\right)=\lim\limits_{x\to  +\infty}\left[x\cdot\left(\sqrt{1+\dfrac{4}{x}}-2\right)\right]$.\\
			Mà $\lim\limits_{x\to  +\infty}x=+\infty$ và $\lim\limits_{x\to  +\infty}\left(\sqrt{1+\dfrac{4}{x^2}}-2\right)=-1<0$.\\ 
			Do đó $\lim\limits_{x\to  +\infty}\left(\sqrt{x^2+4x}-2x\right)=-\infty$.
		\end{enumerate}
	}
\end{vd}
\begin{vd}
	Tính các giới hạn sau đây
	\begin{listEX}[2]
		\item $\lim\limits_{x\to  -\infty}\left(-x^3-2x^2+1\right)$.
		\item $\lim\limits_{x\to  -\infty}\left(x^3-2x^2+1\right)$.
		\item $\lim\limits_{x\to  +\infty}\left(-x^3-2x^2+1\right)$.
		\item $\lim\limits_{x\to  +\infty}\left(x^3-2x^2+1\right)$.
		\item $\lim\limits_{x\to  +\infty}\left(2x+\sqrt{4x^2+1} \right)$.
	\end{listEX}
	\loigiai{
		\begin{enumerate}
			\item Ta có $\lim\limits_{x\to  -\infty}\left(-x^3-2x^2+1\right)=\lim\limits_{x\to  -\infty}\left[x^3\left(-1-\dfrac{2}{x}+\dfrac{1}{x^3}\right)\right]= +\infty$
			vì $\heva{&\lim\limits_{x\to  -\infty}x=-\infty\\&\lim\limits_{x\to  -\infty}\left(-1-\dfrac{2}{x}+\dfrac{1}{x^3}\right)=-1<0.} $
			\item Ta có $\lim\limits_{x\to  -\infty}\left(x^3-2x^2+1\right)=\lim\limits_{x\to  -\infty}\left[x^3\left(1-\dfrac{2}{x}+\dfrac{1}{x^3}\right)\right]= -\infty$
			vì $\heva{&\lim\limits_{x\to  -\infty}x=-\infty\\&\lim\limits_{x\to  -\infty}\left(1-\dfrac{2}{x}+\dfrac{1}{x^3}\right)=1>0.} $
			\item Ta có $\lim\limits_{x\to  +\infty}\left(-x^3-2x^2+1\right)=\lim\limits_{x\to  +\infty}\left[x^3\left(-1-\dfrac{2}{x}+\dfrac{1}{x^3}\right)\right]= -\infty$
			vì $\heva{&\lim\limits_{x\to  +\infty}x=+\infty\\&\lim\limits_{x\to  +\infty}\left(-1-\dfrac{2}{x}+\dfrac{1}{x^3}\right)=-1<0.} $
			\item Ta có $\lim\limits_{x\to  +\infty}\left(x^3-2x^2+1\right)=\lim\limits_{x\to  +\infty}\left[x^3\left(1-\dfrac{2}{x}+\dfrac{1}{x^3}\right)\right]= +\infty$
			vì $\heva{&\lim\limits_{x\to  +\infty}x=+\infty\\&\lim\limits_{x\to  +\infty}\left(1-\dfrac{2}{x}+\dfrac{1}{x^3}\right)=1>0.} $ 
			\item Ta có $\lim\limits_{x\to  +\infty}\left(2x+\sqrt{4x^2+1} \right)=\lim\limits_{x\to  +\infty}\left[2x\cdot\left(1+\sqrt{1+\dfrac{1}{4x^2}}\right)\right]$.\\\
			Mà $\lim\limits_{x\to  +\infty}(2x)=+\infty$ và $\lim\limits_{x\to  +\infty}\left(1+\sqrt{1+\dfrac{1}{4x^2}}\right)=2>0$.\\
			Do đó $\lim\limits_{x\to  +\infty}\left(2x+\sqrt{4x^2+1} \right)=+\infty$.
		\end{enumerate}}
	
\end{vd}
\textbf{Chú ý:} Với $L=\lim\limits_{x\to  \pm\infty}\left( a_{2n}x^{2n}+a_{2n-1}x^{2n-1}+a_{2n-2}x^{2n-2}+\cdots + a_{1}x+a_{0}\right) $.\\
\begin{itemize}
	\item Nếu $a_{2n}>0$ thì $L=+\infty$.
	\item Nếu $a_{2n}<0$ thì $L=-\infty$.
\end{itemize}
	Với $L=\lim\limits_{x\to  \pm\infty}\left( a_{2n+1}x^{2n+1}+a_{2n}x^{2n}+a_{2n-1}x^{2n-1}+\cdots + a_{1}x+a_{0}\right) $.
	\begin{itemize}
		\item Nếu $\heva{&x\to +\infty\\&a_{2n+1}>0 }$ thì $L=+\infty$.
		\item Nếu $\heva{&x\to -\infty\\&a_{2n+1}>0 }$ thì $L=-\infty$.
		\item Nếu $\heva{&x\to -\infty\\&a_{2n+1}<0 }$ thì $L=+\infty$.
		\item Nếu $\heva{&x\to +\infty\\&a_{2n+1}<0 }$ thì $L=-\infty$.
	\end{itemize}
\begin{dang}{Giới hạn một bên của hàm số}
	Nếu $\lim\limits_{x\to x_0} f(x)=L\ne 0$ và $\lim\limits_{x\to x_0} g(x)=+ \infty$ thì 
	\begin{multicols}{2}
		\begin{itemize}
		\item $\lim\limits_{x\to x_0} \left[f(x)\cdot g(x)\right]=\heva{&+\infty\text{ khi }L>0\\&-\infty\text{ khi }L<0.}$
		\item $\lim\limits_{x\to x_0} \dfrac{f(x)}{g(x)}=\heva{&0\text{ khi }L\neq 0\\&+\infty&\text{ khi }L>0\\&-\infty&\text{ khi }L<0.}$
	\end{itemize}
	\end{multicols}
	Tương tự cho trường hợp $\lim\limits_{x\to x_0} f(x)=L\ne 0$ và $\lim\limits_{x\to x_0} g(x)=-\infty$.
	\begin{multicols}{2}
		\begin{itemize}
		\item $\lim\limits_{x\to x_0} \left[f(x)\cdot g(x)\right]=\heva{&+\infty\text{ khi }L<0\\&-\infty\text{ khi }L>0.}$
		\item $\lim\limits_{x\to x_0} \dfrac{f(x)}{g(x)}=\heva{&0\text{ khi }L\neq 0\\&+\infty&\text{ khi }L<0\\&-\infty&\text{ khi }L>0.}$
	\end{itemize}
	\end{multicols}
\begin{luuy}
	\begin{multicols}{2}
		\begin{itemize}
		\item $\lim\limits_{x\to a^+} \dfrac{1}{x-a}=+\infty$;
		\item $\lim\limits_{x\to a^-} \dfrac{1}{x-a}=-\infty$;.
	\end{itemize}
	\end{multicols}
\end{luuy}
\end{dang}
\begin{vd}
	Tính các giới hạn sau đây
	\begin{listEX}[4]
		\item $\lim\limits_{x\to 2^+}\dfrac{2x+3}{2-x}$.
		\item  $\lim\limits_{x\to 1^+} \dfrac{3-x^2}{1-x}$.
		\item  $\lim\limits_{x\to 2^-} \dfrac{\left|2-x\right|}{2x^2-5x+2}$.
		\item $\lim\limits_{x\to 1^-}\dfrac{\left|x-1\right|}{2-x^2-3x}$.
	\end{listEX}
		\loigiai{
			\begin{enumerate}
				\item Ta có
				\begin{itemize}
					\item $\lim\limits_{x\to 2^+}\left(2x+3\right)=\lim\limits_{x\to 2^+}\left(2x\right)+3=2\cdot 2+3=7$.
					\item $\lim\limits_{x\to 2^+}\dfrac{1}{2-x}=-\lim\limits_{x\to 2^+}\dfrac{1}{x-2}=-\infty$.
				\end{itemize}
			Do đó $\lim\limits_{x\to 2^+}\dfrac{2x+3}{2-x}=\lim\limits_{x\to 2^+}\left[(2x+3)\cdot\dfrac{1}{2-x}\right]=-\infty$.				
			\item Ta có 
			\begin{itemize}
				\item $\lim\limits_{x\to 1^+} \left(3-x^2\right)= 3-\lim\limits_{x\to 1^+}x^2=3-1^2=2$.
				\item $\lim\limits_{x\to 1^+} \dfrac{1}{1-x}=\lim\limits_{x\to 1^+} \dfrac{-1}{x-1}=-\infty$.
			\end{itemize}
				Do đó $\lim\limits_{x\to 1^+} \dfrac{3-x^2}{1-x}=\lim\limits_{x\to 1^+}\left[\left(3-x^2\right)\cdot\dfrac{1}{1-x}\right]=-\infty$.	
			\item Do $x\to 2^{-} \Rightarrow x < 2\Rightarrow x-2< 0\Rightarrow 2-x > 0$.\\
			Nên $\lim\limits_{x\to 2^-} \dfrac{\left|2-x\right|}{2x^2-5x+2}=\lim\limits_{x\to 2^-} \dfrac{2-x}{\left(2-x\right)\left(1-2x\right)}=\lim\limits_{x\to 2^-} \dfrac{1}{1-2x}=-\dfrac{1}{3}$.
			\item Do $x\to 1^{-} \Rightarrow x < 1\Rightarrow x-1< 0$ nên $\lim\limits_{x\to 1^-}\dfrac{\left|x-1\right|}{2-x^2-3x}=\lim\limits_{x\to 1^-}\dfrac{1-x}{\left(1-x\right)\left(x+2\right)}=\lim\limits_{x\to 1^-}\dfrac{1}{x+2}=-\dfrac{1}{3}$.
			\end{enumerate}
		}
\end{vd}	

\begin{vd}
	\begin{listEX}[2]
		\item $f(x)=\heva{&\dfrac{2x}{\sqrt{1-x}}&\text{ với }x < 1\\&\sqrt{3x^2+1}&\text{ với }x\ge 1.}$  Tính $\lim\limits_{x\to 1} f(x)$.
		\item $f(x)=\heva{&\dfrac{x^2+1}{1-x}&\text{ với }x < 1\\&\sqrt{2x-2}&\text{ với }x\ge 1.}$ Tính $\lim\limits_{x\to 1}f(x)$.
	\end{listEX}
	\loigiai{
		\begin{enumerate}
			\item Ta có
			\begin{itemize}
				\item $f(1)=\lim\limits_{x\to 1^+}f(x)=\sqrt{3\cdot 1^2+1}=2$.
				\item $\lim\limits_{x\to 1^-}f(x)=\lim\limits_{x\to 1^-}\dfrac{2x}{\sqrt{1-x}}=+\infty$ vì $\lim\limits_{x\to 1^-}{2x}=2$, $\lim\limits_{x\to 1^-}\sqrt{1-x}=0$ và $x\to 1^-$ nên $x<1\Rightarrow \sqrt{1-x}>0$.
			\end{itemize}
		Vì $\lim\limits_{x\to 1^+}f(x)\neq \lim\limits_{x\to 1^-}f(x)$ nên không tồn tại $\lim\limits_{x\to 1} f(x)$.
			\item Ta có
			\begin{itemize}
				\item $f(1)=\lim\limits_{x\to 1^+}f(x)=\sqrt{2\cdot 1-2}=0$.
				\item $\lim\limits_{x\to 1^-}f(x)=\lim\limits_{x\to 1^-}\dfrac{x^2+1}{1-x}=+\infty$ vì $\lim\limits_{x\to 1^-}(x^2+1)=2$, $\lim\limits_{x\to 1^-}(1-x)=0$ và $x\to 1^-\Rightarrow x<1$.
			\end{itemize}
			Vì $\lim\limits_{x\to 1^+}f(x)\neq \lim\limits_{x\to 1^-}f(x)=1$ nên không tồn tại $\lim\limits_{x\to 1} f(x)$.
		\end{enumerate}
	}
\end{vd}

\begin{dang}{Câu thực tế và liên môn}
	Sử dụng các định nghĩa, định lý để áp dụng vào giải các bài toán.
\end{dang}

\begin{vd}
	Một hồ nuôi tôm chứa $600$ m$^3$ nước mặn với nồng độ muối $1$ kg/m$^3$. Chủ hồ nuôi tôm dự định chuyển đổi giống mới nên bơm thêm nước vào hồ với vận tốc $3$ m$^3$ / phút để làm ngọt hóa nước trong hồ.
	\begin{enumerate}
		\item Viết biểu thức $C(x)$ biểu thị nồng độ muối trong hồ sau $x$ phút kể từ khi bắt đầu bơm.
		\item Tính $\lim\limits_{x\to  +\infty}C(x)$ và giải thích ý nghĩa của kết quả này.
	\end{enumerate}
	\loigiai{
		\begin{enumerate}
			\item \begin{itemize}
				\item Khối lượng muối có trong hồ nuôi tôm là $1\cdot 600=600$ (kg).
			\item Sau $x$ phút, lượng nước trong hồ là $600+3x$ (m$^3$).
			\item Nồng độ muối tại thời điểm $x$ phút kể từ khi bơm thêm nước ngọt vào là $C(x)=\dfrac{600}{600+3x}$.
			\end{itemize}
			\item Ta có $\lim\limits_{x\to  +\infty}C(x)=\lim\limits_{x\to  +\infty}\dfrac{600}{600+3x}=0$.
		\end{enumerate}
	\textbf{\textit{Ý nghĩa}}: 
	Điều này có nghĩa là khi $t$ càng lớn thì nồng độ muối trong hồ sẽ dần về $0$. 
	Tức là đến một thời điểm nào đó muối trong hồ không còn đáng kể và nước trong hồ coi như nước ngọt.
}
\end{vd}

\begin{vd}
	Một công ty sản xuất giày da đã xác định được rằng, tính trung bình một công nhân có thể làm được $f(x)=\dfrac{16x}{15+2x}$ đôi giày mỗi ngày sau khi được đào tạo $x$ ngày. Tính $\lim\limits_{x\to  +\infty}f(x)$ và giải thích ý nghĩa của kết quả này
	\loigiai{
		Ta có $\lim\limits_{x\to  +\infty}f(x)=\lim\limits_{x\to  +\infty}\dfrac{16x}{15+2x}=\lim\limits_{x\to  +\infty}\dfrac{16}{\dfrac{15}{x}+2}=\dfrac{16}{2}=8$.\\
		\textbf{\textit{Ý nghĩa}}: Khi thời gian đào tạo tăng lên thì số đôi giày mỗi công nhân sản xuất được trong một ngày cũng được cũng tăng lên nhưng không quá $8$ đôi giày/ ngày.
	}
\end{vd}
\begin{vd}
	Chi phí để sản xuất $x$ chai nước ngọt của công ty nước giải khát A được xác định bởi hàm số $F(x)=50000+15x$ (đơn vị: nghìn đồng).
	\begin{enumerate}
		\item Tính chi phí trung bình $\overline{F}(x)$ để công ty sản xuất một sản phẩm.
		\item Tính $\lim\limits_{x\to  +\infty}\overline{F}(x)$ và cho biết ý nghĩa của kết quả.
	\end{enumerate}
	\loigiai{
		\begin{enumerate}
			\item  Chi phí trung bình để sản xuất $x$ sản phẩm là: $\overline{F}(x)=\dfrac{50000+15x}{x}$ (nghìn đồng).
			\item  Ta có $\lim\limits_{x\to  +\infty}\overline{F}(x)=\lim\limits_{x\to  +\infty}\dfrac{50000+15x}{x}=\lim\limits_{x\to  +\infty}\dfrac{\dfrac{50000}{x}+15}{1}=15$ (nghìn đồng).
		\end{enumerate}
	\textbf{\textit{Ý nghĩa}}: Khi số sản phẩm công ty sản xuất ra càng nhiều thì chi phí trung bình để sản xuất ra một sản phẩm sẽ giảm dần, số sản phẩm $x$ đủ lớn thì chi phí trung bình xấp xỉ $15$ nghìn đồng mỗi sản phẩm và không thể thấp hơn.
	}
\end{vd}

%-----------------------------------------------------------------------------
\subsection{Bài tập rèn luyện}
\ind{PHẦN I.} \inden{Câu trắc nghiệm nhiều phương án lựa chọn. Mỗi câu hỏi học sinh chỉ chọn một phương án.}\\
\setcounter{ex}{0}
\Opensolutionfile{ans}[ans/2D1-Bai1-TN]%--Đặt tên 2D1-Bai1-Dang1-TN
\begin{ex}[THPT Nguyễn Du -- BRVT. NH 24--25]%[1D3N2-1]%[Dự án D - đợt 2 NH24-25- Dương Công Tạo]
	Giả sử $\lim\limits_{x\to x_0}f(x)=L$ và $\lim\limits_{x\to x_0}g(x)=M$ với $L$, $M$ là các số thực. Khẳng định nào sau đây \textbf{sai}?
	\choice
	{$\lim\limits_{x\to x_0}[f(x)+g(x)]=L+M$}
	{$\lim\limits_{x\to x_0}[f(x)-g(x)]=L-M$}
	{$\lim\limits_{x\to x_0}[f(x)\cdot g(x)]=L\cdot M$}
	{\True $\lim\limits_{x\to x_0}\dfrac{f(x)}{g(x)}=\dfrac{L}{M}$}
	\loigiai{Ta có $\lim\limits_{x\to x_0}\dfrac{f(x)}{g(x)}=\dfrac{L}{M}$ chỉ tồn tại khi $M\neq 0$.}
\end{ex}
\begin{ex}[THPT Chu Văn An -- Quãng Nam. NH23-24]%[1D3N2-1]%[Dự án D - đợt 2 NH24-25- Dương Công Tạo]
	Cho hàm số $y = f(x)$ có giới hạn hữu hạn khi $x$ dần tới $x_0$. Mệnh đề nào sau đây \textbf{sai}?
	\choice
	{$\lim\limits_{x \to x_0} f(x) = L$}
	{$\lim\limits_{x \to x_0} c = c$ (với $c$ là hằng số)}
	{$\lim\limits_{x \to x_0} x = x_0$}
	{\True $\lim\limits_{x \to +\infty} f(x) = L$}
	\loigiai{Nếu $x$ dần tới $x_0$ thì hàm số $y=f(x)$ có giới hạn hữu hạn nên mệnh đề sai là $\lim\limits_{x \to +\infty} f(x) = L$.
	}
\end{ex}
\begin{ex}[THPT Nguyễn Quốc Trình -- Hà Nội. GK1-24-25]%[1D3N2-1]%[Dự án D - đợt 2 NH24-25- Dương Công Tạo]
	Cho các giới hạn: $\lim \limits_{x \rightarrow x_{0}} f(x)=2$; $\lim \limits_{x \rightarrow x_{0}} g(x)=3$. Tính $\lim \limits_{x \rightarrow x_{0}}[3 f(x)+4 g(x)]$.
	\choice
	{$-6$}
	{$5$}
	{\True $18$}
	{$17$}
	\loigiai{
		$\lim \limits_{x \rightarrow x_{0}}[3 f(x)+4 g(x)]=3\lim \limits_{x \rightarrow x_{0}} f(x)+4\lim \limits_{x \rightarrow x_{0}} g(x)=3\cdot2+4\cdot3=18$.
	}
\end{ex}


\begin{ex}[THPT Chuyên Lê Hồng Phong - Tp HCM. HKI NH24-25]%[1D2N3-2]%[Dự án D - đợt 2 NH24-25- Dương Công Tạo]
	Cho $\lim\limits_{x\to1} f(x)=3$. Tìm khẳng định \textbf{sai}.
	\choice
	{$\lim\limits_{x\to1} \sqrt{f(x)+1}=2$}
	{$\lim\limits_{x\to1} \left[f(x)+3\right]=6$}
	{$\lim\limits_{x\to1} \left[f(x)-2x\right]=1$}
	{\True $\lim\limits_{x\to1} \left[f(x)-x^2\right]=1$}
	\loigiai
	{
		Ta có $\lim\limits_{x\to1} \left[f(x)-x^2\right]=3-1=2$.
	}
\end{ex}
\begin{ex}[HKI-THPT thị xã Quảng Trị -- NH 24-25]%[1D3N2-1]%[Dự án D - đợt 2 NH24-25- Dương Công Tạo]
	Tính $\displaystyle \lim_{x\to 3^{+}}\dfrac{x-1}{x-3}$.
	\choice
	{$-\dfrac{1}{3}$}
	{$0$}
	{$-\infty $}
	{\True $+\infty$}
	\loigiai{
		Vì $\displaystyle \lim_{x\to 3^{+}}(x-1)=3-1=2>0$, $\displaystyle \lim_{x \to 3^{+}} (x-3)=0$ và $x-3>0$, $\forall x>3$ nên $\displaystyle \lim_{x\to 3^{+}} \dfrac{x-1}{x-3}=+\infty$.
	}
\end{ex}
\begin{ex}[Ôn tập học kì 1 - THPT Thái Bình. NH 24-25]%[1D3N2-7]%[Dự án D - đợt 2 NH24-25- Dương Công Tạo]
	Tính giới hạn $\lim\limits_{x \to(-1)^{+}} \dfrac{\sqrt{3x^2+1}-x}{x-1}$
	\choice
	{$-\dfrac{1}{2}$}
	{$\dfrac{1}{2}$}
	{\True $-\dfrac{3}{2}$}
	{$\dfrac{3}{2}$}
	\loigiai{
		$\lim\limits_{x \to(-1)^{+}} \dfrac{\sqrt{3x^2+1}-x}{x-1}= \dfrac{\sqrt{3(-1)^2+1}-(-1)}{(-1)-1}=-\dfrac{3}{2}$.
	}
\end{ex}
\begin{ex}[Ôn tập học kì 1 - THPT Thái Bình. NH 24-25]%[1D3H2-4]%[Dự án D - đợt 2 NH24-25- Dương Công Tạo]
	Tính giới hạn $\lim\limits_{x \to-\infty}\left(2x^3-x^2+1\right)$.
	\choice
	{$0$}
	{$+\infty$}
	{\True $-\infty$}
	{$2$}
	\loigiai{
		$\lim\limits_{x \to-\infty}\left(2x^3-x^2+1\right)=\lim\limits_{x \to-\infty}\left[x^3\cdot\left(2-\dfrac{1}{x}+\dfrac{1}{x^3}\right)\right]=-\infty$ vì \begin{itemize}
			\item $\lim\limits_{x \to-\infty}x^3=-\infty$.
			\item $\lim\limits_{x \to-\infty}\left(2-\dfrac{1}{x}+\dfrac{1}{x^3}\right)=2>0$.
		\end{itemize}
	}
\end{ex}
\begin{ex}[DHKI THPT Mạc Đĩnh Chi. NH24-25]%[]%[1D3H2-3]%[Dự án D - đợt 2 NH24-25- Dương Công Tạo]
	Tính giới hạn $\lim\limits_{x\to-\infty}\dfrac{3x+\sqrt{4x^2+1}}{2x-\sqrt{9x^2+10}}$ bằng
	\choice
	{\True $\dfrac{1}{5}$}
	{$-5$}
	{$-\dfrac{1}{2}$}
	{$\dfrac{1}{2}$}
	\loigiai{
		$\lim\limits_{x\to-\infty}\dfrac{3x+\sqrt{4x^2+1}}{2x-\sqrt{9x^2+10}}=\lim\limits_{x\to-\infty}\dfrac{x\left(3-\sqrt{4+\tfrac{1}{x^2}}\right)}{x\left(2+\sqrt{9+\tfrac{10}{x^2}}\right)}=\lim\limits_{x\to-\infty}\dfrac{3-\sqrt{4+\tfrac{1}{x^2}}}{2+\sqrt{9+\tfrac{10}{x^2}}}=\dfrac{3-2}{2+3}=\dfrac{1}{5}$.
	}
\end{ex}
\begin{ex}[HK1 - THPT Thuận Thành 1 - Bắc Ninh. NH 24-25]%[Dương Công Tạo]%[1D3N2-4]%[Dự án D - đợt 2 NH24-25- Dương Công Tạo]
	Trong các mệnh đề sau, mệnh đề nào {\bf sai}?
	\choice
	{$\lim\limits _{x \rightarrow 0^{+}} \dfrac{1}{x^4}=+\infty$}
	{$\lim\limits _{x \rightarrow 0^{+}} \dfrac{1}{\sqrt{x}}=+\infty$}
	{$\lim\limits _{x \rightarrow 2^{+}} \dfrac{1}{x-2}=+\infty$}
	{\True $\lim\limits _{x \rightarrow 2^{+}} \dfrac{1}{x-2}=-\infty$}
	\loigiai{
		Khi $x \rightarrow 2^{+}$ thì $x-2>0$ nên $\lim\limits _{x \rightarrow 2^{+}} \dfrac{1}{x-2}=+\infty$
	}
\end{ex}
\begin{ex}[HK1 THPT Thuận Thành 1 - Bắc Ninh]%[1D3N2-4]%[Dự án D - đợt 2 NH24-25- Dương Công Tạo]
	Tìm $\lim\limits _{x \rightarrow-\infty} \dfrac{\sqrt{x^2+3 x+5}}{4 x-1}$
	\choice
	{\True $-\dfrac{1}{4}$}
	{$1$}
	{$0$}
	{$\dfrac{1}{4}$}
	\loigiai{
		$\lim\limits _{x \rightarrow-\infty} \dfrac{\sqrt{x^2+3 x+5}}{4 x-1}=\lim\limits _{x \rightarrow-\infty} \dfrac{-x\sqrt{1+\dfrac{3}{x}+\dfrac{5}{x^2}}}{x\left(4-\dfrac{1}{x}\right)}=\lim\limits _{x \rightarrow-\infty} \dfrac{-\sqrt{1+\dfrac{3}{x}+\dfrac{5}{x^2}}}{4-\dfrac{1}{x}}=-\dfrac{1}{4}$
	}
\end{ex}

\begin{ex}[HKI THPT - Lê Quý Đôn - TPHCM. NH24-25]%[1D3H2-3]%[Dự án D - đợt 2 NH24-25- Dương Công Tạo]
	$\lim\limits_{x \to +\infty} \dfrac{2x^2 - 1}{x^2 + x}$ bằng
	\choice
	{\True $2$}
	{$1$}
	{$0$}
	{$-1$}
	\loigiai{
		Ta có $\lim\limits_{x \to +\infty} \dfrac{2x^2 - 1}{x^2 + x}=\lim\limits_{x \to +\infty} \dfrac{2-\tfrac{1}{x^2}}{1+\tfrac{1}{x}}=2$.
	}
\end{ex}
\begin{ex} [ChuyenLeQuyDon-NinhThuan-HKI-NH24-25]%[1D3H2-3]%[Dự án D - đợt 2 NH24-25- Dương Công Tạo]
	Kết quả của $\lim\limits_{x\to-\infty} \dfrac{2x+3}{x+1}$ là
	\choice
	{\True $2$}
	{$-2$}
	{$0$}
	{$+\infty$}
	\loigiai{Ta có $\lim\limits_{x\to-\infty} \dfrac{2x+3}{x+1}=\lim\limits_{x\to-\infty} \dfrac{2+\dfrac{3}{x}}{1+\dfrac{1}{x}}=2$.}
\end{ex}
\begin{ex}%[1D3V2-3]%[Dự án D - đợt 2 NH24-25- Dương Công Tạo]
	Giới hạn $\lim\limits_{x\to 1}\dfrac{x^3-x^2}{7x-7}$ bằng
	\choice
	{$0$}{$\dfrac{2}{3}$}{$-\dfrac{1}{7}$}{\True $\dfrac{1}{7}$}
	\loigiai{
		\begin{eqnarray*}
			&&\lim\limits_{x\to 1}\dfrac{x^3-x^2}{7x-7}\\ 
			&=&\lim\limits_{x\to 1}\dfrac{x^2\left(x-1\right)}{7(x-1)}\\ 
			&=&\lim\limits_{x\to 1}\dfrac{x^2}{7}\\ 
			&=&\dfrac{1}{7}. 
		\end{eqnarray*}
	}
\end{ex}
\begin{ex}[HK1 SGD - Tỉnh Bắc Ninh. NH24-25]%[1D3H2-7]%[Dự án D - đợt 2 NH24-25- Dương Công Tạo]
	Giới hạn $\lim\limits_{x \to 3^{-}} \dfrac{2x+5}{3-x}$ bằng
	\choice
	{$-\infty$}
	{\True $+\infty$}
	{$0$}
	{$11$}
	\loigiai{
		Ta có $\lim\limits_{x \to 3^{-}} (2x+5)=\lim\limits_{x \to 3^{-}} (2x)+5=2 \cdot 3+5=11>0$; $\lim\limits_{x \to 3^{-}}\dfrac{1}{3-x}=0$.\\
		Do đó $\lim\limits_{x \to 3^{-}} \dfrac{2x+5}{3-x}=\lim\limits_{x \to 3^{-}} \left[(2x+5)\cdot\dfrac{1}{3-x}\right]=+\infty$.
	}
\end{ex}
\begin{ex}[ĐỀ KSCL CUỐI HK1 - THPT BẮC YÊN THÀNH - NGHỆ AN. NH24-25]%[1D3H2-2]%[Dự án D - đợt 2 NH24-25- Dương Công Tạo]
	Giá trị của giới hạn $\lim\limits_{x \to - 3} \dfrac{\sqrt{x^2 +16}}{x+1} $ là
	\choice
	{$1$}
	{\True $\dfrac{-5}{2}$}
	{$\dfrac{5}{2}$}
	{$ +\infty$}
	\loigiai{ Ta có $\lim\limits_{x \to - 3} \dfrac{\sqrt{x^2 +16}}{x+1} = \dfrac{\sqrt{(-3)^2 +16}}{(-3)+1}=\dfrac{-5}{2}$.
		
	}
\end{ex}
\begin{ex}[Dự án đề kiểm tra HKI NH24-25]%[1D3H2-3]%[Dự án D - đợt 2 NH24-25- Dương Công Tạo]
	Kết quả của $\lim\limits_{x \rightarrow-2} \dfrac{x^{2}-4}{x+2}$ bằng
	\choice
	{$0$}
	{\True $-4$}
	{$+\infty$}
	{$-\infty$}
	\loigiai{
		Ta có $\lim\limits_{x \rightarrow-2} \dfrac{x^{2}-4}{x+2} = \lim\limits_{x \rightarrow-2} \dfrac{\left(x-2\right)\left(x+2\right)}{x+2} = \lim\limits_{x \rightarrow-2} (x-2) = -4$.
	}
\end{ex}

%Câu 3
\begin{ex}[Dự án đề kiểm tra Toán 11 HKI NH24-25]%[1D3H2-7]%[Dự án D - đợt 2 NH24-25- Dương Công Tạo]
	Kết quả của $\lim\limits _{x \rightarrow-2^{+}} \dfrac{x^{2}}{x+2}$ bằng
	\choice
	{$-4$}
	{$0$}
	{$4$}
	{\True $+\infty$}
	\loigiai{
		Ta có $\lim\limits _{x \rightarrow-2^{+}} x^2 = 4 > 0$ và $\lim\limits _{x \rightarrow-2^{+}} \dfrac{1}{x+2} = +\infty$.\\
		Do đó $\lim\limits _{x \rightarrow-2^{+}} \dfrac{x^{2}}{x+2} = \lim\limits _{x \rightarrow-2^{+}} \left[ x^2 \cdot  \dfrac{1}{x+2}\right] =+\infty$.
	}
\end{ex}
%Câu 4
\begin{ex}[Dự án đề kiểm tra Toán 11 HKI NH24-25]%[1D3N2-3]%[Dự án D - đợt 2 NH24-25- Dương Công Tạo]
	Ta có $\lim\limits_{x \rightarrow-\infty} \dfrac{x+4}{x+2}$ bằng
	\choice
	{$-1$}
	{$0$}
	{\True $1$}
	{$+\infty$}
	\loigiai{
		Ta có $\lim\limits_{x \rightarrow-\infty} \dfrac{x+4}{x+2} = \lim\limits_{x \rightarrow-\infty} \dfrac{1+\dfrac{4}{x}}{1+\dfrac{2}{x}} = 1$.
	}
\end{ex}

\begin{ex}[Đề thi HK1 lớp 11 THPT Bình Chiểu HCM. NH 24-25]%[1D3N2-4]%[Dự án D - đợt 2 NH24-25- Dương Công Tạo]
	Tính $\lim\limits_{x \to +\infty} \dfrac{x^2 - x + 1}{2 - x}$.
	\choice
	{$-1$}
	{$0$}
	{$+\infty$}
	{\True $-\infty$}
	\loigiai{
		$\lim\limits_{x \to +\infty} \dfrac{x^2 - x + 1}{2 - x} = \lim\limits_{x \to +\infty} \dfrac{x^2\left(1 - \dfrac{1}{x} + \dfrac{1}{x^2}\right)}{x\left(\dfrac{2}{x} - 1\right)} = \lim\limits_{x \to +\infty} \dfrac{x\left(1 - \dfrac{1}{x} + \dfrac{1}{x^2}\right)}{\dfrac{2}{x} - 1}=-\infty$.\\
		Vì $\lim \limits_{x \to +\infty} x=+\infty$ và $\lim \limits_{x \to +\infty} \dfrac{1-\dfrac{1}{x}+\dfrac{1}{x^2}}{\dfrac{2}{x}-1}=-1<0$.
	}
\end{ex}

\begin{ex}[Đề thi HK1 lớp 11 THPT Bình Chiểu HCM. NH 24-25]%[1D3N2-7]%[Dự án D - đợt 2 NH24-25- Dương Công Tạo]
	Tìm giới hạn $\lim\limits_{x \to 4^-} \dfrac{x^2 + x}{x - 4}$.
	\choice
	{$+\infty$}
	{\True $-\infty$}
	{$0$}
	{$-\dfrac{1}{4}$}
	\loigiai{
		$\lim\limits_{x \to 4^-} \dfrac{x^2 + x}{x - 4}=-\infty$ vì $\heva{& \lim \limits_{x \to 4^-}( x^2+x)=20>0 \\ & \lim \limits_{x \to 4^-} (x-4)=0 \\ & x\to 4^-\Rightarrow x-4<0.}$
	}
\end{ex}

\Closesolutionfile{ans}

\ind{PHẦN II.} \inden{Câu trắc nghiệm đúng sai. Trong mỗi ý a), b), c), d) ở mỗi câu, học sinh chọn đúng hoặc sai.}\\
\setcounter{ex}{0}
\Opensolutionfile{ans}[ans/2D1-Bai1-DS]%--Đặt tên 2D1-Bai1-DS
\begin{ex}[Chuyên Lê Quý Đôn-NinhThuận-HKI-NH24-25]%[1D3V2-7]%[Dự án D - đợt 2 NH24-25- Dương Công Tạo]
	Cho hàm số $f(x)=\heva{&\dfrac{3x^2-8x-3}{x-3}\ &\text{khi}& \ x<3\\&2mx+5 \ &\text{khi}& \ x \ge 3}$ ($m$ là tham số).
	\choiceTF
	{$\displaystyle \lim_{x\to 0 }f(x)=+\infty$}
	{\True $\displaystyle \lim_{x\to 5 }f(x)=10m+5$}
	{\True $\displaystyle \lim_{x\to 3^- }f(x)=10$}
	{Với $m=\dfrac{a}{b}$ ($a$, $b\in \mathbb{R}$, $\dfrac{a}{b}$ là phân số tối giản) thì tồn tại giới hạn $\displaystyle \lim_{x\to 3 }f(x)$. Ta có $a+b=12$}
	\loigiai{
		\begin{itemchoice}
			\itemch \textbf{Sai}.\\
			$\displaystyle \lim_{x\to 0 }f(x)=\displaystyle \lim_{x\to 0 }\dfrac{3x^2-8x-3}{x-3}=\dfrac{3\cdot 0^2-8\cdot 0 -3}{0-3}=1$.
			\itemch \textbf{Đúng}.\\
			$\displaystyle \lim_{x\to 5 }f(x)=\displaystyle \lim_{x\to 5 }(2mx+5)=2m\cdot5+5=10m+5$.
			\itemch \textbf{Đúng}.\\
			\begin{eqnarray*}
				\displaystyle \lim_{x\to 3^- }f(x)&=&\displaystyle \lim_{x\to 3^- }\dfrac{3x^2-8x-3}{x-3}\\
				&=&\displaystyle \lim_{x\to 3^- }\dfrac{3(x-3)(x+\dfrac{1}{3})}{x-3}\\&=&\displaystyle \lim_{x\to 3^-}3(x+\dfrac{1}{3})\\&=&3\cdot(3+\dfrac{1}{3})\\&=&10.
			\end{eqnarray*}
			\itemch \textbf{Sai}.\\
			\begin{itemize}
				\item $\displaystyle \lim_{x\to 3^+ }f(x)=\displaystyle \lim_{x\to 3^+ }(2mx+5)=6m+5$.
				\item 
				$\displaystyle \lim_{x\to 3^- }f(x)=10.$
				\item $f(3)=6m+5$.
			\end{itemize}
			Để $\displaystyle \lim_{x\to 3 }f(x)$ tồn tại thì $\displaystyle \lim_{x\to 3^+ }f(x)=\displaystyle \lim_{x\to 3^- }f(x)=f(3)$.
			\\
			Suy ra $6m+5=10$.\\
			$\Rightarrow m = \dfrac{5}{6}$.\\
			Do đó $a=5$, $b=6$.\\		
			Vậy $a+b=11$.
			
		\end{itemchoice}
	}
\end{ex}
\begin{ex}[TH-THCS-THPT-HoangViet-DakLak-HKI-NH24-25]%[1D3V3-6]%[Dự án D - đợt 2 NH24-25- Dương Công Tạo]
	Một bãi đỗ xe ôtô tính phí $60\,000$ cho giờ đầu tiên (hoặc một phần của giờ đầu tiên) và thêm $40\,000$ đồng cho mỗi giờ (hoặc một phần của mỗi giờ) tiếp theo, tối đa là $200\,000$ đồng. Gọi $C=C(t)$ là hàm số biểu thị chi phí theo thời gian đỗ xe.
	\choiceTF
	{Số tiền đỗ xe của một người với thời gian $2{,}5$ giờ là $140\,000$ đồng}
	{Hàm số $C(t)$ liên tục trên $[0;+\infty)$}
	{$\lim\limits _{t\rightarrow 3}C(t)=140\,000$}
	{\True Chênh lệch chi phí đối với hai khách hàng đỗ xe có thời gian $t_{1}$; $t_{2}$ thay đổi với $2<t_{1}\leq 3$; $3<t_{2}\leq 4$ là không đổi}
	\loigiai{
		Hàm số biểu thị chi phí theo thời gian đỗ xe\\
		$C(t)= \heva{&60\,000 & 0<t \leq 1 \\& 60\,000+40\,000(t-1) & 1<t \leq 4 \\& 200\,000 & t>4.}$
		\begin{itemchoice}
			\itemch 
			Số tiền đỗ xe của một người với thời gian $2{,}5$ giờ là\\
			$C(2{,}5)	=60\,000+40\,000(2{,}5-1)=120\,000$.
			\itemch 
			Ta có $\heva{&\lim\limits _{t\rightarrow 4^-}C(t)=180\,000\\&\lim\limits _{t\rightarrow 4^+}C(t)=200\,000.}$\\
			Suy ra hàm số không liên tục tại $t=4$.
			\itemch 
			Ta có $\lim\limits _{t\rightarrow 3}C(t)=\lim\limits _{t\rightarrow 3}\left[60\,000+40\,000(t-1)\right]=140\,000$.
			\itemch 
			Chi phí đối với khách hàng đỗ xe của hai khách hàng tại thời điểm $t_{1}$, $t_{2}$ lần lượt là\\
			$C(t_{1})=60\,000+40\,000(3-1)=140\,000$.\\
			$C(t_{2})=60\,000+40\,000(4-1)=180\,000$.\\
			Chênh lệch chi phí đối với hai khách hàng đỗ xe là
			\[C(t_{1})-C(t_{2})=180\,000-140\,000=40\,000.\]
		\end{itemchoice}
	}
\end{ex}
\begin{ex}[HKI THPT Bắc Yên Thành. NH24-25]%[1D3H2-7]%[Dự án D - đợt 2 NH24-25- Dương Công Tạo]
	Cho hàm số $f(x)=\heva{&x^2-3 x+1 \text{ khi }x<0 \\& \sqrt{x^2+1} \text{ khi }x \geq 0}$. Khi đó
	\choiceTF
	{Giới hạn $\displaystyle\lim\limits _{x \rightarrow 2}f(x)=-1$}
	{Giới hạn $\displaystyle\lim\limits _{x \rightarrow 0^{-}}f(x)=-1$}
	{\True Giới hạn $\displaystyle\lim\limits _{x \rightarrow 0^{+}}f(x)=1$}
	{\True Giới hạn $\displaystyle\lim\limits _{x \rightarrow 0}f(x)=1$}
	\loigiai{
		\begin{itemchoice}
			\itemch Giới hạn $\displaystyle\lim\limits _{x \rightarrow 2}f(x)=\displaystyle\lim\limits_{x\rightarrow 2} \sqrt{x^2+1}=\sqrt{5}$.
			\itemch Giới hạn $\displaystyle\lim\limits _{x \rightarrow 0^{-}}f(x)=\displaystyle\lim\limits _{x \rightarrow 0^{-}}(x^2-3x+1)=1$.
			\itemch Giới hạn $\displaystyle\lim\limits _{x \rightarrow 0^{+}}f(x)=\displaystyle\lim\limits _{x \rightarrow 0^{+}}\sqrt{x^2+1}=1$.
			\itemch Vì $\displaystyle\lim\limits _{x \rightarrow 0^{-}}f(x)=\displaystyle\lim\limits _{x \rightarrow 0^{-}}(x^2-3x+1)=1$  và $\displaystyle\lim\limits _{x \rightarrow 0^{+}}f(x)=\displaystyle\lim\limits _{x \rightarrow 0^{+}}\sqrt{x^2+1}=1$ nên \break $\displaystyle\lim\limits _{x \rightarrow 0}f(x)=\displaystyle\lim\limits _{x \rightarrow 0}=1$.
		\end{itemchoice}
	}
\end{ex}
\begin{ex}[Ôn tập cuối học kì I THPT NGUYỄN GIA THIỀU-Hà Nội. NH24-25]%[Phạm Văn Long]%[1D3V2-8]%[Dự án D - đợt 2 NH24-25- Dương Công Tạo]
	Chi phí (đơn vị: nghìn đồng) để sản xuất $x$ sản phẩm của một công ty được xác định bởi hàm số $C(x)=600+500x$.
	\choiceTF
	{Chi phí để sản xuất $1$ sản phẩm là $1\,100$ đồng}
	{Chi phí để sản xuất $10$ sản phẩm là $560\,000$ đồng}
	{\True Công ty sản xuất $20$ sản phẩm thì chi phí trung bình của mỗi sản phẩm là $530\,000$ đồng}
	{\True Nếu công ty sản xuất được số sản phẩm tăng lên rất nhiều thì chi phí trung bình của mỗi sản phẩm giảm dần về mức $500\,000$ đồng}
	\loigiai{
		\begin{itemchoice}
			\itemch \textbf{Sai}.\\
			Chi phí để sản xuất $1$ sản phẩm là $C(1)=600+500\cdot 1=1\,100$ nghìn đồng.
			\itemch \textbf{Sai}.\\
			Chi phí để sản xuất $10$ sản phẩm là $C(10)=600+500\cdot 10=5\,600$ nghìn đồng.
			\itemch \textbf{Đúng}.\\
			Công ty sản xuất $20$ sản phẩm thì chi phí trung bình của mỗi sản phẩm là $$C(20)=\dfrac{600+500\cdot 20}{20}=530 \text{ nghìn đồng}.$$
			\itemch \textbf{Đúng}.\\
			Ta có 
			$$\lim\limits_{x\to +\infty}\dfrac{600+500x}{x}=500.$$
			Vậy nếu công ty sản xuất được số sản phẩm tăng lên rất nhiều thì chi phí trung bình của mỗi sản phẩm giảm dần về mức $500\,000$ đồng
		\end{itemchoice}
	}
\end{ex}

\begin{ex}[HKI THPT Văn Bàn 1 - Tỉnh Lào Cai. NH24-25]%[1D3H2-7]%[Dự án D - đợt 2 NH24-25- Dương Công Tạo]
	Cho hàm số $f(x) = \begin{cases}
		\sqrt{x+2} & \text{khi } x \geq 2 \\
		2-x & \text{khi } x < 2.
	\end{cases}$
	\choiceTF[1]
	{\True $f(2) = 2$}
	{$\lim\limits_{x \to 2^-} f(x) = 2$}
	{\True $\lim\limits_{x \to 2^+} f(x) = 2$}
	{Tồn tại giới hạn của hàm số $f(x)$ khi $x \to 2$}
	\loigiai{
		\begin{itemchoice}
			\itemch Ta có
			$f(2) = \sqrt{2+2} = 2 $.
			\itemch $\lim\limits_{x \to 2^-} f(x) = 2-2=0 $.
			\itemch $\lim\limits_{x \to 2^+} f(x) = \sqrt{2+2} = 2.$
			\itemch Vì $ \lim\limits_{x \to 2^-} f(x) \neq \lim\limits_{x \to 2^+} f(x)$
			nên không tồn tại giới hạn của hàm số $f(x)$ khi $x \to 2$.
		\end{itemchoice}
	}
\end{ex}
\Closesolutionfile{ans}


\ind{PHẦN III.} \inden{Câu trả lời ngắn. Thí sinh ghi kết quả.}\\
\setcounter{ex}{0}
\Opensolutionfile{ans}[ans/2D1-Bai1-DS]%--Đặt tên 2D1-Bai1-DS
\begin{ex}[HK1 - THPT Thuận Thành 1 - Bắc Ninh. NH 24-25]%[1D3H2-5]%[Dự án D - đợt 2 NH24-25- Dương Công Tạo]
	Cho $\lim\limits _{x \rightarrow-\infty}\left(\sqrt{x^2+a x+5}+x\right)=5$. Khi đó giá trị $a$ là bao nhiêu?
	\shortans[oly]{-10}
	\loigiai{
		Ta có: $\lim\limits _{x \rightarrow-\infty}\left(\sqrt{x^2+a x+5}+x\right)=\lim\limits _{x \rightarrow-\infty}\dfrac{ax+5}{\sqrt{{{x}^{2}}+ax+5}-x}=\lim\limits _{x \rightarrow-\infty}\dfrac{x\left( a+\dfrac{5}{x} \right)}{x\left( -\sqrt{1+\dfrac{a}{x}+\dfrac{5}{{{x}^{2}}}}-1 \right)}=\lim\limits _{x \rightarrow-\infty}\dfrac{a+\dfrac{5}{x}}{-\sqrt{1+\dfrac{a}{x}+\dfrac{5}{{{x}^{2}}}}-1}=-\dfrac{a}{2}$\\
		Theo giả thiết, ta suy ra $-\dfrac{a}{2}=5 \Leftrightarrow a=-10$
	}
\end{ex}
\begin{ex}[HKI Lê Quý Đôn - Tp HCM. NH24-25]%[1D3H2-3]%[Dự án D - đợt 2 NH24-25- Dương Công Tạo]
	Một công ty sản xuất máy tính đã xác định được rằng, tính trung bình một nhân viên có thể lắp ráp được $N(t)=\dfrac{45t}{t+4}$ $(t\geq 0)$ bộ phận mỗi ngày sau $t$ ngày đào tạo. Tính $\lim\limits_{t\to+\infty} N(t)$.
	\shortans[oly]{45}
	\loigiai{
		Ta có $\lim\limits_{t\to+\infty} N(t)=\lim\limits_{t\to+\infty}\dfrac{45t}{t+4}=\lim\limits_{t\to+\infty}\dfrac{45}{1+\dfrac{4}{t}}=45$.
	}
\end{ex}
\begin{ex}[HKI -- Nguyễn Gia Thiều-Hà Nội. NH 24-25]%[1D3V2-3]%[Dự án D - đợt 2 NH24-25- Dương Công Tạo]
	Cho $\lim\limits_{x\to 1}\dfrac{x^2+bx+c}{x-1}=0$. Tính giá trị biểu thức $\dfrac{c}{b}$.
	\shortans[oly]{-0{,}5}
	\loigiai{
		Vì $\lim\limits_{x\to 1}\dfrac{x^2+bx+c}{x-1}$ hữu hạn và $\lim\limits_{x\to 1}(x-1)=0$ nên $1^2+b\cdot 1+c=0\Leftrightarrow c=-b-1$.\\
		Ta có 
		\begin{eqnarray*}
			&&\lim\limits_{x\to 1}\dfrac{x^2+bx+c}{x-1}\\
			&=&\lim\limits_{x\to 1}\dfrac{x^2+bx-b-1}{x-1}\\
			&=&\lim\limits_{x\to 1}\dfrac{(x-1)(x+1+b)}{x-1}\\
			&=&\lim\limits_{x\to 1}(x+1+b)=2+b.
		\end{eqnarray*}
		Theo giả thiết $2+b=0\Leftrightarrow b=-2\Rightarrow c=1\Rightarrow \dfrac{c}{b}=-\dfrac{1}{2}=-0{,}5$.
	}
\end{ex}
\begin{ex}[HKI THCS-THPT-Vam-Dinh-Ca-Mau. NH24-25]%[]%[1D3V2-3]%[Dự án D - đợt 2 NH24-25- Dương Công Tạo]
	Cho $a$, $b$ là số nguyên và $\lim\limits_{x \to 3}\dfrac{x^2+ax+b}{x-3}=3$. Giá trị của biểu thức $a^2+b^2$ bằng bao nhiêu?
	\shortans[oly]{9}
	\loigiai{
		Để $\lim\limits_{x \to 3}\dfrac{x^2+ax+b}{x-3}=3$ thì ta phải có $x^2+ax+b=(x-3)(x-m)$.\\
		Khi đó
		\allowdisplaybreaks
		\begin{eqnarray*}
			&&\lim\limits_{x \to 3}\dfrac{x^2+ax+b}{x-3}=3 \\
			&\Leftrightarrow&\lim\limits_{x\to3} (x-m)=3 \\
			&\Leftrightarrow&3-m=3 \\
			&\Leftrightarrow&m=0.\\
		\end{eqnarray*}
		Suy ra $(x-3)x=x^2-3x$. Do đó $x^2+ax+b=x^2-3x$.\\
		Đồng nhất hệ số ta được $a=-3$, $b=0$.\\
		Vậy $a^2+b^2=(-3)^2+0^2=9$.
	}
\end{ex}

\begin{ex}[Ôn tập cuối kì I. NH 24-25]%[1D3V2-5]%[Dự án D - đợt 2 NH24-25- Dương Công Tạo]
	Tính giới hạn $\lim\limits_{x \to 0} \dfrac{4x}{3-\sqrt{9+x}}$.
	\shortans[oly]{-24}
	\loigiai{
		\begin{eqnarray*}
			\lim\limits_{x \to 0} \dfrac{4x}{3-\sqrt{9+x}}&=&\lim\limits_{x \to 0} \dfrac{4x\left(3+\sqrt{9+x}\right)}{\left(3-\sqrt{9+x}\right)\cdot\left(3+\sqrt{9+x}\right)}\\
			&=&\lim\limits_{x \to 0} \dfrac{4x\left(3+\sqrt{9+x}\right)}{9-9-x}\\
			&=&-\lim\limits_{x \to 0} \dfrac{4x\left(3+\sqrt{9+x}\right)}{x}\\
			&=&-\lim\limits_{x \to 0} \left[4\left(3+\sqrt{9+x}\right)\right]\\
			&=&4\cdot \left(3+\sqrt{9+0}\right)\\
			&=&24.
		\end{eqnarray*}
	}
\end{ex}

\Closesolutionfile{ans}


\ind{PHẦN IV.} \inden{Tự luận.}\\
\setcounter{ex}{0}
\begin{ex}[Đề thi HK1 lớp 11 THPT Thuận Thành Số 1. NH 24-25]%[1D3N2-4]%[Dự án D - đợt 2 NH24-25- Dương Công Tạo]
	($1{,}0$ điểm) Tính 
	\begin{listEX}[2] 
		\item $\lim\limits_{x \rightarrow 2} \dfrac{x^2-5 x+6}{x-2}$;
		\item $\lim\limits_{x \rightarrow+\infty} \dfrac{5x+6}{\sqrt{9x^2-2}}$. 
	\end{listEX}
	\loigiai{
		\begin{enumerate}
			\item Ta có $\lim\limits_{x \rightarrow 2} \dfrac{x^2-5 x+6}{x-2}=\lim\limits_{x \rightarrow 2} \dfrac{(x-2)(x-3)}{x-2}=\lim\limits_{x \rightarrow 2} (x-3)=2-3=-1$.
			\item Ta có $\lim\limits_{x \rightarrow+\infty} \dfrac{5x+6}{\sqrt{9 x^2-2}}=\lim\limits_{x \rightarrow+\infty} \dfrac{x\left(5+\dfrac{6}{x}\right)}{x \sqrt{9-\dfrac{2}{x^2}}}=\dfrac{5}{3}$.
		\end{enumerate}
	}
\end{ex}
\begin{ex}[Đề thi HK1 lớp 11 THPT Thuận Thành Số 1. NH 24-25]%[1D3C2-5]%[Dự án D - đợt 2 NH24-25- Dương Công Tạo]
	($0{,}5$ điểm) Cho biết $\lim\limits_{x \rightarrow 1} \dfrac{\sqrt{ax^2+1}-bx-2}{x^3-3 x+2}$ ($a$, $b \in \mathbb{R}$) có kết quả là một số thực. Tìm $a$, $ b$.
	\loigiai{
		Ta có $x^3-3x+2={{\left( x-1 \right)}^2}\left( x+2 \right)$.\\
		Suy ra $\lim\limits_{x \rightarrow 1} \dfrac{\sqrt{ax^2+1}-bx-2}{x^3-3 x+2}$ có kết quả là một số thực khi và chỉ khi
		$\sqrt{ax^2+1}-bx-2=0$ có nghiệm kép $x=1$.\\
		$\Rightarrow ax^2+1=(bx+2)^2$ có nghiệm kép $x=1$\\
		$\Rightarrow \left( a-b^2 \right)x^2-4bx-3=0$ có nghiệm kép $x=1$\\
		{\allowdisplaybreaks 
			\begin{eqnarray*}
				&\Leftrightarrow &\left\{\begin{aligned}& a-b^2\ne 0 \\& \Delta =0 \\& a-b^2-4b-3=0\\\end{aligned} \right.
				\Leftrightarrow \left\{ \begin{aligned}& a\ne b^2 \\& 16b^2+12\left(a-b^2 \right)=0 \\& a-b^2-4b-3=0 \end{aligned} \right.
				\Leftrightarrow \left\{\begin{aligned}& a\ne b^2 \\& a=-\dfrac{1}{3}b^2 \\& a-b^2-4b-3=0 \end{aligned} \right.\\
				&\Leftrightarrow & \left\{ \begin{aligned}& a\ne b^2 \\& a=-\dfrac{1}{3}b^2 \\& -\dfrac{1}{3}b^2-b^2-4b-3=0 \end{aligned} \right.
				\Leftrightarrow \left\{ \begin{aligned}& a\ne b^2 \\& a=-\dfrac{3}{4} \\& b=-\dfrac{3}{2}\end{aligned} \right.
				\Leftrightarrow \left\{ \begin{aligned}& a=-\dfrac{3}{4} \\& b=-\dfrac{3}{2}.\end{aligned} \right.
		\end{eqnarray*}  }
	}
\end{ex}
\begin{ex}[Ôn tập GKI THPT Nguyễn Thị Minh Khai Tp HCM. NH 24-25]%[1D3H1-4]%[Dự án D - đợt 2 NH24-25- Dương Công Tạo]
	\begin{enumerate}
		\item Tính giới hạn $\lim\limits_{x\to +\infty} \left(\sqrt{4x^2+x}-2024x \right)$.
		\item Tính giới hạn $\lim\limits_{x\to \left(-2\right)^{-} } \dfrac{1-\sqrt{x+3}}{x^2+4x+4}$.
	\end{enumerate}
	
	\loigiai{
		\begin{enumerate}
			\item Ta có $\lim\limits_{x\to +\infty} \left(\sqrt{4x^2+x}-2024x \right)=\lim\limits_{x\to +\infty}\left[ x\cdot \left( \sqrt{4+\dfrac{1}{x}}-2024 \right) \right]$.\\
			Vì $\heva{&\lim\limits_{x\to +\infty} x =+\infty\\
				&\lim\limits_{x\to +\infty} \left( \sqrt{4+\dfrac{1}{x}} -2024 \right)=-2022 <0 }$ nên $\lim\limits_{x\to +\infty}\left[ x\cdot \left( \sqrt{4+\dfrac{1}{x}}-2024 \right) \right] =-\infty$.\\
			Vậy $\lim\limits_{x\to +\infty} \left(\sqrt{4x^2+x}-2024x \right)=-\infty$.
			\item Ta có $\lim\limits_{x\to \left(-2\right)^{-}} \dfrac{1-\sqrt{x+3}}{x^2+4x+4} = \lim\limits_{x\to (-2)^-} \dfrac{-(x+2)}{(x+2)^2\left(1+\sqrt{x+3} \right)} = \lim\limits_{x\to (-2)^-} \left( \dfrac{1}{x+2}\cdot \dfrac{-1}{1+\sqrt{x+3}} \right)$.\\
			Vì $\heva{&\lim\limits_{x\to (-2)^-} \dfrac{1}{x+2} = -\infty\\ &\lim\limits_{x\to (-2)^-} \dfrac{-1}{1+\sqrt{x+3}}=-\dfrac{1}{2} }$ nên $\lim\limits_{x\to (-2)^-} \left( \dfrac{1}{x+2}\cdot \dfrac{-1}{1+\sqrt{x+3}} \right)=+\infty$.\\
			Vậy $\lim\limits_{x\to \left(-2\right)^{-}} \dfrac{1-\sqrt{x+3}}{x^2+4x+4} =+\infty$.
		\end{enumerate}
	}
\end{ex}
\begin{ex}[HKI THPT Chuyên Lê Hồng Phong - Tp HCM. NH24-25]%[1D3H2-5]%[Dự án D - đợt 2 NH24-25- Dương Công Tạo]
	Cho $\lim\limits_{x \rightarrow -\infty}\sqrt{9x^2 + ax + b}+3x=-2$. Tính giá trị của $a$.
	\loigiai{
		Ta có
		\begin{eqnarray*}		
			\lim\limits_{x \rightarrow -\infty}\left(\sqrt{9 x^2+a x}+3 x\right)
			&=&\lim\limits_{x \rightarrow -\infty}\left(\dfrac{a x}{\sqrt{9 x^2+a x}-3 x}\right)\\
			&=&\lim\limits_{x \rightarrow -\infty} \dfrac{a}{-\sqrt{9+\frac{a}{x}}-3}=-\dfrac{a}{6}.\\
		\end{eqnarray*}	
		Do đó $-\dfrac{a}{6}=-2 \Leftrightarrow a=12$.
	}
\end{ex}
\begin{ex}[Chuyên Lê Quý Đôn --Ninh Thuận-HKI-NH24-25]%[1D3H2-3]%[Dự án D - đợt 2 NH24-25- Dương Công Tạo]
	Một hồ nước chứa $800$ m$^3$ nước ngọt. Người ta bơm nước biển có nồng độ muối $35$ gam/lít vào hồ với tối độ $20$ lít/phút. Khi bơm nước biển vào hồ thời gian dài thì nồng độ muối của nước trong hồ xấp xỉ bẳng bao nhiêu gam/lít?
	\loigiai{
		Gọi $t$ là thời gian bơm nước biển vào hồ. $(t>0,\text{ phút})$.\\
		Số lít nước biển bơm vào sau $t$ phút là $20t$ (lít).\\
		Số gam muối sau khi bơm $t$ phút là $35\cdot20t=700t$ (gam).\\
		Thể tích trong bể nước sau khi bơm $t$ phút là $800\ 000+20t$ (lít).\\
		Vậy nồng độ muối trong bể là $f(t)=\dfrac{700t}{800\ 000+20t}$ (gam/lít).\\
		Ta có \allowdisplaybreaks
		\begin{eqnarray*}
			\lim\limits_{t \to +\infty}\dfrac{700t}{800\ 000+20t}=\lim\limits_{t \to +\infty} \dfrac{700}{20+\dfrac{800\ 000}{t}}=35.
		\end{eqnarray*}
		Vậy khi bơm nước vào bể thời gian dài thì nồng độ muối trong hồ xấp xỉ $35$ gam/lít.
	}
\end{ex}
\begin{ex}[HKI TH,THCS \& THPT Hoàng Việt - Đăklăk. NH24-25]%[1D3V2-5]%[Dự án D - đợt 2 NH24-25- Dương Công Tạo]
	Tính giới hạn $\underset{x\to-\infty}{\lim}\,\left(\sqrt{4x^2-x+2}+2x-1\right)$ .
	\loigiai{
		Ta có
		\begin{eqnarray*}
			& &\lim\limits_{x\to-\infty}\left(\sqrt{4x^2-x+2}+2x-1\right)\\
			&=&\lim\limits_{x\to-\infty}\left(\dfrac{4x^2-x+2-\left(2x-1\right)^2}{\sqrt{4x^2-x+2}-2x+1}\right)\\
			&=&\lim\limits_{x\to-\infty}\left(\dfrac{3x+1}{\sqrt{4x^2-x+2}-2x+1}\right)\\
			&=&\lim\limits_{x\to-\infty}\left(\dfrac{3+\dfrac{1}{x}}{-\sqrt{4-\dfrac{1}{x}+\dfrac{2}{x^2}}-2+\dfrac{1}{x}}\right)\\
			&=&-\dfrac{3}{4}.
	\end{eqnarray*}}
\end{ex}
\begin{ex}[ĐỀ KIẾM TRA CUỐI HK1 - SỞ GD BẮC NINH. NH 24-25]%[1D3H2-5]%[Dự án D - đợt 2 NH24-25- Dương Công Tạo]
	Tính các giới hạn sau
	\begin{multicols}{3}
		\begin{enumerate}
			\item $\displaystyle \lim\limits_{x \to -8} |2x - 1|$;
			\item $\displaystyle \lim\limits_{x \to 5} \dfrac{x^2 - 25}{5 - x}$;
			\item $\displaystyle \lim\limits_{x \to -\infty} \dfrac{\sqrt{4x^2 - x + 11}}{x + 2025}$.
		\end{enumerate}
	\end{multicols}
	\loigiai{
		\begin{enumerate}
			\item Ta có $\lim\limits_{x \to - 8} |2x - 1| = |2\cdot (-8)-1| = 17$.
			\item Ta có $\lim\limits_{x \to 5} \dfrac{x^2 - 25}{5-x} = \lim\limits_{x \to 5} \dfrac{(x-5)(x+5)}{-(x-5)} = \lim\limits_{x \to 5} (-x-5) = -5 -5 = -10$. 
			\item Ta có
			\allowdisplaybreaks
			\begin{eqnarray*}
				\lim\limits_{x \to -\infty} \dfrac{\sqrt{4x^2 - x + 11}}{x + 2025} &=& \lim\limits_{x \to -\infty} \dfrac{\sqrt{x^2 \left( 4 - \dfrac{1}{x} + \dfrac{11}{x^2} \right)}}{x + 2025} \\
				&=& \lim\limits_{x \to -\infty} \dfrac{|x| \sqrt{4 - \dfrac{1}{x} + \dfrac{11}{x^2}}}{x + 2025} \\
				&=& \lim\limits_{x \to -\infty} \dfrac{-x \cdot \sqrt{4 - \dfrac{1}{x} + \dfrac{11}{x^2}}}{x \left( 1 + \dfrac{2025}{x} \right)} \\
				&=& \lim\limits_{x \to -\infty} \dfrac{-\sqrt{4 - \dfrac{1}{x} + \dfrac{11}{x^2}}}{1 + \dfrac{2025}{x}} \\
				&=& \dfrac{-\sqrt{4}-0+0}{1+0} \\
				&=& -2.
			\end{eqnarray*}
		\end{enumerate}
	}
\end{ex}

\begin{ex}[De thi cuoi hoc ki 1 lop 11 NH24-25]%[1D3V2-5]%[Dự án D - đợt 2 NH24-25- Dương Công Tạo]
	Tìm giới hạn của các hàm số sau: 
	\begin{enumEX}{3}
		\item $\lim\limits_{x\to 2}\dfrac{3x^2-4x-4}{x-2}$.
		\item $\lim\limits_{x\to 0}\dfrac{\sqrt{1+3x}-1}{2x}$.
		\item $\lim\limits_{x\to 1}\dfrac{\sqrt{x^2+3}+\sqrt{x}-3}{x^2-1}$.
	\end{enumEX}
	\loigiai{
		\begin{enumerate}
			\item $\lim\limits_{x\to 2}\dfrac{3x^2-4x-4}{x-2}=\lim\limits_{x\to 2}\dfrac{(x-2)(3x+2)}{x-2}=\lim\limits_{x\to 2} (3x+2)=8$.
			\item $\lim\limits_{x\to 0}\dfrac{\sqrt{1+3x}-1}{2x}=\lim\limits_{x\to 0}\dfrac{1+3x-1}{2x(\sqrt{1+3x}+1)}=\lim\limits_{x\to 0}\dfrac{3}{2(\sqrt{1+3x}+1)}=\dfrac{3}{4}$.
			\item 
			\begin{eqnarray*}
				\lim\limits_{x\to 1}\dfrac{\sqrt{x^2+3}+\sqrt{x}-3}{x^2-1} &=&
				\lim\limits_{x\to 1}\dfrac{\sqrt{x^2+3}-2}{x^2-1}+\lim\limits_{x\to 1}\dfrac{\sqrt{x}-1}{x^2-1}\\
				&=& \lim\limits_{x\to 1}\dfrac{x^2-1}{\left(x^2-1\right)\left(\sqrt{x^2+3}+2\right)}+\lim\limits_{x\to 1}\dfrac{x-1}{\left(x^2-1\right)\left(\sqrt{x}+1\right)}\\
				&=&\lim\limits_{x\to 1}\dfrac{1}{\sqrt{x^2+3}+2}+\lim\limits_{x\to 1}\dfrac{1}{\left(x+1\right)\left(\sqrt{x}+1\right)}\\
				&=&\dfrac{1}{2}.
			\end{eqnarray*}
		\end{enumerate}
	}
\end{ex}
\begin{ex}[THPT Thị xã Quảng Trị. NH 24-25]%[1D3H2-5]%[Dự án D - đợt 2 NH24-25- Dương Công Tạo]
	Tính giới hạn của hàm số $L= \lim\limits_{x \to 5} \dfrac{\sqrt{x+4}-3}{x^2-25}$.
	\loigiai
	{Ta có 
		\allowdisplaybreaks
		\begin{align*}
			L&= \lim\limits_{x \to 5} \dfrac{\sqrt{x+4}-3}{x^2-25} = \lim\limits_{x \to 5} \dfrac{x+4-9}{(x-5)(x+5)\left(\sqrt{x+4}+3\right)} \\
			&= \lim\limits_{x \to 5}\dfrac{1}{(x+5) \left(\sqrt{x+4}+3\right)} \\
			&= \dfrac{1}{(5+5) \left(\sqrt{5+4}+3\right)} = \dfrac{1}{60}.
		\end{align*}
	}
\end{ex}
\begin{ex}[HKI THPT Số 1 Văn Bàn - Lào Cai. NH24-25]%[1D3V2-8]%[Dự án D - đợt 2 NH24-25- Dương Công Tạo] 
	Một bể chứa $8\,000$ lít nước tinh khiết. Người ta bơm vào bể đó nước muối có nồng độ $50$ gam muối cho mỗi lít nước với tốc độ $25$ lít/phút. 
	\begin{enumerate}
		\item Tìm hàm số biểu thị nồng độ muối trong bể sau $t$ phút.
		\item Nêu nhận xét về nồng độ muối trong bể sau thời gian $t$ ngày càng lớn.
	\end{enumerate}
	\loigiai{
		\begin{enumerate}
			\item Tìm hàm số biểu thị nồng độ muối trong bể sau $t$ phút. 
			\begin{itemize}
				\item Sau $t$ phút, lượng nước muối được bơm vào bể là $25t$ (lít). 
				\item Lượng muối trong bể sau $t$ phút là $25t \cdot 50 = 1\,250t$ (gam). 
				\item Lượng nước trong bể sau $t$ phút là $8\,000 + 25t$ (lít). \\
				Vậy hàm số biểu thị nồng độ muối trong bể sau $t$ phút là
				$$
				y = \dfrac{1\,250t}{8\,000 + 25t} = \dfrac{50t}{320 + t} \quad \text{ (gam/lít)}.
				$$
			\end{itemize}
			\item Nêu nhận xét về nồng độ muối trong bể sau thời gian $t$ ngày càng lớn.\\
			Ta có: 
			$$
			\lim\limits_{t\to +\infty} \dfrac{50t}{320 + t} = \lim\limits_{t\to +\infty} \dfrac{50}{\dfrac{320}{t} + 1} = 50.
			$$
			Vậy sau thời gian $t$ ngày càng lớn, nồng độ muối trong bể càng ngày càng tăng và dần tiến đến $50$ gam/lít.
		\end{enumerate}
	}
\end{ex}