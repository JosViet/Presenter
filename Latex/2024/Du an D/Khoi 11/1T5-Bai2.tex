\newpage
\section{TRUNG VỊ VÀ TỨ PHÂN VỊ CỦA MẪU SỐ LIỆU GHÉP NHÓM}
\subsection{LÝ THUYẾT CẦN NHỚ}
\subsubsection{Trung vị}
\indam{Công thức xác định trung vị của mẫu số liệu ghép nhóm}
\begin{boxdn}
\begin{itemize}
	\item Gọi $ n $ là cỡ mẫu.
	\item Giả sử nhóm $ \left[ u_m ; u_{m+1}\right) $ chứa trung vị;
	\item $ n_m $ là tần số của nhóm chứa trung vị;
	\item $ C=n_1+n_2+ \cdots +n_{m-1} $.
\end{itemize}
Khi đó $ M_e = u_m+\dfrac{\dfrac{n}{2}-C}{n_m}\cdot \left( u_{m+1}-u_m\right). $
\end{boxdn}

\indam{Ý nghĩa của trung vị của mẫu số liệu ghép nhóm}\\
\begin{boxdn}
	Từ dữ liệu ghép nhóm nói chung không thể xác định chính xác trung vị của mẫu số liệu gốc. Trung vị của mẫu số liệu ghép nhóm là giá trị xấp xỉ cho mẫu số liệu gốc và có thể lấy làm giá trị đại diện cho mẫu số liệu.
\end{boxdn}

\subsubsection{Tứ phân vị}
\indam{Công thức xác định tứ phân vị của mẫu số liệu ghép nhóm}
\begin{boxdn}
	\begin{itemize}
		\item Tứ phân vị thứ hai của mẫu số liệu ghép nhóm, kí hiệu $Q_2$, cũng chính là trung vị của mẫu số liệu ghép nhóm.
		\item Để tìm tứ phân vị thứ nhất của mẫu số liệu ghép nhóm, kí hiệu $ Q_1 $, ta thực hiện như sau
		\begin{itemize}
			\item Giả sử nhóm $ \left[ u_m ; u_{m+1}\right)$ chứa tứ phân vị thứ nhất;
			\item $ n_m $ là tần số của nhóm tứ phân vị thứ nhất;
			\item $ C=n_1+n_2+\cdots +n_{m-1} $.
		\end{itemize}
		Khi đó $ Q_1=u_m+\dfrac{\dfrac{n}{4}-C}{n_m}\cdot \left( u_{m+1}-u_m\right).$
		\item Tương tự, để tìm tứ phân vị thứ ba của mẫu số liệu ghép nhóm, kí hiệu $ Q_3 $, ta thực hiện như sau
		\begin{itemize}
			\item Giả sử nhóm $ \left[u_j ; u_{j+1} \right) $ chứa tứ phân vị thứ ba;
			\item $ n_j $ là tần số của nhóm chứa tứ phân vị thứ ba;
			\item $ C=n_1+n_2+\ldots+n_{j-1}$.
		\end{itemize}
		Khi đó $Q_3=u_j+\dfrac{\dfrac{3n}{4}-C}{n_j}\cdot \left(u_{j+1}-u_j \right).$
	\end{itemize}
\end{boxdn} 
\indam{Ý nghĩa của tứ phân vị của mẫu số liệu ghép nhóm}
\begin{boxdn}
	\begin{itemize}
		\item Ba điểm tứ phân vị chia mẫu số liệu đã sắp xếp theo thứ tự không giảm thành bốn phần đều nhau.
		\item Giống với trung vị, nói chung không thể xác định chính xác các điểm tứ phân vị của mẫu số liệu ghép nhóm.
		\item Bộ ba tứ phân vị của mẫu số liệu ghép nhóm là giá trị xấp xỉ cho tứ phân vị của mẫu số liệu gốc và được sử dụng làm giá trị đo xu thế trung tâm của mẫu số liệu.
		\item Tứ phân vị thứ nhất và thứ ba đo xu thế trung tâm của nửa dưới (các dữ liệu nhỏ hơn $ Q_2 $) và nửa trên (các dữ liệu lớn hơn $ Q_2 $) của mẫu số liệu.
	\end{itemize}
\end{boxdn}

%-------------------------------------------------------------------------------------------------------------
\subsection{PHÂN LOẠI VÀ PHƯƠNG PHÁP GIẢI TOÁN}
\begin{dang}{Câu hỏi lý thuyết}
\end{dang}

\begin{vd}%[1D5B2-1]%[Dự án đề cương 3 Khối NH24-25-Dot1-Nguyễn Sĩ Đạt]
	Cho mẫu số liệu ghép nhóm về thống kê điểm số của học sinh tham dự kỳ thi học sinh giỏi
	toán, ta có bảng số liệu sau
	\begin{center}
		\begin{tabular}{|c|c|c|c|c|c|c|c|}
			\hline
			Điểm&$[8;10)$&$[10;12)$&$[12;14)$&$[14;16)$&$[16;18)$&$[18;20)$\\
			\hline
			Số học sinh &$6$&$21$&$30$&$25$&$14$&$4$\\
			\hline
		\end{tabular}
	\end{center}
	Tìm nhóm chứa trung vị của mẫu số liệu ghép nhóm trên.
	\loigiai{
		Số phần tử của mẫu là $n=6+21+30=25+14+4=100$, ta có $\dfrac{n}{2}=\dfrac{100}{2}=50$.\\
		Do $cf_2=27<50<57=cf_3$ nên trung vị của mẫu nằm ở nhóm $3$ là nhóm $[12;14)$.
	}
\end{vd}

\begin{vd}%[1D5B2-1]%[Dự án đề cương 3 Khối NH24-25-Dot1-Nguyễn Sĩ Đạt]
	Khảo sát thời gian xem điện thoại trong một ngày của một số học sinh khối 11 thu được mẫu số liệu ghép nhóm sau\begin{center}
		\begin{tabular}{|c|c|c|c|c|c|}
			\hline Thời gian (phút) & {$[0 ; 20)$} & {$[20 ; 40)$} & {$[40 ; 60)$} & {$[60 ; 80)$} & {$[80 ; 100)$} \\
			\hline Số học sinh & 5 & 9 & 12 & 10 & 6 \\
			\hline
		\end{tabular}
	\end{center}
	Xác định nhóm chứa các tứ phân vị của mẫu số liệu trên.
	\loigiai{Ta có cỡ mẫu là $42$. Khi đó: \\
	Tứ phân vị thứ nhất thuộc nhóm $[20 ; 40)$.\\
	Tứ phân vị thứ hai thuộc nhóm $[40;60)$.\\
	Tứ phân vị thứ ba thuộc nhóm $[60;80)$.
	}
\end{vd}

\begin{dang}{Tìm trung vị}
\end{dang}

\begin{vd}%[1D5H2-2]%[Dự án đề cương 3 Khối NH24-25-Dot1-Nguyễn Sĩ Đạt]
	Kết quả khảo sát cân nặng của $ 25 $ quả bơ ở một lô hàng cho trong bảng sau
	\begin{center}
		\begin{tabular}{|c|c|c|c|c|c|}
			\hline 
			\textbf{Cân nặng}\textbf{ (g)}	& $ \left[150 ; 155 \right) $ & $ \left[ 155 ; 160\right)  $ & $ \left[160 ; 165\right)  $ & $ \left[ 165 ; 170\right)  $ & $ \left[170 ; 175 \right)  $ \\ 
			\hline 
			\textbf{Số quả bơ}	& $ 1 $ & $ 7 $ & $ 12 $ & $ 3 $ & $ 2 $ \\ 
			\hline 
		\end{tabular} 
	\end{center}
	Hãy tìm trung vị của mẫu số liệu ghép nhóm trên.
	\loigiai{
		Gọi $ x_1; x_2; \ldots ; x_{25} $ là cân nặng của $ 25$ quả bơ xếp theo thứ tự không giảm.\\
		Do $ x_1\in \left[150 ; 155 \right) $; $ x_2, \ldots, x_8 \in \left[ 155 ; 160\right) $; $ x_9, \ldots, x_{20} \in \left[ 160 ; 165\right) $ nên trung vị của mẫu số liệu $ x_1; x_2; \ldots; x_{25} $ là \\$ x_{13}\in \left[ 160 ; 165\right)$.\\
		Ta xác định được $ n=25 $, $ n_m=12 $, $ C=1+7=8 $, $ u_m=160 $, $ u_{m+1}=165 $.\\
		Vậy trung vị của mẫu số liệu ghép nhóm là $$ M_e=160+\dfrac{\dfrac{25}{2}-8}{12}\cdot(165-160) =161{,}875.$$
	}
\end{vd}

\begin{vd}%[1D5V2-2]%[Dự án đề cương 3 Khối NH24-25-Dot1-Nguyễn Sĩ Đạt]
	Trong một hội thao, thời gian chạy $ 200 $ m của một nhóm các vận động viên được ghi lại ở bảng sau
	\begin{center}
		\begin{tabular}{|c|c|c|c|c|c|}
			\hline 
			\textbf{Thời gian} \textbf{(giây)}& $ \left[21 ; 21{,}5 \right)  $ & $ \left[ 21{,}5 ; 22\right)  $ & $ \left[ 22 ; 22{,}5\right)  $ & $ \left[ 22{,}5 ; 23\right)  $ & $ \left[ 23 ; 23{,}5\right)  $ \\ 
			\hline 
			\textbf{Số vận động viên} & $ 5 $ & $ 12 $ & $ 32 $ & $ 45 $ & $ 30 $ \\ 
			\hline 
		\end{tabular} 
	\end{center}
	Dựa vào bảng số liệu trên, ban tổ chức muốn chọn ra khoảng $ 50 \% $ số vận động viên chạy nhanh nhất để tiếp tục thi vòng $ 2 $. Ban tổ chức nên chọn các vận động viên có thời gian chạy không quá bao nhiêu giây?
	\loigiai{
		Số vận động viên tham gia là $$n=5+12+32+45+30=124.$$
		Gọi $ x_1; x_2; \ldots ; x_{124} $ lần lượt là thời gian chạy $ 200 $ m của $ 124 $ vận động viên được xếp theo thứ tự không giảm.\\
		Do $ x_1, \ldots, x_5 \in \left[ 21 ; 21{,}5 \right)$, $ x_6, \ldots, x_{17} \in \left[ 21{,}5 ; 22\right) $, $ x_{18}, \ldots, x_{49} \in \left[22 ; 22{,}5\right) $, $ x_{50},\ldots, x_{94} \in \left[ 22{,}5 ; 23\right) $ nên trung vị của mẫu số liệu $ x_1; x_2; \ldots ;x_{124} $ là\\
		$$\dfrac{1}{2}\cdot \left( x_{62}+x_{63}\right) \in  \left[ 22{,}5 ; 23\right).$$
		Ta xác định được $ n=124 $; $ n_m=45$; $ C=5+12+32=49 $; $ u_m= 22{,}5$; $ u_{m+1}=23$.\\
		Trung vị của mẫu số liệu ghép nhóm là $$M_e=22{,}5 +\dfrac{\dfrac{124}{2}-49}{45}\cdot \left( 23-22{,}5\right)= \dfrac{1019}{45}\approx 22{,}64.$$
		Vậy ban tổ chức nên chọn các vận động viên  có thời gian chạy không quá $ 22{,}64$ (giây) được tiếp tục thi vòng hai.
	}
\end{vd}

\begin{dang}{Tìm tứ phân vị}
\end{dang}

\begin{vd}%[1D5V2-3]%[Dự án đề cương 3 Khối NH24-25-Dot1-Nguyễn Sĩ Đạt]
	Một người thống kê lại thời gian thực hiện các cuộc gọi điện thoại của người đó trong một tuần ở bảng sau
	\begin{center}
		\begin{tabular}{|c|c|c|c|c|c|c|}
			\hline 
			\begin{tabular}{c}
				\textbf{Thời gian}\\
				\textbf{ (đơn vị: giây)}
			\end{tabular} 
			& $ \left[ 0;60\right)  $ & $ \left[ 60 ; 120\right)  $ & $ \left[ 120 ; 180\right)  $ & $ \left[ 180; 240\right)  $ & $ \left[240 ; 300 \right)  $ & $ \left[300 ; 360 \right)  $ \\ 
			\hline 
			\textbf{Số cuộc gọi}	& $ 8 $ & $ 10 $ & $ 7 $ & $ 5 $ &$ 2 $  & $ 1 $ \\ 
			\hline 
		\end{tabular} 
	\end{center}
	Hãy ước lượng các tứ phân vị của mẫu số liệu ghép nhóm trên.	
	\loigiai{
		Gọi $ x_1; x_2; \ldots; x_{33} $ là mẫu số liệu được xếp theo thứ tự không giảm.\\
		Ta có $ x_1, \ldots, x_{8}\in \left[ 0 ; 60\right)  $; $ x_9, \ldots, x_{18} \in \left[ 60 ; 120\right) $, $ x_{19}, \ldots, x_{25} \in \left[ 120 ; 180\right) $;  $ x_{26}, \ldots, x_{30} \in \left[ 180 ; 240\right) $; $ x_{31}, x_{32} \in \left[ 240 ; 300\right) $;\\ $ x_{33}\in \left[ 300 ; 360\right) $.\\
		Tứ phân vị thứ nhất của dãy số liệu $ x_1; x_2; \ldots; x_{33} $ là  $ x_9\in \left[60 ; 120 \right) $ nên tứ phân vị thứ nhất của mẫu số liệu ghép nhóm là $$ Q_1=60+\dfrac{\dfrac{33}{4}-8}{10}\cdot (120-60)=\dfrac{123}{2} \approx 61{,}5.$$
		Tứ phân vị thứ hai của dãy số liệu $ x_1; x_2; \ldots; x_{33} $ là  $ x_{17}\in \left[60 ; 120 \right) $ nên tứ phân vị thứ hai của mẫu số liệu ghép nhóm là $$ Q_2=60+\dfrac{\dfrac{33}{2}-8}{10}\cdot (120-60)=111.$$
		Tứ phân vị thứ ba của dãy số liệu $ x_1; x_2; \ldots; x_{33} $ là  $ x_{25}\in \left[120 ; 180 \right) $ nên tứ phân vị thứ ba của mẫu số liệu ghép nhóm là $$ Q_3=120+\dfrac{\dfrac{33\cdot3}{4}-(8+10)}{7}\cdot (180-120)=\dfrac{1245}{7}\approx 177{,}857.$$
	}
\end{vd}

\begin{vd}%[1D5V2-3]%[Dự án đề cương 3 Khối NH24-25-Dot1-Nguyễn Sĩ Đạt]
	Một hãng xe ô tô thống kê lại số lần gặp sự cố về động cơ của $ 100 $ chiếc xe cùng loại sau $ 2 $ năm sử dụng đầu tiên ở bảng sau
	\begin{center}
		\begin{tabular}{|c|c|c|c|c|c|}
			\hline 
			\textbf{Số lần gặp sự cố}	& $ \left[ 1 ; 2\right]  $ & $ \left[3 ; 4\right]  $ & $ \left[ 5 ; 6\right]  $ & $ \left[ 7 ; 8\right]  $ & $ \left[ 9 ; 10\right]  $ \\ 
			\hline 
			\textbf{Số xe}	& $ 17 $  & $ 33 $ & $ 25 $ & $ 20 $ & $ 5 $ \\ 
			\hline 
		\end{tabular} 
	\end{center}
	\begin{enumerate}
		\item Hãy ước lượng các tứ phân vị của mẫu số liệu ghép nhóm trên.
		\item Một người cho rằng có trên $ 25\% $ xe của hãng gặp không ít hơn $ 4 $ sự cố về động cơ trong $ 2 $ năm sử dụng đầu tiên. Nhận định trên có hợp lí không?
	\end{enumerate}
	\loigiai{
		\begin{enumerate}
			\item Do số lần gặp sự cố là số nguyên nên ta hiệu chỉnh lại như sau
			\begin{center}
				\begin{tabular}{|c|c|c|c|c|c|}
					\hline 
					\textbf{Số lần gặp sự cố}	& $ \left[ 0{,}5 ; 2{,}5\right)  $ & $ \left[ 2{,}5 ; 4{,}5\right)  $ & $ \left[4{,}5 ; 6{,}5 \right)  $ & $ \left[ 6{,}5 ; 8{,}5\right) $ &$ \left[8{,}5 ; 10{,}5 \right)  $  \\ 
					\hline 
					\textbf{Số xe}	& $ 17 $ & $ 33 $ &$ 25 $  & $ 20 $ & $ 5 $ \\ 
					\hline 
				\end{tabular} 
			\end{center}
			Gọi $ x_1; x_2; \ldots; x_{100} $ là mẫu số liệu được xếp theo thứ tự không giảm.\\
			Ta có $ x_1, \ldots, x_{17}\in \left[ 0{,}5 ; 2{,}5\right)  $;  $ x_{18}, \ldots, x_{50}\in \left[ 2{,}5 ; 4{,}5\right)  $; $ x_{51}, \ldots, x_{75}\in \left[ 4{,}5 ; 6{,}5\right) $; $ x_{76}, \ldots, x_{95}\in \left[ 6{,}5 ; 8{,}5\right)  $; $ x_{96}, \ldots, x_{100}\in \left[ 8{,}5 ; 10{,}5\right)  $.\\
			Tứ phân vị thứ hai của dãy số liệu $ x_1; x_2; \ldots; x_{100} $ là $ \dfrac{1}{2}\left( x_{50}+x_{51}\right) $. Do $ x_{50}\in \left[ 2{,}5 ; 4{,}5\right) $ và $ x_{51}\in \left[4{,}5 ; 6{,}5 \right) $ nên tứ phân vị thứ hai của mẫu số liệu ghép nhóm là $ Q_2=4{,}5$.\\
			Tứ phân vị thứ nhất của dãy số liệu $ x_1; x_2; \ldots; x_{100} $ là $ \dfrac{1}{2}\left( x_{25}+x_{26}\right) $. Do $ x_{25} $ và $ x_{26} $ thuộc nhóm $ \left[2{,}5 ; 4{,}5 \right)$ nên tứ phân vị thứ nhất của mẫu số liệu ghép nhóm là $$ Q_1=2{,}5+\dfrac{\dfrac{1\cdot100}{4}-17}{33}\cdot (4{,}5-2{,}5)=\dfrac{197}{66}\approx 2{,}98.$$
			Tứ phân vị thứ ba của dãy số liệu $ x_1; x_2; \ldots; x_{100} $ là $ \dfrac{1}{2}\left(x_{75}+x_{76} \right)$. Do $ x_{75}\in \left[4{,}5 ; 6{,}5 \right) $ và $ x_{76}\in \left[ 6{,}5 ; 8{,}5\right)$ nên tứ phân vị thứ ba của mẫu số liệu ghép nhóm là $ Q_3=6{,}5$.
			\item Do tứ phân vị thứ nhất $ Q_1\approx 2{,}98 $ nên nhận định trên là hợp lý.
		\end{enumerate}
	}
\end{vd}

\begin{vd}%[1D5C2-3]%[Dự án đề cương 3 Khối NH24-25-Dot1-Nguyễn Sĩ Đạt]
	Thời gian luyện tập trong một ngày (tính theo giờ) của một số vận động viên được ghi lại ở bảng sau
	\begin{center}
		\begin{tabular}{|c|c|c|c|c|c|}
			\hline 
			\textbf{Thời gian luyện tập (giờ)}	& $ \left[ 0 ; 2\right)  $ & $ \left[ 2 ; 4\right)  $ & $ \left[ 4 ; 6\right)  $ & $ \left[ 6 ; 8\right)  $ & $ \left[8 ; 10 \right)  $ \\ 
			\hline 
			\textbf{Số vận động viên}	& $ 3 $ & $ 8 $ & $ 12 $ & $ 12 $ & $ 4 $ \\ 
			\hline 
		\end{tabular} 
	\end{center}
	Hãy xác định các tứ phân vị của mẫu số liệu đã cho.\\
	Huấn luyện viên muốn xác định nhóm gồm $ 25\% $ các vận động viên có số giờ luyện tập cao nhất. Hỏi huấn luyện viên nên chọn các vận động viên có thời gian luyện tập từ bao nhiêu giờ trở lên vào nhóm này?
	\loigiai{
		Số vận động viên được khảo sát là $ n=3+8+12+12+4=39$.\\
		Gọi $ x_1 $; $ x_2 $; \ldots ;$ x_{39} $ là thời gian luyện tập của $ 39 $ vận động viên được xếp theo thứ tự không giảm.\\
		Ta có $ x_1, x_2, x_3 \in \left[ 0 ; 2\right) $; $ x_4, \ldots, x_{11}\in \left[2 ; 4 \right) $; $ x_{12}, \ldots, x_{23}\in \left[ 4 ; 6\right) $; $ x_{24}, \ldots, x_{35}\in \left[6;8\right) $; $ x_{36},\ldots , x_{39}\in \left[ 8 ; 10\right) $. Do đó đối với dãy số liệu $ x_1 $; $ x_2 $; \ldots; $ x_{39} $ thì
		\begin{itemize}
			\item Tứ phân vị thứ nhất là $ x_{10} $ thuộc nhóm $\left[ 2 ; 4\right)$;
			\item Tứ phân vị thứ hai là $ x_{20} $ thuộc nhóm $ \left[ 4 ; 6\right)$;
			\item Tứ phân vị thứ ba là $ x_{30} $ thuộc nhóm $ \left[ 6 ; 8\right) $.
		\end{itemize}
		Ta nói nhóm $ \left[2 ; 4 \right)$ là nhóm chứa tứ phân vị thứ nhất; nhóm $ \left[ 4 ; 6\right) $ là nhóm tứ phân vị thứ hai; nhóm $ \left[ 6 ; 8\right) $ là nhóm chứa tứ phân vị thứ ba.\\
		Tứ phân vị thứ hai của dãy số liệu $ x_1; x_2; \ldots ;  x_{39}$ là $ x_{20}\in \left[ 4 ; 6\right)  $. Do đó tứ phân vị thứ hai của mẫu số liệu ghép nhóm là $$Q_2=4+\dfrac{\dfrac{2\cdot 39}{4}-(3+8)}{12}\cdot(6-4)=\dfrac{65}{12}\approx 5{,}417.$$ 
		Tứ phân vị thứ nhất của dãy số liệu $ x_1; x_2; \ldots ;  x_{39}$ là $ x_{10}\in \left[ 2 ; 4\right) $. Do đó tứ phân vị thứ nhất của mẫu số liệu ghép nhóm là $$Q_1=2+\dfrac{\dfrac{1\cdot 39}{4}-3}{8}\cdot(4-2)=\dfrac{59}{16}\approx 3{,}6875.$$ 
		Tứ phân vị thứ ba của dãy số liệu $ x_1; x_2; \ldots ;  x_{39}$ là $ x_{30}\in \left[ 6 ; 8\right) $. Do đó tứ phân vị thứ ba của mẫu số liệu ghép nhóm là $$Q_3=6+\dfrac{\dfrac{3\cdot 39}{4}-(3+8+12)}{12}\cdot(8-6)=\dfrac{169}{24}\approx 7{,}042.$$ 
		Huấn luyện viên muốn xác định nhóm gồm $ 25\% $ các vận động viên có số giờ luyện tập cao nhất thì huấn luyện viên nên chọn các vận động viên có thời gian tập luyện từ $ Q_3 \approx 7{,}042$ (giờ).
	}
\end{vd}

%-----------------------------------------------------------------------------
\subsection{Bài tập rèn luyện}
\ind{PHẦN I.} \inden{Câu trắc nghiệm nhiều phương án lựa chọn. Mỗi câu hỏi học sinh chỉ chọn một phương án.}\\
\setcounter{ex}{0}
\Opensolutionfile{ans}[ans/1D5-Bai5-TN]

\begin{ex}[Nguyễn Du - TPHCM 24-25]%[1D5H2-1]%[Dự án đề cương 3 Khối NH24-25-Dot1-Nguyễn Sĩ Đạt]
	Trong các số đặc trưng đo xu thế trung tâm dưới đây, số nào thỏa mãn có $25 \%$ giá trị trong mẫu số liệu nhỏ hơn nó và $75 \%$ giá trị trong mẫu số liệu lớn hơn nó?
	\choice
	{tứ phân vị thứ ba}
	{trung vị}
	{số trung bình}
	{\True tứ phân vị thứ nhất}	
	\loigiai{
		Số thỏa mãn có $25 \%$ giá trị trong mẫu số liệu nhỏ hơn nó và $75 \%$ giá trị trong mẫu số liệu lớn hơn nó là tứ phân vị thứ nhất.
	}
\end{ex}

%\begin{ex}[Mạc Đĩnh Chi - TPHCM 24-25]%[1D5B2-1]%[Dự án đề cương 3 Khối NH24-25-Dot1-Nguyễn Sĩ Đạt]
%	Thống kê chiều cao của học sinh lớp $ 11A$ ta có bảng số liệu sau
%	\begin{center}
%		$\begin{array}{|c|c|c|c|c|c|}
%			\hline \text { Chiều cao (cm)} & {[150 ; 156)} & {[156 ; 162)} & {[162 ; 168)} & {[168 ; 174)} & {[174 ; 180)} \\
%			\hline \text { Số học sinh } & 8 & 12 & 13 & 7 & 4 \\
%			\hline
%		\end{array}$
%	\end{center}
%	Hỏi nhóm chứa trung vị là nhóm nào sau đây?
%	\choice
%	{\True $[162;168)$}
%	{$[150;156)$}
%	{$[156;162)$}
%	{$[168;174)$}
%	\loigiai{
%		Ta có bảng sau
%		\begin{center}
%			$\begin{array}{|c|c|c|c|c|c|}
%				\hline \text { Chiều cao (cm)} & {[150 ; 156)} & {[156 ; 162)} & {[162 ; 168)} & {[168 ; 174)} & {[174 ; 180)} \\
%				\hline \text { Số học sinh } & 8 & 12 & 13 & 7 & 4 \\
%				\hline \text { Tần số tích lũy } & 8 & 20 & 33 & 40 & 44 \\
%				\hline
%			\end{array}$
%		\end{center}
%		Do $N=44$ là số chẵn nên trung vị của mẫu số liệu là trung bình của số thứ $22$ và $23$ thuộc nhóm $[162 ; 168)$.\\
%		Vậy $[162 ; 168)$ là nhóm chứa trung vị.
%	}
%\end{ex}

\begin{ex}[Lê Thánh Tôn - TPHCM 24-25]%[1D5B2-1]%[Dự án đề cương 3 Khối NH24-25-Dot1-Nguyễn Sĩ Đạt]
	Kiểm tra điện lượng của một số viên pin tiểu của một hãng sản xuất thu được kết quả sau
	\begin{center}
		\begin{tabular}{|>{\centering\arraybackslash}p{3cm}*{5}{|>{\centering\arraybackslash}p{2cm}}|}
			\hline
			Điện lượng\newline
			(nghìn mAh) & $[0{,}9; 0{,}95)$ & $[0{,}95; 1)$ & $[1; 1{,}05)$ & $[1{,}05; 1{,}1)$ & $[1{,}1; 1{,}15)$ \\
			\hline
			Số viên pin & $11$ & $19$ & $35$ & $12$ & $5$ \\
			\hline
		\end{tabular}
	\end{center}
	Trung vị của mẫu số liệu trên thuộc nhóm nào sau đây?
	\choice
	{$[0{,}9; 0{,}95)$}
	{\True $[1; 1{,}05)$}
	{$[0{,}95; 1)$}
	{$[1{,}05; 1{,}1)$}
	\loigiai{
		Số viên pin được kiểm tra điện lượng là
		$$n=11+19+35+12+5=82.$$
		Gọi $x_1$, $x_2$, $\ldots$, $x_{82}$ lần lượt là mức điện lượng (nghìn mAh) được xếp theo thứ tự không giảm.\\
		Do $x_1$, $\ldots$, $x_{11}\in [0{,}9; 0{,}95)$; $x_{12}$, $\ldots$, $x_{30}\in [0{,}95; 1)$; $x_{31}$, $\ldots$, $x_{65}\in [1; 1{,}05)$ nên trung vị của mẫu số liệu $x_1$, $x_2$, $\ldots$, $x_{82}$ là
		$$\dfrac{1}{2}\cdot\left(x_{41}+x_{42}\right)\in [1; 1{,}05).$$
		Vậy trung vị của mẫu số liệu trên thuộc nhóm $[1; 1{,}05)$.
	}
\end{ex}

%\begin{ex}[THTH ĐHSP - TPHCM 24-25]%[1D5H2-2]%[Dự án đề cương 3 Khối NH24-25-Dot1-Nguyễn Sĩ Đạt]
%	Bảng sau là thống kê $120$ điểm thi đánh giá năng lực môn Toán của một trường THPT qua thang điểm $100$:
%	\begin{center}
%		\begin{tabular}{|c|c|c|c|c|c|}
%			\hline Điểm&{$[0;20)$}&{$[20;40)$}&$[40;60)$&{$[60;80)$}&{$[80;100)$}\\
%			\hline Số học sinh&$25$&$35$&$37$&$15$&$8$\\
%			\hline
%		\end{tabular}
%	\end{center}
%	Trung vị của mẫu số liệu ghép nhóm thuộc nửa khoảng nào sau đây?
%	\choice
%	{$[50;55)$}
%	{$[45;50)$}
%	{$[55;60)$}
%	{\True $[40;45)$}
%	\loigiai
%	{Gọi $x_1$;$x_2$;...$x_{120}$ là số điểm thi đánh giá năng lực môn Toán của một trường THPT xếp theo thứ tự không giảm.\\
%		Do $x_1$;...;$x_{25}\in \left[0;20\right)$; $x_{26}$;...;$x_{60}\in \left[20;40\right)$; $x_{61}\in [40;60)$ nên trung vị của mẫu số liệu là $Q_2=40\in [40;45)$.
%	}
%\end{ex}

\begin{ex}[Nguyễn Trãi - Khánh Hòa 24-25]%[1D5H2-2]%[Dự án đề cương 3 Khối NH24-25-Dot1-Nguyễn Sĩ Đạt]
	Một cửa hàng sách thống kê số sách tham khảo bán được trong hai tháng ở bảng sau
	\begin{center}
		\begin{tabular}{|l|c|c|c|c|c|}
			\hline 
			Số sách & {$\left[15;21\right)$} & {$\left[21;27\right)$} & {$\left[27;33\right)$} & {$\left[33;39\right)$} & {$\left[39;45\right)$} \\
			\hline Số ngày & $13$ & $18$ & $20$ & $7$ & $3$ \\
			\hline
		\end{tabular}
	\end{center}
	Trung vị của mẫu số liệu ghép nhóm trên thuộc nhóm nào?
	\choice
	{$\left[27 ; 33\right)$}
	{$\left[33 ; 39\right)$}
	{$\left[15 ; 21\right)$}
	{\True $\left[21 ; 27\right)$}
	\loigiai{
		Cỡ mẫu $n=13+18+20+7+3=61\Rightarrow \dfrac{n}{2}=30{,}5$ nên ta xét nhóm $\left[21;27\right)$.\\
		Trung vị là $M_e=21+\dfrac{30{,}5-13}{18}\cdot 6\approx 21{,}97$.
	}
\end{ex}

\begin{ex}[Phan Bội Châu - Lâm Đồng 24-25]%[1D5H2-2]%[Dự án đề cương 3 Khối NH24-25-Dot1-Nguyễn Sĩ Đạt]
	Đo cân nặng của lớp 10A gồm 40 học sinh, ta thu được kết quả sau
	\begin{center}
		\begin{tabular}{|>{\centering\arraybackslash}m{3cm}|>{\centering\arraybackslash}m{1.5cm}|>{\centering\arraybackslash}m{1.5cm}|>{\centering\arraybackslash}m{1.5cm}|>{\centering\arraybackslash}m{1.5cm}|>{\centering\arraybackslash}m{1.5cm}|>{\centering\arraybackslash}m{1.5cm}|>{\centering\arraybackslash}m{1.5cm}|>{\centering\arraybackslash}m{1.5cm}|}
			\hline
			Khối lượng (kg)&$[40;45)$&$[45;50)$&$[50;55)$&$[55;60)$&$[60;65)$&$[65;70)$&$[70;75)$&$[75;80]$\\
			\hline
			Số học sinh&$4$&$13$&$7$&$5$&$6$&$2$&$1$&$2$\\ 
			\hline
		\end{tabular}
	\end{center}
	Trung vị của mẫu số liệu ghép nhóm thuộc khoảng nào sau đây?
	\choice 
	{\True $[50;55)$}{$[40;45)$}{$[45;50)$}{$[55;60)$}
	\loigiai{
		Ta có $40:2=20$, vậy $M_e$ thuộc $[50;55)$.
	}
\end{ex}

\begin{ex}[Lê Quý Đôn - TPHCM 24-25]%[1D5B2-1]%[Dự án đề cương 3 Khối NH24-25-Dot1-Nguyễn Sĩ Đạt]
	Khảo sát thời gian tập thể dục của một số học sinh khối 11 thu được mẫu số liệu ghép nhóm sau
	\begin{center}
		\begin{tabular}{|c|c|c|c|c|c|}
			\hline
			Thời gian (phút) & $[0;20)$ & $[20;40)$ & $[40;60)$ & $[60;80)$ & $[80;100)$ \\
			\hline
			Số học sinh & $5$ & $9$ & $12$ & $10$ & $6$ \\
			\hline
		\end{tabular}
	\end{center}
	Nhóm chứa tứ phân vị thứ nhất của mẫu số liệu trên là
	\choice
	{$[40;60)$}
	{$[80;100)$}
	{\True $[20;40)$}
	{$[60;80)$}
	\loigiai{
		Tổng số học sinh của mẫu số liệu là $n=5+9+12+10+6=42$.\\ Khi đó $\dfrac{n}{4}=\dfrac{42}{4}=10{,}5$.\\
		Do đó nhóm chứa tứ phân vị thứ nhất là $[20;40)$.
	}
\end{ex}

\begin{ex}[Phan Bội Châu - Lâm Đồng 24-25]%[1D5H2-3]%[Dự án đề cương 3 Khối NH24-25-Dot1-Nguyễn Sĩ Đạt]
	Đo chiều dài của $60$ lá dương xỉ trưởng thành, ta thu được kết quả sau
	\begin{center}
		\begin{tabular}{|>{\centering\arraybackslash}m{3cm}|>{\centering\arraybackslash}m{2cm}|>{\centering\arraybackslash}m{2cm}|>{\centering\arraybackslash}m{2cm}|>{\centering\arraybackslash}m{2cm}|}
			\hline
			Chiều dài (cm)&$[10;20)$&$[20;30)$&$[30;40)$&$[40;50]$\\
			\hline
			Số lá&$8$&$18$&$24$&$10$\\ 
			\hline
		\end{tabular}
	\end{center}
	Tứ phân vị thứ nhất $Q_1$ của mẫu số liệu ghép nhóm thuộc khoảng nào sau đây?
	\choice 
	{\True $[20;30)$}{$[10;20)$}{$[40;50]$}{$[30;40)$}
	\loigiai{
		Ta có $60:4=15$, vậy $Q_1$ thuộc $[20;30)$.
	}
\end{ex}

%\begin{ex}[Lê Quý Đôn - TPHCM 24-25]%[1D5H2-2]%[Dự án đề cương 3 Khối NH24-25-Dot1-Nguyễn Sĩ Đạt]
%	Cho bảng số liệu về chiều cao của 100 học sinh một trường trung học phổ thông dưới đây:
%	\begin{center}
%		\begin{tabular}{|c|c|c|}
%			\hline
%			Nhóm & Chiều cao (cm) & Số học sinh \\
%			\hline
%			1 & $[150;153)$ & $7$ \\
%			\hline
%			2 & $[153;156)$ & $13$ \\
%			\hline
%			3 & $[156;159)$ & $40$ \\
%			\hline
%			4 & $[159;162)$ & $21$ \\
%			\hline
%			5 & $[162;165)$ & $13$ \\
%			\hline
%			6 & $[165;168)$ & $6$ \\
%			\hline
%		\end{tabular}
%	\end{center}
%	Ước lượng trung vị của mẫu số liệu ghép nhóm trên được kết quả là
%	\choice
%	{$157{,}25$}
%	{$157{,}76$}
%	{$160{,}45$}
%	{\True $158{,}25$}
%	\loigiai{
%		Cỡ mẫu $n = 100$.\\
%		Gọi $x_1$, $x_2$, $\ldots$, $x_{100}$ là mẫu số liệu gốc gồm chiều cao của $100$ học sinh được xếp theo thứ tự không giảm.\\
%		Ta có $x_1,\ldots, x_7 \in [150; 153]$; $x_8, \ldots, x_{20} \in [153; 156]$;\ldots;$x_{95}\ldots, x_{100} \in [165;168]$.\\
%		Trung vị của mẫu số liệu gốc là $\dfrac{1}{2}\left( x_{50}+x_{51}\right) \in [156;159]$.\\
%		Do đó, trung vị của mẫu số liệu ghép nhóm là
%		$$M_\text{e}=156+\dfrac{\tfrac{100}{2}-20}{40}\cdot (159-156)=158{,}25.$$
%	}
%\end{ex}

\begin{ex}[Nguyễn Thượng Hiền - TPHCM 24-25]%[1D5H2-2]%[Dự án đề cương 3 Khối NH24-25-Dot1-Nguyễn Sĩ Đạt]
	Thời gian (phút) xem tivi mỗi buổi tối của một số học sinh được cho trong bảng sau
	\begin{center}
		\begin{tabular}{|c|c|c|c|c|c|c|c|}
			\hline
			Thời gian   & {$[6{,}5; 9{,}5)$} & {$[9{,}5; 12{,}5)$} & {$[12{,}5; 15{,}5)$} & {$[15{,}5; 18{,}5)$} & {$[18{,}5; 21{,}5)$} & {$[21{,}5; 24{,}5)$} & {$[24{,}5; 27{,}5)$} \\
			\hline
			Số học sinh & $2$                & $3$                 & $12$                 & $15$                 & $24$                 & $2$                  & $2$                  \\
			\hline
		\end{tabular}
	\end{center}
	Trung vị của mẫu số liệu ghép nhóm này bằng
	\choice
	{\True $18{,}1$}
	{$21{,}1$}
	{$15$}
	{$15{,}1$}
	\loigiai{
		Cỡ mẫu $n = 60$. \\
		Vậy trung vị của mẫu số liệu thuộc $[15,5;18,5)$. \\
		$M_e=15{,}5+\dfrac{\tfrac{60}{2}-(3+2+12)}{15} \cdot (18{,}5-15{,}5)=18{,}1$.
	}
\end{ex}

\begin{ex}[Hoàng Việt - Đăklăk 24-25]%[1D5H2-2]%[Dự án đề cương 3 Khối NH24-25-Dot1-Nguyễn Sĩ Đạt]
	Người ta tiến hành phỏng vấn $40$ người về một mẫu áo khoác. Người điều tra yêu cầu cho điểm
	mẫu áo đó theo thang điểm là $100$. Kết quả được trình bày trong bảng ghép nhóm sau
	\begin{center}
		\begin{tabular}{|c|c|c|c|c|c|}
			\hline
			Nhóm&$[50;60)$&$[60;70)$&$[70;80)$&$[80;90)$&$[90;100)$\\
			\hline
			Tần số &$4$&$5$&$23$&$6$&$2$\\
			\hline
		\end{tabular}
	\end{center}
	Trung vị của mẫu số liệu ghép nhóm trên gần nhất với giá trị
	\choice
	{$74$}
	{\True $75$}
	{$76$}
	{$77$}
	\loigiai{
		Ta có $\dfrac{n}{2}=\dfrac{40}{2}=20$ và $4+5<20<4+5+23$ nên nhóm $[70;80)$ là nhóm đầu tiên có tần số tích luỹ lớn hơn hoặc bằng $20$.\\
		Nhóm $[70;80)$ có $r=70$, $d=10$, $n_3=23$, nhóm $[60;70)$ có $cf_2=9$.\\
		Vậy trung vị của mẫu ghép nhóm là
		$$M_e=70+\dfrac{20-9}{23}\cdot 10\approx74{,}79.$$
		Vậy trung vị của mẫu số liệu ghép nhóm trên gần nhất với giá trị $75$.
	}
\end{ex}

\begin{ex}[Chuyên Hạ Long - Quảng Ninh 24-25]%[1D5H2-2]%[Dự án đề cương 3 Khối NH24-25-Dot1-Nguyễn Sĩ Đạt]
	Thời gian đi từ nhà đến trường của $56$ học sinh được cho trong bảng sau
	\begin{center}
		\begin{tabular}{|c|c|c|c|c|c|}
			\hline Thời gian (phút) & {$[9{,}5 ; 12{,}5)$} & {$[12{,}5 ; 15{,}5)$} & {$[15{,}5 ; 18{,}5)$} & {$[18{,}5 ; 21{,}5)$} & {$[21{,}5 ; 24{,}5)$} \\
			\hline Số học sinh & 3 & 12 & 15 & 24 & 2 \\
			\hline
		\end{tabular}
	\end{center}
	Tính trung vị của mẫu số liệu ghép nhóm này.
	\choice {$18{,}3$}
	{$18{,}2$}
	{$18$}
	{\True $18{,}1$}
	\loigiai{
		Ta có $\dfrac{56}{2}=28$ nên trung vị thuộc nhóm 	$[15{,}5 ; 18{,}5)$ nên $u_m=15{,}5$, $n_m=15$, $C=15$, $n=56$, $u_{m+1}-u_m=3$ nên
		$$M_e=u_m+\dfrac{\tfrac n2-C}{n_m}\cdot (u_{m+1}-u_m)=15{,}5+\dfrac{\tfrac {56}{2}-15}{15}\cdot (3)=18{,}1.$$
	}
\end{ex}

%\begin{ex}[Nguyễn Du - Bình Phước 24-25]%[1D5H2-2]%[Dự án đề cương 3 Khối NH24-25-Dot1-Nguyễn Sĩ Đạt]
%	Thời gian (phút) truy cập Internet mỗi buổi tối của một số học sinh được cho trong bảng sau
%	\begin{center}
%		\begin{tabular}{|l*{5}{|c}|}
%			\hline
%			Thời gian & $[9,5;12,5)$ & $[12,5;15,5)$ & $[15,5;18,5)$ & $[18,5;21,5)$ & $[21,5;24,5)$ \\
%			\hline
%			Số học sinh & $3$ & $12$ & $15$ & $24$ & $2$ \\
%			\hline
%		\end{tabular}
%	\end{center}
%	Tính trung vị của mẫu số liệu ghép nhóm này
%	\choice
%	{$18,3$}
%	{$18$}
%	{\True $18,1$}
%	{$18,2$}
%	\loigiai{
%		Ta có $M_e= a_p+\dfrac{\dfrac{n}{2}-\left(m_1+...+m_{p-1}\right)}{m_p}.\left(a_{p+1}-a_p\right)=15,5+\dfrac{\dfrac{56}{2}-(3+12)}{15}.3=18,1$.}
%\end{ex}

\begin{ex}[Lý Tự Trọng - Khánh Hòa 24-25]%[1D5H2-2]%[Dự án đề cương 3 Khối NH24-25-Dot1-Nguyễn Sĩ Đạt]
	Bảng sau thống kê số lượt chở khách mỗi ngày của một lái xe Taxi xanh SM (thuộc Tập đoàn Vingroup) trong 30 ngày
	\begin{center}
		\begin{tabular}{|l|c|c|c|c|c|}
			\hline
			Số lượt khách & $[4{,}5;7{,}5)$ & $[7{,}5;10{,}5)$ & $[10{,}5;13{,}5)$ & $[13{,}5;16{,}5)$ & $[16{,}5;19{,}5)$ \\
			\hline
			Số ngày & $7$ & $9$ & $7$ & $5$ & $2$\\
			\hline  
		\end{tabular}
	\end{center}
	Trung vị của mẫu số liệu trên gần với giá trị nào sau đây?
	\choice
	{\True $10{,}5$}
	{$12{,}1$}
	{$15{,}17$}
	{$9{,}8$}
	\loigiai{
		Ta có $n = 7 + 9 + 7 + 5 + 2 = 30$.\\
		Gọi $x_{1}, x_{2},\dots, x_{30}$ là số lượt chở khách của $30$ ngày của một lái xa Taxi xanh SM đã được sắp xếp theo thứ tự không giảm.\\
		Khi đó $x_{1},\dots, x_{7} \in [4{,}5;7{,}5)$; $x_{8},\dots, x_{16} \in [7{,}5;10{,}5)$; $x_{17},\dots, x_{23} \in [10{,}5;13{,}5)$; $x_{24},\dots, x_{28} \in [13{,}5;16{,}5)$; $x_{29},\dots, x_{30} \in [16{,}5;19{,}5)$.\\
		Do đó trung vị của số liệu trên là $M_e = \dfrac{1}{2}\left(x_{15} + x_{16}\right) \in [7{,}5;10{,}5)$.\\
		Với $C = 7$, ta có
		$$ M_e = u_m + \dfrac{n-C}{n_m}\cdot (u_{m+1} - u_{m}) = 7{,}5 + \dfrac{15-7}{9}\cdot (10{,}5 - 7{,}5) \approx 10{,}2. $$
	}
\end{ex}

\begin{ex}[Quảng Oai - Hà Nội 24-25]%[1D5H2-2]%[Dự án đề cương 3 Khối NH24-25-Dot1-Nguyễn Sĩ Đạt]
	Thời gian (phút) xem tivi mỗi buổi tối của một số học sinh được cho trong bảng sau
	\begin{center}
		\begin{tabular}{|c|c|c|c|c|c|c|c|}
			\hline
			{Thời gian}&{$[6{,}5;9{,}5]$} & {$[9{,}5;12{,}5]$} &{$[12{,}5;15{,}5]$}&{$[15{,}5;18{,}5]$} &{$[18{,}5;21{,}5]$}&{$[21{,}5;24{,}5]$}&{$[24{,}5;27{,}5]$}\\
			\hline
			{Số học sinh}&{$2$} & {$3$} &{$12$}&{$15$} &{$24$}&{$2$}&{$2$}\\
			\hline
		\end{tabular}
	\end{center}
	Số trung vị của mẫu số liệu ghép nhóm này là
	\choice
	{\True $18{,}1$}{$15{,}1$}{$21{,}1$}{$15$}
	\loigiai{
		Cỡ mẫu $n=60$.\\
		Nhóm chứa trung vị là $[15{,}5;18{,}5]$.\\
		$n_m=15{,} c=2+3+12=17$.\\
		Số trung vị của mẫu số liệu ghép nhóm này là
		$$M_e=u_m+\dfrac{\dfrac{n}{2}-c}{n_m}\cdot (u_{m+1}-u_m)=15{,}5+\dfrac{\dfrac{60}{2}-17}{15}\cdot (18{,}5-15{,}5)=18{,}1.$$
	}
\end{ex}

\begin{ex}[Trần Phú - Đà Nẵng 24-25]%[1D5H2-2]%[Dự án đề cương 3 Khối NH24-25-Dot1-Nguyễn Sĩ Đạt]
	Khảo sát chiều cao (đơn vị: cm) của $32$ bạn học sinh ta có bảng số liệu sau
	\begin{center}
		\begin{tabular}{|c|c|c|c|c|c|}
			\hline 
			Chiều cao (cm) & $\left[150 ; 155\right)$ & $\left[155 ; 160\right)$ & $\left[160 ; 165\right)$ & $\left[165 ; 170\right)$ & $\left[170 ; 175\right)$\\
			\hline Số học sinh & $3$ & $9$ & $12$ & $5$ & $3$ \\
			\hline
		\end{tabular}
	\end{center}
	Số trung vị của mẫu số liệu ghép nhóm trong bảng trên là (kết quả làm tròn đến hàng phần trăm)
	\choice
	{$167{,}22$}
	{$162{,}92$}
	{\True $161{,}67$}
	{$157{,}78$}
	\loigiai{
		Gọi $x_1$; $x_2$; $\ldots$; $x_{32}$ là chiều cao của $32$ học sinh theo thứ tự không giảm.\\
		Do $x_1,\,\ldots,\,x_3\in [150;155)$; $x_4,\,\ldots,\,x_9\in [155;160)$;  $x_{10},\,\ldots,\,x_{24}\in [160;165)$; $x_{25},\,\ldots,\,x_{29}\in [165;170)$ và $x_{30},\,\ldots,\,x_{32}\in [170;175)$ nên trung vị của mẫu số liệu $x_1$; $x_2$; $\ldots$; $x_{32}$ là $\dfrac{1}{2}\left(x_{16}+x_{17}\right)\in [160;165)$.\\
		Ta xác định được $n=32$; $n_m=12$; $C=3+9=12$; $u_m=160$; $u_{m+1}=165$.\\
		Vậy trung vị của mẫu số liệu ghép nhóm là
		$$M_e=u_m+\dfrac{\dfrac{n}{2}-C}{n_m}(u_{m+1}-u_m)=160+\dfrac{\dfrac{32}{2}-12}{12}(165-160)\approx161{,}67.$$
	}
\end{ex}

\begin{ex}[Hàng Hải - Hải Phòng 24-25]%[1D5H2-2]%[Dự án đề cương 3 Khối NH24-25-Dot1-Nguyễn Sĩ Đạt]
	Tìm hiểu thời gian chạy cự li $1000\mathrm{m}$ (đơn vị: giây) của các bạn học sinh trong một lớp 11 thu được kết quả sau
	\begin{center}
		\begin{tabular}{|c|c|c|c|c|c|}
			\hline
			Thời gian & $[125; 127)$ & $[127; 129)$ & $[129; 131)$ & $[131; 133)$ & $[133; 135)$ \\
			\hline
			Số bạn & 3 & 7 & 15 & 10 & 5 \\
			\hline
		\end{tabular}
	\end{center}
	Số trung vị của mẫu số liệu ghép nhóm ở trên là
	\choice
	{\True $M_e=\dfrac{391}{3}$}
	{$M_e=\dfrac{394}{3}$}
	{$M_e=\dfrac{392}{3}$}
	{$M_e=\dfrac{395}{3}$}
	\loigiai{Ta có
		\begin{itemize}
			\item cỡ mẫu $n=3+7+15+10+5=40$.
			\item Nhóm chứa trung vị $[129;131)$.
		\end{itemize}
		Do đó, trung vị của mẫu là
		\[M_e=129+\dfrac{20-10}{15}\cdot \left(131-129\right)=\dfrac{391}{3}.\]}
\end{ex}

\begin{ex}[Hàng Hải - Hải Phòng 24-25]%[1D5H2-3]%[Dự án đề cương 3 Khối NH24-25-Dot1-Nguyễn Sĩ Đạt]
	Thời gian (phút) truy cập Internet mỗi buổi tối của một số học sinh được cho trong bảng sau
	\begin{center}
		\begin{tabular}{|c|c|c|c|c|c|} 
			\hline Thời gian(giờ) & {$[9,5;12,5)$} & {$[12,5;15,5)$} & {$[15,5;18,5)$} & {$[18,5;21,5)$} & {$[21,5;24,5)$}\\ 
			\hline Số học sinh & $3$ & $12$ & $15$ & $24$ & $2$\\ 
			\hline 
		\end{tabular}
	\end{center}
	Tứ phân vị thứ nhất của mẫu số liệu ghép nhóm ở trên là 
	\choice
	{$Q_1=18{,}1$}
	{\True $Q_1=15{,}25$}
	{$Q_1=15{,}57$}
	{$Q_1=20$}
	\loigiai{
		Từ bảng số liệu 
		\begin{center}
			\begin{tabular}{|l|c|c|c|c|c|}
				\hline
				Dòng 1 & $[9{,}5; 12{,}5)$ & $[12{,}5; 15{,}5)$ & $[15{,}5; 18{,}5)$ & $[18{,}5; 21{,}5)$ & $[21{,}5; 24{,}5)$\\
				\hline
				Tần Số &$ 3$ & $12$ & $15$ & $24$ & $2$\\
				\hline
			\end{tabular}
		\end{center}
		Tứ phân vị thứ nhất $Q_1$.\\ 
		Cỡ mẫu là $n=56$.\\ 
		Gọi $x_1;x_2;\ldots;x_{56}$ là mẫu số liệu được sắp xếp theo thứ tự không giảm.\\ 
		Tứ phân vị thứ nhất  $Q_1$ là $\dfrac{x_{14}+x_{15}}{2}$.\\ 
		Do $\dfrac{x_{14}+x_{15}}{2}$ thuộc nhóm $[12{,}5;15{,}5)$ nên nhóm này chứa $Q_1$.\\ 
		Do đó $p=2; a_2=12{,}5; m_2=12; m_1=3; a_3-a_2=3{,}0$ và ta có $Q_1=12{,}5+\dfrac{\tfrac{56}{4}-3}{12}\cdot 3=15{,}25$.		
	}
\end{ex}

\begin{ex}[Lê Thánh Tôn - TPHCM 24-25]%[1D5H2-3]%[Dự án đề cương 3 Khối NH24-25-Dot1-Nguyễn Sĩ Đạt]
	Cân nặng (đơn vị: kg) của một số heo con mới sinh được cho trong bảng dưới đây
	\begin{center}
		\begin{tabular}{|l|c|c|c|c|}
			\hline Cân nặng $(\mathrm{kg})$ & $[1{,}0 ; 1{,}2)$ & $[1{,}2 ; 1{,}4)$ & $[1{,}4 ; 1{,}6)$ & $[1{,}6 ; 1{,}8)$ \\
			\hline Số con & $13$ & $14$ & $24$ & $15$ \\
			\hline
		\end{tabular}
	\end{center}
	Tứ phân vị thứ $3$ của mẫu số liệu bằng bao nhiêu? (Làm tròn đến hàng phần trăm)
	\choice 
	{\True $1{,}59$}
	{$1{,}60$}
	{$1{,}55$}
	{$1{,}49$}
	\loigiai{
		Cỡ mẫu $13+14+24+15=66$.\\
		Gọi $x_1$, $x_2$, $\ldots$, $x_{66}$ là cân nặng của $66$ heo con được sắp xếp theo thứ tự không giảm.\\
		$Q_3=x_{50}\in \left[1{,}4; 1{,}6\right)$.\\
		Do đó $u_3=1{,}4$; $n_3=24$; $n_1+n_2=13+14=27$; $u_4-u_3=1{,}6-1{,}4=0{,}2$.\\
		Khi đó \begin{eqnarray*}Q_3&=&u_3+\dfrac{\dfrac{3\cdot 66}{4}-(n_1+n_2)}{n_3}\left(u_4-u_3\right)\\
			&=&1{,}4+\dfrac{\dfrac{3\cdot 66}{4}-(27)}{24}\cdot 0{,}2\approx1{,}59.\end{eqnarray*}
	}
\end{ex}

\begin{ex}[Nguyễn Thượng Hiền - TPHCM 24-25]%[1D5H2-3]%[Dự án đề cương 3 Khối NH24-25-Dot1-Nguyễn Sĩ Đạt]
	Khảo sát thời gian (phút) tập thể dục trong ngày của một số học sinh khối $11$ thu được mẫu số liệu ghép nhóm sau
	\begin{center}
		\begin{tabular}{|c|c|c|c|c|c|}
			\hline
			Thời gian (phút) & {$[0; 20)$} & {$[20; 40)$} & {$[40; 60)$} & {$[60; 80)$} & {$[80; 100)$} \\
			\hline
			Số học sinh & $5$ & $9$ & $12$ & $10$ & $6$ \\
			\hline
		\end{tabular}	
	\end{center}
	Tứ phân vị thứ nhất $Q_1$ của mẫu số liệu ghép nhóm này bằng
	\choice
	{$\dfrac{500}{9}$}
	{$\dfrac{175}{6}$}
	{$\dfrac{710}{9}$}
	{\True $\dfrac{290}{9}$}
	\loigiai{
		Ta có bảng tần số tích lũy như sau
		\begin{center}
			\begin{tabular}{|c|c|c|c|c|c|}
				\hline
				Nhóm & {$[0; 20)$} & {$[20; 40)$} & {$[40; 60)$} & {$[60; 80)$} & {$[80; 100)$} \\
				\hline
				Tấn số & $5$ & $9$ & $12$ & $10$ & $6$ \\
				\hline
				Tần số tích lũy & $5$ & $14$ & $26$ & $36$ & $42$ \\
				\hline
			\end{tabular}	
		\end{center}
		Vì $n = 42$ nên $\dfrac{n}{4}= \dfrac{21}{2}$. Suy ra nhóm $[20;40)$ là nhóm đầu tiên có tần số tích lũy lớn hơn $\dfrac{n}{4}$.\\
		Vậy tứ phân vị thứ nhất của mẫu số liệu là
		\[ Q_1 = 20 + \dfrac{\dfrac{42}{4}-5}{9}\times (40-20) = \dfrac{290}{9}.\]
	}


\end{ex}

\begin{ex}[Châu Văn Liêm - Cần Thơ 24-25]%[1D5H2-3]%[Dự án đề cương 3 Khối NH24-25-Dot1-Nguyễn Sĩ Đạt]
	Một công ty bảo hiểm thống kê lại độ tuổi khách hàng mua bào hiểm xe ô tô ở bảng sau
	\begin{center}
		\begin{tabular}{|c|c|c|c|c|c|c|}
			\hline
			Độ tuổi & { $[25; 30)$ } & { $[30; 35)$ } & { $[35; 40)$ } & { $[40; 45)$ } & { $[45; 50)$ } & { $[50; 55)$ } \\
			\hline
			Số khách hàng & 25 & 38 & 62 & 42 & 37 & $29$ \\
			\hline		
		\end{tabular}
	\end{center}	
	Tứ phân vị thứ nhất của mẫu số liệu ghép nhóm trên (làm tròn kết quả đến hàng phần trăm) là
	\choice{$39{,}97$}
	{\True $34{,}38$}
	{$46{,}05$}
	{$39{,}31$}
	\loigiai{
		Ta có $\dfrac{n}{4}=\dfrac{233}{4}=58{,}25$ nên nhóm chứa tứ phân vị thứ nhất là nhóm $[30; 35)$.\\
		Khi đó $Q_1=30+\dfrac{58{,}25-25}{38}\cdot 5=34{,}375\approx 34{,}38$.
	}
\end{ex}

\begin{ex}[TPHCM 24-25]%[1D5V2-3]%[Dự án đề cương 3 Khối NH24-25-Dot1-Nguyễn Sĩ Đạt]
	Số người xem trong $60$ buổi chiếu phim của một rạp chiếu phim nhỏ được cho trong bảng sau
	\begin{center}
		\begin{tabular}{|c|c|c|c|c|c|c|}
			\hline
			Người xem & { $[0; 10)$ } & { $[10; 20)$ } & { $[20; 30)$ } & { $[30; 40)$ } & { $[40; 50)$ } & { $[50; 60)$ } \\
			\hline
			Tần số & $5$ & $9$ & $11$ & $15$ & $12$ & $8$ \\
			\hline
		\end{tabular}
	\end{center}
	Tìm tứ phân vị của mẫu số liệu ghép nhóm trên.
	\choice
	{$Q1=21; Q2=33; Q3=44$}
	{\True $Q1=20,9; Q2=33,3; Q3=44,2$}
	{$Q1=20,9; Q2=33; Q3=44,2$}
	{$Q1=21; Q2=33,3; Q3=44$}
	\loigiai{
		Gọi $x_1,x_2,\ldots, x_{60}$ là số người xem trong $60$ buổi chiếu phim của một rạp phim theo thứ tự không giảm.
		\\Từ bảng số liệu ta thấy $x_1,\ldots, x_5 \in [0;10); x_6,\ldots,x_{14} \in [10;20); x_{15},\ldots, x_{25} \in [20;30);$\\$x_{26},\ldots, x_{40} \in [30;40); x_{41},\ldots, x_{52} \in [40;50); x_{53},\ldots, x_{60} \in [50;60)$.
		\begin{itemize}
			\item Tứ phân vị thứ nhất là $\dfrac{1}{2} (x_{15}+x_{16})$ thuộc nhóm $[20;30)$ suy ra
			$$Q_1 = 20 + \dfrac{\dfrac{60}{4}-(5+9)}{11}\cdot 10 \approx 20{,}9.$$
			\item Tứ phân vị thứ hai là $\dfrac{1}{2} (x_{30}+x_{31})$ thuộc nhóm $[30;40)$
			suy ra
			$$Q_2 = 30 + \dfrac{\dfrac{60}{2}-(5+9+11)}{15}\cdot10 \approx 33{,}3.$$
			\item Tứ phân vị thứ ba là $\dfrac{1}{2} (x_{45}+x_{46})$ thuộc nhóm $[40;50)$ suy ra
			$$Q_3 = 40 + \dfrac{\dfrac{3\cdot60}{4}-(5+9+11+15)}{12}\cdot10 = 33{,}3 \approx 44{,}2.$$
		\end{itemize}
	}
\end{ex}

\begin{ex}[Hoàng Việt - Đăklăk 24-25]%[1D5V2-3]%[Dự án đề cương 3 Khối NH24-25-Dot1-Nguyễn Sĩ Đạt]
	Một bảng xếp hạng đã tính điểm chuẩn hóa cho chỉ số nghiên cứu của một số trường đại học ở Việt Nam và thu được kết quả như sau\\
	\centerline{
		\begin{tabular}{|c|c|c|c|c|c|c|}
			\hline
			\textbf{Điểm}	& Dưới $20$  & $[20;30)$  & $[30;40)$ & $[40;60)$ & $[60;80)$ & $[80;100)$ \\
			\hline
			\textbf{Số trường}	& $4$ & $19$ & $6$ & $2$ & $3$ & $1$ \\
			\hline
	\end{tabular}}\\
	Để đưa ra danh sách $25\%$ trường đại học có chỉ số nghiên cứu tốt nhất Việt Nam thì phải lấy các trường có điểm chuẩn hóa trên bao nhiêu?
	\choice
	{$35$}
	{\True $35{,}42$}
	{$35{,}4$}
	{$34$}
	\loigiai{
		Điểm ngưỡng để đưa ra danh sách $25\%$ trường đại học có chỉ số nghiên cứu tốt nhất Việt Nam là tứ phân vị thứ $3$.\\
		Ta có cỡ mẫu $n=35$ và $Q_3=30+\dfrac{\tfrac{3\cdot 35}{4} -23}{6}\cdot 10\approx 35{,}42$.\\
		Vậy để đưa ra danh sách $25\%$ trường đại học có chỉ số nghiên cứu tốt nhất Việt Nam thì phải lấy các trường có điểm chuẩn hóa trên $35{,}42$.
	}
\end{ex}

\begin{ex}[Nguyễn Khuyến - An Giang 24-25]%[1D5V2-3]%[Dự án đề cương 3 Khối NH24-25-Dot1-Nguyễn Sĩ Đạt]
	Cân nặng của một số lợn con mới sinh thuộc hai giống A và B được cho ở biểu đồ dưới đây (đơn vị: kg)
	\begin{center}
		\begin{tikzpicture}[>=stealth,line join=round,line cap=round,font=\footnotesize,scale=.7]
			\draw[->] (0,0) -- (0,4.5) node[left] {\scriptsize (Số con) };
			\draw[->] (0,0) -- (14,0) node[below] {\scriptsize (kg) };
			\fill[blue!50] 
			(0,0)rectangle(1.5,0.8)node[above left]{8}
			(3,0)rectangle(4.5,2.8)node[above left]{28}(6,0)rectangle(7.5,3.2)node[above left]{32}(9,0)rectangle(10.5,1.7)node[above left]{17};%Vẽ đường màu xanh
			\draw (0,4)--(0,0)--(12,0) (0,0)--(0,-2pt);
			\foreach \y in {0,1,...,4}{
				\pgfmathsetmacro\yy{int(\y*10)}
				\draw[shift={(0,\y)}] (-2pt,0)--(0,0)node[left=1.5mm]{$\yy$};}
			\draw (1.5,0)node[below]{$[1,0;1,1)$} (4.5,0)node[below]{$[1,1;1,2)$} (7.5,0)node[below]{$[1,2;1,3)$} (10.5,0)node[below]{$[1,3;1,4)$};
			\filldraw[draw=black,fill=yellow!50!red] 
			(1.5,0)rectangle(3,1.3)node[above left]{13}
			(4.5,0)rectangle(6,1.4)node[above left]{14}
			(7.5,0)rectangle(9,2.4)node[above left]{24}
			(10.5,0)rectangle(12,1.4)node[above left]{14};%vẽ đường màu cam
			\path (3,-2) node[draw,fill=blue!50,inner sep=2mm](a1){}node[right=2mm,text width=5cm](b1){Giống A} (8,-2) node[draw,fill=yellow!50!red,inner sep=2mm](a2){}node[right=2mm,text width=5cm](b2){Giống B};
		\end{tikzpicture}
	\end{center}
	Tổng của tứ phân vị thứ nhất và thứ ba của cân nặng lợn con mới sinh giống A và cân nặng của lợn con mới sinh giống B gần nhất với số nào trong các số dưới đây?
	\choice
	{$4{,}55$m}
	{$4{,}5$m}
	{$5{,}0$m}
	{\True $4{,}85$m}
	\loigiai{
		\begin{center}
			\begin{tabular}{|l|c|c|c|c|}
				\hline Cân nặng (kg) & {$[1,0 ; 1,1)$} & {$[1,1 ; 1,2)$} & {$[1,2 ; 1,3)$} & {$[1,3 ; 1,4)$} \\
				\hline Số con giống A & 8 & 28 & 32 & 17 \\
				\hline Số con giống B & 13 & 14 & 24 & 14 \\
				\hline
			\end{tabular}
		\end{center}
		* Con giống A:\\
		Cỡ mẫu là $8+28+32+17=85$ nên tứ phân vị thứ nhất là $\dfrac{x_{21}+x_{22}}{2}$.\\
		Do $x_{21} ; x_{22} \in[1,1 ; 1,2)$ nên tứ phân vị thứ nhất của mẫu số liệu là $$Q_{1 A}=1{,}1+\dfrac{\dfrac{85}{4}-8}{28} \cdot 0{,}1 \approx 1,15.$$	
		Tứ phân vị thứ ba là $\dfrac{x_{64}+x_{65}}{2}$.\\
		Do $x_{64} ; x_{65}$ thuộc $[1,2 ; 1,3)$ nên tứ phân vị thứ ba của mẫu số liệu là $$Q_{3 A}=1{,}2+\dfrac{3 \cdot \dfrac{85}{4}-36}{32} \cdot 0{,}1 \approx 1{,}29.$$
		* Con giống B:\\	
		Cỡ mẫu là $13+14+24+14=65$ nên tứ phân vị thứ nhất là $\dfrac{y_{16}+y_{17}}{2}$.\\
		Do $y_{16} ; y_{17} \in[1,1 ; 1,2)$ nên tứ phân vị thứ nhất của mẫu số liệu là $$Q_{1 B}=1{,}1+\dfrac{\dfrac{65}{4}-13}{14} \cdot 0{,}1 \approx 1{,}12.$$
		Tứ phân vị thứ ba là $\dfrac{y_{49}+y_{50}}{2}$.\\
		Do $y_{64} ; y_{65}$ thuộc $[1,2 ; 1,3)$ nên tứ phân vị thứ ba của mẫu số liệu là $$Q_{3 B}=1{,}2+\dfrac{3 \cdot \dfrac{65}{4}-27}{24} \cdot 0{,}1 \approx 1{,}29.$$
		Vậy $Q_{1 A}+Q_{3 A}+Q_{1 B}+Q_{3 B} \approx 1{,}15+1{,}29+1{,}12+1{,}29=4{,}85$.
	}
\end{ex}


\Closesolutionfile{ans}

\ind{PHẦN II.} \inden{Câu trắc nghiệm đúng sai. Trong mỗi ý a), b), c), d) ở mỗi câu, học sinh chọn đúng hoặc sai.}\\
\setcounter{ex}{0}
\Opensolutionfile{ans}[ans/1D5-Bai5-DS]
\begin{ex}[Chuyên Hùng Vương - Phú Thọ 24-25]%[1D5V2-3]%[Dự án đề cương 3 Khối NH24-25-Dot1-Nguyễn Sĩ Đạt]
	Cho mẫu số liệu ghép nhóm về Điểm khảo sát chất lượng lần $1$ môn Toán của một lớp $11$ trường THPT chuyên Hùng Vương năm học $2024-2025$ như sau
	\begin{center}
		\begin{tabular}{|c|c|c|c|c|c|c|}
			\hline
			Điểm  &$[4;5)$  &$[5;6)$&$[6;7)$  &$[7;8)$  &$[8;9)$  &$[9;10)$  \\
			\hline
			Số học sinh  &$1$  &$6$ &$11$ &$15$  &$9$  &$3$  \\
			\hline
		\end{tabular}
	\end{center}
	\choiceTF
	{Cỡ mẫu là $44$}
	{\True Mốt và số trung vị của mẫu số liệu thuộc cùng một nhóm}
	{Số trung bình của mẫu số liệu là $7{,}2$}
	{\True Có khoảng $50$\% học sinh trong lớp đạt từ $7{,}3$ điểm trở lên}
	\loigiai{
		\begin{center}
			\begin{tabular}{|c|c|c|c|c|c|c|}
				\hline
				Điểm  &$[4;5)$  &$[5;6)$&$[6;7)$  &$[7;8)$  &$[8;9)$  &$[9;10)$  \\
				\hline
				Giá trị đại diện &$4{,}5$  &$5{,}5$&$6{,}5$  &$7{,}5$  &$8{,}5$  &$9{,}5$  \\
				\hline
				Số học sinh  &$1$  &$6$ &$11$ &$15$  &$9$  &$3$  \\
				\hline
				Tần số tích lũy  &$1$  &$7$ &$18$ &$33$  &$42$  &$45$  \\
				\hline
			\end{tabular}
		\end{center}
		\begin{itemchoice}
			\itemch \textbf{Sai}.\\
			Cỡ mẫu là $n=1+6+11+15+9+3=45$.
			\itemch \textbf{Đúng}.\\
			Nhóm có tần số lớn nhất là $[7,8)$, suy ra mốt thuộc nhóm $[7;8)$.\\
			Vì $18<\dfrac{n}{2}=\dfrac{45}{2}<33$ nên 
			$M_e=6+\dfrac{\dfrac{45}{2}-18}{15}\cdot (7-6)=\dfrac{73}{10}=7{,}3$.\\
			Vậy mốt và số trung vị của mẫu số liệu thuộc cùng một nhóm $[7;8)$.
			\itemch \textbf{Sai}.\\
			Số trung bình $\overline{x}=\dfrac{1\cdot 4{,}5+6\cdot 5{,}5+11\cdot 6{,}5+15\cdot 7{,}5+9\cdot 8{,}5+3\cdot 9{,}5}{45}=\dfrac{653}{90}\approx 7{,}3$.
			\itemch \textbf{Đúng}.\\
			Vì $M_e=7{,}3$ nên có khoảng $50$\% học sinh trong lớp đạt từ $7{,}3$ điểm trở lên.
		\end{itemchoice}
	}
\end{ex}


\begin{ex}[Lê Quý Đôn - TPHCM 24-25]%[1D5V2-3]%[Dự án đề cương 3 Khối NH24-25-Dot1-Nguyễn Sĩ Đạt]
	Người ta đo đường kính của một số cây gỗ được trồng sau $ 12 $ năm (đơn vị: centimét), họ thu được bảng tần số ghép nhóm sau
	\begin{center}
		\begin{tabular}{|c|c|c|c|c|c|}
			\hline Đường kính (cm) &{$[20; 25)$} &{$[25; 30)$} &{$[30; 35)$} &{$[35; 40)$} &{$[40; 45)$} \\
			\hline Số cây & $ 4 $ & $ 12 $ & $ 26 $ & $ 13 $ & $ 6 $ \\
			\hline
		\end{tabular}
	\end{center}
	\choiceTF
	{\True Cỡ mẫu của mẫu số liệu là $n=61$}
	{\True Tứ phân vị thứ ba của mẫu số liệu ghép nhóm là $Q_3 \approx 36,44$}
	{\True Tứ phân vị thứ hai của mẫu số liệu ghép nhóm là $Q_2 \approx 32,79$}
	{Tứ phân vị thứ nhất của mẫu số liệu ghép nhóm là $Q_1=28,69$}
	\loigiai{
		\begin{itemchoice}
			\itemch \textbf{Đúng}.
			Ta có cỡ mẫu là $ n= 4 + 12 + 26 + 13 + 6 = 61$.
			\itemch \textbf{Đúng}.
			Do $ n=61 $, nên tứ phân vị thứ ba là $ \dfrac{1}{2}\left(x_{45}+x_{46}\right) $.\\
			Mà $ x_{45},x_{46}\in [35;40) $ nên ta có
			$$Q_3=35+\dfrac{\tfrac{3\cdot61}{4}-(4+12+26)}{13}\left(40-35\right)\approx 36{,}44.$$
			\itemch \textbf{Đúng}.
			Do $ n=61 $ nên trung vị là giá trị $x_{31}\in [30;35)$, do đó
			$$Q_2=30+\dfrac{\tfrac{61}{2}-(4+12)}{26}\left(40-35\right)\approx 32{,}79.$$
			\itemch \textbf{Sai}.
			Do $n=61$, nên tứ phân vị thứ nhất là $\dfrac{1}{2}\left(x_{15}+x_{16}\right)$.\\
			Mà $ x_{15}$, $x_{16}\in [25;30) $ nên ta có
			$$Q_1=25+\dfrac{\tfrac{61}{4}-4}{12}\left(30-25\right)\approx 29{,69}.$$
		\end{itemchoice}
	}
\end{ex}

\begin{ex}[Hoàng Việt - Đăklăk 24-25]%[1D5C2-3]%[Dự án đề cương 3 Khối NH24-25-Dot1-Nguyễn Sĩ Đạt]
	Một bảng xếp hạng đã tính điểm chuẩn hoá cho chỉ số nghiên cứu của một số trường đại học ở Việt Nam và thu được kết quả sau
	\begin{center}
		\begin{tabular}{|c|c|c|c|c|c|c|}
			\hline
			Điểm & $[10;20)$ & $[20;30)$ & $[30;40)$ & $[40;50)$ & $[50,60)$ & $[60,70)$\\
			\hline
			Số trường & $4$ & $19$ & $6$ & $2$ & $3$ & $1$\\
			\hline
		\end{tabular}
	\end{center}
	\choiceTF
	{\True Số liệu đã cho cho có $35$ mẫu số liệu}
	{Số trung bình của mẫu số liệu đã cho là $28$}
	{Số trung vị của mẫu số liệu là $M_{e}=12$}
	{\True Ngưỡng điểm để đưa ra danh sách $25\%$ trường đại học có chỉ số nghiên cứu tốt nhất Việt Nam là trên $35{,}42$}
	\loigiai{
		\begin{itemchoice}
			\itemch 
			Số các số liệu đã cho có mẫu số liệu là $n=4+19+6+2+3+1=35$.
			\itemch 
			Số trung bình cộng của mẫu số liệu đã cho là\\
			\[\overline{x}=\dfrac{4\cdot 15+19\cdot 25+6\cdot 35+2\cdot 45+3\cdot 55+1\cdot 65}{35}\approx30{,}43.\]
			\itemch 
			Vị trí của số trung vị là $\dfrac{35+1}{2}=18$.\\
			Do đó, giá trị trung vị của mẫu số liệu nằm trong khoảng $[20;30)$.
			\itemch 
			Gọi $x_1$, $x_2$, $\ldots$, $x_{35}$ là điểm chuẩn hóa cho chỉ số nghiên cứu của các trường đại học và giả sử dãy này đã sắp xếp theo thứ tự tăng dần. Khi đó, trung vị của mẫu số liệu là $M_e=x_{18}$ và tứ phân vị thứ ba $Q_3$ của mẫu số liệu là trung vị của nửa số liệu bên phải $M_e$, đó là dãy gồm 17 số liệu $x_{19}$, $x_{20}$, $\ldots$, $x_{35}$, do đó $Q_3=x_{27}$.\\
			Do $x_{27}$ thuộc nhóm $[30;40$ ) nên nhóm này chứa $Q_3$.\\
			Do đó, $p=3$; $a_3=30$; $m_3=6$; $m_1+m_2=4+19=23$; $a_4-a_3=40-30=10$ và ta có 		
			\[Q_3=30+\dfrac{\dfrac{3{,}35}{4}-23}{6} \cdot 10 \approx 35{,}42.\]	
			$25\%$ trường đại học có chỉ số nghiên cứu tốt nhất là $25\%$ giá trị nằm bên phải $Q_3$.\\
			Vậy điểm ngưỡng để đưa ra danh sách $25\%$ trường đại học có chỉ số nghiên cứu tốt nhất Việt Nam là những trường có điểm chuẩn hóa lớn hơn hoặc bằng $35{,}42$.
		\end{itemchoice}
	}
\end{ex}

\begin{ex}[Sở Bắc Giang 24-25]%[1D5V2-3]%[Dự án đề cương 3 Khối NH24-25-Dot1-Nguyễn Sĩ Đạt]
	Thống kê điểm giữa kì I môn Toán của 82 học sinh khối 11 tại một trường THPT được bảng số liệu ghép nhóm sau
	\begin{center}
		\begin{tabular}{|c|c|c|c|c|c|c|c|}
			\hline Điểm &{$[6,5 ; 7)$}&{$[7 ; 7,5)$}&{$[7,5 ; 8)$}&{$[8 ; 8,5)$}&{$[8,5 ; 9)$}&{$[9 ; 9,5)$}&{$[9,5 ; 10)$}\\
			\hline Số học $\sinh$ & 8 & 10 & 16 & 24 & 13 & 7 & 4 \\
			\hline
		\end{tabular}
	\end{center}	
	\choiceTF
	{Tứ phân vị thứ nhất của mẫu số liệu trên là $Q_1=7,8$}
	{Trung vị của mẫu số liệu trên thuộc nhóm $[8,5 ; 9)$}
	{\True Nhóm chứa mốt của mẫu số liệu trên là nhóm $[8 ; 8,5)$}
	{\True  Điểm trung bình giữa kì $I$ môn Toán của 82 học sinh trên nằm trong khoảng $(8 ; 8,5)$}
	\loigiai{
		Ta có bảng sau
		\begin{center}
			\begin{tabular}{|c|c|c|c|c|c|c|c|}
				\hline Điểm & {[6,5;7,0)} & {[7,0;7,5)} & {[7,5;8,0)} & {[8,0;8,5)} & {[8,5;9,0)} & {[9,0;9,5)} & {[9,5;10)} \\
				\hline Giá trị đại diện & 6,75 & 7,25 & 7,75 & 8,25 & 8,75 & 9,25 & 9,75 \\
				\hline Tần số & 8 & 10 & 16 & 24 & 13 & 7 & 4 \\
				\hline Tần số tích lũy & 8 & 18 & 34 & 58 & 71 & 78 & 82\\
				\hline
			\end{tabular}
		\end{center}
		Ta có cỡ mẫu $ n=82 $, suy ra $\dfrac{n}{4} =20,5$, $\dfrac{n}{2} =41$, $\dfrac{3n}{4}= 61,5$.\\
		\begin{itemchoice}
			\itemch Do đó, nhóm đầu tiên có tần số tích lũy  lớn hơn hoặc bằng $20,5$ là nhóm $[7,5;8,0).$ \\
			Suy ra tứ phân vị thứ nhất là $Q_1 =7,5+\dfrac{20,5-18}{16}\cdot 0{,}5=7,578125$. 
			\itemch Nhóm đầu tiên có tần số tích lũy lớn hơn hoặc bằng $41$ là nhóm $[8{,}0;8{,}5) $.
			Do đó nhóm chứa trung vị của mẫu số liệu trên thuộc nhóm $[8{,}0;8{,}5] $.
			\itemch Nhóm $[8{,}0;8{,5})$ là nhóm có tần số lớn nhất nên là nhóm chưa mốt của mẫu số liệu.
			\itemch Điểm trung bình giữa học kì $I$  môn toán của $82$ học sinh trong mẫu số liệu là 
			$$ \overline{x}=\dfrac{8\cdot6,75+10\cdot7,25+16\cdot7,75+24\cdot8,25+13\cdot8,75+7\cdot9,25+4\cdot9,75}{82}\approx 8,12. $$
		\end{itemchoice}
	}
\end{ex}

\begin{ex}[Nguyễn Khuyến - An Giang 24-25]%[1D5V2-3]%[Dự án đề cương 3 Khối NH24-25-Dot1-Nguyễn Sĩ Đạt]
	Cho mẫu số liệu về cân nặng (đơn vị: kg) của các em học sinh trong lớp $10$A đã ghép nhóm dưới dạng bảng tần số như sau
	\begin{center}
		\begin{tabular}{|l|c|c|c|c|c|c|}
			\hline
			Nhóm & $[30; 40)$ & $[40; 50)$ & $[50; 60)$ & $[60; 70)$ & $[70; 80)$ & $[80; 90)$ \\
			\hline
			Tần số & $2$ & $10$ & $16$ & $8$ & $2$ & $2$ \\
			\hline
		\end{tabular}
	\end{center}
	\choiceTF
	{Cỡ của mẫu số liệu là $n=42$}
	{\True Số trung bình của mẫu số liệu trên là $56$}
	{\True Trung vị của mẫu số liệu đã cho bằng $55$}
	{Hiệu của tứ phân vị thứ ba và thứ nhất là $Q_3-Q_1=14$}
	\loigiai{
		Ta có, mẫu số liệu ghép nhóm bao gồm cả tần số tích lũy như sau
		\begin{center}
			\begin{tabular}{|l|c|c|c|c|c|c|}
				\hline
				Nhóm & $[30; 40)$ & $[40; 50)$ & $[50; 60)$ & $[60; 70)$ & $[70; 80)$ & $[80; 90)$ \\
				\hline
				Giá trị đại diện & $35$ & $45$ & $55$ & $65$ & $75$ & $85$\\
				\hline
				Tần số & $2$ & $10$ & $16$ & $8$ & $2$ & $2$ \\
				\hline
				Tần số tích lũy & $2$ & $12$ & $28$ & $36$ & $38$ & $40$\\
				\hline
			\end{tabular}
		\end{center}
		\begin{itemchoice}
			\itemch \textbf{Sai}. Cỡ của mẫu số liệu là $n=2+10+16+8+2+2=40$.
			\itemch \textbf{Đúng}. Số trung bình của mẫu số liệu là
			$$\overline{x}=\dfrac{2\cdot 35+10\cdot 45+16\cdot 55+8\cdot 65+2\cdot 75+2\cdot 85}{40} = 56.$$
			\itemch \textbf{Đúng}. Ta có $\dfrac{n}{2}=20$.\\
			Nhóm đầu tiên có tần số tích lũy lớn hơn hoặc bằng $20$ là nhóm $[50;60)$.\\
			Do đó, trung vị của mẫu số liệu đã cho là
			$M_e=50+\dfrac{20-12}{16}\cdot (60-50)=55.$
			\itemch \textbf{Sai}. Ta có $\dfrac{n}{4}=10$.\\
			Nhóm đầu tiên có tần số tích lũy lớn hơn hoặc bằng $10$ là nhóm $[40;50)$.\\
			Do đó, tứ phân vị thứ nhất của mẫu số liệu đã cho là
			$Q_1=40+\dfrac{10-2}{10}\cdot (50-40)=48.$\\
			Ta có $3\cdot \dfrac{n}{4}=30$.\\
			Nhóm đầu tiên có tần số tích lũy lớn hơn hoặc bằng $30$ là nhóm $[60;70)$.\\
			Do đó, tứ phân vị thứ ba của mẫu số liệu đã cho là
			$Q_3=60+\dfrac{30-28}{8}\cdot (70-60)=62{,}5.$\\
			Vậy $Q_3-Q_1=62{,}5-48=14{,}5$.
		\end{itemchoice}
	}
\end{ex}

\Closesolutionfile{ans}

\ind{PHẦN III.} \inden{Trả lời ngắn}\\
\setcounter{ex}{0}
\Opensolutionfile{ans}[ans/1D5-Bai5-TLN]
\begin{ex}[Chuyên Trần Phú - Hải Phòng 24-25]	%[1D5H2-2]%[Dự án đề cương 3 Khối NH24-25-Dot1-Nguyễn Sĩ Đạt]
	Khối lượng của $30$ củ khoai lang thu hoạch ở một hộ gia đình được trình bày trong bảng ghép nhóm sau
	\begin{center}
		\begin{tabular}{|l|c|c|c|c|c|} 
			\hline Khối lượng (gam) & { $[70 ; 80)$ } & { $[80 ; 90)$ } & { $[90 ; 100)$ } & { $[100 ; 110)$ } & { $[110 ; 120)$ } \\ 
			\hline Số củ khoai & $2$ & $7$ & $12$ & $6$ & $3$ \\ 
			\hline 
		\end{tabular}
	\end{center}
	Tính trung vị của mẫu số liệu ghép nhóm trên.
	\shortans{$95$}
	\loigiai{
		Gọi $x_1,x_2,\hdots,x_{30}$ lần lượt là khối lượng của các củ khoai được sắp xếp theo thứ tự không giảm.\\
		Do $x_1,x_2 \in [70;80)$; $x_3,\hdots,x_9 \in [80;90)$; $x_{10},\hdots,x_{21} \in [90;100)$; $x_{22},\hdots,x_{27} \in [100;110)$ và $x_{28},x_{29},x_{30} \in [110;120)$ nên trung vị của mẫu số liệu $x_1,x_2,\hdots,x_{30}$ là
		$$
		\dfrac{1}{2} \left(x_{15}+x_{16}\right) \in [90;100).
		$$
		Ta xác định được
		$$
		n = 30, \quad
		n_m = 12, \quad
		C = 2+7 = 9, \quad
		u_m = 90, \quad
		u_{m+1} = 100.
		$$
		Vậy trung vị của mẫu số liệu ghép nhóm là
		$$
		M_e = 90 + \dfrac{\dfrac{30}{2}-9}{12} \cdot (100-90)
		= 95.
		$$
	}
\end{ex}

\begin{ex}[Hoàng Việt - Đăk Lăk 24-25] %[1D5H2-2]%[Dự án đề cương 3 Khối NH24-25-Dot1-Nguyễn Sĩ Đạt]
	Kết quả thu thập điểm số môn Toán của $25$ học sinh khi tham gia kì thi học sinh giỏi
	Toán $11$ (thang điểm $20$) cho ta bảng tần số ghép nhóm sau
	\begin{center}
		\begin{tabular}{|c|c|c|c|c|c|}
			\hline
			Nhóm & $[0;4)$ & $[4;8)$ & $[8;12)$ & $[12;16)$ & $[16;20)$ \\
			\hline
			Số học sinh  & $1$ & $7$ & $12$ & $3$ & $2$ \\
			\hline
		\end{tabular}
	\end{center}
	Tìm trung vị của mẫu số liệu ghép nhóm trên.	
	\shortans{$9{,}5$}
	\loigiai{
		Cỡ mẫu của mẫu số liệu là $n=25$.\\
		Gọi $x_1$, $x_2$, $x_3$, $\ldots$, $x_{25}$ là điểm số của $25$ học sinh trong kì thi đó và dãy này được sắp xếp theo thứ tự không giảm.\\
		Trung vị của mẫu số liệu là $x_{13}\in[8;12)$.\\
		Ta có $n=25$, $n_m=12$, $C=1+7=8$, $u_m=8$, $u_{m+1}=12$.\\
		Trung vị mẫu số liệu ghép nhóm là
		$$M_e=u_m+\dfrac{\dfrac{n}{2}-C}{n_m}\left(u_{m+1}-u_m\right)=8+\dfrac{\dfrac{25}{2}-8}{12}(12-8)=9{,}5.$$}
\end{ex}

\begin{ex}%[1D5H2-2]%[Dự án đề cương 3 Khối NH24-25-Dot1-Nguyễn Sĩ Đạt]
	Trong tuần lễ bảo vệ môi trường, các học sinh khối $ 11 $ tiến hành thu nhặt vỏ chai nhựa để tái chế. Nhà trường thống kê kết quả thu nhặt vỏ chai của học sinh khối $ 11 $ ở bảng sau
	\begin{center}
		\begin{tabular}{|c|c|c|c|c|c|}
			\hline 
			\textbf{Số vỏ chai nhựa}	& $ \left[ 11 ; 15\right]  $ & $ \left[ 16 ; 29\right]  $ & $ \left[21 ; 25 \right]  $ & $ \left[ 26 ; 30\right]  $ & $ \left[31 ; 35 \right]  $ \\ 
			\hline 
			\textbf{Số học sinh}	& $ 53 $ & $ 82 $ & $ 48 $ & $ 39 $ & $ 18 $ \\ 
			\hline 
		\end{tabular} 
	\end{center}
	Hãy tìm trung vị của mẫu số liệu ghép nhóm trên.
	\shortans{$19{,}6$}
	\loigiai{
		Do số vỏ chai là số nguyên nên ta hiệu chỉnh lại như sau
		\begin{center}
			\begin{tabular}{|c|c|c|c|c|c|}
				\hline 
				\textbf{Số vỏ chai nhựa}	& $ \left[ 10{,}5 ; 15{,}5\right) $ & $ \left[ 15{,}5 ; 20{,}5\right) $ & $ \left[ 20{,}5 ; 25{,}5\right) $ & $ \left[ 25{,}5 ; 30{,}5\right) $ & $ \left[30{,}5 ; 35{,}5 \right)  $ \\ 
				\hline 
				\textbf{Số học sinh}& $ 53 $ & $ 82 $ & $ 48 $ & $ 39 $ & $ 18 $ \\ 
				\hline 
			\end{tabular} 
		\end{center}
		Số học sinh tham gia thu nhặt vỏ chai nhựa là $$ n=53+82+48+39+18=240.$$
		Gọi $ x_1; x_2; \ldots ; x_{240} $ lần lượt là số vỏ chai $ 240 $ học sinh khối $ 11 $ thu nhặt được xếp theo thứ tự không giảm.\\
		Do $ x_1, \ldots, x_{53}\in \left[10{,}5 ; 15{,}5 \right) $; $ x_{54}, \ldots, x_{135}\in \left[ 15{,}5 ; 20{,}5\right)$ nên trung vị của mẫu số liệu $ x_1; x_2; \ldots;x_{240} $ là $$ \dfrac{1}{2}\left( x_{120}+x_{121}\right)\in \left[ 15{,}5 ; 20{,}5\right).$$
		Ta xác định được $ n=240$; $ n_m=82 $; $ C=53 $; $ u_m=15{,}5 $; $ u_{m+1}=20{,}5 $.\\
		Trung vị của mẫu số liệu ghép nhóm là $$ M_e=15{,}5+\dfrac{\dfrac{240}{2}-53}{82}\cdot \left( 20{,}5-15{,}5\right)=\dfrac{803}{41}\approx 19{,}6. $$
	}
\end{ex}

\begin{ex}[Ngô Gia Tự - Đắk Lắk 24-25]%[1D5V2-3]%[Dự án đề cương 3 Khối NH24-25-Dot1-Nguyễn Sĩ Đạt]
	Thầy giáo thống kê lại số lần kéo xà đơn của các học sinh nam khối 11 ở bảng sau
	\begin{center}
		\begin{tabular}{|c|c|c|c|c|c|}
			\hline
			Số lần & [6;10] & [11;15] & [16;20] & [21;25] & [26;30] \\
			\hline
			Số học sinh & 35 & 54 & 32 & 17 & 5 \\
			\hline
		\end{tabular}
	\end{center}
	Thầy giáo dự định chọn $25$\% học sinh có thành tích kéo co thấp nhất để bồi dưỡng thể lực thêm. Thầy giáo nên chọn học sinh có thành tích kéo xà đơn dưới bao nhiêu lần để bồi dưỡng thể lực?
	
	\shortans{$12$}
	\loigiai{
		Thầy giáo dự định chọn $25$\% học sinh có thành tích kéo co thấp nhất để bồi dưỡng thể lực thêm nên ta tìm tứ phân vị thứ nhất.\\
		Vì $n = 143$ nên $\dfrac{n}{4}= 37{,}75$. Suy ra nhóm $[11;15)$ là nhóm đầu tiên có tần số tích lũy lớn hơn $\dfrac{n}{4}$.\\
		Tứ phân vị thứ nhất của mẫu số liệu là
		\[ Q_1 = 11 + \dfrac{\dfrac{143}{4}-35}{54}\cdot (15-11) = \dfrac{290}{9}\approx 11{,}06.\]
		Vậy thầy giáo chọn học sinh có thành tích kéo co thấp hơn $12$ để bồi dưỡng thể lực thêm.
	}
\end{ex}

\begin{ex}[DTNT - Phú Yên 24-25]%[1D5H2-3]%[Dự án đề cương 3 Khối NH24-25-Dot1-Nguyễn Sĩ Đạt]
	Cho mẫu số liệu ghép nhóm về thống kê điểm số (thang điểm $10$) của $50$ học sinh tham dự kỳ thi giữa kỳ I của lớp $11$A, ta có bảng số liệu sau
	\begin{center}
		\begin{tabular}{|c|c|c|c|c|c|}
			\hline
			Điểm & {$[0; 2)$} & {$[2; 4)$} & {$[4; 6)$} & {$[6; 8)$} & {$[8; 10)$} \\
			\hline
			Số học sinh & 5 & 7 & 13 & 18 & 7 \\
			\hline
		\end{tabular}
	\end{center}
	Tìm tứ phân vị thứ nhất của mẫu số liệu ghép nhóm trên. (Kết quả làm tròn đến hàng phần trăm).
	
	\shortans{$4{,}04$}
	\loigiai{
		Gọi $x_1, x_2, \ldots, x_{50}$ là điểm số (thang điểm $10$) của $50$ học sinh tham dự kỳ thi giữa kỳ I của lớp $11$A.\\
		Khi đó tứ phân vị thứ nhất $Q_1$ là trung vị của dãy $x_1$, $x_2$,$\ldots$, $x_{25}$. \\
		Do đó $Q_1$ thuộc nhóm $[4;6)$. và \begin{align*}
			Q_1=4+\dfrac{\dfrac{50}{4}-12}{13} \cdot (6-4)=4{,}04.
		\end{align*}
	}
\end{ex}


\Closesolutionfile{ans}


\ind{PHẦN IV.} \inden{Tự luận}\\
\setcounter{ex}{0}
\begin{ex}%[1D1B2-2]
	Một trang báo điện tử thống kê thời gian người sử dụng đọc thông tin trên trang trong mối lần truy cập ở bảng sau:
	\begin{center}
		\begin{tabular}{|l|c|c|c|c|c|}
			\hline Thời gian đọc (phút) & {$[0 ; 2)$} & {$[2 ; 4)$} & {$[4 ; 6)$} & {$[6 ; 8)$} & {$[8 ; 10)$} \\
			\hline Số lượt truy cập & 45 & 34 & 23 & 18 & 5 \\
			\hline
		\end{tabular}
	\end{center}
	Hãy ước lượng các tứ phân vị của mẫu sô liệu ghép nhóm trên.
	\loigiai{
		$Q_1=\dfrac{25}{18}$; $Q_2=\dfrac{103}{34}$; $Q_3=\dfrac{243}{46}$.
	}	
\end{ex}


%---Bài 3
\begin{ex}%[1D1B2-2]
	Thâm niên công tác của các công nhân hai nhà máy $A$ và $B$.
	\begin{center}
		\begin{tabular}{|l|c|c|c|c|c|}
			\hline Thâm niên công tác (năm) & {$[0 ; 5)$} & {$[5 ; 10)$} & {$[10 ; 15)$} & {$[15 ; 20)$} & {$[20 ; 25)$} \\
			\hline Số công nhân nhà máy $A$ & 35 & 13 & 12 & 12 & 8 \\
			\hline Số công nhân nhà máy $B$ & 14 & 26 & 24 & 11 & 5 \\
			\hline
		\end{tabular}
	\end{center}
	\begin{listEX}
		\item Hãy so sánh thâm niên công tác của nhân viên hai nhà máy theo số trung bình và trung vị.
		\item Hãy ước lượng tứ phân vị thứ nhất và thứ ba của hai mẫu số liệu ghép nhóm trên.
	\end{listEX}
	\loigiai{
		\begin{listEX}
			\item So sánh theo số trung bình: $\bar{x}_A=9{,}0625$; $\bar{x}_B=10{,}4375$, suy ra $\bar{x}_A<\bar{x}_B$.\\
			So sánh theo trung vị: $M_e(A)=\dfrac{90}{13}$; $M_e(B)=10$, $M_e(A)<M_e(B)$.
			\item $Q_1(A)=\dfrac{20}{7}$, $Q_3(A)=15$; $Q_1(B)=\dfrac{80}{13}$, $Q_3(B)=\dfrac{85}{6}$.
		\end{listEX}
	}	
\end{ex}

%---Bài 4
\begin{ex}%[1D1B2-2]
	Thầy giáo thống kê lại số lần kéo xà đơn của các học sinh nam khối 11 ở bảng sau
	\begin{center}
		\begin{tabular}{|l|c|c|c|c|c|}
			\hline Số lần & {$[6 ; 10]$} & {$[11 ; 15]$} & {$[16 ; 20]$} & {$[21 ; 25]$} & {$[26 ; 30]$} \\
			\hline Số học sinh & 35 & 54 & 32 & 17 & 5 \\
			\hline
		\end{tabular}
	\end{center}
	\begin{listEX}
		\item Hãy ước lượng số trung bình, mốt và trung vị của mẫu số liệu ghép nhóm trên.
		\item Thầy giáo dự định chọn $25 \%$ học sinh co số lần kéo thấp nhất để bồi dưỡng thế lực thềm. Thầy giáo nên chọn học sinh có thành tích kéo xà đơn dưới bao nhiêu lần để bồi dưỡng thể lực?
	\end{listEX}
	\loigiai{
		\begin{listEX}
			\item $\bar{x}=\dfrac{2089}{143}$; $M_0=\dfrac{1051}{82}$; $M_e=\dfrac{1499}{108}$.
			\item $Q_1=\dfrac{797}{72} \approx 11{,}07$. Thầy giáo nên chọn các bạn có thành tích kéo xà dưới 12 lần để bồi dưỡng thể lực thêm.
		\end{listEX}
	}	
\end{ex}



\begin{ex}%[1D1B1-2]%[1D1Y1-1]%[1D1B1-3]%[1D1B2-1]%[1D1B2-2]
	\immini{
		Cho mẫu số liệu ghép nhóm thống kê thời gian sử dụng điện thoại trước khi ngủ (đơn vị  phút) của một người trong 120 ngày như bảng bên. Xác định các số đặc trưng đo xu thế trung tâm của mẫu số liệu đó (làm tròn đến kết quả hàng phần mười).}
	{\begin{tabular}{|c|c|}
			\hline
			Nhóm & Tần số\\
			\hline
			$[0;4)$&13\\
			\hline
			$\left[4;8\right)$&29\\
			\hline
			$\left[8;12\right)$&48\\
			\hline
			$\left[12;16\right)$&22\\
			\hline
			$\left[16;20\right)$&8\\
			\hline
			&$n=120$\\
			\hline
	\end{tabular}}
	\loigiai{Số trung bình cộng là $$\overline{x}=\dfrac{13\cdot2+29\cdot6+48\cdot10+22\cdot14+8\cdot18}{120}\approx 9,4.$$
		Bảng tần số ghép nhóm bao gồm cả tần số tích lũy được cho như bảng sau:
		\begin{center}
			\begin{tabular}{|c|c|c|}\hline
				Nhóm&Tần số&Tần số tích lũy\\
				\hline$[0;4)$&13&13\\
				\hline$[4;8)$&29&42\\
				\hline$[8;12)$&48&90\\
				\hline$[12;16)$&22&112\\
				\hline$[16;20)$&$8$&$120$\\
				\hline&$n=120$&\\
				\hline
			\end{tabular}
		\end{center}
		Ta có  $\dfrac{n}{2}=60,\,\dfrac{n}{4},\,\dfrac{3n}{4}=90$. Vì $42<60<90$ nên nhóm 3 là nhóm đầu tiên có tần số tích lũy lớn hơn hoặc bằng 60. Suy ra trung vị là $$M_e=8+\left(\dfrac{60-42}{48}\right)\cdot4=9{,}5.$$
			Tứ phân vị thứ hai là  $Q_2=M_e=9{,}5.$
			Vì $13<30<42$ nên nhóm 2 là nhóm đầu tiên có tần số tích lũy lớn hơn hoặc bằng 30. Suy ra tứ phân vị thứ nhất là $$Q_1=4+\left(\dfrac{30-13}{29}\right)\cdot4\approx 6{,}3.$$
		Vì $42<90\leq 90$ nên nhóm 3 là nhóm đầu tiên có tần số tích lũy lớn hơn hoặc bằng 90. Suy ra tứ phân vị thứ 3 là  $$Q_3=8+\left(\dfrac{90-42}{48}\right)\cdot4\approx 12.$$
		Trong các nhóm, nhóm 3 có tần số lớn nhất. Suy ra mốt là $$M_o=8+\left(\dfrac{48-29}{2\cdot48-29-22}\right)\cdot4\approx 9{,}7.$$}
\end{ex}

\begin{ex}%[1T5K2-2]
	Kiểm tra điện lượng của một số viên pin tiểu do một hãng sản xuất thu được kết quả như sau:
	\begin{center}
		\begin{tabular}{|c|c|c|c|c|c|}
			\hline 
			\begin{tabular}{c}
				\textbf{Điện lượng} \\	\textbf{(nghìn mAh)}
			\end{tabular} 
			& $ \left[ 0{,}9 ; 0{,}95\right)  $ & $ \left[ 0{,}95 ; 1{,}0\right)  $ & $ \left[ 1{,}0 ; 1{,}05\right)  $ &$ \left[ 1{,}05 ; 1{,}1\right)  $  &  $ \left[ 1{,}1 ; 1{,}15\right)  $\\ 
			\hline 
			\textbf{Số viên pin}& $ 10 $ & $ 20 $ & $ 35 $ & $ 15 $ & $ 5 $ \\ 
			\hline 
		\end{tabular} 
	\end{center}
	Hãy ước lượng số trung bình, mốt và tứ phân vị của mẫu số liệu ghép nhóm trên.
	\loigiai{
		\begin{itemize}
			\item Tìm số trung bình của mẫu số liệu ghép nhóm.\\
			Ta có bảng thống kê điện lượng của pin theo giá trị đại diện là:
			\begin{center}
				\begin{tabular}{|c|c|c|c|c|c|}
					\hline 
					\begin{tabular}{c}
						\textbf{Điện lượng} \\	\textbf{(nghìn mAh)}
					\end{tabular} 
					& $ \left[ 0{,}9 ; 0{,}95\right)  $ & $ \left[ 0{,}95 ; 1{,}0\right)  $ & $ \left[ 1{,}0 ; 1{,}05\right)  $ &$ \left[ 1{,}05 ; 1{,}1\right)  $  &  $ \left[ 1{,}1 ; 1{,}15\right)  $\\ 
					\hline 
					\textbf{Giá trị đại diện}& $ 0{,}925 $ & $ 0{,}975 $ & $ 1{,}025 $ & $ 1{,}075 $ & $ 1{,}125 $ \\ 
					\hline
					\textbf{Số viên pin}& $ 10 $ & $ 20 $ & $ 35 $ & $ 15 $ & $ 5 $ \\ 
					\hline 
				\end{tabular} 
			\end{center}
			Số trung bình của mẫu số liệu ghép nhóm theo dõi điện lượng của một số viên pin xấp xỉ bằng: $$ \dfrac{0{,}925\cdot 10 + 0{,}975\cdot 20 +1{,}025 \cdot 35 +1{,}075 \cdot 15+1{,}125 \cdot 5}{10+20+35+15+5}\approx 1{,}016. $$
			\item Tìm mốt của mẫu số liệu ghép nhóm.\\
			Nhóm chứa mốt của mẫu số liệu là $ \left[ 1{,}0 ; 1{,}05\right) $.\\
			Do đó $ u_m=1$; $ n_{m-1}=20 $; $ n_m=35 $; $ n_{m+1}=15 $; $ u_{m+1}-u_m=1{,}05-1{,}0=0{,}05 $.\\
			Mốt của mẫu số liệu ghép nhóm là $$M_0=1+\dfrac{35-20}{(35-20)+(35-15)}\cdot 0{,}05 =\dfrac{143}{140}\approx 1{,}021.$$  
			\item Tìm các tứ phân vị của mẫu số liệu ghép nhóm.\\
			Gọi $ x_1 $, $ x_2 $, $ \ldots $, $ x_{85} $ là mẫu số liệu được xếp theo thứ tự không giảm.\\
			Ta có $ x_1, \ldots, x_{10}\in  \left[ 0{,}9 ; 0{,}95\right)$; $ x_{11}, \ldots, x_{30}\in  \left[ 0{,}95 ; 1{,}0\right) $; $ x_{31}, \ldots, x_{65}\in   \left[ 1{,}0 ; 1{,}05\right) $; $ x_{66}, \ldots, x_{80}\in \left[ 1{,}05 ; 1{,}1\right) $; $ x_{81}, \ldots, x_{85}\in \left[ 1{,}1 ; 1{,}15\right)$.\\
			Tứ phân vị thứ nhất của mẫu số liệu là $ x_{22} \in  \left[ 0{,}95 ; 1{,}0\right) $ nên tứ phân vị thứ nhất của mẫu số liệu ghép nhóm là 
			$$Q_1=0{,}95+\dfrac{\dfrac{85}{4}-10}{20}\cdot \left( 1{,}0-0{,}95\right)=  \dfrac{313}{320}\approx 0{,}978.$$
			Tứ phân vị thứ hai của mẫu số liệu là $ x_{43} \in  \left[ 1{,}0 ; 1{,}05\right) $ nên tứ phân vị thứ hai của mẫu số liệu ghép nhóm là 
			$$Q_2=1{,}0+\dfrac{\dfrac{85}{2}-(10+20)}{35}\cdot \left( 1{,}05-1{,}0\right)=\dfrac{57}{56} \approx 1{,}018.$$
			Tứ phân vị thứ ba của mẫu số liệu là $ x_{64} \in  \left[ 1{,}0 ; 1{,}05\right) $ nên tứ phân vị thứ ba của mẫu số liệu ghép nhóm là 
			$$Q_3=1{,}0+\dfrac{\dfrac{85\cdot 3}{4}-(10+20)}{35}\cdot \left( 1{,}05-1{,}0\right)=\dfrac{587}{560} \approx 1{,}048.$$
		\end{itemize}
	}
\end{ex}
\begin{ex}%[1T5G2-2]
	Cân nặng của một số lợn con mới sinh thuộc hai giống $ A $ và $ B $ được cho ở biểu đồ dưới đây (đơn vị: kg).
	\begin{center}
		\begin{tikzpicture}[>=stealth,line join=round,line cap=round,font=\footnotesize,scale=0.85,line width=1pt]
			\draw[->] (0,0)--(0,5)node[left]{(\text{Số con})};
			\foreach \y in {1,2,3,4}
			\draw[shift={(0,\y)}] (0,0)--(-2pt,0) node[left]{\scriptsize ${\y}0$};
			%	\path (4.5,6) node {\normalsize{\textbf{Cân nặng của một số lợn con mới sinh}}};
			\path (4.5,5.5) node {
				$\begin{array}{c}
					\normalsize{\textbf{Cân nặng của một số}}\\
					\normalsize{\textbf{lợn con mới sinh}}
				\end{array}$
			};
			%% nhãn
			\path (2.5,-1.5) node[rectangle,fill=cyan,draw=none]{};
			\path (3.6,-1.5) node {\text{Giống $ A $}};
			\path (5,-1.5) node[rectangle,fill=orange,draw=none]{};
			\path (6.1,-1.5) node {\text{Giống $ B $}};
			% đường gióng
			\foreach \y in {1,2,3,4}{
				\draw[line width=0.2pt] (0,\y)--(8.4,\y);
			}
			%% cột
			\draw[fill=cyan,draw=none] (0,0)--(0,0.8)--(1,0.8)node[midway,above]{$ 8 $}--(1,0)--cycle;
			\draw[fill=orange,draw=none] (1,0)--(1,1.3)--(2,1.3)node[midway,above]{$ 13 $}--(2,0)--cycle;
			\draw[fill=cyan,draw=none] (2,0)--(2,2.8)--(3,2.8)node[midway,above]{$ 28 $}--(3,0)--cycle;
			\draw[fill=orange,draw=none] (3,0)--(3,1.4)--(4,1.4)node[midway,above]{$ 14 $}--(4,0)--cycle;
			\draw[fill=cyan,draw=none] (4,0)--(4,3.2)--(5,3.2)node[midway,above]{$ 32 $}--(5,0)--cycle;
			\draw[fill=orange,draw=none] (5,0)--(5,2.4)--(6,2.4)node[midway,above]{$ 24 $}--(6,0)--cycle;
			\draw[fill=cyan,draw=none] (6,0)--(6,1.7)--(7,1.7)node[midway,above]{$ 17 $}--(7,0)--cycle;
			\draw[fill=orange,draw=none] (7,0)--(7,1.4)--(8,1.4)node[midway,above]{$ 14 $}--(8,0)--cycle;
			%% miền
			\node [below] at (1,0){$ \left[1{,}0 ; 1{,}1 \right)$};
			\node [below] at (3,0){$ \left[1{,}1 ; 1{,}2 \right)$};
			\node [below] at (5,0){$ \left[1{,}2 ; 1{,}3 \right)$};
			\node [below] at (7,0){$ \left[1{,}3 ; 1{,}4 \right)$};
			\draw[->] (0,0)node [below left=-2pt]{$ O $}--(9,0)node[below]{(\text{kg})};
		\end{tikzpicture}
	\end{center}
	\begin{enumerate}[a)]
		\item Hãy so sánh cân nặng của lợn con mới sinh giống $ A $ và giống $ B $ theo số trung bình và trung vị.
		\item Hãy ước lượng tứ phân vị thứ nhất và thứ ba của cân nặng lợn con mới sinh giống $ A $ và của cân nặng lợn con mới sinh giống $ B $.
	\end{enumerate}
	\loigiai{
		\begin{enumerate}[a)]
			\item Bảng tần số ghép nhóm thống kê cân nặng của lợn con mới sinh giống $ A $ và giống $ B $ như sau:
			\begin{center}
				\begin{tabular}{|c|c|c|c|c|}
					\hline 
					\textbf{Cân nặng (kg)}	& $ \left[1{,}0 ; 1{,}1 \right)$  &$ \left[1{,}1 ; 1{,}2 \right)$  &$ \left[1{,}2 ; 1{,}3 \right)$  & $ \left[1{,}3 ; 1{,}4 \right)$ \\ 
					\hline 
					\textbf{Giá trị đại diện (kg)}	& $1{,}05 $ & $ 1{,}15 $ & $ 1{,}25 $ & $ 1{,}35 $ \\ 
					\hline 
					\begin{tabular}{c}
						\textbf{Giống A}
						\\ 
						\textbf{(đơn vị: con)}
					\end{tabular} 	& $ 8 $ & $ 28 $ & $ 32 $ & $ 17 $ \\ 
					\hline 
					\begin{tabular}{c}
						\textbf{Giống B}
						\\ 
						\textbf{(đơn vị: con)}
					\end{tabular} 	& $ 13 $ & $ 14 $ & $ 24 $ & $ 14 $ \\ 
					\hline 
				\end{tabular} 
			\end{center}
			Cân nặng trung bình của lợn con mới sinh giống $ A $ là $$ \dfrac{1{,}05\cdot 8 + 1{,}15 \cdot 28 + 1{,}25 \cdot 32 + 1{,}35 \cdot 17}{8+28+32+17}=\dfrac{2071}{1700}\approx 1{,}218.$$
			Cân nặng trung bình của lợn con mới sinh giống $ B $ là $$ \dfrac{1{,}05\cdot 13 + 1{,}15 \cdot 14 + 1{,}25 \cdot 24 + 1{,}35 \cdot 14}{13+14+24+14}=\dfrac{121}{100} \approx 1{,}21.$$
			Suy ra cân nặng trung bình của lợn con mới sinh giống $ A $  lớn hơn cân nặng trung bình của lợn con mới sinh giống $ B $. \\
			Trung vị của mẫu số liệu ghép nhóm cân nặng của lợn con  giống $ A $ là $$M_e=1{,}2+\dfrac{\dfrac{85}{2}-(8+28)}{32}\cdot (1{,}3-1{,}2)=\dfrac{781}{640}\approx 1{,}22.$$
			Trung vị của mẫu số liệu ghép nhóm cân nặng của lợn con  giống  $ B $ là $$M_e=1{,}2+\dfrac{\dfrac{65}{2}-(13+14)}{24}\cdot (1{,}3-1{,}2)=\dfrac{587}{480}\approx 1{,}22.$$
			Suy ra trung vị của mẫu số liệu ghép nhóm cân nặng của của lợn con  giống  $ A $ bằng trung vị của mẫu số liệu ghép nhóm cân nặng của của lợn con  giống $ B $.
			\item Hãy ước lượng tứ phân vị thứ nhất và thứ ba của cân nặng lợn con mới sinh giống $ A $ và của cân nặng lợn con mới sinh giống $ B $.
			\begin{itemize}
				\item Đối với mẫu dữ liệu ghép nhóm cân nặng lợn con mới sinh giống $ A $ có:\\
				Tứ phân vị thứ nhất của mẫu số liệu ghép nhóm cân nặng lợn con mới sinh giống $ A $ thuộc $ \left[ 1{,}1 ; 1{,}2\right)  $ nên $$ Q_1= 1{,}1+\dfrac{\dfrac{85}{4}-8}{28}\cdot \left( 1{,}2-1{,}1\right)= \dfrac{257}{224}\approx 1{,}147.$$
				Tứ phân vị thứ ba của mẫu số liệu ghép nhóm cân nặng lợn con mới sinh giống $ A $ thuộc $ \left[ 1{,}2 ; 1{,}3\right)  $ nên $$ Q_3= 1{,}2+\dfrac{\dfrac{85\cdot3 }{4}-(8+28)}{32}\cdot \left( 1{,}3-1{,}2\right)= \dfrac{1647}{1280}\approx 1{,}287.$$
				\item Đối với mẫu dữ liệu ghép nhóm cân nặng lợn con mới sinh giống $ B $ có:\\
				Tứ phân vị thứ nhất của mẫu số liệu ghép nhóm cân nặng lợn con mới sinh giống $ B $ thuộc $ \left[ 1{,}1 ; 1{,}2\right)  $ nên $$ Q_1= 1{,}1+\dfrac{\dfrac{65}{4}-13}{14}\cdot \left( 1{,}2-1{,}1\right)= \dfrac{629}{560}\approx 1{,}123.$$	
				Tứ phân vị thứ ba của mẫu số liệu ghép nhóm cân nặng lợn con mới sinh giống $ B $ thuộc $ \left[ 1{,}2 ; 1{,}3\right)  $ nên $$ Q_3= 1{,}2+\dfrac{\dfrac{65\cdot 3}{4}-(13+14)}{24}\cdot \left( 1{,}3-1{,}2\right)= \dfrac{413}{320}\approx 1{,}291.$$	
			\end{itemize}
		\end{enumerate}		
	}
\end{ex}

\begin{ex}[An Lạc - TPHCM 24-25]%[1D5V2-3]%[Dự án đề cương 3 Khối NH24-25-Dot1-Nguyễn Sĩ Đạt]
	Một trang báo điện tử thống kê thời gian người sử dụng đọc thông tin trên trang trong mỗi lần truy cập ở bảng sau
	\begin{longtable}{|l|c|c|c|c|c|}
		\hline
		Thời gian đọc (phút) & $[0;2)$ & $[2;4)$ & $[4;6)$ & $[6;8)$ & $[8;10)$\\
		\hline
		Số lượt truy cập & $45$ & $34$ & $23$ & $18$ & $5$\\
		\hline
	\end{longtable}
	\noindent
	Hãy ước lượng các tứ phân vị của mẫu số liệu ghép nhóm trên.
	\loigiai{
		Bảng dữ liệu có cỡ $n=45+34+23+18+5=125$.\\
		Sắp $125$ giá trị theo thứ tự không giảm: $x_1$, $x_2$, \ldots, $x_{125}$.\\
		Khi đó, các tứ phân vị $\heva{&Q_1=\dfrac{x_{31}+x_{32}}{2} \in [0;2)\\&Q_2=x_{63} \in [2;4) \\&Q_3=\dfrac{x_{94}+x_{95}}{2} \in [4;6)}$, do đó $\heva{&Q_1=0+\dfrac{\tfrac{125}{4}-0}{45}\cdot (2-0)=\dfrac{25}{18}\\&Q_2=2+\dfrac{\tfrac{125}{2}-45}{34}\cdot (4-2)=\dfrac{103}{34}\\&Q_3=4+\dfrac{\tfrac{3\cdot 125}{4}-79}{23}\cdot (6-4)=\dfrac{243}{46}.}$
	}
\end{ex}

\begin{ex}[Nguyễn Thượng Hiền - TPHCM 24-25]%[1D5V2-2]%[Dự án đề cương 3 Khối NH24-25-Dot1-Nguyễn Sĩ Đạt]
	Phòng tập thể dục thể thao của một trường THPT đã ghi lại số giờ học sinh khối 11 sử dụng cơ sở vật chất để tập luyện trong một tháng. Dữ liệu thu được trong bảng dưới đây:
	\begin{center}
		\begin{tabular}{|>{\centering\arraybackslash}p{4cm}|>{\centering\arraybackslash}p{1.5cm}|>{\centering\arraybackslash}p{1.5cm}|>{\centering\arraybackslash}p{1.5cm}|>{\centering\arraybackslash}p{1.5cm}|>{\centering\arraybackslash}p{1.5cm}|>{\centering\arraybackslash}p{1.5cm}|}
			\hline
			Thời gian (giờ)&$[1;5)$& $[5;9)$ & $[9;13)$ &$[13;17)$  & $[17;21)$ &$[21;25)$  \\
			\hline
			Tần số (số học sinh) & $10$ &$14$ & $31$ & $2$ & $5$ & $23$ \\
			\hline
		\end{tabular}
	\end{center}	
	\begin{enumerate}
		\item Tính số trung bình của mẫu số liệu (làm tròn kết quả đến hàng phần mười).
		\item Tính trung vị của mẫu số liệu (làm tròn kết quả đến hàng phần mười).
		\item Trong trường hợp này thì số trung bình hay trung vị đại diện tốt hơn cho mẫu số liệu? Giải thích.
	\end{enumerate}
	\loigiai{
		\begin{enumerate}
			\item Số trung bình của mẫu số liệu ghép nhóm là\\
			$$\overline{x}=\dfrac{10\cdot3+14\cdot7+31\cdot11+2\cdot15+5\cdot19+23\cdot23}{85} \approx 13{,}2.$$
			\item Gọi $x_1;x_2;\ldots;x_{85}$ lần lượt là thời gian tập luyện của $85$ học sinh khối $11$ được sắp xếp theo thứ tự không giảm. Trung vị của mẫu số liệu $x_1;x_2;\ldots;x_{85}$ là $x_{43} \in [9;13).$\\
			Ta có $n=85$;~$n_m=31$;~$C=10+14=24$;~$u_m=9$;~$u_{m+1}=13$.\\
			Trung vị của mẫu số liệu ghép nhóm là $M_e=9+\dfrac{\dfrac{85}{2}+14}{31} \cdot (13-9) \approx 11{,}4$.
			\item Như vậy trung bình thuộc nhóm $[13;17)$ cho thấy trong $85$ số liệu, đã có ít nhất $55$ số liệu nhỏ hơn trung bình $\overline{x}$. Suy ra, trong trường hợp này thì trung vị là số đại diện tốt hơn cho mẫu số liệu.
		\end{enumerate}
	}
\end{ex}

\begin{ex}[Ngô Gia Tự - Phú Yên 24-25]%[1D5H2-2]%[Dự án đề cương 3 Khối NH24-25-Dot1-Nguyễn Sĩ Đạt]
	Một học viện bóng đá điều tra về lứa tuổi của 100 học viên trẻ đăng kí đầu tiên để tham gia khóa học mới và thu được bảng sau
	\begin{center}
		\begin{tabular}{|c|c|c|c|c|c|}
			\hline Nhóm tuổi &{$[8;9]$}&{$[10;11]$}&{$[12;13]$}&{$[14;15]$}&{$[16;17]$}\\
			\hline Số học viên & 14 & 20 & 33 & 18 & 15 \\
			\hline
		\end{tabular}
	\end{center}
	Tìm cỡ mẫu và số trung vị của mẫu số liệu ghép nhóm trên.
	\loigiai{
		Ta có cỡ mẫu là $n=100$.\\
		Ta lập lại bảng số liệu ghép nhóm như sau
		\begin{center}
			\begin{tabular}{|c|c|c|c|c|c|}
				\hline Nhóm tuổi &{$[7,5;9,5)$}&{$[9,5;11,5)$}&{$[11,5;13,5)$}&{$[13,5;15,5)$}&{$[15,5;17,5)$}\\
				\hline Số học viên & 14 & 20 & 33 & 18 & 15 \\
				\hline
			\end{tabular}
		\end{center}
		Ta có $\dfrac{n}{2}=\dfrac{100}{2}=50$ nên $M_e\in [11,5;13,5)$.\\
		$M_e=11,5+\dfrac{50-(14+20)}{33}\cdot 2 \approx 12,5$.
	}
\end{ex}

\begin{ex}%[1D5C2-3]%[Dự án đề cương 3 Khối NH24-25-Dot1-Nguyễn Sĩ Đạt]
	Một phòng khám thống kê số bệnh nhân đến khám bệnh mỗi ngày trong tháng $ 4 $ năm $ 2022 $ ở bảng sau
	\begin{center}
		\begin{tabular}{|c|c|c|c|c|c|}
			\hline 
			Số bệnh nhân	& $ \left[1 ; 10 \right]  $ & $ \left[ 11 ; 20\right]  $ & $ \left[ 21 ; 30\right]  $ & $ \left[ 31 ; 40\right]  $ & $ \left[41 ; 50 \right]  $ \\ 
			\hline 
			Số ngày	& $ 7 $ & $ 8 $ & $ 7 $ & $ 6 $ & $ 2 $ \\ 
			\hline 
		\end{tabular} 
	\end{center}
	\begin{enumerate}
		\item Hãy ước lượng các tứ phân vị của mẫu số liệu ghép nhóm trên.
		\item Quản lý phòng khám cho rằng có khoảng $ 25\% $ số ngày khám có nhiều hơn $ 35 $ bệnh nhân đến khám. Nhận định trên có hợp lý không?
	\end{enumerate}
	\loigiai{
		\begin{enumerate}
			\item Do số bệnh nhân đến khám là số nguyên nên ta hiệu chỉnh lại như sau
			\begin{center}
				\begin{tabular}{|c|c|c|c|c|c|}
					\hline 
					\textbf{Số bệnh nhân}	& $ \left[0{,}5 ; 10{,}5 \right)  $ & $ \left[ 10{,}5 ; 20{,}5\right)  $ & $ \left[ 20{,}5 ; 30{,}5\right)  $ & $ \left[ 30{,}5 ; 40{,}5\right)  $ & $ \left[40{,}5 ; 50{,}5 \right)  $ \\ 
					\hline 
					\textbf{Số ngày}	& $ 7 $ & $ 8 $ & $ 7 $ & $ 6 $ & $ 2 $ \\ 
					\hline 
				\end{tabular} 
			\end{center}
			Gọi $ x_1; x_2; \ldots; x_{30} $ là mẫu số liệu được xếp theo thứ tự không giảm.\\
			Ta có $ x_1, \ldots, x_{7}\in \left[ 0{,}5 ; 10{,}5\right)  $; $ x_8, \ldots, x_{15} \in \left[ 10{,}5 ; 20{,}5\right) $, $ x_{16}, \ldots, x_{22} \in \left[ 20{,}5 ; 30{,}5\right) $;  $ x_{23}, \ldots, x_{28} \in \left[ 30{,}5 ; 40{,}5\right) $; $ x_{29}, x_{30} \in \left[ 40{,}5 ; 50{,}5\right) $.\\
			Tứ phân vị thứ nhất của dãy số liệu $ x_1; x_2; \ldots; x_{30} $ là  $ x_8 \in \left[10{,}5 ; 20{,}5 \right) $ nên tứ phân vị thứ nhất của mẫu số liệu ghép nhóm là $$ Q_1=10{,}5+\dfrac{\dfrac{30}{4}-7}{8}\cdot (20{,}5-10{,}5)=\dfrac{89}{8}\approx 11{,}125.$$
			Tứ phân vị thứ hai của dãy số liệu $ x_1; x_2; \ldots; x_{30} $ là $ \dfrac{1}{2}\left( x_{15}+x_{16}\right) $. Do $ x_{15} \in \left[ 10{,}5 ; 20{,}5 \right) $ và $ x_{16}\in \left[ 20{,}5 ; 30{,}5\right) $ nên tứ phân vị thứ hai của mẫu số liệu ghép nhóm là $ Q_2=20{,}5 $.\\
			Tứ phân vị thứ hai của dãy số liệu $ x_1; x_2; \ldots; x_{30} $ là $ x_{23}\in  \left[ 30{,}5 ; 40{,}5\right)$ nên tứ phân vị thứ ba của mẫu số liệu ghép nhóm là $$Q_3= 30{,}5+\dfrac{\dfrac{3\cdot 30}{4}-(7+8+7)}{6}\cdot(40{,}5-30{,}5)=\dfrac{94}{3}\approx 31{,}333.$$
			\item  Do $ Q_1\approx 11{,}125 $ nên nhận định $ 25\% $ số ngày khám nhiều hơn $ 35 $ bệnh nhân là không hợp lý.
		\end{enumerate}
	}
\end{ex}