\newpage
\section{Ôn tập chương 3}
\def\thoigian{90}%--Thời gian
\de{Đề số 1}{ÔN TẬP CHƯƠNG III - GIỚI HẠN - HÀM SỐ LIÊN TỤC}

\begin{center}
	\textbf{PHẦN 1 - CÂU TRẮC NGHIỆM BỐN PHƯƠNG ÁN}
\end{center}
\Opensolutionfile{ans}[ans/ans-TN-ONTAPCHUONG-DE1]
\begin{ex}%[1D3N1-1] %[Đề thi HK1 trường THPT thị xã Quảng Trị, Quảng Trị Năm học 2024-2025]
	Cho dãy số $(u_n)$ thỏa mãn $\lim(u_n+4)=9$. Tính $\lim u_n$.
	\choice
	{\True $5$}
	{$9$}
	{$4$}
	{$13$}
	\loigiai
	{Theo định nghĩa dãy số có giới hạn $a$, ta có $\lim(u_n+4)=9 \Leftrightarrow \lim(u_n-(-4))=9 \Leftrightarrow \lim(u_n)=9-4=5$.}
\end{ex}
\begin{ex}%[1D3N1-2] %[Đề thi HK1 trường THPT Thuận Thành số 1, Năm học 2024-2025]
	Dãy số nào sau đây có giới hạn bằng $0$?
	\choice
	{$u_n=\pi^n$}
	{\True $u_n=\left(\dfrac{2}{5}\right)^n$}
	{$u_n=\left(\dfrac{12}{5}\right)^n$}
	{$u_n=\dfrac{n^5}{2n+3}$}
	\loigiai{
		Ta thấy 	$\lim\left(\dfrac{2}{5}\right)^n=0$, vì $0<q=\dfrac{2}{5}<1$.
	}	
\end{ex}
\begin{ex}%[1D3H1-2] %[Đề thi HK1 Sở GD-ĐT Bắc Giang, Năm học 2024-2025]
	$\lim\limits_{n \to+\infty}\dfrac{3n-n^4}{4n^4-5}$ bằng
	\choice
	{$0$}
	{$\dfrac{3}{4}$}
	{$-\infty $}
	{\True $-\dfrac{1}{4}$}
	\loigiai
	{
		Ta có 
		$\lim\limits_{n \to +\infty}\dfrac{3n-n^4}{4n^4-5} = \lim\limits_{n \to +\infty} \dfrac{\dfrac{3}{n^3}-1}{4- \dfrac{5}{n^4}}= \dfrac{0-1}{4-0}= -\dfrac{1}{4}$.
	}
\end{ex}
\begin{ex}%[1D3N1-4] %[Đề thi HK1 Sở Bắc Ninh, Năm học 2024-2025]
	Giới hạn $\lim\limits_{n \to +\infty} \left(2^n+3^n-4^n\right)$ bằng
	\choice
	{$1$}
	{$-4$}
	{$+\infty$}
	{\True $-\infty$}
	\loigiai{
		Ta có $\lim\limits_{n \to +\infty} \left(2^n+3^n-4^n\right)
		=\lim\limits_{n \to +\infty} 4^n\left(\left(\dfrac{2}{4}\right)^n+\left(\dfrac{3}{4}\right)^n-1\right)
		=-\infty$.\\
		Vì $\lim\limits_{n \to +\infty} 4^n=+\infty$ và $\lim\limits_{n \to +\infty} \left(\left(\dfrac{2}{4}\right)^n+\left(\dfrac{3}{4}\right)^n-1\right)=-1<0$.
	}
\end{ex}
% GH hàm số
\begin{ex}%[1D3N2-2] %[Đề thi HK1 trường THPT Đoàn Thượng, Hải Dương. Năm học 2024-2025]
	Tính giới hạn $\lim\limits_{x\to 0}\dfrac{2x+1}{x+1}$ là 
	\choice
	{$\dfrac{1}{2}$}
	{$-1$}
	{$2$}
	{\True $1$}
	\loigiai 
	{
		Ta có $\lim\limits_{x\to 0}\dfrac{2x+1}{x+1}=\dfrac{2\cdot0+1}{0+1}=1$.
	}
\end{ex}
\begin{ex}%[1D3H2-4]
	Giá trị của giới hạn $\lim \limits_{x\to -\infty} \dfrac{x^4+x^3-2}{2x^3+x}$ là
	\choice
	{$2$}
	{$+\infty$}
	{\True $-\infty$}
	{$\dfrac{1}{2}$}
	\loigiai{$\displaystyle \lim_{x\to -\infty} \dfrac{x^4+x^3-2}{2x^3+x}=\lim_{x\to -\infty} \dfrac{x^3\left(x+1-\dfrac{2}{x^3}\right)}{x^3\left(2+\dfrac{1}{x^2}\right)}=\lim_{x\to -\infty} \dfrac{x+1-\dfrac{2}{x^3}}{2+\dfrac{1}{x}}=-\infty$.
	}
\end{ex}
\begin{ex}%[1D3H2-3]% [Đề thi HK1 trường THPT Lê Quý Đôn - HCM, Năm học 2024-2025]
	$\lim\limits_{x\to +\infty} \dfrac{2x^2-1}{x^2+x}$ bằng
	\choice
	{$1$}
	{$0$}
	{\True $2$}
	{$-1$}
	\loigiai
	{
		Ta có $\lim\limits_{x\to +\infty} \dfrac{2x^2-1}{x^2+x} = \lim\limits_{x\to +\infty} \dfrac{x^2\left(2-\dfrac{1}{x^2}\right)}{x^2\left(1+\dfrac{1}{x}\right)} = \lim\limits_{x\to +\infty} \dfrac{2-\dfrac{1}{x^2}}{1+\dfrac{1}{x}} = \dfrac{2-0}{1+0} = 2$.
	}
\end{ex}
\begin{ex}%[1D3H2-7] %Đề thi HK1 trường THPT thị xã Quảng Trị, Quảng Trị, Năm học 2024-2025]
	Tính $\lim\limits_{x \to 3^+}\dfrac{x-1}{x-3}$.
	\choice
	{$-\dfrac{1}{3}$}
	{$0$}
	{$-\infty$}
	{\True $+\infty$}
	\loigiai
	{Ta có 
		$\heva{&\lim\limits_{x \to 3^+} (x-3)=0\\&x \to 3^+\Rightarrow x-3>0\\& \lim\limits_{x \to 3^+} (x-1)=2>0.}$\\
	Vậy	$\lim\limits_{x \to 3^+}\dfrac{x-1}{x-3}= +\infty$.}
\end{ex}
\begin{ex}%[1D3H3-4] % Đề thi HK1 trường Nguyễn Thị Minh Khai-TPHCM, Năm học 2024-2025]
	Trong các hàm số sau, hàm số nào liên tục trên $\mathbb{R}$?
	\choice
	{$y=\sqrt{x^2-1}$}
	{$y=\cot x$}
	{$y=\dfrac{2x-1}{x-1}$}
	{\True $y=x^3-x$}
	\loigiai{
		\begin{itemize}
			\item Hàm số $y=\sqrt{x^2-1}$ có tập xác định là $(-\infty;-1]\cup [1;+\infty)$ nên không liên tục trên $\mathbb{R}$.
			\item Hàm số $y=\cot x$ có tập xác định là $\mathbb{R}\setminus \{k\pi,\ k\in\mathbb{Z}\}$ nên không liên tục trên $\mathbb{R}$.
			\item Hàm số $y=\dfrac{2x-1}{x-1}$ có tập xác định là $\mathbb{R}\setminus \{1\}$ nên không liên tục trên $\mathbb{R}$.
			\item Hàm số $y=x^3-x$ có tập xác định là $\mathbb{R}$ nên liên tục trên $\mathbb{R}$.
		\end{itemize}
	}
\end{ex}
\begin{ex}%[1D3H3-3] % Đề thi HK1 trường THPT Nguyễn Thị Minh Khai-TPHCM, Năm học 2024-2025]
	Cho hàm số $y=f(x)=\heva{&\dfrac{x^2-3x+2}{x^2-1} & \text{khi}\  x\ne 1 \\ &a & \text{khi}\ x=1}$. Xác định giá trị của tham số $a$ để hàm số $y=f(x)$ liên tục tại điểm $x=1$.
	\choice
	{\True $a=-\dfrac{1}{2}$}
	{$a=-1$}
	{$a=0$}
	{$a=\dfrac{1}{2}$}
	\loigiai{
		Ta có $f(1)=a$.\\
		Và $\lim\limits_{x\to 1}f(x)=\lim\limits_{x\to 1}\dfrac{x^2-3x+2}{x^2-1}=\lim\limits_{x\to 1} \dfrac{( x-1 )( x-2)}{(x-1)(x+1)}=\lim\limits_{x\to 1}\dfrac{x-2}{x+1}=-\dfrac{1}{2}$.\\
		Vậy để hàm số liên tục tại điểm $x=1$ thì $a=-\dfrac{1}{2}$.
	}
\end{ex}
\begin{ex}%[1D3H3-3]
	Cho hàm số $f(x) = \dfrac{x^2 + 3x - 4}{x + 4}$ với $x \neq -4$. Để hàm số $f(x)$ liên tục tại $x = -4$ thì giá trị $f(-4)$ là
	\choice
	{$0$}
	{$3$}
	{$5$}
	{\True $-5$}
	\loigiai{
	Để hàm số $f(x)$ liên tục tại $x = -4$ thì
		$$f(-4) = \lim\limits_{x \to -4}\dfrac{x^2 + 3x - 4}{x + 4} = \lim\limits_{x \to -4}\dfrac{(x - 1)(x + 4)}{x + 4} = \lim\limits_{x \to -4}(x - 1) = -5.$$	
	}
\end{ex}
\begin{ex}%[1D3H3-2]
	Các đồ thị của các hàm số $y=f(x)$, $y=g(x)$, $y=h(x)$, $y=t(x)$ như hình vẽ bên dưới. Đồ thị nào thể hiện hàm số không liên tục trên khoảng $(-2;2)$?
	\def\dotEX{}
	\choice
	{	\begin{tikzpicture}[>=stealth]
		\draw[->] (-3,0) -- (3,0);
		\draw[->] (0,-1.2) -- (0,5);
		\clip(-3.2,-1.2) rectangle (3.2,5.2);
		\draw[line width=1pt,smooth,samples=100,domain=-3.2:3.2] plot(\x,{(\x)*(\x)});
		\begin{scriptsize}
		\foreach \x in {-2,-1,1,2}
		\draw (\x,0.05) -- (\x,-0.05) (\x,-0.2) node {$\x$};
		\foreach \y in {1,2,3,4}
		\draw (0.05,\y) -- (-0.05,\y) (-0.2,\y) node {$\y$};
		\foreach \y in {-2,-1}
		\draw (0.05,\y) -- (-0.05,\y) (-0.3,\y) node {$\y$};
		\draw (-0.2,-0.2) node {$O$};
		\draw (-0.2,5) node {$y$};
		\draw (3,-0.2) node {$x$};
		\draw (-1,3) node {$y=f(x)$};
		\end{scriptsize}
		\end{tikzpicture}
	}
	{
		\begin{tikzpicture}[>=stealth]
		\draw[->] (-3,0) -- (3,0);
		\draw[->] (0,-5) -- (0,1.2);
		\clip(-3.2,-5.2) rectangle (3.2,1.2);
		\draw[line width=1pt,smooth,samples=100,domain=-3.2:3.2] plot(\x,{-(\x)*(\x)});
		\begin{scriptsize}
		\foreach \x in {-2,-1,1,2}
		\draw (\x,0.05) -- (\x,-0.05) (\x,0.2) node {$\x$};
		\foreach \y in {1,2}
		\draw (0.05,\y) -- (-0.05,\y) (-0.2,\y) node {$\y$};
		\foreach \y in {-4,-3,-2,-1}
		\draw (0.05,\y) -- (-0.05,\y) (-0.3,\y) node {$\y$};
		\draw (0.2,0.2) node {$O$};
		\draw (-0.2,3) node {$y$};
		\draw (3,0.2) node {$x$};
		\draw (-1.2,-4) node {$y=g(x)$};
		\end{scriptsize}
		\end{tikzpicture}
	}
	{
		\begin{tikzpicture}[>=stealth]
		\draw[->] (-3,0) -- (3,0);
		\draw[->] (0,-3) -- (0,3);
		\clip(-3.2,-3.2) rectangle (3.2,3.2);
		\draw[line width=1pt,smooth,samples=100,domain=-3.2:3.2] plot(\x,{(\x)+1});
		\begin{scriptsize}
		\foreach \x in {-2,-1,1,2}
		\draw (\x,0.05) -- (\x,-0.05) (\x,-0.2) node {$\x$};
		\foreach \y in {1,2}
		\draw (0.05,\y) -- (-0.05,\y) (-0.2,\y) node {$\y$};
		\foreach \y in {-2,-1}
		\draw (0.05,\y) -- (-0.05,\y) (-0.3,\y) node {$\y$};
		\draw (-0.2,-0.2) node {$O$};
		\draw (-0.2,3) node {$y$};
		\draw (3,-0.2) node {$x$};
		\draw (1.5,1.7) node {$y=h(x)$};
		\end{scriptsize}
		\end{tikzpicture}
	}
	{\True 
		\begin{tikzpicture}[>=stealth]
		\draw[->] (-3,0) -- (3,0);
		\draw[->] (0,-3) -- (0,3);
		\clip (-3.2,-3.2) rectangle (3.2,3.2);
		\draw (-2,-2) -- (-1.5,0) (-1.5,1.5) -- (2,-2);
		\draw [dashed] (-1.5,0) -- (-1.5,1.5);
		\begin{scriptsize}
		\foreach \x in {-2,-1,1,2}
		\draw (\x,0.05) -- (\x,-0.05) (\x,-0.2) node {$\x$};
		\foreach \y in {1,2}
		\draw (0.05,\y) -- (-0.05,\y) (-0.2,\y) node {$\y$};
		\foreach \y in {-2,-1}
		\draw (0.05,\y) -- (-0.05,\y) (-0.3,\y) node {$\y$};
		\draw (-0.2,-0.2) node {$O$};
		\draw (-0.2,3) node {$y$};
		\draw (3,-0.2) node {$x$};
		\draw (-1.4,-2) node {$y=t(x)$};
		\end{scriptsize}
		\end{tikzpicture}
	}
	\loigiai{
		Nhìn trên hình vẽ ta thấy đồ thị các hàm số $y=f(x)$, $y=g(x)$, $y=h(x)$ đều là các nét liền nên nó biểu diễn hàm số liên tục.\\
		Hàm số $y=t(x)$ có đồ thị gián đoạn tại $x=-1{,}5$, do $\lim\limits_{x\rightarrow -1{,}5^{-}}t(x)\ne \lim\limits_{x\rightarrow -1{,}5^{+}}t(x)$.
	}
\end{ex}
\Closesolutionfile{ans}
%\begin{center}
%	\textbf{ĐÁP ÁN}
%	\inputansbox{10}{ans/ans}	
%\end{center}

\begin{center}
	\textbf{PHẦN 2 - CÂU TRẮC NGHIỆM ĐÚNG SAI}
\end{center}

\Opensolutionfile{ans}[ans/answer-DS-ONTAPCHUONG-DE1]
\begin{ex}%[1D3H3-3]
	Cho hàm số $f(x)=\heva{&\dfrac{2x^2-3x+1}{2(x-1)} &\text{khi } x \neq 1\\
		&m &\text{khi } x=1&}$, với $m$ là tham số.
	\choiceTF
	{\True Tập xác định của hàm số $f(x)$ là $\mathscr{D}=\mathbb{R}$}
	{\True $f(1)=m$}
	{$\lim\limits_{x \to 1} f(x)=1$}
	{\True Hàm số $f(x)$ liên tục tại $x=1$ khi $m=\dfrac{1}{2}$}
	\loigiai 
	{
		\begin{itemchoice}
			\itemch 
			Tập xác định của hàm số $f(x)$ là $\mathscr{D}=\mathbb{R}$.
			\itemch 
			Ta có $f(1)=m$.
			\itemch 
			$\lim\limits_{x \to 1} f(x)=\lim\limits_{x \to 1}\dfrac{2x^2-3x+1}{2(x-1)}=\lim\limits_{x \to 1}\dfrac{2x-1}{2}=\dfrac{1}{2}$.
			\itemch 
			Hàm số $f(x)$ liên tục tại $x=1$ khi $\lim\limits_{x \to 1} f(x)=f(1)\Leftrightarrow m=\dfrac{1}{2}$.
		\end{itemchoice}
	}
\end{ex}
\begin{ex}%[1D3V3-6]%[TH-THCS-THPT-HoangViet-DakLak-HKI-NH24-25]%[Trần Hưng]
	Một bãi đỗ xe ôtô tính phí $60\,000$ cho giờ đầu tiên (hoặc một phần của giờ đầu tiên) và thêm $40\,000$ đồng cho mỗi giờ (hoặc một phần của mỗi giờ) tiếp theo, tối đa là $200\,000$ đồng. Gọi $C=C(t)$ là hàm số biểu thị chi phí theo thời gian đỗ xe.
	\choiceTF
	{Số tiền đỗ xe của một người với thời gian $2{,}5$ giờ là $140\,000$ đồng}
	{Hàm số $C(t)$ liên tục trên $[0;+\infty)$}
	{$\lim\limits _{t\rightarrow 3}C(t)=140\,000$}
	{\True Chênh lệch chi phí đối với hai khách hàng đỗ xe có thời gian $t_{1}$; $t_{2}$ thay đổi với $2<t_{1}\leq 3$; $3<t_{2}\leq 4$ là không đổi}
	\loigiai{
		Hàm số biểu thị chi phí theo thời gian đỗ xe\\
		$C(t)= \heva{&60\,000 & 0<t \leq 1 \\& 60\,000+40\,000(t-1) & 1<t \leq 4 \\& 200\,000 & t>4.}$
		\begin{itemchoice}
			\itemch 
			Số tiền đỗ xe của một người với thời gian $2{,}5$ giờ là\\
			$C(2{,}5)	=60\,000+40\,000(2{,}5-1)=120\,000$.
			\itemch 
			Ta có $\heva{&\lim\limits _{t\rightarrow 4^-}C(t)=180\,000\\&\lim\limits _{t\rightarrow 4^+}C(t)=200\,000.}$\\
			Suy ra hàm số không liên tục tại $t=4$.
			\itemch 
			Ta có $\lim\limits _{t\rightarrow 3}C(t)=\lim\limits _{t\rightarrow 3}\left[60\,000+40\,000(t-1)\right]=140\,000$.
			\itemch 
			Chi phí đối với khách hàng đỗ xe của hai khách hàng tại thời điểm $t_{1}$, $t_{2}$ lần lượt là\\
			$C(t_{1})=60\,000+40\,000(3-1)=140\,000$.\\
			$C(t_{2})=60\,000+40\,000(4-1)=180\,000$.\\
			Chênh lệch chi phí đối với hai khách hàng đỗ xe là
			\[C(t_{1})-C(t_{2})=180\,000-140\,000=40\,000.\]
		\end{itemchoice}
	}
\end{ex}
\Closesolutionfile{ans}
%\inputansbox[2]{2}{ans/answer.tex}

\begin{center}
\textbf{PHẦN 3 - CÂU TRẮC NGHIỆM TRẢ LỜI NGẮN}
\end{center}
\setcounter{ex}{0}
\Opensolutionfile{ans}[ans-KQ-ONTAPCHUONG-DE1]
\begin{ex}%[1D3V2-8]
	Chi phí (đơn vị: nghìn đồng) để sản xuất $x$ sản phẩm của một công ty được xác định bởi hàm số $C(x)=50000+ 120x$. Biết rằng số lượng sản phẩm sản xuất càng nhiều thì chi phí sản xuất sẽ càng giảm. Khi đó chi phí sản xuất một sản phẩm thấp nhất là bao nhiêu nghìn đồng?
	\shortans[0]{$120$}
	\loigiai{
		$-$ Chi phí trung bình sản xuất một sản phẩm là $\overline{C}(x)= \dfrac{50000+ 120x }{x}$, $(x>0)$.\\
		Vì số lượng sản phẩm sản xuất càng nhiều thì chi phí sản xuất sẽ càng giảm nên  chi phí sản xuất một sản phẩm thấp nhất là\\
		$\lim\limits_{x\to +\infty} \overline{C}(x)= \lim\limits_{x\to +\infty} \dfrac{50000+ 120x }{x}=\lim\limits_{x\to +\infty}\left(\dfrac{50000}{x}+120\right)= 120$.\\
		Vậy chi phí sản xuất một sản phẩm thấp nhất là $120$ nghìn đồng.
	}
\end{ex}
\begin{ex}%[1D3V1-6]
	Bạn Lan thả quả bóng cao su từ độ cao $12$ mét theo phương thẳng đứng. Mỗi khi chạm đất nó lại nảy lên theo phương thẳng đứng với độ cao bằng $\dfrac{2}{3}$ độ cao trước đó. Tính tổng quãng đường bóng đi được đến khi bóng dừng hẳn.
	\shortans[]{$60$}
	\loigiai{
		Các quãng đường khi bóng đi xuống tạo thành một cấp số nhân có $u_1=12$ và $q=\dfrac{2}{3}$.\\
		Tổng các quãng đường khi bóng đi xuống là $S=\dfrac{u_1}{1-q}=\dfrac{12}{\left(1-\dfrac{2}{3}\right)}=36$ mét.\\
		Vậy tổng quãng đường bóng đi được (cả lên và xuống) đến khi bóng dừng hẳn là
		$$2 S-12=2\cdot 36-12=60 \text{ (mét).}$$
	}
\end{ex}
\begin{ex}%[1D3V3-6]
	Hãng taxi Xanh Việt Đức đưa ra giá cước tại tỉnh Đắk Lắk dựa trên số quãng đường di chuyển cho bởi hàm $T\left(x\right)$ (đồng) khi đi quãng đường $x$ (km) cho loại xe $4$ chỗ như sau 
	$$T\left(x\right)=\heva{&10000 \hspace*{2.55cm} \text{khi $0<x\le 1$}\\&a+13000\left(x-1\right) \hspace*{0.5cm} \text{khi $1<x \le 30$}\\&b+11000\left(x-30\right) \hspace*{0.3cm} \text{khi $x>30$.}
	}$$
	Biết rằng tiền cước được cho bởi hàm số liên tục trên $\left(0;+\infty\right)$, khi đó $\dfrac{b}{a}$ bằng bao nhiêu?.
	\shortans{$38,7$}
	\loigiai{Ta có $T\left(x\right)$ liên tục trên $\left(0;+\infty\right)$ nên $T\left(x\right)$ liên tục tại $x=1$ và $x=30$.\\
		Vì $T\left(x\right)$ liên tục tại $x=1$ nên 
		$$\lim\limits_{x \to 1} T\left(x\right)=T\left(1\right) \Rightarrow \lim\limits_{x \to 1^+} T\left(x\right)=T\left(1\right) \Rightarrow a+13000 \cdot \left(1-1\right)=10000 \Rightarrow a=10000.$$
		Vì $T\left(x\right)$ liên tục tại $x=30$ nên 
		\begin{align*}
		\lim\limits_{x \to 1} T\left(x\right)=T\left(1\right) &\Rightarrow \lim\limits_{x \to {30}^+} T\left(x\right)=T\left(30\right)\\
		& \Rightarrow b+11000 \cdot \left(30-30\right)=a+13000 \cdot \left(30-1\right)\\ &\Rightarrow b=10000+13000 \cdot 29\\
		&\Rightarrow b=387000.
		\end{align*}
		Vậy $\dfrac{b}{a}=\dfrac{387000}{10000}=38{,}7$.}
\end{ex}
\begin{ex}%[1D3H3-3]
	Cho hàm số $y = f(x) = \heva{&\dfrac{x^2-5x+6}{x-2} & \text{nếu } x \ne 2 \\& a^2-2a & \text{nếu } x=2}$. Có bao nhiêu giá trị thực của tham số $a$ để hàm số liên tục tại điểm $x_0=2$.
	\shortans{$1$}
	\loigiai{
		Tập xác định của hàm số là $\mathscr{D}=\mathbb{R}$.\\
		Ta có $f(2) = a^2-2a$.\\
		Ta tính giới hạn của hàm số khi $x \to 2$
		$$ \lim\limits_{x \to 2}f(x) = \lim\limits_{x \to 2}\dfrac{x^2-5x+6}{x-2} = \lim\limits_{x \to 2}\dfrac{(x-2)(x-3)}{x-2} = \lim\limits_{x \to 2}(x-3) = 2-3=-1.$$
		Để hàm số đã cho liên tục tại điểm $x_0=2$ thì $\lim\limits_{x \to 2}f(x)=f(2)\Leftrightarrow -1 = a^2-2a\Leftrightarrow a^2-2a+1=0 \Leftrightarrow a=1$.\\
		Vậy có một giá trị thực của $a$ là $a=1$ thỏa mãn yêu cầu bài toán.
	}
\end{ex}
\Closesolutionfile{ans}
\begin{center}
	\textbf{PHẦN 4 - TỰ LUẬN}
\end{center}
%Câu 1...........................
\begin{ex}%[1D3H1-2]
	Tính giới hạn
	$L=\lim\displaystyle\dfrac{2^n-3^{n-2}+3\cdot 5^{n+2}}{2^{n-1}+3^{n+2}+5^{n+1}}$. 
	\loigiai{
		Chia tử và mẫu cho $5^n$, ta được
		$$L=\lim\displaystyle\dfrac{ \left(\dfrac{2}{5}\right)^n-\dfrac{1}{9}\cdot \left(\dfrac{3}{5}\right)^n+75}{\dfrac{1}{2}\cdot \left(\dfrac{2}{5}\right)^n+9\cdot \left(\dfrac{3}{5}\right)^n+5}=\dfrac{0-0+75}{0+0+5}=15.$$
	}
\end{ex}
%Câu 2...........................
\begin{ex}%[1D3V2-5]
	Tính giới hạn $\lim\limits_{x\to 0}\dfrac{\sqrt{2x+1}-\sqrt[3]{3x+1}}{x^2}$.
	\loigiai{
		Ta có
		\begin{align*}
		&\lim\limits_{x\to 0}\dfrac{\sqrt{2x +1} -\sqrt[3]{3x+1}}{x^2}\\
		=&\lim\limits_{x\to 0}\left[\dfrac{\sqrt{2x +1}-(1+x)}{x^2} -\dfrac{\sqrt[3]{3x+1}-(1+x)}{x^2}\right]\\
		=&\lim\limits_{x\to 0}\left[\dfrac{2x +1-x^2-2x-1}{x^2\left(\sqrt{2x +1}+(1+x)\right)} -\dfrac{{3x+1}-x^3-3x^2-3x-1}{x^2\left(\sqrt[3]{(3x+1)^2}+(1+x)\sqrt[3]{3x+1}+(x+1)^2\right)}\right]\\
		=&\lim\limits_{x\to 0}\left[\dfrac{-x^2}{x^2\left(\sqrt{2x +1}+(1+x)\right)} +\dfrac{x^3+3x^2}{x^2\left(\sqrt[3]{(3x+1)^2}+(1+x)\sqrt[3]{3x+1}+(x+1)^2\right)}\right]\\
		=&\lim\limits_{x\to 0}\left[\dfrac{-1}{\sqrt{2x +1}+(1+x)} +\dfrac{x+3}{\sqrt[3]{(3x+1)^2}+(1+x)\sqrt[3]{3x+1}+(x+1)^2}\right]\\
		=&1-\dfrac{1}{2}=\dfrac{1}{2}.
		\end{align*}
	}
\end{ex}
%Câu 3...........................
\begin{ex}%[1D3B3-3]
	Xét tính liên tục của hàm số $y=f(x)=\heva{& \dfrac{x|x-3|}{x-3}&&\text{ khi } x\ne3 \\& 3&&\text{ khi } x=3}$ tại điểm $x=3$.
	\loigiai{
		Tập xác định của hàm số là $\mathscr{D}=\mathbb R$, chứa điểm $3$.\\
		Ta có $f(3)=3$;\\
		Khi $x>3$, $f(x)=\dfrac{x(x-3)}{x-3}=x$ nên $\lim\limits_{x\to 3^+}f(x)=\lim\limits_{x\to 3^+} x=3$;\\
		Khi $x<3$, $f(x)=\dfrac{x(3-x)}{x-3}=-x$ nên $\lim\limits_{x\to 3^-}f(x)=\lim\limits_{x\to 3^-} x=-3$.\\
		Do $\lim\limits_{x\to 3^+}f(x)\ne \lim\limits_{x\to 3^-}f(x)$ nên không tồn tại $\lim\limits_{x\to 3}f(x)$.\\
		Do đó, hàm số $y=f(x)$ không liên tục tại điểm $x=3$.
	}
\end{ex}