\section{ĐẠO HÀM}
\subsection{LÝ THUYẾT CẦN NHỚ}
\subsubsection{Đạo hàm}
\indam{Định nghĩa:}
	\begin{boxdn}
		Cho hàm số $y=f(x)$ xác định trên $(a;b)$ và $x_0\in (a;b)$.\\
		Nếu tồn tại giới hạn hữu hạn \[\lim \limits_{x\to x_0}\dfrac{f\left(x\right)-f\left(x_0\right)}{x-x_0}\] thì giới hạn này được gọi là \textit{\textbf{đạo hàm}} của hàm số $f(x)$ tại $x_0$, kí hiệu là $f'\left(x_0\right)$ hoặc $y'\left(x_0\right)$.\\ 
		Vậy \[f'\left(x_0\right)=\lim \limits_{x\to x_0}\dfrac{f\left(x\right)-f\left(x_0\right)}{x-x_0}\]
	\end{boxdn}
	
\begin{khung4}{Chú ý $1$}
	Cho hàm số $y=f(x)$ xác định trên khoảng $(a;b)$. Nếu hàm số này có đạo hàm tại mọi điểm $x\in(a;b)$ thì ta nói nó có \textit{đạo hàm trên khoảng} $(a;b)$, kí hiệu $y'$ hoặc $f'(x)$.
\end{khung4}
\begin{khung4}{Chú ý $2$}
	Cho hàm số $y=f(x)$ xác định trên khoảng $(a;b)$, có đạo hàm tại $x_0\in (a;b)$.
	\begin{enumerate}
		\item Đại lượng $\Delta x=x-x_0$ gọi là số gia của biến tại $x_0$. Đại lượng $\Delta y=f(x)-f\left(x_0\right)$ gọi là số gia tương ứng của hàm số. Khi đó, $x=x_0+\Delta x$ và \[f'\left(x_0\right)=\lim \limits_{\Delta x\to 0}\dfrac{\Delta y}{\Delta x}=\lim \limits_{\Delta x\to 0}\dfrac{f\left(x_0+\Delta x\right)-f\left(x_0\right)}{\Delta x}.  \]
		\item Tỉ số $\dfrac{\Delta y}{\Delta x}$ biểu thị tốc độ thay đổi trung bình của đại lượng $y$ theo đại lượng $x$ trong khoảng từ $x_0$ đến $x_0+\Delta x$; còn $f'\left(x_0\right)$ biểu thị tốc độ thay đổi (tức thời) của đại lượng $y$ theo đại lượng $x$ tại điểm $x_0$.
	\end{enumerate}
\end{khung4}
\begin{khung4}{Ý nghĩa vật lí của đạo hàm}
	\begin{itemize}
		\item Nếu hàm số $s=f(t)$ biểu thị quãng đường di chuyển của vật theo thời gian $t$ thì $f'\left(t_0\right)$ biểu thị tốc độ tức thời của chuyển động tại thời điểm $t_0$.
		\item Nếu hàm số $T=f(t)$ biểu thị nhiệt độ $T$ theo thời gian $t$ thì $f'\left(t_0\right)$ biểu thị tốc độ thay đổi nhiệt độ theo thời gian tại thời điểm $t_0$.
	\end{itemize}
\end{khung4}

\subsubsection{Ý nghĩa hình học của đạo hàm}
\indam{Định nghĩa:}
\begin{boxdn}
	Đạo hàm của hàm số $y=f(x)$ tại điểm $x_0$ là hệ số góc của tiếp tuyến $M_0T$ của $(C)$ tại điểm $M_0\left(x_0;f\left(x_0\right)\right)$.\\ 
	Tiếp tuyến $M_0T$ có phương trình là $y-f\left(x_0\right)=f'\left(x_0\right)\left(x-x_0\right)$.
\end{boxdn}

\subsubsection{Số e}
\indam{Định nghĩa:}
\begin{boxdn}
	Người ta chứng minh được rằng có giới hạn hữu hạn \[\lim \limits_{x\to +\infty}\left(1+\dfrac{1}{x}  \right)^x=\mathrm{e}.  \]
	Hơn nữa, người ta còn biết rằng $\mathrm{e}$ là số vô tỉ và $\mathrm{e}=2{,718281828\ldots}$ (số thập phân vô hạn không tuần hoàn).
\end{boxdn}
%-------------------------------------------------------------------------------------------------------------
\subsection{PHÂN LOẠI VÀ PHƯƠNG PHÁP GIẢI TOÁN}
\begin{dang}{Tính đạo hàm của hàm số bằng định nghĩa}
	Để tính đạo hàm của hàm số $y=f(x)$ tại $x_0\in (a;b)$, ta thực hiện theo các bước sau
	\begin{enumerate}[label=\arabic*.]
		\item Tính $f(x)-f(x_0)$.
		\item Lập và rút gọn tỉ số $\dfrac{f(x)-f(x_0)}{x-x_0}$ với $x\in(a;b)$, $x\neq x_0$.
		\item Tính $\lim \limits_{x\to x_0}\dfrac{f(x)-f(x_0)}{x-x_0}$.
		\item Kết luận.
	\end{enumerate}
	Hàm số $y=f(x)$ có đạo hàm tại điểm $x=x_0\Leftrightarrow f'\left({x_0}^+\right)=f'\left({x_0}^-\right)$.\\ 
	Hàm số $y=f(x)$ có đạo hàm tại điểm thì trước hết phải liên tục tại điểm đó.
\end{dang}

\begin{vd}%[1D7N1-1]%[Dự án đề cương 3 Khối NH24-25-Dot 1-Viet Hoang Pham]
	Tính đạo hàm bằng định nghĩa tại một điểm của các hàm số sau
	\begin{multicols}{2}
		\begin{enumerate}
			\item $y=f(x)=2x^3+x-1$ tại $x_0=0$.
			\item $y=f(x)=x^2+2x-1$ tại $x_0=1$.
		\end{enumerate}
	\end{multicols}
	\loigiai{
		\begin{enumerate}
			\item Ta có $f(x)-f(0)=2x^3+x-1-(-1)=2x^3+x$.\\ 
			Suy ra $\dfrac{f(x)-f(0)}{x-0}=\dfrac{2x^3+x}{x}=2x^2+1$.\\ 
			$\lim\limits_{x\to 0}\dfrac{f(x)-f(0)}{x-0}=\lim\limits_{x\to 0}\left(2x^2+1\right)=2\cdot 0^2+1=1$.\\ Vậy $f'(0)=1$.
			\item Ta có $f(x)-f(1)=x^2+2x-1-(1^2+2\cdot 1-1)=x^2+2x-3$.\\ 
			Suy ra $\dfrac{f(x)-f(1)}{x-1}=\dfrac{x^2+2x-3}{x-1}=\dfrac{(x-1)(x+3)}{x-1}=x+3$.\\ 
			$\lim\limits_{x\to 1}\dfrac{f(x)-f(1)}{x-1}=\lim\limits_{x\to 1}(x+3)=1+3=4$.\\ Vậy $f'(1)=4$.
		\end{enumerate}
	}
\end{vd}
\begin{vd}%[1D7H1-1]%[Dự án đề cương 3 Khối NH24-25-Dot 1-Viet Hoang Pham]
	Tính đạo hàm bằng định nghĩa tại một điểm của các hàm số sau
	\begin{multicols}{2}
		\begin{enumerate}
			\item $y=f(x)=\dfrac{1}{x^2+x+1}$ tại $x_0=-2$.
			\item $y=f(x)=\dfrac{x^2+x-3}{2x-1}$ tại $x_0=3$.
		\end{enumerate}
	\end{multicols}
	\loigiai{
		\begin{enumerate}
			\item Ta có $f(x)-f(-2)=\dfrac{1}{x^2+x+1}-\dfrac{1}{(-2)^2+(-2)+1}=\dfrac{1}{x^2+x+1}-\dfrac{1}{3}=\dfrac{3-\left(x^2+x+1\right)}{3\left(x^2+x+1\right)}=\dfrac{-x^2-x+2}{5\left(x^2+x+1\right)}$.\\ 
			Suy ra $\dfrac{f(x)-f(-2)}{x-(-2)}=\dfrac{\dfrac{-x^2-x+2}{5\left(x^2+x+1\right)}}{x+2}=\dfrac{-(x+2)(x-1)}{5\left(x^2+x+1\right)(x+2)}=\dfrac{-x+1}{5\left(x^2+x+1\right)}$.\\ 
			Do đó $\lim\limits_{x\to -2}\dfrac{f(x)-f(-2)}{x-(-2)}=\lim\limits_{x\to -2}\dfrac{-x+1}{5\left(x^2+x+1\right)}=\dfrac{-(-2)+1}{5\left((-2)^2+(-2)+1\right)}=\dfrac{1}{5}$.\\ Vậy $f'(-2)=\dfrac{1}{5}$.
			\item Ta có $f(x)-f(3)=\dfrac{x^2+x-3}{2x-1}-\dfrac{3^2+3-3}{2\cdot 3-1}=\dfrac{x^2+x-3}{2x-1}-\dfrac{9}{5}=\dfrac{5\left(x^2+x-3\right)-9(2x-1)}{5\left(2x-1\right)}=\dfrac{5x^2-13x-6}{5\left(2x-1\right)}$.\\ 
			Suy ra $\dfrac{f(x)-f(3)}{x-3}=\dfrac{\dfrac{5x^2-13x-6}{5\left(2x-1\right)}}{x-3}=\dfrac{(x-3)(5x+2)}{5\left(2x-1\right)(x-3)}=\dfrac{5x+2}{5\left(2x-1\right)}$.\\ 
			Do đó $\lim\limits_{x\to 3}\dfrac{f(x)-f(3)}{x-3}=\lim\limits_{x\to 3}\dfrac{5x+2}{5\left(2x-1\right)}=\dfrac{5\cdot 3+2}{5\left(2\cdot 3-1\right)}=\dfrac{17}{25}$\\ Vậy $f'(3)=\dfrac{17}{25}$.
		\end{enumerate}
	}
\end{vd}
\begin{vd}%[1D7V1-1]%[Dự án đề cương 3 Khối NH24-25-Dot 1-Viet Hoang Pham]
	Tính đạo hàm bằng định nghĩa tại một điểm của các hàm số sau
	\begin{multicols}{2}
		\begin{enumerate}
			\item $f(x)=\heva{&\dfrac{x^2+3x-1}{x-1}&\text{khi }x\neq 1\\& -3&\text{ khi }x= 1}$ tại $x=1$.
			\item $f(x)=\heva{&\dfrac{\sqrt{x^3+x^2+1}-1}{x}&\text{khi }x\neq 0\\& 0&\text{ khi }x=0}$ tại $x=0$.
		\end{enumerate}
	\end{multicols}
	\loigiai{
		\begin{enumerate}
			\item Xét tính liên tục của hàm số tại $x=1$.\\ 
				Ta có $\heva{&\lim\limits_{x\to 1^+}f(x)=\lim\limits_{x\to 1^+}\dfrac{x^2+3x-1}{x-1}=+\infty\\ 
				&f(1)=-3}.\\ 
				\Rightarrow \lim\limits_{x\to 1^+}f(x)\neq f(1)$.\\ 
				$\Rightarrow$ Hàm số không liên tục tại $x=1$.\\ 
				$\Rightarrow$ Hàm số không có đạo hàm tại $x=1$.
			\item \begin{itemize}
				\item Xét tính liên tục của hàm số tại $x=0$.\\ 
				Ta có $\heva{&\lim\limits_{x\to 0}f(x)=\lim\limits_{x\to 0}\dfrac{\sqrt{x^3+x^2+1}-1}{x}=0\\ 
				&f(0)=0}.\\ 
				\Rightarrow \lim\limits_{x\to 0}f(x)= f(0)$.\\ 
				$\Rightarrow$ Hàm số liên tục tại $x=0$.
				\item Tính $f'(0)$.\\ 
				Ta có $f(x)-f(0)=\dfrac{\sqrt{x^3+x^2+1}-1}{x}-0=\dfrac{\sqrt{x^3+x^2+1}-1}{x}$.\\ 
				$\dfrac{f(x)-f(0)}{x-0}=\dfrac{\dfrac{\sqrt{x^3+x^2+1}-1}{x}}{x}=\dfrac{\sqrt{x^3+x^2+1}-1}{x^2}\cdot \dfrac{\sqrt{x^3+x^2+1}+1}{\sqrt{x^3+x^2+1}+1}=\dfrac{x^3+x^2+1-1}{x^2\left(\sqrt{x^3+x^2+1}+1\right)}=\dfrac{x}{\sqrt{x^3+x^2+1}+1}$.\\ 
				$\lim\limits_{x\to 0}\dfrac{f(x)-f(0)}{x-0}=\lim\limits_{x\to 0}\dfrac{x}{\sqrt{x^3+x^2+1}+1}=0$.\\ 
				Vậy $f'(0)=0$.
			\end{itemize}
		\end{enumerate}
	}
\end{vd}

\begin{dang}{Tính đạo hàm tại một điểm bất kì trên \textit{(a;b)} bằng định nghĩa}
	Để tính đạo hàm của hàm số $y=f(x)$ tại $x_0\in (a;b)$, ta thực hiện theo các bước sau
	\begin{enumerate}[label=\arabic*.]
		\item Lập và rút gọn tỉ số $\dfrac{f(x)-f(x_0)}{x-x_0}$ với $x\in(a;b)$, $x\neq x_0$.
		\item Tính $\lim \limits_{x\to x_0}\dfrac{f(x)-f(x_0)}{x-x_0}$.
		\item Kết luận.
	\end{enumerate}
	Hàm số $y=f(x)$ có đạo hàm tại điểm thì trước hết phải liên tục tại điểm đó.
\end{dang}
\setcounter{vd}{0}
%Ví dụ 1
\begin{vd}%[1D7N1-1]%[Dự án đề cương 3 Khối NH24-25-Dot 1-Viet Hoang Pham]
	Tính đạo hàm bằng định nghĩa của các hàm số sau
	\begin{multicols}{2}
		\begin{enumerate}
			\item $y=f(x)=4x+3$.
			\item $y=f(x)=2\,024x+2\,025$.
		\end{enumerate}
	\end{multicols}
	\loigiai{
		\begin{enumerate}
			\item Với bất kì $x_0$, ta có\\ 
			$\dfrac{f(x)-f(x_0)}{x-x_0}=\dfrac{4x+3-(4x_0+3)}{x-x_0}=\dfrac{4(x-x_0)}{x-x_0}=4$.\\ 
			$\lim\limits_{x\to x_0}\dfrac{f(x)-f(x_0)}{x-x_0}=\lim\limits_{x\to x_0}4=4$.\\ 
			Vậy $f'(x_0)=4$.
			\item Với bất kì $x_0$, ta có\\ 
			$\dfrac{f(x)-f(x_0)}{x-x_0}=\dfrac{2\,024x+2\,025-(2\,024x_0+2\,025)}{x-x_0}=\dfrac{2\,024(x-x_0)}{x-x_0}=2\,024$.\\ 
			$\lim\limits_{x\to x_0}\dfrac{f(x)-f(x_0)}{x-x_0}=\lim\limits_{x\to x_0}2/,024=2\,024$.\\
			Vậy $f'(x_0)=2\,024$.
		\end{enumerate}
	}
\end{vd}
\begin{vd}%[1D7H1-1]%[Dự án đề cương 3 Khối NH24-25-Dot 1-Viet Hoang Pham]
	Tính đạo hàm bằng định nghĩa của các hàm số sau
	\begin{multicols}{2}
		\begin{enumerate}
			\item $y=f(x)=2x^2+2\,024$.
			\item $y=f(x)=x^2-3x+1$.
		\end{enumerate}
	\end{multicols}
	\loigiai{
		\begin{enumerate}
			\item Với bất kì $x_0$, ta có\\ 
			$\dfrac{f(x)-f(x_0)}{x-x_0}=\dfrac{2x^2+2\,024-\left(2{x_0}^2+2\,024\right)}{x-x_0}=\dfrac{2\left(x^2-{x_0}^2\right)}{x-x_0}=2(x+x_0)$.\\ 
			$\lim\limits_{x\to x_0}\dfrac{f(x)-f(x_0)}{x-x_0}=\lim\limits_{x\to x_0}2(x+x_0)=4x_0$.\\
			Vậy $f'(x_0)=4x_0$.
			\item Với bất kì $x_0$, ta có\\ 
			$\dfrac{f(x)-f(x_0)}{x-x_0}=\dfrac{x^2-3x+1-\left({x_0}^2-3x_0+1\right)}{x-x_0}=\dfrac{\left(x^2-{x_0}^2\right)-3(x-x_0)}{x-x_0}=\dfrac{\left(x-{x_0}\right)(x+x_0-3)}{x-x_0}=x+x_0-3$.\\ 
			$\lim\limits_{x\to x_0}\dfrac{f(x)-f(x_0)}{x-x_0}=\lim\limits_{x\to x_0}(x+x_0-3)=2x_0-3$.\\
			Vậy $f'(x_0)=2x_0-3$.
		\end{enumerate}
	}
\end{vd}
\begin{vd}%[1D7V1-1]%[Dự án đề cương 3 Khối NH24-25-Dot 1-Viet Hoang Pham]
	Tính đạo hàm bằng định nghĩa của các hàm số sau
	\begin{multicols}{2}
		\begin{enumerate}
			\item $y=f(x)=x^3-2x$.
			\item $y=f(x)=x^4-2x^2+2$.
		\end{enumerate}
	\end{multicols}
	\loigiai{
		\begin{enumerate}
			\item Với bất kì $x_0$, ta có\\ 
			$\dfrac{f(x)-f(x_0)}{x-x_0}=\dfrac{x^3-2x-\left({x_0}^3-2x_0\right)}{x-x_0}=\dfrac{\left(x^3-{x_0}^3\right)-2(x-x_0)}{x-x_0}=\dfrac{(x-x_0)\left(x^2+xx_0+{x_0}^2-2\right)}{x-x_0}=x^2+xx_0+{x_0}^2-2$.\\ 
			$\lim\limits_{x\to x_0}\dfrac{f(x)-f(x_0)}{x-x_0}=\lim\limits_{x\to x_0}(x^2+xx_0+{x_0}^2-2)=3{x_0}^2-2$.\\
			Vậy $f'(x_0)=3{x_0}^2-2$.
			\item Với bất kì $x_0$, ta có\\ 
			$\dfrac{f(x)-f(x_0)}{x-x_0}=\dfrac{x^4-2x^2+2-\left({x_0}^4-2{x_0}^2+2\right)}{x-x_0}=\dfrac{\left(x^4-{x_0}^4\right)-2\left(x^2-{x_0}^2\right)}{x-x_0}=\dfrac{(x-x_0)(x+x_0)\left(x^2+{x_0}^2-2\right)}{x-x_0}=(x+x_0)\left(x^2+{x_0}^2-2\right)$.\\ 
			$\lim\limits_{x\to x_0}\dfrac{f(x)-f(x_0)}{x-x_0}=\lim\limits_{x\to x_0}(x+x_0)\left(x^2+{x_0}^2-2\right)=2{x_0}\left({x_0}^2+{x_0}^2-2\right)=4{x_0}^3-4x_0$.\\
			Vậy $f'(x_0)=4{x_0}^3-4x_0$.
		\end{enumerate}
	}
\end{vd}
\begin{dang}{Ý nghĩa hình học của đạo hàm}
	Để viết phương trình tiếp tuyến của đồ thị hàm số $y=f(x)$ tại $x_0=a$.
	\begin{enumerate}[label=\arabic*.]
		\item Tính $y_0=f(x_0)$.
		\item Tính $f'(x_0)$.
		\item Hoàn thiện phương trình tiếp tuyến cần tìm $y=f'(x_0)(x-x_0)+f(x_0)$
	\end{enumerate}
	Tiếp tuyến có hệ số góc $k=f'(x_0)$.
\end{dang}
\setcounter{vd}{0}
\begin{vd}%[1D7N1-3]%[Dự án đề cương 3 Khối NH24-25-Dot 1-Viet Hoang Pham]
	Cho hàm số $y=x^2+2x-4$ có đồ thị $(C)$. Tìm hệ số góc của tiếp tuyến của $(C)$ tại điểm có hoành độ $x_0=1$ thuộc $(C)$.
	\loigiai{
	Tiếp tuyến tại $x_0=1$ có hệ số góc là 
	\begin{eqnarray*}
		k&=&y'(1)\\ 
		&=&\lim\limits_{x\to 1}\dfrac{y(x)-y(1)}{x-1}\\ 
		&=&\lim\limits_{x\to 1}\dfrac{x^2+2x-4-\left(1^2+2\cdot 1-4\right)}{x-1}\\
		&=&\lim\limits_{x\to 1}\dfrac{x^2+2x-3}{x-1}\\
		&=&4.
	\end{eqnarray*}
	Vậy tiếp tuyến tại $x_0=1$ có hệ số góc là $k=4$.
	}
\end{vd}
\begin{vd}%[1D7H1-3]%[Dự án đề cương 3 Khối NH24-25-Dot 1-Viet Hoang Pham]
	Viết phương trình tiếp tuyến của
	\begin{enumerate}
		\item Đồ thị hàm số $y=x^2+2x-4$ $(C)$ tại điểm có hoành độ $x_0=0$.
		\item Đồ thị hàm số $y=x^3+1$ $(C)$ tại điểm có hoành độ $x_0=1$.
	\end{enumerate}
	\loigiai{
		\begin{enumerate}
			\item \begin{itemize}
				\item $y(0)=0^2+2\cdot 0-4=-4$
				\item $y'(0)=\lim\limits_{x\to 0}\dfrac{y(x)-y(0)}{x-0}=\lim\limits_{x\to 0}\dfrac{x^2+2x}{x}=\lim\limits_{x\to 0}(x+2)=2$.
				\item Phương trình tiếp tuyến là $y=y'(0)(x-0)+y(0)=2(x-0)-4=2x-4$.
			\end{itemize}
			\item \begin{itemize}
				\item $y(1)=1^3+1=2$
				\item $y'(1)=\lim\limits_{x\to 1}\dfrac{y(x)-y(1)}{x-1}=\lim\limits_{x\to 1}\dfrac{x^3-1}{x-1}=\lim\limits_{x\to 1}\left(x^2+x+1\right)=3$.
				\item Phương trình tiếp tuyến là $y=y'(1)(x-1)+y(1)=3(x-1)+2=3x-1$.
			\end{itemize}
		\end{enumerate}
	}
\end{vd}
\begin{vd}%[1D7V1-3]%[Dự án đề cương 3 Khối NH24-25-Dot 1-Viet Hoang Pham]
	Viết phương trình tiếp tuyến của đồ thị hàm số $y=2x^2-3$ $(C)$ tại điểm có tung độ $y_0=-1$.
	\loigiai{
		\begin{itemize}
			\item $y_0=-1\Rightarrow 2{x_0}^2-3=-1\Rightarrow \hoac{&x_0=1\\&x_0=-1}$
			\item Với $x_0=1$
				\begin{itemize}
				 	\item $y'(1)=\lim\limits_{x\to 1}\dfrac{y(x)-y(1)}{x-1}=\lim\limits_{x\to 1}\dfrac{2x^2-2}{x-1}=\lim\limits_{x\to 1}2\left(x+1\right)=4$.
				 	\item Phương trình tiếp tuyến là $y=y'(1)(x-1)+y(1)=4(x-1)-1=4x-5$.
				\end{itemize} 
			\item Với $x_0=-1$
				\begin{itemize}
				 	\item $y'(-1)=\lim\limits_{x\to -1}\dfrac{y(x)-y(-1)}{x-(-1)}=\lim\limits_{x\to -1}\dfrac{2x^2-2}{x+1}=\lim\limits_{x\to -1}2\left(x-1\right)=-4$.
				 	\item Phương trình tiếp tuyến là $y=y'(-1)(x+1)+y(-1)=-4(x+1)-1=-4x-5$.
				\end{itemize}
		\end{itemize}	
	}
\end{vd}
\begin{dang}{Ý nghĩa vật lý của đạo hàm}
	\begin{itemize}
		\item Nếu hàm số $s=f(t)$ biểu thị \textit{quãng đường} di chuyển của vật theo thời gian $t$ thì $f'(t_0)$ biểu thị tốc độ tức thời của chuyển động tại thời điểm $t_0$.
		\item Nếu hàm số $T=f(t)$ biểu thị \textit{nhiệt độ} $T$ theo thời gian $t$ thì $f'(t_0)$ biểu thị tốc độ thay đổi nhiệt độ theo thời gian tại thời điểm $t_0$.
	\end{itemize}
\end{dang}
\setcounter{vd}{0}
\begin{vd}%[1D7N1-4]%[Dự án đề cương 3 Khối NH24-25-Dot 1-Viet Hoang Pham]
	Một chất điểm chuyển động có phương trình chuyển động là $s=f(t)=t^2+4t+6$ ($t$ được tính bằng giây, $s$ được tính bằng mét).
	\begin{enumerate}
		\item Tính đạo hàm của hàm số $f(t)$ tại điểm $t_0$.
		\item Tính vận tốc tức thời của chuyển động tại thời điểm $t=5$.
	\end{enumerate}
	\loigiai{
	\begin{enumerate}
		\item Ta có 
		\begin{eqnarray*}
			f'(t_0)&=&\lim\limits_{t\to t_0}\dfrac{f(t)-f(t_0)}{t-t_0}\\
			&=&\lim\limits_{t\to t_0}\dfrac{t^2+4t+6-\left({t_0}^2+4t_0+6\right)}{t-t_0}\\
			&=&\lim\limits_{t\to t_0}\dfrac{t^2-{t_0}^2+4(t-t_0)}{t-t_0}\\
			&=&\lim\limits_{t\to t_0}\dfrac{(t-t_0)(t+t_0)+4(t-t_0)}{t-t_0}\\
			&=&\lim\limits_{t\to t_0}(t+t_0+4)\\
			&=&2t_0+4.
		\end{eqnarray*}
		\item $f'(5)=2\cdot 5+4=14$ (m/s).
	\end{enumerate}
	}
\end{vd}
\begin{vd}%[1D7H1-4]%[Dự án đề cương 3 Khối NH24-25-Dot 1-Viet Hoang Pham]
	Cho biết điện lượng trong một dây dẫn theo thời gian biểu thị bởi hàm số $Q=6t+5$ ($t$ được tính bằng giây, $Q$ được tính bằng Coulomb). Tính cường độ của dòng điện trong dây dẫn tại thời điểm $t=10$.
	\loigiai{
		Cường độ của dòng điện trong dây dẫn tại thời điểm $t=10$ là
		\begin{eqnarray*}
			I(10)=Q'(10)&=&\lim\limits_{t\to 10}\dfrac{Q(t)-Q(10)}{t-10}\\
			&=&\lim\limits_{t\to 10}\dfrac{6t+5-(6\cdot 10+5)}{t-10}\\
			&=&\lim\limits_{t\to 10}\dfrac{6t-60}{t-10}\\
			&=&\lim\limits_{t\to 10}6\\
			&=&6.
		\end{eqnarray*}
	}
\end{vd}
\begin{vd}%[1D7V1-4]%[Dự án đề cương 3 Khối NH24-25-Dot 1-Viet Hoang Pham]
	Một chất điểm chuyển động thẳng biến đổi đều với phương trình $s=s(t)=2t^2+t-1$ (m)
	\begin{enumerate}
		\item Tìm vận tốc tức thời của vật tại thời điểm $t=2$.
		\item Tìm vận tốc trung bình của chất điểm trong khoảng thời gian từ $t=0$ tới $t=2$.
	\end{enumerate}
	\loigiai{
		\begin{enumerate}
			\item Vận tốc tức thời của vật tại thời điểm $t=2$ là
			\begin{eqnarray*}
				v(2)=s'(2)&=&\lim\limits_{t\to 2}\dfrac{s(t)-s(2)}{t-2}\\
				&=&\lim\limits_{t\to 2}\dfrac{2t^2+t-1-\left(2\cdot 2^2+2-1\right)}{t-2}\\
				&=&\lim\limits_{t\to 2}\dfrac{2t^2+t-10}{t-2}\\
				&=&\lim\limits_{t\to 2}\dfrac{(t-2)(2t+5)}{t-2}\\
				&=&\lim\limits_{t\to 2}2t+5\\
				&=&9\text{ (m/s)}.
			\end{eqnarray*}
			\item Vận tốc trung bình của chất điểm trong khoảng thời gian từ $t=0$ tới $t=2$ là
				$$v_{tb}=\dfrac{s(2)-s(0)}{2-0}=\dfrac{2\cdot 2^2+2-1-\left(2\cdot 0^2+0-1\right)}{2}=5 \text{ (m/s)}.$$
		\end{enumerate}
	}
\end{vd}
\begin{dang}{Tìm tham số để hàm số có đạo hàm tại $x_0$
}
	Cho hàm số $y=f(x)\heva{&g(x;m)&\text{khi }x\neq a\\&h(x:m)&\text{khi }x=a}$ hoặc $y=\heva{&g(x;m)&\text{khi }x\geq a\\&h(x;m)&\text{khi }x<=>a}$.\\ 
	Tìm tham số $m$ để hàm số có đạo hàm tại $x=a$.
	\begin{enumerate}[label=\arabic*.]
		\item Xác định $\lim\limits_{x\to a^+}f(x)$, $\lim\limits_{x\to a^-}f(x)$, $f(a)$.
		\item Hàm số liên tục tại $x=a \Leftrightarrow \lim\limits_{x\to a^+}f(x)=\lim\limits_{x\to a^-}f(x)=f(a)\Rightarrow a=?$.
		\item Tính $f'(a^+)=\lim\limits_{x\to a^+}\dfrac{f(x)-f(a)}{x-a}$, $f'(a^-)=\lim\limits_{x\to a^-}\dfrac{f(x)-f(a)}{x-a}$.
		\item Hàm số có đạo hàm tại $x=a\Leftarrow f'(a^+)=f'(a^-)$.
	\end{enumerate}
\end{dang}
\setcounter{vd}{0}
\begin{vd}%[1D7V1-1]%[Dự án đề cương 3 Khối NH24-25-Dot 1-Viet Hoang Pham]
	Tìm $a$ để hàm số $f(x)=\heva{&\dfrac{x^2-1}{x-1}&\text{khi }x\neq 1\\ 
	&a&\text{khi }x=1}$ có đạo hàm tại $x=1$.
	\loigiai{
		\begin{itemize}
			\item 
			Ta có $\lim\limits_{x\to 1}f(x)=\lim\limits_{x\to 1}\dfrac{x^2-1}{x-1}=2$, $f(1)=a$.\\ 
			$\Rightarrow$ Hàm số liên tục tại $x=1$ khi $\Leftrightarrow \lim\limits_{x\to 1}f(x)=f(1)\Rightarrow a=2$.\\ 
			\item  
			Với $a=2$, ta có \\ 
			$f(x)-f(1)=\dfrac{x^2-1}{x-1}-2=\dfrac{x^2-1-2(x-1)}{x-1}=\dfrac{x^2-1-2x+2}{x-1}=\dfrac{x^2-2x+1}{x-1}=x-1$.\\ 
			$\dfrac{f(x)-f(1)}{x-1}=\dfrac{x-1}{x-1}=1$.\\ 
			$f'(1)=\lim\limits_{x\to 1}\dfrac{f(x)-f(1)}{x-1}=\lim\limits_{x\to 1}(x-1)=0$.\\ 
			Vậy hàm số $f(x)$ có đạo hàm tại $x=1$ khi $a=2$.
		\end{itemize}
	}
\end{vd}
\begin{vd}%[1D7V1-1]%[Dự án đề cương 3 Khối NH24-25-Dot 1-Viet Hoang Pham]
	Tìm $a$ để hàm số $f(x)=\heva{&2x^2-x+a&\text{khi }x\neq 0\\ 
	&x^2&\text{khi }x=0}$ có đạo hàm tại $x=0$.
	\loigiai{
		\begin{itemize}
			\item 
			Ta có $\lim\limits_{x\to 0}f(x)=\lim\limits_{x\to 0}\left(2x^2-x+a\right)=a$, $f(0)=0^2=0$.\\ 
			$\Rightarrow$ Hàm số liên tục tại $x=0$ khi $\Leftrightarrow \lim\limits_{x\to 0^+}f(x)=\lim\limits_{x\to 0^-}f(x)=f(0)\Rightarrow a=0$.\\ 
			\item  
			Với $a=0$, ta có \\ 
			$f(x)-f(0)=2x^2-x-0=2x^2-x$.\\ 
			$\dfrac{f(x)-f(0)}{x-0}=\dfrac{2x^2-x}{x}=2x-1$.\\ 
			$f'(0)=\lim\limits_{x\to 0}\dfrac{f(x)-f(0)}{x-0}=\lim\limits_{x\to 0}(2x-1)=-1$.
			Vậy hàm số $f(x)$ có đạo hàm tại $x=0$ khi $a=0$.
		\end{itemize}
	}
\end{vd}
\begin{vd}%[1D7C1-1]%[Dự án đề cương 3 Khối NH24-25-Dot 1-Viet Hoang Pham]
	Tìm $a$, $b$ để hàm số $f(x)=\heva{&\dfrac{1}{2}x^2&\text{khi }x\leq 1\\ 
	&ax+b&\text{khi }x>1}$ có đạo hàm tại $x=1$.
	\loigiai{
		\begin{itemize}
			\item 
			Ta có $\lim\limits_{x\to 1^-}f(x)=\lim\limits_{x\to 1^-}\left(\dfrac{1}{2}x^2\right)=\dfrac{1}{2}$, $\lim\limits_{x\to 1^+}f(x)=\lim\limits_{x\to 1^+}\left(ax+b\right)=a+b$, $f(1)=\dfrac{1}{2}\cdot 1^2=\dfrac{1}{2}$.\\ 
			$\Rightarrow$ Hàm số liên tục tại $x=1$ khi $\Leftrightarrow \lim\limits_{x\to 1^+}f(x)=\lim\limits_{x\to 1^-}f(x)=f(1)\Rightarrow a+b=\dfrac{1}{2}$.\\ 
			\item  
			Ta có \\ 
			Đạo hàm phải $f'\left(1^+\right)=\lim\limits_{x\to 1^+}\dfrac{f(x)-f(1)}{x-1}=\lim\limits_{x\to 1^+}\dfrac{ax+b-(a\cdot 1+b)}{x-1}=\lim\limits_{x\to 1^+}\dfrac{ax-a}{x-1}=a$.\\ 
			Đạo hàm trái $f'\left(1^-\right)=\lim\limits_{x\to 1^-}\dfrac{f(x)-f(1)}{x-1}=\lim\limits_{x\to 1^-}\dfrac{\dfrac{1}{2}x^2-\dfrac{1}{2}}{x-1}=\lim\limits_{x\to 1^-}\dfrac{1}{2}\left(x+1\right)=1$.\\
			Hàm số có đạo hàm tại $x=1\Leftrightarrow f'\left(1^+\right)=f'\left(1^-\right)\Leftrightarrow a=1$.\\ 
			Vậy hàm số $f(x)$ có đạo hàm tại $x=1$ khi $a=1\Rightarrow b=-\dfrac{1}{2}$.
		\end{itemize}
	}
\end{vd}
%-----------------------------------------------------------------------------
\subsection{BÀI TẬP RÈN LUYỆN}
\ind{PHẦN I.} \inden{Câu trắc nghiệm nhiều phương án lựa chọn. Mỗi câu hỏi học sinh chỉ chọn một phương án.}\\
\setcounter{ex}{0}
\Opensolutionfile{ans}[ans/1D7-Bai1-TN]%--Đặt tên 2D1-Bai1-Dang1-TN

\begin{ex}%[1D7N1-1]%[Dự án đề cương 3 Khối NH24-25-Dot 1-Viet Hoang Pham]
\textit{(Trích đề thi HKII - Trường THPT Đan Phượng - Hà Nội - Năm học 2024-2025)}\\
Cho hàm số $y=f(x)$ xác định trên $\mathbb{R}$ và
$\displaystyle \lim_{x \to 1} \frac{f(x) - f(1)}{x - 1} = -1$.
Kết quả nào sau đây đúng?
\choice
{\True $f'(1) = -1$}
{$f'(-1)=1$}
{$f(1)=-1$}
{$f'(-1)=-1$}
\loigiai{
Theo định nghĩa đạo hàm, ta có $x = 1$ là $\displaystyle \lim_{x \to 1} \frac{f(x) - f(1)}{x - 1} = f'(1)$, nên $f'(1) = -1$.
}
\end{ex}

\begin{ex}%[1D7N1-1]%[Dự án đề cương 3 Khối NH24-25-Dot 1-Viet Hoang Pham]
\textit{(Trích đề thi HKII - Trường THPT Chuyên Lê Quý Đôn - Ninh Thuận - Năm học 2024-2025)}\\
	Nếu hàm số $y=f(x)$ thỏa mãn $\lim\limits_{x\to 4}\dfrac{f(x)-f(4)}{x-4}=-9$ thì ta kết luận
	\choice
	{\True $f'(4)=-9$}
	{$f'\left(-9\right)=4$}
	{$f'\left(-4\right)=9$}
	{$f'(9)=-4$}
	\loigiai{
		Theo định nghĩa đạo hàm tại $x=4$ ta có\\
		$f'(4)=\displaystyle\lim\limits_{x\to 4}\dfrac{f(x)-f(4)}{x-4}=-9$.\\
		Vậy $f'(4)=-9$.	}
\end{ex}

\begin{ex}%[1D7N1-1]%[Dự án đề cương 3 Khối NH24-25-Dot 1-Viet Hoang Pham]
\textit{(Trích đề thi HKII - Trường THPT Nguyễn Hữu Cảnh - An Giang - Năm học 2024-2025)}\\
	Cho hàm số $y=f(x)$ có đạo hàm tại điểm $x=3$. Khẳng định nào sau đây đúng?
	\choice
	{$f'(3)=\lim\limits_{x \to 3} \dfrac{f(x)+f(3)}{x-3}$}
	{\True $f'(3)=\lim\limits_{x \to 3} \dfrac{f(x)-f(3)}{x-3}$}
	{$f'(3)=\lim\limits_{x \to 3} \dfrac{f(x)-f(3)}{x+3}$}
	{$f'(3)=\lim\limits_{x \to 3} \dfrac{f(x)+f(3)}{x+3}$}
	\loigiai{
		Theo định nghĩa đạo hàm ta có $f'(3)=\lim\limits_{x \to 3} \dfrac{f(x)-f(3)}{x-3}$.
	}
\end{ex}

\begin{ex}%[1D7N1-1]%[Dự án đề cương 3 Khối NH24-25-Dot 1-Viet Hoang Pham]
	Cho hàm số $y=f(x)$ xác định trên $\mathbb{R}$ thỏa mãn $\lim\limits_{x\to 3}\dfrac{f(x)-f(3)}{x-3}=2$. Kết quả đúng là
	\choice
	{$f'(2)=3$}
	{$f'(x)=2$}
	{$f'(x)=3$}
	{\True $f'(3)=2$}
	\loigiai{
		Theo định nghĩa đạo hàm của hàm số tại một điểm, ta có $\lim\limits_{x\to 3}\dfrac{f(x)-f(3)}{x-3}=f'(3)$.\\
		Do đó $f'(3)=2$.
	}
\end{ex}

\begin{ex}%[1D7N1-1]%[Dự án đề cương 3 Khối NH24-25-Dot 1-Viet Hoang Pham]
	\textit{(Trích đề thi HKII - Trường THPT Lương Thế Vinh - Hà Nội - Năm học 2024-2025)}\\
	Nếu hàm số $y=f(x)$ có đạo hàm trên $\mathbb{R}$ và thỏa mãn $f'(2)=5$ thì $\lim\limits_{x\to 2}\dfrac{f(x)-f(2)}{x-2}$ là
	\choice 
	{\True $5$}
	{$\dfrac{1}{5}$}
	{$\dfrac{1}{2}$}
	{$2$}
	\loigiai{
		Ta có $\lim\limits_{x\to 2}\dfrac{f(x)-f(2)}{x-2}=f'(2)=5$.
	}
\end{ex}

\begin{ex}%[1D7N1-1]%[Dự án đề cương 3 Khối NH24-25-Dot 1-Viet Hoang Pham]
	Cho hàm số $y=f(x)$ có đạo hàm tại điểm $x_0$. Tìm khẳng định đúng trong các khẳng định sau
	\choice
	{\True $f'(x_0)=\lim\limits_{x\to x_0}\dfrac{f(x)-f(x_0)}{x-x_0}$}
	{$f'(x_0)=\lim\limits_{x\to x_0}\dfrac{f(x)+f(x_0)}{x-x_0}$}
	{$f'(x_0)=\lim\limits_{x\to x_0}\dfrac{f(x)-f(x_0)}{x+x_0}$}
	{$f'(x_0)=\lim\limits_{x\to x_0}\dfrac{f(x)+f(x_0)}{x+x_0}$}
	\loigiai{
		Theo định nghĩa đạo hàm của hàm số tại một điểm, ta có $f'(x_0)=\lim\limits_{x\to x_0}\dfrac{f(x)-f(x_0)}{x-x_0}$.
	}
\end{ex}

\begin{ex}%[1D7N1-1]%[Dự án đề cương 3 Khối NH24-25-Dot 1-Viet Hoang Pham]
	Cho hàm số $y=f(x)$ có đạo hàm thỏa mãn $f'(6)=2$. Giá trị của biểu thức $\lim\limits_{x\to 6}\dfrac{f(x)-f(6)}{x-6}$ bằng
	\choice
	{$12$}
	{\True $2$}
	{$\dfrac{1}{3}$}
	{$\dfrac{1}{2}$}
	\loigiai{
		Ta có $\lim\limits_{x\to 6}\dfrac{f(x)-f(6)}{x-6} = f'(6) = 2$.
	}
\end{ex}
%TH
\begin{ex}%[1D7H1-1]%[Dự án đề cương 3 Khối NH24-25-Dot 1-Viet Hoang Pham]
\textit{(Trích đề thi HKII - Trường THPT Đan Phượng - Hà Nội - Năm học 2024-2025)}\\
Đạo hàm cấp hai của hàm số $f(x)=x^{3}-3x+2$ là
\choice
{$f''(x)=3x^2 - 3$}
{$f''(x)=6x-3$}
{\True $f''(x)=6x$}
{$f''(x)=3x-3$}
\loigiai{
Ta có $f'(x) = 3x^2 - 3$, suy ra $f''(x) = 6x$.
}
\end{ex}


\begin{ex}%[1D7H1-4%[Dự án đề cương 3 Khối NH24-25-Dot 1-Viet Hoang Pham]
\textit{(Trích đề thi HKII - Trường THPT Nguyễn Hữu Cảnh - An Giang - Năm học 2024-2025)}\\
	Một chất điểm chuyển động với phương trình $s=f(t)=t^2-t+2$ ($s$ tính bằng mét và $t$ tính bằng giây). Vận tốc tức thời của chuyển động tại thời điểm $t=2$ s là
	\choice
	{\True $3$ m/s}
	{$2$ m/s}
	{$4$ m/s}
	{$1$ m/s}
	\loigiai{
		Ta có $f(2)=2^2-2+2=4$.\\
		Vận tốc của chất điểm tại $t=2$ là
		\begin{eqnarray*}
			v(2)=f'(2)&=&\lim\limits_{t\to 2}\dfrac{t^2-t+2-4}{t-2}\\
			&=&\lim\limits_{t\to 2}(t+1)\\
			&=&3\text{ (m/s)}.
		\end{eqnarray*}
	}
\end{ex}

\begin{ex}%[1D7H1-1]%[Dự án đề cương 3 Khối NH24-25-Dot 1-Viet Hoang Pham]
\textit{(Trích đề thi HKII - Trường THPT Lương Thế Vinh - Hà Nội - Năm học 2024-2025)}
	Cho hàm số $f(x)=\dfrac{x+1}{x-2}$. Đạo hàm của hàm số tại $x=1$ là
	\choice
	{$f'(1)=-4$}
	{$f'(1)=-5$}
	{\True $f'(1)=-3$}
	{$f'(1)=-2$}
	\loigiai{
		Ta có $f(1)=-2$.\\
		$f'(1)=\lim\limits_{x\to 1}\dfrac{f(x)-f(1)}{x-1}=\lim\limits_{x\to 1}\dfrac{\dfrac{x+1}{x-2}+2}{x-1}=\lim\limits_{x\to 1}\dfrac{3x-3}{(x-1)(x-2)}=\lim\limits_{x\to 1}\dfrac{3}{x-2}=-3$.
	}
\end{ex}

%Câu 4
\begin{ex}%[1D7H1-1]%[Dự án đề cương 3 Khối NH24-25-Dot 1-Viet Hoang Pham]
	Cho hàm số $y=f(x)=\heva{&x^2+1, & x\ge 1 \\ &2x, & x<1 }$. Mệnh đề \textbf{sai} là
	\choice
	{$f'(1)=2$}
	{\True $f$ không có đạo hàm tại $x_0=1$}
	{$f'(0)=2$}
	{$f'(2)=4$}
	\loigiai{
	\begin{itemize}
		\item Ta có $\lim\limits_{x\to 1^+}f(x)=\lim\limits_{x\to 1^+}(x^2+1)=2$ và $\lim\limits_{x\to 1^-}f(x)=\lim\limits_{x\to 1^-}(2x)=2$.\\ 
		Vì $\lim\limits_{x\to 1^+}f(x)=\lim\limits_{x\to 1^-}f(x)=f(1)=2$ nên hàm số liên tục tại $x=1$.\\
		\item $\lim\limits_{x\to 1^+}\dfrac{f(x)-f(1)}{x-1}=\lim\limits_{x\to 1^+}\dfrac{x^2+1-2}{x-1}=\lim\limits_{x\to 1^+}(x+1)=2$.\\
		\item $\lim\limits_{x\to 1^-}\dfrac{f(x)-f(1)}{x-1}=\lim\limits_{x\to 1^-}\dfrac{2x-2}{x-1}=\lim\limits_{x\to 1^-}2=2$.\\
		Do đó $f$ có tồn tại đạo hàm tại $x_0=1$.
	\end{itemize}
	}
\end{ex}

%Câu 8
\begin{ex}%[1D7H1-2]%[Dự án đề cương 3 Khối NH24-25-Dot 1-Viet Hoang Pham]
	Tỉ số $\dfrac{\Delta y}{\Delta x}$ của hàm số $f(x)=2x(x-1)$ theo $x$ và $\Delta x$ là
	\choice
	{$4x+2\Delta x+2$}
	{$4x+2(\Delta x)^2+2$}
	{\True $4x+2\Delta x-2$}
	{$4x\cdot\Delta x+2(\Delta x)^2-2\Delta x$}
	\loigiai{
		Ta có $f(x)=2x^2-2x$.\\
		Số gia của hàm số là
		\begin{eqnarray*}
			\Delta y &=& f(x+\Delta x)-f(x) = [2(x+\Delta x)^2-2(x+\Delta x)]-(2x^2-2x)\\
			&=& 2(x^2+2x\Delta x+(\Delta x)^2)-2x-2\Delta x-2x^2+2x\\
			&=& 4x\Delta x+2(\Delta x)^2-2\Delta x.
		\end{eqnarray*}
		Tỉ số $\dfrac{\Delta y}{\Delta x}=\dfrac{4x\Delta x+2(\Delta x)^2-2\Delta x}{\Delta x}=4x+2\Delta x-2$.
	}
\end{ex}

%Câu 9
\begin{ex}%[1D7H1-1]%[Dự án đề cương 3 Khối NH24-25-Dot 1-Viet Hoang Pham]
	Tính đạo hàm của hàm số $f(x)=\heva{&\dfrac{\sqrt{x^3-2x^2+x+1}-1}{x-1}, & x\neq 1 \\ &0, & x=1 }$ tại điểm $x_0=1$.
	\choice
	{$\dfrac{1}{3}$}
	{$\dfrac{1}{5}$}
	{\True $\dfrac{1}{2}$}
	{$\dfrac{1}{4}$}
	\loigiai{
		Ta có 
		\begin{eqnarray*}
			f'(1)=\lim\limits_{x\to 1}\dfrac{f(x)-f(1)}{x-1}&=&\lim\limits_{x\to 1}\dfrac{\dfrac{\sqrt{x^3-2x^2+x+1}-1}{x-1}-0}{x-1}\\
			&=&\lim\limits_{x\to 1}\dfrac{\sqrt{x(x-1)^2+1}-1}{(x-1)^2}\\
			&=&\lim\limits_{x\to 1}\dfrac{x(x-1)^2+1-1}{(x-1)^2\left(\sqrt{x(x-1)^2+1}+1\right)}\\
			&=&\lim\limits_{x\to 1}\dfrac{x(x-1)^2}{(x-1)^2\left(\sqrt{x(x-1)^2+1}+1\right)}\\
			&=&\lim\limits_{x\to 1}\dfrac{x}{\sqrt{x(x-1)^2+1}+1}=\dfrac{1}{\sqrt{1}+1}=\dfrac{1}{2}.
		\end{eqnarray*}
	}
\end{ex}

%Câu 10
\begin{ex}%[1D7H1-1]%[Dự án đề cương 3 Khối NH24-25-Dot 1-Viet Hoang Pham]
	Tính đạo hàm của hàm số $f(x)=\dfrac{x^2+|x+1|}{x}$ tại $x=1$.
	\choice
	{$2$}
	{\True $0$}
	{$3$}
	{Đáp án khác}
	\loigiai{
		Với $x$ trong lân cận của $1$, ta có $x+1>0$ nên $|x+1|=x+1$.\\
		Do đó $f(x)=\dfrac{x^2+x+1}{x}=x+1+\dfrac{1}{x}$.\\
		Suy ra $f'(x)=1-\dfrac{1}{x^2}$.\\
		Vậy $f'(1)=1-\dfrac{1}{1^2}=0$.
	}
\end{ex}

%Câu 11
\begin{ex}%[1D7H1-1]%[Dự án đề cương 3 Khối NH24-25-Dot 1-Viet Hoang Pham]
	Cho hàm số $f(x)=x^2-x$, đạo hàm của hàm số ứng với số gia của đối số $x$ tại $x_0$ là
	\choice
	{$\lim\limits_{\Delta x\to 0}\left((\Delta x)^2+2x_0\Delta x-\Delta x\right)$}
	{\True $\lim\limits_{\Delta x\to 0}(\Delta x+2x_0-1)$}
	{$\lim\limits_{\Delta x\to 0}(\Delta x+2x_0+1)$}
	{$\lim\limits_{\Delta x\to 0}\left((\Delta x)^2+2x_0\Delta x+\Delta x\right)$}
	\loigiai{
		Ta có 
		\begin{eqnarray*}
			\Delta y = f(x_0+\Delta x)-f(x_0) &=& (x_0+\Delta x)^2-(x_0+\Delta x)-(x_0^2-x_0)\\
			&=&x_0^2+2x_0\Delta x+(\Delta x)^2-x_0-\Delta x-x_0^2+x_0\\
			&=&(\Delta x)^2+2x_0\Delta x-\Delta x.
		\end{eqnarray*}
			Đạo hàm của hàm số là\\ 
			$$f'(x_0)=\lim\limits_{\Delta x\to 0}\dfrac{\Delta y}{\Delta x}=\lim\limits_{\Delta x\to 0}\dfrac{(\Delta x)^2+2x_0\Delta x-\Delta x}{\Delta x}=\lim\limits_{\Delta x\to 0}(\Delta x+2x_0-1).$$
		
	}
\end{ex}

%Câu 12
\begin{ex}%[1D7H1-2]%[Dự án đề cương 3 Khối NH24-25-Dot 1-Viet Hoang Pham]
	Tính số gia của hàm số $y=\sqrt{2x+1}$ tại $x_0=1$.
	\choice
	{$\dfrac{2\Delta x}{\sqrt{3+2\Delta x}-\sqrt{3}}$}
	{\True $\dfrac{2\Delta x}{\sqrt{3+2\Delta x}+\sqrt{3}}$}
	{$\dfrac{2\Delta x}{\sqrt{3-2\Delta x}+\sqrt{3}}$}
	{Đáp án khác}
	\loigiai{
		Số gia của hàm số tại $x_0=1$ là
		\begin{eqnarray*}
			\Delta y = f(1+\Delta x)-f(1) &=& \sqrt{2(1+\Delta x)+1}-\sqrt{2\cdot 1+1}\\
			&=& \sqrt{3+2\Delta x}-\sqrt{3}\\
			&=& \dfrac{\left(\sqrt{3+2\Delta x}-\sqrt{3}\right)\left(\sqrt{3+2\Delta x}+\sqrt{3}\right)}{\sqrt{3+2\Delta x}+\sqrt{3}}\\
			&=& \dfrac{(3+2\Delta x)-3}{\sqrt{3+2\Delta x}+\sqrt{3}}=\dfrac{2\Delta x}{\sqrt{3+2\Delta x}+\sqrt{3}}.
		\end{eqnarray*}
	}
\end{ex}

%Câu 2
\begin{ex}%[1D7H1-1]%[Dự án đề cương 3 Khối NH24-25-Dot 1-Viet Hoang Pham]
	Cho hàm số $f(x)$ xác định bởi $f(x)=\heva{&\dfrac{\sqrt{4x^2+1}-1}{x}, &x\neq 0\\&0, &x=0}$. Giá trị $f'(0)$ bằng
	\choice
	{\True $2$}
	{$0$}
	{$\dfrac{1}{2}$}
	{Không tồn tại}
	\loigiai{
		Theo định nghĩa, ta có:
		\begin{eqnarray*}
			f'(0) &=& \lim_{x\to 0}\dfrac{f(x)-f(0)}{x-0} = \lim_{x\to 0}\dfrac{\dfrac{\sqrt{4x^2+1}-1}{x}-0}{x}\\
			&=& \lim_{x\to 0}\dfrac{\sqrt{4x^2+1}-1}{x^2}\\
			&=& \lim_{x\to 0}\dfrac{\left(\sqrt{4x^2+1}-1\right)\left(\sqrt{4x^2+1}+1\right)}{x^2\left(\sqrt{4x^2+1}+1\right)}\\
			&=& \lim_{x\to 0}\dfrac{4x^2}{x^2\left(\sqrt{4x^2+1}+1\right)}\\
			&=& \lim_{x\to 0}\dfrac{4}{\sqrt{4x^2+1}+1} = \dfrac{4}{\sqrt{1}+1}=2.
		\end{eqnarray*}
		Vậy $f'(0)=2$.
	}
\end{ex}

%Câu 8
\begin{ex}%[1D7H1-2]%[Dự án đề cương 3 Khối NH24-25-Dot 1-Viet Hoang Pham]
	Số gia của hàm số $f(x)=x^3$ ứng với $x_0=2$ và $\Delta x=1$ bằng bao nhiêu?
	\choice
	{$-19$}
	{$7$}
	{\True $19$}
	{$-7$}
	\loigiai{
		Số gia của hàm số là $\Delta y = f(x_0+\Delta x)-f(x_0) = f(2+1)-f(2)=f(3)-f(2)$.\\
		Ta có $f(3)=3^3=27$ và $f(2)=2^3=8$.\\
		Vậy $\Delta y=27-8=19$.
	}
\end{ex}
\begin{ex}%[1D7H1-2]%[Dự án đề cương 3 Khối NH24-25-Dot 1-Viet Hoang Pham]
	Số gia của hàm số $f(x)=\dfrac{x^2}{2}$ ứng với số gia $\Delta x$ của đối số $x$ tại $x_0=-1$ là
	\choice
	{\True $\dfrac{1}{2}(\Delta x)^2-\Delta x$}
	{$\dfrac{1}{2}[(\Delta x)^2-\Delta x]$}
	{$\dfrac{1}{2}[(\Delta x)^2+\Delta x]$}
	{$\dfrac{1}{2}(\Delta x)^2+\Delta x$}
	\loigiai{
		Số gia của hàm số là 
		\begin{eqnarray*}
			\Delta y = f(x_0+\Delta x)-f(x_0) &=& f(-1+\Delta x)-f(-1).\\
		&=& \dfrac{(-1+\Delta x)^2}{2}-\dfrac{(-1)^2}{2} \\
		&=& \dfrac{1-2\Delta x+(\Delta x)^2}{2}-\dfrac{1}{2}\\
		&=& \dfrac{(\Delta x)^2-2\Delta x}{2}\\ 
		&=& \dfrac{1}{2}(\Delta x)^2-\Delta x.
		\end{eqnarray*}
	}
\end{ex}
%Câu 11
\begin{ex}%[1D7H1-1]%[Dự án đề cương 3 Khối NH24-25-Dot 1-Viet Hoang Pham]
	Tính đạo hàm của hàm số $y=2x^2+x+1$ tại điểm $x=2$
	\choice
	{\True $9$}
	{$4$}
	{$7$}
	{$6$}
	\loigiai{
	Ta có $y(2)=2\cdot 2^2+2+1=11$.\\
	$y'(2)=\lim\limits_{x\to 2}\dfrac{2x^2+x+1-11}{x-2}=\lim\limits_{x\to 2}(2x+5)=9$.
	}
\end{ex}


\Closesolutionfile{ans}

\ind{PHẦN II.} \inden{Câu trắc nghiệm đúng sai. Trong mỗi ý a), b), c), d) ở mỗi câu, học sinh chọn đúng hoặc sai.}\\
\setcounter{ex}{0}
\Opensolutionfile{ans}[ans/2D1-Bai1-DS]%--Đặt tên 2D1-Bai1-DS

%Câu 1
\begin{ex}%[1D7H1-1]%[Dự án đề cương 3 Khối NH24-25-Dot 1-Viet Hoang Pham]
	Dùng định nghĩa để tính đạo hàm của hàm số $y=f(x)=x^2+2x$ tại điểm $x_0=1$. 
	\choiceTF
	{\True $f'(1)=\lim\limits_{x\to 1}\dfrac{f(x)-f(1)}{x-1}$}
	{\True $f'(1)=\lim\limits_{x\to 1}\dfrac{x^2+2x-3}{x-1}$}
	{$f'(1)=\lim\limits_{x\to 1}(x+4)$}
	{$f'(1)=a \Rightarrow a>5$}
	\loigiai{
		Ta có $f(1)=1^2+2(1)=3$.
		\begin{itemchoice}
			\itemch Theo định nghĩa đạo hàm tại một điểm, $f'(1)=\lim\limits_{x\to 1}\dfrac{f(x)-f(1)}{x-1}$.
			\itemch Thay $f(x)=x^2+2x$ và $f(1)=3$ vào công thức trên, ta được $f'(1)=\lim\limits_{x\to 1}\dfrac{x^2+2x-3}{x-1}$.
			\itemch Ta có $f'(1)=\lim\limits_{x\to 1}\dfrac{(x-1)(x+3)}{x-1}=\lim\limits_{x\to 1}(x+3)=4$. 
			\itemch Ta có $f'(1)=4\Rightarrow a=4$.
		\end{itemchoice}
	}
\end{ex}

%Câu 2
\begin{ex}%[1D7H1-1]%[Dự án đề cương 3 Khối NH24-25-Dot 1-Viet Hoang Pham]
	Dùng định nghĩa để tính đạo hàm của hàm số $f(x)=2x^3$.
	\choiceTF
	{\True Với bất kì $x_0$, $f'(x_0)=\lim\limits_{x\to x_0}\dfrac{f(x)-f(x_0)}{x-x_0}$}
	{$f'(1)=-6$}
	{\True $f'(0)=0$}
	{\True $f'(2)=24$}
	\loigiai{
		\begin{itemchoice}
			\itemch Với bất kì $x_0$, $f'(x_0)=\lim\limits_{x\to x_0}\dfrac{f(x)-f(x_0)}{x-x_0}$.
			\itemch Ta có $f(1)=2\cdot 1^3=2$.\\ 
			$f'(1)=\lim\limits_{x\to 1}\dfrac{2x^3-2}{x-1}=\lim\limits_{x\to 1}2\left(x^2+x+1\right)=6$.
			\itemch Ta có $f(0)=2\cdot 0^3=0$.\\ 
			$f'(0)=\lim\limits_{x\to 0}\dfrac{2x^3-0}{x-0}=\lim\limits_{x\to 0}2x^2=0$.
			\itemch Ta có $f(2)=2\cdot 2^3=16$.\\ 
			$f'(2)=\lim\limits_{x\to 2}\dfrac{2x^3-16}{x-2}=\lim\limits_{x\to 2}2\left(x^2+2x+4\right)=24$.
		\end{itemchoice}
	}
\end{ex}

%Câu 3
\begin{ex}%[1D7H1-3]%[Dự án đề cương 3 Khối NH24-25-Dot 1-Viet Hoang Pham]
	Cho hàm số $y=f(x)=2x^3$ có đồ thị $(C)$ và điểm $M$ thuộc $(C)$ có hoành độ $x_0=-1$. 
	\choiceTF
	{\True Hệ số góc của tiếp tuyến của $(C)$ tại điểm $M$ bằng $6$}
	{\True Phương trình tiếp tuyến của $(C)$ tại $M$ đi qua điểm $A(0;4)$}
	{Tiếp tuyến của $(C)$ tại $M$ cắt đường thẳng $d: y=3x$ tại điểm có hoành độ bằng $4$}
	{\True Tiếp tuyến của $(C)$ tại $M$ vuông góc với đường thẳng $d\colon y=-\dfrac{1}{6}x$}
	\loigiai{
		Tại $x_0=-1$, ta có $y_0=2(-1)^3=-2$ và $y'(-1)=\lim\limits_{x\to -1}\dfrac{2x^3-(-2)}{x-(-1)}=\lim\limits_{x\to -1}2\left(x^2-x+1\right)=6$.\\
		Phương trình tiếp tuyến của $(C)$ tại $M(-1;-2)$ là $y-y_0=y'(x_0)(x-x_0) \Leftrightarrow y-(-2)=6(x-(-1)) \Leftrightarrow y=6x+4$.
		\begin{itemchoice}
			\itemch Hệ số góc của tiếp tuyến tại $M$ là $k=y'(-1)=6$.
			\itemch Vì tọa độ điểm $A(0;4)$ thỏa mãn phương trình tiếp tuyến tại $M$ nên tiếp tuyến đi qua $A(0;4)$.
			\itemch Xét phương trình hoành độ giao điểm: $6x+4=3x \Leftrightarrow 3x=-4 \Leftrightarrow x=-\dfrac{4}{3}$.
			\itemch Tiếp tuyến có hệ số góc $k_1=6$. Đường thẳng $d$ có hệ số góc $k_2=-\dfrac{1}{6}$.\\
			Vì $k_1\cdot k_2=6\cdot\left(-\dfrac{1}{6}\right)=-1$ nên hai đường thẳng vuông góc.
		\end{itemchoice}
	}
\end{ex}

%Câu 4
\begin{ex}%[1D7H1-1]%[Dự án đề cương 3 Khối NH24-25-Dot 1-Viet Hoang Pham]
	\choiceTF
	{\True $y=x^3-x$ tại $x_0=1$ có $f'(1)=2$}
	{$y=\sqrt{x}$ tại $x_0=1$ có $f'(1)=1$}
	{\True $y=\dfrac{1}{x^2+1}$ tại $x_0=0$ có $f'(0)=0$}
	{$y=\dfrac{1}{x+1}$ tại $x_0=2$ có $f'(2)=\dfrac{1}{9}$}
	\loigiai{
		\begin{itemchoice}
			\itemch Ta có $y(1)=1^3-1=0$, $y'(1)=\lim\limits_{x\to 1}\dfrac{x^3-x-0}{x-1}=\lim\limits_{x\to 1}x\left(x+1\right)=2$.
			\itemch Ta có $y(1)=\sqrt{1}=1$, $y'(1)=\lim\limits_{x\to 1}\dfrac{\sqrt{x}-1}{x-1}=\lim\limits_{x\to 1}\dfrac{1}{1+\sqrt{x}}=\dfrac{1}{2}$.
			\itemch Ta có $y(0)=\dfrac{1}{0^2+1}=1$, $y'(0)=\lim\limits_{x\to 0}\dfrac{\dfrac{1}{x^2+1}-1}{x-0}=\lim\limits_{x\to 0}\dfrac{-x}{x^2+1}=0$.
			\itemch Ta có $y(2)=\dfrac{1}{2+1}=\dfrac{1}{3}$, $y'(2)=\lim\limits_{x\to 2}\dfrac{\dfrac{1}{x+1}-1}{x-2}=\lim\limits_{x\to 2}\dfrac{-1}{3(x+1)}=-\dfrac{1}{9}$.
		\end{itemchoice}
	}
\end{ex}
%Câu 5
\begin{ex}%[1D7V1-3]%[Dự án đề cương 3 Khối NH24-25-Dot 1-Viet Hoang Pham]
	Viết được phương trình tiếp tuyến của đồ thị hàm số $y=\dfrac{x+9}{x+1}$ biết tiếp tuyến vuông góc với đường thẳng $d\colon x-2y+2=0$. Khi đó
	\choiceTF
	{\True Có $2$ phương trình tiếp tuyến thỏa mãn}
	{\True Hệ số góc của tiếp tuyến bằng $-2$}
	{Các tiếp tuyến đi qua điểm $A(1;5)$}
	{\True Có một tiếp tuyến đi qua điểm $B(-1;-7)$}
	\loigiai{
		Đường thẳng $d\colon x-2y+2=0 \Leftrightarrow y=\dfrac{1}{2}x+1$ có hệ số góc $k_d=\dfrac{1}{2}$.\\
		Vì tiếp tuyến vuông góc với $d$ nên hệ số góc của tiếp tuyến là $k_{tt}=-2$.
		Ta có $y'=\dfrac{1(x+1)-1(x+9)}{(x+1)^2}=\dfrac{-8}{(x+1)^2}$.\\
		Gọi $M(x_0;y_0)$ là tiếp điểm.\\
		Ta có $y'(x_0)=\lim\limits_{x\to x_0}\dfrac{\dfrac{x+9}{x+1}-\dfrac{x_0+9}{x_0+1}}{x-x_0}=\lim\limits_{x\to x_0}\dfrac{-8}{(x+1)(x_0+1)}=-\dfrac{8}{(x_0+1)^2}$\\ 
		Ta có $k_{tt}=-2\Leftrightarrow y'(x_0)=-2 \Leftrightarrow \dfrac{-8}{(x_0+1)^2}=-2 \Leftrightarrow \heva{&x_0=1\\&x_0=-3}$.\\
		Có hai tiếp điểm nên có hai phương trình tiếp tuyến.
		\begin{itemize}
			\item Với $x_0=1 \Rightarrow y_0=5$. PTTT là $y-5=-2(x-1) \Leftrightarrow y=-2x+7$.
			\item Với $x_0=-3 \Rightarrow y_0=-3$. PTTT là $y-(-3)=-2(x-(-3)) \Leftrightarrow y=-2x-9$.
		\end{itemize}
		\begin{itemchoice}
			\itemch Có $2$ tiếp điểm nên có $2$ PTTT.
			\itemch Hệ số góc của tiếp tuyến là $k_{tt}=-2$. 
			\itemch $A(1;5)$ thỏa mãn TT $y=-2x+7$ nhưng không thỏa mãn TT $y=-2x-9$ nên chỉ có một tiếp tuyến đi qua $A$.
			\itemch $B(-1;-7)$ thỏa mãn TT $y=-2x-9$ nhưng không thỏa mãn TT $y=-2x+7$ nên chỉ có một tiếp tuyến đi qua $B$.
		\end{itemchoice}
	}
\end{ex}
\Closesolutionfile{ans}


\ind{PHẦN III.} \inden{Câu trắc nghiệm trả lời ngắn.}\\
\setcounter{ex}{0}
\Opensolutionfile{ans}[ans/2D1-Bai1-TLN]%--Đặt tên 2D1-Bai1-DS
\begin{ex}%[1D7V1-4]%[%[Dự án đề cương 3 Khối NH24-25-Dot 1-Viet Hoang Pham]
\textit{Trích đề HKII - Trường THPT Chuyên Lê Quý Đôn - Ninh Thuận - Năm học 2024-2025}
	Một chất điểm chuyển động theo phương trình $s(t)=\dfrac{1}{3}t^3+t^2-18t+4$, trong đó $t>0$ tính bằng giây, $s(t)$ tính bằng mét. Tính vận tốc (đơn vị m/s) của chất điểm tại thời điểm vật đi được quãng đường $4$ mét.
	\shortans[oly]{30}
	\loigiai{
		\begin{itemize}
			\item Thời điểm chất điểm đi được quãng đường $4$ mét là 
			\begin{eqnarray*}
				\dfrac{1}{3}t^3+t^2-18t+4=4&\Leftrightarrow&
				\dfrac{1}{3}t^3+t^2-18t=0\\
				&\Leftrightarrow&\hoac{&t=6&& \text{(nhận)}\\&t=-9&& \text{(loại)}\\&t=0&& \text{(loại).}}
			\end{eqnarray*}
			\item Ta có $s(6)=\dfrac{1}{3}6^3+6^2-18\cdot 6+4=4$.\\
			Vận tốc của chất điểm tại thời điểm vật đi được quãng đường $4$ mét là
			\[v(6)=s'(6)=\lim\limits_{t\to 6}\dfrac{\dfrac{1}{3}t^3+t^2-18t+4-4}{t-6}=\lim\limits_{t\to 6}\dfrac{1}{3}t(t+9)=30\text{ (m/s)}. \]
		\end{itemize}
	}
\end{ex}
%Câu 2
\begin{ex}%[1D7V1-3]%[Dự án đề cương 3 Khối NH24-25-Dot 1-Viet Hoang Pham]
	\textit{(Trích đề thi HKII - Trường THPT Lê Thánh Tông - TP.Hồ Chí Minh - Năm học 2024-2025)}
	Cho hàm số $y=x^3+3x^2+9x-1$ có đồ thị là $(C)$. Hệ số góc của tiếp tuyến của $(C)$ tại điểm có hoành độ $x=1$ bằng bao nhiêu?
	\shortans[oly]{15}
	\loigiai{
		Ta có $x=1\Rightarrow y=12$.
		Hệ số góc của tiếp tuyến tại điểm có hoành độ $x=1$ là
		$$k=f'(1)=\lim\limits_{x\to 1}\dfrac{f(x)-f(1)}{x-1}=\lim\limits_{x\to 1}\dfrac{x^3+3x^2+9x-1-12}{x-1}=\lim\limits_{x\to 1}\dfrac{(x-1)\left(x^2+x+13\right)}{x-1}=\lim\limits_{x\to 1}\left(x^2+x+13\right)=15.$$
	}
\end{ex}
%Câu 3
\begin{ex}%[1D7V1-4]%[Dự án đề cương 3 Khối NH24-25-Dot 1-Viet Hoang Pham]
	\textit{(Trích đề thi HKII - Trường THPT Ngô Gia Tự - Phú Yên - Năm học 2024-2025)}
	Vị trí của một vật chuyển động thẳng được cho bởi phương trình $s=f(t)=t^3-6t^2+9t$, trong đó $t$ tính bằng giây và $s$ tính bằng mét. Tính vận tốc của vật tại thời điểm $t=5$ giây.
	\shortans[oly]{24}
	\loigiai{
		Ta có $s(5)=20$.\\
		Vận tốc của chất điểm tại thời điểm $t=5$ giây là
		\[v(5)=s'(5)=\lim\limits_{t\to 5}\dfrac{t^3-6t^2+9t-20}{t-5}=\lim\limits_{t\to 5}\dfrac{(t-5)\left(t^2-t+4\right)}{t-5}=\lim\limits_{t\to 5}\left(t^2-t+4\right)=24\text{ (m/s)}. \]
	}
\end{ex}
%Câu 4 
\begin{ex}%[1D7V1-3]%[Dự án đề cương 3 Khối NH24-25-Dot 1-Viet Hoang Pham]
	Cho hàm số $y=f(x)=\dfrac{x+1}{3x}$ có đồ thị $(C)$. Phương trình tiếp tuyến của đồ thị hàm số $(C)$ tại giao điểm của $(C)$ với trục hoành có dạng $y=ax+b$. Khi đó, giá trị của biểu thức $a+b$ là bao nhiêu? (làm tròn đến hàng phần mười)
	\shortans[oly]{-0,7}
	\loigiai{
		Giao điểm của $(C)$ với trục hoành có $y=0 \Rightarrow \dfrac{x+1}{3x}=0 \Rightarrow x=-1$.
		Vậy tiếp điểm là $M(-1;0)$.\\
		Hệ số góc của tiếp tuyến là $a=y'(-1)=\lim\limits_{x\to -1}\dfrac{\dfrac{x+1}{3x}-0}{x-(-1)}=\lim\limits_{x\to -1}\dfrac{1}{3x}=-\dfrac{1}{3}$.\\
		Phương trình tiếp tuyến là $y-0=-\dfrac{1}{3}(x-(-1)) \Leftrightarrow y=-\dfrac{1}{3}x-\dfrac{1}{3}$.\\
		Suy ra $a=-\dfrac{1}{3}$ và $b=-\dfrac{1}{3}$.\\
		Giá trị của biểu thức $a+b=-\dfrac{1}{3}-\dfrac{1}{3}=-\dfrac{2}{3}\approx -0,7$.
	}
\end{ex}

%Câu 5
\begin{ex}%[1D7V1-1]%[Dự án đề cương 3 Khối NH24-25-Dot 1-Viet Hoang Pham]
Cho $f(x)=\heva{&\dfrac{\sqrt{x^2+1}-1}{x}&\text{khi }x\neq0\\ &0&\text{khi }x=0}$. Tính $f'(0)$.
\shortans[oly]{0,5}
\loigiai{
	Ta có $\lim\limits_{x\to 0}f(x)=\lim\limits_{x\to 0}\dfrac{\sqrt{x^2+1}-1}{x}=\lim\limits_{x\to 0}\dfrac{x^2}{x\left(\sqrt{x^2+1}+1\right)}=\lim\limits_{x\to 0}\dfrac{x}{\sqrt{x^2+1}+1}=0$ và $f(0)=0$.\\ 
	Vì $\lim\limits_{x\to 0}f(x)=f(0)=0$ nên hàm số liên tục tại $x=0$.\\ 
	Ta có $f'(0)=\lim\limits_{x\to 0}\dfrac{\dfrac{\sqrt{x^2+1}-1}{x}-0}{x-0}=\lim\limits_{x\to 0}\dfrac{\sqrt{x^2+1}-1}{x^2}=\lim\limits_{x\to 0}\dfrac{x^2}{x^2\left(\sqrt{x^2+1}+1\right)}=\lim\limits_{x\to 0}\dfrac{1}{\sqrt{x^2+1}+1}=0{,}5$.
}
\end{ex}
\Closesolutionfile{ans}


\ind{PHẦN IV.} \inden{Tự luận.}\\
\Opensolutionfile{ans}[ans/2D1-Bai1-TLN]
\setcounter{ex}{0}
%Câu 1
\begin{ex}%[1D7H1-1]%[Dự án đề cương 3 Khối NH24-25-Dot 1-Viet Hoang Pham]
	Tính đạo hàm của hàm số $f(x)=\sqrt{x+1}$ tại $x=1$.
	\loigiai{
		Ta có $f'(1)=\lim\limits_{x\to 1}\dfrac{f(x)-f(1)}{x-1}=\lim\limits_{x\to 1}\dfrac{\sqrt{x+1}-\sqrt{2}}{x-1}=\lim\limits_{x\to 1}\dfrac{x-1}{x-1\left(\sqrt{x+1}+\sqrt{2}\right)}=\lim\limits_{x\to 1}\dfrac{1}{\sqrt{x+1}+\sqrt{2}}=\dfrac{1}{2\sqrt{2}}$.
	}
\end{ex}
%câu 2
\begin{ex}%[1D7H1-1]%[Dự án đề cương 3 Khối NH24-25-Dot 1-Viet Hoang Pham]
	Tính đạo hàm của hàm số $f(x)=\heva{&x^2-3x&\text{khi }x\neq 1\\&-2&\text{khi }x=1}$ tại $x=1$.
	\loigiai{
		Ta có $\heva{&\lim\limits_{x\to 1}f(x)=\lim\limits_{x\to 1}\left(x^2-3x\right)=-2\\&f(1)=-2}\Rightarrow$ Hàm số liên tục tại $x=1$.\\ 
		Khi đó $f'(1)=\lim\limits_{x\to 1}\dfrac{f(x)-f(1)}{x-1}=\lim\limits_{x\to 1}\dfrac{x^2-3x+2}{x-1}=\lim\limits_{x\to 1}\dfrac{(x-1)(x-2)}{x-1}=\lim\limits_{x\to 1}(x-2)=-1$.
	}
\end{ex}
%Câu 3
\begin{ex}%[1D7V1-1]%[Dự án đề cương 3 Khối NH24-25-Dot 1-Viet Hoang Pham]
	Tìm $a$, $b$ để hàm số $f(x)=\heva{&ax^2+bx+1&\text{khi }x\geq 0\\&ax-b-1&\text{khi }x<0}$ có đạo hàm tại điểm $x=0$.
	\loigiai{
		Ta có $\heva{&f(0)=1\\ 
		&\lim\limits_{x\to 0^+}f(x)=\lim\limits_{x\to 0^+}\left(ax^2+bx+1\right)=1\\
		&\lim\limits_{x\to 0^-}f(x)=\lim\limits_{x\to 0^+}\left(ax-b-1\right)=-b-1}$\\ 
		Hàm số liên tục tại $x=0$ khi $\lim\limits_{x\to 0^+}f(x)=\lim\limits_{x\to 0^-}f(x)=f(0)\Rightarrow -b-1=1\Leftrightarrow b=-2$.\\ 
		Đạo hàm phải $f'(0^+)=\lim\limits_{x\to 0^+}\dfrac{f(x)-f(0)}{x-0}=\lim\limits_{x\to 0^+}\dfrac{ax^2-2x+1-1}{x}=\lim\limits_{x\to 0^+}(ax-2)=-2$.\\ 
		Đạo hàm trái $f'(0^-)=\lim\limits_{x\to 0^-}\dfrac{f(x)-f(0)}{x-0}=\lim\limits_{x\to 0^-}\dfrac{ax+1-1}{x}=\lim\limits_{x\to 0^-}a=a$.\\ 
		Hàm số có đạo hàm tại $x=0$ khi $f'(0^+)=f'(0^-)\Rightarrow a=-2$.\\ 
		Vậy với $a=-2$, $b=-2$ thì hàm số có đạo hàm tại $x=0$.
	}
\end{ex}
%Câu 4
\begin{ex}%[1D7V1-3]%[Dự án đề cương 3 Khối NH24-25-Dot 1-Viet Hoang Pham]
Cho hàm số $y=x^2+2x-4$. Viết phương trình tiếp tuyến của đồ thị hàm số, biết tiếp tuyến đó song song với đường thẳng $y=1-3x$.
\loigiai{
	Gọi $M(x_0;y_0)$ là tiếp điểm, ta có \\ 
	$k=f'(x_0)=\lim\limits_{x\to x_0}\dfrac{f(x)-f(x_0)}{x-x_0}=\lim\limits_{x\to x_0}\dfrac{x^2+2x-4-{x_0}^2-2x_0+4}{x-x_0}=\lim\limits_{x\to x_0}\dfrac{(x-x_0)(x+x_0+2)}{x-x_0}=\lim\limits_{x\to x_0}(x+x_0+2)=2x_0+2$.\\ 
	Vì tiếp tuyến song song với đường thẳng $y=1-3x$ nên $k=-3\Leftrightarrow 2x_0+2=-3\Leftrightarrow x_0=-\dfrac{5}{2}\Rightarrow y_0=-\dfrac{11}{4}$.\\ 
	Phương trình tiếp tuyến là $y=-3\left(x+\dfrac{5}{2}\right)-\dfrac{11}{4}\Leftrightarrow y=-3x-\dfrac{41}{4}$.
}
\end{ex}
%Câu 5
\begin{ex}%[1D7V1-3]%[Dự án đề cương 3 Khối NH24-25-Dot 1-Viet Hoang Pham]
Cho hàm số $y=\dfrac{x+1}{3x}$ có đồ thị $(C)$. Viết phương trình tiếp tuyến của đồ thị hàm số tại giao điểm của $(C)$ với đường thẳng $y=x+1$.
\loigiai{
Tại $x_0\in \mathbb{R}\setminus {0}$ tùy ý, ta có\\ 
$f'(x_0)=\lim\limits_{x\to x_0}\dfrac{f(x)-f(x_0)}{x-x_0}=\lim\limits_{x\to x_0}\dfrac{\dfrac{x+1}{3x}-\dfrac{x_0+1}{3x_0}}{x-x_0}=\lim\limits_{x\to x_0}\dfrac{-(x-x_0)}{3xx_0(x-x_0)}=\lim\limits_{x\to x_0}\dfrac{-1}{3xx_0}=-\dfrac{1}{3{x_0}^2}$.\\ 
Phương trình hoành độ giao điểm
$$\dfrac{x+1}{3x}=x+1\Leftrightarrow3x^2+2x-1=0\Leftrightarrow\hoac{&x=-1\Rightarrow y=0\\&x=\dfrac{1}{3}\Rightarrow y=\dfrac{4}{3}}.$$
\begin{itemize}
	\item Với $x_0=-1$, $y_0=0$ ta có\\
	$f'(-1)=-\dfrac{1}{3\cdot(-1)^2}=-\dfrac{1}{3}$.\\ 
	Phương trình tiếp tuyến là $y=-\dfrac{1}{3}\left(x+1\right)+0\Leftrightarrow y=-\dfrac{1}{3}x-\dfrac{1}{3}$.
	\item Với $x_0=\dfrac{1}{3}$, $y_0=\dfrac{4}{3}$ ta có\\
	$f'(\dfrac{1}{3})=-\dfrac{1}{3\cdot(\dfrac{1}{3})^2}=-3$.\\ 
	Phương trình tiếp tuyến là $y=-3\left(x-\dfrac{1}{3}\right)+\dfrac{4}{3}\Leftrightarrow y=-3x+\dfrac{7}{3}$.
\end{itemize}
}
\end{ex}
%Câu 6
\begin{ex}%[1D7V1-4]%[Dự án đề cương 3 Khối NH24-25-Dot 1-Viet Hoang Pham]
\textit{(Trích đề thi HKII - Trường THPT Lương Thế Vinh - Hà Nội - Năm học 2024-2025)}
	Một chất điểm chuyển động theo phương trình $s(t)=-5t^2+20t+4$, trong đó $t>0$, $t$ được tính bằng giây (s) và $s(t)$ được tính bằng mét (m). Tính vận tốc (đơn vị m/s) của chất điểm tại thời điểm $t=1$ giây.
	\loigiai{
	Ta có $s(1)=-5\cdot 1^2+20\cdot 1+4=19$ (m).\\ 
	Vận tốc của vật tại thời điểm $t=1$ giây là
	$$v(1)=s'(1)=\lim\limits_{t\to 1}\dfrac{s(t)-s(1)}{t-1}=\lim\limits_{t\to 1}\dfrac{-5t^2+20t+4-19}{t-1}=\lim\limits_{t\to 1}\dfrac{-5(t-1)(t-3)}{t-1}=\lim\limits_{t\to 1}-5(t-3)=10\text{ (m/s)}.$$
	}
\end{ex}
%Câu 7
\begin{ex}%[1D7V1-5]%[Dự án đề cương 3 Khối NH24-25-Dot 1-Viet Hoang Pham]
	Giả sử chi phí $C$ (USD) để sản xuất $Q$ máy vô tuyến là $C(Q)=Q^2+80Q+3\,500$.
	\begin{enumerate}
		\item Ta gọi chi phí biên là chi phí gia tăng để sản xuất thêm $1$ sản phẩm từ $Q$ sản phẩm lên $Q+1$ sản phẩm. Giả sử chi phí biên được xác định bởi hàm số $C'(Q)$. Tìm hàm chi phí biên.
		\item Tìm $C'(90)$ và giải thích ý nghĩa kết quả tìm được.
		\item Hãy tính chi phí sản xuất máy vô tuyến thứ $100$.
	\end{enumerate}
	\loigiai{
	\begin{enumerate}
		\item Hàm chi phí biên để sản xuất thêm $1$ sản phẩm từ $Q_0$ sản phẩm là 
		\begin{eqnarray*}
			C'(Q_0)&=&\lim\limits_{Q\to Q_0}\dfrac{C(Q)-C(Q_0)}{Q-Q_0}\\ 
			&=&\lim\limits_{Q\to Q_0}\dfrac{Q^2+80Q+3\,500-{Q_0}^2-80Q_0-3\,500}{Q-Q_0}\\ 
			&=&\lim\limits_{Q\to Q_0}\dfrac{(Q-Q_0)(Q+Q_0+80)}{Q-Q_0}\\ 
			&=&\lim\limits_{Q\to Q_0}(Q+Q_0+80)\\ 
			&=&2Q_0+80.
		\end{eqnarray*}
		\item $C'(90)=2\cdot 90+80=260$.\\ 
		Dựa vào kết quả trên, ta thấy chi phí gia tăng để sản xuất thêm $1$ sản phẩm từ $90$ lên $91$ sản phẩm là $260$ (USD).
		\item Chi phí sản xuất máy vô tuyến thứ $100$ là $C(100)=100^2+80\cdot 100+3\,500=21\,500$ (USD).
	\end{enumerate}
	}
\end{ex}
%Câu 8
\begin{ex}%[1D7C1-4]%[Dự án đề cương 3 Khối NH24-25-Dot 1-Viet Hoang Pham]
	Một chất điểm chuyển động theo phương trình $s=s(t)=-t^3+6t^2$, trong đó $t$ được tính bằng giây và $s$ được tính bằng mét. Tìm thời điểm để vận tốc của chất điểm đạt giá trị lớn nhất.
	\loigiai{
	Vận tốc tại thời điểm $t_0$ là
	\begin{eqnarray*}
		v(t_0)=s'(t_0)=\lim\limits_{t\to t_0}\dfrac{s(t)-s(t_0)}{t-t_0}&=&\lim\limits_{t\to t_0}\dfrac{-t^3+6t^2+{t_0}^3-6{t_0}^2}{t-t_0}\\ 
		&=&\lim\limits_{t\to t_0}\dfrac{(t-t_0)\left[6(t+t_0)-\left(t^2+tt_0+{t_0}^2\right)\right]}{t-t_0}\\ 
		&=&\lim\limits_{t\to t_0}\left[6(t+t_0)-\left(t^2+tt_0+{t_0}^2\right)\right]\\ 
		&=&12t_0-3{t_0}^2\text{ (m/s)}.
	\end{eqnarray*}
	Ta có $12t_0-3{t_0}^2=-3(t-2)^2+12\leq 12$.
	Vậy vận tốc đạt giá trị lớn nhất bằng $12$ khi $t-2=0\Leftrightarrow t=2$. 
	}
\end{ex}
%Câu 9
\begin{ex}%[1D7C1-4]%[Dự án đề cương 3 Khối NH24-25-Dot 1-Viet Hoang Pham]
	Một chất điểm chuyển động theo phương trình $s=s(t)=-\dfrac{1}{3}t^3+6t^2$, trong đó $t$ được tính bằng giây và $s$ được tính bằng mét. Tìm thời điểm để vận tốc của chất điểm đạt giá trị lớn nhất.
	\loigiai{
	Vận tốc tại thời điểm $t_0$ là
	\begin{eqnarray*}
		v(t_0)=s'(t_0)=\lim\limits_{t\to t_0}\dfrac{s(t)-s(t_0)}{t-t_0}&=&\lim\limits_{t\to t_0}\dfrac{-\dfrac{1}{3}t^3+6t^2+\dfrac{1}{3}{t_0}^3-6{t_0}^2}{t-t_0}\\ 
		&=&\lim\limits_{t\to t_0}\dfrac{(t-t_0)\left[6(t+t_0)-\dfrac{1}{3}\left(t^2+tt_0+{t_0}^2\right)\right]}{t-t_0}\\ 
		&=&\lim\limits_{t\to t_0}\left[6(t+t_0)-\dfrac{1}{3}\left(t^2+tt_0+{t_0}^2\right)\right]\\ 
		&=&12t_0-{t_0}^2\text{ (m/s)}.
	\end{eqnarray*}
	Ta có $12t_0-{t_0}^2=-(t-6)^2+36\leq 36$.
	Vậy vận tốc đạt giá trị lớn nhất bằng $36$ khi $t-6=0\Leftrightarrow t=6$. 
	}
\end{ex}
%Câu 10
\begin{ex}%[1D7C1-4]%[Dự án đề cương 3 Khối NH24-25-Dot 1-Viet Hoang Pham]
	Một chất điểm chuyển động theo phương trình $s=s(t)=-\dfrac{1}{2}t^3+9t^2$, trong đó $t$ được tính bằng giây và $s$ được tính bằng mét. Tìm thời điểm để vận tốc của chất điểm đạt giá trị lớn nhất.
	\loigiai{
	Vận tốc tại thời điểm $t_0$ là
	\begin{eqnarray*}
		v(t_0)=s'(t_0)=\lim\limits_{t\to t_0}\dfrac{s(t)-s(t_0)}{t-t_0}&=&\lim\limits_{t\to t_0}\dfrac{-\dfrac{1}{2}t^3+9t^2+\dfrac{1}{2}{t_0}^3-9{t_0}^2}{t-t_0}\\ 
		&=&\lim\limits_{t\to t_0}\dfrac{(t-t_0)\left[9(t+t_0)-\dfrac{1}{2}\left(t^2+tt_0+{t_0}^2\right)\right]}{t-t_0}\\ 
		&=&\lim\limits_{t\to t_0}\left[9(t+t_0)-\dfrac{1}{2}\left(t^2+tt_0+{t_0}^2\right)\right]\\ 
		&=&18t_0-\dfrac{3}{2}{t_0}^2\text{ (m/s)}.
	\end{eqnarray*}
	Ta có $18t_0-\dfrac{3}{2}{t_0}^2=-\dfrac{3}{2}(t-6)^2+54\leq 54$.
	Vậy vận tốc đạt giá trị lớn nhất bằng $54$ khi $t-6=0\Leftrightarrow t=6$. 
	}
\end{ex}
\Closesolutionfile{ans}

