\newpage
\def\thoigian{90}%--Thời gian
\de{Đề số 3}{Chương II. Dãy số. Cấp số cộng. Cấp số nhân}

\begin{center}
	\textbf{PHẦN 1 - CÂU TRẮC NGHIỆM BỐN PHƯƠNG ÁN}
\end{center}
\Opensolutionfile{ans}[ans/ans-TN-1D2-DE2]

	\begin{ex}%[1D2N1-3]%[Dự án D - đợt 4 NH24-25- Nguyễn Hữu Đức]
		Cho dãy số $\left(u_n\right)$, biết $u_n=\left(-1\right)^n\cdot 2n$. Mệnh đề nào sau đây \textbf{sai}?
		\choice
		{$u_1=-2$}
		{$u_2=4$}
		{$u_3=-6$}
		{\True $u_4=-8$}
		\loigiai{
			Thay trực tiếp vào kiểm tra, ta có
			\begin{eqnarray*}
				u_1&=&(-1)\cdot2=-2\\
				u_2&=&(-1)^2\cdot2\cdot2=4\\
				u_3&=&(-1)^3\cdot2\cdot3=-6\\
				u_4&=&(-1)^4\cdot2\cdot4=8.
			\end{eqnarray*}
		}
	\end{ex}
	\begin{ex}%[1D2H1-3]%[Dự án D - đợt 4 NH24-25- Nguyễn Hữu Đức]
		Cho dãy số $\left(u_n\right)$, biết $u_n=\left(-1\right)^n\cdot\dfrac{2^n}{n}$. Tìm số hạng $u_3$.
		\choice
		{$u_3=\dfrac{8}{3}$}
		{$u_3=2$}
		{$u_3=-2$}
		{\True $u_3=-\dfrac{8}{3}$}
		\loigiai{
			Thay trực tiếp vào kiểm tra, ta có
			\begin{center}
				$u_3=(-1)^3\cdot\dfrac{2^3}{3}=-\dfrac{8}{3}$.
			\end{center}
		}
	\end{ex}
	\begin{ex}%[1D2H1-3]%[Dự án D - đợt 4 NH24-25- Nguyễn Hữu Đức]
		Cho dãy số $\left(u_n\right)$, biết $u_n=\dfrac{2n+5}{5n-4}$. Số $\dfrac{7}{12}$ là số hạng thứ mấy của dãy số?
		\choice
		{\True $8$}
		{$6$}
		{$9$}
		{$10$}
		\loigiai{
			Ta có
			\allowdisplaybreaks
			\begin{eqnarray*}
				&&u_n=\dfrac{2n+5}{5n-4}\\
				&\Leftrightarrow&\dfrac{7}{12}=	\dfrac{2n+5}{5n-4}\\
				&\Leftrightarrow&24n+60=35n-28\\
				&\Leftrightarrow&11n=88\\
				&\Leftrightarrow&n=8.
			\end{eqnarray*}
			Vậy số $\dfrac{7}{12}$ là số hạng thứ $8$.
		}
	\end{ex}
	
	\begin{ex}%[1D2N2-2]%[Dự án D - đợt 4 NH24-25- Nguyễn Hữu Đức]
		Cho cấp số cộng $(u_n)$ có công sai $d=11$. Khẳng định nào sau đây đúng?
		\choice
		{$u_n=u_{n+1}+11$, $ \forall n \in \mathbb{N}^{*}$}
		{$u_{n+1}=u_n-10$, $ \forall n \in \mathbb{N}^{*}$}
		{$u_{n+1}=u_n \cdot 11$, $ \forall n \in \mathbb{N}^{*}$}
		{\True $u_{n+1}=u_n+11$, $ \forall n \in \mathbb{N}^{*}$}
		\loigiai{
			\begin{itemize}
				\item  Xét $(u_n)$ với $u_n=u_{n+1}+11 \Leftrightarrow u_{n+1}-u_n=-11, \forall n \in \mathbb{N}^{*}$.\\
				Do đó $(u_n)$ là cấp số cộng với công sai $d=-11$.
				\item  Xét $(u_n)$ với $u_{n+1}=u_n-10 \Leftrightarrow u_{n+1}-u_n=-10, \forall n \in \mathbb{N}^{*}$.\\
				Do đó $(u_n)$ là cấp số cộng với công sai $d=-10$.
				\item  Xét $(u_n)$ với $u_{n+1}=u_n \cdot 11, \forall n \in \mathbb{N}^{*}$.\\
				Do đó $(u_n)$ là cấp số nhân với công bội $q=11$.
				\item Xét $(u_n)$ với $u_{n+1}=u_n+11 \Leftrightarrow u_{n+1}-u_n=11, \forall n \in \mathbb{N}^{*} $.\\
				Do đó $(u_n)$ là cấp số cộng với công sai $d=11$.
			\end{itemize}
		}
	\end{ex}
	\begin{ex}%[1D2N2-6]%[Dự án D - đợt 4 NH24-25- Nguyễn Hữu Đức]
		Cho cấp số cộng $(u_n)$ có số hạng đầu $u_1$, số hạng tổng quát $u_n$, tổng của $n$ số hạng đầu $S_n$. Khẳng định nào sau đây đúng?
		\choice
		{$S_n=\dfrac{1}{2}\left(u_1+u_n\right)$, $ \forall n \in \mathbb{N}^{*}$}
		{$S_n=n\left(u_1+u_n\right)$, $ \forall n \in \mathbb{N}^{*}$}
		{$S_n=\dfrac{n}{2}\left(2u_1+u_n\right)$, $ \forall n \in \mathbb{N}^{*}$}
		{\True $S_n=\dfrac{n}{2}\left(u_1+u_n\right)$, $ \forall n \in \mathbb{N}^{*}$}
		\loigiai{
			Ta có công thức tính tổng của $n$ số hạng đầu $S_n$ là
			$$S_n=\dfrac{n}{2}\left(u_1+u_n\right), \forall n \in \mathbb{N}^{*}.$$
		}
	\end{ex}
	
	\begin{ex}%[1D2N2-3]%[Dự án D - đợt 4 NH24-25- Nguyễn Hữu Đức]
		Cho cấp số cộng $(u_n)$ với số hạng đầu $u_1=9$ và công sai $d=2$. Số hạng thứ hai của cấp số cộng đó bằng
		\choice
		{$\dfrac{9}{2}$}
		{\True $11$}
		{$7$}
		{$18$}
		\loigiai
		{
			Ta có $u_2=u_1+d=9+2=11$.
		}
	\end{ex}
	\begin{ex}%[1D2H2-4]%[Dự án D - đợt 4 NH24-25- Nguyễn Hữu Đức]
		Cho cấp số cộng có số hạng đầu $u_1=-\dfrac{1}{2}$, công sai $d=\dfrac{1}{2}$. Năm số hạng liên tiếp đầu tiên của cấp số này là
		\choice
		{$-\dfrac{1}{2}$; $0$; $1$; $\dfrac{1}{2};1$}
		{$-\dfrac{1}{2}$; $0$; $\dfrac{1}{2}$; $0$; $\dfrac{1}{2}$}
		{$\dfrac{1}{2}$; $1$; $\dfrac{3}{2}$; $2$; $\dfrac{5}{2}$}
		{\True $-\dfrac{1}{2}$; $0$; $\dfrac{1}{2}$; $1$; $\dfrac{3}{2}$}
		\loigiai{
			Ta dùng công thức tổng quát $u_n=u_1+(n-1)d=-\dfrac{1}{2}+(n-1)\dfrac{1}{2}=-1+\dfrac{n}{2}$ để tính các số hạng của một cấp số cộng.\\
			 Ta có $u_1=-\dfrac{1}{2}$, $u_2=0$, $u_3=\dfrac{1}{2}$, $u_4=1$, $u_5=\dfrac{3}{2}$.			
		}
	\end{ex}
	\begin{ex}%[1D2H2-4]%[Dự án D - đợt 4 NH24-25- Nguyễn Hữu Đức]
		Viết ba số hạng xen giữa các số $2$ và $22$ để được một cấp số cộng có năm số hạng.
		\choice
		{\True $7$; $12$; $17$}
		{$6$; $10$; $14$}
		{$8$; $13$; $18$}
		{$6$; $12$; $18$}
		\loigiai{
			Giữa $2$ và $22$ thêm ba số hạng nữa lập thành cấp số cộng, xem như ta có một cấp số cộng có năm số hạng với $u_1=2$; $u_5=22$, ta cần tìm $u_2$; $u_3$; $u_4$. Ta có
			\begin{eqnarray*}
				&&u_5=u_1+4d\\
				&\Leftrightarrow&d=\dfrac{u_5-u_1}{4}\\&\Leftrightarrow&d=5\\
				&\Rightarrow&\heva{&u_2=7\\&u_3=12\\&u_4=17.}
			\end{eqnarray*}
		}
	\end{ex}
	
	\begin{ex}%[1D2N3-2]%[Dự án D - đợt 4 NH24-25- Nguyễn Hữu Đức]
		Trong các dãy số $(u_n)$ cho bởi số hạng tổng quát sau, dãy số nào là một cấp số nhân
		\choice
		{\True 
			$u_n=\dfrac{1}{3^{n-2}}$}
		{$u_n=\dfrac{1}{3^{n}}-1$}
		{$u_n=n+\dfrac{1}{3}$}
		{$u_n=n^2-\dfrac{1}{3}$}
		\loigiai{
			Dãy $u_n=\dfrac{1}{3^{n-2}}=3\cdot\left(\dfrac{1}{3}\right)^{n-1}$ là cấp số nhân có $u_1=3$, $q=\dfrac{1}{3}$.
		}
	\end{ex}
	\begin{ex}%[1D2H3-2]%[Dự án D - đợt 4 NH24-25- Nguyễn Hữu Đức]
		Một cấp số nhân có $6$ số hạng, số hạng đầu bằng $2$ và số hạng thứ sáu bằng $486$. Tìm công bội $q$ của cấp số nhân đã cho.
		\choice
		{\True $q=3$}
		{$q=-3$}
		{$q=2$}
		{$q=-2$}
		\loigiai{
			Ta có $\heva{&u_1=2\\&u_6=486}\Rightarrow u_6=u_1q^5\Leftrightarrow 486=2\cdot q^5\Leftrightarrow q=3$.
		}
	\end{ex}
	\begin{ex}%[1D2H3-4]%[Dự án D - đợt 4 NH24-25- Nguyễn Hữu Đức]
		Cho cấp số nhân $\left(u_n\right)$ có $u_1=-3$ và $q=\dfrac{2}{3}$. Mệnh đề nào sau đây đúng?
		\choice
		{$u_5=-\dfrac{27}{16}$}
		{\True $u_5=-\dfrac{16}{27}$}
		{$u_5=\dfrac{16}{27}$}
		{$u_5=\dfrac{27}{16}$}
		\loigiai{
			Ta có $\heva{&u_1=-3\\&q=\dfrac{2}{3}}\Rightarrow u_5=u_1\cdot q^4=-3\cdot \left(\dfrac{2}{3}\right)^4=-\dfrac{16}{27}$.
		}
	\end{ex}

	\begin{ex}%[1D2H3-6]%[Dự án D - đợt 4 NH24-25- Nguyễn Hữu Đức]
		Cho cấp số nhân $\left(u_n\right)$ có $u_1=-3$ và $q=-2$. Tính tổng $10$ số hạng đầu tiên của cấp số nhân đã cho.
		\choice
		{$S_{10}=-511$}
		{$S_{10}=-1\,025$}
		{$S_{10}=1\,025$}
		{\True $S_{10}=1\,023$}
		\loigiai{
			Ta có $\heva{&u_1=-3\\&q=-2}\Rightarrow S_{10}=u_1\cdot \dfrac{q^{n}-1}{q-1}=(-3)\cdot \dfrac{(-2)^{10}-1}{-2-1}=1\,023$.
		}
	\end{ex}
	
\Closesolutionfile{ans}
%\begin{center}
%	\textbf{ĐÁP ÁN}
%	\inputansbox{10}{ans/ans-TN-1D2-DE2}	
%\end{center}
\begin{center}
	\textbf{PHẦN 2 - CÂU TRẮC NGHIỆM ĐÚNG SAI}
\end{center}
\setcounter{ex}{0}
\Opensolutionfile{ans}[ans/ans-DS-1D2-DE2]

\begin{ex}%[1D2H1-4]%[Dự án D - đợt 4 NH24-25- Nguyễn Hữu Đức]
	Cho dãy số $(u_n)$ được xác định bởi $u_1=1$ và $u_n=u_{n-1}+2n$ với mọi $n \geq 2$.
	\choiceTF
	{\True Ba số hạng đầu tiên của dãy số lần lượt là $1$; $5$; $11$}
	{Số hạng thứ tư của dãy là $17$}
	{\True Ta có $u_4> u_3$}
	{$(u_n)$ là một dãy số giảm}
	\loigiai{
		\begin{itemchoice}
			\itemch
			Ta có $u_1=1$, $u_2=u_1+2\cdot 2=5$, $u_3=u_2+2\cdot 3=11$.
			\itemch
			Số hạng thứ tư của dãy là $u_4=u_3+2\cdot 4=19$.
			\itemch
			Ta có $u_4=19$ và $u_3=11$ nên $u_4> u_3$.
			\itemch
			Ta có $u_n=u_{n-1}+2n\Rightarrow u_n-u_{n-1}=2n>0$ với mọi $n\in \mathbb{N}^*$ nên $(u_n)$ là một dãy số tăng.
		\end{itemchoice}
	}
\end{ex}
\begin{ex}%[1D2V2-7]%[Dự án D - đợt 4 NH24-25- Nguyễn Hữu Đức]
	Một nhà hát có $25$ hàng ghế với $16$ ghế ở hàng thứ nhất, $18$ ghế ở hàng thứ hai, $20$ ghế ở hàng thứ ba và cứ tiếp tục theo quy luật đó, tức là hàng sau nhiều hơn hàng liền trước đó $2$ ghế. Gọi $u_n$ là số ghế ở hàng thứ $n$ ($n \in \mathbb{N}^{*}$, $1 \le n \le 25$).
	\choiceTF
	{\True $u_2=18$}
	{\True Dãy số $(u_n)$ là cấp số cộng có công sai $d=2$}
	{Số ghế ở hàng thứ $20$ nhỏ hơn $54$ ghế}
	{Tổng số ghế trong nhà hát nhiều hơn $1\,000$ ghế}
	\loigiai{
		\begin{itemchoice}
			\itemch Vì $u_n$ là số ghế ở hàng thứ $n$ ($n \in \mathbb{N}^{*}$, $1 \le n \le 25$) nên $u_2$ là số ghế ở hàng thứ $2$. Do đó $u_2=18$.
			\itemch  Do số ghế ở hàng sau nhiều hơn hàng liền trước đó $2$ ghế nên dãy số $(u_n)$ lập thành cấp số cộng với số hạng đầu $u_1=16$ và công sai $d=2$.
			\itemch Số ghế ở hàng thứ $20$ là $u_{20}=u_1+19d=16+19\cdot 2=54$.
			\itemch Tổng số ghế trong nhà hát là
			$$S_{25}=u_1+u_2+\cdots+u_{25}=\dfrac{n(2u_1+24d)}{2}=\dfrac{25(2\cdot 16+24\cdot 2)}{2}=1\, 000.$$
		\end{itemchoice}
	}
\end{ex}
\Closesolutionfile{ans}
%\begin{center}
%\textbf{ĐÁP ÁN}
%\inputansbox[2]{2}{ans/ans-DS-1D2-DE2}
%\end{center}
\begin{center}
	\textbf{PHẦN 3 - CÂU TRẮC NGHIỆM TRẢ LỜI NGẮN}
\end{center}
\setcounter{ex}{0}
\Opensolutionfile{ans}[ans/ans-KQ-1D2-DE2]
\begin{ex}%[1D2V2-7]%[Dự án D - đợt 4 NH24-25- Nguyễn Hữu Đức]
	Một CLB Toán học tổ chức trò chơi sử dụng đồng xu để xếp thành một kim tự tháp. Nhóm đã xử dụng $23\,520$ đồng tiền xu để xếp một mô hình kim tự tháp. Biết rằng tầng dưới cùng có $3\,020$ đồng xu và cứ lên thêm một tầng thì số đồng xu giảm đi $120$ đồng. Hỏi mô hình kim tự tháp này có tất cả bao nhiêu tầng?
	\shortans{$42$}
	\loigiai{
		Vì tầng dưới cùng của mô hình kim tự tháp có $3\, 020$ đồng xu và cứ lên thêm một tầng thì số đồng xu giảm đi $120$ đồng nên ta có một cấp số cộng với số hạng đầu $a_1 = 3\,020$ công sai $d = -120$.\\
		Gọi $n$ là số tầng kim tự tháp nên $n \in \mathbb{N}^*$.\\
		Áp dụng công thức tổng của cấp số cộng
		\[
		S_n = \frac{n}{2} \cdot \left[2a_1 + (n - 1)d\right]
		\]
%		\[
%		23\,520 = \frac{n}{2} \times \left[2 \cdot 3\,020 + (n - 1) \cdot (-120)\right]
%		\]
		\begin{align*}
			S_n = 23\,520 &\Leftrightarrow 3\,020n+\dfrac{n(n-1)(-120)}{2}= 23\,520 \\
			&\Leftrightarrow -60n^2 + 3\,080n - 23\,520 = 0\\
			&\Leftrightarrow \hoac{&n = 42 \\ &n = \dfrac{28}{3}.}
		\end{align*}
		Do $n \in \mathbb{N}^*$ nên $n = 42$.\\
		Vậy kim tự tháp có $42$ tầng.
	}
\end{ex}

\begin{ex}%[1D2V2-7]%[Dự án D - đợt 4 NH24-25- Nguyễn Hữu Đức]
	Một nhà thi đấu có $20$ hàng ghế dành cho khán giả. Hàng thứ nhất có $20$ ghế, hàng thứ hai có $21$ ghế, hàng thứ ba có $22$ ghế,$\ldots$. Cứ như thế, số ghế ở hàng sau nhiều hơn số ghế ở hàng trước là $1$ ghế. Trong một giải thi đấu, ban tổ chức đã bán được hết số vé phát ra và số tiền thu được từ bán vé là $70\,800\,000$ đồng. Tính giá tiền của mỗi vé (đơn vị: nghìn đồng), biết số vé bán ra bằng số ghế dành cho khán giả của nhà thi đấu và các vé là đồng giá.
	\par \shortans[4]{120}
	\loigiai{Các hàng ghế lập thành cấp số cộng có $u_1=20$ và $d=1$.\\
		$S_{20}=\dfrac{20}{2}\cdot\left(2\cdot 20+19\cdot 1\right)=590$ (ghế).\\
		Giá tiền của một vé là $\dfrac{70\,800}{590}=120$ nghìn đồng.}
\end{ex}
\begin{ex}%[1D2H2-3]%[Dự án D - đợt 4 NH24-25- Nguyễn Hữu Đức]
	Người ta trồng $3\,003$ cây theo một hình tam giác quy luật như sau: hàng thứ nhất trồng $1$ cây, hàng thứ hai trồng $2$ cây, hàng thứ ba trồng $3$ cây,\ldots Hỏi có tất cả bao nhiêu hàng cây?
	\shortans[4]{77}
	\loigiai{
		Gọi $u_n$ là số cây trồng ở hàng thứ $n$.\\
		Theo đề bài ta có $u_1=1$; $u_2=2$; $u_3=3$, $\ldots$.\\
		Dễ thấy $u_n$ là cấp số cộng với số hạng đầu là $u_1=1$, công sai $d=1$.\\
		Giả sử với $3\,003$ cây trồng được $n_o$ hàng. Khi đó ta có
		$$\begin{aligned}
			\dfrac{\left[ 2u_1+(n_o-1)d\right] n_o}{2}=3\,003 
			\Leftrightarrow  n^2_o+n_o-6\,006=0 
			\Leftrightarrow \hoac{&n_o=77 \ (\text{thỏa})\\&n_o=-78 \ (\text{loại}).}
		\end{aligned}$$
		Vậy có $77$ hàng được trồng.
	}
\end{ex}
\begin{ex}%[1D6V4-6][Dự án D - đợt 4 NH24-25- Nguyễn Hữu Đức]
	Anh An kí hợp đồng lao động $5$ năm và được trả lương như sau: Tháng thứ nhất tiền lương là $16$ triệu đồng, kể từ tháng thứ hai trở đi mỗi tháng tiền lương được tăng lên $1$\%. Anh An đã lên kế hoạch quản lý tài chính cá nhân như sau: Ngay từ tháng lương đầu tiên, hàng tháng chuyển tiết kiệm vào tài
	khoản đúng bằng $30$\% tiền lương của tháng đó, biết lãi suất trong tài khoản nhỏ coi như không có. Hỏi anh An cần tối thiểu bao nhiêu tháng kể từ tháng lương đầu tiên để tiền tiết kiệm này đủ mua chiếc xe máy giá $120$ triệu đồng mà không phải vay mượn ai?
	\shortans{$23$}
	\loigiai{
		Tiền lương ở tháng thứ nhất là $L_1=16$ (triệu đồng).\\
		Tiền lương ở tháng thứ hai là $L_2=L_1+0{,}01\cdot L_1=16+16\cdot 0{,}01=16\cdot 1{,}01$ (triệu đồng).\\
		Tiền lương ở tháng thứ ba là $L_3=L_2+0{,}01\cdot L_2=(16\cdot 1{,}01)+16\cdot 1{,}01\cdot 0{,}01=16\cdot 1{,}01^{2}$ (triệu đồng).\\
		$\cdots$\\
		Tiền lương ở tháng thứ $n$ là $L_n=16\cdot 1{,}01^{n-1}$ (triệu đồng).\\
		Tiền tiết kiệm ở tháng thứ $n$ là $T_n=0{,}3\cdot L_n=0{,}3\cdot 16\cdot 1{,}01^{n-1}=4{,}8\cdot 1{,}01^{n-1}$ (triệu đồng).\\
		Tổng số tiền tiết kiệm sau $n$ tháng là 
		\begin{center}
			$S_n=T_1+T_2+\cdots+T_n=4{,}8\cdot \dfrac{(1{,}01)^n-1}{1{,}01-1}=480\left[(1{,}01)^n-1\right]$ (triệu đồng).
		\end{center}
		Ta có 
		\begin{eqnarray*}
			S_n\ge 120&\Leftrightarrow& 480\left[(1{,}01)^n-1\right]\ge 120\\
			&\Leftrightarrow& (1{,}01)^n\ge 1{,}25\\
			&\Leftrightarrow& n\ge \log_{1{,}01}1{,}25\\
			&\Rightarrow& n\ge 22{,}4.
		\end{eqnarray*}
		Vậy anh An cần tối thiểu $23$ tháng để có đủ tiền mua xe.
	}
\end{ex}

\Closesolutionfile{ans}
\begin{center}
	\textbf{PHẦN 4 - TỰ LUẬN}
\end{center}
\setcounter{ex}{0}
\begin{ex}%[1D2H2-7]%[Dự án D - đợt 4 NH24-25- Nguyễn Hữu Đức]
	Một hội trường lớn có $27$ ghế ở hàng đầu tiên, $29$ ghế ở hàng thứ hai, $31$ ghế ở hàng thứ ba và cứ tiếp tục theo quy luật như vậy (số ghế ở hàng ghế sau luôn nhiều hơn so với hàng ghế kề ngay sát phía trước nó là $2$ ghế). Hỏi để xếp hết $1\,275$ ghế vào hội trường thì hàng cuối cùng có bao nhiêu ghế?
	\loigiai{
		Dễ thấy số ghế ở mỗi hàng là một cấp số cộng với số hạng đầu tiên $u_1=27$ và công sai $d=2$.\\
		Gọi $S_n$ là tổng $n$ số hạng đầu tiên của cấp số cộng này. Ta có
		$$
		S_n = \dfrac{n\left[2u_1+(n-1)d\right]}{2} = 1\,275.
		$$
		Từ đó, ta được phương trình
		\begin{align*}
			\dfrac{n\left[54+(n-1) \cdot 2\right]}{2} &= 1\,275 \\
			n^2+26n-1\,275 &= 0 \\
			n=25~\text{(nhận)} \quad &\text{hoặc} \quad n=-51~\text{(loại).}
		\end{align*}
		Khi đó
		$$
		u_{25} = u_1+24d = 27 + 24 \cdot 2 = 75.
		$$
		Vậy hàng cuối có $75$ ghế.
	}
\end{ex}
\begin{ex}%[1D2V3-3]%[Dự án D - đợt 4 NH24-25- Nguyễn Hữu Đức]
	Một công ty mua một cái máy với giá $1$ tỉ $800$ triệu đồng. Công ty nhận thấy, trong vòng $5$ năm đầu, tốc độ khấu hao là $25\%$ trên một năm (tức là sau mỗi một năm, giá trị còn lại của chiếc máy bằng $75\%$ giá trị của năm trước đó). Sau $5$ năm, giá trị của cái máy đó còn khoảng bao nhiêu triệu đồng (làm tròn kết quả đến hàng đơn vị)?

	\loigiai{Sau một năm giá trị còn lại của một cái máy là $u_1=1\,800\cdot \dfrac{3}{4}$.\\
		Quy luật này xác định một cấp số nhân có $q=\dfrac{3}{4}$.\\
		Sau $5$ năm, giá trị của cái máy còn lại là $u_5=1\,800\cdot \left( \dfrac{3}{4}\right)^5=\dfrac{54\,675}{128}\approx 427$ triệu đồng.
	}
\end{ex}
\begin{ex}%[1D3V1-5]%[Dự án D - đợt 4 NH24-25- Nguyễn Hữu Đức]
	Cho tam giác $T_1$ có diện tích bằng $2$. Giả sử tam giác $T_2$ đồng dạng với tam giác $T_1$, tam giác $T_3$ đồng dạng với tam giác $T_2 \ldots$ tam giác $T_n$ đồng dạng với tam giác $T_{n-1}$ với tỉ số đồng dạng là $\dfrac{1}{3}$. Khi $n$ tiến tới vô cùng thì tổng diện tích của tất cả các tam giác này bằng bao nhiêu? (làm tròn đến hàng phần trăm).
	\loigiai{
		Gọi diện tích các tam giác $T_1, T_2, \ldots, T_{n-1}, T_n$ lần lượt là $S_1, S_2, \ldots, S_{n-1}, S_n$.\\		
		Vì tam giác $T_n$ đồng dạng với tam giác $T_{n-1}$ với tỉ số đồng dạng $\dfrac{1}{3}$ nên diện tích tam giác $T_n$ bằng $\dfrac{1}{9}$ diện tích tam giác $T_{n-1}$, hay
		$$ S_n = \dfrac{1}{9}S_{n-1}. $$
		Vậy $S_1, S_2, \ldots, S_{n-1}, S_n, \ldots$ lập thành một cấp số nhân lùi vô hạn có số hạng đầu $S_1 = 2$ và công bội $q = \dfrac{1}{9}.$\\
		Khi đó, tổng diện tích của tất cả các tam giác nếu $n$ tiến tới vô cùng là
		$$ S = S_1 + S_2 + \dots + S_{n-1} + S_n + \ldots =\dfrac{2}{1 - \frac{1}{9}} = \dfrac{9}{4}=2{,}25.$$
	}
\end{ex}

