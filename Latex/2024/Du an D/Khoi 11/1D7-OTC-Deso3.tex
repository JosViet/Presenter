\newpage
\def\thoigian{90}%--Thời gian
\de{Đề số 3}{Chương VII. Đạo hàm}


\begin{center}
	\textbf{PHẦN 1 - CÂU TRẮC NGHIỆM BỐN PHƯƠNG ÁN}
\end{center}
\Opensolutionfile{ans}[ans/ans-TN-ONTAPCHUONG-VII-DE3]
%Cau1
\begin{ex}%[Dự án D - Dot 4 - NH25-26-Nắng Đông]
	Cho hàm số $y=f(x)$ xác định trên khoảng $K$ và $4\in K$. Khẳng định nào trong bốn khẳng định bên dưới là đúng?
	\choice
	{\True $f'(4)=\lim\limits_{x\to 4} \dfrac{f(x)-f(4)}{x-4}$}
	{$f'(4)=\lim\limits_{x\to 4} \dfrac{f(x)-f(4)}{x+4}$}
	{$f'(4)=\lim\limits_{x\to 4} \dfrac{f(x)+f(4)}{x+4}$}
	{$f'(4)=\lim\limits_{x\to 4} \dfrac{f(x)+f(4)}{x-4}$}
	\loigiai{
		Theo định nghĩa thì $f'(4)=\lim\limits_{x\to 4} \dfrac{f(x)-f(4)}{x-4}$.
	}
\end{ex}
%Cau2
\begin{ex}%[Dự án D - Dot 4 - NH25-26-Nắng Đông]
	Tiếp tuyến của đồ thị $(C)$ của hàm số $y=f(x)$ tại điểm $M_0(x_0;y_0)$ thuộc $(C)$ có phương trình là
	\choice
	{$y=f'\left(x_0\right)\left(x-x_0\right)-y_0$}
	{\True $y=f'\left(x_0\right)\left(x-x_0\right)+y_0$}
	{$y=f'\left(x_0\right)\left(x+x_0\right)-y_0$}
	{$y=f'\left(x\right)\left(x-x_0\right)-y_0$}
	\loigiai{
		Phương trình tiếp tuyến của đồ thị hàm số $y=f(x)$ tại điểm $M_0(x_0;y_0)$ là $$y=f'\left(x_0\right)\left(x-x_0\right)+y_0.$$
	}
\end{ex}

%Cau3
\begin{ex}%[Dự án D - Dot 4 - NH25-26-Nắng Đông]
	Với $x\in \mathbb{R}$, hàm số $y=5x^7$ có đạo hàm là
	\choice
	{$y'=5x^6$}
	{$y'=35x^7$}
	{\True $y'=35x^6$}
	{$y'=7x^5$}
	\loigiai{
		Ta có $y=5x^7 \Rightarrow y'=35x^6$.
	}
\end{ex}

%Cau4
\begin{ex}%[Dự án D - Dot 4 - NH25-26-Nắng Đông]
	Với $x>0$, đạo hàm của hàm số $y=\log x$ là 
	\choice
	{\True $y'=\dfrac{1}{x\cdot \ln 10}$}
	{$y'=\dfrac{1}{x}$}
	{$y'=\dfrac{1}{x^2}$}
	{$y'=-\dfrac{1}{x}$}
	\loigiai{
		Với $x>0$, thì $y=\log x \Rightarrow y'=\dfrac{1}{x\cdot \ln 10}$.
	}
\end{ex}

%Cau5
\begin{ex}%[Dự án D - Dot 4 - NH25-26-Nắng Đông]
	Tiếp tuyến của đồ thị hàm số $y=x^2-1$ tại điểm $M(-2; 3)$ có hệ số góc là
	\choice
	{$k=-6$}
	{$k=2$}
	{$k=-3$}
	{\True $k=-4$}
	\loigiai{
		Ta có $y'=2x$.\\
		Hệ số góc $k$ của tiếp tuyến của đồ thị hàm số $y=x^2-1$ tại điểm $M(-2; 3)$ là $$y'(-2)=2\cdot (-2)=-4.$$	
	}
\end{ex}


%Cau6
\begin{ex}%[Dự án D - Dot 4 - NH25-26-Nắng Đông]
	Với $x\in \mathbb{R}$, đạo hàm của hàm số $y=2x^3-3x^2+1$ là
	\choice
	{\True $y'=6x^2-6x$}
	{$y'=6x^2-6x+1$}
	{$y'=3x^2-2x$}
	{$y'=x^2-x$}
	\loigiai{
		Ta có $y=2x^3-3x^2+1\Rightarrow y'=6x^2-6x$.
	}
\end{ex}


%Cau7
\begin{ex}%[Dự án D - Dot 4 - NH25-26-Nắng Đông]
	Khẳng định nào sau đây là khẳng định sai?
	\choice
	{$(\sqrt{x})'=\dfrac{1}{2\sqrt{x}}$, với $x>0$}
	{$(\sin x)'=\cos x$}
	{\True $(\cos x)'=\sin x$}
	{$(\mathrm{e}^x)'=\mathrm{e}^x$}
	\loigiai{
		Ta có $(\cos x)'=-\sin x$.
	}
\end{ex}


%Cau8
\begin{ex}%[Dự án D - Dot 4 - NH25-26-Nắng Đông]
	Đạo hàm của hàm số $y=2\sqrt{x}-\ln x$ trên khoảng $(0;+\infty)$ là
	\choice
	{$y'=\dfrac{2}{\sqrt{x}}-\dfrac{1}{x}$}
	{$y'=\dfrac{1}{2\sqrt{x}}-\dfrac{1}{x}$}
	{$y'=\dfrac{1}{\sqrt{x}}+\dfrac{1}{x}$}
	{\True $y'=\dfrac{1}{\sqrt{x}}-\dfrac{1}{x}$}
	\loigiai{
		Ta có $y'=\dfrac{1}{\sqrt{x}}-\dfrac{1}{x}$.
	}
\end{ex}


%Cau9
\begin{ex}%[Dự án D - Dot 4 - NH25-26-Nắng Đông]
	Cho hàm số $f(x)=\ln x$, với $x>0$. Khẳng định nào sau đây là đúng?
	\choice
	{$f''(x)=-\dfrac{1}{x}$}
	{\True $f''(x)=-\dfrac{1}{x^2}$}
	{$f''(x)=-\dfrac{\ln x}{x^2}$}
	{$f''(x)=\ln x$}
	\loigiai{
		Với $x>0$ thì $f'(x)=\dfrac{1}{x}$.\\
		Suy ra $f''(x)=-\dfrac{1}{x^2}$.
	}
\end{ex}


%Cau10
\begin{ex}%[Dự án D - Dot 4 - NH25-26-Nắng Đông]
	Với $x\ne 1$, hàm số $y=\dfrac{3x+4}{x-1}$ có đạo hàm là 
	\choice
	{$y'=\dfrac{7}{(x-1)^2}$}
	{$y'=\dfrac{1}{(x-1)^2}$}
	{$y'=\dfrac{-7}{(x-1)^2}$}
	{$y'=\dfrac{-1}{(x-1)^2}$}
	\loigiai{
		Với $x\ne 1$ thì $y'=\dfrac{3(x-1)-(3x+4)}{(x-1)^2}=\dfrac{-7}{(x-1)^2}$.
	}
\end{ex}


%Cau11

\begin{ex}%[Dự án D - Dot 4 - NH25-26-Nắng Đông]
	Với $x\ne \dfrac{1}{2}$, hàm số $y=\dfrac{3x^2+2x-5}{2x-1}$ có đạo hàm là $y'=\dfrac{ax^2-bx+4c}{(2x-1)^2}$, với $a,b,c \in \mathbb{R}$. Giá trị của biểu thức $T=a^2-2b+3c$ là
	\choice
	{$72$}
	{\True $30$}
	{$80$}
	{$24$}
	\loigiai{
		Với $x\ne \dfrac{1}{2}$ thì $y'=\dfrac{(6x+2)(2x-1)-2\left(3x^2+2x-5\right)}{(2x-1)^2}=\dfrac{6x^2-6x+8}{(2x-1)^2}$.\\
		Suy ra $a=6$, $b=6$ và $c=2$.\\
		Vậy giá trị của biểu thức $T=a^2-2b+3c=36-12+6=30$.
	}
\end{ex}
%Cau12
\begin{ex}%[Dự án D - Dot 4 - NH25-26-Nắng Đông]
	Một vật chuyển động theo phương trình $s(t)=3t^2+5t+5$ (m), trong đó $t$ là thời gian chuyển động được tính bằng giây (s), $s(t)$ là quãng đường chuyển động của vật theo thời gian $t$, tính bằng mét (m). Tại thời điểm $t=2$ giây, vận tốc của vật là
	\choice
	{$13$ m/s}
	{$14$ m/s}
	{$18$ m/s}
	{\True $17$ m/s}
	\loigiai{
		Vận tốc tức thời của chuyển động trên là $v(t)=s'(t)=6t+5$ (m/s).\\
		Tại thời điểm $t=2$ giây, vận tốc của vật là $v(2)=6\cdot 2+5=17$ (m/s).
	}
\end{ex}


\Closesolutionfile{ans}

\begin{center}
	\textbf{PHẦN 2 - CÂU TRẮC NGHIỆM ĐÚNG SAI}
\end{center}
\setcounter{ex}{0}
\Opensolutionfile{ans}[ans/answer-DS-ONTAPCHUONG-VII-DE3]
%Cau1
\begin{ex}%[Dự án D - Dot 4 - NH25-26-Nắng Đông]
	Cho hàm số $y=f(x)=\sqrt{x^2+3x+5}$ có đồ thị $(C)$.
	\choiceTF
	{\True Tập xác định của hàm số $y=f(x)=\sqrt{x^2+3x+5}$ là $\mathscr{D}=\mathbb{R}$}
	{$y'=f'(x)=\dfrac{1}{2\sqrt{x^2+3x+5}}$}
	{\True $f'(1)=\dfrac{5}{6}$}
	{\True Gọi $M$ là điểm thuộc $(C)$ và có $x_M=1$. Phương trình tiếp tuyến của $(C)$ tại $M$ có phương trình là $5x-6y+13=0$}
	\loigiai{
		\begin{itemchoice}
			\itemch Điều kiện xác định: $x^2+3x+5\geq 0 \Leftrightarrow x\in \mathbb{R}$ hay hàm số đã cho có tập xác định $\mathscr{D}=\mathbb{R}$.
			\itemch Ta có $y=f(x)=\sqrt{x^2+3x+5}$ nên $y'=f'(x)=\dfrac{2x+3}{2\sqrt{x^2+3x+5}}$.
			\itemch Vì $f'(x)=\dfrac{2x+3}{2\sqrt{x^2+3x+5}}$ nên $f'(1)=\dfrac{5}{6}$.
			\itemch Với $x_M=1$ thì $y_M=3$ hay $M(1;3)$.\\
			Phương trình tiếp tuyến của $(C)$ tại $M(1;3)$ có phương trình là
			$$y=\dfrac{5}{6}(x-1)+3\Leftrightarrow 5x-6y+13=0.$$
		\end{itemchoice}
	}
\end{ex}

%Cau2
\begin{ex}%[Dự án D - Dot 4 - NH25-26-Nắng Đông]
	Một vật chuyển động thẳng không đều xác định bởi phương trình $s(t)=t^3-6t^{2}+18t+3$, trong đó $s(t)$ là quãng đường chuyển động của vật theo thời gian $t$, tính bằng mét (m) và $t$ tính bằng giây (s). 
	\choiceTF
	{Quãng đường vật đi được sau $5$ giây kể từ khi bắt đầu chuyển động là $52$ (m)}
	{Vận tốc tức thời của vật là $v(t)=3t^2-6t+18$ (m/s)}
	{\True Gia tốc tức thời của vật là $a(t)=6t-12$ (m/s$^2$)}
	{\True Vận tốc nhỏ nhất vật đạt được trong $10$ giây từ lúc bắt đầu chuyển động là $6$ (m/s)}
	\loigiai{ 
		\begin{itemchoice}
			\itemch Quãng đường vật đi được sau $5$ giây kể từ khi bắt đầu chuyển động là $s(5)=68$ (m).
			\itemch Vận tốc tức thời của vật là $v(t)=s'(t)=3t^2-12t+18$ (m/s).
			\itemch Gia tốc tức thời của vật là $a(t)=v'(t)=6t-12$ (m/s$^2$).
			\itemch Ta có $v(t)=3t^2-12t+18=3(t-2)^2+6\geq 6$, với mọi $t\geq 0$.\\
			Dấu \lq\lq$=$\rq\rq \, xảy ra khi và chỉ khi $t-2=0$ hay $t=2$.\\
			Vậy vận tốc nhỏ nhất vật đạt được trong $10$ giây từ lúc bắt đầu chuyển động là $6$ (m/s).
		\end{itemchoice}
	}
\end{ex}






\Closesolutionfile{ans}
%\inputansbox[2]{2}{ans/answer.tex}



\begin{center}
	\textbf{PHẦN 3 - CÂU TRẮC NGHIỆM TRẢ LỜI NGẮN}
\end{center}
\setcounter{ex}{0}
\Opensolutionfile{ans}[ans-KQ-ONTAPCHUONG-VII-DE3]
%Cau1
\begin{ex}%[Dự án D - Dot 4 - NH25-26-Nắng Đông]
	Cho hàm số $y=\sqrt{x^2-2x+4}$ có đồ thị $(C)$. Tiếp tuyến của $(C)$ tại giao điểm với trục tung có dạng $y=Ax+B$. Tính $2\,026A+2\,025B$.
	\shortans[]{$1012$}
	\loigiai{
		Tập xác định $\mathscr{D}=\mathbb{R}$.\\
		Gọi $M$ là giao điểm của đồ thị $(C)$ và trục tung.\\
		Khi đó $x=0$, suy ra $y=2$ hay $M(0;2)$.\\
		Ta có $y'=\dfrac{x-1}{\sqrt{x^2-2x+4}}\Rightarrow y'(0)=-\dfrac{1}{2}$.\\
		Phương trình tiếp tuyến của $(C)$ tại $M(0;2)$ là 
		$$y=-\dfrac{1}{2}(x-0)+2\Leftrightarrow y=-\dfrac{1}{2}x+2.$$
		Vậy $A=-\dfrac{1}{2}$, $B=2$ nên $2\,026A+2\,025B=-1\,013+2\,025=1\,012$.
	}
\end{ex}

%Cau2
\begin{ex}%[Dự án D - Dot 4 - NH25-26-Nắng Đông]
	Một chất điểm có phương trình chuyển động $s(t)=3 \sin \left(t+\dfrac{\pi}{3}\right)$, trong đó $t>0$, $t$ tính bằng giây (s), $s(t)$ là quãng đường chuyển động của vật theo thời gian $t$, tính bằng mét (m). Tính vận tốc tức thời của chất điểm tại thời điểm $t=\dfrac{5\pi}{3}$ (s).
	\shortans[]{$3$}
	\loigiai{
		Ta có $s(t)=3\sin{\left(t+\dfrac{\pi}{3}\right)}$, suy ra vận tốc tức thời của chất điểm là $v(t)=s'(t)=3\cos{\left(t+\dfrac{\pi}{3}\right)}$ (m/s).\\
		Vậy vận tốc tức thời của chất điểm tại thời điểm $t=\dfrac{5\pi}{3}$ (s) là
		$$v\left(\dfrac{5\pi}{3}\right) =3\cos{\left(\dfrac{5\pi}{3}+\dfrac{\pi}{3}\right)=3} \, \text{ (m/s)}.$$
	}
\end{ex}

%Cau3

\begin{ex}%[Dự án D - Dot 4 - NH25-26-Nắng Đông]
	Cân nặng trung bình của một em bé trong độ tuổi từ $0$ đến $36$ tháng có thể được tính gần đúng bởi hàm số $w(t)=0{,}00076t^3-0{,}06t^2+1{,}8t+8{,}2$, trong đó $t$, $(0\leq t\leq 36)$ là độ tuổi, được tính bằng tháng và $w(t)$ là cân nặng trung bình của một em bé sau $t$ tháng, được tính bằng pound. Tính tốc độ thay đổi cân nặng của em bé đó tại thời điểm $15$ tháng tuổi. (kết quả làm tròn đến hàng phần trăm)
	
	\shortans[]{$0{,}51$}
	\loigiai{
		Ta có $w(t)=0{,}00076t^3-0{,}06t^2+1{,}8t+8{,}2$ suy ra $w'(t)=\dfrac{57}{25\,000}t^2-\dfrac{3}{25}t+1{,}8$.\\
		Tốc độ thay đổi cân nặng của em bé đó tại thời điểm $15$ tháng tuổi là
		$$w'(15)=\dfrac{57}{25\,000}\cdot 15^2-\dfrac{3}{25}\cdot 15+1{,}8\approx 0{,}51 \text{ (pound/tháng).}$$		
	}
\end{ex}
%Cau4
\begin{ex}%[Dự án D - Dot 4 - NH25-26-Nắng Đông]
	Một ô tô chuyển động có phương trình chuyển động là $s(t)= -t^3 + 6t^2 + 5t$ với $t$ là khoảng thời gian tính từ lúc ô tô bắt đầu chuyển động, tính bằng giây (s) và $s(t)$ là quãng đường ô tô đi được trong khoảng thời gian $t$, tính bằng mét (m). Trong khoảng thời gian $8$ giây đầu tiên, vận tốc của ô tô đạt giá trị lớn nhất bằng bao nhiêu m/s?
	
	\shortans[]{$17$}
	\loigiai{
		Vận tốc tức thời của ô tô là $v(t)=s'(t)=-3t^2+12t+5$ (m/s).\\
		Ta có $v(t)=-3t^2+12t+5=-3(t-2)^2+17\leq 17$, với $t\in (0;8)$.\\
		Dấu \lq\lq$=$\rq\rq xảy ra khi và chỉ khi $t-2=0$ hay $t=2$.\\
		Vậy trong khoảng thời gian $8$ giây đầu tiên ô tô đạt vận tốc lớn nhất là $17$ m/s.
	}
\end{ex}
\Closesolutionfile{ans}

\begin{center}
	\textbf{PHẦN 4 - TỰ LUẬN}
\end{center}
\setcounter{ex}{0}
%Cau1
\begin{ex}%[Dự án D - Dot 4 - NH25-26-Nắng Đông]
	Tính đạo hàm của hàm số $f(x)=\left(31x^2-7x-25\right)^{26}$.
	\loigiai{
		Ta có
		\begin{eqnarray*}
			f'(x)&=&\left[\left(31x^2-7x-25\right)^{26}\right]'\\
			&=&26\cdot\left(31x^2-7x-25\right)^{25} \cdot \left(31x^2-7x-25\right)'\\
			&=&26\cdot\left(31x^2-7x-25\right)^{25} \cdot (62x-7).
		\end{eqnarray*}
		Vậy $f'(x)=26(62x-7)\left(31x^2-7x-25\right)^{25}$.
	}
\end{ex}

%Cau2
\begin{ex}%[Dự án D - Dot 4 - NH25-26-Nắng Đông]
	Biết hàm số $y=\left(x^2-1\right)\mathrm{e}^{2x}$ có đạo hàm $y'=\left(ax^2+bx+c\right)\mathrm{e}^{2x}$. Tính giá trị của biểu thức $T=3a^2+2b^3-7c$.
	\loigiai{
		Ta có 
		\begin{eqnarray*}
			y'&=&\left(x^2-1\right)'\cdot\mathrm{e}^{2x}+\left(x^2-1\right)\cdot\left(\mathrm{e}^{2x}\right)'\\
			&=&2x\cdot\mathrm{e}^{2x}+\left(x^2-1\right)\cdot 2\cdot\mathrm{e}^{2x}\\
			&=&\left(2x^2+2x-2\right) \cdot\mathrm{e}^{2x}.
		\end{eqnarray*}
		Suy ra $a=2$, $b=2$ và $c=-2$.\\
		Vậy $T=3a^2+2b^3-7c=12+16+14=42$.
	}
\end{ex}

%Cau3
\begin{ex}%[Dự án D - Dot 4 - NH25-26-Nắng Đông]
	Một con lắc lò xo dao động điều hòa theo phương ngang trên mặt phẳng không ma sát như bên dưới và có phương trình chuyển động $x(t)=4\sin\left(2t+\dfrac{\pi}{3}\right)$, trong đó $t$ tính bằng giây và $x$ tính bằng centimét (cm). Tìm gia tốc tức thời của con lắc tại thời điểm $t=\dfrac{2\pi}{3}$ (s) và tại thời điểm đó, con lắc di chuyển theo hướng như thế nào so với phương $Ot$?
	\begin{center}
		\begin{tikzpicture}[line cap=round,line join=round,>=triangle 45,x=1cm,y=1cm]
			%	\draw[decoration={aspect=0.3, segment length=1.5mm, amplitude=3mm,coil},decorate] (0,1) -- (4,1); 
			\draw [smooth,domain=3.13:15*pi,scale=0.1,samples=100] plot (\x, {4*sin(\x r)+7});
			\fill [pattern = north east lines] (-0.2,0) rectangle (0,4);
			\draw[fill=gray] (4.9,1.5)--(5.5,1.5)--(5.5,0)--(4.9,0)--cycle;
			\draw[->](0,0)--(6,0) ;
			\draw[->] (2.5,2.5)--(2.5,1.2);
			\draw[thick] (0,0) -- (0,4) (2.5,0.1)--(2.5,-0.1) (5.2,0.1)--(5.2,-0.1);
			\draw (2.5,3) node {vị trí cân bằng}
			(2.5,-0.3) node {$O$}
			(5.2,-0.3) node {$x(t)$}
			(6,-0.3) node {$t$}
			;
			\draw (0,0.7)--(0.3,0.7) (4.72,0.7)--(4.9,0.7);
%			\path (0,0) node{\hypersetup{hidelinks}\href{1Krv7eN}{ }};
%			\path (0,0) node{\hypersetup{hidelinks}\href{1Krv7eN}{ }};
%			\path (0,0) node{\hypersetup{hidelinks}\href{1Krv7eN}{ }};
%			\path (0,0) node{\hypersetup{hidelinks}\href{1Krv7eN}{ }};
		\end{tikzpicture}
	\end{center}
	\loigiai{
		Vận tốc tức thời của con lắc là $v(t)=x'(t)=8\cos \left(2t+\dfrac{\pi}{3}\right)$ (cm/s).\\
		Gia tốc tức thời của con lắc là $a(t)=v'(t)=-16\sin \left(2t+\dfrac{\pi}{3}\right)$ (cm/s$^2$).\\
		Khi đó gia tốc tức thời của con lắc tại thời điểm $t=\dfrac{2\pi}{3}$ (s) là
		$$a\left(\dfrac{2\pi}{3}\right)=-16\sin \left(2\cdot \dfrac{2\pi}{3}+\dfrac{\pi}{3}\right)=8\sqrt{3}\, \text{ (cm/s$^2$)}.$$
		Và $x\left(\dfrac{2\pi}{3}\right)=4\sin\left(2\cdot \dfrac{2\pi}{3}+\dfrac{\pi}{3}\right)=-2\sqrt{3}$ (cm).\\
		Vì $x\left(\dfrac{2\pi}{3}\right)=-2\sqrt{3}<0$ nên tại thời điểm $t=\dfrac{2\pi}{3}$ (s), con lắc di chuyển ngược hướng với phương $Ot$.
	}
\end{ex}

