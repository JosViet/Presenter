\newpage
\def\thoigian{90}%--Thời gian
\de{Đề số 3}{Chương V. THỐNG KÊ}

\begin{center}
	\textbf{PHẦN 1 - CÂU TRẮC NGHIỆM BỐN PHƯƠNG ÁN}
\end{center}
\Opensolutionfile{ans}[ans/ans-TN-ONTAPCHUONGV-DE3]
\begin{ex}%[1D5N1-2]%[Dự án D đợt 4 - Nguyễn Hoàng Anh]%[Đề ôn tập Chương V. THỐNG KÊ - Khối 11 - Đề số 3]
	Số lượng khách hàng nữ mua hàng thời trang trong một ngày của một cửa hàng được thống kê trong bảng tần số ghép nhóm sau
	\begin{center}
		\begin{tabular}{|c|c|c|c|c|c|}
			\hline
			Mức thưởng & $[20;30)$ & $[30;40)$ & $[40;50)$ & $[50;60)$ & $[60;70)$ \\
			\hline
			Số khách hàng nữ & $3$ & $9$ & $6$ & $4$ & $2$ \\
			\hline
		\end{tabular}
	\end{center}
	Cỡ mẫu của bảng tần số ghép nhóm là
	\choice
	{$40$}
	{\True $24$}
	{$10$}
	{$20$}
	\loigiai
	{
		Cỡ mẫu của bảng tần số ghép nhóm là $n=3+9+6+4+2=24$.
	}
\end{ex}
\begin{ex}%[1D5N1-1]%[Dự án D đợt 4 - Nguyễn Hoàng Anh]%[Đề ôn tập Chương V. THỐNG KÊ - Khối 11 - Đề số 3]
	Số lượng khách hàng nữ mua bảo hiểm nhân thọ trong một ngày được thống kê trong bảng tần số ghép nhóm sau
	\begin{center}
		\begin{tabular}{|c|c|c|c|c|c|}
			\hline
			Khoảng tuổi & { $[20; 30)$ } & { $[30; 40)$ } & { $[40; 50)$ } & { $[50; 60)$ } & { $[60; 70)$ } \\
			\hline
			Số khách hàng nữ & 3 & 9 & 6 & 4 & 2 \\
			\hline
		\end{tabular}
	\end{center}
	Giá trị đại diện của nhóm $[30;40)$ là
	\choice
	{$9$}
	{$40$}
	{$30$}
	{\True $35$}
	\loigiai{Giá trị đại diện của nhóm $[30;40)$ là $\dfrac{30+40}{2}=35$.
	}
\end{ex}
\begin{ex}%[1D5N1-2]%[Dự án D đợt 4 - Nguyễn Hoàng Anh]%[Đề ôn tập Chương V. THỐNG KÊ - Khối 11 - Đề số 3]
	Cho mẫu số liệu ghép nhóm về thời gian (phút) đi từ nhà đến nơi làm việc của các nhân viên một công ty như sau
	\begin{center}
		\begin{tabular}{|l|c|c|c|c|c|c|c|}
			\hline Thời gian &{$[15 ; 20)$}&{$[20 ; 25)$}&{$[25 ; 30)$}&{$[30 ; 35)$}&{$[35 ; 40)$}&{$[40 ; 45)$}&{$[45 ; 50)$}\\
			\hline Số nhân viên & 6 & 14 & 25 & 37 & 21 & 13 & 9 \\
			\hline
		\end{tabular}
	\end{center}
	Có bao nhiêu nhân viên có thời gian đi từ nhà đến nơi làm việc là từ $15$ phút đến dưới $20$ phút?
	\choice
	{$9$ }
	{$13$}
	{$14$ }
	{\True $6$}
	\loigiai{
		Số 	nhân viên có thời gian đi từ nhà đến nơi làm việc là từ 15 phút đến dưới 20 phút là $6$.
	}
\end{ex}
\begin{ex}%[1D5H1-4]%[Dự án D đợt 4 - Nguyễn Hoàng Anh]%[Đề ôn tập Chương V. THỐNG KÊ - Khối 11 - Đề số 3]
	Một công ty xây dựng khảo sát khách hàng xem họ có nhu cầu mua nhà ở mức giá nào. Kết quả khảo sát được ghi lại ở bảng sau\\
	\begin{center}
		\begin{tabular}{|c|c|c|c|c|c|}
			\hline
			Mức giá (Triệu đồng/m$^2$)&$[10; 14)$& $[14; 18)$&$[18; 22)$& $[22; 26)$& $[26; 30)$\\
			\hline
			Số khách hàng &$54$& $78$ & $120$ &$45$ & $12$\\
			\hline
		\end{tabular}
	\end{center}
	Mốt của mẫu số liệu ghép nhóm  trên gần bằng giá trị nào sau đây?
	\choice
	{$20{,}4$}
	{$21{,}4$}
	{\True$19{,}4$}
	{$18{,}4$}
	\loigiai{Ta có nhóm chứa mốt là $[18;22)$, khi đó mốt của mẫu số liệu trên là
		\begin{center}
			$\begin{aligned}	M_0&=u_m+\dfrac{n_m-n_{m-1}}{\left(n_m-n_{m-1}\right)+\left(n_m-n_{m+1}\right)}\cdot\left(u_{m+1}-u_m\right)\\
				&=18+\dfrac{120-78}{(120-78)+(120-45)}\cdot(22-18)\\
				&=\dfrac{758}{39}\approx 19{,}44.
			\end{aligned}$
	\end{center}}
\end{ex}
\begin{ex}%[1D5H1-2]%[Dự án D đợt 4 - Nguyễn Hoàng Anh]%[Đề ôn tập Chương V. THỐNG KÊ - Khối 11 - Đề số 3]
	Thời gian luyện tập trong ngày (tính theo giờ) của một số vận động viên được ghi lại ở bảng sau
	\begin{center}
		\begin{tabular}{|c|c|c|c|c|c|}
			\hline
			Thời gian luyện tập & $[0;2)$ & $[2;4)$ & $[4;6)$ & $[6;8)$ & $[8;10)$ \\
			\hline
			Số vận động viên & $3$ & $8$ & $15$ & $12$ & $4$\\
			\hline
		\end{tabular}
	\end{center}
	Giá trị đại diện của nhóm có tần số lớn nhất là bao nhiêu?
	\choice
	{$10$}
	{$7$}
	{$15$}
	{\True $5$}
	\loigiai{
		Nhóm có tần số lớn nhất là $[4;6)$ và có giá trị đại diện là $\dfrac{4+6}{2} = 5$.
	}
\end{ex}
\begin{ex}%[1D5N1-1]%[Dự án D đợt 4 - Nguyễn Hoàng Anh]%[Đề ôn tập Chương V. THỐNG KÊ - Khối 11 - Đề số 3]
	Khảo sát thời gian tập thể dục trong ngày của một số học sinh lớp $11$ thu được mẫu số liệu ghép nhóm sau
	\begin{center}
		\begin{tabular}{|c|c|c|c|c|c|}
			\hline
			Thời gian (phút)& $[0;20)$ & $[20;40)$ & $[40;60)$ & $[60;80)$ & $[80;100)$ \\
			\hline
			Số học sinh& $5$ & $9$ & $12$ & $10$ & $16$ \\
			\hline
		\end{tabular}
	\end{center}
	Giá trị đại diện của nhóm $[20;40)$ là 
	\choice
	{$40$}
	{$10$}
	{$20$}
	{\True $30$}
	\loigiai{
		Giá trị đại diện của nhóm $[20;40)$ là $\dfrac{20+40}{2}=30$.	
	}
\end{ex}
\begin{ex}%[1D5H2-2]%[Dự án D đợt 4 - Nguyễn Hoàng Anh]%[Đề ôn tập Chương V. THỐNG KÊ - Khối 11 - Đề số 3]
	Thời gian (phút) truy bài trước mỗi buổi học của một số học sinh trong một tuần được ghi lại ở bảng sau:
	\begin{center}
		\begin{tabular}{|c|c|c|c|c|c|}
			\hline Thời gian &{$[9{,}5 ; 12{,}5)$}&{$[12{,}5 ; 15{,}5)$}&{$[15{,}5 ; 18{,}5)$}&{$[18{,}5 ; 21{,}5)$}&{$[21{,}5 ; 24{,}5)$}\\
			\hline Số học sinh & 3 & 12 & 15 & 24 & 2 \\
			\hline
		\end{tabular}		
	\end{center}
	Trung vị của mẫu số liệu trên bằng
	\choice
	{$16{,}2$}
	{\True $18{,}1$}
	{$9$}
	{$15$}
	\loigiai{
		Ta có tổng số học sinh $n=56$.\\
		Gọi $x_1$; $x_2$; $\ldots$; $x_{56}$ là cân nặng của 56 quả bơ xếp theo thứ tự không giảm. \\
		Trung vị của mẫu số liệu $x_1$; $\ldots$; $x_{56}$ là $\dfrac{1}{2}\left(x_{28}+x_{29}\right)\in [15{,}5 ; 18{,}5)$.\\
		Ta xác định được $n=56, n_m=15, C=3+12=15, u_m=15{,}5, u_{m+1}=18{,}5$.
		Vậy trung vị của mẫu số liệu ghép nhóm là
		$$
		M_e=15{,}5+\dfrac{\dfrac{56}{2}-15}{15} \cdot(18{,}5-15{,}5)=18{,}1.
		$$	
}\end{ex}
\begin{ex}%%[1D5H2-3]%[Dự án D đợt 4 - Nguyễn Hoàng Anh]%[Đề ôn tập Chương V. THỐNG KÊ - Khối 11 - Đề số 3]
	Cô Minh Hiền rất thích nhảy hiện đại. Thời gian tập nhảy mỗi ngày trong thời gian gần đây của Cô Minh Hiền được thống kê lại ở bảng sau
	\begin{center}
		\begin{tabular}{|l|c|c|c|c|c|}
			\hline Thời gian (phút)	& $[20; 25)$&$[25; 30)$&$[30; 35)$ &$[35; 40)$ &$[40; 45)$ \\
			\hline Số ngày& $6$	&  $6$& $4$ & $4$ & $1$ \\			
			\hline
		\end{tabular}
	\end{center}	
	Nhóm chứa tứ phân vị thứ nhất $Q_1$ là
	\choice
	{$[30; 35)$}
	{$[25; 30)$}
	{\True $[20; 25)$}
	{$[35; 40)$}
	\loigiai{
		Cỡ mẫu $n=21$.\\
		Ta có $Q_1=\dfrac{x_5+x_6}{2}$.\\
		Ta lại có $x_5$, $x_6\in [20; 25)$ nên $Q_1\in [20; 25)$. 
	}
\end{ex}
\begin{ex}%[1D5V2-3]%[Dự án D đợt 4 - Nguyễn Hoàng Anh]%[Đề ôn tập Chương V. THỐNG KÊ - Khối 11 - Đề số 3]
	Một hãng ô tô thống kê lại số lần gặp sự cố về động cơ của $100$ chiếc xe cùng loại sau $2$ năm sử dụng đầu tiên ở bảng sau
	\begin{center}
		\begin{tabular}{|l|c|c|c|c|c|}
			\hline Số lần gặp sự cố & {$[1 ; 2]$} & {$[3 ; 4]$} & {$[5 ; 6]$} & {$[7 ; 8]$} & {$[9 ; 10]$} \\
			\hline Số xe & $17$ & $33$ & $25$ & $20$ & $5$ \\
			\hline
		\end{tabular}
	\end{center}
	Hãy ước lượng tứ phân vị thứ nhất của mẫu số liệu ghép số trên.
	\choice
	{$2{,}64$}
	{$2,89 $}
	{$2,73$}
	{\True  $2,98$}
	\loigiai{
		Do số lần gặp sự cố là số nguyên nên ta hiệu chỉnh lại như sau:
		\begin{center}
			\begin{tabular}{|l|c|c|c|c|c|}
				\hline Số lần gặp sự cố & {$[0{,}5 ; 2{,}5)$} & {$[2{,}5 ; 4{,}5)$} & {$ [4{,}5 ; 6{,}5)$} & {$[6{,}5 ; 8{,}5)$} & {$[8{,}5 ; 10{,}5)$} \\
				\hline Số xe & $17$ & $33$ & $25$ & $20$ & $5$ \\
				\hline
			\end{tabular}
		\end{center}
		Cỡ mẫu là $n=100$.\\
		Gọi $x_1; x_2; \ldots; x_{100}$ là số lần gặp sự cố về động cơ của $100$ xe và giả sử dãy này được sắp xếp theo thứ tự tăng dần. Khi đó, tứ phân vị thứ nhất $Q_1$ là $\dfrac{x_{25}+x_{26}}{2}$. Do $ x_{25}$ và $x_{26}$ thuộc nhóm $[2{,}5 ; 4{,}5)$ nên tứ phân vị thứ nhất của mẫu số liệu ghép nhóm là
		\begin{center}
		$Q_{1}=2{,}5+\dfrac{\dfrac{1\cdot 100}{4}-17}{33}\cdot (4{,}5-2{,}5)=\dfrac{197}{66} \approx 2{,}98$.
		\end{center}
	}
\end{ex}
\begin{ex}%[1D5H2-3]%[Dự án D đợt 4 - Nguyễn Hoàng Anh]%[Đề ôn tập Chương V. THỐNG KÊ - Khối 11 - Đề số 3]
	Kiểm tra điện lượng của một số viên pin tiểu do một hãng sản xuất thu được kết quả sau
	\begin{center}
		\begin{tabular}{|l|c|c|c|c|c|}
			\hline Điện lượng (nghìn mAh ) & {$ [0{,}9 ; 0{,}95) $} & {$ [0{,}95 ; 1{,}0) $} & {$ [1{,}0 ; 1{,}05) $} & {$ [1{,}05 ; 1{,}1) $} & {$ [1{,}1 ; 1{,}15) $} \\
			\hline Số viên pin & $10$ & $20$ & $35$ & $15$ & $5$ \\
			\hline
		\end{tabular}
	\end{center}
	Hãy ước lượng tứ phân vị của mẫu số liệu ghép nhóm trên.
	\choice
	{ $ Q_{1}=0{,}58 ; Q_{2}=1{,}02 ; Q_{3}=1{,}048 $}
	{ $ Q_{1}=0{,}98 ; Q_{2}=1{,}02 ; Q_{3}=1{,}248 $}
	{ $ Q_{1}=0{,}98 ; Q_{2}=1{,}22 ; Q_{3}=1{,}048 $}
	{\True $ Q_{1}=0{,}98 ; Q_{2}=1{,}02 ; Q_{3}=1{,}048 $}
	\loigiai{
		Cỡ mẫu là $n=10+20+35+15+5=85$.\\
		Gọi $ x_{1} ; x_{2} ; x_{3} ; \ldots ; x_{85} $ lần lượt là điện lượng của $85$ viên pin và giả sử dãy này được sắp xếp theo thứ tự tăng dần.\\
		Tứ phân vị thứ hai của dãy số liệu là $\dfrac{1}{2}\left(x_{42}+x_{43}\right)$ thuộc nhóm $[1{,}0 ; 1{,}05)$ nên tứ phân vị thứ hai của mẫu số liệu là $ Q_{2}=1{,}0+\dfrac{\dfrac{85}{2}-30}{35}\cdot (1{,}05-1{,}0)=1{,}02$.\\
		Tứ phân vị thứ nhất của dãy số liệu là $\dfrac{1}{2}\left(x_{21}+x_{22}\right)$ thuộc nhóm $[0{,}95 ; 1{,}0)$ nên tứ phân vị thứ nhất của mẫu số liệu là $ Q_{1}=0{,}95+\dfrac{\dfrac{85}{4}-10}{20}\cdot (1{,}0-0{,}95)=0{,}98$.\\
		Tứ phân vị thứ ba của dãy số liệu là $\dfrac{1}{2}\left(x_{63}+x_{64}\right)$ thuộc nhóm $ [1{,}0 ; 1{,}05) $ nên tứ phân vị thứ ba của mẫu số liệu là $ Q_{3}=1{,}0+\dfrac{\dfrac{3\cdot 85}{4}-30}{35}\cdot (1{,}05-1{,}0)=1{,}048$.
	}
\end{ex}
\begin{ex}%[1D5N1-3]%[Dự án D đợt 4 - Nguyễn Hoàng Anh]%[Đề ôn tập Chương V. THỐNG KÊ - Khối 11 - Đề số 3]
	Khảo sát thời gian (đơn vị: phút) tập thể dục của một số học sinh khối $12$ thu được mẫu số liệu ghép nhóm sau
	\begin{center}
		\begin{tabular}{|>{\centering\arraybackslash}m{2.5cm}
				|>{\centering\arraybackslash}m{1.5cm}
				|>{\centering\arraybackslash}m{1.5cm}
				|>{\centering\arraybackslash}m{1.5cm}
				|>{\centering\arraybackslash}m{1.5cm}
				|>{\centering\arraybackslash}m{1.5cm}|}
			\hline
			\textbf{Thời gian} & $[0;20)$ & $[20;40)$ & $[40;60)$ & $[60;80)$ & $[80;100)$ \\
			\hline
			\textbf{Số học sinh} & $5$ & $9$ & $12$ & $10$ & $6$ \\
			\hline
		\end{tabular}
	\end{center}
	Trung bình mỗi học sinh tập bao nhiêu phút mỗi ngày (Làm tròn đến hàng phần mười)?
	\choice
	{\True $51{,}4$}
	{$52{,}3$}
	{$51{,}6$}
	{$52{,}5$}
	\loigiai{Bảng thống kê bổ sung giá trị đại diện
		\begin{center}
		\begin{tabular}{|>{\centering\arraybackslash}m{3cm}
				|>{\centering\arraybackslash}m{1.5cm}
				|>{\centering\arraybackslash}m{1.5cm}
				|>{\centering\arraybackslash}m{1.5cm}
				|>{\centering\arraybackslash}m{1.5cm}
				|>{\centering\arraybackslash}m{1.5cm}|}
				\hline
				\textbf{Thời gian} & $[0;20)$ & $[20;40)$ & $[40;60)$ & $[60;80)$ & $[80;100)$ \\
				\hline
				\textbf{Số học sinh} & $5$ & $9$ & $12$ & $10$ & $6$ \\
				\hline
				\textbf{Giá trị đại diện} & $10$ & $30$ & $50$ & $70$ & $90$\\
				\hline 
			\end{tabular}
		\end{center}
		
		Khi đó $\overline{x} = \dfrac{10\cdot 5 + 30\cdot 9 + 50\cdot 12 + 70\cdot 10 + 90\cdot 6}{5 + 9 + 12 + 10 + 6} \approx 51{,}4.$
	}
\end{ex}
\begin{ex}%[1D5H2-3]%[Dự án D đợt 4 - Nguyễn Hoàng Anh]%[Đề ôn tập Chương V. THỐNG KÊ - Khối 11 - Đề số 3]
	Cân nặng (đơn vị: kg) của một số heo con mới sinh được cho trong bảng dưới đây
	\begin{center}
		\begin{tabular}{|l|c|c|c|c|}
			\hline Cân nặng $(\mathrm{kg})$ & $[1{,}0 ; 1{,}2)$ & $[1{,}2 ; 1{,}4)$ & $[1{,}4 ; 1{,}6)$ & $[1{,}6 ; 1{,}8)$ \\
			\hline Số con & $13$ & $14$ & $24$ & $15$ \\
			\hline
		\end{tabular}
	\end{center}
	Tứ phân vị thứ $3$ của mẫu số liệu bằng bao nhiêu (Làm tròn đến hàng phần trăm)?
	\choice 
	{\True $1{,}59$}
	{$1{,}60$}
	{$1{,}55$}
	{$1{,}49$}
	\loigiai{
		Cỡ mẫu $13+14+24+15=66$.\\
		Gọi $x_1$, $x_2$, $\ldots$, $x_{66}$ là cân nặng của $66$ heo con được sắp xếp theo thứ tự không giảm.\\
		$Q_3=x_{50}\in \left[1{,}4; 1{,}6\right)$.\\
		Do đó $u_3=1{,}4$; $n_3=24$; $n_1+n_2=13+14=27$; $u_4-u_3=1{,}6-1{,}4=0{,}2$.\\
		Khi đó \begin{eqnarray*}Q_3&=&u_3+\dfrac{\dfrac{3\cdot 66}{4}-(n_1+n_2)}{n_3}\left(u_4-u_3\right)\\
			&=&1{,}4+\dfrac{\dfrac{3\cdot 66}{4}-(27)}{24}\cdot 0{,}2\approx1{,}59.\end{eqnarray*}
	}
\end{ex} 
\Closesolutionfile{ans}
%\begin{center}
%	\textbf{ĐÁP ÁN}
%	\inputansbox{10}{ans/ans}	
%\end{center}
\begin{center}
	\textbf{PHẦN 2 - CÂU TRẮC NGHIỆM ĐÚNG SAI}
\end{center}
\Opensolutionfile{ans}[ans/answer-DS-ONTAPCHUONGV-DE3]
\begin{ex}%[1D5V1-2]%[1D5V2-3]%[Dự án D đợt 4 - Nguyễn Hoàng Anh]%[Đề ôn tập Chương V. THỐNG KÊ - Khối 11 - Đề số 3]
	Bảng thống kê doanh số bán hàng của các nhân viên một siêu thị điện máy trong một dịp Black Friday như sau:
	\begin{center}
		\begin{tabular}{|c|c|c|c|c|c|}
			\hline \begin{tabular}{c} 
				Doanh số \\
				(triệu đồng)
			\end{tabular} & $[20;30)$ & $[30;40)$ & $[40;50)$ & $[50;60)$ & $[60;70)$ \\
			\hline Số nhân viên & $4$ & $8$ & $12$ & $7$ & $5$ \\
			\hline
		\end{tabular}
	\end{center}
	\choiceTF
	{Cỡ mẫu của mẫu số liệu trên là $n=35$}
	{\True Số trung bình của mẫu số liệu trên là $\overline{x}\approx 45{,}28$ triệu đồng (\textit{kết quả làm tròn đến hàng phần trăm})}
	{Tứ phân vị thứ nhất của mẫu số liệu trên là $Q_1=36{,}5$}
	{Cửa hàng dự định sẽ thưởng $25\%$ số nhân viên có doanh số bán hàng cao nhất. Theo mẫu sổ liệu trên, của hàng nên khen thưởng các nhân viên có doanh số bán hàng ít nhất từ $53{,}29$ triệu đồng (kết quả làm tròn đến hàng phần trăm)}
	\loigiai{
		Theo bài ra ta có bảng tần số ghép nhóm sau
		\begin{center}
			\begin{tabular}{|c|c|c|c|c|c|}
				\hline \begin{tabular}{c} 
					Doanh số \\
					(triệu đồng)
				\end{tabular} & $[20;30)$ & $[30;40)$ & $[40;50)$ & $[50;60)$ & $[60;70)$ \\
				\hline Giá trị đại diện & $25$ & $35$ & $45$ & $55$ & $65$ \\		
				\hline Số nhân viên & $4$ & $8$ & $12$ & $7$ & $5$ \\
				\hline
			\end{tabular}
		\end{center}
		\begin{itemchoice}
			\itemch \textbf{Sai}.\\
			Cỡ mẫu của mẫu số liệu trên là $n=4+8+12+7+5=36$.
			\itemch \textbf{Đúng}.\\
			Giá trị trung bình của mẫu số liệu 
			\[ \overline{x}=\dfrac{25\cdot 4+35\cdot 8+45\cdot 12+55\cdot 7+65\cdot 5}{36}=\dfrac{815}{18}\approx 45{,}28.\]
			\itemch \textbf{Sai}.\\
			Ta có $\dfrac{n}{4}=9$ mà $4<9<12$ nên suy ra tứ phân vị thứ nhất là 
			\[ Q_1=30+\dfrac{9-4}{8}\cdot 10=36{,}25.\]
			\itemch \textbf{Sai}.\\
			Ta có $\dfrac{3n}{4}=27$ mà $4+8+12<27<4+8+12+7$ nên suy ra tứ phân vị thứ ba là 
			\[ Q_3=50+\dfrac{27-24}{7}\cdot 10=\dfrac{380}{7}\approx 54{,}29. \]
			Vậy cửa hàng nên khen thưởng các nhân viên có doanh số bán hàng ít nhất là $54{,}29$ triệu đồng.
		\end{itemchoice}
	}
\end{ex}
\begin{ex}%[1D5H1-4]%[Dự án D đợt 4 - Nguyễn Hoàng Anh]%[Đề ôn tập Chương V. THỐNG KÊ - Khối 11 - Đề số 3]
	\immini{
		Bảng bên cho ta bảng tần số ghép nhóm về số liệu thống kê tỉ lệ che phủ rừng (đơn vị: \%) của $60$ tỉnh, thành phố ở Việt Nam (không bao gồm Hưng Yên, Vĩnh Long, Cần Thơ) tính đến ngày $31 / 12 / 2020$
		\textit{(Nguồn: https://bandolamnghiep.com).}
		\choiceTF
		{\True Tỉ lệ che phủ rừng trung bình trên một tỉnh, thành phố được thống kề ở trên là lớn hơn $33 \%$}
		{\True Trung vị của mẫu số liệu trên là $40 \%$}
		{Có $20$ tỉnh, thành phố có tỉ lệ che phủ rừng nhỏ hơn $10 \%$}
		{Mốt của mẫu số liệu trên là $5 \%$}}{\begin{tabular}{|c|c|}
			\hline
			Nhóm & Tần số \\
			\hline
			$[0 ; 10)$ & 17 \\
			\hline
			$[10 ; 20)$ & 6 \\
			\hline
			$[20 ; 30)$ & 3 \\
			\hline
			$[30 ; 40)$ & 4 \\
			\hline
			$[40 ; 50)$ & 9 \\
			\hline
			$[50 ; 60)$ & 15 \\
			\hline
			$[60 ; 70)$ & 5 \\
			\hline
			$[70 ; 80)$ & 1 \\
			\hline
	\end{tabular}}
	\loigiai{
		Ta có bảng sau
		\begin{center}
			\begin{tabular}{|c|c|c|c|c|c|c|c|c|}
				\hline
				Nhóm &  $[0 ; 10)$ & $[10 ; 20)$ &$[20 ; 30)$& $[30 ; 40)$ & $[40 ; 50)$ & $[50 ; 60)$&$[60 ; 70)$ &  $[70 ; 80)$ \\ \hline 
				Giá trị đại diện & $5$&  $15$& $25$& $35$& $45$& $55$& $65$& $75$\\ \hline
				Tần số &  $17 $ &  $6 $ &  $3$ &  $4$ &  $9$ & $15$ & $5$ & $1$ \\ \hline 
				Tần số tích lũy & $17$ & $23$& $26$& $30$& $39$& $54$&  $59$& $60$\\ \hline 
			\end{tabular}
		\end{center}
		\begin{itemchoice}
			\itemch \textbf{Đúng.}\\ Tỉ lệ che phủ rừng trung bình trên một tỉnh, thành phố là
			$$\overline{x}=\dfrac{17\cdot 5+6\cdot15+3 \cdot 25+4\cdot35+9\cdot45+15 \cdot 55+5 \cdot 65+1\cdot75}{60}=\dfrac{101}{3} \, \% >33\, \%.$$
			\itemch\textbf{ Đúng.}\\ Ta có $\dfrac{n}{2}= 30$.\\ 
			Suy ra nhóm $[30;43)$ là nhóm đầu tiên có tần số tích lũy lớn hơn hay bằng $30$.\\
			Vậy trung vị của mẫu số liệu đã cho bằng $M_e= 30 + \dfrac{30- 26}{4} \cdot 10=40 \ \%$.
			\itemch \textbf{Sai.}\\ 
			Vì có $17$ tỉnh thành có tỉ lệ rừng che phủ nhỏ hơn $10 \, \%$.
			\itemch \textbf{Sai.}\\ 
			Ta có nhóm có tần số lớn nhất là nhóm $[0;10)$.\\
			Suy ra  Mốt của mẫu số liệu là $M_0 = 0 + \dfrac{17-0}{2\cdot 17 - 0 -6 } \approx 6 \, \%$.
		\end{itemchoice}
	}
\end{ex}
\Closesolutionfile{ans}
%\inputansbox[2]{2}{ans/answer.tex}
\begin{center}
\textbf{PHẦN 3 - CÂU TRẮC NGHIỆM TRẢ LỜI NGẮN}
\end{center}
\setcounter{ex}{0}
\Opensolutionfile{ans}[ans-KQ-ONTAPCHUONGV-DE3]
\begin{ex}%[1D5H2-2]%[Dự án D đợt 4 - Nguyễn Hoàng Anh]%[Đề ôn tập Chương V. THỐNG KÊ - Khối 11 - Đề số 3]
	Bảng bên dưới cho ta bảng tần số ghép nhóm về số liệu thống kê chiều dài đường bờ biển (đơn vị: ki-lô-mét) của $28$ tỉnh, thành phố có giáp biển ở Việt Nam.\\
	Trung vị của mẫu số liệu đó bằng bao nhiêu (làm tròn kết quả đến hàng đơn vị)?
	\begin{center}
		\begin{tabular}{|c|c|}
			\hline
			Nhóm & Tần số \\
			\hline
			$[0 ; 100)$ & 13 \\
			\hline
			$[100 ; 200)$ & 11 \\
			\hline
			$[200 ; 300)$ & 3 \\
			\hline
			$[300 ; 400)$ & 1 \\
			\hline
		\end{tabular}
	\end{center}
	\shortans{$109$}
	\loigiai{Ta có 
		\begin{center}
			\begin{tabular}{|c|c|c|}
				\hline
				Nhóm & Tần số & Tần số tích lũy \\
				\hline
				$[0 ; 100)$ & $13 $ & $13$\\
				\hline
				$[100 ; 200)$ & $11$ & $24$\\
				\hline
				$[200 ; 300)$ & $3$& $27$  \\
				\hline
				$[300 ; 400)$ & $1$ & $28$\\
				\hline
			\end{tabular}
		\end{center}
		Ta có $\dfrac{n}{2}= 14$.\\
		Nhóm đầu tiên có tần số tích lũy lớn hơn hay bằng $14$ là $[100; 200)$.\\
		Suy ra trung vị của mẫu số liệu là
		\begin{center}
		$M_e = 100 +\dfrac{14- 13}{11}\cdot 100\approx 109$\,km.
		\end{center}
		}
\end{ex}

\begin{ex}%[1D5H1-4]%[Dự án D đợt 4 - Nguyễn Hoàng Anh]%[Đề ôn tập Chương V. THỐNG KÊ - Khối 11 - Đề số 3]
	Người ta đo chiều cao của $35$ cây bạch đàn (đơn vị: m) và thu được kết quả như sau
	\begin{center}
		\begin{tabular}{|l|l|l|l|l|l|l|l|l|l|}
			\hline
			$6{,}6$ & $6{,}9$ & $7{,}0$ & $7{,}2$ & $7{,}3$ & $7{,}4$ & $7{,}5$ & $7{,}5$ & $7{,}6$ & $7{,}7$ \\
			\hline
			$7{,}7$ & $7{,}8$ & $7{,}8$ & $7{,}9$ & $7{,}9$ & $8{,}0$ & $8{,}0$ & $8{,}0$ & $8{,}1$ & $8{,}1$ \\
			\hline
			$8{,}2$ & $8{,}2$ & $8{,}2$ & $8{,}3$ & $8{,}3$ & $8{,}4$ & $8{,}5$ & $8{,}5$ & $8{,}6$ & $8{,}7$ \\
			\hline
			$8{,}7$ & $8{,}8$ & $8{,}9$ & $9{,}0$ & $9{,}4$ & & & & & \\
			\hline
		\end{tabular}
	\end{center}
	Từ mẫu số liệu không ghép nhóm trên, hãy ghép các số liệu thành $6$ nhóm có độ dài bằng nhau trong đó có nhóm $[6{,}5; 7{,}0)$. Uớc lượng mốt của mẫu số liệu ghép nhóm (kết quả làm tròn đến hàng phần trăm).
	\shortans{$8{,}17$}
	\loigiai{
		Ghép các số liệu thành $6$ nhóm có độ dài bằng nhau trong đó có nhóm $[6{,}5; 7{,}0)$ ta được mẫu số liệu ghép nhóm sau
		\begin{center}
			\begin{tabular}{|c|c|c|c|c|c|c|}
				\hline
				Chiều cao (m) & $[6{,}5;7{,}0)$ & $[7{,}0;7{,}5)$ & $[7{,}5;8{,}0)$ & $[8{,}0;8{,}5)$ & $[8{,}5;9{,}0)$ & $[9{,}0;9{,}5)$ \\
				\hline
				Số cây & $2$ & $4$ & $9$ & $11$ & $7$ & $2$ \\
				\hline
			\end{tabular}
		\end{center}
		Tần số lớn nhất của mẫu dữ liệu ghép nhóm trên là $11$ nên nhóm chứa mốt là nhóm $[8{,}0;8{,}5)$.\\
		Ta có $j=4$, $a_4=8{,}0$, $m_4=11$, $m_3=9$, $m_5=7$, $h=0{,}5$. Do đó
		$$M_o=a_j+\dfrac{m_j-m_{j-1}}{\left(m_j-m_{j-1}\right)+\left(m_j-m_{j+1}\right)} \cdot h=8+\dfrac{11-9}{(11-9)+(11-7)}\cdot 0{,}5=\dfrac{49}{6}\approx8{,}17.$$
		
	}
\end{ex}
\begin{ex}%%[1D5H2-3]%[Dự án D đợt 4 - Nguyễn Hoàng Anh]%[Đề ôn tập Chương V. THỐNG KÊ - Khối 11 - Đề số 3]
	Thời gian (phút) truy cập Internet mỗi buổi tối của một số học sinh được cho trong bảng sau
	\begin{center}
		\begin{tabular}{|c|c|c|c|c|c|}
			\hline 	Thời gian (phút) & {$[9{,}5;12{,}5)$} & {$[12{,}5;15{,}5)$} & {$[15{,}5;18{,}5)$} & {$[18{,}5;21{,}5)$} & {$[21{,}5;24{,}5)$} \\
			\hline Số học sinh & $3$ & $12$ & $15$ & $24$ & $2$ \\
			\hline
		\end{tabular}
	\end{center}
	Gọi $Q_1$, $Q_2$, $Q_3$ lần lượt là các tứ phân vị của mẫu số liệu ghép nhóm trên. Tính giá trị biểu thức $\Delta_Q=Q_3-Q_1$.
	\shortans{$4{,}75$}
	\loigiai{
		Cỡ mẩu $n=3+12+15+24+2=56$.\\
		Gọi $x_1$; $x_2$; $\ldots$; $x_{56}$ là thời gian truy cập Internet mỗi buổi tối của $56$ học sinh và giả sử dãy này được sắp xếp theo thứ tự không giảm.\\
		Ta có 
		\begin{itemize}
		\item $x_1$, $x_2$, $x_3 \in \left[9{,}5;12{,}5\right)$; 
		\item $x_4$, $\ldots$, $x_{15}\in \left[12{,}5;15{,}5\right)$; 
		\item $x_{16}$, $\ldots$, $x_{30} \in \left[15{,}5;18{,}5\right)$; 
		\item $x_{31}$, $\ldots$, $x_{54} \in \left[18{,}5;21{,}5\right)$; 
		\item $x_{55}$, $x_{56} \in \left[21{,}5;24{,}5\right)$.
		\end{itemize}
		Tứ phân vị thứ nhất của mẫu số liệu là $\dfrac{1}{2} \left(x_{14}+x_{15} \right)\in \left[12{,}5;15{,}5\right)$.\\
		Do đó, tứ phân vị thứ nhất của mẫu số liệu ghép nhóm là
		\[Q_1=12{,}5+\dfrac{\dfrac{56}{4}-3}{12} (15{,}5-12{,}5)=15{,}25.\]
		Tứ phân vị thứ ba của mẫu số liệu là $\dfrac{1}{2} \left(x_{42}+x_{43} \right)\in \left[18{,}5;21{,}5\right)$.\\
		Do đó, tứ phân vị thứ ba của mẫu số liệu ghép nhóm là
		\[Q_3= 18{,}5+\dfrac{\dfrac{3\cdot 56}{4}-(3+12+15)}{24} (21{,}5-18{,}5)=20.\]
		Vậy \[\Delta_Q=20-15{,}25=4{,}75.\]
	}
\end{ex}
\begin{ex}%[1D5H2-2]%[Dự án D đợt 4 - Nguyễn Hoàng Anh]%[Đề ôn tập Chương V. THỐNG KÊ - Khối 11 - Đề số 3]
	Mẫu số liệu dưới đây ghi lại tốc độ của $40$ ô tô khi đi qua một trạm đo tốc độ (đơn vị: km/h).
	\begin{center}
		\begin{tabular}{cccccccccc}
			$49$ & $42$ & $51$ & $55$ & $45$ & $60$ &$53$ & $55$& $44$ & $65$ \\
			$5$2 &$62$ &$41$ &$44$& $57$ &$56$& $68$&$48$ &$46$&$53$ \\
			$63$ &$49$& $54$ &$61$ &$59$& $57$& $47$& $50$&$60$&$62$\\
			$48$ &$52$& $58$& $47$& $60$& $55$& $45$&$47$& $48$&$61$\\
		\end{tabular}
	\end{center}
	Sau khi ghép nhóm mẫu số liệu trên thành sáu nhóm ứng với sáu nửa khoảng
	\[
	[40;45), \,\, [45;50),\,\, [50;55),\,\, [55;60),\,\, [60;65),\,\, [65;70)
	\]
	thì trung vị của mẫu số liệu ghép nhóm nhận được bằng $\dfrac{a}{b}$ (km/h) ($\dfrac{a}{b}$ là phân số tối giản). Khi đó giá trị của $a$ bằng bao nhiêu?
	\shortans{$375$}
	\loigiai{
		Ta có 
		\begin{center}
			\begin{tabular}{|c|c|c|c|c|c|c|}
				\hline
				Nhóm & $[40;45)$& $[45;50)$&$[50;55)$&$[55;60)$&$[60;65)$&$[65;70)$\\
				\hline
				Tần số &$4$& $11$&$7$&$8$&$8$&$2$\\
				\hline
			\end{tabular}
		\end{center}
		Gọi $x_1;\ldots;x_{40}$ (km/h) là vận tốc xếp không giảm của $40$ ô tô.\\
		Ta có $x_1;\ldots;x_4\in [40;45)$; $x_5;\ldots;x_{15}\in [45;50)$; $x_{16};\ldots;x_{22}\in [50;55)$; \\$x_{23};\ldots;x_{30}\in [55;60)$; $x_{31};\ldots;x_{38}\in [60;65)$; $x_{39};x_{40}\in [65;70)$.\\
		Suy ra trung vị của mẫu số liệu $x_1;\ldots; x_{40}$ là $\dfrac{1}{2}\left(x_{20}+x_{21}\right)\in [50;55)$.\\
		Ta xác định được $n=40$; $n_m=7$; $C=15$; $u_{m}=50$; $u_{m+1}=55$.\\
		Trung vị của mẫu số liệu ghép nhóm là
		\[
		M_\mathrm{e}= 50+\dfrac{\dfrac{40}{2}-15}{7}\left(55-50\right) =\dfrac{375}{7} \,\, (\text{km/h}).
		\]
	}
\end{ex}

\Closesolutionfile{ans}

\begin{center}
	\textbf{PHẦN 4 - TỰ LUẬN}
\end{center}
\begin{bt}%[1D5V2-3]%[Dự án D đợt 4 - Nguyễn Hoàng Anh]%[Đề ôn tập Chương V. THỐNG KÊ - Khối 11 - Đề số 3]
	\immini{Biểu đồ trong hình bên thể hiện điểm kiểm tra học kỳ $2$ môn Toán của $501$ học sinh khối $12$ một trường THPT. Điểm trung bình môn Toán của các học sinh đó bằng bao nhiêu? (Kết quả làm tròn đến hàng phần trăm).}{\begin{tikzpicture}[scale=0.7, font=\footnotesize, line join=round, line cap=round, >=stealth]
			% \draw[](6,9.6)node[scale=1.2]{Thời gian trong ngày của Nam };
			\coordinate [label= left: $0$](O) at (0,0) ;
			%\coordinate [label= left: (mm)](y) at (0,9.5) ;
			\coordinate [label= right: Số học sinh ](y) at (0,6) ;
			\coordinate [label= above: Điểm ](x) at (13.3,0) ;
			%\coordinate [label= below right: $M$](M) at ($(B)!0.5!(C)$) ;
			%		\draw[](0,8)node[left]{$40$};
			%		\draw[](0,7)node[left]{$35$};
			%		\draw[](0,6)node[left]{$30$};
			\draw[](0,5)node[left]{$250$};
			\draw[](0,4)node[left]{$200$};
			\draw[](0,3)node[left]{$150$};
			\draw[](0,2)node[left]{$100$};
			\draw[](0,1)node[left]{$50$};
			%
			\draw[-,color=gray!50] (0,1)--(12.5,1);
			\draw[-,color=gray!50] (0,2)--(12.5,2);
			\draw[-,color=gray!50] (0,3)--(12.5,3);
			\draw[-,color=gray!50] (0,4)--(12.5,4);
			\draw[-,color=gray!50] (0,5)--(12.5,5);
			%		\draw[-,color=gray!50] (0,6)--(12.5,6);
			%		\draw[-,color=gray!50] (0,7)--(11,7);
			%		\draw[-,color=gray!50] (0,8)--(11,8);
			%		\draw[-,color=gray!50] (0,9)--(11,9);
			%
			\draw[](1.5,0.5)node[above]{$25$};
			\draw[](3.5,1)node[above]{$50$};
			\draw[](5.5,2.03)node[above]{$102$};
			\draw[](7.5,4.03)node[above]{$202$};
			\draw[](9.5,2.07)node[above]{$112$};
			\draw[](11.5,0.2)node[above]{$10$};
			%
			\draw[](1.5,0)node[below]{[4;5)};
			\draw[](3.5,0)node[below]{[5;6)};
			\draw[](5.5,0)node[below]{[6;7)};
			\draw[](7.5,0)node[below]{[7;8)};
			\draw[](9.5,0)node[below]{[8;9)};
			\draw[](11.5,0)node[below]{[9;10)};
			
			\draw[fill=blue!80] (1,0) rectangle (2,0.5);
			\draw[fill=blue!80] (3,0) rectangle (4,1);
			\draw[fill=blue!80] (5,0) rectangle (6,2.03);
			\draw[fill=blue!80] (7,0) rectangle (8,4.03);
			\draw[fill=blue!80] (9,0) rectangle (10,2.07);
			\draw[fill=blue!80] (11,0) rectangle (12,0.2);
			\draw[->] (O)--(x);
			\draw[->] (O)--(y);
	\end{tikzpicture}}
	\loigiai{
	Từ biểu đồ hình cột ta có bảng tần số ghép nhóm:
		\begin{center}
		\begin{tabular}{|c|c|c|c|c|c|c|}
			\hline
			Điểm &$[4;5)$&$[5;6)$&$[6;7)$&$[7;8)$&$[8;9)$&$[9;10)$\\
			\hline
			Giá trị đại diện&$4{,}5$&$5{,}5$&$6{,}5$&$7{,}5$&$8{,}5$&$9{,}5$\\
			\hline
			Tần số&$25$&$50$&$102$&$202$&$112$&$10$\\
			\hline
		\end{tabular}
	\end{center}
Điểm trung bình môn Toán của $501$ học sinh lớp $12$ đó là
\begin{center}
	$\overline{x}=\dfrac{4{,}5 \cdot 25 + 5{,}5 \cdot 50 + 6{,}5 \cdot 102 + 7{,}5 \cdot 202 + 8{,}5\cdot 112 + 9{,}5 \cdot 10}{501}\approx 7{,}21$.
\end{center}
}

\end{bt}
\begin{bt}%[1D5H2-3]%[Dự án đề kiểm tra Toán 11 GHKI NH23-24- Tacgia]%[THPT - Tp HCM]%[THPT Quang Trung- Hải Dương]
	An tìm hiểu hàm lượng chất béo (đơn vị: g) có trong $100\,$g mỗi loại thực phẩm. Sau khi thu thập dữ liệu về $60$ loại thực phẩm, An lập bảng thống kê sau
	\begin{center}
		\begin{tabular}{|c|c|c|c|c|c|c|}
			\hline
			Hàm lượng chất béo (g) &$[2;6)$&$[6;10)$&$[10;14)$&$[14;18)$&$[18;22)$&$[22;26)$\\
			\hline
			Tần số&$2$&$6$&$10$&$13$&$16$&$13$\\
			\hline
		\end{tabular}
	\end{center}
	Xác định giá trị trung bình, trung vị, tứ phân vị của mẫu số liệu.
	\loigiai{
		\begin{center}
			\begin{tabular}{|c|c|c|c|c|c|c|}
				\hline
				Hàm lượng chất béo (g) &$[2;6)$&$[6;10)$&$[10;14)$&$[14;18)$&$[18;22)$&$[22;26)$\\
				\hline
				Giá trị đại diện&$4$&$8$&$12$&$16$&$20$&$24$\\
				\hline
				Tần số&$2$&$6$&$10$&$13$&$16$&$13$\\
				\hline
			\end{tabular}
		\end{center}	
		Gọi $x_1,x_2,\ldots,x_{60}$ là hàm lượng chất béo của $60$ loại thực phẩm được xếp theo thứ tự không giảm.\\
		Giá trị trung bình $\bar{x}=\dfrac{4\cdot 2+8\cdot 6+12\cdot 10+16\cdot 13+20\cdot 16+24\cdot 13}{60}=16{,}9$.\\
		Ta có
		\begin{itemize}
		 \item $x_1,x_2\in [2;6)$; \item $x_3,\ldots,x_8\in [6;10)$;
		 \item $x_9,\ldots,x_{18}\in [10;14)$; 
		 \item $x_{19},\ldots,x_{31}\in [14;18)$;
		 \item $x_{32},\ldots,x_{47}\in [18;22)$; 
		 \item $x_{48},\ldots,x_{60}\in [22;26)$.
		 \end{itemize}
		  Do đó trung vị của mẫu số liệu là $\dfrac{1}{2} \left(x_{30}+x_{31}\right)\in [14;18)$.\\
		Suy ra trung vị $M_e=14+\dfrac{\dfrac{60}{2}-18}{13}\cdot (18-14)=17{,}7$.\\
		Tứ phân vị thứ hai $Q_2=M_e=17{,}7$.\\
		Tứ phân vị thứ nhất là $\dfrac{1}{2}\left(x_{15}+x_{16} \right)\in [10;14)\Rightarrow Q_1=10+\dfrac{\dfrac{60}{4}-8}{10}\left(14-10\right)=12{,}8$.\\
		Tứ phân vị thứ ba là $\dfrac{1}{2}\left(x_{45}+x_{46}\right)\in[18;22)\Rightarrow Q_3=18+\dfrac{\dfrac{3}{4}.60-31}{16}\left(22-18\right)=21{,}5$.
	}
\end{bt}
	\begin{bt}%[1D5H2-3]%[Dự án D đợt 4 - Nguyễn Hoàng Anh]%[Đề ôn tập Chương V. THỐNG KÊ - Khối 11 - Đề số 3]
	Một công ty viễn thông đã tính điểm chỉ số hài lòng của khách hàng (thang điểm $100$) cho $150$ đại lý bán hàng của mình và thu được kết quả sau:
	\vspace{-.2cm}
	\begin{center}
		\begin{tabular}{|c|c|c|c|c|c|c}
			\hline
			Điểm &$[0; 20)$&$[20; 40)$&$[40; 60)$&$[60; 80)$&$[80; 100]$\\
			\hline
			Số đại lý &$10$ &$28$ &$52$ &$48$ &$12$\\
			\hline
		\end{tabular}
	\end{center}
	\vspace{-.2cm}
	\hspace{1cm} Hãy xác định điểm ngưỡng để đưa ra danh sách $25 \%$ số đại lý có chỉ số hài lòng của khách hàng tốt nhất (viết dưới dạng số thập phân và làm tròn đến chữ số hàng phần chục).
\shortans{ $69,4$}
	\loigiai{
	Điểm ngưỡng để đưa ra danh sách $25 \%$ số đại lý có chỉ số hài lòng của khách hàng tốt nhất chính là tứ phân vị thứ ba của mẫu số liệu ghép nhóm trên.\\
	Gọi $x_1$, $x_2$, $\ldots$, $x_{150}$ là điểm chỉ số hài lòng của khách hàng được xếp theo thứ tự không giảm.\\
	Tứ phân vị thứ ba là $Q_3=x_{113} \in [60; 80)$.\\
	Do đó $Q_3=60+\dfrac{\dfrac{3}{4}\cdot 150-(10+28+52)}{48}\cdot (80-60)=69{,}375 \approx 69{,}4$.
}
\end{bt}
