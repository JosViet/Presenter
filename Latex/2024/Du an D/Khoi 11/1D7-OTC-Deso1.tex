\newpage
\section{Ôn tập chương 7}
\def\thoigian{90}%--Thời gian
\de{Đề số 1}{Chương VII. Đạo hàm}


\begin{center}
	\textbf{PHẦN 1 - CÂU TRẮC NGHIỆM BỐN PHƯƠNG ÁN}
\end{center}
\Opensolutionfile{ans}[ans/ans-TN-ONTAPCHUONG-VII-DE1]
\begin{ex}
	Hệ số góc của tiếp tuyến của đồ thị hàm số $y=f(x)$ tại điểm $M\left(x_0;y_0\right)$, trong đó $y_0=f\left(x_0\right)$ là
	\choice
	{\True $k=f'\left(x_0\right)$}
	{$k=f'\left(y_0\right)$}
	{$k=\left(x_0\right)$}
	{$k=f\left(y_0\right)$}
	\loigiai{
		Theo định nghĩa, hệ số góc của tiếp tuyến của đồ thị hàm số $y=f(x)$ tại điểm có hoành độ $x_0$ là $k=f'\left(x_0\right)$.
	}
\end{ex}

\begin{ex}
	Tiếp tuyến của đồ thị hàm số $y=f(x)$ tại điểm $M\left(x_0;y_0\right)$, trong đó $y_0=f\left(x_0\right)$ có phương trình là
	\choice
	{$y-y_0=f\left(x_0\right)\left(x-x_0\right)$}
	{$y-y_0=f(x)\left(x-x_0\right)$}
	{$y-y_0=f'\left(y_0\right)\left(x-x_0\right)$}
	{\True $y-y_0=f'\left(x_0\right)\left(x-x_0\right)$}
	\loigiai{
		Phương trình tiếp tuyến của đồ thị hàm số $y=f(x)$ tại điểm $M(x_0;y_0)$ có dạng $y-y_0=f'\left(x_0\right)\left(x-x_0\right)$.
	}
\end{ex}
\begin{ex}
	Cho hàm số $y=f(x)$ xác định trên $\mathbb{R}$ thỏa mãn $\lim\limits_{x \to 3} \dfrac{f(x)-f(3)}{x-3}=2$. Khẳng định nào sau đây đúng?
	\choice
	{$f'(2)=3$}
	{$f'(x)=2$}
	{\True $f'(3)=2$}
	{$f(3)=2$}
	\loigiai{
		Theo định nghĩa đạo hàm tại một điểm, ta có
		\[f'(x_0)=\lim\limits_{x \to x_0} \dfrac{f(x)-f(x_0)}{x-x_0}.\]
		Từ giả thiết $\lim\limits_{x \to 3} \dfrac{f(x)-f(3)}{x-3}=2$, suy ra $f'(3)=2$.
	}
\end{ex}

\begin{ex}
	Cho hàm số $y=f(x)$ có đạo hàm thỏa mãn $f'(6)=2$. Tính $\lim\limits_{x \to 6} \dfrac{f(x)-f(6)}{x-6}$.
	\choice
	{$12$}
	{$\dfrac{1}{6}$}
	{$\dfrac{1}{2}$}
	{\True $2$}
	\loigiai{
		Theo định nghĩa đạo hàm của hàm số $y=f(x)$ tại điểm $x=6$, ta có
		\[f'(6)=\lim\limits_{x \to 6} \dfrac{f(x)-f(6)}{x-6}.\]
		Vì $f'(6)=2$ nên $\lim\limits_{x \to 6} \dfrac{f(x)-f(6)}{x-6}=2$.
	}
\end{ex}

\begin{ex}
	Hệ số góc của tiếp tuyến của đồ thị hàm số $y=f(x)=3x+1$ tại điểm $M(1;4)$ là
	\choice
	{\True $k=3$}
	{$k=4$}
	{$k=5$}
	{$k=13$}
	\loigiai{
		Theo định nghĩa, hệ số góc của tiếp tuyến của đồ thị hàm số $y=f(x)=3x+1$ tại điểm có hoành độ $x=1$ là
		\begin{eqnarray*}
			k&=&f'(1)\\
			&=&\lim\limits_{x \to a}\dfrac{f(x)-f(1)}{x-1}\\
			&=&\lim\limits_{x \to a}\dfrac{3x+1-4}{x-1}\\
			&=&\lim\limits_{x \to a}\dfrac{3(x-1)}{x-1}\\
			&=&\lim\limits_{x \to a} 3\\
			&=&3.
		\end{eqnarray*}
	}
\end{ex}

\begin{ex}
	Cho hàm số $f(x)=\heva{&\dfrac{\sqrt{x^2+1}-1}{x} & \text{khi}\,\, x \neq 0 \\& 0 & \text{khi}\,\, x=0}$. Tính giá trị của $f'(0)$.
	\choice
	{$f'(0)=0$}
	{$f'(0)=1$}
	{\True $f'(0)=\dfrac{1}{2}$}
	{không tồn tại}
	\loigiai{
		Ta có
		\begin{eqnarray*}
			&&\lim\limits_{x\to 0} \dfrac{f\left(x\right)-f\left(0\right)}{x-0} \\
			&=&\lim\limits_{x\to 0} \dfrac{\dfrac{\sqrt{x^2+1}-1}{x}-0}{x} \\
			&=&\lim\limits_{x\to 0} \dfrac{\sqrt{x^2+1}-1}{x^2} \\
			&=&\lim\limits_{x\to 0} \dfrac{\left(\sqrt{x^2+1}-1\right)\left(\sqrt{x^2+1}+1\right)}{x^2\left(\sqrt{x^2+1}+1\right)} \\
			&=&\lim\limits_{x\to 0} \dfrac{x^2+1-1}{x^2\left(\sqrt{x^2+1}+1\right)} \\
			&=&\lim\limits_{x\to 0} \dfrac{1}{\sqrt{x^2+1}+1}\\
			&=&\dfrac{1}{2}.
		\end{eqnarray*}
		Vậy $f'(0)=\dfrac{1}{2}$.
	}
\end{ex}
\begin{ex}
	Trên khoảng $\left(0;+\infty\right)$, hàm số $y=x^4-\ln x$ có đạo hàm là
	\choice
	{$y'=4x^3+\dfrac{1}{x}$}
	{\True $y'=4x^3-\dfrac{1}{x}$}
	{$y'=4x^3+1$}
	{$y'=x^3+\dfrac{1}{x^2}$}
	\loigiai{
		Ta có $y'=\left(x^4 - \ln x\right)'=\left(x^4\right)'-\left(\ln x\right)'= 4x^3-\dfrac{1}{x}$.
	}
\end{ex}

\begin{ex}
	Với mọi $x\neq 1$, hàm số $y=\dfrac{2x+1}{x-1}$ có đạo hàm là
	\choice
	{$y'=2$}
	{$y'=\dfrac{1}{(x-1)^2}$}
	{\True $y'=-\dfrac{3}{(x-1)^2}$}
	{$y'=\dfrac{-1}{(x-1)^2}$}
	\loigiai{
		Áp dụng quy tắc đạo hàm của thương, ta có
		\begin{eqnarray*}
			y'&=&\dfrac{(2x+1)'(x-1)-(2x+1)(x-1)'}{(x-1)^2}\\
			&=&\dfrac{2(x-1)-1(2x+1)}{(x-1)^2}\\
			&=&\dfrac{2x-2-2x-1}{(x-1)^2}\\
			&=&\dfrac{-3}{(x-1)^2}.
		\end{eqnarray*}
	}
\end{ex}

\begin{ex}
	Với mọi $x$ dương, hàm số $y=\log_2 x$ có đạo hàm là
	\choice
	{$y'=2^x \ln 2$}
	{$y'=\dfrac{2}{x}$}
	{$y'=x \ln 2$}
	{\True $y'=\dfrac{1}{x\ln 2}$}
	\loigiai{
		Theo công thức đạo hàm hàm số logarit, $\left(\log_a x\right)'=\dfrac{1}{x\ln a}$.\\
		Do đó, $\left(\log_2 x\right)'=\dfrac{1}{x\ln 2}$.
	}
\end{ex}

\begin{ex}
	Đạo hàm của hàm số $y=x^2-2x-\sin x$ là
	\choice
	{$y'=x-2-\cos x$}
	{$y'=2x-2x-\cos x$}
	{\True $y'=2x-2-\cos x$}
	{$y'=x-2+\cos x$}
	\loigiai{
		Ta có $y'=\left(x^2\right)'-\left(2x\right)'-\left(\sin x\right)' = 2x-2-\cos x$.
	}
\end{ex}

\begin{ex}
	Trên khoảng $\left(0;+\infty\right)$, hàm số $y=2\sqrt{x}-3\sin x$ có đạo hàm là
	\choice
	{\True $y'=\dfrac{1}{\sqrt{x}}-3\cos x$}
	{$y'=\dfrac{1}{\sqrt{x}}+3\cos x$}
	{$y'=\dfrac{1}{2\sqrt{x}}-3\cos x$}
	{$y'=-\dfrac{1}{\sqrt{x}}-3\cos x$}
	\loigiai{
		Ta có $y'=\left(2\sqrt{x}\right)'-\left(3\sin x\right)' = 2\cdot\dfrac{1}{2\sqrt{x}} - 3\cos x = \dfrac{1}{\sqrt{x}}-3\cos x$.
	}
\end{ex}

\begin{ex}
	Với mọi $x\neq \dfrac{\pi}{2}+k\pi$, $k\in \mathbb{Z}$, thì đạo hàm của hàm số $y=x^3-x-\tan x$ là
	\choice
	{\True $y'=3x^2-2-\tan^2 x$}
	{$y'=3x^2+\tan^2 x$}
	{$y'=3x^2-1-\tan^2 x$}
	{$y'=x^2-2-\tan^2 x$}
	\loigiai{
		Ta có $y'=\left(x^3\right)'-\left(x\right)'-\left(\tan x\right)'=3x^2-1-\dfrac{1}{\cos^2 x}$.\\
		Sử dụng đẳng thức $\dfrac{1}{\cos^2 x}=1+\tan^2 x$, ta được
		$y'=3x^2-1-\left(1+\tan^2 x\right) = 3x^2-2-\tan^2 x$.
	}
\end{ex}

\Closesolutionfile{ans}

\begin{center}
	\textbf{PHẦN 2 - CÂU TRẮC NGHIỆM ĐÚNG SAI}
\end{center}
\setcounter{ex}{0}
\Opensolutionfile{ans}[ans/answer-DS-ONTAPCHUONG-VII-DE1]
\begin{ex}
	Cho hai hàm số $ f(x)=\dfrac{3}{x+1}$ và $ g(x)=\dfrac{x^2}{x+2}$. 
	\choiceTF
	{\True $f'(x)=-\dfrac{3}{\left(x+1\right)^2}$, $\forall x\ne-1$}
	{$g'(1)=-1$}
	{\True $\left[g(x)\right]'=\dfrac{x^2+4x}{\left(x+2\right)^2}$, $\forall x\ne-2$}
	{ $\left[f(x)\cdot g(x)\right]'=f'(x)\cdot g'(x)$ ,$\forall x\in \mathbb{R} \setminus \left\{-1,-2\right\}$}
	\loigiai{
		\begin{itemchoice}
			\itemch $f'(x)=-\dfrac{3}{\left(x+1\right)^2}$, $\forall x\ne-1$.
			\itemch $g'(1)=\dfrac{5}{9}$.
			\itemch  $\left[g(x)\right]'=\dfrac{2x\cdot \left(x+2\right)-x^2}{\left(x+2\right)^2}=\dfrac{x^2+4x}{\left(x+2\right)^2}$, $\forall x\ne-2$.
			\itemch $\left[f(x).g(x)\right]'=f'(x)\cdot g'(x)+f(x)\cdot g'(x)$, $\forall x\in \mathbb{R} \setminus \left\{-1;-2\right\}$.
	\end{itemchoice}}
\end{ex}
\begin{ex}
	Một chuyển động thẳng có quãng đường di chuyển được xác định bởi phương trình $s(t)=2t^2+t-1$, trong đó $s$ tính bằng mét và $t$ tính bằng giây.
	\choiceTF
	{Tại thời điểm $t=2$ tốc độ tức thời của chuyển động là $10$ m/s}
	{$s(3)-s'(1)=3$}
	{\True Tốc độ tức thời của chuyển động tại thời điểm $t_0$ là $s'(t_0)={\lim\limits_{t\to t_0}} \dfrac{s(t)-s(t_0)}{t-t_0}$}
	{\True Phương trình $s(t)-\left[\left(t+1\right)\cdot s(t)\right]'+27=0$ có $2$ nghiệm trái dấu}
	\loigiai{
		\begin{itemchoice}
			\itemch Tốc độ tức thời tại thời điểm $t=2$ là \allowdisplaybreaks
			\begin{eqnarray*}
			v(2)&=&s'(2)\\
			&=&\lim\limits_{t\to 2} \dfrac{s(t)-s(2)}{t-2}\\
			&=&{\lim\limits_{t\to 2}} \dfrac{2t^2+t-1-9}{t-2}\\
			&=&{\lim\limits_{t\to 2}} \dfrac{\left(t-2\right)\cdot\left(2t+5\right)}{t-2}\\
			&=&{\lim\limits_{t\to 2}} \left(2t+5\right)\\
			&=&9.
			\end{eqnarray*}
			\itemch Ta có $s(3)=2\cdot3^2+3-1=20$.\\
			 Lại có $s'(t)=4t+1\Rightarrow s'(1)=5$.\\
			 Suy ra $s(3)-s'(1)=20-5=15$.
			\itemch Tốc độ tức thời của chuyển động tại thời điểm $t_0$ là $s'(t_0)={\lim\limits_{t\to t_0}} \dfrac{s(t)-s(t_0)}{t-t_0}$.
			\itemch Ta có
			\begin{eqnarray*}
				&&\left[\left(t+1\right)\cdot s(t)\right]'\\
				&=&\left(t+1\right)' s(t)+\left(t+1\right)s'(t)\\
				&=&1\cdot \left(2t^2+t-1\right)+\left(t+1\right)\left(4t+1\right)\\
				&=&2t^2+t-1+4t^2+5t+1\\
				&=&6t^2+6t.
			\end{eqnarray*}
			Xét phương trình
			\begin{eqnarray*}
				&&s(t)-\left[\left(t+1\right)\cdot s(t)\right]'+ 27=0\\
				&\Leftrightarrow	& 2t^2+t-1-\left(6t^2+6t\right)+27=0\\
				&\Leftrightarrow	&-4t^2-5t+26=0\\
				&\Leftrightarrow	& \hoac{&{t=2} \\
					&{t=\dfrac{-13}{4}}.}
			\end{eqnarray*}
			Vậy phương trình đã cho có $2$ nghiệm trái dấu.
		\end{itemchoice}
	}
\end{ex}
\Closesolutionfile{ans}
%\inputansbox[2]{2}{ans/answer.tex}



\begin{center}
	\textbf{PHẦN 3 - CÂU TRẮC NGHIỆM TRẢ LỜI NGẮN}
\end{center}
\setcounter{ex}{0}
\Opensolutionfile{ans}[ans-KQ-ONTAPCHUONG-VII-DE1]
\begin{ex}
	Một chất điểm chuyển động có quãng đường được cho bởi phương trình $s(t)=\dfrac{1}{6} t^4-\dfrac{4}{3} t^3+5t^2-7$, trong đó $t > 0$ với $t$ tính bằng giây (s), $s$ tính bằng mét (m). Vận tốc chuyển động của chất điểm tại thời điểm chất điểm có gia tốc chuyển động nhỏ nhất là $\dfrac{a}{b}$ với $\dfrac{a}{b}$ là phân số tối giản và $a,b\in \mathbb{Z}$. Tính $T=a-2b$.
	\shortans[oly]{22}
	\loigiai{
		Ta có vận tốc chuyển động của chất điểm tại thời điểm $t$ là $v(t)=s'(t)=\dfrac{2}{3} t^3-4t^2+10t$;\\
		Gia tốc chuyển động của chất điểm tại thời điểm $t$ là $a(t)=v'(t)=2t^2-8t+10=2\left(t^2-4t+4\right)+2=2\left(t-2\right)^2+2\ge 2$.\\
		Dấu \lq\lq $=$ \rq\rq\, xảy ra khi $t=2$.\\
		Suy ra gia tốc chuyển động của chất điểm có giá trị nhỏ nhất là $2$ khi $t=2$ (s). \\
		Khi đó vận tốc chuyển động của chất điểm là
		$v(2)=\dfrac{2}{3}\cdot 2^3-4\cdot 2^2+10\cdot 2=\dfrac{28}{3}$ (m/s).\\
		Khi đó $\heva{	&{a=28} \\
			&{b=3}
			&} \Rightarrow T=a-2b=28-2\cdot 3=22$.
	}
\end{ex}
\begin{ex}
	Gọi $M\left(x_0; y_0\right)$ là điểm trên đồ thị hàm số $y=x^3-3x^2-1$ mà tiếp tuyến tại đó có hệ số góc bé nhất trong các tiếp tuyến của đồ thị hàm số. Khi đó $x_0^2+y_0^2$ bằng bao nhiêu?
	\shortans[oly]{10}
	\loigiai{Ta có $y'=3x^2-6x$. \\
		Hệ số góc của tiếp tuyến tại $M$ là $k=y'(x_0)=3x_0^2-6x_0$.\\
		Ta có $3x_0^2-6x_0=3\left(x_0^2-2x_0+1\right)-4=3\left(x_0-1\right)^2-3\geq-3$.\\
		Suy ra $k_{\min}=-3$ khi $x_0=1$, khi đó $y_0=-3$.\\
		Vậy $x_0^2+y_0^2=1^2+(-3)^2=10$.}
\end{ex}
\begin{ex}
	Một chuyển động theo quy luật là $S(t)=-t^3+6t^2+3t+9$ với $t$ (giây) là khoảng thời gian tính từ khi vật bắt đầu chuyển động và $s$ (mét) là quãng đường vật di chuyển được trong khoảng thời gian đó. Tính quãng đường vật đi được bắt đầu từ lúc vật chuyển động tới thời điểm vật đạt được vận tốc lớn nhất.
	\shortans[oly]{22}
	\loigiai{
		Ta có $S(t)=-t^3+6t^2+3t+9$, suy ra
		\begin{eqnarray*}
			v(t)&=&S'(t)\\
			&=&-3t^2+12t+3\\
			&=&-3(t-2)^2+15\leq 15, \forall t\in\mathbb{R}.
		\end{eqnarray*}
		Vậy vận tốc đạt được giá trị lớn nhất tại thời điểm $t=2$ (giây).\\
		Khi đó quãng đường vật đi được là $s(2)-s(0)=31-9=22$ (mét).
	}
\end{ex}
\begin{ex}
	Một con lắc lò xo dao động điều hòa theo phương ngang trên mặt phẳng không ma sát, có phương trình chuyển động là $x(t)=4\cos\left(\pi t - \dfrac{2\pi}{3}\right)+4 \text{ (cm)}$, trong đó $t$ là thời gian tính bằng giây. Tìm thời điểm đầu tiên mà vận tốc tức thời của con lắc bằng $0$ (làm tròn kết quả đến chữ số thập phân thứ hai).
	\shortans[oly]{0{,}67}
	\loigiai{
		Phương trình vận tốc của con lắc là đạo hàm của phương trình li độ theo thời gian $t$ là
		\[v(t) = x'(t) = -4\pi\sin\left(\pi t - \dfrac{2\pi}{3}\right).\]
		Vận tốc của con lắc bằng $0$ khi $v(t)=0$, tức là
		\allowdisplaybreaks
		\begin{eqnarray*}
			& & -4\pi\sin\left(\pi t - \dfrac{2\pi}{3}\right) = 0 \\
			&\Leftrightarrow& \sin\left(\pi t - \dfrac{2\pi}{3}\right) = 0 \\
			&\Leftrightarrow& \pi t - \dfrac{2\pi}{3} = k\pi, (k \in \mathbb{Z}) \\
			&\Leftrightarrow& t = \dfrac{2}{3} + k.
		\end{eqnarray*}
		Thời điểm đầu tiên ứng với giá trị $k=0$, suy ra thời điểm đầu tiên là $t=\dfrac{2}{3}\approx 0{,}67$ (giây).
	}
\end{ex}
\Closesolutionfile{ans}

\begin{center}
	\textbf{PHẦN 4 - TỰ LUẬN}
\end{center}
\setcounter{ex}{0}
\begin{ex}
	Một khối lập phương tại thời điểm ban đầu có thể tích bằng $8$ (cm$^3$). Mỗi giây thể tích khối lập phương tăng thêm $8$ (cm$^3$). Hỏi khi cạnh hình lập phương là $12$ cm thì tốc độ thay đổi diện tích toàn phần là bao nhiêu cm$^2$/s (làm tròn đến hàng đơn vị)?
	\loigiai{
		Thể tích của khối lập phương tính bởi $V(t)=8t+8$ (cm$^3$).\\ Cạnh hình lập phương là $x(t)=2\sqrt[3]{t+1}$. \\
		Suy ra diện tích toàn phần của hình lập phương là $S(t)=24\sqrt[3]{(t+1)^2}$.\\		
		Hình lập phương có cạnh bằng $12$ cm khi
		\[ 2\sqrt[3]{t+1}=12 \Leftrightarrow t=215\,\, (\text{ s}). \]
		Ta có
			\allowdisplaybreaks
		\begin{eqnarray*}
			\lim\limits_{t\to 215} \dfrac{S(t)-S(215)}{t-215}
			&=& 24\cdot \lim\limits_{t\to 215}\dfrac{\sqrt[3]{(t+1)^2}-\sqrt[3]{216^2}}{t-215} \\
			&=& 24\cdot \lim\limits_{t\to 215} \dfrac{(t-215)(t+217)}{(t-215)\left[\sqrt[3]{(t+1)^4}+\sqrt[3]{(t+1)^2\cdot 216^2}+\sqrt[3]{216^4}\right]} \\
			&=& 24\cdot \lim\limits_{t\to 215} \dfrac{t+217}{\left[\sqrt[3]{(t+1)^4}+\sqrt[3]{(t+1)^2\cdot 216^2}+\sqrt[3]{216^4}\right]} \\
			&=& \dfrac{8}{3}.
		\end{eqnarray*}
		Vậy tốc độ thay đổi diện tích toàn phần của khối lập phương là $2{,}7$ cm$^2$/s.
	}
\end{ex}
\begin{ex}
	\immini[thm]{
		Độ dốc của mộ con đường được tính bởi góc tạo bởi tiếp tuyến của đường cong mặt đường tại điểm đang xét và phương nằm ngang. Góc $90^\circ$ được coi là độ dốc $100\%$.	Xét một con dốc được mô hình bởi đồ thị hàm số $f(x)=\dfrac{x^2}{200}$, $0\le x\le 25$. Độ dốc tại điểm $A(20;2)$ là bao nhiêu phần trăm? (kết quả làm tròn đến hàng phần chục). Biết rằng $\alpha$ là góc kẻ từ tia $Ox$ đến tiếp tuyến của đường cong $y=f(x)$ tại điểm $M\left(x_0;y_0\right)$ theo ngược chiều kim đồng hồ thì $\tan \alpha=f'\left(x_0\right)$.
	}{
		\begin{tikzpicture}[line cap=round, line join=round,font=\footnotesize,>=stealth, scale=0.7,x=0.4cm, y=1cm]
			\tikzset{label style/.style={font=\footnotesize}}
			\draw[->] (0,0)--(25,0) node[below]{$x$};
			\draw[->] (0,0)--(0,4) node[left]{$y$};
			\draw[smooth, samples=100] plot[domain=0:25] (\x, { (\x)^2/200 });
			\draw[fill=black] (0,0) circle (1.2pt) node[below left]{$O$};
			\draw[fill=black] (20,2) circle (1.2pt) node[above]{$A$};
			\draw[dashed] (20,0) node[below]{$20$}--(20,2)--(0,2) node[left]{$2$};
		\end{tikzpicture}
	}
	\loigiai{
		Ta có
		\begin{eqnarray*}
			f'(20)&=&\lim\limits_{x\to 20}\dfrac{f(x)-f(20)}{x-20}\\
			&=&\lim\limits_{x\to 20}\dfrac{\dfrac{x^2}{200}-2}{x-20}\\
			&=&\lim\limits_{x\to 20}\dfrac{(x-20)(x+20)}{200(x-20)}\\
			&=&\lim\limits_{x\to 20}\dfrac{x+20}{200}\\
			&=&\dfrac{1}{5}. 
		\end{eqnarray*}
		Do đó hệ số góc của tiếp tuyến của con dốc tại $A$ là $k=f'(20)=\dfrac{1}{5}$.\\ 
		Suy ra góc $\alpha$ kẻ từ tia $Ox$ đến tiếp tuyến nói trên theo ngược chiều kim đồng hồ thỏa mãn $\tan \alpha=\dfrac{1}{5}$, hay $\alpha\approx 11{,}3^\circ$.\\		
		Vậy độ dốc cần tìm là $\dfrac{\alpha}{90^\circ}\approx \dfrac{11{,}3^\circ}{90^\circ}\approx 12{,}6\%$.
	}
\end{ex}

\begin{ex}
	Chứng minh rằng $\mathrm{C}_n^1+2\mathrm{C}_n^2+3\mathrm{C}_n^3+\cdots+n\mathrm{C}_n^n=n\cdot 2^{n-1}$ (với $n$ nguyên dương).
	\loigiai{
		Xét khai triển nhị thức $(1+x)^n=\mathrm{C}_n^0+\mathrm{C}_n^1x+\mathrm{C}_n^2x^2+\cdots+\mathrm{C}_n^nx^n$. \\
		Đạo hàm hai vế ta được $$n(1+x)^{n-1}=\mathrm{C}_n^1+2\mathrm{C}_n^2x+3\mathrm{C}_n^3x^2+\cdots+n\mathrm{C}_n^nx^{n-1}.$$
		Chọn $x=1$, ta được $$n(1+1)^{n-1}=\mathrm{C}_n^1+2\mathrm{C}_n^2+3\mathrm{C}_n^3+\ldots+n\mathrm{C}_n^n
		\Leftrightarrow \mathrm{C}_n^1+2\mathrm{C}_n^2+3\mathrm{C}_n^3+\cdots+n\mathrm{C}_n^n=n.2^{n-1}.$$
		Vậy ta có điều phải chứng minh.
	}
\end{ex}
