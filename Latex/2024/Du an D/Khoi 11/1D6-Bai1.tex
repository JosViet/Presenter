\section{PHÉP TÍNH LŨY THỪA}
\subsection{LÝ THUYẾT CẦN NHỚ}
\subsubsection{Lũy thừa với số mũ nguyên}
Nhắc lại lũy thừa với số mũ tự nhiên (ở cấp THCS)
$$a^n=\underbrace{a\cdot a \cdots a}_{n \text { thừa số } a}\ \left(n\in\mathbb{N},\ n>0,\ a\in\mathbb{R}\right); \quad a^0=1\ (a\neq0).$$
\begin{dn}
	Với số nguyên dương $n$, số thực khác $a\neq0$, \textbf{\textit{lũy thừa}} của $a$ với số mũ $-n$ xác định bởi
			$$a^{-n}=\dfrac{1}{a^n}.$$
\end{dn}
\begin{luuy}
$0^0$ và $0^{-n}$ (với $n>0$) không có nghĩa.
\end{luuy}
\subsubsection{Căn bậc $n$}
\begin{dn}
	Cho số nguyên dương $n$ ($n\ge2$) và số thực $b$ bất kỳ. Nếu có số thực $a$ sao cho $a^n=b$ thì $a$ được gọi là một \textbf{căn bậc $n$} của $b$.
\end{dn}
\begin{note}
	\begin{itemize}
	\item Khi $n$ là số nguyên dương lẻ, mỗi số thực $b$ chỉ có duy nhất một căn bậc $n$ và kí hiệu là $\sqrt[n]{b}$.
	\item Khi $n$ là số nguyên dương chẵn, mỗi số thực dương $b$ có đúng hai căn bậc $n$ là hai số đối nhau, giá trị dương kí hiệu là $\sqrt[n]{b}$, giá trị âm kí hiệu là $-\sqrt[n]{b}$.
	\item Khi $n$ là số nguyên dương chẵn, số $0$ có một căn bậc $n$ là $0$ và số âm không tồn tại căn bậc $n$.	
\end{itemize}	 
\end{note}
\begin{luuy}
	Nếu $n$ lẻ thì căn thức $\sqrt[n]{b}$ luôn có nghĩa với mọi số thực $b$;	\\
	Nếu $n$ chẵn thì căn thức $\sqrt[n]{b}$ luôn có nghĩa chỉ khi $b\ge0$.	
\end{luuy}
\begin{tc}
	(Với điều kiện các căn thức đều có nghĩa).
$$\begin{array}{c@{\hskip 2.5em} c@{\hskip 2.5em} c@{\hskip 2.5em} c@{\hskip 2.5em} c}
\bullet	\sqrt[n]{a}\cdot\sqrt[n]{b} = \sqrt[n]{ab} &
\bullet	\dfrac{\sqrt[n]{a}}{\sqrt[n]{b}} = \sqrt[n]{\dfrac{a}{b}} &
\bullet	\left(\sqrt[n]{a}\right)^m = \sqrt[n]{a^m} &
\bullet	\sqrt[n]{\sqrt[k]{a}} = \sqrt[nk]{a} &
\bullet \sqrt[n]{a^n} =
	\begin{cases}
		a & \text{khi } n \text{ lẻ} \\
		|a| & \text{khi } n \text{ chẵn}
	\end{cases}
\end{array}
$$
\end{tc}
\subsubsection{Lũy thừa với số mũ hữu tỉ}
\begin{dn}
	Cho số thực $a$ dương và số hữu tỉ $r=\dfrac{m}{n}$, trong đó $m$ là một số nguyên và $n$ là số nguyên dương. \textbf{\textit{Lũy thừa }}của $a$ với số mũ $r$, kí hiệu $a^r$, xác định bởi $$a^r=a^{\tfrac{m}{n}}=\sqrt[n]{a^m}.$$
\end{dn}
\subsubsection{Lũy thừa với số mũ thực}
Cho $a$ là số thực dương và $\alpha$ là một số vô tỉ bất kỳ. Khi đó có dãy số hữu tỉ $(r_n)$ mà $\lim\limits_{n\to +\infty}r_n=\alpha$ và dãy số $\left(a^{r_n}\right)$ có giới hạn không phụ thuộc việc chọn $(r_n)$.
\begin{dn}	
	Giới hạn của dãy số $\left(a^{r_n}\right)$ được gọi là \textbf{\textit{lũy thừa}} của số thực dương $a$ với số mũ $\alpha$, kí hiệu là $a^{\alpha}$.
	$$a^{\alpha}=\lim\limits_{n\to +\infty}a^{r_n} \text{ với}\ \alpha=\lim\limits_{n\to +\infty}r_n.$$
\end{dn}
\subsubsection{Tính chất của phép tính lũy thừa}
	Phép tính lũy thừa với số mũ thực có tính chất tương tự như lũy thừa với số mũ tự nhiên.
	\begin{tc}
	Với $a$, $b$ là những số thực dương; $\alpha$, $\beta$ là những số thực bất kỳ. Khi đó
	\begin{multicols}{5}
		\begin{itemize}
			\item $(ab)^\alpha=a^\alpha b^\alpha$
			\item $\left(\dfrac{a}{b}\right)^\alpha=\dfrac{a^\alpha}{b^\alpha}$
			\item $a^\alpha\cdot a^\beta=a^{\alpha+\beta}$
			\item $\dfrac{a^\alpha}{a^\beta}=a^{\alpha-\beta}$
			\item $(a^\alpha)^\beta=a^{\alpha \beta}$
		\end{itemize}
	\end{multicols}
\end{tc}
\begin{luuy}
	Với mọi $\alpha \in \mathbb{R}$, ta có $1^\alpha=1$.\\
	Với $a>1$, ta có $a^\alpha>1\Leftrightarrow \alpha>0$.\\
	Với $0<a<1$, ta có $a^\alpha>1\Leftrightarrow \alpha<0$.
\end{luuy}
%-------------------------------------------------------------------------------------------------------------
\subsection{PHÂN LOẠI VÀ PHƯƠNG PHÁP GIẢI TOÁN}
%%Dang1%%=========
\begin{dang}{Tính giá trị của biểu thức chứa lũy thừa}
	Sử dụng phối hợp linh hoạt các định nghĩa và các tính chất của lũy thừa.
	\begin{multicols}{3}
		\begin{itemize}
		\item $a^{-n}=\dfrac{1}{a^n}$
		\item $a^{\tfrac{m}{n}}=\sqrt[n]{a^m}$
		\item $a^{\tfrac{1}{n}}=\sqrt[n]{a}$		
		\item $a^\alpha \cdot a^\beta=a^{\alpha+\beta}$
		\item $(a^\alpha)^\beta=a^{\alpha \cdot\beta}$		
		\item $\dfrac{a^\alpha}{a^\beta}=a^{\alpha-\beta}$
		\item $(ab)^\alpha=a^\alpha \cdot b^\alpha$
		\item $\left( \dfrac{a}{b}\right) ^\alpha=\dfrac{a^\alpha}{b^\alpha}$
		\item $\left( \dfrac{a}{b}\right) ^{-\alpha}=\left(\dfrac{b}{a}\right) ^\alpha$
	\end{itemize}
	\end{multicols}
\end{dang}
%%Vidu1=========
\begin{vd}%[1D6N1-1]%[Dự án đề cương 3 Khối NH24-25-Đợt 2 - BCTuan]
	Tính giá trị các biểu thức sau
		\begin{enumEX}[a)]{4}
		\item $\left(\dfrac{3}{7}\right)^{-1}-\dfrac{3}{4}\cdot \left(\dfrac{9}{4}\right)^{-1}$;
		\item $ \sqrt[3]{-27} $;
		\item $25^{\tfrac{3}{2}} $;
		\item $\left(\dfrac{1}{16}\right)^{-0,75}+\left(\dfrac{1}{8}\right)^{-\tfrac{4}{3}}$.
	\end{enumEX}
	\loigiai{
		\begin{enumerate}[a)]
			\item Ta có $\left(\dfrac{3}{7}\right)^{-1}-\dfrac{3}{4}\cdot \left(\dfrac{9}{4}\right)^{-1}=\dfrac{7}{3}-\dfrac{3}{4}\cdot \dfrac{4}{9}=2$.
			\item Ta có $\sqrt[3]{-27}=\sqrt[3]{(-3)^3}=-3$.
			\item Ta có $ 25^{\tfrac{3}{2}}=\left(5^2\right)^{\tfrac{3}{2}}=5^{2\cdot\tfrac{3}{2}}=5^3=125 $.
			\item Ta có  $\left(\dfrac{1}{16}\right)^{-0{,}75}=\left(\dfrac{1}{16}\right)^{-0{,}75}=\left(2^{-4}\right)^{-0{,}75}=2^{-4\cdot(-0{,}75)}=2^3=8$ \\
			và $\left(\dfrac{1}{8}\right)^{-\tfrac{4}{3}}=\left(\dfrac{1}{8}\right)^{-\tfrac{4}{3}}=\left(2^{-3}\right)^{-\tfrac{4}{3}}=2^{-3\cdot\left(-\tfrac{4}{3}\right)}=2^4=16$. \\
			Vậy $\left(\dfrac{1}{16}\right)^{-0{,}75}+\left(\dfrac{1}{8}\right)^{-\tfrac{4}{3}}=8+16=24$.
		\end{enumerate}
	}
\end{vd}
%%Vidu2=========
\begin{vd}%[1D6H1-1]%[Dự án đề cương 3 Khối NH24-25-Đợt 2 - BCTuan]
	Biết rằng $3^x=2$, tính giá trị của biểu thức $A=3^{2x-1}\cdot{\left(\dfrac{1}{3}\right)}^{2x-1}+9^{x+1}$.
	\loigiai{
	\begin{multicols}{2}
	Ta có 
	\begin{eqnarray*}
			A&=&3^{2x-1}\cdot\left(3^{-1}\right)^{2x-1}+9\cdot 9^x\\
			&=&3^{2x-1}\cdot3^{-2x+1}+9\cdot(3^x)^2\\
			&=&3^{2x-1-2x+1}+9\cdot(3^x)^2\\
			&=&3^0+9\cdot2^2\\
			&=&1+9\cdot4\\
			&=&37.
	\end{eqnarray*}
	Cách khác, từ $3^x=2\Rightarrow 3=2^{\tfrac{1}{x}}$ suy ra
	\begin{eqnarray*}
		A&=&3^{2x-1}\cdot\left(3^{-1}\right)^{2x-1}+9\cdot 3^{2x}\\
		&=&\left(2^{\tfrac{1}{x}}\right)^{2x-1}\cdot\left(2^{\tfrac{1}{x}}\right)^{-2x+1}+9\cdot\left(2^{\tfrac{1}{x}}\right)^{2x}\\
		&=&\left(2^{\tfrac{1}{x}}\right)^0+9\cdot2^2\\
		&=&1+9\cdot4=37.
	\end{eqnarray*}
	\end{multicols}
		}
\end{vd}
\vspace{0.4em}
%%Vidu3=========
\begin{vd}%[1D6V1-1]%[Dự án đề cương 3 Khối NH24-25-Đợt 2 - BCTuan]
Tính giá trị biểu thức $A=\left(7+5\sqrt{2}\right)^{2\,020{,}5}\cdot \left(5\sqrt{2}-7\right)^{2\,019{,}5}$.
\loigiai{
	Ta có
	\begin{eqnarray*}
		A &=& \left(7+5\sqrt{2}\right)\left(7+5\sqrt{2}\right)^{2\,019{,}5} \cdot \left(5\sqrt{2}-7\right)^{2\,019{,}5} \\
		&=& \left(7+5\sqrt{2}\right) \cdot \left[\left(7+5\sqrt{2}\right)\left(5\sqrt{2}-7\right)\right]^{2\,019{,}5} \\
		&=& \left(7+5\sqrt{2}\right) \cdot \left[\left(5\sqrt{2}+7\right)\left(5\sqrt{2}-7\right)\right]^{2\,019{,}5} \\
		&=& \left(7+5\sqrt{2}\right) \cdot \left[\left(5\sqrt{2}\right)^2 - 7^2\right]^{2\,019{,}5} \\
		&=& \left(7+5\sqrt{2}\right) \cdot 1^{2\,019{,}5} \\
		&=& \left(7 + 5\sqrt{2}\right)\cdot 1\\
		&=& 7 + 5\sqrt{2}.
	\end{eqnarray*}
}
\end{vd}
%%Dang2%%=========
\begin{dang}{Rút gọn biểu thức chứa lũy thừa}
	Sử dụng phối hợp linh hoạt các định nghĩa và các tính chất của lũy thừa.
	\begin{multicols}{3}
	\begin{itemize}
		\item $a^{-n}=\dfrac{1}{a^n}$
		\item $a^{\tfrac{m}{n}}=\sqrt[n]{a^m}$
		\item $a^{\tfrac{1}{n}}=\sqrt[n]{a}$		
		\item $a^\alpha \cdot a^\beta=a^{\alpha+\beta}$
		\item $(a^\alpha)^\beta=a^{\alpha \cdot\beta}$		
		\item $\dfrac{a^\alpha}{a^\beta}=a^{\alpha-\beta}$
		\item $(ab)^\alpha=a^\alpha \cdot b^\alpha$
		\item $\left( \dfrac{a}{b}\right) ^\alpha=\dfrac{a^\alpha}{b^\alpha}$
		\item $\left( \dfrac{a}{b}\right) ^{-\alpha}=\left(\dfrac{b}{a}\right) ^\alpha$
	\end{itemize}
\end{multicols}
\end{dang}
%%Vidu4=========
\begin{vd}%[1D6N1-2]%[Dự án đề cương 3 Khối NH24-25-Đợt 2 - BCTuan]
	Viết mỗi biểu thức sau dưới dạng luỹ thừa của $2$ với số mũ $\alpha$.
	\begin{enumEX}[a)]{3}
			\item $\sqrt[4]{2^{-3}}$
			\item $\dfrac{1}{\sqrt[5]{2^3}}$
			\item $\left(\sqrt[5]{4}\right)^{4\tfrac{1}{2}}$
	\end{enumEX}
	\loigiai{
		\begin{enumerate}[a)]
			\item Ta có $\sqrt[4]{2^{-3}}=2^{-\tfrac{3}{4}}$.
			\item Ta có  $\dfrac{1}{\sqrt[5]{2^3}}=\dfrac{1}{2^{\tfrac{3}{5}}}=\left(2^{\tfrac{3}{5}}\right)^{-1}=2^{-\tfrac{3}{5}}$.
			\item Ta có  $\left(\sqrt[5]{4}\right)^{4\tfrac{1}{2}}=\left(\sqrt[5]{2^2}\right)^{\tfrac{9}{2}}=\left(2^{\tfrac{2}{5}}\right)^{\tfrac{9}{2}}=2^{\tfrac{2}{5}\cdot\tfrac{9}{2}}=2^{\tfrac{9}{5}}$.
		\end{enumerate}
	}
\end{vd}
%%Vidu5=========
\begin{vd}%[1D6H1-2]%[Dự án đề cương 3 Khối NH24-25-Đợt 2 - BCTuan]
	Với $a>0$, hãy viết mỗi biểu thức sau về dạng $a^\alpha$. Từ đó suy ra giá trị của $\alpha$.
	\begin{enumEX}{3}
		\item $A =a^{\tfrac{1}{6}} \sqrt[3]{a}$;
		\item $B=a^{\tfrac{5}{3}}\colon \sqrt[3]{a}$;
		\item $C= \dfrac{ \sqrt[3]{ a^2 \sqrt{a} } }{ a^3 }$;
		\item $D= a^{\sqrt{2}}\left(\dfrac{1}{a}\right)^{\sqrt{2}-\sqrt{5}}$;
		\item $E= a^{-\sqrt{3}}: a^{\left(\sqrt{3}-1\right)^2} $.
	\end{enumEX}
	\loigiai{
		\begin{enumerate}[a)]
			\item $A =a^{\tfrac{1}{6}} \sqrt[3]{a}= a^{\tfrac{1}{6}}\cdot a^{\tfrac{1}{3}} = a^{\tfrac{1}{6}+\tfrac{1}{3}}= a^{\tfrac{1}{2}}\Rightarrow\alpha=\dfrac{1}{2}$.
			\item $B=a^{\tfrac{5}{3}}:\sqrt[3]{a}=a^{\tfrac{5}{3}}:a^{\tfrac{1}{3}}=a^{\tfrac{5}{3}-\tfrac{1}{3}}=a^{\tfrac{4}{3}}\Rightarrow\alpha=\dfrac{4}{3}$.
			\item $C= \dfrac{ \sqrt[3]{  a^2 \sqrt{a} } }{a^3} = \dfrac{ \left(a^2 \cdot a^{\tfrac{1}{2}}\right)^{\tfrac{1}{3}} }{a^3} = \dfrac{ \left( a^{\tfrac{5}{2}}\right)^{\tfrac{1}{3}} }{a^3}= \dfrac{a^{ \tfrac{5}{6} }}{a^3} = a^{ \tfrac{5}{6} -3} = a^{-\tfrac{13}{6} }\Rightarrow\alpha=-\dfrac{13}{6}$.
			\item $ D=a^{\sqrt{2}}\left(\dfrac{1}{a}\right)^{\sqrt{2}-\sqrt{5}}=a^{\sqrt{2}}\cdot a^{\sqrt{5}-\sqrt{2}} =a^{\sqrt{2}+\sqrt{5}-\sqrt{2}} =a^{\sqrt{5}} \Rightarrow\alpha=\sqrt{5}$.
			\item $E= a^{-\sqrt{3}}\colon a^{\left(\sqrt{3}-1\right)^2}=a^{-\sqrt{3}}\colon a^{4-2\sqrt{3}}=a^{-\sqrt{3}-4+2\sqrt{3}}=a^{-4+\sqrt{3}} \Rightarrow\alpha=-4+\sqrt{3}$.
		\end{enumerate}
	}
\end{vd}
%%Vidu6=========
\begin{vd}%[1D6V1-2]%[Dự án đề cương 3 Khối NH24-25-Đợt 2 - BCTuan]
	Cho $a>0$, $b>0$. Rút gọn các biểu thức sau
	\begin{enumEX}[a)]{3}
		\item $A=\left(a^{\tfrac{1}{2}}+b^{-\tfrac{1}{2}}\right)\left(a^{\tfrac{1}{2}}-b^{-\tfrac{1}{2}}\right)$;
		\item $B=\left(a+b^{\tfrac{2}{3}}\right)\left(a^2-a b^{\tfrac{2}{3}}+b^{\tfrac{4}{3}}\right)$;
		\item $C=\dfrac{\left(a^{\sqrt{2}}\right)^{\sqrt{2}}-b^0}{a+1}$.
	\end{enumEX}
	\loigiai{
	\begin{enumerate}[a)]
			\item $A=\left(a^{\tfrac{1}{2}}+b^{-\tfrac{1}{2}}\right)\left(a^{\tfrac{1}{2}}-b^{-\tfrac{1}{2}}\right)=\left( a^{\tfrac{1}{2}}\right)^2-\left( b^{-\tfrac{1}{2}}\right)^2=a-b^{-1}=a-\dfrac{1}{b}$.
			\item $B=\left(a+b^{\tfrac{2}{3}}\right)\left(a^2-a b^{\tfrac{2}{3}}+b^{\tfrac{4}{3}}\right)=a^3+\left( b^{\tfrac{2}{3}}\right)^3= a^3+ b^{\tfrac{2}{3}\cdot3}=a^3+b^2$.
			\item $C=\dfrac{\left(a^{\sqrt{2}}\right)^{\sqrt{2}}-b^0}{a+1}=\dfrac{a^{\sqrt{2}\cdot\sqrt{2}}-1}{a+1}=\dfrac{a^2-1}{a+1}=\dfrac{(a-1)(a+1)}{a+1}=a-1$.
	\end{enumerate}
	}
\end{vd} 
%%Dang3%%=========
\begin{dang}{So sánh hai lũy thừa}
	\begin{multicols}{3}
	\begin{itemize}
		\item Với mọi $\alpha \in \mathbb{R}$, $1^\alpha=1$ .\\
		\item Với $a>1$, $a^\alpha>1\Leftrightarrow \alpha>0$.
		\item Với $0<a<1$, $a^\alpha>1\Leftrightarrow \alpha<0$.
		\item Với $a>1$, $a^\alpha >a^\beta \Leftrightarrow \alpha>\beta$.
		\item Với $0<a<1$, $a^\alpha >a^\beta \Leftrightarrow \alpha<\beta$.
	\end{itemize}
	\end{multicols}
\end{dang}
%%Vidu7=========
\begin{vd}%[1D6H1-4]%[Dự án đề cương 3 Khối NH24-25-Đợt 2 - BCTuan]
	Không sử dụng máy tính cầm tay, hãy so sánh
	\begin{enumEX}[a)]{3}
		\item $ 2^{\tfrac{3}{4}}\  \text{và}\ 2^{\tfrac{5}{6}}$;
		\item $0{,}3^{\pi}\  \text{và}\ 0{,}3^{\sqrt{10}}$;
		\item $\sqrt[15]{0{,}1^{-7}}\  \text{và}\ \sqrt[5]{100}$.
	\end{enumEX}
	\loigiai{
		\begin{enumerate}
			\item Ta có $ \dfrac{3}{4}<\dfrac{5}{6}$. Vì $2>1$ nên $2^{\tfrac{3}{4}}<2^{\tfrac{5}{6}}$.
			\item Ta có $ \pi< \sqrt{10}$. Vì $ 0<0{,}3<1$ nên $0{,}3^{\pi}>0{,}3^{\sqrt{10}}$.
			\item Ta có $\sqrt[15]{0{,}1^{-7}}=\sqrt[15]{10^7}=10^{\tfrac{7}{15}}$; $\sqrt[5]{100}=\sqrt[5]{10^2}=10^{\tfrac{2}{5}}$.\\
			Lại có, $\dfrac{7}{15}>\dfrac{2}{5}$ và vì $10>1$ nên $10^{\tfrac{7}{15}}>10^{\tfrac{2}{5}}$ hay $\sqrt[15]{a^7}>\sqrt[5]{a^2}$.
		\end{enumerate}
	}
\end{vd}
%%Vidu8=========
\begin{vd}%[1D6H1-4]%[Dự án đề cương 3 Khối NH24-25-Đợt 2 - BCTuan]
	Xác định các giá trị dương của $a$ thoả mãn điều kiện
	\begin{enumEX}[a)]{3}
		\item $a^{\tfrac{1}{2}}>a^{\sqrt{3}}$;
		\item $a^{-\tfrac{3}{2}}<a^{\tfrac{2}{3}}$;
		\item $\left(\sqrt{2}\right)^{3a-1}>\left(\sqrt{3}\right)^{3a-1}$.
	\end{enumEX}
	\loigiai{
		\begin{enumerate}
			\item Ta có $\dfrac{1}{2}<\sqrt{3}$ nên từ $a^{\tfrac{1}{2}}>a^{\sqrt{3}}$ suy ra $0<a<1$.
			\item Ta có $-\dfrac{3}{2}<\dfrac{2}{3}$ nên từ $a^{-\tfrac{3}{2}}<a^{\tfrac{2}{3}}$ suy ra $a>1$.
			\item Ta có $\left(\sqrt{2}\right)^{3a-1}>\left(\sqrt{3}\right)^{3a-1}\Leftrightarrow \left(\dfrac{\sqrt{2}}{\sqrt{3}}\right)^{3a-1}>1\Leftrightarrow \left(\sqrt{\dfrac{2}{3}}\right)^{3a-1}>\left(\sqrt{\dfrac{2}{3}}\right)^0$\quad ($1$).\\
			Ta có $0<\sqrt{\dfrac{2}{3}}<1$ nên từ ($1$) suy ra $3a-1<0$ hay $a<\dfrac{1}{3}$.\\
			Kết hợp điều kiện $a>0$ ta được giá trị của $a$ là $0<a<\dfrac{1}{3}$.
		\end{enumerate}
	}
\end{vd}
%%Dang4%%=========
\begin{dang}{Ứng dụng}
{\bf Bài toán lãi kép:}
	\begin{itemize}
	\item Lãi kép là hình thức tính lãi trong đó tiền lãi được cộng vào vốn gốc sau mỗi kỳ hạn, để kỳ sau tính lãi trên tổng số đó (gồm cả gốc lẫn lãi kỳ trước).
	\item Công thức $\boxed{T_n =T_0(1+r)^n}$, trong đó 
		$\begin{array}{ll}
	&T_n \text{ là số tiền cả vốn lẫn lãi sau}\ n \text{ kỳ hạn;}\\
	&T_0 \text{ là số tiền gửi ban đầu;}\\
	&n \text{ là số kỳ hạn tính lãi và } r \text{ là lãi suất một kỳ, tính theo }\%.
		\end{array}$
	\end{itemize}
{\bf Bài toán chu kỳ bán rã:}
		\begin{itemize}
		\item Chu kỳ bán rã là khoảng thời gian cần thiết để một nửa lượng chất phóng xạ ban đầu phân rã thành chất khác (không còn phóng xạ).
		\item Công thức $\boxed{m=m_0\cdot 2^{-\tfrac{t}{T}}}$, trong đó 
		$\begin{array}{ll}
			&m \text{ là khối lượng của chất phóng xạ còn lại sau}\ t \text{ năm;}\\
			&m_0 \text{ là khối lượng chất phóng xạ ban đầu;}\\
			&t \text{ là năm và } T \text{ là chu kì bán rã}.
		\end{array}$
	\end{itemize}
\end{dang}
%%Vidu9=========
\begin{vd}%[1D6V1-1]%[Dự án đề cương 3 Khối NH24-25-Đợt 2 - BCTuan]
	Một người gửi $100$ triệu đồng vào ngân hàng với lãi suất $0{,}4\% /$tháng. Biết rằng nếu không rút tiền ra khỏi ngân hàng thì cứ sau mỗi tháng, số tiền lãi sẽ được nhập vào vốn ban đầu để tính lãi cho tháng tiếp theo. Hỏi sau $6$ tháng, người đó được lĩnh số tiền bao nhiêu, nếu trong khoảng thời gian này người đó không rút tiền ra và lãi xuất không thay đổi?
	\loigiai{
		Áp dụng công thức lãi kép với $n=6$, $r=0{,}4\%$, $T_0=100$ (triệu), ta có 
		$$T_6=100\cdot \left(1+0{,}4\%\right)^6=100\cdot \left(1+\dfrac{0{,}4}{100} \right)^6=102{,}424128\text{ (triệu đồng)}.$$
		Vậy sau $6$ tháng, người đó lĩnh được số tiền là $102{,}424128$ triệu đồng.
	}
\end{vd}
\vspace{0.5em}
%%Vidu10=========
\begin{vd}%[1D6V1-1]%[Dự án đề cương 3 Khối NH24-25-Đợt 2 - BCTuan]
	Một chất phóng xạ có chu kì bán rã là $25$ năm, tức là cứ sau $25$ năm, khối lượng của chất phóng xạ đó giảm đi một nửa. Giả sử lúc đầu có $10 \mathrm{~g}$ chất phóng xạ đó. Viết công thức tính khối lượng của chất đó còn lại sau $t$ năm và tính khối lượng của chất đó còn lại sau $120$ năm (làm tròn kết quả đến hàng phần nghìn theo đơn vị gam).
	\loigiai{
		Áp dụng công thức chu kỳ bán rã với $t=120$, $T=25$, $m_0=10$ (gam), ta có 
$$m=10\cdot 2^{-\tfrac{120}{25}}\approx0{,}359\ \text{gam}.$$
%		Cứ $25$ năm, khối lượng chất phóng xạ đó giảm đi một nửa nên sau mỗi năm, khối lượng chất phóng xạ đó giảm đi $\left(\dfrac{1}{2}\right)^{\tfrac{1}{25}}$. \\
%		Sau $t$ năm, khối lượng của chất phóng xạ đó giảm đi $\left[\left(\dfrac{1}{2}\right)^{\tfrac{1}{25}}\right]^t=\left(\dfrac{1}{2}\right)^{\tfrac{t}{25}}$.\\
%		Vậy $10$\,gam chất phóng xã còn lại sau $120$ năm là $$m=10\cdot \left(\dfrac{1}{2}\right)^{\tfrac{120}{25}}\approx0{,}359\ \text{gam}.$$
	}
\end{vd}
%%Vidu11=========
\begin{vd}%[1D6V1-1]%[Dự án đề cương 3 Khối NH24-25-Đợt 2 - BCTuan]
	Cường độ ánh sáng tại độ sâu $h$ (m) dưới một mặt hồ được tính bằng công thức $I_h=I_0\left(\dfrac{1}{2}\right)^{\tfrac{h}{4}}$, trong đó $I_0$ là cường độ ánh sáng tại mặt hồ đó.
	\begin{enumEX}[a)]{1}
		\item Cường độ ánh sáng tại độ sâu $1$ m bằng bao nhiêu phần trăm so với cường độ ánh sáng tại mặt hồ (kết quả làm tròn đến hàng đơn vị phần trăm)?
		\item Cường độ ánh sáng tại độ sâu $3$ m gấp bao nhiêu lần cường độ ánh sáng tại độ sâu $6$ m (kết quả làm tròn đến hàng phần trăm)?
	\end{enumEX}
	\loigiai{
		\begin{enumerate}
			\item Ta có cường độ ánh sáng tại độ sâu $1$ (m) dưới một mặt hồ bằng $I_1=I_0\left(\dfrac{1}{2}\right)^{\tfrac{1}{4}}$.\\
			Suy ra, $\dfrac{I_1}{I_0}=\left(\dfrac{1}{2}\right)^{\tfrac{1}{4}} \approx 0{,}84=84 \%$.
			\item Ta có $I_3=I_0\left(\dfrac{1}{2}\right)^{\tfrac{3}{4}}$ và $I_6=I_0\left(\dfrac{1}{2}\right)^{\tfrac{6}{4}}$ nên					
			$\dfrac{I_3}{I_6}=\left(\dfrac{1}{2}\right)^{\tfrac{3}{4}-\tfrac{6}{4}}=\left(\dfrac{1}{2}\right)^{-\tfrac{3}{4}} \approx 1{,}68$ (lần).
		\end{enumerate}	
	}
\end{vd}

%-----------------------------------------------------------------------------
\subsection{Bài tập rèn luyện}
\ind{PHẦN I.} \inden{Câu trắc nghiệm nhiều phương án lựa chọn. Mỗi câu hỏi học sinh chỉ chọn một phương án.}\\
\setcounter{ex}{0}
\Opensolutionfile{ans}[ans/1D6-Bai1-TN]
\begin{ex}[Đề thi học kỳ 2 năm học 2024-2025, THPT Xuyên Mộc]%[1D6N1-2]%[Dự án D đợt 2, BCTuan]
	Cho $a$ là số thực dương. Rút gọn biểu thức $P=a^{\frac{2}{5}} \cdot \sqrt[3]{a}$ được kết quả là
	\choice
	{$a^{\frac{2}{15}}$}
	{$a^{\frac{17}{5}}$}
	{\True $a^{\frac{11}{15}}$}
	{$a^{\frac{6}{5}}$}
	\loigiai{
		Với $a$ là số thực dương, ta có $P=a^{\frac{2}{5}} \cdot \sqrt[3]{a}=a^{\frac{2}{5}} \cdot a^{\frac{1}{3}}=a^{\frac{11}{15}}$.
	}
\end{ex}
\begin{ex}[ĐỀ THI HK2 TRƯỜNG PHAN ĐĂNG LƯU - NĂM HỌC 2024-2025]%[1D6N1-2]%[Dự án D đợt 2, BCTuan]
	Với $a$ là số thực dương tùy ý, $\sqrt[3]{a^2}$ bằng
	\choice
	{$a^6$}
	{\True $a^{\tfrac{2}{3}}$}
	{$a^{\tfrac{3}{2}}$}
	{$a^{\tfrac{1}{6}}$}
	\loigiai{
		Ta có $\sqrt[3]{a^2}=a^{\tfrac{2}{3}}$.
	}
\end{ex}
\begin{ex}[Đề thi HK2 THPT Chuyên Trần Đại Nghĩa năm học 2024-2025]%[1D6N1-2]%[Dự án D đợt 2, BCTuan]
	Với các số thực $a$, $b$ bất kỳ, mệnh đề nào dưới đây đúng?
	\choice
	{$2^a\cdot 2^b=2^{a-b}$}
	{$2^a\cdot 2^b=2^{ab}$}
	{\True $2^a\cdot 2^b=2^{a+b}$}
	{$2^a\cdot 2^b=4^{ab}$}
	\loigiai{
		Với các số thực $a$, $b$ bất kỳ ta có $2^a\cdot 2^b=2^{a+b}$.
	}
\end{ex}
\begin{ex}[Đề thi HK2 THPT Phạm Văn Đồng - Khánh Hoà- Năm học 2023-2024]%[1D6H1-2]%[Dự án D đợt 2, BCTuan]
	Cho biểu thức $P=\sqrt[4]{x\cdot \sqrt[3]{x^4}}$, với $x > 0$. Mệnh đề nào dưới đây đúng?
	\choice
	{$P=x^{\tfrac{13}{24}}$}
	{$P=x^{\tfrac{7}{2}}$}
	{$P=x^{\tfrac{1}{4}}$}
	{\True $P=x^{\tfrac{7}{12}}$}
	\loigiai{
		Ta có 
		\allowdisplaybreaks
		\begin{align*}
			P=\sqrt[4]{x\cdot \sqrt[3]{x^4}} = \sqrt[4]{x\cdot x^{\tfrac{4}{3}}}=\sqrt[4]{x^{1+\tfrac{4}{3}}} = \sqrt[4]{x^{\tfrac{7}{3}}} = x^{\tfrac{7}{12}}.
		\end{align*}
	}
\end{ex}
\begin{ex}[Đề thi HK2 năm 2023-2024, Sở Giáo Dục và Đào tạo Nam Định]%[1D6H1-2]%[Dự án D đợt 2, BCTuan]
	Cho biểu thức $P=\dfrac{a^{\sqrt{3}+1} \cdot a^{2-\sqrt{3}}}{\left(a^{\sqrt{3}-2} \right)^{\sqrt{3}+2}}$, với $a > 0$. Phát biểu nào sau đây đúng?
	\choice
	{$P=a^3$}
	{$P=a^{\tfrac{1}{2}}$}
	{$P=a^{-1}$}
	{\True $P=a^4$}
	\loigiai{Ta có 
			$$P=\dfrac{a^{\sqrt{3}+1} \cdot a^{2-\sqrt{3}}}{\left(a^{\sqrt{3}-2} \right)^{\sqrt{3}+2}}
			=\dfrac{a^3}{a^{-1}}
			=a^4.$$
	}
\end{ex}
\begin{ex}[ĐỀ THI HK2-THPT HƯỚNG HÓA - QUẢNG TRỊ NĂM HỌC 2024-2025]%[1D6N1-1]%[Dự án D đợt 2, BCTuan]
	Với $a>0$, $b>0$, $\alpha$, $\beta$ là các số thực bất kì, đẳng thức nào sau đây \textbf{sai}?
	\choice
	{$a^\alpha \cdot a^\beta=a^{\alpha+\beta}$}
	{\True $\dfrac{a^\alpha}{b^\beta}=\left(\dfrac{a}{b}\right)^{\alpha-\beta}$}
	{$a^\alpha \cdot b^\alpha=(a b)^\alpha$}
	{$\dfrac{a^\alpha}{a^\beta}=a^{\alpha-\beta}$}
	\loigiai{
		Đẳng thức sai là $\dfrac{a^\alpha}{b^\beta}=\left(\dfrac{a}{b}\right)^{\alpha-\beta}$.
	}
\end{ex}

\begin{ex}[Đề thi GHK2, THPT Nguyễn Quốc Trinh - Hà Nội, năm học 2024-2025]%[1D6H1-4]%[Dự án D đợt 2, BCTuan]
	Cho số dương $a$ thỏa mãn $a^4<a^3$, mệnh đề nào sau đây là đúng?
	\choice
	{$a<3$}
	{$a>1$}
	{$3<a<4$}
	{\True $0<a<1$}
	\loigiai
	{Vì $4>3$ mà $a^4<a^3$ nên $0<a<1$.}
\end{ex}

\begin{ex}[Đề thi GHK2- THPT Trần Phú năm học 2024-2025]%[1D6H1-2]%[Dự án D đợt 2, BCTuan]
	Cho $K=\dfrac{x^4\cdot \sqrt{x}\cdot (x^3)^2}{\sqrt[3]{x^2}}$ với $x>0$. Rút gọn $K$ ta được
	\choice
	{$K=x^{\tfrac{41}{6}}$}
	{$K=x^{\tfrac{47}{3}}$}
	{\True $K=x^{\tfrac{59}{6}}$}
	{$K=x^{\tfrac{55}{4}}$}
	\loigiai{
		Ta có $K = \dfrac{x^4 \cdot \sqrt{x} \cdot (x^3)^2}{\sqrt[3]{x^2}} 
		= \dfrac{x^4 \cdot x^{\tfrac{1}{2}} \cdot x^6}{x^{\tfrac{2}{3}}} 
		= \dfrac{x^{4 + \tfrac{1}{2} + 6}}{x^{\tfrac{2}{3}}} 
		= \dfrac{x^{\tfrac{21}{2}}}{x^{\tfrac{2}{3}}} 
		= x^{\tfrac{21}{2} - \tfrac{2}{3}} 
		= x^{\tfrac{59}{6}}$.
	}
\end{ex}
\begin{ex}%[1D6H1-1]%[Dự án D đợt 2, BCTuan]
	Biết rằng  $10^\alpha=2$; $10^\beta = 5$. Tính $10^{\alpha+\beta}$+ $10^{\alpha-\beta}$.
	\choice
	{10}
	{\True $\dfrac{52}{5}$}
	{$\dfrac{2}{5}$}
	{$\dfrac{5}{52}$}
	\loigiai{
		Ta có  $10^{\alpha+\beta}=10^\alpha \cdot 10^\beta=2\cdot5=10$. \\
		Mà $10^{\alpha-\beta}=\dfrac{10^\alpha}{ 10^\beta}=\dfrac{2}{5}$.\\
		Vậy $10^{\alpha+\beta}$+ $10^{\alpha-\beta}$ = $\dfrac{52}{5}$. 
	}
\end{ex}	
\begin{ex}%[1D6V1-1]%[Dự án D đợt 2, BCTuan]
	Cho $a$, $b$ là các số thực thỏa mãn  $3\cdot2^a + 2^b=7\sqrt{2}$ và $5\cdot2^a - 2^b=9\sqrt{2}$. Tính $S=2^{a+b}+2^{a-b}$.
	\choice
	{$6\sqrt{2}$}
	{$2$}
	{$4$}
	{\True $6$}
	\loigiai{
		Ta có  $\left\{
		\begin{aligned}
			3\cdot2^a + 2^b=7\sqrt{2}\\
			5\cdot2^a - 2^b=9\sqrt{2}
		\end{aligned}
		\right.$
		$\Rightarrow 8\cdot2^a=16\sqrt{2}\Rightarrow 2^a = 2\sqrt{2}$.\\
		Mà $2^b=7\sqrt{2}-3\cdot 2^a\Rightarrow 2^b = 7\sqrt{2}-3\cdot 2\sqrt{2}\Rightarrow 2^b = \sqrt{2}$.\\
		Xét $2^a \cdot 2^b=2^{a+b}\Rightarrow 2^{a+b}=2\sqrt{2}\cdot \sqrt{2}\Rightarrow 2^{a+b}=4$. \\
		Mặt khác $\dfrac{2^a}{2^b}=2^{a-b}\Rightarrow 2^{a-b}=\dfrac{2\sqrt{2}}{\sqrt{2}}=2$. \\
		Vậy $S=2^{a+b}+2^{a-b}=4+2=6$.
	}
\end{ex}	

\begin{ex}%[1D6V3-5]%[Dự án D đợt 2, BCTuan]
	Anh An gửi số tiền 58 triệu đồng vào một ngân hàng theo hình thức lãi kép và ổn định trong $9$ tháng thì lĩnh về được $61\,758\,000$ đồng. Hỏi lãi suất ngân hàng hàng tháng là bao nhiêu? Biết rằng lãi suất không thay đổi trong thời gian gửi (làm tròn kết quả đến hàng phần chục).
	\choice
	{$0{,}8 \%$}
	{ $0{,}6 \%$}
	{\True $0{,}7 \%$}
	{$0{,}5\%$}
	\loigiai{
		Áp dụng công thức $A_n=A_0\left(1+r\right)^n$ với $n$ là số kỳ hạn, $A_0$ là số tiền ban đầu, $A_n$ là số tiền sau $n$ kỳ hạn, $r$ là lãi suất.\\
		Suy ra $A_9=A_0\left(1+r\right)^9\Rightarrow r=\sqrt[9]{\dfrac{A_9}{A_0}}-1\approx 0{,}7\%$.
	}
\end{ex}	
\begin{ex}%[1D6V3-5]%[Dự án D đợt 2, BCTuan]
	Ông An gửi tiết kiệm $50$  triệu đồng vào ngân hàng với kỳ hạn $3$  tháng, lãi suất $8{,}4\%$  một năm theo hình thức lãi kép. Ông gửi được đúng  $3$ kỳ hạn thì ngân hàng thay đổi lãi suất, ông gửi tiếp $12$  tháng nữa với kỳ hạn như cũ và lãi suất trong thời gian này là $12\%$  một năm thì ông rút tiền về. Số tiền ông An nhận được cả gốc lẫn lãi là
	\choice
	{$62\,255\,910$ đồng}
	{\True $59\,895\,767$ đồng}
	{$59\,993\,756$ đồng}
	{$63\,545\,193$ đồng}
	\loigiai{
		\begin{itemize}
			\item Đợt I, ông An gửi số tiền $P_0 = 50$  triệu, lãi suất  $8{,}4\%$/năm tức là  $2{,}1\%$ mỗi kỳ hạn. \\
			Số tiền cả gốc và lãi ông thu được sau  $3$ kỳ hạn là $P_3=50\,000\,000 \cdot (1{,}021)^3$.
			\item Đợt II, do ông không rút ra nên số tiền  $P_3$ được xem là số tiền gửi ban đầu của đợt II, lãi suất đợt II là $3\%$  mỗi kỳ hạn. Ông gửi tiếp $12$  tháng bằng $4$ kỳ hạn nên số tiền thu được cuối cùng là
			\[P=P_3 \cdot (1{,}03)^4=50\,000\,000 \cdot (1{,}021)^3\cdot(1{,}03)^4\approx 59\,895\,767 \ \text{(đồng)}.\]
		\end{itemize}
	}
\end{ex}	
\begin{ex}[Đề thi GHK2 THPT An Lương Đông - Huế năm học 2024-2025]%[1D6H1-2]%[Dự án D đợt 2, BCTuan]
	Giá trị của $K=\left(\dfrac{1}{81}\right)^{-0{,}75}+\left(\dfrac{1}{27}\right)^{-\tfrac{4}{3}}$ bằng
	\choice
	{$K=18$}
	{\True $K=108$}
	{$K=180$}
	{$K=54$}
	\loigiai{
		Ta có
		\begin{eqnarray*}
			K&=&\left(\dfrac{1}{81}\right)^{-0{,}75}+\left(\dfrac{1}{27}\right)^{-\tfrac{4}{3}}\\
			&=&81^{0{,}75}+27^{\frac{4}{3}}\\
			&=&3^{4\cdot 0{,}75}+3^{3\cdot \tfrac{4}{3}}\\
			&=&3^3+3^{4}\\
			&=&27+81\\
			&=&=108.
		\end{eqnarray*}
		Vậy $K=108$.
	}
\end{ex}
\begin{ex}[Đề thi GHK2 năm học 2024-2025, THPT Phan Bội Châu - Tỉnh Bình Thuận]%[1D6H1-1]%[Dự án D đợt 2, BCTuan]
	Nếu $2^\alpha = 9$ thì $\left(\dfrac{1}{16}\right)^{\tfrac{\alpha}{8}}$ bằng
	\choice
	{$\dfrac{1}{\sqrt{3}}$}
	{$\dfrac{1}{9}$}
	{\True $\dfrac{1}{3}$}
	{$3$}
	\loigiai{
		Ta có $\left(\dfrac{1}{16}\right)^{\tfrac{\alpha}{8}} = \left( \dfrac{1}{2^4} \right)^{\tfrac{\alpha}{8}} = \left(2^{-4}\right)^{\tfrac{\alpha}{8}} = 2^{-4 \cdot \frac{n}{8}} = 2^{-\tfrac{\alpha}{2}} = (2^\alpha)^{-\tfrac{1}{2}}$.\\
		Thay $2^\alpha = 9$ vào biểu thức: $(9)^{-\tfrac{1}{2}} = \dfrac{1}{9^{\tfrac{1}{2}}} = \dfrac{1}{\sqrt{9}} = \dfrac{1}{3}$.
	}  
\end{ex}

%Cau7%[1D6-Bai1-Dang2-TN]
	\begin{ex}%[1D6N1-2]%[Dự án đề cương 3 Khối NH24-25-Đợt 2- Bùi Lương Phúc]
	(\textit{\footnotesize Trích đề thi GKII - THPT Chuyên Vị Thanh - Hậu Giang - Năm học 2024-2025})\\	
	Cho $x, y > 0$ và $\alpha, \beta \in \mathbb{R}$. Đẳng thức nào dưới đây là \textbf{sai}?	
	\choice
	{\True $x^{\alpha} + y^{\alpha} = (x + y)^\alpha$}
	{$(xy)^\alpha = x^{\alpha} \cdot y^{\alpha}$}
	{$\left(x^{\alpha}\right)^\beta = x^{\alpha\beta}$}
	{$x^{\alpha} \cdot x^{\beta} = x^{\alpha + \beta}$}
	
	\loigiai{
		Nói chung, $x^{\alpha} + y^{\alpha} \neq (x + y)^\alpha$ (với $x, y > 0$ và $\alpha, \beta \in \mathbb{R}$).\\
		Vì vậy mệnh đề $x^{\alpha} + y^{\alpha} = (x + y)^\alpha$ là sai \quad ($1$).
		Thật vậy, lấy $x = y = 1$, $\alpha = 2$ thì 
		\[
		x^{\alpha} + y^{\alpha} = 1^2 + 1^2 = 2\text{ và } (x + y)^\alpha = (1 + 1)^2 = 4
		\Rightarrow \text{ hai vế của (1) không bằng nhau.}\]
	}
\end{ex}


%Cau12%[1D6-Bai1-Dang1-TN]
\begin{ex}%[1D6H1-1]%[Dự án đề cương 3 Khối NH24-25-Đợt 2 - Bùi Lương Phúc]
	Giá trị của biểu thức $P=\dfrac{6^{3+\sqrt{5}}}{2^{2+\sqrt{5}}\cdot 3^{1+\sqrt{5}}}$ là
	\choice
	{$\dfrac{1}{\sqrt{5}}$}
	{$36$}
	{$12$}
	{\True $18$}
	\loigiai
	{Ta có $\dfrac{6^{3+\sqrt{5}}}{2^{2+\sqrt{5}}\cdot 3^{1+\sqrt{5}}}=\dfrac{6^{3+\sqrt{5}}}{2^{2+\sqrt{5}}\cdot 3^{1+\sqrt{5}}}=\dfrac{2^{3+\sqrt{5}}\cdot 3^{3+\sqrt{5}}}{2^{2+\sqrt{5}}\cdot 3^{1+\sqrt{5}}}=2\cdot 3^2 =18$.
	}
\end{ex}


%Cau15%[1D6-Bai1-Dang2-TN]
\begin{ex}%[1D6H1-2]%[Dự án đề cương 3 Khối NH24-25-Đợt 2 - Bùi Lương Phúc]
(\textit{\footnotesize Trích đề thi GKII - THPT Chuyên Nguyễn Đình Chiểu - Đồng Tháp - Năm học 2024-2025})\\
	Rút gọn biểu thức $P = x^{\tfrac{1}{6}} \cdot \sqrt[3]{x}$ với $x>0$.
\choice
	{$P = x^{\tfrac{1}{8}}$}
	{\True $P = \sqrt{x}$}
	{$P = x^{\tfrac{2}{9}}$}
	{$P = x^2$}
\loigiai{	
Ta có
$$ P = x^{\tfrac{1}{6}} \cdot \sqrt[3]{x} = x^{\tfrac{1}{6}} \cdot x^{\tfrac{1}{3}} = x^{\tfrac{1}{6}+\tfrac{1}{3}} = x^{\tfrac{1}{2}} = \sqrt{x}.$$
}
\end{ex}	

\begin{ex}%[1D6V3-5]%[Dự án D đợt 2, BCTuan]
	Một học sinh $A$  khi $15$  tuổi được hưởng tài sản thừa kế $200\,000\,000$   VNĐ. Số tiền này được bảo quản trong ngân hàng $B$  với kì hạn thanh toán  $1$ năm và học sinh $A$  chỉ nhận được số tiền này khi  $18$ tuổi. Biết rằng khi $18$  tuổi, số tiền mà học sinh $A$  được nhận sẽ là $231\,525\,000$  VNĐ. Vậy lãi suất kì hạn một năm của ngân hàng  $B$ là bao nhiêu? (làm tròn kết quả đến hàng đơn vị)
	\choice
	{$8 \%$/năm}
	{$7\%$/năm}
	{$6 \%$/năm}
	{\True $5\%$/năm}
	\loigiai{
		Số tiền nhận được của gốc và lãi là
		\begin{eqnarray*}
			200\,000\,000(1+r)^3&=&231\,525\,000\\
			(1+r)^3&=&1{,}157625\\
			1+r&=&\sqrt[3]{1{,}157625}\\
			r&=&\sqrt[3]{1{,}157625}-1\approx 0{,}05=5\%.
		\end{eqnarray*} 
	}
\end{ex}	

\begin{ex}%[1D6V3-5]%[Dự án D đợt 2, BCTuan]
	Ông An gửi vào ngân hàng $60$ triệu đồng theo hình thức lãi kép. Lãi suất ngân hàng là  $8\%$/năm. Sau $5$  năm ông An tiếp tục gửi thêm  $60$ triệu đồng nữa. Hỏi sau  $10$ năm kể từ lần gửi đầu tiên ông An đến rút toàn bộ tiền gốc và tiền lãi được là bao nhiêu triệu đồng?
	\choice
	{$231{,}815$}
	{ $197{,}201$}
	{\True $217{,}695$}
	{$190{,}271$}
	\loigiai{
		Số tiền ông An nhận được sau $5$ năm đầu là $60(1+8\%)^5=88{,}160$ (triệu đồng). \\
		Số tiền ông An nhận được sau $10$ năm là
		\[(88{,}16+60)(1+8\%)^5\approx 217{,}695 \ \text{(triệu đồng)}.\]
	}
\end{ex}	
\begin{ex}%[1D6V3-5]%[Dự án D đợt 2, BCTuan]
	Anh Nam gửi $100$ triệu đồng vào ngân hàng theo thể thức lãi kép kì hạn là một quý với lãi suất $3\%$/quý. Sau đúng $6$ tháng anh Nam gửi thêm $100$ triệu đồng với kì hạn và lãi suất như trước đó. Hỏi sau $1$ năm số tiền anh Nam nhận được là bao nhiêu?
	\choice
	{\True $218{,}64$ triệu đồng}
	{ $208{,}25$ triệu đồng}
	{$210{,}45$ triệu đồng}
	{$209{,}25$ triệu đồng}
	\loigiai{
		\begin{itemize}
			\item Số tiền anh Nam nhận được sau 6 tháng là
			\[T_1=100(1+3\%)^2=106{,}09 \ \text{triệu đồng}.\]
			\item Số tiền anh Nam nhận được sau một năm là
			\[T_2=(106{,}09+100)(1+3\%)^2\approx 218{,}64 \  \text{triệu đồng}.\]
		\end{itemize}
	}
\end{ex}	
\Closesolutionfile{ans}

\ind{PHẦN II.} \inden{Câu trắc nghiệm đúng sai. Trong mỗi ý a), b), c), d) ở mỗi câu, học sinh chọn đúng hoặc sai.}\\
\setcounter{ex}{0}
\Opensolutionfile{ans}[ans/1D6-Bai1-DS]
%%Cau1
\begin{ex}%[1D6H1-1]%[Dự án đề cương 3 Khối NH24-25-Đợt 2 - BCTuan]
	Cho hai biểu thức $A=9^{\tfrac{2}{5}}\cdot 27^{\tfrac{2}{5}}$ và $B=144^{\tfrac{3}{4}}:9^{\tfrac{3}{4}}$.
	\choiceTF
	{\True $A=(9\cdot27)^{\tfrac{2}{5}}$}
	{Nếu $A=3^m$ thì $m=3$}
	{Nếu $B=4^k$ thì $k=3$}
	{\True Thực hiện phép tính $A-B$ ta được kết quả là một số tự nhiên}
	\loigiai{
		\begin{itemchoice}
			\itemch {\bf Đúng}. Vì $9^{\tfrac{2}{5}}\cdot 27^{\tfrac{2}{5}}=(9\cdot27)^{\tfrac{2}{5}}$.
			\itemch {\bf Sai}. Vì 
			$A=9^{\tfrac{2}{5}}\cdot 27^{\tfrac{2}{5}}=(9\cdot27)^{\tfrac{2}{5}}=\left(3^2\cdot3^3\right)^{\tfrac{2}{5}}=(3^5)^{\tfrac{2}{5}}=3^2$. Suy ra $m=2$.
			\itemch {\bf Sai}. Vì  $B=144^{\tfrac{3}{4}}:9^{\tfrac{3}{4}}=\left(\dfrac{144}{9}\right)^{\tfrac{3}{4}}=16^{\tfrac{3}{4}}=\left(4^2\right)^{\tfrac{3}{4}}=4^{\tfrac{3}{2}}$. Suy ra $k=\dfrac{3}{2}$.
			\itemch {\bf Đúng}. Vì $A=9^{\tfrac{2}{5}}\cdot 27^{\tfrac{2}{5}}=(9\cdot27)^{\tfrac{2}{5}}=\left(3^2\cdot3^3\right)^{\tfrac{2}{5}}=(3^5)^{\tfrac{2}{5}}=3^2=9$.\\
			$B=144^{\tfrac{3}{4}}:9^{\tfrac{3}{4}}=\left(\dfrac{144}{9}\right)^{\tfrac{3}{4}}=16^{\tfrac{3}{4}}=\left(2^4\right)^{\tfrac{3}{4}}=2^3=8$.\\
			Khi đó $A-B=1$.
		\end{itemchoice}
	}
\end{ex}
%%Cau2
\begin{ex}%[1D6H1-2]%[Dự án đề cương 3 Khối NH24-25-Đợt 2 - BCTuan]
	Cho số thực dương $a$ với $a\neq1$ và hai số thực $x$, $y$ tùy ý. 
	\choiceTF
	{$a^x\cdot a^y=a^{xy}$}
	{\True $a^0=1$ và $\sqrt[5]{a}=a^{\tfrac{1}{5}}$}
	{$a^{x-y}>1\Leftrightarrow x>y$}
	{\True Nếu $a^{2x}+a^{-2x}=23$ thì $a^x+a^{-x}=5$}
	\loigiai{
		\begin{itemchoice}
			\itemch {\bf Sai}.\\
			Lý thuyết $a^x\cdot a^y=a^{x+y}$.
			\itemch {\bf Đúng}.\\
			Ta có $a^0=1$ và $\sqrt[5]{a}=a^{\tfrac{1}{5}}$.
			\itemch {\bf Sai}.\\
			Ta có $a^{x-y}>1\Leftrightarrow a^{x-y}>a^0$. Khi đó có hai trường hợp xảy ra\\
			$\bullet$ Nếu $a>1$, suy ra $x-y>0$ tức là $x>y$.\\
			$\bullet$ Nếu $1>a>0$, suy ra $x-y<0$ tức là $x<y$.
			\itemch {\bf Đúng}.\\
			Nhận xét$\colon a^x+a^{-x}>0$, khi đó
			\begin{eqnarray*}
				&&a^{2x}+a^{-2x}=23\\
				&\Leftrightarrow&\left(a^x\right)^2+\left(a^{-x}\right)^2+2\cdot a^x\cdot a^{-x} =23+2\cdot a^x\cdot a^{-x}\\
				&\Leftrightarrow&\left(a^x+a^{-x}\right)^2 =23+2\cdot a^{x-x}\\
				&\Leftrightarrow&\left(a^x+a^{-x}\right)^2 =23+2\cdot a^0\\
				&\Leftrightarrow&\left(a^x+a^{-x}\right)^2 -2\cdot 1=23\\
				&\Leftrightarrow&\left(a^x+a^{-x}\right)^2 =25\\
				&\Leftrightarrow&a^x+a^{-x} =5.
			\end{eqnarray*}
		\end{itemchoice}
	}
\end{ex}
%%Cau3
\begin{ex}%[1D6H1-3]%[Dự án đề cương 3 Khối NH24-25-Đợt 2 - Bùi Lương Phúc]
	Cho các biểu thức $A=\sqrt[4]{x-1}$ và $B=(x-1)^{\sqrt{2}}$.
	\choiceTF
	{Điều kiện xác định của biểu thức $A$ là $x>1$}
	{\True Điều kiện xác định của biểu thức $B$ là $x>1$}
	{Thực hiện phép tính đối với biểu thức $A\cdot B$ ta được kết quả là $B=(x-1)^{\tfrac{\sqrt{2}}{4}}$}
	{Nếu $A>B$ thì $x>2$}
	\loigiai
	{
		\begin{itemchoice}
			\itemch \textbf{Sai}.\\
			Điều kiện xác định của biểu thức $A$ là $x\ge 1$.
			\itemch \textbf{Đúng}.\\
			Điều kiện xác định của biểu thức $B$ là $x>1$.
			\itemch \textbf{Sai}.\\
			$A\cdot B$ có nghĩa khi và chỉ khi  $x>1$. Ta có 
			\begin{align*}
			A\cdot B
			&=\sqrt[4]{x-1}\cdot(x-1)^{\sqrt{2}}\\
			&=(x-1)^{\tfrac{1}{4}}\cdot(x-1)^{\sqrt{2}}\\
			&=(x-1)^{\tfrac{1}{4}+\sqrt{2}}.			
			\end{align*}
			\itemch \textbf{Sai}.\\
			$A$ và $B$ đồng thời có nghĩa khi và chỉ khi  $x>1$. Ta có 
			\begin{align*}
				A> B
				&\Leftrightarrow \sqrt[4]{x-1}>(x-1)^{\sqrt{2}}\\
				&\Leftrightarrow (x-1)^{\tfrac{1}{4}}>(x-1)^{\sqrt{2}}\\
				&\Leftrightarrow x-1<1\ \text{do } \dfrac{1}{4}<\sqrt{2}\\
				&\Leftrightarrow x<2.		
			\end{align*}
			Kết hợp điều kiện ban đầu ta được $1<x<2$. 		
		\end{itemchoice}
	}
\end{ex}
%%Cau4
\begin{ex}%[1D6V1-1]%[Dự án đề cương 3 Khối NH24-25-Đợt 2 - BCTuan]
	Anh Nam gửi $100$ triệu đồng vào ngân hàng theo thể thức lãi kép kì hạn là một quý với lãi suất $3\%$/quý. Sau đúng $6$ tháng anh Nam gửi thêm $100$ triệu đồng với kì hạn và lãi suất như trước đó. 
	\choiceTF
	{\True Sau $3$ tháng đầu tiên, số tiền lãi anh Nam nhẩm tính được là $3$ triệu đồng}
	{\True Giả sử sau $6$ tháng, anh Nam không gửi thêm $100$ triệu đồng như đã nêu ở trên. Lúc đó số tiền anh Nam nhận được (cả gốc và lãi) là $106{,}09$ triệu đồng}
	{Sau $1$ năm số tiền anh Nam nhận được là $209{,}25$ triệu đồng}
	{Sau $2$ năm số tiền anh Nam nhận được là $247$ triệu đồng (làm tròn đến hàng đơn vị của triệu đồng)}
	\loigiai{
		\begin{itemchoice}
			\itemch \textbf{Đúng}.\\
			Sau $3$ tháng đầu tiên, tức là $1$ quý, số tiền lãi anh Nam nhẩm tính được là $3\%\cdot100=3$ triệu đồng
			\itemch \textbf{Đúng}.\\
			Số tiền anh Nam nhận được sau $6$ tháng đầu là
			\[T_1=100(1+3\%)^2=106{,}09 \ \text{triệu đồng}.\]
			\itemch \textbf{Sai}.\\
			Sau $1$ năm, tức là $2$ quý sau đó (trừ đi $2$ quý đầu), số tiền anh Nam nhận được là
			\[T_2=(106{,}09+100)(1+3\%)^2\approx 218{,}64\  \text{triệu đồng}.\]
			\itemch \textbf{Sai}.\\
			Sau $2$ năm, tức là $6$ quý sau đó, số tiền anh Nam nhận được là
			\[T_6=(106{,}09+100)(1+3\%)^6\approx 246\  \text{triệu đồng}.\]
		\end{itemchoice}
	}
\end{ex}	
%%Cau5
\begin{ex}%[1D6V1-1]%[Dự án đề cương 3 Khối NH24-25-Đợt 2 - BCTuan]
	Tại một xí nghiệp, công thức $P(t)=500\cdot\left(\dfrac{1}{2}\right)^{\tfrac{t}{3}}$ được dùng để tính giá trị còn lại (tính theo triệu đồng) của một chiếc máy sau thời gian $t$ (tính theo năm) kể từ khi đưa vào sử dụng.
	\choiceTF
	{\True Giá trị còn lại của máy sau $3$ năm sử dụng là $250$ triệu đồng}
	{Giá trị còn lại của máy sau $4$ năm $3$ tháng sử dụng gần bằng $180$ triệu đồng}
	{\True Sau $2$ năm đưa vào sử dưng thì giá trị của chiếc máy giảm $185$ triệu đồng so với giá trị ban đầu}
	{\True Sau $1$ năm đưa vào sử dụng thì giá trị của chiếc máy giảm $20{,}6\%$ so với giá trị ban đầu của nó}
	\loigiai
	{
		\begin{itemchoice}
			\itemch \textbf{Đúng}.\\
			Giá trị còn lại của máy sau $3$ năm sử dụng là
			$$
			P(3)=500 \cdot\left(\dfrac{1}{2}\right)^{\tfrac{3}{3}}=250 \text { (triệu đồng).}
			$$
			\itemch \textbf{Sai}.\\
			Đổi: $4$ năm $3$ tháng $=4{,}25$ năm.\\
			Giá trị còn lại của máy sau $4$ năm $3$ tháng sử dụng là
			$$
			P(4{,}25)=500\cdot\left(\dfrac{1}{2}\right)^{\tfrac{4{,}25}{3}} \approx 187 \text { (triệu đồng).}
			$$
			\itemch \textbf{Đúng}.\\
			Giá trị của máy lúc ban đầu là $P(0)=500$ (triệu đồng).\\
			Giá trị còn lại của máy sau $2$ năm đưa vào sử dụng là	
			$$
			P(2)=500\cdot\left(\dfrac{1}{2}\right)^{\tfrac{2}{3}} \approx 315 \text { (triệu đồng). }
			$$
			Vậy sau $2$ năm đưa vào sử dụng thì giá trị của chiếc máy giảm $500-315=185$ triệu đồng so với giá trị ban đầu.
			\itemch \textbf{Đúng}.\\
			Giá trị của máy lúc ban đầu là $P(0)=500$ (triệu đồng).\\
			Giá trị còn lại của máy sau $1$ năm đưa vào sử dụng là
			$$
			P(1)=500 \cdot\left(\dfrac{1}{2}\right)^{\tfrac{1}{3}} \approx 397 \text { (triệu đồng). }
			$$
			Vậy sau $1$ năm đưa vào sử dụng thì giá trị của chiếc máy giảm $$100 \%-\dfrac{397}{500} \cdot 100 \approx 20{,}6 \%$$ so với giá trị ban đầu của nó.
		\end{itemchoice}
	}
\end{ex}
\Closesolutionfile{ans}

\ind{PHẦN III.} \inden{Câu trắc nghiệm trả lời ngắn.}\\
\setcounter{ex}{0}
\Opensolutionfile{ans}[ans/1D6-Bai1-KQ]
%%%Cau1
\begin{ex}%[1D6H1-2]%[Dự án đề cương 3 Khối NH24-25-Đợt 2 - BCTuan]
	(\textit{\footnotesize Trích đề thi GKII - THPT Núi Thành - Quảng Nam - Năm học 2024-2025})\\
		Cho $x$ là số thực dương, biết rằng $P=\dfrac{\sqrt[5]{x^3}}{\sqrt{x}}=x^{\tfrac{a}{b}}$ với $\dfrac{a}{b}$ là phân số tối giản. Giá trị biểu thức $5a-b$ bằng bao nhiêu?
		\par
	\shortans{$-5$}
	\loigiai{Với $x>0$, ta có $P=\dfrac{\sqrt[5]{x^3}}{\sqrt{x}}=\dfrac{x^{\tfrac{3}{5}}}{x^{\tfrac{1}{2}}}=x^{\tfrac{3}{5}-\tfrac{1}{2}}=x^{\tfrac{1}{10}}$.\\
		Suy ra $\heva{&a=1\\&b=10}$ và $5a-b=5-10=-5$.}
\end{ex}
%%%Cau2
\begin{ex}%[1D6H1-2]%[Dự án đề cương 3 Khối NH24-25-Đợt 2 - Bùi Lương Phúc]
	(\textit{\footnotesize Theo đề thi GKII - THPT Hậu nghĩa - Long An - Năm học 2024-2025})\\
	Với $a > 0$, viết biểu thức $\frac{\sqrt{a} \cdot \sqrt[3]{a} \cdot \sqrt[4]{a}}{\left( \sqrt[5]{a} \right)^3 \cdot a^{\frac{2}{5}}}$ viết dưới dạng một lũy thừa của $a$ với số mũ là $\dfrac{m}{n}$ trong đó $\dfrac{m}{n}$ là phân số tối giản. Giá trị của $m+n$ bằng bao nhiêu?
		\par
	\shortans{$13$}
	\loigiai{
		Ta có $\sqrt{a} = a^{\tfrac{1}{2}},\quad \sqrt[3]{a} = a^{\tfrac{1}{3}},\quad \sqrt[4]{a} = a^{\tfrac{1}{4}},\quad \left(\sqrt[5]{a}\right)^3 = a^{\tfrac{3}{5}}$.\\
		Biến đổi biểu thức
		\[
		\dfrac{a^{\tfrac{1}{2}} \cdot a^{\tfrac{1}{3}} \cdot a^{\tfrac{1}{4}}}{a^{\tfrac{3}{5}} \cdot a^{\tfrac{2}{5}}} = \dfrac{a^{\tfrac{1}{2} + \tfrac{1}{3} + \tfrac{1}{4}}}{a^{\tfrac{3}{5} + \tfrac{2}{5}}} = \dfrac{a^{\tfrac{13}{12}}}{a^1} = a^{\tfrac{13}{12} - 1} = a^{\tfrac{1}{12}}.
		\]		
		Suy ra $\dfrac{m}{n}=\frac{1}{12}\Rightarrow \heva{&m=1\\&n=12.}$\\
		Vậy $m+n=1+12=13$.
	}
\end{ex}
%%%Cau3
\begin{ex}%[1D6H1-1]%[Dự án đề cương 3 Khối NH24-25-Đợt 2 - BCTuan]
	Biểu thức $B=\dfrac{\left(4+2\sqrt{3} \right)^{2\,018{,}5}\cdot \left(1-\sqrt{3} \right)^{2\,017 }}{\left(1+\sqrt{3} \right)^{\,2019 }}$ sau khi rút gọn được kết quả có dạng $\left(a+b\sqrt{3} \right)\cdot 2^{2\,017}$ với $a$, $b$ là những số nguyên. Giá trị của tổng $a+b$ bằng bao nhiêu?
		\par
\shortans{$-2$}
	\loigiai{
		Ta có
		\begin{eqnarray*}
			B	&= &\dfrac{\left(4+2\sqrt{3} \right)^{2\,018{,}5}\cdot \left(1-\sqrt{3} \right)^{2\,017 }}{\left(1+\sqrt{3} \right)^{2\,019 }}   \\
			&=&\dfrac{\left(1+\sqrt{3} \right)^{4\,037}\cdot \left(1-\sqrt{3} \right)^{2\,017 }}{\left(1+\sqrt{3} \right)^{2\,019 }}\\
			&= & \left(1+\sqrt{3} \right)^{2\,018}\cdot \left(1-\sqrt{3} \right)^{2\,017 }\\
			&=&\left(1+\sqrt{3} \right)\cdot\left[\left(1+\sqrt{3} \right)\cdot \left(1-\sqrt{3} \right) \right]^{2\,017 }\\
			&=& \left(1+\sqrt{3} \right)\cdot(-2)^{2\,017 }\\
			&=& \left(1+\sqrt{3} \right)\cdot\left(-2^{2\,017 }\right)\\
			&=&\left(-1-\sqrt{3} \right)\cdot2^{2\,017}.
		\end{eqnarray*}
			Suy ra $a=-1$, $b=-1$. Vậy $a+b=-1-1=-2$.
			}
\end{ex}
%%%Cau4
\begin{ex}%[1D6H1-1]%[Dự án đề cương 3 Khối NH24-25-Đợt 2 - BCTuan]
	Biết $9^{\alpha}=\dfrac{1}{2}$. Tính $B=\left(3^{\alpha}+3^{-\alpha}\right)^2-\left(81^{\alpha}-81^{-\alpha}\right)$.
		\par
	\shortans{$8{,}25$}
	\loigiai{
		$$
		\begin{aligned}
			B&=\left(3^\alpha\right)^2+2 \cdot 3^\alpha \cdot 3^{-\alpha}+\left(3^{-\alpha}\right)^2-\left(9^2\right)^\alpha+\left(9^2\right)^{-\alpha} \\
			& =3^{2\alpha}+2 \cdot 3^{\alpha+(-\alpha)}+\left(3^2\right)^{-\alpha}-\left(9^\alpha\right)^2+\left(9^{-\alpha}\right)^2 \\
			& =9^\alpha+2 \cdot 3^0+9^{-\alpha}-\left(9^\alpha\right)^2+\left(9^\alpha\right)^{-2}\\
			&=\dfrac{1}{2}+2+\left(\dfrac{1}{2}\right)^{-1}-\left(\dfrac{1}{2}\right)^2+\left(\dfrac{1}{2}\right)^{-2}\\
			&=\dfrac{33}{4}\\
			&= 8{,}25.
		\end{aligned}
		$$
	}
\end{ex}
%%%Cau5
\begin{ex}%[1D6V1-1]%[Dự án đề cương 3 Khối NH24-25-Đợt 2 - BCTuan]
	Một khu rừng có trữ lượng gỗ là $400\,000$ mét khối. Biết tốc độ sinh trưởng của các cây lấy gỗ trong khu rừng này là $4 \%$ mỗi năm. Nếu không khai thác trong vòng $5$ năm, trữ lượng gỗ (đơn vị: nghìn mét khối) của khu rừng sau $5$ năm là bao nhiêu nghìn mét khối (làm tròn kết quả đến hàng đơn vị)?	
		\par
	\shortans{$487$}
	\loigiai{
		Nếu trữ lượng gỗ của khu rừng ban đầu là $A$ thì\\
		Sau năm thứ nhất, lượng gỗ có được là $A+A r=A(1+r), r$ là tốc độ tăng trưởng/năm.\\
		Sau năm thứ hai, lượng gỗ có được là $A(1+r)+A(1+r) \cdot r=A(1+r)^2$.\\
		Ta chứng minh được công thức tính lượng gỗ trong khu rừng là $T_n=A(1+r)^n$.\\
		Vậy sau $5$ năm, lượng gỗ trong khu rừng là
		$$
		T_s=400\left(1+\dfrac{4}{100}\right)^5=486{,}661161\approx 487\text{~nghìn mét khối}.
		$$
	}
\end{ex}
\Closesolutionfile{ans}

\ind{PHẦN IV.} \inden{Tự luận.}\\
\setcounter{ex}{0}
\Opensolutionfile{ans}[ans/1D6-Bai1-TL]
%%%%Cau1
	\begin{ex}%[1D6H1-3]%[Dự án đề cương 3 Khối NH24-25-Đợt 2 - Bùi Lương Phúc]
	(\textit{\footnotesize Theo đề thi GKII - THPT Phan Đình Phùng - Hà Nội - Năm học 2024-2025})\\
		Với những giá trị nào của $a$ thì 
	\[
	(a - 1)^{\frac{4}{5}} > (a - 1)^{\frac{1}{2}} \text{ ?}
	\]
	\par
	\loigiai{
	Điều kiện để cho các biểu thức $(a - 1)^{\frac{4}{5}} $ và $(a - 1)^{\frac{1}{2}}$ đồng thời có nghĩa là $a-1>0\Leftrightarrow a>1$.\\
	Ta có các trường hợp sau
	\begin{enumerate}[\it TH1:]
		\item $a-1>1\Leftrightarrow a>2$, khi đó vì $\frac{4}{5}>\frac{1}{2}$ nên $(a - 1)^{\frac{4}{5}} > (a - 1)^{\frac{1}{2}}$ (thỏa mãn).
		\item $a-1<1\Leftrightarrow a<2$, khi đó vì $\frac{4}{5}>\frac{1}{2}$ nên $(a - 1)^{\frac{4}{5}} < (a - 1)^{\frac{1}{2}}$ (không thỏa mãn).
		\item $a-1=1\Leftrightarrow a=2$, khi đó $(a - 1)^{\frac{4}{5}}=1^{\frac{4}{5}}=1$ và $(a - 1)^{\frac{1}{2}}=1^{\frac{1}{2}}=1$ (không thỏa mãn).	
	\end{enumerate}
	Cùng với điều kiện ban đầu ta được giá trị của $a$ là $a>2$.	
	}	
\end{ex}
%%%%Cau2
\begin{ex}%[1D6H1-1]%[Dự án đề cương 3 Khối NH24-25-Đợt 2 - BCTuan]
	(\textit{\footnotesize Theo đề thi GKII - THPT Phan Bội Châu - Bình Thuận - Năm học 2024-2025})\\
	Cho $2^\alpha = 9$. Tính giá trị biểu thức $\left(\dfrac{1}{16}\right)^{\tfrac{\alpha}{8}}$.
		\par
	\loigiai{
		Ta có $\left(\dfrac{1}{16}\right)^{\tfrac{\alpha}{8}} = \left( \dfrac{1}{2^4} \right)^{\tfrac{\alpha}{8}} = \left(2^{-4}\right)^{\tfrac{\alpha}{8}} = 2^{-4 \cdot \tfrac{\alpha}{8}} = 2^{-\tfrac{\alpha}{2}} = (2^\alpha)^{-\tfrac{1}{2}}$.\\
		Thay $2^\alpha = 9$ vào biểu thức ta được $(9)^{-\tfrac{1}{2}} = \dfrac{1}{9^{\tfrac{1}{2}}} = \dfrac{1}{\sqrt{9}} = \dfrac{1}{3}$.
	}  
\end{ex}
%%%%Cau3
\begin{ex}%[1D6V1-2]%[Dự án đề cương 3 Khối NH24-25-Đợt 2 - BCTuan]
	(\textit{\footnotesize Trích đề thi GKII - THPT Lê Quý Đôn - Ninh Thuận - Năm học 2024-2025})\\
	Cho $a$ là số thực dương và $a \neq 1$. Rút gọn biểu thức sau
	$$A=\dfrac{2 \sqrt{a}}{\left(a+2 a^{\tfrac{1}{2}}+1\right)\left(\sqrt{a}-1\right)}\colon  \dfrac{a^{\tfrac{1}{2}}}{a^{\tfrac{1}{2}}+1}.$$
		\par
	\loigiai{
		Ta có
		$$A=\dfrac{2 \sqrt{a}}{\left(a+2 a^{\tfrac{1}{2}}+1\right)\left(\sqrt{a}-1\right)}:  \dfrac{a^{\tfrac{1}{2}}}{a^{\tfrac{1}{2}}+1}=\dfrac{2 \sqrt{a}}{\left(a+2\sqrt{a}+1\right)\left(\sqrt{a}-1\right)}\cdot \dfrac{\sqrt{a}+1}{ \sqrt{a}}=\dfrac{2}{\left(\sqrt{a}+1\right)\left(\sqrt{a}-1\right)}=\dfrac{2}{a-1}.$$
	}
\end{ex}
%%%%Cau4
\begin{ex}%[1D6V1-2]%[Dự án đề cương 3 Khối NH24-25-Đợt 2 - BCTuan]
	(\textit{\footnotesize Trích đề thi GKII - THPT Thị xã Quảng Trị - Quảng Trị - Năm học 2024-2025})\\
	Rút gọn biểu thức $P=\dfrac{a^{\sqrt{7}+1}\cdot a^{2-\sqrt{7}}}{(a^{\sqrt{2}-2})^{\sqrt{2}+2}}$ về dạng một luỹ thừa của $a$	 với $a>0$.
		\par
	\loigiai{
		Ta có $P=\dfrac{a^{\sqrt{7}+1}\cdot a^{2-\sqrt{7}}}{(a^{\sqrt{2}-2})^{\sqrt{2}+2}}=\dfrac{a^{\sqrt{7}+1+2-\sqrt{7}}}{a^{(\sqrt{2}-2)(\sqrt{2}+2)}}=\dfrac{a^3}{a^{-2}}=a^5$.
	}
\end{ex}
%%%%Cau5
\begin{ex}%[1D6H1-4]%[Dự án đề cương 3 Khối NH24-25-Đợt 2 - BCTuan]
	Tìm điều kiện của $a$ (với $a>1$) để
	\begin{enumEX}[a)]{2}
		\item  $(a-1)^{-\tfrac{2}{3}}<(a-1)^{-\tfrac{1}{3}}$;
		\item  $(a-1)^{-2}>(a-1)^{\sqrt{2}}$.
	\end{enumEX}
	\loigiai{
		\begin{enumerate}[a)]
			\item Ta có $\heva{&-\dfrac{2}{3}<-\dfrac{1}{3}\\&(a-1)^{-\tfrac{2}{3}}<(a-1)^{-\tfrac{1}{3}}}\Rightarrow a-1>1\Rightarrow a>2$
			\item Vì $\heva{& -2<\sqrt{2} \\ & (a-1)^{-2}>(a-1)^{\sqrt{2}}}$ nên $0<a-1<1\Leftrightarrow 1<a<2$.
	\end{enumerate}
}
\end{ex}
%%%%Cau6
\begin{ex}%[1D6V1-1]%[Dự án đề cương 3 Khối NH24-25-Đợt 2 - BCTuan]
	Cho số thực dương $\alpha$ thỏa mãn $3^{\alpha}+3^{-\alpha}=4$. Tính giá trị của  biểu thức $3^{\tfrac{\alpha}{2}}-3^{\tfrac{-\alpha}{2}}$.
	\loigiai{
Ta có $\left(3^{\tfrac{\alpha}{2}}-3^{-\tfrac{\alpha}{2}}\right)^2=\left(3^{\tfrac{\alpha}{2}}\right)^2-2 \cdot 3^{\tfrac{\alpha}{2}} \cdot 3^{-\tfrac{\alpha}{2}}+\left(3^{-\tfrac{\alpha}{2}}\right)^2=3^\alpha+3^{-\alpha}-2\cdot3^0=4-2=2$.\\
Mặt khác, $\alpha>0\Rightarrow \heva{&3^\alpha>1\\&1>3^{-\alpha}}\Rightarrow 3^\alpha>3^{-\alpha}\Rightarrow3^\alpha-3^{-\alpha}>0$.\\
		Suy ra $3^{\tfrac{\alpha}{2}}-3^{\tfrac{\alpha}{2}}=\sqrt{2}$.	
	}
\end{ex}
%%%%Cau7
\begin{ex}%[1D6V1-1]%[Dự án đề cương 3 Khối NH24-25-Đợt 2 - BCTuan]
	Biết rằng $4^x=25^{y}=10$. Tính giá trị của biểu thức $\dfrac{1}{x}+\dfrac{1}{y}$.
	\loigiai{
		Ta có $4^x=10 \Rightarrow 10^{\tfrac{1}{x}}=4$; $25^y=10 \Rightarrow 10^{\tfrac{1}{y}}=25$.\\
		Suy ra $10^{\tfrac{1}{x}+\tfrac{1}{y}}=4\cdot25=100=10^2 \Rightarrow \dfrac{1}{x}+\dfrac{1}{y}=2$.	
	}
\end{ex}
%%%%Cau8
\begin{ex}%[1D6V1-1]%[Dự án đề cương 3 Khối NH24-25-Đợt 2 - BCTuan]
	Số lượng của loại vi khuẩn A trong một phòng thí nghiệm được tính theo công thức $s(t)=s(0) \cdot 2^t$, trong đó $s(0)$ là số lượng vi khuẩn A lúc ban đầu, $s(t)$ là số lượng vi khuẩn A có sau $t$ phút. Biết sau $3$ phút thì số lượng vi khuẩn A là $625$ nghìn con. Hỏi sau bao nhiêu phút, kể từ lúc ban đầu, số lượng vi khuẩn A là $10$ triệu con?
	\loigiai{
		Ta có $s(3)=s(0) \cdot 2^3 \Rightarrow s(0)=\dfrac{s(3)}{8}=78{,}125$ nghìn con.\\
		Do đó $s(t)=10$ triệu con $=10\,000$ nghìn con khi
		$$10\,000=s(0) \cdot 2^t \Rightarrow 2^t=\dfrac{10\,000}{78{,}125}=128=2^7 \Rightarrow t=7\text{ phút}.$$
	}
\end{ex}
%%%%Cau9
\begin{ex}%[1D6V1-1]%[Dự án đề cương 3 Khối NH24-25-Đợt 2 - BCTuan]
	Nếu một khoản tiền gốc $P$ được gửi ngân hàng với lãi suất hằng năm $r$ ($r$ được biểu thị dưới dạng số thập phân), được tính lãi $n$ lần trong một năm, thì tổng số tiền $A$ nhận được (cả vốn lẫn lãi) sau $N$ kì gửi cho bởi công thức sau:
	$$
	A=P\left(1+\dfrac{r}{n}\right)^N .
	$$
	Hỏi nếu bác An gửi tiết kiệm số tiền $120$ triệu đồng theo kì hạn $6$ tháng với lãi suất không đổi là $5 \%$ một năm, thì số tiền thu được (cả vốn lẫn lãi) của bác An sau $2$ năm là bao nhiêu triệu đồng (kết quả làm tròn đến hàng phần trăm)?
	\loigiai{
		Ta có $2$ năm là $24$ tháng ứng với $N=4$ kì hạn.\\
		Do kì hạn là $6$ tháng nên mỗi năm được tính lãi $n=2$ lần.\\
		Vậy số tiền cả vốn lẫn lãi bác An nhận được sau $2$ năm là $A=120\left(1+\dfrac{0{,}05}{2}\right)^4\approx 132{,}46$ triệu đồng.
	}
\end{ex}
%%%%Cau10
\begin{ex}%[1D6V1-1]%[Dự án đề cương 3 Khối NH24-25-Đợt 2 - BCTuan]
	Định luật thứ ba của Kepler nói rằng bình phương của chu kì quỹ đạo $p$ (tính bằng năm Trái Đất) của một hành tỉnh chuyển động xung quanh Mặt Trời (theo quỹ đạo là một đường elip với Mặt Trời nằm ở một tiêu điểm) bẳng lập phương của bán trục lớn d (tính bằng đơn vị thiên văn $AU$).
	\begin{enumEX}[a)]{1}
		\item  Tính $p$ theo $d$.
		\item  Nếu Sao Thỗ có chu kì quỹ đạo là $29{,}46$ năm Trái Đất, hãy tính bán trục lớn quỹ đạo của Sao Thổ đến Mặt Trời (\textit{kết quả tính theo đơn vị thiên văn và làm tròn đến hàng phần trăm}).
	\end{enumEX}
	\loigiai{
		\begin{enumerate}
			\item  Theo định luật thứ ba của Kepler, ta có
			$$
			p^2=d^3\ \text {hay}\ p=\sqrt{d^3}.
			$$			
			\item  Thay $p=29{,}46$ vào công thức $p=\sqrt{d^3}$, ta được $d \approx 9{,}54 AU$.
		\end{enumerate}
	}
\end{ex}
\Closesolutionfile{ans}


































