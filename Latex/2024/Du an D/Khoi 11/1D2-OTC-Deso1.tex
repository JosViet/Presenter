\newpage
\section{Ôn tập chương 2}
\def\thoigian{90}%--Thời gian
\de{Đề số 1}{Chương II. Dãy số. Cấp số cộng. Cấp số nhân}


\begin{center}
	\textbf{PHẦN 1 - CÂU TRẮC NGHIỆM BỐN PHƯƠNG ÁN}
\end{center}
\Opensolutionfile{ans}[ans/ans-TN-1D2-DE2]

\begin{ex}%[1D2N1-3]%[Dự án D - đợt 2 NH24-25- Hứa Chí Ninh]
	Cho dãy số $(u_n)$ với $u_n=\dfrac{2n^2-1}{n^2+3}$, $\forall n \in \mathrm{N^*}$. Số hạng đầu tiên của dãy số là 
	\choice
	{$ u_1=-\dfrac{1}{3}$}
	{$ u_1=\dfrac{2}{3}$}
	{$ u_1=\dfrac{1}{3}$}
	{\True $ u_1=\dfrac{1}{4}$ }
	\loigiai
	{Ta có $u_1=\dfrac{2\cdot 1^2-1}{1^2+3}=\dfrac{1}{4}$.
	}
\end{ex}

\begin{ex}%[1D2H1-3]%[Dự án D - đợt 2 NH24-25- Hứa Chí Ninh]
	Cho dãy số $(u_n)$, biết $u_n=\dfrac{2n^2-1}{n^2+3}$. Tìm số hạng $u_5$.
	\choice
	{$u_5=\dfrac{1}{4}$}
	{\True $u_5=\dfrac{7}{4}$}
	{$u_5=\dfrac{17}{12}$}
	{$u_5=\dfrac{71}{39}$}
	\loigiai{
		Ta có $u_5=\dfrac{2\cdot 5^2-1}{5^2+3}=\dfrac{49}{28}=\dfrac{7}{4}$.}
	
\end{ex}

\begin{ex}%[1D2H1-2]%[Dự án D - đợt 2 NH24-25- Hứa Chí Ninh]
	Cho dãy số $(u_n )$ xác định bởi $\heva{
		& u_1=2 \\
		& u_{n+1}=\dfrac{1}{3}(u_n+1 ) \\
	}.$ Tìm số hạng $u_4$.
	\choice
	{$u_4=\dfrac{2}{3}$}
	{$u_4=1$}
	{$u_4=\dfrac{14}{27}$}
	{ \True $u_4=\dfrac{5}{9}$}
	\loigiai {
		Ta có
		$u_2=\dfrac{1}{3}(u_1+1 )=\dfrac{1}{3}(2+1 )=1$; $u_3=\dfrac{1}{3}(u_2+1 )=\dfrac{2}{3}$; $u_4=\dfrac{1}{3}(u_3+1 )=\dfrac{1}{3}\cdot\left(\dfrac{2}{3}+1\right)=\dfrac{5}{9}$. \\
	}
\end{ex}

\begin{ex}%[1D2N2-2]%[Dự án D - đợt 2 NH24-25- Hứa Chí Ninh]
	Cho cấp số cộng $(u_n)$ với $u_1=2$ và $u_2=6$. Công sai của cấp số cộng đã cho là	
	\choice
	{\True $4$}
	{$-4$}
	{$8$}
	{$3$}
	\loigiai
	{
		Ta có $ u_2=6 \Leftrightarrow 6=u_1+d \Leftrightarrow d=4 $.
	}
\end{ex}

\begin{ex}%[1D2H2-2]%[Dự án D - đợt 2 NH24-25- Hứa Chí Ninh]
	Cho cấp số cộng $ (u_n) $ với $ u_1=-3 $ và $ u_6=27 $. Công sai $ d $ của cấp số cộng đã cho là	
	\choice
	{$ d=7 $}
	{$ d=5 $}
	{$ d=8 $}
	{\True $ d=6 $}
	\loigiai
	{
		Ta có $ u_6=27 \Leftrightarrow 27=u_1+5d \Leftrightarrow d=6$.
	}
\end{ex}

\begin{ex} %[1D2H2-2]%[Dự án D - đợt 2 NH24-25- Hứa Chí Ninh]
	Cho cấp số cộng có $ u_1=11 $ và công sai $ d=4 $. Hãy tính $ u_{99} $.
	\choice
	{$ 401 $}
	{\True $ 403 $}
	{$ 402 $}
	{$ 404 $}
	\loigiai
	{
		Ta có $ u_{99}=u_1+98d=11+98\cdot4=403 $.
	}
\end{ex}

\begin{ex}%[1D2N2-3]%[Dự án D - đợt 2 NH24-25- Hứa Chí Ninh]
	Cho cấp số cộng $(u_n )$ có các số hạng đầu lần lượt là $5;\,9;\,13;\,17;\cdots $. Tìm số hạng tổng quát $u_n$ của cấp số cộng.
	\choice
	{$u_n=5n+1$}
	{$u_n=5n-1$}
	{\True $u_n=4n+1$}
	{$u_n=4n-1$}
	\loigiai{
		Vì các số $5;\,9;\,13;\,17;\cdot$ theo thứ tự đó lập thành cấp số cộng $(u_n)$ nên	$\heva{
			&u_1=5 \\
			& d=u_2-u_1=4\\
		}$.\\
		Suy ra $u_n=u_1+(n-1 )d=5+4(n-1 )=4n+1$.}
\end{ex}

\begin{ex} %[1D2H2-4]%[Dự án D - đợt 2 NH24-25- Hứa Chí Ninh]
	Cho cấp số cộng $(u_n)$ có $u_5=-15$, $u_{20}=60$. Tổng $S_{20}$ của $20$ số hạng đầu tiên của cấp số cộng là
	\choice
	{$S_{20}=600$}
	{$S_{20}=60$}
	{\True $S_{20}=250$}
	{$S_{20}=500$}
	\loigiai{
		Ta có $\left \{\begin{aligned}
			&u_5=-15 \\
			&u_{20}=60
		\end{aligned} \right. \Leftrightarrow \left \{\begin{aligned}
			&u_1+4d=-15 \\
			&u_1+19d=60
		\end{aligned} \right. \Leftrightarrow \left \{\begin{aligned}
			&u_1=-35 \\
			&d=5
		\end{aligned} \right. $ ($d$ là công sai của cấp số cộng).\\
		$\Rightarrow S_{20}=20u_1+ \dfrac{20\cdot 19}{2}d=20(-35)+ \dfrac{20\cdot19}{2}\cdot5=250$.}
\end{ex}

\begin{ex}%[1D2N3-2]%[Dự án D - đợt 2 NH24-25- Hứa Chí Ninh]
	Dãy số $1$; $2$; $4$; $8$; $16$; $32$; $\ldots$ là một cấp số nhân với 
	\choice
	{Công bội là $1$ và số hạng đầu tiên là $2$}
	{\True Công bội là $2$ và số hạng đầu tiên là $1$}
	{Công bội là $2$ và số hạng đầu tiên là $2$}
	{Công bội là $1$ và số hạng đầu tiên là $1$}
	\loigiai{
		Ta có $ q=\dfrac{u_2}{u_1}=\dfrac{u_3}{u_2}=\ldots=2 $. \\
		Vậy dãy số đã cho là một cấp số nhân với công bội là $ q=2 $ và số hạng đầu tiên là $ u_1=1 $.
	}
\end{ex}

\begin{ex}%[1D2N3-2]%[Dự án D - đợt 2 NH24-25- Hứa Chí Ninh]
	Một cấp số nhân có hai số hạng liên tiếp là $3$ và $12$. Số hạng tiếp theo của cấp số nhân là
	\choice
	{$15$}
	{$21$}
	{$36$}
	{\True $48$}
	\loigiai{
		Một cấp số nhân có hai số hạng liên tiếp là $3$ và $12$, do đó $ q=\dfrac{u_{n+1}}{u_n}=\dfrac{12}{3}=4$.\\
		Vậy số hạng tiếp theo của cấp số nhân đó là $ u_{n+2}=u_{n+1}q=12\cdot 4=48 $.
	}
\end{ex}


\begin{ex}%[1D2V3-4]%[Dự án D - đợt 2 NH24-25- Hứa Chí Ninh]
	Cho tứ giác $ABCD$ có bốn góc tạo thành cấp số nhân có công bội $ q=2 $. Góc có số đo nhỏ nhất trong bốn góc đó là
	\choice
	{\True $ 24^\circ $}
	{$ 1^\circ $}
	{$ 12^\circ $}
	{$ 30^\circ $}
	\loigiai{
		Gọi số đo bốn góc của tứ giác $ ABCD $ là $ x $, $ 2x $, $ 4x $, $ 8x $.
		\\ Khi đó $ x+2x+4x+8x=360 \Leftrightarrow 15x=360 \Leftrightarrow x=24 $.}
\end{ex}

\begin{ex}[Trích đề kiểm tra Toán 11 HKI THPT Lê Quý Đôn TPHCM NH24-25]%[1D2H3-5]
	Có bao nhiêu số thực $x$ để $2x-1$; $x$; $2x+1$ theo thứ tự lập thành cấp số nhân?
	\choice
	{$3$}
	{\True $2$}
	{$1$}
	{$0$}
	\loigiai{
		Ta có $(2x-1)(2x+1)=x^2\Leftrightarrow 4x^2-1=x^2\Leftrightarrow x^2=\dfrac{1}{3}\Leftrightarrow \hoac{&x=\dfrac{\sqrt{3}}{3}\\&x=-\dfrac{\sqrt{3}}{3}.}$\\
		Vậy có hai giá trị $x$ cần tìm.
	}
\end{ex}
\Closesolutionfile{ans}
%\begin{center}
%	\textbf{ĐÁP ÁN}
%	\inputansbox{10}{ans/ans-TN-1D2-DE2}	
%\end{center}
\begin{center}
	\textbf{PHẦN 2 - CÂU TRẮC NGHIỆM ĐÚNG SAI}
\end{center}
\setcounter{ex}{0}
\Opensolutionfile{ans}[ans/ans-DS-1D2-DE2]

\begin{ex}%[1D2N1-3]%[Dự án D - đợt 2 NH24-25- Hứa Chí Ninh]
	Cho dãy số $\left(u_n\right)$, biết $u_n=\dfrac{-n}{n+1}$. Khi đó
	\choiceTF
	{\True $u_1=-\dfrac{1}{2} ; u_2=-\dfrac{2}{3} ; u_3=-\dfrac{3}{4} ; u_4=-\dfrac{4}{5} ; u_5=-\dfrac{5}{6}$}
	{\True Số hạng $u_{10}, u_{100}$ lần lượt là $-\dfrac{10}{11} ;-\dfrac{100}{101}$}
	{$-\dfrac{85}{86}$ là số hạng thứ 86 của dãy số $\left(u_n\right)$}
	{$-\dfrac{99}{101}$ là một số hạng của dãy số $\left(u_n\right)$}
	\loigiai{
		\begin{itemchoice}
			\itemch Ta có $u_1=-\dfrac{1}{2} ; u_2=-\dfrac{2}{3} ; u_3=-\dfrac{3}{4} ; u_4=-\dfrac{4}{5} ; u_5=-\dfrac{5}{6}$.
			\itemch  Ta có $u_{10}=-\dfrac{10}{11}, u_{100}=-\dfrac{100}{101}$.
			\itemch  $-\dfrac{85}{86}=\dfrac{-n}{n+1} \Leftrightarrow 85 n+85=86 n \Leftrightarrow n=85$.
			\itemch $-\dfrac{99}{101}=\dfrac{-n}{n+1} \Leftrightarrow 99 n+99=101 n \Leftrightarrow n=\dfrac{99}{2} \notin \mathbb{N}^*$ (loại).
		\end{itemchoice}
	}
\end{ex}

\begin{ex}%[1D2N2-4]%[Dự án D - đợt 2 NH24-25- Hứa Chí Ninh]
	Cho cấp số cộng $\left(u_n\right)$ có số hạng đầu $u_1=\dfrac{3}{2}$, công sai $d=\dfrac{1}{2}$. Khi đó
	\choiceTF
	{Công thức cho số hạng tổng quát $u_n=1+\dfrac{n}{3}$}
	{\True $5$ là số hạng thứ $8$ của cấp số cộng đã cho}
	{$\dfrac{15}{4}$ một số hạng của cấp số cộng đã cho}
	{Tổng $100$ số hạng đầu của cấp số cộng $\left(u_n\right)$ bằng $2620$}
	\loigiai{
		\begin{itemchoice}
			\itemch $u_n=u_1+(n-1) d=\dfrac{3}{2}+(n-1) \cdot \dfrac{1}{2}=1+\dfrac{n}{2}.$
			\itemch  Vì $5=1+\dfrac{n}{2} \Rightarrow n=8 \in \mathbb{N}^*$ nên $u_8=5$.
			\itemch Vì $\dfrac{15}{4}=1+\dfrac{n}{2} \Rightarrow n=\dfrac{11}{2} \notin \mathbb{N}^*$ nên $\dfrac{15}{4}$ không phải là số hạng của cấp số cộng.
			\itemch $S_{100}=\dfrac{100\left[2 \cdot \dfrac{3}{2}+(100-1) \cdot \dfrac{1}{2}\right]}{2}=2625.$
		\end{itemchoice}
	}
\end{ex}
\Closesolutionfile{ans}
%\begin{center}
%\textbf{ĐÁP ÁN}
%\inputansbox[2]{2}{ans/ans-DS-1D2-DE2}
%\end{center}
\begin{center}
	\textbf{PHẦN 3 - CÂU TRẮC NGHIỆM TRẢ LỜI NGẮN}
\end{center}
\setcounter{ex}{0}
\Opensolutionfile{ans}[ans-KQ-1D2-DE2]

\begin{ex}%[1D2N2-6]%[Dự án D - đợt 2 NH24-25- Hứa Chí Ninh]
	Cho cấp số cộng $\left(u_n\right)$ có số hạng đầu $u_1=2$ và công sai $d=-3$. Tính tổng $10$ số hạng đầu của $\left({u_n}\right)$.
	\shortans[oly]{-115}
	\loigiai{
		$S_{10}=\dfrac{10\left(2\cdot2+9\cdot(-3)\right)}{2}=-115$.
	}
\end{ex}

\begin{ex}%[1D2H2-6]%[Dự án D - đợt 2 NH24-25- Hứa Chí Ninh]
	Cho cấp số cộng $\left(u_n\right)$ có $u_4=-12$, $u_{14}=18$. Tính tổng $16$ số hạng đầu tiên của cấp số cộng này. 
	\shortans[oly]{24}
	\loigiai{
		Gọi $d$ là công sai của cấp số cộng $\left(u_n\right)$, ta có
		$$\heva{&u_4=-12\\&u_{14}=18}\Leftrightarrow \heva{&u_1+3d=-12\\&u_1+13d=18}\Leftrightarrow \heva{&u_1=-21\\&d=3.}$$
		Vì $u_{16}=u_1+15d=-21+15\cdot 3=24$.\\
		Nên $S_{16}=\dfrac{n\left(u_1+u_{16}\right)}{2}=\dfrac{16\left(-21+24\right)}{2}=24$.
	}
\end{ex}

\begin{ex}%[1D2H3-8]%[Dự án D - đợt 2 NH24-25- Hứa Chí Ninh]
	Giả sử một người đi làm được lĩnh lương khởi điểm là $2\, 000\, 000$ đồng/tháng. Cứ $3$ năm người ấy lại được tăng lương một lần với mức tăng bằng $7\%$ của tháng trước đó. Hỏi sau $36$ năm làm việc người ấy lĩnh được tất cả bao nhiêu triệu đồng? (làm tròn kết quả đến hàng đơn vị)
	\shortans[oly]{1288}
	\loigiai{
		Ta có $36$ năm tương ứng với $12$ kỳ lương; mỗi kỳ lương có $36$ tháng và kỳ sau tăng $7\%$ so với kỳ trước. Do đó tổng số tiền mỗi kỳ lương là một cấp số nhân với $u_1=36\times 2=72$ (triệu đồng) và công bội $q=1{,}07$.\\
		Vậy tổng số tiền sau $36$ năm là $T=\dfrac{72\cdot \left[(1{,}07)^{12}-1\right]}{1{,}07-1}1\, 287\, 968\, 492\approx 1\,288$ (triệu đồng).
	}
\end{ex}

\begin{ex}%[1D2V3-4]%[Dự án D - đợt 2 NH24-25- Hứa Chí Ninh]
	Cho dãy số $(u_n)$ được xác định bởi $ u_1=2,u_n=2u_{n-1}+3n-1 $. Công thức số hạng tổng quát của dãy số đã cho là biểu thức có dạng $ a2^n+bn+c $, với $ a,b,c\in\mathbb{Z},n\ge 2, n\in\mathbb{N} $. Khi đó tổng $ a+b+c $ có giá trị bằng
	\shortans[oly]{-3}
	\loigiai{
		Ta có $ u_n=2u_{n-1}+3n-1 \Leftrightarrow u_n+3n+5=2\left[u_{n-1}+3(n-1)+5\right] $ với $ n\ge 2, n\in\mathbb{N} $.\\
		Đặt $ v_n=u_n+3n+5 $, ta có $ v_n=2v_{n-1} $ với $ n\ge 2, n\in\mathbb{N} $.\\
		Dễ thấy $ (v_n) $ là cấp số nhân với công bội $ q=2 $ và $ v_1=10 $.\\
		Do đó $ v_n=10\cdot 2^{n-1}=5\cdot 2^n $.\\
		Suy ra $ u_n+3n+5=5\cdot 2^n $ hay $ u_n=5\cdot 2^n-3n-5 $ với $ n\ge 2, n\in\mathbb{N} $.\\
		Vậy $ a=5,b=-3,c=-5 $, suy ra $ a+b+c=-3 $.
	}
\end{ex}

\Closesolutionfile{ans}
\begin{center}
	\textbf{PHẦN 4 - TỰ LUẬN}
\end{center}

\begin{ex}[Trích đề kiểm tra GHK1 Trường THPT Nguyễn Thị Minh Khai, Năm học 2024-2025]%[1D2H2-2]
	Chứng minh dãy số $\left(u_n\right)$ với $u_n=2024n-2025$ là cấp số cộng. Xác định công sai, số hạng đầu của cấp số cộng đó.
	( 1 điểm).
	\loigiai{
		Ta có $u_n=2024n-2025\Rightarrow u_{n+1}=2024(n+1)-2025=2024n-1$.\\
		Vì $u_{n+1}-u_n=2024$ nên dãy số $\left(u_n\right)$ là một cấp số cộng với $\heva{&d=2024\\&u_1=-1.}$
	}
\end{ex}
\begin{ex}%[1D2H2-7]%[Dự án D - đợt 2 NH24-25- Hứa Chí Ninh]
	Người ta trồng  cây theo dạng một hình tam giác như sau: hàng thứ nhất trồng $ 1 $ cây, hàng thứ hai trồng $ 3 $ cây, hàng thứ ba trồng $ 5 $ cây, $\ldots$ cứ tiếp tục trồng như thế cho đến khi hết số cây là $ 6\,561 $. Số hàng cây được trồng là bao nhiêu?
	\loigiai{
		Để giải bài toán này, ta cần tìm số hàng cây được trồng cho đến khi tổng số cây là $ 2023 $. 
		\begin{itemize}
			\item Hàng thứ nhất trồng $ 1 $ cây. 
			\item Hàng thứ hai trồng $ 3 $ cây ($ 1 $ cây $ + 2 $ cây).
			\item Hàng thứ ba trồng $ 5 $ cây ($ 1 $ cây $ + 2 $ cây $ + 2 $ cây).
			\item $\cdots$
		\end{itemize}
		Vậy ta thấy rằng số cây trồng trong hàng thứ $n$ là $(n-1)\cdot 2+1$. \\
		Số cây được trồng trong $n$ hàng đầu tiên là
		$$1 + 3 + 5 + \cdots + (2n-1) = n^2.$$ 
		Để tìm số hàng cây được trồng cho đến khi tổng số cây là $ 6561 $, ta giải phương trình sau\\ 
		$n^2 = 6\,561.$ 
		Vậy số hàng cây được trồng là $ 81 $ hàng.
	}
\end{ex}

\begin{ex}%[1D2V3-8]%[Dự án D - đợt 2 NH24-25- Hứa Chí Ninh]
	Từ độ cao $55{,}8$ (mét) của tháp nghiên Pisa nước Italia người ta thả một quả bóng cao su chạm xuống đất. Giả sử mỗi lần chạm đất bóng lại nảy lên độ cao bằng $\dfrac{1}{10}$ độ cao mà bóng đạt trước đó. Tổng độ dài hành trình (mét) của bóng được thả từ lúc ban đầu cho đến khi nó nằm yên trên mặt đất thuộc khoảng nào trong các khoảng sau đây?
	\loigiai{
		Đặt $u_1=55{,}8$ (mét) là quãng đường bóng rơi khi thả xuống, $u_{n+1}=\dfrac{1}{10^{n}} u_1, n\ge 1$ là quãng đường bóng rơi sau lần nảy lên thứ $n$. \\
		Ta có $(u_n)$ là dãy cấp số nhân với $u_1=55{,}8$ và công bội $q=\dfrac{1}{10}$.\\
		Suy ra tổng quãng đường quả bóng rơi xuống là $$\displaystyle \lim \limits_{n \rightarrow +\infty} u_1 \cdot \dfrac{1-q^n}{1-q}=\displaystyle \lim \limits_{n \rightarrow +\infty}55{,}8\cdot\dfrac{1-\left( \dfrac{1}{10}\right)^n}{1-\dfrac{1}{10}}=62 .$$
		Ngoài ra ta còn phải tính tổng quãng đường mà bóng nảy lên. Ta có tổng quãng đường bóng nảy lên bằng tổng quãng đường rơi của bóng trừ đi quãng đường thả rơi xuống.\\
		Vậy tổng quãng đường hành trình của quả bóng là $62+62-55{,}8=68{,}2$ (mét).
	}
\end{ex}