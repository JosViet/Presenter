\newpage
\def\thoigian{90}%--Thời gian
\de{Đề số 2}{Chương V. THỐNG KÊ}

\begin{center}
	\textbf{PHẦN 1 - CÂU TRẮC NGHIỆM BỐN PHƯƠNG ÁN}
\end{center}
\Opensolutionfile{ans}[ans/ans-TN-ONTAPCHUONGV-DE2]
\begin{ex}%[1D5N1-1]%[Dự án D đợt 4 - Nguyễn Hoàng Anh]%[Đề ôn tập Chương V. THỐNG KÊ - Khối 11 - Đề số 2]
	Cho mẫu số liệu ghép nhóm về thời gian (phút) đi từ nhà đến nơi làm việc của các nhân viên một công ty như sau:
	\begin{center}
		\begin{tabular}{|c|c|c|c|c|c|c|c|}
			\hline
			Thời gian & $[15{;}20$ & $[20{;}25)$ & $[25{;}30)$ & $[30{;}35)$ & $[35{;}40)$ & $[40{;}45)$ & $[45{;}50)$ \\
			\hline
			Số nhân viên & $6$ & $14$ & $25$ & $37$ & $21$ & $13$ & $9$ \\
			\hline
		\end{tabular}
	\end{center}
	Mẫu số liệu được chia thành bao nhiêu nhóm?
	\choice
	{$6$ nhóm}
	{$5$ nhóm}
	{\True $7$ nhóm}
	{$8$ nhóm}
	\loigiai{
		Mẫu số liệu được chia thành $7$ nhóm.
	}
\end{ex}
\begin{ex}%[1D5N1-1]%[Dự án D đợt 4 - Nguyễn Hoàng Anh]%[Đề ôn tập Chương V. THỐNG KÊ - Khối 11 - Đề số 2]
	Cho mẫu số liệu ghép nhóm về số tiền mà sinh viên chi cho thanh toán cước điện thoại trong tháng.
	\begin{center}
			\begin{tabular}{|>{\centering\arraybackslash}m{4cm}
				|>{\centering\arraybackslash}m{1.5cm}
				|>{\centering\arraybackslash}m{1.5cm}
				|>{\centering\arraybackslash}m{1.5cm}
				|>{\centering\arraybackslash}m{1.5cm}
				|>{\centering\arraybackslash}m{1.5cm}|}
			\hline
			Số tiền (nghìn đồng) & $[0{;}50)$ & $[50{;}100)$ & $[100{;}150)$ & $[150{;}200)$ & $[200{;}250)$  \\
			\hline
			Số sinh viên & $5$ & $12$ & $23$ & $17$ & $31$ \\
			\hline
		\end{tabular}
	\end{center}
	Có bao nhiêu sinh viên chi từ $100$ đến dưới $150$ nghìn đồng cho việc thanh toán cước điện thoại trong tháng?
	\choice
	{$5$}
	{\True $23$}
	{$12$}
	{$17$}
	\loigiai{
		Có $23$ sinh viên chi từ $100$ đến dưới $150$ nghìn đồng cho việc thanh toán cước điện thoại trong tháng.
	}
\end{ex}
\begin{ex}%%[1D5N1-2]%[Dự án D đợt 4 - Nguyễn Hoàng Anh]%[Đề ôn tập Chương V. THỐNG KÊ - Khối 11 - Đề số 2]
	Thống kê điểm thi của 30 học sinh trong kì thi học sinh giỏi môn Toán của một trường THPT (thang điểm 20) thu được mẫu số liệu ghép nhóm sau:
	\vspace{-.2cm}
	\begin{center}
		\begin{tabular}{|c|c|c|c|c|c|c}
			\hline
			Điểm &[10; 12)&[12; 14)&[14; 16)&[16; 18)&[18; 20]\\
			\hline
			Số học sinh &4&8&10&6&2\\
			\hline
		\end{tabular}
	\end{center}
	Giá trị đại diện của nhóm $[16; 18)$ là
	\choice
	{$16$}
	{$18$}
	{\True $17$}
	{$6$}
	\loigiai{
		Giá trị đại diện của nhóm $[16; 18)$ là $\dfrac{16+18}{2}=17$}
\end{ex}
\begin{ex}%[1D5N1-2]%[Dự án D đợt 4 - Nguyễn Hoàng Anh]%[Đề ôn tập Chương V. THỐNG KÊ - Khối 11 - Đề số 2]
	Chiều cao của $35$ cây bạch đàn (đơn vị: m)
	\begin{center}
		\begin{tabular}{|c|c|c|c|c|c|c|}
			\hline
			$6{,}6$ & $7{,}5$ & $8{,}2$ & $8{,}2$ & $7{,}8$ & $7{,}9$ &
			$9{,}0$\\
			\hline
			$8{,}9$ & $8{,}2$ & $7{,}2$ & $7{,}5$ & $8{,}3$ & $7{,}4$ & $8{,}7$\\
			\hline
			$7{,}7$ & $7{,}0$ & $9{,}4$ & $8{,}7$ & $8{,}0$ & $7{,}7$& $7{,}8$\\
			\hline 
			$8{,}3$ & $8{,}6$ & $8{,}1$ & $8{,}1$ & $8{,}5$ & $6{,}9$ &		$8{,}0$\\
			\hline
			$7{,}6$ & $7{,}9$ &$7{,}3$ & $8{,}5$ & $8{,}4$ & $8{,}0$ & $8{,}8$\\
			\hline
		\end{tabular}
	\end{center}
	Từ mẫu số liệu không ghép nhóm trên, ghép các số liệu thành $6$ nhóm theo các nửa khoảng có độ dài bằng nhau. Nhóm chiếm tỉ lệ cao nhất là
	\choice
	{$[7{,}0 ; 7{,}5)$}
	{$[7{,}5 ; 8{,}0)$}
	{\True $[8{,}0 ; 8{,}5)$}
	{$[8{,}5 ; 9{,}0)$}
	\loigiai{
		Khoảng biến thiên là $9{,}4-6{,}6=2{,}8$\\
		Ta chia thành các nhóm sau: $[6{,}5 ; 7{,}0),[7{,}0 ; 7{,}5),[7{,}5 ; 8{,}0),[8{,}0 ; 8{,}5),[8{,}5 ; 9{,}0),[9{,}0 ; 9{,}5)$.\\
		Đếm số giá trị của mỗi nhóm, ta có bảng ghép nhóm sau:
		\begin{center}
			\begin{tabular}{|c|c|}
				\hline
				Chiều cao $(\mathrm{m})$ & Số cây \\
				\hline
				$[6{,}5 ; 7{,}0)$ & $2$ \\
				\hline
				$[7{,}0 ; 7{,}5)$ & $4$ \\
				\hline
				$[7{,}5 ; 8{,}0)$ & $9$ \\
				\hline
				$[8{,}0 ; 8{,}5)$ & $11$ \\
				\hline
				$[8{,}5 ; 9{,}0)$ & $7$ \\
				\hline
				$[9{,}0 ; 9{,}5)$ & $2$ \\
				\hline
			\end{tabular}
		\end{center}
		Từ bảng số liệu ta thấy nhóm chiếm tỉ lệ cao nhất là $[8{,}0 ; 8{,}5)$.}
\end{ex}				
\begin{ex}%[1D5N1-2]%[Dự án D đợt 4 - Nguyễn Hoàng Anh]%[Đề ôn tập Chương V. THỐNG KÊ - Khối 11 - Đề số 2]
	Một cuộc khảo sát đã tiến hành xác định tuổi (theo năm) của 120 chiếc ô tô. Kết quả điều tra được cho trong bảng số liệu ghép nhóm sau:
	\vspace{-.2cm}
	\begin{center}
		\begin{tabular}{|c|c|c|c|c|c|c|c}
			\hline
			Thời gian (năm) &[0; 4)&[4; 8)&[8; 12)&[12; 16)&[16; 20)&[20; 24)\\
			\hline
			Số xe &13&28&45&22&8&4\\
			\hline
		\end{tabular}
	\end{center}
	Nhóm chứa mốt của mẫu số liệu là
	\choice
	{$[4; 8)$}
	{\True $[8; 12)$}
	{$[12; 16)$}
	{$[20; 24)$}
	\loigiai{
Từ bảng số liệu ghép nhóm, ta có nhóm $[8; 12)$ là nhóm có tần số lớn nhất. Do đó nhóm chứa mốt của mẫu số liệu là nhóm $[8; 12)$.
}
\end{ex}
\begin{ex}%[1D5N1-1]%[Dự án D đợt 4 - Nguyễn Hoàng Anh]%[Đề ôn tập Chương V. THỐNG KÊ - Khối 11 - Đề số 2]
	Giả sử mẫu số liệu được cho dưới dạng bảng tần số ghép nhóm.
	\begin{center}
		{\tabcolsep = 5mm
			\begin{tabular}{|c|c|c|c|c|}
				\hline
				Nhóm & Nhóm $1$ & Nhóm $2$  & \ldots & Nhóm $k$\\
				\hline
				Giá trị đại diện & $c_1$ & $c_2$ & \ldots & $c_k$\\
				\hline 
				Tần số &  $n_1$ & $n_2$ & \ldots & $n_k$\\
				\hline
			\end{tabular}
		}
	\end{center}
	Đặt $n=n_1+n_2+ \cdots + n_k$.\\
	Số trung bình của mẫu số liệu ghép nhóm, kí hiệu $\overline{x}$, được tính theo công thức nào?
	\choice
	{\True $\overline{x}=\dfrac{n_1 c_1+n_2 c_2 +\cdots + n_k c_k}{n}$}
	{$\overline{x}=\dfrac{n_1 c_1+n_2 c_2+\cdots + n_k c_k}{2n}$}
	{$\overline{x}=\dfrac{n_1^2 c_1+n_2^2 c_2+\cdots + n_k^2 c_k}{n}$}
	{$\overline{x}=\dfrac{n_1 c_1+n_2 c_2 +\cdots + n_k c_k}{\sqrt n}$}
	\loigiai{
		Giả sử mẫu số liệu được cho dưới dạng bảng tần số ghép nhóm
		\begin{center}
			{\tabcolsep = 5mm
				\begin{tabular}{|c|c|c|c|c|}
					\hline
					Nhóm & Nhóm $1$ & Nhóm $2$  & \ldots & Nhóm $k$\\
					\hline
					Giá trị đại diện & $c_1$ & $c_2$ & \ldots & $c_k$\\
					\hline 
					Tần số &  $n_1$ & $n_2$ & \ldots & $n_k$\\
					\hline
				\end{tabular}
			}
		\end{center}
		Số trung bình của mẫu số liệu ghép nhóm, kí hiệu $\overline{x}$, được tính như sau
		$$\overline{x}=\dfrac{n_1 c_1+n_2 c_2 +\cdots + n_k c_k}{n}$$
		trong đó $n=n_1+n_2+ \cdots + n_k$.
	}
\end{ex}
\begin{ex}%[1D5N1-3]%[Dự án D đợt 4 - Nguyễn Hoàng Anh]%[Đề ôn tập Chương V. THỐNG KÊ - Khối 11 - Đề số 2]
	Kết quả khảo sát cân nặng của $25$ quả cam ở lô hàng $A$ được cho ở bảng sau
	\begin{center}
		{\tabcolsep = 2mm
			\begin{tabular}{|c|c|c|c|c|c|}
				\hline
				Cân nặng (g) & $[150; 155)$ & $[155; 160)$  & $[160; 165)$ & $[165; 170)$ & $[170; 175)$ \\
				\hline 
				Số quả cam ở lô hàng A &  $1$ & $3$ & $7$ & $10$ & $4$\\
				\hline
			\end{tabular}
		}
	\end{center}
	Cân nặng trung bình của mỗi quả cam ở lô hàng $A$ xấp xỉ bằng
	\choice
	{$162{,}7$}
	{\True $161{,}7$}
	{$163{,}7$}
	{$164{,7}$}
	\loigiai{
		Ta có bảng thống kê số lượng cam theo giá trị đại diện
		\begin{center}
			{\tabcolsep = 2mm
				\begin{tabular}{|c|c|c|c|c|c|}
					\hline
					Cân nặng (g) & $[150; 155)$ & $[155; 160)$  & $[160; 165)$ & $[165; 170)$ & $[170; 175)$ \\
					\hline
					Cân nặng đại diện & $152{,}5$ & $157{,}5$ & $162{,}5$ & $167{,}5$ & $172{,}5$ \\
					\hline 
					Số quả cam ở lô hàng A &  $2$ & $6$ & $12$ & $4$ & $1$ \\
					\hline
				\end{tabular}
			}
		\end{center}
		Cân nặng trung bình của mỗi quả cam ở lô hàng A là
		$$\overline{x}= \dfrac{2 \cdot 152{,}5 + 6 \cdot 157{,}5 + 12 \cdot 162{,}5 + 4 \cdot 167{,}5 + 172{,}5}{25} \approx 161{,}7 \text{ (g)}.$$
	}
\end{ex}
\begin{ex}%[1D5H2-3]%[Dự án D đợt 4 - Nguyễn Hoàng Anh]%[Đề ôn tập Chương V. THỐNG KÊ - Khối 11 - Đề số 2]
	Doanh thu bán hàng trong $20$ ngày được lựa chọn ngẫu nhiên của một của hàng được ghi lại ở bảng sau (đơn vị: triệu đồng).
	\begin{center}
		\begin{tabular}{|>{\centering\arraybackslash}m{2cm}
			|>{\centering\arraybackslash}m{1.5cm}
			|>{\centering\arraybackslash}m{1.5cm}
			|>{\centering\arraybackslash}m{1.5cm}
			|>{\centering\arraybackslash}m{1.5cm}
			|>{\centering\arraybackslash}m{1.5cm}|}
				\hline
				Doanh thu & $[5; 7)$ & $[7; 9)$  & $[9; 11)$ & $[11; 13)$ & $[13; 15)$ \\
				\hline 
				Số ngày &  $2$ & $7$ & $7$ & $3$ & $1$\\
				\hline
			\end{tabular}
	\end{center}
	Mốt của mẫu số liệu trên thuộc khoảng nào trong các khoảng dưới đây?
	\choice
	{$[7; 9)$}
	{\True $[9; 11)$}
	{$[11; 13)$}
	{$[13; 15)$}
	\loigiai
	{
		Có $2$ nhóm chứa mốt của mẫu số liệu trên đó là $[7; 9)$ và $[9; 11)$. \\
		Do đó, ta xét
		\begin{itemize}
			\item Xét nhóm $[7; 9)$	ta có
			$M_0 = 7 + \dfrac{7-2}{(7-2)+(7-7)} (9-7) =9$.
			\item Xét nhóm $[9; 11)$ ta có
			$M_0 = 9 + \dfrac{7-7}{(7-7)+(7-3)} (11-9) =9$.
		\end{itemize}
		Vậy mốt của mẫu số liệu trên là $9$ thuộc nhóm $[9; 11)$.
	}
\end{ex}
\begin{ex}%[1D5N2-2]%[Dự án D đợt 4 - Nguyễn Hoàng Anh]%[Đề ôn tập Chương V. THỐNG KÊ - Khối 11 - Đề số 2]
	Khảo sát thời gian chạy bộ trong một ngày của một số học sinh khối $11$ thu được mẫu số liệu ghép nhóm sau
	\begin{center}
		\begin{tabular}{|>{\centering\arraybackslash}m{3cm}
			|>{\centering\arraybackslash}m{1.5cm}
			|>{\centering\arraybackslash}m{1.5cm}
			|>{\centering\arraybackslash}m{1.5cm}
			|>{\centering\arraybackslash}m{1.5cm}
			|>{\centering\arraybackslash}m{1.5cm}|}
				\hline
				Thời gian (phút) & $[0; 20)$ & $[20; 40)$  & $[40; 60)$ & $[60; 80)$ & $[80; 100)$ \\
				\hline 
				Số học sinh &  $5$ & $9$ & $12$ & $10$ & $6$\\
				\hline
			\end{tabular}
		\end{center}
	Nhóm chứa trung vị của mẫu số liệu trên là
	\choice
	{\True $[40; 60)$}
	{$[20; 40)$}
	{$[60; 80)$}
	{$[80; 100)$}
	\loigiai{
		Ta có $n=42$ nên trung vị của mẫu số liệu trên là $M_e = Q_2 = \dfrac{x_{21}+ x_{22}}{2}$.\\
		Mà $x_{21}, x_{22} \in [40; 60)$.\\
		Vậy nhóm chứa trung vị của mẫu số liệu trên là nhóm $[40;60)$.
	}
\end{ex}
\begin{ex}%[1D5N2-3]%[Dự án D đợt 4 - Nguyễn Hoàng Anh]%[Đề ôn tập Chương V. THỐNG KÊ - Khối 11 - Đề số 2]
	Khảo sát thời gian chạy bộ trong một ngày của một số học sinh khối $11$ thu được mẫu số liệu ghép nhóm sau
	\begin{center}
			\begin{tabular}{|>{\centering\arraybackslash}m{3cm}
				|>{\centering\arraybackslash}m{1.5cm}
				|>{\centering\arraybackslash}m{1.5cm}
				|>{\centering\arraybackslash}m{1.5cm}
				|>{\centering\arraybackslash}m{1.5cm}
				|>{\centering\arraybackslash}m{1.5cm}|}
				\hline
				Thời gian (phút) & $[0; 20)$ & $[20; 40)$  & $[40; 60)$ & $[60; 80)$ & $[80; 100)$ \\
				\hline 
				Số học sinh &  $5$ & $9$ & $12$ & $10$ & $6$\\
				\hline
			\end{tabular}
		\end{center}
	Nhóm chứa tứ phân vị thứ nhất của mẫu số liệu trên là
	\choice
	{$[40; 60)$}
	{\True $[20; 40)$}
	{$[60; 80)$}
	{$[80; 100)$}
	\loigiai{
		Ta có $n=42$ nên tứ phân vị thứ nhất của mẫu số liệu trên là $Q_1 = x_{11}$.\\
		Mà $x_{11} \in [20; 40)$.\\
		Vậy nhóm chứa trung vị của mẫu số liệu trên là nhóm $[20; 40)$.
	}
\end{ex}
\begin{ex}%[1D5N2-2]%[Dự án D đợt 4 - Nguyễn Hoàng Anh]%[Đề ôn tập Chương V. THỐNG KÊ - Khối 11 - Đề số 2]
	Doanh thu bán hàng trong $20$ ngày được lựa chọn ngẫu nhiên của một của hàng được ghi lại ở bảng sau (đơn vị: triệu đồng)
	\begin{center}
		\begin{tabular}{|>{\centering\arraybackslash}m{2.5cm}
			|>{\centering\arraybackslash}m{1.5cm}
			|>{\centering\arraybackslash}m{1.5cm}
			|>{\centering\arraybackslash}m{1.5cm}
			|>{\centering\arraybackslash}m{1.5cm}
			|>{\centering\arraybackslash}m{1.5cm}|}
				\hline
				Doanh thu & $[5; 7)$ & $[7; 9)$  & $[9; 11)$ & $[11; 13)$ & $[13; 15)$ \\
				\hline 
				Số ngày &  $2$ & $7$ & $7$ & $3$ & $1$\\
				\hline
			\end{tabular}
	\end{center}
	Trung vị của mẫu số liệu trên thuộc khoảng nào trong các khoảng dưới đây?
	\choice
	{$[7; 9)$}
	{\True $[9; 11)$}
	{$[11; 13)$}
	{$[13; 15)$}
	\loigiai
	{
		Gọi $x_1$, $x_2$, \ldots, $x_{20}$ là doanh thu bán hàng trong $20$ ngày xếp theo thứ tự không giảm.\\ 
		Khi đó
		$x_1, x_2 \in [5; 7)$;
		$x_3, \cdots, x_9 \in [7; 9)$;
		$x_{10}, \cdots, x_{16} \in [9; 11)$;
		$x_{17}, x_{18}, x_{19} \in [11; 13)$;
		$x_{20} \in [13; 15)$.\\
		Do đó, trung vị của mẫu số liệu thuộc nhóm $[9; 11)$.
	}
\end{ex}
\begin{ex}%[1D5N2-3]%[Dự án D đợt 4 - Nguyễn Hoàng Anh]%[Đề ôn tập Chương V. THỐNG KÊ - Khối 11 - Đề số 2]
	Khảo sát thời gian chạy bộ trong một ngày của một số học sinh khối $11$ thu được mẫu số liệu ghép nhóm sau
	\begin{center}
			\begin{tabular}{|>{\centering\arraybackslash}m{3cm}
				|>{\centering\arraybackslash}m{1.5cm}
				|>{\centering\arraybackslash}m{1.5cm}
				|>{\centering\arraybackslash}m{1.5cm}
				|>{\centering\arraybackslash}m{1.5cm}
				|>{\centering\arraybackslash}m{1.5cm}|}
				\hline
				Thời gian (phút) & $[0; 20)$ & $[20; 40)$  & $[40; 60)$ & $[60; 80)$ & $[80; 100)$ \\
				\hline 
				Số học sinh &  $5$ & $9$ & $12$ & $10$ & $6$\\
				\hline
			\end{tabular}
	\end{center}
	Nhóm chứa tứ phân vị thứ ba của mẫu số liệu trên là
	\choice
	{$[40; 60)$}
	{$[20; 40)$}
	{\True $[60; 80)$}
	{$[80; 100)$}
	\loigiai{
		Ta có $n=42$ nên tứ phân vị thứ ba của mẫu số liệu trên là $Q_3 = x_{33}$.\\
		Mà $x_{33} \in [60; 80)$.\\
		Vậy nhóm chứa tứ phân vị thứ ba của mẫu số liệu trên là nhóm $[60; 80)$.
	}
\end{ex}
\Closesolutionfile{ans}
%\begin{center}
%	\textbf{ĐÁP ÁN}
%	\inputansbox{10}{ans/ans}	
%\end{center}
\begin{center}
	\textbf{PHẦN 2 - CÂU TRẮC NGHIỆM ĐÚNG SAI}
\end{center}
\Opensolutionfile{ans}[ans/answer-DS-ONTAPCHUONGV-DE2]
\begin{ex}%[1D5H1-3]%[Dự án D đợt 4 - Nguyễn Hoàng Anh]%[Đề ôn tập Chương V. THỐNG KÊ - Khối 11 - Đề số 2]
	Kết quả khảo sát cân nặng của $25$ quả cam ở mỗi lô hàng $A$, $B$ được cho ở bảng sau:
	\begin{center}
		\begin{tabular}{|c|c|c|c|c|c|}
			\hline
			Cân nặng (gam) & $[150 ; 155)$ & $[155 ; 160)$ & $[160 ; 165)$ & $[165 ; 170)$ & $[170 ; 175)$ \\
			\hline
			Số quả cam ở lô hàng $A$ & $2$ & $6$ & $12$ & $4$ & $1$ \\
			\hline
			Số quả cam ở lô hàng $B$ & $1$ & $3$ & $7$ & $10$ & $4$ \\
			\hline
		\end{tabular}
	\end{center}
	Khi đó
	\choiceTF
	{\True Giá trị đại diện nhóm $[150 ; 155)$ bằng $152{,}5$}
	{Cân nặng trung bình của mỗi quả cam ở lô $A$ là $163{,}7$\,(gam)}
	{Cân nặng trung bình của mỗi quả cam ở lô $B$ là $162{,}1$\,(gam)}
	{\True Theo số trung bình thì cam ở lô hàng $B$ nặng hơn cam ở lô hàng $A$}
	\loigiai{
		Bảng thống kê số lượng cam theo giá trị đại diện:
		\begin{center}
			\begin{tabular}{|c|c|c|c|c|c|}
				\hline
				Cân nặng đại diện (gam) & $152{,}5$ & $157{,}5$ & $162{,}5$ & $167{,}5$ & $172{,}5$ \\
				\hline
				Số quả cam ở lô hàng $A$ & $2$ & $6$ & $12$ & $4$ & $1$ \\
				\hline
				Số quả cam ở lô hàng $B$ & $1$ & $3$ & $7$ & $10$ & $4$ \\
				\hline
			\end{tabular}
		\end{center}
		\begin{itemchoice}
			\itemch Giá trị đại diện nhóm $[150 ; 155)$ bằng $152{,}5$.
			\itemch 
			Cân nặng trung bình của mỗi quả cam ở lô A là $163{,}7$\,(gam).\\
			Cân nặng trung bình của mỗi quả cam ở lô $A$ là
			\begin{center}
				$\overline{x}_{A}=\dfrac{152{,}5 \cdot 2+157{,}5 \cdot 6+162{,}5 \cdot 12+167{,}5 \cdot 4+172{,}5 \cdot 1}{25}=161{,}7$\,(gam).
			\end{center}
			\itemch 
			Cân nặng trung bình của mỗi quả cam ở lô B là $162{,}1$ (gam).\\
			Cân nặng trung bình của mỗi quả cam ở lô $B$ là
			\begin{center}
			$\overline{x}_{B}=\dfrac{152{,}5 \cdot 1+157{,}5 \cdot 3+162{,}5 \cdot 7+167{,}5 \cdot 10+172{,}5 \cdot 4}{25}=165{,}1$\,(gam).
			\end{center}
			\itemch 
			Theo số trung bình thì cam ở lô hàng $B$ nặng hơn cam ở lô hàng $A$.\\
			Ta thấy $\overline{x}_{A}<\overline{x}_{B}$.\\
			Vậy nếu so sánh theo số trung bình thì cam ở lô hàng $B$ nặng hơn cam ở lô hàng $A$.
		\end{itemchoice}
	}
\end{ex}	
\begin{ex}%[1D5N1-4]%[Dự án D đợt 4 - Nguyễn Hoàng Anh]%[Đề ôn tập Chương V. THỐNG KÊ - Khối 11 - Đề số 2]
	Số lượng người đi xem một bộ phim mới theo độ tuổi trong một rạp chiếu phim (sau $1$ giờ đầu công chiếu) được ghi lại theo bảng phân phối ghép nhóm sau
	\begin{center}
		\begin{tabular}{|c|c|c|c|c|c|}
			\hline 
			Độ tuổi & $[10;20)$ & $[20;30)$ & $[30;40)$ & $[40;50)$ & $[50;60)$ \\ 
			\hline 
			Số người & $6$ & $12$ & $16$ & $7$ & $2$ \\ 
			\hline 
		\end{tabular} 
	\end{center}
	\choiceTF
	{\True Giá trị đại diện nhóm $[50;60)$ là $\dfrac{50+60}{2}=55$}
	{\True Độ tuổi được dự báo là ít xem phim đó nhất là thuộc nhóm $[50;60)$}
	{\True Nhóm chứa mốt là nửa khoảng $[30;40)$}
	{Độ tuổi được dự báo là thích xem phim đó nhiều nhất là $31$ tuổi}
	\loigiai{
		\begin{itemchoice}
			\itemch Giá trị đại diện nhóm $[50;60)$ là $55$.
			\itemch Độ tuổi được dự báo là ít xem phim đó nhất là thuộc nhóm $[50;60)$ với $2$ người.
			\itemch Nhóm chứa mốt là nửa khoảng $[30;40)$.
			\itemch Khi đó $u_m = 30$, $n_m=16$, $n_{m-1}=12$, $n_{m+1}=7$, $u_{m+1}-u_m =40-30=10$.\\
			Ta có mốt là $M_0 = 30 + \dfrac{16-12}{(16-12)+(16-7)} \cdot 10 = \dfrac{430}{13} \approx 33{,}08$.\\
			Vậy độ tuổi được dự báo là thích xem phim nhiều nhất là $33$ tuổi.
		\end{itemchoice}
	}
\end{ex}
		
\Closesolutionfile{ans}
%\inputansbox[2]{2}{ans/answer.tex}
\begin{center}
\textbf{PHẦN 3 - CÂU TRẮC NGHIỆM TRẢ LỜI NGẮN}
\end{center}
\setcounter{ex}{0}
\Opensolutionfile{ans}[ans-KQ-ONTAPCHUONGV-DE2]
\begin{ex}%%[1D5V2-3]%[Dự án D đợt 4 - Nguyễn Hoàng Anh]%[Đề ôn tập Chương V. THỐNG KÊ - Khối 11 - Đề số 2]
	Một công ty mua bán ô tô đã tính điểm chỉ số hài lòng của khách hàng (thang điểm 100) cho 110 đại lý bán hàng của mình và thu được kết quả sau:
	\vspace{-.2cm}
	\begin{center}
			\begin{tabular}{|>{\centering\arraybackslash}m{2.5cm}
				|>{\centering\arraybackslash}m{2cm}
				|>{\centering\arraybackslash}m{2cm}
				|>{\centering\arraybackslash}m{2cm}
				|>{\centering\arraybackslash}m{2cm}
				|>{\centering\arraybackslash}m{2cm}|}
			\hline
			Điểm &$[0; 20)$&$[20; 40)$&$[40; 60)$&$[60; 80)$&$[80; 100]$\\
			\hline
			Số đại lý &4 &16 &36 &42 &12\\
			\hline
		\end{tabular}
	\end{center}
	\vspace{-.2cm}
	\hspace{1cm} Hãy xác định điểm ngưỡng để đưa ra danh sách $25 \%$ số đại lý có chỉ số hài lòng của khách hàng tốt nhất (viết dưới dạng số thập phân và làm tròn đến chữ số hàng phần chục).
\shortans{$72,6$}
	\loigiai{
		Điểm ngưỡng để đưa ra danh sách $25 \%$ số đại lý có chỉ số hài lòng của khách hàng tốt nhất chính là tứ phân vị thứ ba của mẫu số liệu ghép nhóm trên.\\
	Gọi $x_1$, $x_2$, $\ldots$, $x_{110}$ là điểm chỉ số hài lòng của khách hàng cho $110$ đại lý và giả sử dãy này được xếp theo thứ tự không giảm.\\
	Tứ phân vị thứ ba là $Q_3=x_{83} \in [60; 80)$.\\
	Do đó $Q_3=60+\dfrac{\dfrac{3}{4}\cdot 110-(4+16+36)}{42}\cdot (80-60)=72{,}6$.
}
\end{ex}
\begin{ex}%[1D5H1-4]%[Dự án D đợt 4 - Nguyễn Hoàng Anh]%[Đề ôn tập Chương V. THỐNG KÊ - Khối 11 - Đề số 2]
	Số khách hàng nam mua bảo hiểm ở từng độ tuổi được thống kê như sau
	\begin{center}
		{\tabcolsep = 2mm
			\begin{tabular}{|c|c|c|c|c|c|}
				\hline
				\textbf{Độ tuổi} & $[20; 30)$ & $[30; 40)$  & $[40; 50)$ & $[50; 60)$ & $[60; 70)$ \\
				\hline \textbf{Số khách hàng nam} &  $4$ & $6$ & $10$ & $7$ & $3$\\
				\hline
			\end{tabular}
		}
	\end{center}
	Hãy sử dụng dữ liệu ở trên để tư vấn cho đại lí bảo hiểm xác định khách hàng nam ở tuổi nào hay mua bảo hiểm nhất (làm tròn kết quả đến hàng đơn vị).
	\shortans{$46$}
	\loigiai{
		Nhóm chứa mốt của mẫu số liệu khách hàng nam là $[40; 50)$.\\
		Do đó, $u_m=40$; $u_{m+1}=50$ $\Rightarrow u_{m+1} - u_m = 50-40=10$.\\
		$n_{m-1}=6$; $n_m=10$; $n_{m+1}=7$.\\
		$\mathrm{M}_0 = 40 + \dfrac{10-6}{(10-6)+(10-7)} (50-40) =45{,}7$.\\
		Dựa vào kết quả trên ta có thể dự đoán được khách hàng nam $46$ tuổi có nhu cầu mua bảo hiểm cao nhất.
	}
\end{ex}
\begin{ex}%[1D5H2-3]%[Dự án D đợt 4 - Nguyễn Hoàng Anh]%[Đề ôn tập Chương V. THỐNG KÊ - Khối 11 - Đề số 2]
	Bảng sau biểu diễn mẫu số liệu ghép nhóm về độ tuổi của cư dân trong một khu phố.
	\begin{center}
		\renewcommand{\arraystretch}{1.5}
		\begin{tabular}{|c|c|c|c|c|c|c|}
			\hline
			Nhóm & $[20;30)$ & $[30;40)$ & $[40;50)$ & $[50;60)$ & $[60;70)$ & $[70;80)$ \\
			\hline
			Tần số & $25$ & $22$ & $20$ & $15$ & $14$ & $4$ \\
			\hline
		\end{tabular}
	\end{center}
	Gọi $Q_1$, $Q_3$ lần lượt là tứ phân vị thứ nhất và thứ ba của mẫu số liệu trên. Tính $\Delta_Q=Q_3-Q_1$ (làm tròn tới hàng phần chục).
	\shortans[oly]{25{,}3}
	\loigiai{
		Cỡ mẫu $n=25+22+20+15+14+4=100$.\\		
		Gọi $x_1, x_2, \dots, x_{100}$ là độ tuổi của cư dân trong khu phố và giả sữ dãy này đã được xếp theo thứ tự không giảm.\\
		Ta có 
		\begin{itemize}
		\item $x_1, \dots, x_{25} \in[20;30)$; 
		\item $x_{26}, \dots, x_{47} \in[30;40)$; 
		\item $x_{48}, \dots, x_{67} \in[40;50)$; 
		\item $x_{68}, \dots, x_{82} \in[50;60)$; 
		\item $x_{83}, \dots, x_{96} \in[60;70)$; 
		\item $x_{97}, \dots, x_{100} \in[70;80)$.
		\end{itemize}		
		Tứ phân vị thứ nhất $Q_1$ là $\dfrac{x_{25}+x_{26}}{2}$. Suy ra $Q_1$ thuộc nhóm $[30;40)$.\\		
		Do đó $Q_1=30+\dfrac{\dfrac{100}{4}-25}{22} \cdot 10=30$.\\		
		Tứ phân vị thứ ba $Q_3$ là $\dfrac{x_{75}+x_{76}}{2}$. Suy ra $Q_3$ thuộc nhóm $[50;60)$.\\
		Do đó $Q_3=50+\dfrac{\dfrac{3 \cdot 100}{4}-67}{15} \cdot 10 \approx 55{,}3$.\\
		Vậy $\Delta_Q=Q_3-Q_1=55{,}3-30=25{,}3$.
	}
\end{ex}
\begin{ex}%[1D5H1-3]%[Dự án D đợt 4 - Nguyễn Hoàng Anh]%[Đề ôn tập Chương V. THỐNG KÊ - Khối 11 - Đề số 2]
	Lớp 12A có 50 học sinh, mỗi học sinh làm bài thi trắc nghiệm môn Toán có 50 câu. Biết số câu trả lời đúng bài thi trắc nghiệm môn Toán được cho bởi bảng sau
	\begin{center}
		\begin{tabular}{|l|c|c|c|c|c|}
			\hline
			Số câu đúng & $[14;21)$ & $[21;28)$ & $[28;35)$ & $[35;42)$ & $[42;49)$ \\
			\hline
			Số học sinh & $4$ & $x$ & $25$ & $y$ & $7$ \\
			\hline
		\end{tabular}
	\end{center}
	Tính $3x-2y$ biết số trung bình của mẫu số liệu bằng $32,06$.
	\shortans{$12$}
	\loigiai{
		Tổng số học sinh là $x+y+36=50$, suy ra $x+y=14$ \,(1).\\
		Lại có $\dfrac{ 4\cdot 17,5+x \cdot 24,5+25 \cdot 31,5+y\cdot 38,5+7\cdot 45,5}{50}=32,06$.\\
		Từ đó suy ra $24,5x+38,5y=427$ \,(2).\\
		Từ (1) và (2) ta được $x=8$ và $y=6$ suy ra $3x-2y=12$.
	}
\end{ex}
\Closesolutionfile{ans}

\begin{center}
	\textbf{PHẦN 4 - TỰ LUẬN}
\end{center}
\begin{bt}%[1D5V1-2]%[Dự án D đợt 4 - Nguyễn Hoàng Anh]%[Đề ôn tập Chương V. THỐNG KÊ - Khối 11 - Đề số 2]
	Một cửa hàng bán $3$ loại hoa quả nhập khẩu: Lê, Dưa vàng và Bưởi với số liệu tính toán được cho bởi bảng (trong một quý) sau khi giảm giá mỗi loại lần lượt là $x$, $y$, $z$ trên $1$ kg 
	\begin{center}
		\begin{tabular}{|c|c|c|c|}
			\hline Loại quả&Lê&Dưa vàng&Bưởi\\
			\hline Giá bán (nghìn/kg)&$200-x$&$300-y$&$400-z$\\
			\hline Số lượng bán (kg)&$200+x$&$300+y$&$400+z$\\
			\hline 
		\end{tabular}
	\end{center}
	Biết rằng $x+y+z=90$ (nghìn). Tính giá trị $x$, $y$, $z$ để lợi nhuận bình quân của $1$ kg hoa quả đạt được cao nhất.
	\loigiai{
		Do khối lượng hoa quả bán được là $200+x+300+y+400+z=990$ là cố định, vì thế bình quân mỗi kg hoa quả có giá cao nhất khi tổng số tiền thu được là cao nhất.\\
		Tổng số tiền thu được là
		$$P=(200-x)(200+x)+(300-y)(300+y)+(400-z)(400+z)=290\,000-({x^2}+{y^2}+{z^2}).$$
		Ta có bất đẳng thức ${x^2}+{y^2}+{z^2}\ge\dfrac{1}{3}{(x+y+z)^2}=2700$, từ đó $P\le 287\,300$.\\
		Vậy $P$ lớn nhất khi dấu bằng xảy ra tức là $x=y=z=30$ (nghìn).
	}
\end{bt}
\begin{bt}%[1D5N1-3]%[Dự án D đợt 4 - Nguyễn Hoàng Anh]%[Đề ôn tập Chương V. THỐNG KÊ - Khối 11 - Đề số 2]
	Cân nặng của $28$ học sinh nam lớp $11$ được cho như sau
	\begin{center}
		\begin{tabular}{llllllllllllll}
			$55{,}4$ & $62{,}6$ & $54{,}2$ & $56{,}8$ & $58{,}8$ & $59{,}4$ & $60{,}7$ & $58$ & $59{,}5$ & $63{,}6$ & $61{,}8$ & $52{,}3$ & $63{,}4$ & $57{,}9$\\
			$49{,}7$ & $45{,}1$ & $56{,}2$ & $63{,}2$ & $46{,}1$ & $49{,}6$ & $59{,}1$ & $55{,}3$ & $55{,}8$ & $45{,}5$ & $46{,}8$ & $54$ & $49{,}2$ & $52{,}6$
		\end{tabular}
	\end{center}
	Tìm số trung bình của mẫu số liệu ghép nhóm trên.
	\loigiai
	{
		\begin{center}
			\begin{tabular}{|c|c|c|c|c|c|}
				\hline 
				Cân nặng &{$[45; 49)$} &{$[49; 53)$} &{$[53; 57)$} &{$[57; 61)$} &{$[61; 65)$} \\
				\hline 
				Giá trị đại diện & $47$ & $51$ & $55$ & $59$ & $63$ \\
				\hline 
				Số học sinh & $4$ & $5$ & $7$ & $7$ & $5$ \\
				\hline
			\end{tabular}
		\end{center}
		Cân nặng trung bình của học sinh trong lớp $11$ là\\
		$\overline{x}= \dfrac{47 \cdot 4 + 51 \cdot 5 + 55 \cdot 7 + 59 \cdot 7 + 63 \cdot 5 }{28} \approx 55{,}6$ (kg).
	}
\end{bt}
\begin{bt}%%[1D5H2-2]%[Dự án D đợt 4 - Nguyễn Hoàng Anh]%[Đề ôn tập Chương V. THỐNG KÊ - Khối 11 - Đề số 2]
	Trong một hội thao, thời gian chạy $200$ m của một nhóm các vận động viên được ghi lại ở bảng sau
	\begin{center}
		\begin{tabular}{|c|c|c|c|c|c|}
			\hline
			Thời gian (giây) &$[21;21{,}5)$& $[21{,}5;22)$ &$[22;22{,}5)$&$[22{,}5;23)$ &$[23;23{,}5)$\\
			\hline
			Số vận động viên & $5$ & $12$ & $32$ & $45$ & $30$ \\
			\hline
		\end{tabular}
	\end{center}
	Dựa vào bảng số liệu trên, ban tổ chức muốn chọn ra khoảng $50\%$ số vận động viên chạy nhanh nhất để tiếp tục thi vòng $2$. Ban tổ chức nên chọn các vận động viên có thời gian chạy không quá bao nhiêu giây (kết quả làm tròn đến hàng phần chục)?
	\loigiai{Trung vị của bảng số liệu trên là $M_\text{e}=u_m+\dfrac{\dfrac{n}{2}-C}{n_m}\left(u_{m+1}-u_m\right)$.\\
		$M_\text{e}=22{,}5+\dfrac{\dfrac{124}{2}-49}{45}\left(23-22{,}5\right)=\dfrac{1019}{45}\approx 22{,}64$ (giây).}
\end{bt}