\newpage
\def\thoigian{90}%--Thời gian
\de{Đề số 2}{Chương VII. Đạo hàm}

\begin{center}
	\textbf{PHẦN 1 - CÂU TRẮC NGHIỆM BỐN PHƯƠNG ÁN}
\end{center}
\Opensolutionfile{ans}[ans/ans-TN-ONTAPCHUONG-DE2]
%Câu 1
\begin{ex}%[1D7N1-1]%[Dự án đề kiểm tra Toán 11 GHK2 NH24-25- Huỳnh Đức Vũ]%[THPT PHẠM VĂN ĐỒNG-QUẢNG NGÃI]
	Cho hàm số $y=f(x)$ có đạo hàm tại điểm $x_0$. Chọn phương án đúng?
	\choice
	{\True  $f'\left(x_0\right)=\lim\limits_{x\to{x_0}}\dfrac{f(x)-f\left(x_0\right)}{x-x_0}$}
	{$f'\left(x_0\right)=\lim\limits_{x\to{x_0}}\dfrac{f(x)+f\left(x_0\right)}{x+x_0}$}
	{$f'\left(x_0\right)=\lim\limits_{x\to{x_0}}\dfrac{f(x)+f\left(x_0\right)}{x-x_0}$}
	{$f'\left(x_0\right)=\lim\limits_{x\to{x_0}}\dfrac{f\left(x_0\right)-f(x)}{x-x_0}$}
	\loigiai{Nếu hàm số $y=f(x)$ có đạo hàm tại điểm $x_0$ thì $f'\left(x_0\right)=\lim\limits_{x\to{x_0}}\dfrac{f(x)-f\left(x_0\right)}{x-x_0}$.}
\end{ex}
%Câu 2

\begin{ex}%[1D7N1-1]%[Dự án đề kiểm tra Toán khối 11 HKII NH24-25-Dot 11-Đắc Vũ]%[THPT Phan Đăng Lưu-TP HCM]
	Cho hàm số $y=f(x)$ có đạo hàm tại $x_0$ là $f'\left(x_0\right)$. Khẳng định nào sau đây {\bf sai}?
	\choice
	{$f'\left(x_0\right)=\lim \limits_{\Delta x \to 0} \dfrac{f(x_0+\Delta x)-f\left(x_0\right)}{\Delta x}$}
	{$f'\left(x_0\right)=\lim \limits_{h \to 0} \dfrac{f(x_0+h)-f\left(x_0\right)}{h}$}
	{$f'\left(x_0\right)=\lim \limits_{x \to x_0} \dfrac{f(x)-f\left(x_0\right)}{x-x_0}$}
	{\True $f'\left(x_0\right)=\lim \limits_{x \to x_0} \dfrac{f(x)+f\left(x_0\right)}{x-x_0}$}
	\loigiai{
		Đạo hàm của hàm số $y=f(x)$ tại $x_0$ là 
		\[f'\left(x_0\right)=\lim \limits_{x \to x_0} \dfrac{f(x)-f\left(x_0\right)}{x-x_0}.	\]}
\end{ex}
\begin{ex}%[1D7N1-1]%[Dự án đề kiểm tra Toán 11 HKII NH24-25- Nguyễn Tấn Tài]%[THPT NGÔ GIA TỰ - Daklak]
	Cho hàm số $y=f(x)$ có đạo hàm thỏa mãn $f'(3)=6$. Giá trị của biểu thức $\displaystyle \lim\limits_{x\to 3} \dfrac{f(x)-f(3)}{x-3}$ bằng
	\choice
	{\True $6$}
	{$\dfrac{1}{2}$}
	{$2$}
	{$\dfrac{1}{3}$}
	\loigiai{
		Theo định nghĩa đạo hàm tại điểm $x=3$, ta có
		$$f'(3)=\displaystyle \lim\limits_{x\to 3} \dfrac{f(x)-f(3)}{x-3}=6.$$
	}
\end{ex}

\begin{ex}%[1D7H2-3]%[Dự án đề kiểm tra Toán 11 HK2 NH24-25 - Thành Đức Trung]%[THPT Phan Bội Châu - Bình Thuận]
	Cho hàm số $y = f(x)$ có đạo hàm trên $\mathbb{R}$ và có đồ thị như hình vẽ bên dưới. Đường thẳng $d$ là tiếp tuyến của đồ thị hàm số $y = f(x)$ tại điểm có hoành độ bằng $3$. Giá trị $f'(3)$ bằng
	\begin{center}
		\begin{tikzpicture}[scale=1.2, font=\footnotesize, >=stealth]  
			\draw[->] (-1,0)--(0,0) node[below left]{$O$}--(5.5,0) node[below]{$x$};
			\draw[->] (0,-1) --(0,3) node[right]{$y$};
			\draw [black,thick, domain=-0.4:4.2, samples=100] plot(\x,{0-0.25*(\x)^(3.0)+1.5*(\x)^(2.0)-2.25*(\x)+1.0}) node[below]{$y = f(x)$};
			\draw[fill = black] (0,0) circle (1pt);
			\draw[dashed] (3,0)node[below]{$3$} -- (3,1) -- (0,1)node[left]{$1$} (4,0) node[above right]{$4$} (1,0) node[below]{$1$};
			\draw[fill = black] (1,0) circle (1pt) (3,0) circle (1pt) (0,1) circle (1pt) (3,1) circle (1pt) (4,0) circle (1pt);
			\draw[thick] (1,1)--(4.5,1) node[right]{$d$};
		\end{tikzpicture}
	\end{center}
	\choice
	{$1$}
	{$3$}
	{\True $0$}
	{$4$}
	\loigiai
	{
		Vì $d$ là tiếp tuyến của đồ thị hàm số $y = f(x)$ tại điểm có hoành độ bằng $3$ nên $k = f'(3)$ là hệ số góc của $d$. \\
		Mà $d$ là đường thẳng song song với $Ox$ nên $k = 0$.
	}
\end{ex}
\begin{ex}%[1D7H2-8]%[Dự án đề kiểm tra Toán 11 HKII NH23-24- VU-Ngoc-Hao]%[THPT chuyên Lê Quý Đôn - Tỉnh Ninh Thuận]
	Một vật đang đứng yên tại thời điểm ban đầu $t=0$ và bắt đầu di chuyển theo hướng thẳng có phương trình chuyển động $s(t)=t^3-6t^2+24 t$, trong đó $t$ là thời gian tính bằng giây và $s$ là quãng đường tính bằng mét. Tính quãng đường vật đi được kể từ thời điểm ban đầu cho đến thời điểm vận tốc của vật đạt giá trị nhỏ nhất.
	\choice
	{$30$ m}
	{$25$ m}
	{$20$ m}
	{\True $32$ m}
	\loigiai{
		Ta có $v(t)=s'(t)=3t^2-12t+24=3(t^2-4t+8)=3(t-2)^2+12\geq 12$.\\
		Vật có vận tốc đạt giá trị nhỏ nhất tại thời điểm $t=2$.\\
		Khi đó vật di chuyển được $s(2)=2^3-6\cdot 2^2+24\cdot 2=32$ m.
	}
\end{ex}
\begin{ex}%[1D7H2-8]%[Dự án đề kiểm tra Toán 11 HKII NH23-24- VU-Ngoc-Hao]%[THPT chuyên Lê Quý Đôn - Tỉnh Ninh Thuận]
	Chuyển động của một vật gắn trên con lắc lò xo (khi bỏ qua ma sát và sức cản không khí) được cho bởi phương trình $x(t)=4 \cos \left(2 \pi t+\dfrac{\pi}{3}\right)$, trong đó $x$ tính bằng centimet và thời gian $t$ tính bằng giây. Tìm gia tốc tức thời của vật tại thời điểm $t=5 s$ (làm tròn kết quả đến hàng đơn vị).
	\choice
	{$-40$ cm/s$^2$}
	{\True $-79$ cm/s$^2$}
	{$40$ cm/s$^2$}
	{$79$ cm/s$^2$}
	\loigiai{
		Ta có $v(t)=x'(t)=(-4)\cdot 2\pi\sin \left(2 \pi t+\dfrac{\pi}{3}\right)=-8\pi \sin \left(2 \pi t+\dfrac{\pi}{3}\right)$.\\
		Và $a(t)=v'(t)=-16\pi^2\cos \left(2 \pi t+\dfrac{\pi}{3}\right)$.	\\
		Gia tốc tức thời của vật tại thời điểm $t=5s$ là $a(5)=-16\pi^2\cos \left(2 \pi \cdot 5+\dfrac{\pi}{3}\right)\approx -79$ cm/s$^2$.
	}
\end{ex}
%++++++++================

\begin{ex}%[1D7N2-1]%[Dự án đề kiểm tra học kỳ 2 năm học 2024-2025 - Đợt 10 - Ngô Tất Thành]%[THPT Chuyên Lê Quý Đôn - Ninh Thuận]
	Hàm số $y=\sqrt{x}$ có đạo hàm trên $(0;+\infty)$ là 
	\choice
	{$y'=\dfrac{1}{\sqrt{x}}$}
	{\True $y'=\dfrac{1}{2\sqrt{x}}$}
	{$y'=2\sqrt{x}$}
	{$y'=\dfrac{-1}{\sqrt{x}}$}
	\loigiai{
		Ta có $y=\sqrt{x}\Rightarrow y'=\dfrac{1}{2\sqrt{x}}$.
	}
\end{ex}
\begin{ex}%[1D7N2-2]%[Dự án đề kiểm tra Toán 11 HKII NH23-24-Dot-16-VU-Ngoc-Hao]%[THPT chuyên Lê Quý Đôn - Tỉnh Ninh Thuận]
	Đạo hàm của hàm số $y=x^2\cdot\ln x$ trên $(0;+\infty)$ là
	\choice
	{$y'=3x\ln x$}
	{\True $y'=x+2x\ln x$}
	{$y'=2$}
	{$y'=-x+2x\ln x$}
	\loigiai{
		Ta có $y'=\left(x^{2}\right)'\cdot \ln x+x^{2}\cdot (\ln x)'=2x\cdot \ln x+x^{2}\cdot \dfrac{1}{x}=2x\cdot \ln x+x$.
	}
\end{ex}

\begin{ex}%[HKII-2024-2025, THPT Lương Ngọc Quyến, Thái Nguyên]%[Trần Văn Hùng]%[1D7N3-1]
	Cho hàm số $y=x^5-3x^4+x+1$ với $x\in \mathbb{R}$. Đạo hàm $y''$ của hàm số là
	\choice
	{$y''=5x^4-12x^3$}
	{\True $y''=20x^3-36x^2$}
	{$y''=5x^3-12x^2+1$}
	{$y''=20x^2-36x^3$}
	\loigiai{
	Ta có $y'=5x^4-12x^3+1\Rightarrow y''=20x^3-36x^2$.
	}
\end{ex}

\begin{ex}%[1D7H2-1]%[Dự án đề kiểm tra Toán 11 HKII NH24-25- Đoàn Minh Tâm]%[THPT Mạc Đĩnh Chi - TP Hồ Chí Minh]%Câu 8
	Đạo hàm của hàm số $f(x)=\mathrm{e}^x\cdot\cos x$ là
	\choice
	{\True $f'(x)=\mathrm{e}^x\cdot(\cos x-\sin x)$}
	{$f'(x)=\mathrm{e}^x\cdot(\cos x+\sin x)$}
	{$f'(x)=\mathrm{e}^x-\sin x$}
	{$f'(x)=\mathrm{e}^x+\sin x$}
	\loigiai{Ta có $f'(x)=\left(\mathrm{e}^x\cdot\cos x\right)'=\mathrm{e}^x\cdot\cos x-\mathrm{e}^x\cdot\sin x=\mathrm{e}^x\cdot(\cos x-\sin x)$.}
\end{ex}

\begin{ex}%[1D7N2-3]%[Dự án đề kiểm tra Toán 11 HKII NH23-24-Dot-16-VU-Ngoc-Hao]%[THPT chuyên Lê Quý Đôn - Tỉnh Ninh Thuận]
	Cho hàm số $y=f(x)=x^3-3x^2+4x-1$ có đồ thị $(C)$. Tiếp tuyến của đồ thị $(C)$ tại điểm có hoành độ bằng $2$ có phương trình là
	\choice
	{$y=4x-11$}
	{$y=4x-8$}
	{$y=4x-4$}
	{\True $y=4x-5$}
	\loigiai{
		Ta có $y'=f'(x)=3x^2-6x+4$. \\
		Tiếp tuyến của đồ thị $(C)$ tại điểm có hoành độ bằng $2$ có hệ số góc $k=f'(2)=4$; tung độ tiếp điểm $y_{0}=f(2)=3$.\\
		Phương trình tiếp tuyến có dạng $y=4(x-2)+3$ hay $y=4x-5$.	
	}
\end{ex}


\begin{ex}%[1D7V2-1]%[Dự án đề kiểm tra Toán 11 HKII NH23-24- VU-Ngoc-Hao]%[THPT chuyên Lê Quý Đôn - Tỉnh Ninh Thuận]
	Cho hàm số $f(x)=(1-2 x) \cdot \mathrm{e}^{-x^2+1}$. Số nghiệm nguyên thuộc $[-2\,024 ; 2\,024]$ của phương trình $f'(x) \geq 0$ là
	\choice
	{$0$}
	{$2\,025$}
	{\True $4\,048$}
	{$4\,049$}
	\loigiai{
		Ta có $f'(x)=(-2)\cdot \mathrm{e}^{-x^2+1}+(1-2x)\cdot (-2x)\cdot \mathrm{e}^{-x^2+1}=(-2)\cdot \mathrm{e}^{-x^2+1}\cdot \left(-2x^2+x+1\right)$.\\
		$f'(x) \geq 0 \Leftrightarrow -2x^2+x+1 \leq 0 \Leftrightarrow x\leq -\dfrac{1}{2} $ hoặc $x \geq 1 $  (vì $(-2)\cdot \mathrm{e}^{-x^2+1} <0, \forall x\in \mathbb{R}$).\\
		Kết hợp với điều kiện nghiệm thuộc $[-2\,024 ; 2\,024]$, ta được $x\in \left[-2\,024;-\dfrac{1}{2}\right]\cup [1;2\,024]$.\\
		Các nghiệm nguyên gồm $\{-2\,024;-2\,023;...;-2;-1;1;2;...;2\,024\}$. \\
		Suy ra có $4\,048$ nghiệm nguyên thỏa mãn yêu cầu bài toán. 
	}
\end{ex}

\Closesolutionfile{ans}
%\begin{center}
%	\textbf{ĐÁP ÁN}
%	\inputansbox{10}{ans/ans}	
%\end{center}

\begin{center}
	\textbf{PHẦN 2 - CÂU TRẮC NGHIỆM ĐÚNG SAI}
\end{center}
\setcounter{ex}{0}
\Opensolutionfile{ans}[ans/answer-DS-ONTAPCHUONG-DE1]
\begin{ex}%[1D7H2-3]%[Lớp 11 - Học kì II - THPT XUYÊN MỘC - BÀ RỊA VŨNG TÀU]%[Võ Thị Thùy Trang]
	Cho hàm số $f(x)=x^2+2x$ có đồ thị $(C)$ và điểm $M\in(C)$ có hoành độ bằng $2$.
	\choiceTF
	{\True $f'(x)=2x+2$}
	{Giá trị $f'(1)=\lim\limits_{x \to 1} \dfrac{f(x)+f(1)}{x-1}$}
	{\True Hệ số góc của tiếp tuyến của $(C)$ tại điểm $M$ là $k=f'(2)$}
	{Tiếp tuyến của $(C)$ tại điểm $M$ có phương trình là $y=6x-20$}
	\loigiai{
		\begin{itemchoice}
			\itemch  Ta có $f'(x)=(x^2+2x)'=2x+2$.
			\itemch  Ta có $f'(1)=\lim\limits_{x \to 1} \dfrac{f(x)-f(1)}{x-1}$.
			\itemch  Hệ số góc của tiếp tuyến của $(C)$ tại điểm $M$ là $k=f'(2)$.
			\itemch  Điểm $M$ có hoành độ $x=2$ thì tung động $y=8$. Ta có $M(2;8)\in (C)$, $f'(2)=6$.
			Tiếp tuyến của $(C)$ tại điểm $M$ có phương trình là $y=6(x-2)+8=6x-4$.
		\end{itemchoice}
	}
\end{ex}


\begin{ex}%[1D7V2-4]%[Dự án đề kiểm tra Toán khối 11 HKII NH24-25-Dot 12- Sy Truong]%[THPT Chuyen Le Hong Phong-TP Ho Chi Minh]
	Cho đồ thị hàm số $(C)\colon y=f(x)=\dfrac{x^3}{3}+3 x^2-2$. Gọi $\Delta$ là tiếp tuyến của $(C)$ tại $M\left(x_0, y_0\right)$, $\Delta$ song song với đường thẳng $d: y=-9 x$.
	\choiceTF
	{\True $f'(x)=x^2+6x$}
	{Hệ số góc của đường thẳng $\Delta$ là $ 9 $}
	{\True $x_0=-3$}
	{Phương trình đường thẳng $\Delta\colon y=-9 x-12$}
	\loigiai{
		\begin{itemchoice}
			\itemch Ta có $f'(x)=x^2+6x $.
			\itemch Vì $ \Delta \parallel d $ nên $ k_{\Delta}=k_d=-9 $.
			\itemch Vì $\Delta$ là tiếp tuyến của $(C)$ tại $M\left(x_0, y_0\right)$ và song song với $ d $ nên ta có
			\begin{eqnarray*}
				&&f'\left(x_0\right)=-9\\
				&\Leftrightarrow&x_0^2+6x_0=-9\\
				&\Leftrightarrow&x_0^2+6x_0+9=0\\
				&\Leftrightarrow&x_0=-3.
			\end{eqnarray*}.
			\itemch Với $ x_0=-3 $, ta tính được $ y_0=16 $.\\
			Vậy phương trình đường thẳng $ \Delta $ là
			$$y=-9(x+3)+16=-9x-11.$$
		\end{itemchoice}
	}
\end{ex}

\Closesolutionfile{ans}
%\inputansbox[2]{2}{ans/answer.tex}

\begin{center}
	\textbf{PHẦN 3 - CÂU TRẮC NGHIỆM TRẢ LỜI NGẮN}
\end{center}
\setcounter{ex}{0}
\Opensolutionfile{ans}[ans-KQ-ONTAPCHUONG-DE1]
\begin{ex}%[1D7H2-2]%[Dự án đề kiểm tra Toán 11 GHK2 NH24-25- Huỳnh Đức Vũ]%[THPT PHẠM VĂN ĐỒNG-QUẢNG NGÃI]
	Cho hàm số $ y=f(x)=-2x^3+x$ có đồ thị $(C)$.Tính hệ số góc của tiếp tuyến của đồ thị $(C)$ tại điểm có hoành độ bằng $1$.
	\shortans[oly]{-5}
	\loigiai{
	Ta có $f'(x)=\left(-2 x^3+x\right)'=-6 x^2+1$ nên hệ số góc của tiếp tuyến của $(C)$ tại điểm có hoành độ bằng $1$ là \\
	$f'(1)=-6\cdot1^2+1=-5$.}
\end{ex}
\begin{ex}%[1D7H2-8]%[Dự án đề kiểm tra Toán 11 GHK2 NH24-25- Huỳnh Đức Vũ]%[THPT PHẠM VĂN ĐỒNG-QUẢNG NGÃI]
	Một chất điểm chuyển động có phương trình $S(t)=t^3-3t^2-9t+2$, trong đó $t$ được tính bằng giây và $S$ được tính bằng mét. Gia tốc tại thời điểm vận tốc bị triệt tiêu là bao nhiêu? (đơn vị m/s$^2$).
	\shortans[oly]{12}
	\loigiai{
		Ta có $S(t)=t^3-3t^2-9t+2$ nên vận tốc tức thời và gia tốc tức thời tại thời điểm $ t$ lần lượt là
		$$ v(t)=S'(t)=3t^2-6t-9 \text{ và }a(t)=S''(t)=6t-6.$$
		Khi vận tốc bị triệt tiêu tức $v(t)=0\Leftrightarrow 3t^2-6t-9=0\Leftrightarrow \hoac{&t=3 \\&t=-1}\Rightarrow t=3\,\,(t>0)$.\\
		Khi đó gia tốc tại thời điểm vận tốc bị triệt tiêu là $ a(3)=6\cdot 3-6=12$ (m/s$^2$).}
\end{ex}
\begin{ex}%[1D7N2-8]%[Dự án đề kiểm tra Toán HKII NH24-25 - Đợt 12 - Hieu Phan]%[THPT Phan Bội Châu - Bình Thuận]
	Dân số của một thành phố A được tính bởi hàm số $P(t)=\dfrac{500t}{t^2+9}$ trong đó $P$ được tính theo đơn vị nghìn người, $t$ là thời gian được tính bằng năm. Nếu xem $P'(t)$ là tốc độ tăng dân số tại thời điểm $t$ thì tốc độ tăng dân số tại thời điểm $t=1$ là bao nhiêu nghìn người?
	\shortans[oly]{40}
	\loigiai{
			Ta có $P(t)=\dfrac{500t}{t^2+9}$, suy ra tốc độ tăng dân số tại thời điểm $t$ được xác định bởi công thức
			$$P'(t)=\dfrac{-500t^2+4\,500}{\left(t^2+9\right)^2}.$$
			Do đó $P'(1)=\dfrac{-500\cdot 1^2+4\,500}{\left(1^2+9\right)^2}=\dfrac{4\,000}{100}=40$ nghìn người.}
\end{ex}


\begin{ex}%[1D7H1-4]%[Dự án đề kiểm tra học kỳ 2 năm học 2024-2025 - Đợt 10 - Võ Thị Thùy Trang]%[THPT Chuyên Lê Quý Đôn - Ninh Thuận]
	Một chất điểm chuyển động theo phương trình $s(t)=\dfrac{1}{3}t^3+t^2-18t+4$, trong đó $t>0$ tính bằng giây, $s(t)$ tính bằng mét. Tính vận tốc (đơn vị: mét/giây) của chất điểm tại thời điểm vật đi được quãng đường $4$ mét.
	\shortans[oly]{30}
	\loigiai{
		\begin{itemize}
			\item Thời gian chất điểm đi được quãng đường $4$ mét là 
			\begin{eqnarray*}
				\dfrac{1}{3}t^3+t^2-18t+4=4&\Leftrightarrow&
				\dfrac{1}{3}t^3+t^2-18t=0\\
				&\Leftrightarrow&\hoac{&t=6&& \text{(nhận)}\\&t=-9&& \text{(loại)}\\&t=0&& \text{(loại).}}
			\end{eqnarray*}
			\item Ta có $v(t)=s'(t)=t^2+2t-18$.\\
			Vận tốc của chất điểm tại thời điểm vật đi được quãng đường $4$ mét là
			\[v(6)=s'(6)=6^2+2\cdot 6-18=30~~\text{(m/s).}\]
		\end{itemize}
	}
\end{ex}


\Closesolutionfile{ans}

\begin{center}
	\textbf{PHẦN 4 - TỰ LUẬN}
\end{center}
\setcounter{ex}{0}

\begin{ex}%[1D7V2-8]%[Dự án đề kiểm tra Toán 11 GHK2 NH24-25- Huỳnh Đức Vũ]%[THPT PHẠM VĂN ĐỒNG-QUẢNG NGÃI]	
	Một viên đạn được bắn lên cao theo phương thẳng đứng có phương trình chuyển động $ h(t)=3+196t-4{,}9t^2$, trong đó $t>0$,\, $t$ là thời gian chuyển động và được tính bằng giây; $h$ là độ cao so với mặt đất và được tính bằng mét. Tại thời điểm viên đạn đạt vận tốc tức thời bằng $98$ mét/giây thì viên đạn ở độ cao so với mặt đất bằng bao nhiêu mét?
	\loigiai{
		Ta có vận tốc tức thời của viên đạn tại thời điểm $t$ là $v(t)=h'(t)$.\\
		Viên đạn đạt vận tốc tức thời bằng $98$ mét/giây nên ta có phương trình
		$$98=196-9{,}8t\Leftrightarrow t=10\,\text{(s)}.$$
		Độ cao cần tìm là $h= h\left(10\right)=3+196\cdot 10-4{,}9\cdot (10)^2=1\,473$ (m).}
\end{ex}
\begin{ex}%[1D7H2-8]%[Dự án đề kiểm tra Toán 11 CKII NH23-24-Đợt 16-Lê Hữu Kiệt]%[THPT Chuyên Lê Quý Đôn- Ninh Thuận]
	\immini[thm]
	{Một vật chuyển động trong $4$ giờ với vận tốc $v$ phụ thuộc vào thời gian $t$ và có đồ thị vận tốc như hình bên. Trong khoảng thời gian $4$ giờ kể từ khi bắt đầu chuyển động, đồ thị đó là một phần của đường parabol có đỉnh $I\left( \dfrac{5}{2};\,-\dfrac{9}{4} \right)$ và có trục đối xứng song song trục tung. Tính gia tốc của vật lúc $t=3$ giờ.}
	{\begin{tikzpicture}[font=\footnotesize, >=stealth, x=0.6cm, y=0.6cm]
			\draw[->] (-0.5,0)--(6,0)node[above]{$t$ (giờ)};
			\draw[->] (0,-3)--(0,5)node[left]{$v$ (km/giờ)};
			\draw (0,0) node[below left]{$O$};
			\draw[dashed] (5/2,0) |- (0,-9/4);
			\begin{scope}
				\clip (-0.5,-3) rectangle (5.5,4.8);
				\draw[smooth] plot[domain=0:4] (\x,{(\x)^2-5*(\x)+4});
			\end{scope}
			\foreach \x in {(5/2,0), (0,4), (0,-9/4), (5/2,-9/4)}{\fill[black] \x circle (1pt);}
			\draw (0,4)node[left]{$4$} (0,-9/4)node[left]{$-\dfrac{9}{4}$} (5/2,0)node[above]{$\dfrac{5}{2}$} (5/2,-9/4)node[below]{$I$};
	\end{tikzpicture}}
	\loigiai{
		Gọi $v(t)=at^2+bt+c$ ($a> 0$) là hàm số vận tốc cần tìm (đơn vị: km/giờ). Ta có
		\begin{itemize}
			\item Đồ thị hàm số đi qua điểm $(0;\,4)$, suy ra $c=4$;
			\item Hoành độ đỉnh $I$ là $-\dfrac{b}{2a}=\dfrac{5}{2} \Leftrightarrow b=-5a$; \quad$(1)$
			\item Tung độ đỉnh $I$ là $-\dfrac{b^2-4ac}{4a}=-\dfrac{9}{4} \Leftrightarrow b^2-16a=9a \Leftrightarrow b^2-25a=0$. \quad$(2)$
		\end{itemize}
		Thay $(1)$ vào $(2)$ ta được
		\allowdisplaybreaks
		\begin{eqnarray*}
			&& (-5a)^2-25a=0 \\
			&\Leftrightarrow& 25a^2-25a=0 \\
			&\Leftrightarrow& a=1 \quad\text{hoặc} \quad a=0.
		\end{eqnarray*}
		Vì $a>0$ nên ta nhận $a=1$. Khi đó $b=-5\cdot1=-5$.\\
		Suy ra $v(t)=t^2-5t+4$.\\
		Gọi $a(t)$ là gia tốc của vật (đơn vị: km/giờ$^2$), ta có $a(t)=v'(t)=2t-5$.\\
		Khi đó, gia tốc của vật lúc $t=3$ giờ là $a(3)=1$ (km/giờ$^2$).
	}
\end{ex} 

\begin{ex}%[1D7V2-4]%[Dự án đề kiểm tra Toán HKII NH24-25 - Đợt 12 - Hieu Phan]%[THPT Phan Bội Châu - Bình Thuận]
	Cho hàm số $ y=\dfrac{2}{x}$ có đồ thị là đường cong $(C)$.
	\begin{enumerate}
		\item Tính đạo hàm của hàm số $y=\dfrac{2}{x}$.
		\item Biết tiếp tuyến của đường cong $(C)$ song song với đường thẳng $2x+y+4=0$ và có phương trình $ y=ax+b$. Tính giá trị của $a-b$.
	\end{enumerate}
	\loigiai{
		\begin{enumerate}
			\item $y'=-\dfrac{2}{x^2}$.
			\item $2x+y+4=0\Rightarrow y=-2x-4$.\\
			Tiếp tuyến song song với đường thẳng $y=-2x-4$ nên
			\begin{eqnarray*}
			&&y'\left(x_0\right)=-2 \\
			&\Leftrightarrow&-\dfrac{2}{x_0^2}=-2\\
			&\Leftrightarrow& x_0^2=1 \\
			&\Leftrightarrow& x_0=-1\quad \text{hoặc}\quad x_0=1.
			\end{eqnarray*}
		\end{enumerate}
		\begin{itemize}
			\item $x_0=-1\Rightarrow y_0=\dfrac{2}{-1}=-2$.\\
			Phương trình tiếp tuyến cần tìm là $ y=-2(x+1)-2\Leftrightarrow y=-2x-4$ (loại vì trùng với đường thẳng đã cho).
			\item $x_0=1\Rightarrow y_0=\dfrac{2}{1}=2$.\\
			Phương trình tiếp tuyến cần tìm là $ y=-2(x-1)+2\Leftrightarrow y=-2x+4$ (nhận).\\
		\end{itemize} 
		Vậy $a-b=-2-(-4)=-2+4=2$.}
\end{ex}

%====================











































