\section{Dãy số}
\subsection{Lý thuyết}
\subsubsection{Định nghĩa dãy số}
\begin{dn} \
	\begin{itemize}
		\item Mỗi hàm số $u$ xác định trên tập các số nguyên dương $\mathbb{N}^*$ được gọi là một dãy số vô hạn (gọi tắt là \textbf{dãy số}), kí hiệu là $u = u(n)$.\\
		Ta thường viết $u_n$ thay cho $u(n)$ và kí hiệu dãy số $u = u(n)$ bởi $\left(u_n\right)$.
		\item Do đó, dãy số $\left(u_n\right)$ được viết dưới dạng khai triển $u_1$, $u_2$, $u_3$, \ldots, $u_n$, \ldots. \\ Số $u_1$ gọi là \textbf{số hạng đầu}, $u_n$ là \textbf{số hạng thứ $n$} và gọi là \textbf{số hạng tổng quát} của dãy số.
	\end{itemize}
\end{dn} 
\textcolor{red}{\faExclamationTriangle\ \bf Chú ý.} Nếu $\forall n \in \mathbb{N}^*$, $u_n=c$ thì $\left(u_n\right)$ được gọi là dãy số không đổi. 
\begin{dn} \
	\begin{itemize}
		\item Mỗi hàm số $u$ xác định trên tập $M = \{1 ; 2 ; 3 ; \ldots ; m\}$ với $m \in \mathbb{N}^*$ được gọi là một \textbf{dãy số hữu hạn}.
		\item Dạng khai triển của dãy số hữu hạn là $u_1$, $u_2$, \ldots, $u_m$. \\Số $u_1$ gọi là \textbf{số hạng đầu}, số $u_m$ gọi là \textbf{số hạng cuối}.
	\end{itemize}
\end{dn}
\subsubsection{Các cách cho một dãy số} 
\begin{dn}
	Một dãy số có thể cho bằng:
	\begin{itemize}
		\item Liệt kê các số hạng (chỉ dùng cho các dãy hữu hạn và có ít số hạng);
		\item Công thức của số hạng tổng quát;
		\item Phương pháp mô tả;
		\item Phương pháp truy hồi.
	\end{itemize}
\end{dn}
\subsubsection{Dãy số tăng, dãy số giảm} 
\begin{dn} \
	\begin{itemize}
		\item Dãy số $\left(u_n\right)$ được gọi là \textbf{dãy số tăng} nếu ta có $u_{n+1} > u_n$ với mọi $n \in \mathbb{N}^*$.
		
		\item Dãy số $\left(u_n\right)$ được gọi là \textbf{dãy số giảm} nếu ta có $u_{n+1} < u_n$ với mọi $n \in \mathbb{N}^*$.
	\end{itemize}
	
\end{dn}
\subsubsection{Dãy số bị chặn} 

\begin{dn} \
	\begin{itemize}
		\item Dãy số $\left(u_n\right)$ được gọi là \textbf{bị chặn trên} nếu tồn tại một số $M$ sao cho $u_n \leq M$ với mọi $n \in \mathbb{N}^*$.
		
		\item Dãy số $\left(u_n\right)$ được gọi là \textbf{bị chặn dưới} nếu tồn tại một số $m$ sao cho $u_n \geq m$ với mọi $n \in \mathbb{N}^*$.
		
		\item Dãy số $\left(u_n\right)$ được gọi là \textbf{bị chặn} nếu nó vừa bị chặn trên vừa bị chặn dưới, tức là tồn tại các số $m$, $M$ sao cho $m \leq u_n \leq M$ với mọi $n \in \mathbb{N}^*$.
	\end{itemize}
	
\end{dn}
\textcolor{red}{\faExclamationTriangle\ \bf Chú ý.}
\begin{itemize}
	\item Nếu dãy số $\left(u_n\right)$ \textbf{tăng} thì nó bị chặn dưới bởi $u_1$.
	
	\item Nếu dãy số $\left(u_n\right)$ \textbf{giảm} thì nó bị chặn trên bởi $u_1$.
\end{itemize}
\subsection{Bài tập}
\begin{dang}{Tìm các số hạng của dãy số cho bởi công thức tổng quát}
	Thay $n$ vào công thức để tìm được số hạng thứ $n$. 
\end{dang}
\begin{vd}%[1D2N1-3]
	Cho dãy số $\left(u_n\right)$, biết $u_n=\dfrac{n}{3^n-1}$. Tìm ba số hạng đầu tiên của dãy số.
	\dapso{$\dfrac{1}{2};\dfrac{1}{4};\dfrac{3}{26}$.}
	\loigiai{
		Ta có $u_1=\dfrac{1}{2};u_2=\dfrac{2}{3^2-1}=\dfrac{2}{8}=\dfrac{1}{4};u_3=\dfrac{3}{3^3-1}=\dfrac{3}{26}.$
	}
\end{vd}

\begin{vd}%[1D2H1-3]
	Cho dãy số $(u_n)$ được xác định bởi $u_n=\dfrac{n^2+3n+7}{n+1}$.
	\begin{enumerate}
		\item 	Viết năm số hạng đầu của dãy.  
		\item 	Dãy số có bao nhiêu số hạng nhận giá trị nguyên?
		
	\end{enumerate}
	
	\loigiai{
		
		\begin{enumerate}
			\item Ta có năm số hạng đầu của dãy
			$u_1=\dfrac{1^2+3 \cdot 1+7}{1+1}=\dfrac{11}{2}$; $u_2=\dfrac{17}{3}$; $u_3=\dfrac{25}{4}$; $u_4=7$; $u_5=\dfrac{47}{6}$.
			\item Ta có $u_n=n+2+\dfrac{5}{n+1}$, do đó $u_n$ nguyên khi và chỉ khi $ \dfrac{5}{n+1}$ nguyên hay $ n+1 $ là ước của $5$. \\
			Điều đó xảy ra khi $ n+1=5\Leftrightarrow n=4 $. \\
			Vậy dãy số có duy nhất một số hạng nguyên là $u_4=7 $.
	\end{enumerate}	}
\end{vd}
\begin{vd}%[1D2H1-3]
	Cho dãy số $(u_n)$ có số hạng tổng quát $u_n=\dfrac{2n+1}{n+2}$. 
	\begin{enumerate}
		\item 	Viết năm số hạng đầu của dãy số. 
		\item Tìm số hạng thứ $ 100 $ và $ 200 $. 
		\item Số $ \dfrac{167}{84}$ là số hạng thứ mấy?
		\item Dãy số có bao nhiêu số hạng là số nguyên?
	\end{enumerate}
	\loigiai{
		\begin{enumerate}
			\item Năm số hạng đầu của dãy là $ u_1=1$, $ u_2=\dfrac{5}{4}$, $u_3=\dfrac{7}{5}$, $u_4=\dfrac{3}{2}$, $u_5=\dfrac{11}{7}$. 
			\item Số hạng thứ 100 là $ u_{100}=\dfrac{2 \cdot 100+1}{100+2}=\dfrac{67}{34}$.\\
			Số hạng thứ 200  là $ u_{200}=\dfrac{2\cdot 200+1}{200+2}=\dfrac{401}{202}$.
			\item Giả sử $ u_n=\dfrac{167}{84}\Rightarrow \dfrac{2n+1}{n+2}=\dfrac{167}{84}\Leftrightarrow 84(2n+1)=167(n+2)$
			$ \Leftrightarrow n=250 $. \\
			Vậy $ \dfrac{167}{84}$ là số hạng thứ $ 250 $ của dãy số $(u_n)$. 
			\item Ta có $ u_n=\dfrac{2(n+2)-3}{n+2}=2-\dfrac{3}{n+2}$
			$ \Rightarrow u_n\in \mathbb{Z}\Leftrightarrow \dfrac{3}{n+2}\in \mathbb{Z}\Leftrightarrow 3\;\vdots\; (n+2)\Leftrightarrow n=1 $.\\
			Vậy dãy số có duy nhất một số hạng là số nguyên.
			
		\end{enumerate}
		
	}
\end{vd}

\begin{dang}{Tìm các số hạng của dãy số cho bởi công thức truy hồi}
	Dựa trên hệ thức truy hồi, ta tìm lần lượt từng số hạng từ $u_1$ đến $u_n$.
\end{dang}
\begin{vd}%[1D2N1-3]
	Cho dãy số $\left(u_n\right)$, biết $\heva{
		& u_1=-1 \\
		& u_{n+1}=u_n+3 \\
	}$ với $n\ge 1$. Tìm ba số hạng đầu tiên của dãy số.
	\loigiai{Ta có $u_1=-1$; $u_2=u_1+3=2$; $u_3=u_2+3=5$.}
\end{vd}

\begin{vd}%[1D2H1-3]
	Cho dãy số $(u_n)$ được xác định bởi$\left\{\begin{aligned}&u_1=1 \\&u_n=2u_{n-1}+1, (n\geq 2).\end{aligned}\right.$
	\begin{enumerate}
		\item Hãy viết dạng khai triển của dãy số trên. \dapso{$1;3;7;15; \ldots; 2^n-1; \ldots$.}
		\item Tính $u_8$. \dapso{$u_8=255$.}
	\end{enumerate}
	\loigiai
	{\begin{enumerate}
			\item Dạng khai triển của dãy số đã cho là $1;3;7;15; \ldots$.
			\item 	Ta có $u_1=1$, $u_2=3$, $u_3=7$, $u_4=15$, $u_5=31$, $u_6=63$, $u_7=127$, $u_8=255$.
		\end{enumerate}
	}
\end{vd}

\begin{vd}%[1D2H1-3]
	Cho dãy số $(u_n)$ được xác định bởi
	$\left\{\begin{aligned}&u_1=1, u_2=1 \\&u_n=u_{n-1}+u_{n-2}, (n\geq 3)\end{aligned}\right. \text{ (dãy số Phi-bô-na-xi)}.$
	\begin{enumerate}
		\item Hãy viết dạng khai triển của dãy số trên.
		\item Tính $u_7$.
	\end{enumerate}
	\loigiai
	{
		\begin{enumerate}
			\item Dạng khai triển của dãy số đã cho là $1;1;2;3;5;8;13; \ldots$.
			\item 
			Ta có $u_1=1$, $u_2=1$, $u_3=2$, $u_4=3$, $u_5=5$, $u_6=8$, $u_7=13$.
		\end{enumerate}
		
	}
\end{vd}


\begin{dang}{Xét sự tăng giảm của dãy số}
	
	\begin{enumerate}[\iconMT]
		\item \indam{Phương pháp 1:} Xét dấu của hiệu số $u_{n+1}-u_n$.
		\begin{itemize}
			\item Nếu $u_{n+1}-u_n>0, \forall n \in \mathbb{N}^*$ thì $(u_n)$ là dãy số tăng.
			\item Nếu $u_{n+1}-u_n<0, \forall n \in \mathbb{N}^*$ thì $(u_n)$ là dãy số giảm.
		\end{itemize}
		\item \indam{Phương pháp 2:} Nếu $u_n>0, \forall n\in \mathbb{N}^*$ thì ta có thể so sánh thương $\dfrac{u_{n+1}}{u_n}$ với $1$.
		\begin{itemize}
			\item Nếu $\dfrac{u_{n+1}}{u_n}>1$ thì $(u_n)$ là dãy số tăng.
			\item Nếu $\dfrac{u_{n+1}}{u_n}<1$ thì $(u_n)$ là dãy số giảm.
		\end{itemize}
		Nếu $u_n<0, \forall n\in \mathbb{N}^*$ thì ta có thể so sánh thương $\dfrac{u_{n+1}}{u_n}$ với $1$.
		\begin{itemize}
			\item Nếu $\dfrac{u_{n+1}}{u_n}<1$ thì $(u_n)$ là dãy số tăng.
			\item Nếu $\dfrac{u_{n+1}}{u_n}>1$ thì $(u_n)$ là dãy số giảm.
		\end{itemize}
	\end{enumerate}
\end{dang}

\begin{vd}%[1D2V1-4]
	Xét tính tăng giảm của dãy số sau $(u_n)$ với
	\begin{multicols}{3}
		\begin{enumerate}
			\item $u_n=n^2$.	
			\item $u_n=\dfrac{2n+1}{n+1}$.	
			\item $u_n=\dfrac{4^n-1}{4^n+5}$.	
			\item $u_n=\dfrac{n}{3^n}$.	
			\item $u_n=(-1)^n$.	
			\item $u_n=\sqrt{n}-\sqrt{n+3}$.	
		\end{enumerate}
	\end{multicols}
	\loigiai{
		\begin{enumerate}
			\item 
			Xét $u_{n+1}-u_n=(n+1)^2-n^2=2n+1 >0, \forall n \in \mathbb{N}^{*}$ nên $(u_n)$ là dãy tăng.
			\item
			Ta có $u_n=\dfrac{2n+1}{n+1}=2-\dfrac{1}{n+1}$.\\
			$u_{n+1}-u_n=\left(2-\dfrac{1}{n+1+1}\right)-\left(2-\dfrac{1}{n+1}\right)=\dfrac{1}{n+1}-\dfrac{1}{n+2}>0, \forall n \in \mathbb{N}^*$.\\
			Vậy dãy số $(u_n)$ là dãy số tăng.
			\item 
			Ta có $u_n=\dfrac{4^n-1}{4^n+5}=1-\dfrac{6}{4^n+5}$.\\
			$u_{n+1}-u_n=\left(1-\dfrac{6}{4^{n+1}+5}\right)-\left(1-\dfrac{6}{4^n+5}\right)=\dfrac{6}{4^n+5}-\dfrac{6}{4^{n+1}+5} >0, \forall n \in \mathbb{N}^*$.\\
			Vậy dãy số $(u_n)$ là dãy số tăng.
			\item 
			Ta có $u_n=\dfrac{n}{3^n}>0, \forall n \in \mathbb{N}^*$.\\
			Xét thương $\dfrac{u_{n+1}}{u_n}=\dfrac{n+1}{3^{n+1}}:\dfrac{n}{3^n}=\dfrac{n+1}{3.n}<1, \forall  n \in \mathbb{N}^*$.\\
			Vậy $(u_n)$ là dãy số giảm.
			\item
			Ta có\\ $u_1=(-1)^1=-1.\\
			u_2=(-1)^2=1.\\
			u_3=(-1)^3=-1.$\\
			Vậy $(u_n)$ là dãy không tăng không giảm.
			\item
			Ta có $u_n=\sqrt{n}-\sqrt{n+3}=\dfrac{-3}{\sqrt{n}+\sqrt{n+3}}$.\\
			Xét hiệu\\ 
			$\begin{aligned}
				u_{n+1}-u_n&=\dfrac{-3}{\sqrt{n+1}+\sqrt{n+4}}-\dfrac{-3}{\sqrt{n}+\sqrt{n+3}}\\
				&=\dfrac{3}{\sqrt{n}+\sqrt{n+3}}-\dfrac{3}{\sqrt{n+1}+\sqrt{n+4}}>0, \forall n\in \mathbb{N}^*.
			\end{aligned}$\\
			Vậy $(u_n)$ là dãy số tăng.
		\end{enumerate}
	}
\end{vd}
\begin{vd}%[1D2V1-4]
	Cho dãy số $(u_n)$ biết $u_n=\dfrac{an+2}{3n+1}$. Giá trị nguyên nhỏ nhất của $a$  để dãy số là dãy số tăng bằng bao nhiêu?
	\loigiai{
		Ta có $u_{n+1}=\dfrac{a(n+1)+2}{3(n+1)+1}=\dfrac{an+a+2}{3n+4}=\dfrac{an+2}{3n+4}+\dfrac{a}{3n+4}$.\\
		Khi đó
		\begin{eqnarray*}
			&u_{n+1}-u_n&=\dfrac{an+2}{3n+4}+\dfrac{a}{3n+4}-\dfrac{an+2}{3n+1} \\
			&&= (an+2)\left(\dfrac{1}{3n+4}-\dfrac{1}{3n+1}\right)+\dfrac{a}{3n+4} \\
			&&=\dfrac{-3(an+2)}{(3n+4)(3n+1)}+\dfrac{a}{3n+4} \\
			&&=\dfrac{-3(an+2)+a(3n+1)}{(3n+4)(3n+1)} \\
			&&=\dfrac{a-6}{(3n+4)(3n+1)}.
		\end{eqnarray*}
		Dãy số $(u_n)$ là dãy số tăng khi và chỉ khi
		\begin{eqnarray*}
			&&u_{n+1}-u_n>0 \\
			&\Leftrightarrow&\dfrac{a-6}{(3n+4)(3n+1)}>0 \\
			&\Leftrightarrow&a>6.
		\end{eqnarray*}
		Vậy giá trị nguyên nhỏ nhất của $a$ để dãy số $(u_n)$ là dãy số tăng bằng $7$.
	}
\end{vd}

\begin{vd}%[1D2H1-6]
	Anh Thanh vừa được tuyển dụng vào một công ty công nghệ, được cam kết lương năm đầu sẽ là $ 200 $ triệu đồng và lương mỗi năm tiếp theo sẽ được tăng thêm $ 25 $ triệu đồng. Gọi $s_n$ (triệu đồng) là lương vào năm thứ $n$ mà anh Thanh làm việc cho công ty đó. Khi đó 
	$$s_1=200, s_n=s_{n-1}+25\, \text{ với}\, n \geq 2.$$
	\begin{enumerate}
		\item Tính lương của anh Thanh vào năm thứ $ 5 $ làm việc cho công ty.
		\item Chứng minh $\left(s_n\right)$ là dãy số tăng. Giải thích ý nghĩa thực tế của kết quả này.
	\end{enumerate}
	\loigiai{
		\begin{enumerate}
			\item Ta có $ s_1=200; s_2=225; s_3=250;s_4=275;s_5=300 $.\\
			Vậy lương của anh Thanh vào năm thứ $ 5 $ là $ 300 $ triệu đồng.
			\item Ta có $ s_n-s_{n-1}=25>0$, $\forall n\ge 2 $.\\
			Do đó $\left(s_n\right)$ là dãy số tăng.\\
			Điều này cho thấy ý nghĩa thực tế là một người có nhiều năm làm việc thì nhận được lương nhiều hơn người mới vào làm việc.
		\end{enumerate}	
	}
\end{vd}

\begin{dang}{Xét tính bị chặn của dãy số}
	\begin{itemize}
		\item Để chứng minh dãy số $(u_n)$ bị chặn trên bởi $M$, ta chứng minh $u_n\leq M, \forall n\in \mathbb{N}^*.$
		\item Để chứng minh dãy số $(u_n)$ bị chặn dưới bởi $m$, ta chứng minh $u_n\geq m, \forall n\in \mathbb{N}^*.$
		\item Để chứng minh dãy số bị chặn ta chứng minh nó bị chặn trên và bị chặn dưới.
		\item Nếu dãy số $(u_n)$ tăng thì bị chặn dưới bởi $u_1$; dãy số $(u_n)$ giảm thì bị chặn trên bởi $u_1$.
	\end{itemize}
\end{dang}

\begin{vd}%[1D2V1-5]
	Xét tính bị chặn của các dãy số $(u_n)$ sau, với
	\begin{multicols}{3}
		\begin{enumerate}
			\item $u_n=\dfrac{1}{2^n}$.
			\item $u_n=2\sin{n^2}$
			\item $u_n=\dfrac{3n-1}{3n+1}$.
			\item $u_n=\dfrac{n^2+1}{2n^2-3}$.
			\item $u_{n}=\dfrac{1}{n(n+1)}$.
			\item $u_n=\dfrac{2n-1}{\sqrt{n^2+2}}$.
		\end{enumerate}
	\end{multicols}
	\loigiai
	{
		\begin{enumerate}
			\item $0<u_n=\dfrac{1}{2^n}\leq\dfrac{1}{2}$ với mọi $n\in\mathbb{N}^*$ nên dãy $(u_n)$ bị chặn.
			\item Ta có $-2 \leq u_n=2 \sin n^{2} \leq 2, \forall n \in \mathbb{N}^*$.\\
			Suy ra $(u_n)$ với $u_n=2 \sin n^2$ bị chặn dưới bởi $-2$ và bị chặn trên bởi $2$.
			\item Ta có $u_n=\dfrac{3n-1}{3n+1}=1-\dfrac{2}{3n+1}<1, \forall n \geq 1$. Suy ra dãy số $(u_n)$ bị chặn trên.\\
			Mặt khác, $\dfrac{3n-1}{3n+1}>0$, $\forall n\geq 1$ nên dãy số $(u_n)$ bị chặn dưới.\\
			Vậy dãy số $(u_n)$ bị chặn.
			\item Ta có 
			\[u_n=\dfrac{n^2+1}{2n^2-3}=\dfrac{1}{2}+\dfrac{5}{2(2n^2-3)}.\]
			Dễ thấy, với mọi $n \geq 1$ ta có, $-1 \le \dfrac{1}{2n^2-3} \le \dfrac{1}{5}$.\\ Do đó suy ra $-2 \le u_n \le 1, \forall n \geq 1$.\\
			Từ đó suy ra $(u_n)$ là một dãy số bị chặn.
			\item Ta có $u_n=\dfrac{1}{n(n+1)}>0, \forall n \geq 1$ nên dãy số bị chặn dưới.\\
			Lại có
			\[u_{n+1}-u_n=\dfrac{1}{(n+1)(n+2)}-\dfrac{1}{n(n+1)}=\dfrac{-2}{n(n+1)(n+2)}<0, \forall n \geq 1.\]
			Vậy dãy số $(u_n)$ là dãy số giảm nên $u_{n}\le u_1=\dfrac{1}{2}$. \\
			Suy ra dãy số $(u_n)$ bị chặn trên.\\
			Vậy dãy số $(u_n)$ bị chặn.
			\item Ta có $u_n=\dfrac{2n-1}{\sqrt{n^2+2}}>0, \forall n \geq 1$ nên dãy số bị chặn dưới.\\
			Lại có $\dfrac{2n-1}{\sqrt{n^2+2}} < \dfrac{2n-1}{n} \le \dfrac{2n}{n}=2$ nên dãy số bị chặn trên.\\
			Vậy dãy số $(u_n)$ bị chặn.
		\end{enumerate}	
	}
\end{vd}

\begin{vd}%[1D2H1-5]
	Chứng minh rằng dãy số $(u_n)$ xác định bởi $u_n=\dfrac{8n+3}{3n+5}$ là một dãy số bị chặn.	\dapso{$ 0<u_n< \dfrac{11}{3}$.}
	\loigiai{
		Ta có $u_n>0, \forall n\geq 1$. Suy ra dãy số bị chặn dưới.\\
		Mặt khác $u_n=\dfrac{8n+3}{3n+5}<\dfrac{8n+3}{3n}=\dfrac{8}{3}+\dfrac{1}{n}<\dfrac{8}{3}+1=\dfrac{11}{3}$.	Do đó dãy số bị chặn trên bởi $\dfrac{11}{3}$.\\
		Vậy dãy số đã cho bị chặn.
	}
\end{vd}

\begin{vd}%[1D2V1-5]
	Chứng minh rằng dãy số $(u_n)$ với $u_n=\dfrac{3n}{n^2+9}$ bị chặn trên bởi $\dfrac{1}{2}$.
	\loigiai{
		Với mọi $n\geq 1$, ta có $\dfrac{3n}{n^2+9}\leq \dfrac{1}{2} \Leftrightarrow n^2+9\leq 6n \Leftrightarrow (n-3)^2\leq 0$ (đúng).\\
		Vậy dãy số đã cho bị chặn trên bởi $\dfrac{1}{2}$.
	}
\end{vd}


\begin{dang}{Vận dụng thực tiễn}
	
\end{dang}

\begin{vd}%[1D2H1-6]
	Biết rằng năm $2016$, dân số của Việt Nam là
	$ 93{,}422$ triệu người. Tỷ lệ tăng dân số hàng năm của Việt Nam là $ 1{,}07\% $ thì số dân $P_n$ (triệu người) của Việt Nam sau $n$ năm, kể từ năm $2016$, được tính bằng công thức $P_n = 93{,}422(1 +0{,}0107 )^n$. Hỏi nếu tăng trưởng theo quy luật như vậy thì vào năm $2026$, dân số Việt Nam khoảng bao nhiêu triệu người?
	\loigiai
	{
		Từ năm $2016$ đến năm $2026$ là $10$ năm, nên $n=10$. Thay vào công thức đã cho ta được
		$ P_{10}\approx 103{,} 9 $ triệu người.
	}
\end{vd}
\begin{vd}%[1D2V1-6]
	Chị Hương vay trả góp một khoản tiền $ 100 $ triệu đồng và đồng ý trả dần $ 2 $ triệu đồng mỗi tháng với lãi suất $0,8 \%$ số tiền còn lại của mỗi tháng.\\
	Gọi $A_n(n \in \mathbb{N}$) là số tiền còn nợ (triệu đồng) của chị Hương sau $n$ tháng.
	\begin{enumerate}
		\item Tìm lần lượt $A_0$, $A_1$, $A_2$, $A_3$, $A_4$, $A_5$, $A_6$ để tính số tiền còn nợ của chị Hương sau $ 6 $ tháng.
		\item Dự đoán hệ thức truy hồi đối với dãy số $\left(A_n\right)$.
	\end{enumerate}
	\loigiai{
		\begin{enumerate}
			\item Đặt $A_0=100$ (triệu đồng). Khi đó
			\begin{itemize}
				\item  $A_1=A_0+\dfrac{0,8}{100}A_0-2=1{,}008A_0-2=98{,}8$ (triệu đồng).
				\item  $A_2=A_1+\dfrac{0,8}{100}A_1-2=1{,}008A_1-2=97{,}5904$ (triệu đồng).
				\item  $A_3=A_2+\dfrac{0,8}{100}A_2-2=1{,}008A_2-2=96{,}3711232$ (triệu đồng).
				\item  $A_4=A_3+\dfrac{0,8}{100}A_3-2=1{,}008A_3-2=95{,}1420932$ (triệu đồng).
				\item  $A_5=A_4+\dfrac{0,8}{100}A_4-2=1{,}008A_4-2=93{,}9032332$ (triệu đồng).
				\item  $A_6=A_5+\dfrac{0,8}{100}A_5-2=1{,}008A_5-2=92{,}6544632$ (triệu đồng).
			\end{itemize}
			\item Dự đoán hệ thức truy hồi đối với dãy số $\left(A_n\right)$ là $A_n=1{,}008A_{n-1}-2$, với $n \ge 1$.
		\end{enumerate}
	}
\end{vd}
\subsection{Bài tập}
\ind{PHẦN I.} \inden{Câu trắc nghiệm nhiều phương án lựa chọn. Mỗi câu hỏi học sinh chỉ chọn một phương án.}\\
\setcounter{ex}{0}
\Opensolutionfile{ans}[ans/1D2-Bai1-TN]
\begin{ex}%[1D2N1-3]
	Cho dãy số $(u_n)$ biết $u_n=\dfrac{3n-2}{n+1}$. Số hạng $u_8$ của dãy số $(u_n)$ là
	\choice
	{$\dfrac{8}{9}$}
	{$\dfrac{24}{9}$}
	{\True $\dfrac{22}{9}$}
	{$\dfrac{22}{8}$}
	\loigiai{Ta có $u_8=\dfrac{3\cdot 8-2}{8+1}=\dfrac{22}{9}$.}
\end{ex}
\begin{ex}%[1D2N1-4]
	(\textit{Đề kiểm tra GK1 -- Trường THPT Vàm Đình -- Cà Mau -- Năm 2024--2025})\\
	Cho dãy số có các số hạng: $1$; $2$; $3$; $4$,$\ldots $, $n$; $\ldots $ Khẳng định nào sau đây đúng?
	\choice
	{\True Dãy số đã cho là dãy số tăng}
	{Dãy số đã cho là dãy số chẵn}
	{Dãy số đã cho là dãy số giảm}
	{Dãy số đã cho là dãy số lẻ}
	\loigiai{
		Các số hạng sau lớn hơn số hạng trước nó nên dãy số đã cho là dãy số tăng.
	}
\end{ex}
\begin{ex}%[1D2N1-2]
	Cho dãy số $1$; $4$; $16$; $64$;\ldots (số hạng sau bằng $4$ lần số hạng đứng trước nó). Công thức số hạng tổng quát của dãy số đã cho là
	\choice
	{$u_n=4^{n+1}$}
	{\True $u_n=4^{n-1}$}
	{$u_n=4^{n}$}
	{$u_n=4+4^{n}$}
	\loigiai
	{
		Ta thấy $u_1=4^0$; $u_2=4^1$; $u_3=4^2$; $u_4=4^3$; $\ldots$ \\
		Vậy công thức số hạng tổng quát của dãy số đã cho là $u_n=4^{n-1}$.
	}
\end{ex}
\begin{ex}%[1D2N1-3]
	Cho dãy số $(u_n)$, biết $u_n=\dfrac{n}{2^n}$. Chọn đáp án đúng.
	\choice
	{\True $u_4=\dfrac{1}{4}$}
	{$u_5=\dfrac{1}{32}$}
	{$u_3=\dfrac{1}{8}$}
	{$u_5=\dfrac{1}{16}$}
	\loigiai{
		Ta có  $u_n=\dfrac{n}{2^n}$. Suy ra $u_4=\dfrac{4}{2^4}=\dfrac{1}{4}$.
	}
\end{ex}
\begin{ex}%[1D2N1-3]
	Cho dãy số $\left(u_n \right)$ với $u_n=\dfrac{-1}{n+1}, \forall n\in \mathbb{N}^*$. Giá trị của $u_3$ là
	\choice
	{$-\dfrac{1}{3}$}{$-4$}{\True $-\dfrac{1}{4}$}{$-\dfrac{1}{2}$}
	\loigiai{
		Ta có
		$u_3=\dfrac{-1}{3+1}=-\dfrac{1}{4}$.
	}
\end{ex}
\begin{ex}%[1D2N1-4]
	Dãy số nào trong các dãy số sau là dãy số tăng?
	\choice
	{$-1$; $1$; $-1$; $1$; $-1$; $\ldots$}
	{\True $1$; $3$; $5$; $7$; $9$}
	{$-2$; $-4$; $-6$; $-8$}
	{$\dfrac{1}{2}$; $\dfrac{1}{22}$; $\dfrac{1}{222}$; $\dfrac{1}{2\,222}$; $\ldots$}
	\loigiai{
		Dãy số $1; 3; 5; 7; 9$ là dãy số tăng.
	}
\end{ex}
\begin{ex}%[1D2H1-4]
	Cho dãy số $(u_n)$ biết $u_n=\dfrac{10}{3^n}$. Mệnh đề nào sau đây đúng?
	\choice
	{Dãy số tăng}
	{\True Dãy số giảm}
	{Dãy số không tăng, không giảm}
	{Dãy số vừa tăng vừa giảm}
	\loigiai{
		Ta có $u_{n+1}-u_n=\dfrac{10}{3^{n+1}}-\dfrac{10}{3^n}=\dfrac{10}{3\cdot 3^n}-\dfrac{10}{3^n}=\dfrac{-20}{3\cdot 3^n} < 0$.\\
		Vậy $u_{n+1}-u_n < 0\Leftrightarrow u_{n+1} < u_n, \forall n\in \mathbb{N}^{*}$.
		Do đó, dãy số đã cho là dãy số giảm.
	}
\end{ex}
\begin{ex}%[1D2N1-3]
	Cho dãy số $\left(u_n\right): \heva{&u_1=\dfrac{1}{2}\\& u_{n+1}=\dfrac{1}{2+u_n}}$ $ (n\ge 1)$. Tìm số hạng thứ ba của dãy số $\left(u_n\right)$.
	\choice
	{$u_3=\dfrac{2}{5}$}
	{$u_3=\dfrac{12}{5}$}
	{\True $u_3=\dfrac{5}{12}$}
	{$u_3=\dfrac{5}{13}$}
	\loigiai{
		Ta có $ u_2=\dfrac{1}{2+u_1}=\dfrac{1}{2+\dfrac{1}{2}}=\dfrac{2}{5} $ và $ u_3=\dfrac{1}{2+u_2}=\dfrac{1}{2+\dfrac{2}{5}}=\dfrac{5}{12} $.
	}
\end{ex}
\begin{ex}%[1D2N1-1]
	Cho dãy số $\left(u_n\right)$ được xác định bởi $\left\{\begin{aligned}&u_1=1 \\& u_{n+1}=2 u_n\end{aligned}, \forall n \in \mathbb{N}^*\right.$. Số hạng $u_2$ của dãy số đã cho là
	\choice
	{$u_2=1$}
	{\True $u_2=2$}
	{$u_2=3$}
	{$u_2=4$}
	\loigiai{
		Ta có $u_2=2u_1=2\cdot 1=2$.		
	}
\end{ex}
\begin{ex}%[1D2N1-2]
	Cho dãy số ($u_{n}$), biết $u_{n}=-\dfrac{n}{n+1}$. Năm số hạng đầu tiên của dãy số đó lần lượt là
	\choice
	{$-\dfrac{2}{3}$; $-\dfrac{3}{4}$; $-\dfrac{4}{5}$; $-\dfrac{5}{6}$; $-\dfrac{6}{7}$}
	{\True $-\dfrac{1}{2}$; $-\dfrac{2}{3}$; $-\dfrac{3}{4}$; $-\dfrac{4}{5}$; $-\dfrac{5}{6}$}
	{$\dfrac{1}{2}$; $\dfrac{2}{3}$; $\dfrac{3}{4}$; $\dfrac{4}{5}$; $\dfrac{5}{6}$}
	{$\dfrac{2}{3}$; $\dfrac{3}{4}$; $\dfrac{4}{5}$; $\dfrac{5}{6}$; $\dfrac{6}{7}$}
	\loigiai{
		Dãy ($u_{n}$) có năm số hạng đầu là $-\dfrac{1}{2}$; $-\dfrac{2}{3}$; $-\dfrac{3}{4}$; $-\dfrac{4}{5}$; $-\dfrac{5}{6}$.
	}
\end{ex}
\begin{ex}%[1D2N1-3]
	Cho dãy số $\left(u_n\right)$ biết $u_n = \dfrac{2n^2 - 1}{n^2 + 2}$. Số hạng thứ $10$ của dãy số là?
	\choice
	{$u_{10} = \dfrac{19}{2}$}
	{$u_{10} = \dfrac{33}{34}$}
	{\True $u_{10} = \dfrac{199}{102}$}
	{$u_{10} = \dfrac{3}{4}$}
	\loigiai{
		Để tìm số hạng thứ 10, ta thay $n = 10$ vào công thức của dãy số
		$$u_{10} = \dfrac{2(10)^2 - 1}{10^2 + 2} = \dfrac{2(100) - 1}{100 + 2} = \dfrac{200 - 1}{102} = \dfrac{199}{102}.$$
	}
\end{ex}
\begin{ex}%[1D2N1-4]
	Trong các dãy số sau đây số nào là dãy số giảm?
	\choice
	{$-5$, $-4$, $-3$, $-2$, $-1$}
	{\True $0$, $-1$, $-3$, $-5$, $-7$}
	{$0$, $3$, $12$, $16$, $19$}
	{$24$, $15$, $14$, $16$, $19$}
	\loigiai{Ta có $0$, $-1$, $-3$, $-5$, $-7$ là dãy số giảm.
	}
\end{ex}
\begin{ex}%[1D2N1-4]
	Trong các dãy số sau, dãy số nào là dãy số tăng?
	\choice
	{$1; 1; 1; 1; 1; 1; \ldots$}
	{$1;-\dfrac{1}{2}; \dfrac{1}{4};-\dfrac{1}{8}; \dfrac{1}{16}; \ldots$}
	{\True $1; 3; 5; 7; 9; \ldots$}
	{$1; \dfrac{1}{2}; \dfrac{1}{4}; \dfrac{1}{8}; \dfrac{1}{16}; \ldots$}
	\loigiai{Ta có
		\begin{itemize}
			\item $1; 1; 1; 1; 1; 1; \ldots$ là dãy số không đổi.
			\item $1;-\dfrac{1}{2}; \dfrac{1}{4};-\dfrac{1}{8}; \dfrac{1}{16}; \ldots$ là dãy số không tăng không giảm.
			\item $1; 3; 5; 7; 9; \ldots$ là dãy số tăng.
			\item $1; \dfrac{1}{2}; \dfrac{1}{4}; \dfrac{1}{8}; \dfrac{1}{16}; \ldots$ là dãy số giảm.
		\end{itemize}
	}
\end{ex}
\begin{ex}%[1D2N1-2]
	Cho dãy số có các số hạng đầu là $-2$; $0$; $2$; $4$; $6$; $\ldots$. Số hạng tổng quát của dãy số này là công thức nào dưới đây?
	\choice
	{$u_n=-2n$}
	{\True $u_n=2n-4$}
	{$u_n=-2(n+1)$}
	{$u_n=n-2$}
	\loigiai{Cho dãy số có các số hạng đầu là $-2$; $0$; $2$; $4$; $6$; $\ldots$.\\
		Số hạng tổng quát của dãy số này là công thức $u_n=2n-4$.}
\end{ex}
\begin{ex}%[1D2H1-4]
	Dãy số nào sau đây là dãy số tăng?
	\choice
	{Dãy số $\left(a_n\right)$ với $a_n=\dfrac{1}{n}$}
	{\True Dãy số $\left(b_n\right)$ với $b_n=2n+1$}
	{Dãy số $\left(c_n\right)$ với $c_n=-n^2$}
	{Dãy số $\left(d_n\right)$ với $d_n=\dfrac{1}{2^n}$}
	\loigiai{Kiểm tra các số hạng đầu của các dãy số $(a_n)$, $(c_n)$ và $(d_n)$, dễ dàng nhận thấy đây là các dãy số giảm.\\
		Ta có $b_{n+1}-b_n=2\left(n+1\right)+1-2n-1=2>0$.\\
		Vậy $\left(b_n\right)$ là dãy số tăng.	
	}
\end{ex}

%C:\Users\Admin\Downloads\SpDuAn-A-Dot7\SpDuAn-A-Dot7\data\HKI\1-TN-DS-TLN-TL-THPT-TanChau-AnGiang-HKI-NH24-25.tex
\begin{ex}%[Dự án đề kiểm tra Toán 11 HKI NH24-25- Nguyễn Quang Hiệp]%[THPT Tân Châu - An Giang]%[1D2H1-3]
	Cho dãy số $(u_n)$ với $\heva{&u_1=5 \\ &u_{n+1}=u_n+n, \ n \ge 2}$. Tìm số hạng thứ $4$ của dãy số đã cho.
	\choice
	{$12$}
	{\True $11$}
	{$16$}
	{$15$}
	\loigiai{
		Ta có
		\begin{itemize}
			\item $u_1=5$.
			\item $u_2=u_1+1=5+1=6$.
			\item 	$u_3=u_2+2=6+2=8$.
			\item 	$u_4=u_3+3=8+3=11$.
		\end{itemize}
		
		Vậy số hạng thứ $4$ của dãy số là $u_4=11$.
	}
\end{ex}

%C:\Users\Admin\Downloads\SpDuAn-A-Dot5\SpDuAn-A-Dot5\data\HKI\1-TN-DS-TLN-THPT-ChuyenHungVuong-PhuTho-HKI-NH24-25.tex
\begin{ex}%[1D2H1-3]%[Lớp 11 - Ôn tập cuối học kì 1 - Chuyên Hùng Vương-Phú Thọ-24-25]%[Phạm Văn Long]
	Cho dãy số $(u_n)$, biết $u_n=\dfrac{2}{n+3}$ với $n \in \mathbb{N}^*$. Số hạng thứ ba của dãy số là
	\choice
	{$u_3=\dfrac{1}{2}$}
	{$u_3=\dfrac{1}{4}$}
	{$u_3=\dfrac{2}{5}$}
	{\True $u_3=\dfrac{1}{3}$}
	\loigiai{
		Ta có $u_3=\dfrac{2}{3+3}=\dfrac{2}{6}=\dfrac{1}{3}$.
	}
\end{ex}

%C:\Users\Admin\Downloads\SpDuAn-A-Dot5\SpDuAn-A-Dot5\data\HKI\1-TN-DS-TLN-THPT-LeQuyDon-HCM-HKI-NH24-25.tex
\begin{ex}%[1D2H1-3]%[Dự án đề kiểm tra Toán 11 HKI NH24-25- Vanle Vo]%[THPT - Lê Quý Đôn - TPHCM]
	Cho dãy số $\left(u_n\right)$ xác định bởi 
	$\heva{&u_1=5\\&u_{n+1}=u_n+3, (n\geq1)}$.
	Tìm $u_3$.
	\choice
	{\True $11$}
	{$8$}
	{$14$}
	{$5$}
	\loigiai{
		Ta có
		\begin{itemize}
			\item 	$u_2=u_1+3=5+3=8$;
			\item $u_3=u_2+3=8+3=11$.
		\end{itemize}
		
	}
\end{ex}

%C:\Users\Admin\Downloads\SpDuAn-A-Dot4\SpDuAn-A-Dot4\data\GHKI\1-TN-DS-TLN-THPT-NguyenAnNinh-HCM-GHKI-NH24-25.tex
\begin{ex}%[1D2H1-3]%[Dự án đề kiểm tra toán khối 11 GHKI THPT NguyễnAnNinh]%[NGUYỄN HỮU ĐỨC]
	%%%%%-----Câu 7.-----%%%%%
	Cho dãy $(u_4)$, biết $u_n = \dfrac{2n^2 + 4}{n + 5}$. Tìm số hạng $u_4$.
	\choice
	{$u_4=6$}
	{$u_4=7$}
	{\True $u_4=4$}
	{$u_4=5$}
	\loigiai{
		Thay $n=4$ vào dãy ta được $u_4 = \dfrac{2 \cdot 4^2 + 4}{4 + 5} = \dfrac{2 \cdot 16 + 4}{9} = \dfrac{32 + 4}{9} = \dfrac{36}{9} = 4$.
	}
	
\end{ex}

%C:\Users\Admin\Downloads\SpDuAn-A-Dot6\SpDuAn-A-Dot6\data\HKI\1-TN-DS-TL-SGD-BacNinh-HKI-NH24-25.tex
\begin{ex}%[1D2H1-4]%[Dự án đề kiểm tra Khối 11 HK1 NH24-25 - Đợt 6 - Nguyễn Trần Anh Tuấn]%[SGD - Tỉnh Bắc Ninh]
	Dãy số $(u_n)$ có số hạng tổng quát là một trong bốn phương án A, B, C, D. Dãy số nào là dãy số giảm?
	\choice
	{$u_n=2n$, $ \forall n \in \mathbb{N}^{*}$}
	{\True $u_n=1-3n$, $ \forall n \in \mathbb{N}^{*}$}
	{$u_n=(-1)^n$, $ \forall n \in \mathbb{N}^{*}$}
	{$u_n=2008$, $ \forall n \in \mathbb{N}^{*}$}
	\loigiai{
		\begin{itemize}
			\item  Xét $(u_n)$ với $u_n=2n$ có $u_{n+1}=2(n+1)$, suy ra $u_{n+1}-u_n=2>0$.\\
			Do đó $u_{n+1}>u_n$ nên $(u_n)$ là dãy số tăng.
			\item  Xét $(u_n)$ với $u_n=1-3n$ có $u_{n+1}=1-3(n+1)=-3n-2$, suy ra $u_{n+1}-u_n=-3<0$.\\
			Do đó $u_{n+1}<u_n$ nên $(u_n)$ là dãy số giảm.
			\item  Xét $(u_n)$ với $u_n=(-1)^n$ có\\
			Nếu $n$ là số chẵn $\Rightarrow u_n=1>0$.\\
			Nếu $n$ là số lẻ $\Rightarrow u_n=-1<0$.\\
			$\Rightarrow u_n$ không tăng không giảm.
			\item  Xét $(u_n)$ với $u_n=2\,008$ là dãy không tăng không giảm.
		\end{itemize}
	}
\end{ex}
\Closesolutionfile{ans}
\ind{PHẦN II.} \inden{Câu trắc nghiệm đúng sai. Trong mỗi ý a), b), c), d) ở mỗi câu, học sinh chọn đúng hoặc sai.}\\
\setcounter{ex}{0}
\Opensolutionfile{ans}[ans/2D1-Bai1-DS]
\begin{ex}%[1D2H1-4]
	(\textit{Đề ôn tập HK1 -- Trường THPT Nguyễn Gia Thiều -- Hà Nội -- Năm 2024--2025})\\
	Cho dãy số $(u_n)$ được xác định bởi $u_1=1$ và $u_n=u_{n-1}+2n$ với mọi $n \geq 2$.
	\choiceTF
	{\True Ba số hạng đầu tiên của dãy số lần lượt là $1; 5; 11$}
	{Số hạng thứ tư của dãy là $17$}
	{\True Ta có $u_4> u_3$}
	{$(u_n)$ là một dãy số giảm}
	\loigiai{
		\begin{itemchoice}
			\itemch 
			Ta có $u_1=1$, $u_2=u_1+2\cdot 2=5$, $u_3=u_2+2\cdot 3=11$.
			\itemch
			Số hạng thứ tư của dãy là $u_4=u_3+2\cdot 4=19$.
			\itemch 
			Ta có $u_4=19$ và $u_3=11$ nên $u_4> u_3$.
			\itemch
			Ta có $u_n=u_{n-1}+2n\Rightarrow u_n-u_{n-1}=2n>0$ với mọi $n\in \mathbb{N}^*$ nên $(u_n)$ là một dãy số tăng.
		\end{itemchoice}
	}
\end{ex}
\begin{ex}%[1D2H1-3]
	Cho dãy số $(u_n)$, biết $u_n=-\dfrac{n}{n+1}$. Khi đó
	\choiceTF
	{\True Năm số hạng đầu tiên của dãy số là $u_1=-\dfrac{1}{2}$;  $u_2=-\dfrac{2}{3}$; $u_3=-\dfrac{3}{4}$; $u_4=-\dfrac{4}{5}$;  $u_5=-\dfrac{5}{6}$}
	{\True Số hạng $u_{10}$, $u_{100}$ lần lượt là $-\dfrac{10}{11}$; $-\dfrac{100}{101}$}
	{$-\dfrac{85}{86}$ là số hạng thứ $86$ của dãy số $(u_n)$}
	{$-\dfrac{99}{101}$ là một số hạng của dãy số $(u_n)$}
	\loigiai{
		\begin{itemchoice}
			\itemch Ta có $u_1=-\dfrac{1}{2}$;  $u_2=-\dfrac{2}{3}$; $u_3=-\dfrac{3}{4}$; $u_4=-\dfrac{4}{5}$; $u_5=-\dfrac{5}{6}$.
			\itemch Ta có $u_{10}=-\dfrac{10}{11}$, $u_{100}=-\dfrac{100}{101}$.
			\itemch Ta có $-\dfrac{85}{86}=\dfrac{-n}{n+1} \Leftrightarrow 85n+85=86n \Leftrightarrow n=85$.\\
			Vậy $-\dfrac{85}{86}$ là số hạng thứ $85$ của dãy $(u_n)$.
			\itemch Ta có $-\dfrac{99}{101}=\dfrac{-n}{n+1} \Leftrightarrow 99n+99=101n \Leftrightarrow n=\dfrac{99}{2} \notin \mathbb{N}^{*}$ (loại).\\
			Vậy $-\dfrac{99}{101}$ không phải là số hạng của dãy số $(u_n)$.
		\end{itemchoice}
	}
\end{ex}
\begin{ex}%[1D2H1-3]
	Cho dãy số $(u_n)$ có số hạng tổng quát $u_n=\dfrac{2n+1}{n+2}$. Khi đó
	\choiceTF
	{\True Số hạng đầu tiên của dãy số là $1$}
	{\True Số hạng $u_2=\dfrac{5}{4}; u_3=\dfrac{7}{5}$}
	{$u_4 > u_5$}
	{Số $\dfrac{167}{84}$ là số hạng thứ $252$ của dãy số $(u_n)$}
	\loigiai{
		\begin{itemchoice}
			\itemch Ta có $u_1=\dfrac{2\cdot 1+1}{1+2}=1$.
			\itemch Ta có $u_2=\dfrac{2\cdot 2+1}{2+2}=\dfrac{5}{4}$, $u_3=\dfrac{2\cdot 3+1}{3+2}=\dfrac{7}{5}$.
			\itemch Ta có $u_4=\dfrac{2\cdot 4+1}{4+2}=\dfrac{3}{2}$, $u_5=\dfrac{2\cdot 5+1}{5+2}=\dfrac{11}{7}$. \\
			Suy ra $u_4 < u_5$.
			\itemch Ta có $\dfrac{2n+1}{n+2}=\dfrac{167}{84} \Leftrightarrow 84(2n+1)=167(n+2) \Leftrightarrow n=250$.\\
			Vậy $\dfrac{167}{84}$ là số hạng thứ $250$ của dãy số $(u_n)$.
		\end{itemchoice}
	}
\end{ex}
\begin{ex}%[1D2V1-3]
	Cho dãy số $(u_n)$ xác định bởi $u_n=\dfrac{1}{1\cdot 3}+\dfrac{1}{3\cdot 5}+\dfrac{1}{5\cdot 7}+\cdots+\dfrac{1}{(2n-1) \cdot(2n+1)}$. Khi đó
	\choiceTF
	{Số hạng thứ $2\,021$ là $\dfrac{2\,021}{4\,040}$}
	{Số hạng thứ $2\,022$ là $\dfrac{2\,022}{4\,043}$}
	{\True Số hạng thứ $2\,023$ là $\dfrac{2\,023}{4\,047}$}
	{\True Tổng các số hạng thứ $2\,021$; $2\,022$; $2\,023$ và $2\,024$ nhỏ hơn $2$}
	\loigiai{
		Với $k$ là số nguyên dương, ta có \\
		$ \dfrac{1}{(2k-1) \cdot(2k+1)}=\dfrac{1}{2}\left[\dfrac{(2k+1)-(2k-1)}{(2k-1) \cdot(2k+1)}\right]=\dfrac{1}{2}\left(\dfrac{1}{(2k-1)}-\dfrac{1}{(2k+1)}\right)$.\\	
		Khi đó
		\allowdisplaybreaks
		\begin{align*}
			u_n&=\dfrac{1}{2}\left[\left(\dfrac{1}{1}-\dfrac{1}{3}\right)+\left(\dfrac{1}{3}-\dfrac{1}{5}\right)+\left(\dfrac{1}{5}-\dfrac{1}{7}\right)+\cdots+\left(\dfrac{1}{(2n-1)}-\dfrac{1}{(2n+1)}\right)\right]
			\\
			&=\dfrac{1}{2}\left[1-\dfrac{1}{(2n+1)}\right]
			\\
			&=\dfrac{n}{2n+1}.
		\end{align*}
		Vậy $u_n=\dfrac{n}{2n+1}$, với mọi $n \in \mathbb{N}^{*}$.
		\begin{itemchoice}
			\itemch Áp dụng công thức số hạng tổng quát ta có
			$u_{2\,021}=\dfrac{2\,021}{2\cdot 2\,021+1}=\dfrac{2\,021}{4\,043}$.
			\itemch Ta có $u_{2\,022}=\dfrac{2\,022}{2\cdot 2\,022+1}=\dfrac{2\,022}{4\,045}$.
			\itemch Ta có $u_{2\,023}=\dfrac{2\,023}{2\cdot 2\,023+1}=\dfrac{2\,023}{4\,047}$.
			\itemch Ta có $u_{2\,024}=\dfrac{2\,024}{2\cdot 2\,024+1}=\dfrac{2\,024}{4\,049}$.\\
			Vậy $u_{2\,021}+u_{2\,022}+u_{2\,023}+u_{2\,024}=\dfrac{2\,021}{4\,043}+\dfrac{2\,022}{4\,045}+\dfrac{2\,023}{4\,047}+\dfrac{2\,024}{4\,049} \approx 1{,}9995< 2$.
		\end{itemchoice}
	}
\end{ex}
\begin{ex}%[1D2H1-4]
	Cho dãy số $(u_n)$ có số hạng tổng quát $u_n=1-\dfrac{1}{n}$. Khi đó
	\choiceTF
	{\True $u_3=\dfrac{2}{3}$}
	{$u_7-u_8=\dfrac{1}{56}$}
	{$u_{n+1}-u_n=-\dfrac{1}{n(n+1)}$}
	{\True Dãy số $(u_n)$ là dãy số tăng}
	\loigiai{\begin{itemchoice}
			\itemch Ta có $u_3=1-\dfrac{1}{3}=\dfrac{2}{3}$.
			\itemch Ta có $u_7-u_8=\left(1-\dfrac{1}{7}\right)-\left(1-\dfrac{1}{8}\right)=-\dfrac{1}{56}$.
			\itemch Ta có $u_{n+1}-u_n=\left(1-\dfrac{1}{n+1}\right)-\left(1-\dfrac{1}{n}\right)=\dfrac{1}{n}-\dfrac{1}{n+1}=\dfrac{1}{n(n+1)}$.
			\itemch Do $u_{n+1}-u_n=\dfrac{1}{n(n+1)}>0$, $\forall n\in\mathbb{N}^*$ nên $u_{n+1}>u_n$, $\forall n\in\mathbb{N}^*$.\\ 
			Suy ra $(u_n)$ là dãy số tăng.
	\end{itemchoice}}
\end{ex}
\Closesolutionfile{ans}
\ind{PHẦN III.} \inden{Trắc nghiệm trả lời ngắn.}
\setcounter{ex}{0}

\begin{ex}%[1D2H1-3]
	(\textit{Đề kiểm tra HK1 -- Trường THPT Chuyên Hùng Vương -- Phú Thọ -- Năm 2024--2025})\\
	Số $\dfrac{9}{41}$ là số hạng thứ mấy của dãy số $(u_n)$ với $u_n=\dfrac{2n}{n^2+1}$?
	\par \shortans[oly]{$9$}
	\loigiai{
		Ta có 
		$\dfrac{2n}{n^2+1}=\dfrac{9}{41}\Leftrightarrow 2n\cdot 41 = (n^2+1)\cdot 9\Leftrightarrow 82n=9n^2+9\Leftrightarrow9n^2-82n+9=0\Leftrightarrow\hoac{&n=9 \text{\,(nhận)}\\&n=\dfrac{1}{9} \text{\,(loại).}}$\\
		Vậy số $\dfrac{9}{41}$ là số hạng thứ $9$ của dãy số $u_n=\dfrac{2n}{n^2+1}$.
	}
\end{ex}
\begin{ex}%[1D2H1-3]
	(\textit{Đề kiểm tra HK1 -- Trường THPT Nguyễn Gia Thiều -- Hà Nội -- Năm 2024--2025})\\
	Các nhà nghiên cứu đã chỉ ra công thức tính cân nặng lí tưởng theo tuổi ở trẻ em từ $2$ tuổi đến $12$ tuổi là $u_n=2n+8$, trong đó $n$ là số tuổi của trẻ và $u_n$ là cân nặng lí tưởng đơn vị kilôgam. Hỏi theo công thức trên thì cân nặng lí tưởng của trẻ $6$ tuổi là bao nhiêu kilôgam?
	\par \shortans[oly]{$20$}
	\loigiai{
		Cân nặng lí tưởng của trẻ $6$ tuổi là $u_6=2\cdot 6+8=20$ kg.
	}
\end{ex}

\begin{ex}%[1D2H1-2]
	(\textit{Dự án đề kiểm tra lớp 11 -- Toán Từ Tâm })\\
	Số hạng tổng quát $u_n$ theo $n$ của dãy số $(u_n) \colon \heva{&u_1=2 \\& u_{n+1}=2u_n,\ \forall n \ge 1}$ có dạng $u_n=a^n$, với $a$ là số tự nhiên. Xác định giá trị của $a$.
	\shortans[oly]{$2$}
	\loigiai{
		Từ công thức $u_{n+1}=2u_n,\ \forall n \ge 1$ suy ra $\dfrac{u_{n+1}}{u_n}=2,\ \forall n \ge 1$. Từ đó ta có:
		$$
		\dfrac{u_2}{u_1}=2; \dfrac{u_3}{u_2}=2; \dfrac{u_4}{u_3}=2; \ldots ; \dfrac{u_n}{u_{n-1}}=2.$$
		Nhân theo vế tất cả các đẳng thức trên ta được:
		$$\dfrac{u_n}{u_1}=2^{n-1} \Leftrightarrow u_n=u_1\cdot 2^{n-1} \Leftrightarrow u_n=2\cdot 2^{n-1} \Leftrightarrow u_n=2^n.$$
	}
\end{ex}
\begin{ex}%[1D2H1-2]
	Số hạng tổng quát $u_n$ theo $n$ của dãy số $(u_n)$ : $\heva{&u_1=2 \\& u_{n+1}=2u_n,\ \forall n \ge 1}$ có dạng $u_n=a^n$, với $a$ là số tự nhiên. Xác định giá trị của $a$.
	\par \shortans[oly]{$2$}
	\loigiai{
		Từ công thức $u_{n+1}=2u_n,\ \forall n \ge 1$ suy ra $\dfrac{u_{n+1}}{u_n}=2,\ \forall n \ge 1$. Từ đó ta có
		$$
		\dfrac{u_2}{u_1}=2; \dfrac{u_3}{u_2}=2; \dfrac{u_4}{u_3}=2; \ldots ; \dfrac{u_n}{u_{n-1}}=2.$$
		Nhân theo vế tất cả các đẳng thức trên ta được
		$$\dfrac{u_n}{u_1}=2^{n-1} \Leftrightarrow u_n=u_1\cdot 2^{n-1} \Leftrightarrow u_n=2\cdot 2^{n-1} \Leftrightarrow u_n=2^n.$$
	}
\end{ex}
\begin{ex}%[1D2H1-4]
	Có bao nhiêu giá trị nguyên của $m$ trong đoạn $[-20;20]$ để dãy số $(u_n)$ với $u_n=\dfrac{mn+1}{n+1}$ là dãy số tăng?
	\par \shortans[oly]{$19$}
	\loigiai{
		Ta có $u_{n+1}=\dfrac{mn+m+1}{n+2}$. Xét
		$$u_{n+1}-u_n=\dfrac{mn+m+1}{n+2}-\dfrac{mn+1}{n+1}=\dfrac{m-1}{(n+2)(n+1)}.$$
		Để dãy số $(u_n)$ là dãy số tăng thì $u_{n+1}-u_n>0$, $\forall n\in\mathbb{N}^*$. Từ đó suy ra $m-1>0\Leftrightarrow m>1$.\\
		Mà $m$ nguyên nên $m\in \{2;3;4;\ldots;20\}$.\\
		Vậy có tất cả $19$ giá trị nguyên của $m$ thỏa mãn.
	}
\end{ex}
\Closesolutionfile{ans}
\ind{PHẦN IV.} \inden{Tự luận.}\\
\setcounter{ex}{0}
\begin{ex}%[1D2H1-4]
	(\textit{Dự án đề kiểm tra lớp 11 -- Toán Từ Tâm })\\
	Xét tính đơn điệu của dãy số $(u_n)$ biết
	\begin{multicols}{2}
		\begin{enumerate}[1)]
			\item $u_n = 3n+6$;
			\item $u_n = \dfrac{n+5}{n+2}$.
		\end{enumerate}
	\end{multicols}
	\loigiai{
		\begin{enumerate}[1)]
			\item $u_n = 3n+6$.\\
			Ta có $u_n = 3n+6 \Rightarrow u_{n+1} = 3(n+1)+6 = 3n+9$.\\
			Xét hiệu $u_{n+1} - u_n = (3n+9)-(3n+6) = 3 >0$, $\forall n \in \mathbb{N}^{\ast}$.\\
			Vậy $(u_n)$ là dãy số tăng.
			\item $u_n = \dfrac{n+5}{n+2}$.\\
			Ta có $u_n = \dfrac{n+5}{n+2} = 1+\dfrac{3}{n+2} \Rightarrow u_{n+1} = 1+\dfrac{3}{n+3}$.\\
			Xét hiệu $u_{n+1} - u_n = \dfrac{3}{n+3}-\dfrac{3}{n+2} = \dfrac{-3}{(n+2)(n+3)} <0$, $\forall n \in \mathbb{N}^{\ast}$.\\
			Vậy $(u_n)$ là dãy số giảm.
		\end{enumerate}
	}
\end{ex}
\begin{ex}%[1D2V1-2]
	Cho dãy số $(u_n)$ với $u_n=\dfrac{n^2+3n+7}{n+1}$. Hỏi dãy số trên có bao nhiêu số hạng nhận giá trị nguyên?
	\loigiai{
		Ta có $u_n=\dfrac{n^2+3n+7}{n+1}=n+2+\dfrac{5}{n+1}$, $\left(n\in \mathbb{N}^{*} \right)$.\\
		Để $u_n$ nhận giá trị nguyên thì $\dfrac{5}{n+1}$, $\left(n\in \mathbb{N}^{*} \right)$ là số nguyên hay $n=4$.\\
		Vậy dãy số $(u_n)$ chỉ có một số hạng nhận giá trị nguyên.
	}
\end{ex}

\begin{ex}%[1D2H1-3]
	(\textit{Dự án đề kiểm tra lớp 11 -- Toán Từ Tâm })\\
	Viết năm số hạng đầu và số hạng thứ $100$ của các dãy số $\left(u_n\right)$ có số hạng tổng quát cho bởi
	\begin{enumerate}[(1)]
		\item $u_n=3 n-2$;
		\item $u_n=3 \cdot 2^n$;
		\item $u_n=\left(1+\dfrac{1}{n}\right)^n$.
	\end{enumerate}	
	\loigiai{	\begin{enumerate}[(1)]
			\item $u_n=3 n-2$. \\
			Ta có
			$
			u_1=1$; $ u_2=4$; $ u_3=7$; $ u_4=10$; $ u_5=13$; $ u_{100}=298
			$.
			\item $u_n=3 \cdot 2^n$.\\ 
			Ta có
			$
			u_1=6$; $ u_2=12$; $ u_3=24$; $ u_4=48$; $ u_5=96, u_{100}=3{,}803 \cdot 10^{30}
			$.
			\item $u_n=\left(1+\dfrac{1}{n}\right)^n$. \\
			Ta có
			$
			u_1=2$; $ u_2=\dfrac{9}{4}$; $ u_3=\dfrac{64}{27}$; $ u_4=\dfrac{625}{256}$; $ u_5=\dfrac{7\,736}{3\,125}$; $ u_{100}=2{,}7148
			$.
	\end{enumerate}	}	
\end{ex}
\begin{ex}%[1D2H1-3]
	(\textit{Dự án đề kiểm tra lớp 11 -- Toán Từ Tâm })\\
	Cho dãy số $\left(u_n\right)$ biết $u_n=\dfrac{1}{\sqrt{5}}\left[\left(\dfrac{1+\sqrt{5}}{2}\right)^n-\left(\dfrac{1-\sqrt{5}}{2}\right)^n\right]$. Tìm số hạng $u_6$.
	\loigiai{ Ta có
		$u_6=\dfrac{1}{\sqrt{5}}\left[\left(\dfrac{1+\sqrt{5}}{2}\right)^6-\left(\dfrac{1-\sqrt{5}}{2}\right)^6\right]=8$.
		
	}		
\end{ex}
\begin{ex}%[1D2H1-3]
	(\textit{Dự án đề kiểm tra lớp 11 -- Toán Từ Tâm })\\
	Cho dãy số $(u_n)$ biết $\heva{&u_1=1\\&u_{n+1} = \dfrac{u_n+2}{u_n+1}}$. Tìm số hạng $u_{10}$.
	\loigiai{ Ta có
		\begin{itemize}
			\item $u_2 = \dfrac{u_1+2}{u_1+1} = \dfrac{1+2}{1+1}=\dfrac{3}{2}$; $u_3 = \dfrac{u_2+2}{u_2+1} = \dfrac{\dfrac{3}{2}+2}{\dfrac{3}{2}+1} = \dfrac{7}{5}$; $u_4 = \dfrac{u_3+2}{u_3+1} = \dfrac{\dfrac{7}{5}+2}{\dfrac{7}{5}+1} = \dfrac{17}{12}$;
			\item $u_5 = \dfrac{u_4+2}{u_4+1} = \dfrac{\dfrac{17}{12}+2}{\dfrac{17}{12}+1} = \dfrac{41}{29}$; $u_6 = \dfrac{u_5+2}{u_5+1} = \dfrac{\dfrac{41}{29}+2}{\dfrac{41}{29}+1} = \dfrac{99}{70}$; $u_7 = \dfrac{u_6+2}{u_6+1} = \dfrac{\dfrac{99}{70}+2}{\dfrac{99}{70}+1} = \dfrac{239}{169}$;
			\item 	$u_8 = \dfrac{u_7+2}{u_7+1} = \dfrac{\dfrac{239}{169}+2}{\dfrac{239}{169}+1} = \dfrac{577}{408}$; $u_9 = \dfrac{u_8+2}{u_8+1} = \dfrac{\dfrac{577}{408}+2}{\dfrac{577}{408}+1} = \dfrac{1\,393}{985}$;
			\item 	$u_{10} = \dfrac{u_9+2}{u_9+1} = \dfrac{\dfrac{1\,393}{985}+2}{\dfrac{1\,393}{985}+1} = \dfrac{3\,363}{2\,378}$.
		\end{itemize}
	}
\end{ex}
\begin{ex}%[1D2H1-3]
	(\textit{Dự án đề kiểm tra lớp 11 -- Toán Từ Tâm })\\
	Cho dãy số $(u_n)$ được xác định như sau $\heva{&u_1=1; \, u_2=2\\&u_{n+2}=2u_{n+1}+3u_n+5}$. Tìm số hạng $u_8$.
	\loigiai{
		Ta có
		\begin{itemize}
			\item $u_3 = 2u_2+3u_1+5=12$; $u_4 = 2u_3+3u_2+5=35$; $u_5 = 2u_4+3u_3+5=111$;
			\item $u_6 = 2u_5+3u_4+5=332$; $u_7 = 2u_6+3u_5+5=\,1002$; $u_8 = 2u_7+3u_6+5=3\,005$.
		\end{itemize}
	}
\end{ex}
\begin{ex}%[1D2V1-3]
	(\textit{Dự án đề kiểm tra lớp 11 -- Toán Từ Tâm })\\
	Cho dãy số $(u_n)$ được xác minh như sau $\heva{&u_1=0\\&u_{n+1}=\dfrac{n}{n+1}\left(u_n+1\right)}$. Tìm số hạng $u_{11}$ 
	\loigiai{
		Ta có
		\begin{itemize}
			\item $u_2=\dfrac{1}{2}\left( u_1 +1 \right) = \dfrac{1}{2}$; $u_3=\dfrac{2}{3}\left( u_2 +1 \right) = 1$; $u_4=\dfrac{3}{4}\left( u_3 +1 \right) = \dfrac{3}{2}$; $u_5=\dfrac{4}{5}\left( u_4 +1 \right) = 2$;
			\item $u_6=\dfrac{5}{6}\left( u_5 +1 \right) = \dfrac{5}{2}$; $u_7=\dfrac{6}{7}\left( u_6 +1 \right) = 3$; $u_8=\dfrac{7}{8}\left( u_7 +1 \right) = \dfrac{7}{2}$;$u_9=\dfrac{8}{9}\left( u_8 +1 \right) = 4$;
			\item 	$u_{10}=\dfrac{9}{10}\left( u_9 +1 \right) = \dfrac{9}{2}$; $u_{11}=\dfrac{10}{11}\left( u_{10} +1 \right) = 5$. 
		\end{itemize}
	}
\end{ex}
\begin{ex}%[1D2V1-4]
	
	Có bao nhiêu giá trị nguyên của $m$ trong đoạn $[-15;15]$ để dãy số $(u_n)$ với $u_n=\dfrac{n+m}{n+2}$ là dãy số giảm?
	\loigiai{
		Ta có $u_{n+1}=\dfrac{n+1+m}{n+3}$. Xét
		$$u_{n+1}-u_n=\dfrac{n+1+m}{n+3}-\dfrac{n+m}{n+2}=\dfrac{-m-1}{(n+3)(n+2)}.$$
		Để dãy số $(u_n)$ là dãy số giảm thì $u_{n+1}-u_n<0,\ \forall n \in \mathbb{N}^*$. Suy ra $-m-1<0 \Leftrightarrow m>-1$.\\
		Mà $m$ nguyên, $m\in[-15;15]$ nên $m\in \{0;1;2;\ldots;15\}$.\\
		Vậy có tất cả $16$ giá trị nguyên của $m$ thỏa mãn.
	}
\end{ex}
\begin{ex}%[1D2H1-3]
	Xét dãy số $u_n = \dfrac{3n+1}{n^2+1}$. Hỏi:
	\begin{enumerate}
		\item Số $\dfrac{10}{37}$ là số hạng thứ mấy của dãy số $(u_n)$?
		\item Số $\dfrac{13}{50}$ có là một số hạng của dãy số $(u_n)$ không?
	\end{enumerate}
	\loigiai{
		Ta có
		\begin{enumerate}
			\item Giải phương trình 
			\begin{eqnarray*}
				\dfrac{3n+1}{n^2+1} = \dfrac{10}{37} \Leftrightarrow (3n+1)\cdot 37 = (n^2+1)\cdot 10 \Leftrightarrow 111n + 37 = 10n^2 + 10 \Rightarrow 10n^2 - 111n - 27 = 0 \Leftrightarrow \hoac{&n=4\ (\text{nhận})\\ &n=\dfrac{27}{10}\ (\text{loại}).}
			\end{eqnarray*}
			Vậy $\dfrac{10}{37}$ là số hạng thứ $4$ của dãy số.
			\item Giải phương trình
			\begin{eqnarray*}
				\dfrac{3n+1}{n^2+1} = \dfrac{13}{50} \Leftrightarrow (3n+1)\cdot 50 = (n^2+1)\cdot 13 \Leftrightarrow 150n + 50 = 13n^2 + 13 \Rightarrow 13n^2 - 150n - 37 = 0 \Leftrightarrow \hoac{&x=\dfrac{-\sqrt{6\,106}+75}{13}\\&x=\dfrac{\sqrt{6\,106}+75}{13}.}
			\end{eqnarray*}
			Vậy phương trình không có nghiệm nguyên $\Rightarrow$ không tồn tại $n \in \mathbb{N}^*$ để $u_n = \dfrac{13}{50}$.
		\end{enumerate}
	}
\end{ex}
\begin{ex}%[1D2H1-5]
	Xét các dãy số $(u_n)$ sau. Hãy cho biết dãy số nào trong các dãy sau là \lq\lq bị chặn \rq\rq?
	\begin{enumerate}
		\item $u_n = n - \sin 3n$.
		\item $u_n = \dfrac{n^2 + 1}{n}$.
		\item $u_n = \dfrac{1}{n(n+1)}$.
		\item $u_n = n \cdot \sin(3n - 1)$.
	\end{enumerate}
	\loigiai{
		\begin{itemize}
			\item Với $u_n = n - \sin 3n$ thì $\sin 3n \in [-1; 1] \Rightarrow u_n \in [n - 1;\ n + 1]$, nên $u_n \to \infty$ khi $n \to \infty$ $\Rightarrow$ dãy không bị chặn.
			\item Với $u_n = \dfrac{n^2 + 1}{n} = n + \dfrac{1}{n} \to \infty$ khi $n \to \infty$ $\Rightarrow$ dãy không bị chặn.
			\item Với $u_n = \dfrac{1}{n(n+1)}$, ta có
			\begin{itemize}
				\item $u_n > 0,\ \forall n \in \mathbb{N}^*$.
				\item Xét hiệu:
				\begin{eqnarray*}
					u_{n+1} - u_n &=& \dfrac{1}{(n+1)(n+2)} - \dfrac{1}{n(n+1)} \\
					&=& \dfrac{n - (n+2)}{n(n+1)(n+2)} = -\dfrac{2}{n(n+1)(n+2)} < 0.
				\end{eqnarray*}
				\item Dãy giảm, dương nên bị chặn trên bởi $u_1 = \dfrac{1}{2}$.
			\end{itemize}
			$\Rightarrow$ Dãy bị chặn.
			\item Với $u_n = n \cdot \sin(3n - 1)$ thì $\sin(3n - 1) \in [-1; 1] \Rightarrow u_n \in [-n;\ n] \Rightarrow$ không bị chặn khi $n \to \infty$.
		\end{itemize}
		Vậy dãy bị chặn là $u_n = \dfrac{1}{n(n+1)}$.
	}
\end{ex}