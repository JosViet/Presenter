\newpage
\section{Ôn tập chương 5}
\def\thoigian{90}%--Thời gian
\de{Đề số 1}{Chương V. MỘT SỐ YẾU TỐ THỐNG KÊ}


\begin{center}
	\textbf{PHẦN 1 - CÂU TRẮC NGHIỆM BỐN PHƯƠNG ÁN}
\end{center}
\Opensolutionfile{ans}[ans/ans-TN-ONTAPCHUONG9-DE1]

\begin{ex}%[Dự án đề cương 3 khối NH24-25 - Đợt 3 - Hải Phụng]%[1D5N1-1]
	Khảo sát thời gian sử dụng Internet trong một ngày của $45$ học sinh lớp $11$A, cô giáo chủ nhiệm thu được mẫu số liệu ghép nhóm (đơn vị: phút), với năm nhóm, như sau\\
	\centerline{
		\begin{tabular}{|c|c|c|c|c|c|c|}
			\hline
			Thời gian (phút) & $[0;60)$ & $[60;120)$ & $[120;180)$ & $[180;240)$ & $[240;300)$ & \\ \hline
			Số học sinh & $3$ & $12$ & $23$ & $5$ & $2$ & $n=45$ \\ \hline
		\end{tabular}
	}\\
	Giá trị đại diện của nhóm $[120;180)$ bằng
	\choice
	{$300$}
	{$180$}
	{\True $150$}
	{$120$}
	\loigiai{Giá trị đại diện của nhóm $[120;180)$ bằng $\dfrac{120+180}{2}=150$.}
\end{ex}

\begin{ex}%[Dự án đề cương 3 khối NH24-25 - Đợt 3 - Hải Phụng]%[1D5N1-2]
	\immini[thm]{Thống kê chiều cao của học sinh lớp $11 \mathrm{A}$ ta có bảng số liệu như hình bên. Hỏi lớp có bao nhiêu học sinh có chiều cao từ $168$ cm trở lên?
		\choice
		{\True$11$}
		{$20$}
		{$31$}
		{$8$}}{\begin{tabular}{|c|c|}
			\hline
			Chiều cao (cm) & Số học sinh\\
			\hline
			$[150 ; 156)$ & $8$\\
			$[156 ; 162)$ & $12$\\
			$[162 ; 168)$ & $11$\\
			$[168 ; 174)$ & $8$\\
			$[174 ; 180)$ & $3$\\
			\hline
	\end{tabular}}
	\loigiai{
		Số học sinh có chiều cao từ $168$ cm trở lên là $8+3=11$.
	}
\end{ex}

\begin{ex}%[Dự án đề cương 3 khối NH24-25 - Đợt 3 - Hải Phụng]%[1D5N1-2]
	Đo chiều cao (tính bằng cm) của $300$ học sinh một trường THPT thu được kết quả như sau
	\begin{center}
		\begin{tabular}{|l|c|c|c|c|c|}
			\hline
			Chiều cao & $[150;154)$ & $[154;158)$ & $[158;162)$ & $[162;166)$ & $[166;170)$ \\ 
			\hline
			Số học sinh & $60$ & $65$ & $52$ & $55$ & $68$ \\
			\hline
		\end{tabular}
	\end{center}
	Tần số tích lũy của nhóm $[154;158)$ là
	\choice
	{$65$}
	{\True $125$}
	{$156$}
	{$117$}
	\loigiai
	{
		Vì nhóm $[154;158)$ là nhóm số $2$ nên tần số tích lũy của nhóm này là
		$$cf_2=n_1+n_2=60+65=125.$$
	}
\end{ex}

\begin{ex}%[Dự án đề cương 3 khối NH24-25 - Đợt 3 - Hải Phụng]%[1D5H1-3]
	Cho mẫu số liệu ghép nhóm về cân nặng và số người như sau
	\begin{center}
		\begin{tabular}{|c|c|c|c|c|c|c|}
			\hline
			Cân nặng & [39;48) & [48;57) & [57;66) & [66;75) & [75;84) & [84;93)\\ \hline
			Số người & 28	& 2 & 13 & 32 & 26 & 31 \\ \hline
		\end{tabular}
	\end{center}
	Tính cân nặng trung bình từ mẫu số liệu ghép nhóm trên.
	\choice
	{\True $69{,}61$}
	{$66{,}00$}
	{$44{,}00$}
	{$65{,}11$}
	\loigiai{Ta có bảng thống kê cân nặng tính theo gia trị đại diện
		\begin{center}
			\begin{tabular}{|c|c|c|c|c|c|c|}
				\hline
				Cân nặng đại diện & 43{,}5 & 52{,}5 & 61{,}5 & 70{,}5 & 79{,}5 & 88{,}5\\ \hline
				Số người & 28	& 2 & 13 & 32 & 26 & 31 \\ \hline
			\end{tabular}
		\end{center}
		Cân nặng trung bình là $\dfrac{28 \cdot 43{,}5 + 2 \cdot 52{,}5 + 13 \cdot 61{,}5 + 32 \cdot 70{,}5 + 26 \cdot 79{,}5 + 31 \cdot 88{,}5}{28+2+13+32+26+31} \approx 69{,}61$.
	}
\end{ex}

\begin{ex}%[Dự án đề cương 3 khối NH24-25 - Đợt 3 - Hải Phụng]%[1D5N1-3]
	Sau khi tăng cường xử lý vi phạm nồng độ cồn của tài xế điều khiển phương tiện giao thông trong $30$ ngày ở thành phố nọ, người ta chia mẫu số liệu đó thành năm nhóm căn cứ vào số lượng người vi phạm mỗi ngày (đơn vị: người) và lập bảng tần số ghép nhóm bao gồm cả tần số tích lũy như bảng sau
	\begin{center}
		\begin{tabular}{|c|c|c|}
			\hline
			Nhóm & Tần số & Tần số tích lũy\\
			\hline
			$[30;54)$ & $5$ & $5$\\
			\hline
			$[54;78)$ & $3$ & $8$\\
			\hline
			$[78;102)$ & $6$ & $14$\\
			\hline
			$[102;126)$ & $9$ & $23$\\
			\hline
			$[126;150)$ & $7$ & $30$\\
			\hline
			& $n = 30$ & \\
			\hline
		\end{tabular}
	\end{center}
	Số trung bình của mẫu số liệu ghép nhóm trên là
	\choice
	{$30$}
	{$48$}
	{\True $98$}
	{$14$}
	\loigiai{
		Số trung bình của mẫu số liệu ghép nhóm trên là
		$$ \bar{x} = \dfrac{5\cdot 42 + 3\cdot 66 + 6\cdot 99 + 9\cdot 114 + 7\cdot 138}{30} = 98. $$
	}
\end{ex}

\begin{ex}%[Dự án đề cương 3 khối NH24-25 - Đợt 3 - Hải Phụng]%[1D5H1-4]
	An tìm hiểu hàm lượng chất béo (đơn vị: g) có trong $100$g mỗi loại thực phẩm. Sau khi thu thập dữ liệu về $60$ loại thực phẩm, An lập được bảng thống kê. Tìm mốt của mẫu số liệu.
	\begin{center}
		\begin{tabular}{|c|c|c|c|c|c|c|}
			\hline Hàm lượng chất béo g & {$[2 ; 6)$} & {$[6 ; 10)$} & {$[10 ; 14)$} & {$[14 ; 18)$} & {$[18 ; 22)$} & {$[22 ; 26)$} \\
			\hline Tần số & $2$ & $6$ & $10$ & $13$ & $16$ & $13$ \\
			\hline
		\end{tabular}
	\end{center}
	\choice
	{$10$}
	{\True $20$}
	{$97$}
	{$95$}
	\loigiai{
		Nhóm chứa mốt của mẫu số liệu trên là nhóm $[18; 22)$.\\
		Do đó $u_m=18$, $n_{m-1}=13$, $n_m=16$, $n_{m+1}=13$, $u_{m+1}-u_m=22-18=4$.\\
		Mốt của mẫu số liệu ghép nhóm là
		$$
		M_o=18+\dfrac{16-13}{(16-13)+(16-13)}\cdot 4=20.
		$$
	}
\end{ex}

\begin{ex}%[Dự án đề cương 3 khối NH24-25 - Đợt 3 - Hải Phụng]%[1D5H2-2]
	Thời gian (phút) truy cập Internet mỗi buổi tối của một số học sinh được cho trong bảng sau
	\begin{center}
		\begin{tabular}{|l|c|c|c|c|c|}
			\hline Thời gian (phút) & $[9{,}5; 12{,}5)$ & $[12{,}5 ; 15{,}5)$ & $[15{,}5 ; 18{,}5)$ & $[18{,}5 ; 21{,}5)$ & $[21{,}5 ; 24{,}5)$ \\
			\hline Số học sinh & $3$ & $12$ & $15$ & $24$ & $2$ \\
			\hline
		\end{tabular}
	\end{center}
	Tính trung vị của mẫu số liệu ghép nhóm này.
	\choice
	{$18{,}3$}
	{\True$18$}
	{\True $18{,}1$}
	{$18{,}2$}
	\loigiai{Cỡ mẫu là $3+12+15+24+2=56$.\\ 
		Gọi $x_1$; $x_2 $; $\ldots$; $x_{56}$ là thời gian truy cập Internet của một số học sinh xếp theo thứ tự không giảm.\\
		Khi đó trung vị là $\dfrac{x_{28}+x_{29}}{2}$.\\
		Do hai giá trị $x_{28}$, $x_{29}$ đều thuộc nhóm $[15{,}5 ; 18{,}5)$ nên 
		nhóm chứa trung vị của mẫu số liệu là $[15{,}5 ; 18{,}5)$.\\
		Trung vị \begin{eqnarray*}
			M_e&=&u_p+\dfrac{\dfrac{n}{2}-\left(n_1+n_2+\ldots +n_{p-1}\right)}{n_p}\cdot (u_{p+1}-u_p)\\
			&=&u_3+\dfrac{\dfrac{56}{2}-\left(n_1+n_2\right)}{n_3}\cdot (u_{4}-u_3)\\
			&=&15{,}5+\dfrac{\dfrac{56}{2}-\left(3+13\right)}{15}\cdot \left(18{,}5-15{,}5\right) \\
			&=&18{,}1.
		\end{eqnarray*}
	}
\end{ex}

\begin{ex}%[Dự án đề cương 3 khối NH24-25 - Đợt 3 - Hải Phụng] %[1D5H2-2]
	Thời gian đi từ nhà đến trường của $56$ học sinh được cho trong bảng sau
	\begin{center}
		\begin{tabular}{|c|c|c|c|c|c|}
			\hline Thời gian (phút) & {$[9{,}5 ; 12{,}5)$} & {$[12{,}5 ; 15{,}5)$} & {$[15{,}5 ; 18{,}5)$} & {$[18{,}5 ; 21{,}5)$} & {$[21{,}5 ; 24{,}5)$} \\
			\hline Số học sinh & 3 & 12 & 15 & 24 & 2 \\
			\hline
		\end{tabular}
	\end{center}
	Tính trung vị của mẫu số liệu ghép nhóm này.
	\choice{$18{,}3$}
	{$18{,}2$}
	{$18$}
	{\True $18{,}1$}
	\loigiai{
		Ta có $\dfrac{56}{2}=28$ nên trung vị thuộc nhóm 	$[15{,}5 ; 18{,}5)$ nên $u_m=15{,}5$, $n_m=15$, $C=15$, $n=56$, $u_{m+1}-u_m=3$ nên
		$$M_e=u_m+\dfrac{\tfrac n2-C}{n_m}\cdot (u_{m+1}-u_m)=15{,}5+\dfrac{\tfrac {56}{2}-15}{15}\cdot (3)=18{,}1.$$
	}
\end{ex}

\begin{ex}%[Dự án đề cương 3 khối NH24-25 - Đợt 3 - Hải Phụng]%[1D5H2-3]
	Cho mẫu số liệu ghép nhóm về chiều cao của $25$ cây dừa giống như sau
	\begin{center}
		\begin{tabular}{|l|c|c|c|c|c|}
			\hline
			Chiều cao $(\mathrm{cm})$ & $[0 ; 10)$ & $[10 ; 20)$ & $[20 ; 30)$ & $[30 ; 40)$ & $[40 ; 50)$ \\
			\hline
			Số cây & $4$ & $6$ & $7$ & $5$ & $3$ \\
			\hline
		\end{tabular}
	\end{center}
	Tứ phân vị thứ nhất của mẫu số liệu ghép nhóm này là
	\choice
	{$Q_{1}=13{,}5$}
	{$Q_{1}=13{,}9$}
	{$Q_{1}=15{,}75$}
	{\True $Q_{1}=13{,}75$}
	\loigiai{
		Cỡ mẫu $n=4+6+7+5+3=25$.\\
		Tứ phân vị thứ nhất $Q_{1}$ là $\dfrac{x_{6}+x_{7}}{2}$. \\
		Do $x_{6}, x_{7}$ đều thuộc nhóm $[10 ; 20)$ nên nhóm này chứa $Q_{1}$.\\
		Do đó $p=2$, $a_{2}=10$, $m_{2}=6$, $m_{1}=4$, $a_{3}-a_{2}=10$.\\
		Ta có $Q_{1}=10+\dfrac{\dfrac{25}{4}-4}{6} \cdot 10=13{,}75$.
	}
\end{ex}

\begin{ex}%[Dự án đề cương 3 khối NH24-25 - Đợt 3 - Hải Phụng]%[1D5H2-3]
	Khảo sát thời gian xem điện thoại trong một ngày của một số học sinh khối 11 thu được mẫu số liệu ghép nhóm sau:\begin{center}
		\begin{tabular}{|c|c|c|c|c|c|}
			\hline Thời gian (phút) & {$[0 ; 20)$} & {$[20 ; 40)$} & {$[40 ; 60)$} & {$[60 ; 80)$} & {$[80 ; 100)$} \\
			\hline Số học sinh & 5 & 9 & 12 & 10 & 6 \\
			\hline
		\end{tabular}
	\end{center}
	Nhóm chứa tứ phân vị thứ nhất là
	\choice{$[60 ; 80)$}
	{$[0 ; 20)$}
	{\True $[20 ; 40)$}
	{$[40 ; 60)$}
	\loigiai{Ta có cỡ mẫu là $42$ khi đó tứ phân vị tứ nhất thuộc nhóm $[20 ; 40)$.
	}
\end{ex}

\begin{ex}%[Dự án đề cương 3 khối NH24-25 - Đợt 3 - Hải Phụng]%[1D5H2-3]
	Thời gian (phút) truy cập Internet mỗi buổi tối của một số học sinh được cho trong bảng sau
	\begin{center}
		\begin{tabular}{|c|c|c|c|c|c|} 
		\hline Thời gian(giờ) & {$[9,5;12,5)$} & {$[12,5;15,5)$} & {$[15,5;18,5)$} & {$[18,5;21,5)$} & {$[21,5;24,5)$}\\ 
		\hline Số học sinh & $3$ & $12$ & $15$ & $24$ & $2$\\ 
		\hline 
	\end{tabular} 
	\end{center}
	Tứ phân vị thứ nhất của mẫu số liệu ghép nhóm ở trên là 
	\choice
	{$Q_1=18{,}1$}
	{\True $Q_1=15{,}25$}
	{$Q_1=15{,}57$}
	{$Q_1=20$}
	\loigiai{
		Từ bảng số liệu 
		\begin{center}
			\begin{tabular}{|l|c|c|c|c|c|}
				\hline
				Dòng 1 & $[9{,}5; 12{,}5)$ & $[12{,}5; 15{,}5)$ & $[15{,}5; 18{,}5)$ & $[18{,}5; 21{,}5)$ & $[21{,}5; 24{,}5)$\\
				\hline
				Tần Số &$ 3$ & $12$ & $15$ & $24$ & $2$\\
				\hline
			\end{tabular}
		\end{center}
		Tứ phân vị thứ nhất $Q_1$.\\ 
		Cỡ mẫu là $n=56$.\\ 
		Gọi $x_1;x_2;\ldots;x_{56}$ là mẫu số liệu được sắp xếp theo thứ tự không giảm.\\ 
		Tứ phân vị thứ nhất  $Q_1$ là $\dfrac{x_{14}+x_{15}}{2}$.\\ 
		Do $\dfrac{x_{14}+x_{15}}{2}$ thuộc nhóm $[12{,}5;15{,}5)$ nên nhóm này chứa $Q_1$.\\ 
		Do đó $p=2; a_2=12{,}5; m_2=12; m_1=3; a_3-a_2=3{,}0$ và ta có $Q_1=12{,}5+\dfrac{\tfrac{56}{4}-3}{12}\cdot 3=15{,}25$.		
	}
\end{ex}

\begin{ex}%[Dự án đề cương 3 khối NH24-25 - Đợt 3 - Hải Phụng]%[1D5H2-3]
	Doanh thu bán hàng trong $20$ ngày được lựa chọn ngẫu nhiên cửa một của hàng được ghi lại ở bảng sau (đơn vị: triệu đồng):
	\begin{center}
		\begin{tabular}{|c|c|c|c|c|c|}
			\hline Doanh thu & {$[5 ; 7)$} & {$[7 ; 9)$} & {$[9 ; 11)$} & {$[11 ; 13)$} & {$[13 ; 15)$}\\
			\hline Số ngày & $2$ & $7$ & $7$ & $3$ & $1$\\
			\hline
		\end{tabular}
	\end{center}
	Tứ phân vị thứ nhất của mẫu số liệu gần nhất với giá trị nào trong các giá trị dưới đây?
	\choice
	{$8{,}6$}
	{$8$}
	{$7{,}6$}
	{\True $7$}
	\loigiai{
		Gọi $x_1;x_2;\ldots;x_{20}$ là doanh thu bán hàng của $20$ ngày sắp xếp theo thứ tự không giảm\\
		Do $x_1,x_2\in\left[5;7\right)$; $x_3,\ldots,x_9\in\left[7;9\right)$; $x_{10},\ldots,x_{16}\in\left[9;11\right)$; $x_{17},x_{18},x_{19}\in\left[11;13\right)$; $x_{20}\in\left[13;15\right)$.\\
		Tứ phân vị thứ nhất của dãy số liệu $x_1;x_2;\ldots;x_{20}$ là $\dfrac{1}{2}\left(x_5+x_6\right)$.\\
		Do $x_5$ và $x_6$ đều thuộc nhóm $\left[7;9\right)$ nên tứ phân vị thứ nhất của mẫu số liệu là \[Q_1=7+\dfrac{\dfrac{10}{4}-2}{7}\cdot\left(9-7\right)\approx7{,}14.\]
	}
\end{ex}
\Closesolutionfile{ans}
%\begin{center}
%	\textbf{ĐÁP ÁN}
%	\inputansbox{12}{ans/ans-TN-ONTAPCHUONG9-DE1}	
%\end{center}


\begin{center}
	\textbf{PHẦN 2 - CÂU TRẮC NGHIỆM ĐÚNG SAI}
\end{center}
\setcounter{ex}{0}
\Opensolutionfile{ans}[ans/ans-DS-ONTAPCHUONG9-DE1]

\begin{ex}%[Dự án đề cương 3 khối NH24-25 - Đợt 3 - Hải Phụng]%[1D5H1-3]
	Khảo sát thời gian tập thể dục của một số học sinh khối $11$ thu được mẫu số liệu ghép nhóm sau
	\begin{center}
		\begin{tabular}{|c|c|c|c|c|c|}
			\hline
			Thời gian (phút)	&$[0;20)$  &$[20;40)$  &$[40;60)$  &$[6;80)$  &$[80;100)$  \\
			\hline
			Số học sinh	&$5$  &$9$  &$12$  &$10$  &$6$  \\
			\hline
		\end{tabular}
	\end{center}
	Xét tính đúng - sai của các khẳng định sau:
	\choiceTF
	{\True Tổng số học sinh được khảo sát là $42$ học sinh}
	{Giá trị đại diện của nhóm $[20;40)$ là $25$}
	{\True Có $16$ học sinh tập thể dục ít nhất $1$ giờ trong ngày}
	{Số trung bình của mẫu số liệu trên thuộc nhóm $[20;40)$}
	\loigiai{
		\begin{itemchoice}
			\itemch Tổng số học sinh được khảo sát là $n=5+9+12+10+6=42$.
			\itemch Giá trị đại diện của nhóm $[20;40)$ là $\dfrac{20+40}{2}=30$.
			\itemch Số học sinh tập thể dục ít nhất $1$ giờ trong ngày là $10+6=16$ học sinh.
			\itemch Ta có
			\begin{center}
				\begin{tabular}{|c|c|c|c|c|c|}
					\hline
					Thời gian (phút)	&$[0;20)$  &$[20;40)$  &$[40;60)$  &$[6;80)$  &$[80;100)$  \\
					\hline
					Giá trị đại diện &10&30&50&70&90\\
					\hline
					Tần số	&$5$  &$9$  &$12$  &$10$  &$6$  \\
					\hline
				\end{tabular}
			\end{center}
			Cỡ mẫu là $n=5+9+12+10+6=42$.\\
			Thời gian trung bình học sinh tập thể dục là
			$$\overline{x} = \dfrac{5\cdot 10 + 9 \cdot 30 + 12 \cdot 50 + 10\cdot 70 + 6\cdot 90}{42} = \dfrac{360}{7} \in [40;60).$$
		\end{itemchoice}
	}
\end{ex}

\begin{ex}%[Dự án đề cương 3 khối NH24-25 - Đợt 3 - Hải Phụng]%[1D5H2-3]
	Thống kê cân nặng của $40$ học sinh lớp $11$A trong một trường THPT (đơn vị: kilôgam) được cho ở bẳng sau 
	\begin{center}
		\begin{tabular}{|c|c|c|}
			\hline
			Nhóm	&Tần số  & Tần số tích lũy \\
			\hline
			$[30;40)$	&$4$  &$4$\\
			\hline
			$[40;50)$	&$10$  &$14$\\
			\hline
			$[50;60)$	&$18$  &$32$\\
			\hline
			$[60;70)$	&$10$  &$42$\\
			\hline
			$[70;80)$	&$2$  &$44$\\
			\hline
		\end{tabular}
	\end{center}
	Xét tính đúng - sai của các khẳng định sau:
	\choiceTF
	{Số học sinh cân nặng từ $40$ kg đến dưới $50$ kg là $18$}
	{\True $[50;60)$ là nhóm chứa mốt}
	{\True Cân nặng trung bình của các bạn học sinh trong lớp $11$A là $54$ kg}
	{Tứ phân vị thứ nhất của mẫu số liệu ghép nhóm trên là $50$ kg}
	\loigiai{
		\begin{itemchoice}
			\itemch Số học sinh có cân nặng từ $40$ kg đến dưới $50$ kg là $10$.
			\itemch Nhóm chứa mốt là nhóm có tần số lớn nhất nên $[50;60)$ là nhóm chứa mốt.
			\itemch Cân nặng của học sinh lớp $11$A là
			$$\overline{x} = \dfrac{4\cdot 35 + 10\cdot 45 + 18 \cdot 55 + 10\cdot 65 + 2 \cdot 75}{44} \approx 54 \, \text{(kg)}.$$
			\itemch Số phần tử của mẫu là $n=44$.\\
			Ta có $\dfrac{n}{4} = \dfrac{44}{4} = 11$.\\ Mà $14>11$ suy ra nhóm $2$ là nhóm đầu tiên có tần số tích lũy lớn hơn hoặc bằng $11$.\\
			Xét nhóm $2$ là nhóm $[40;50)$ có $s=40$; $h=10$; $n_2=10$ và nhóm $1$ là nhóm có $cf_1 = 4$.\\
			Ta có tứ phân vị thứ nhất là $Q_1 = 40 + \dfrac{11-4}{10} \cdot 10 = 47$ (kg).
		\end{itemchoice}
	}
\end{ex}
\Closesolutionfile{ans}
%\inputansbox[2]{2}{ans/ans-DS-ONTAPCHUONG9-DE1}

\begin{center}
	\textbf{PHẦN 3 - CÂU TRẮC NGHIỆM TRẢ LỜI NGẮN}
\end{center}
\setcounter{ex}{0}
\Opensolutionfile{ans}[ans/ans-KQ-ONTAPCHUONG9-DE1]

\begin{ex}%[Dự án đề cương 3 khối NH24-25 - Đợt 3 - Hải Phụng]%[1D5H1-4]
	Một người thống kê lại thời gian thực hiện các cuộc gọi điện thoại của người đó trong một tuần ở bảng sau
	\begin{center}
		\begin{tabular}{|c|c|c|c|c|c|c|}
			\hline Thời gian (giây) & {$[0 ; 60)$} & {$[60 ; 120)$} & {$[120 ; 180)$} & {$[180 ; 240)$} & {$[240 ; 300)$} & {$[300 ; 360)$} \\
			\hline Số cuộc gọi & 8 & 10 & 7 & 5 & 2 & 1 \\
			\hline
		\end{tabular}
	\end{center}
	Xác định mốt của mẫu số liệu ghép nhóm trên (làm tròn đến hàng đơn vị).
	\par\shortans[oly]{$84$}
	\loigiai{
		Ta có bảng thống kê thời gian thực hiện các cuộc gọi điện thoại theo giá trị đại diện
		\begin{center}
			\newcolumntype{L}{>{\raggedright\arraybackslash}m{3cm}}
			\newcolumntype{M}{>{\centering\arraybackslash}m{1cm}}
			\begin{tabular}{|L|M|M|M|M|M|M|}
				\hline 	Thời gian (giây) & 30 & 90 & 150 & 210 & 270 & 330 \\
				\hline Số cuộc gọi & 8 & 10 & 7 & 5 & 2 & 1 \\
				\hline
			\end{tabular}
		\end{center}
		Nhóm chứa mốt của mẫu số liệu trên là nhóm $\left[60 ; 120\right)$.\\
		Mốt của mẫu số liệu ghép nhóm là $M_O=60+\dfrac{10-8}{(10-8)+(10-7)} \cdot 60=84$.}
\end{ex}

\begin{ex}%[Dự án đề cương 3 khối NH24-25 - Đợt 3 - Hải Phụng]%[1D5H1-3]
	Bảng thống kê sau cho biết thời gian sử dụng điện thoại một ngày (đơn vị: giờ) của bạn Bình trong $120$ ngày.
	\begin{center}
		\begin{tabular}{|c|c|c|c|c|c|}
			\hline
			Thời gian (giờ)& $[0;4)$ & $[4;8)$ & $[8;12)$ & $[12;16)$ & $[16;20)$\\
			\hline
			Số ngày & $12$ & $65$ & $30$ & $9$ & $4$\\
			\hline
		\end{tabular}
	\end{center}
	Tính giá trị trung bình của mẫu số liệu trên.
	\par
	\shortans[oly]{$7{,}6$}
	\loigiai{
		Bảng giá trị
		\begin{center}
			\begin{tabular}{|c|c|c|c|c|c|}
				\hline
				Thời gian (giờ)& $[0;4)$ & $[4;8)$ & $[8;12)$ & $[12;16)$ & $[16;20)$\\
				\hline
				Giá trị đại diện & $2$ & $6$ & $10$ & $14$ & $18$\\
				\hline
				Số ngày & $12$ & $65$ & $30$ & $9$ & $4$\\
				\hline
			\end{tabular}
		\end{center}
		Giá trị trung bình
		\[\overline{x}=\dfrac{2\cdot 12 + 6\cdot 65 + 10\cdot 30 + 14\cdot 9 +18\cdot 4}{120}=\dfrac{38}{5}= 7{,}6.\]	
	}
\end{ex}

\begin{ex}%[Dự án đề cương 3 khối NH24-25 - Đợt 3 - Hải Phụng]%[1D5V2-3]
	Khảo sát thời gian tập thể dục của một số học sinh khối 11 thu được mẫu số liệu ghép nhóm sau:
	\begin{center}
		\begin{tabular}{|l|c|c|c|c|c|}
			\hline Thời gian (phút) & {$[0; 20)$} & {$[20; 40)$} & {$[40; 60)$} & {$[60; 80)$} & {$[80; 100)$} \\
			\hline Số học sinh & $5$ & $9$ & $12$ & $10$ & $6$ \\
			\hline
		\end{tabular}
	\end{center}
	Tứ phân vị thứ ba của mẫu số liệu là bao nhiêu?
	
	\shortans[oly]{$71$}
	\loigiai{
		Cỡ mẫu là $n=5+9+12+10+6=42$.\\
		Tứ phân vị thứ ba $Q_3$ là $x_{33}$, do nhóm chứa $x_{33}$ là $[60; 80)$, do đó $p=4$; $a_4=60$; $m_1+m_2+m_3=5+9+12=26$; $a_5-a_4=20$.\\
		Ta có $Q_3=a_4+\dfrac{\dfrac{3 n}{4}-\left(m_1+m_2+m_3\right)}{m_4}\left(a_5-a_4\right)=60+\dfrac{\dfrac{3.42}{4}-26}{10} \cdot 20=71$.
	}
\end{ex}

\begin{ex}%[Dự án đề cương 3 khối NH24-25 - Đợt 3 - Hải Phụng]%[1D5H2-2]
	Mẫu số liệu dưới đây ghi lại tốc độ của $40$ ô tô khi đi qua một trạm đo tốc độ (đơn vị: km/h).
	\begin{center}
		\begin{tabular}{cccccccccc}
			$49$ & $42$ & $51$ & $55$&$45$& $60$ & $53$ & $55$ & $44$ & $65$\\
			$52$ & $62$ & $41$ & $44$&$57$& $56$ & $68$ & $48$ & $46$ & $53$\\
			$63$ & $49$ & $54$ & $61$&$59$& $57$ & $47$ & $50$ & $60$ & $62$\\
			$48$ & $52$ & $58$ & $47$&$60$& $55$ & $45$ & $47$ & $48$ & $61$\\
		\end{tabular}
	\end{center}
	Sau khi ghép nhóm mẫu số liệu trên thành sáu nhóm tương ứng với sáu nửa khoảng: $[40;45)$, $[45;50)$, $[50;55)$, $[55;60)$, $[60;65)$, $[65;70)$ thì số trung vị của mẫu số liệu ghép nhóm bằng $\dfrac{a}{b}$ km/h với  $\dfrac{a}{b}$ là phân số tối giản. Giá trị của $a$ bằng bao nhiêu?
	
	\shortans[oly]{$375$}
	\loigiai{
		Ta có bảng số liệu ghép nhóm của mẫu số liệu trên là
		\begin{center}
			\begin{tabular}{|c|c|c|}
				\hline
				\textbf{Nhóm} & \textbf{Tần số} & \textbf{Tần số tích luỹ}\\
				\hline
				$[40;45)$ & $4$ & $4$ \\
				$[45;50)$ & $11$ & $15$ \\
				$[50;55)$ & $7$ & $22$ \\
				$[55;60)$ & $8$ & $30$ \\
				$[60;65)$ & $8$ & $38$ \\
				$[65;70)$ & $2$ & $40$ \\
				\hline
				&$n=40$ &\\
				\hline
			\end{tabular}\\
			\textit{Bảng $8$}
		\end{center}
		Số phần tử của mẫu là $n=40$.\\
		 Ta có $\dfrac{n}{2}=20$, lại có $15<20<22$ nên nhóm $3$ là nhóm đầu tiên có tần số tích luỹ lớn hơn hoặc bằng $20$.\\
		Xét nhóm $3$ có $r=50$; $d=5$; $n_3=7$ và nhóm $2$ có $cf_2=15$.\\
		Trung vị của mẫu số liệu ghép nhóm là $$M_e=50+\dfrac{20-15}{7}\cdot 5 = \dfrac{375}{7} \text{ km/h.}$$
		Suy ra $a=375$.
	}	
\end{ex}
\Closesolutionfile{ans}
%\inputansbox[3]{5}{ans/ans-KQ-ONTAPCHUONG9-DE1}

\begin{center}
	\textbf{PHẦN 4 - TỰ LUẬN}
\end{center}

\begin{bt}%[Dự án đề cương 3 khối NH24-25 - Đợt 3 - Hải Phụng]%[1D5H1-3]
	Dựa vào điểm trung bình cuối năm học $2022-2023$ của $487$ học sinh khối $11$ đạt danh hiệu \lq\lq Học sinh giỏi cả năm\rq\rq, ta thu được bảng số liệu sau
	\begin{center}
		\begin{tabular}{|c|c|c|c|c|c|c|c|c|}
			\hline \begin{tabular}{c} 
				{\bf Điểm trung bình}
			\end{tabular} & {$[8.0;8.2)$} & {$[8.2;8.4)$} & {$[8.4 ;8.6)$} & {$[8.6;8.8)$} & {$[8.8;9.0)$} & {$[9.0; 9.2)$} & {$[9.2;9.4)$} & {$[9.4;9.6)$} \\
			\hline \begin{tabular}{c} 
				{\bf Số học sinh}
			\end{tabular} & $12$ & $31$ & $66$ & $107$ & $114$ & $93$ & $51$ & $13$ \\
			\hline
		\end{tabular}
	\end{center}
	\begin{enumEX}[a)]{1}
		\item Hãy ước lượng điểm trung bình cả năm của $487$ học sinh trên.
		\item Hãy dự đoán điểm trung bình cả năm mà số học sinh đạt được là nhiều nhất.
	\end{enumEX}
	\loigiai{
		\begin{enumEX}[a)]{1}
			\item Điểm trung bình cả năm của $487$ học sinh trên bằng
			$$\dfrac{8{,}1\cdot 12+8{,}3\cdot 31+8{,}5\cdot 66+8{,}7\cdot 107+8{,}9\cdot 114+9{,}1\cdot 93+9{,}3\cdot 51+9{,}5\cdot 13}{487}\approx 8{,}84.$$
			\item Nhóm chứa mốt của mẫu số liệu trên là nhóm $[8{,}8;9{,}0)$.\\
			Do đó $u_m=8{,}8$, $n_{m-1}=107$, $n_m=114$, $n_{m+1}=93$, $u_{m+1}-u_m=0{,}2$.\\
			Mốt của mẫu số liệu trên là
			$$M_o=8{,}8+\dfrac{114-107}{(114-107)+(114-93)}\cdot 0{,}2=8{,}85$$
			Do vậy điểm trung bình cả năm mà số học sinh đạt được nhiều nhất là $8{,}8$.
		\end{enumEX}
	}
\end{bt}

\begin{bt}%[Dự án đề cương 3 khối NH24-25 - Đợt 3 - Hải Phụng]%[1D5H2-3]
	An tìm hiểu hàm lượng chất béo (đơn vị: g) có trong $100\,$g mỗi loại thực phẩm. Sau khi thu thập dữ liệu về $60$ loại thực phẩm, An lập bảng thống kê sau
	\begin{center}
		\begin{tabular}{|c|c|c|c|c|c|c|}
			\hline
			Hàm lượng chất béo (g) &$[2;6)$&$[6;10)$&$[10;14)$&$[14;18)$&$[18;22)$&$[22;26)$\\
			\hline
			Tần số&$2$&$6$&$10$&$13$&$16$&$13$\\
			\hline
		\end{tabular}
	\end{center}
	Xác định giá trị trung bình, trung vị, tứ phân vị của mẫu số liệu.
	\loigiai{
		\begin{center}
			\begin{tabular}{|c|c|c|c|c|c|c|}
				\hline
				Hàm lượng chất béo (g) &$[2;6)$&$[6;10)$&$[10;14)$&$[14;18)$&$[18;22)$&$[22;26)$\\
				\hline
				Giá trị đại diện&$4$&$8$&$12$&$16$&$20$&$24$\\
				\hline
				Tần số&$2$&$6$&$10$&$13$&$16$&$13$\\
				\hline
			\end{tabular}
		\end{center}	
		Gọi $x_1, x_2,\ldots,x_{60}$ là hàm lượng chất béo của $60$ loại thực phẩm được xếp theo thứ tự không giảm.\\
		Giá trị trung bình $\bar{x}=\dfrac{4\cdot 2+8\cdot 6+12\cdot 10+16\cdot 13+20\cdot 16+24\cdot 13}{60}=16{,}9$.\\
		Do $x_1,x_2\in [2;6)$; $x_3,\ldots,x_8\in [6;10)$; $x_9$,\ldots,$x_{18}\in [10;14)$; $x_{19}$,\ldots,$x_{31}\in [14;18)$; $x_{32}$,\ldots,$x_{47}\in [18;22)$, $x_{48}$,\ldots,$x_{60}\in [22;26)$ nên trung vị của mẫu số liệu là $\dfrac{1}{2} \left(x_{30}+x_{31}\right)\in [14;18)$.\\
		Ta xác định được $n=60$, $n_m=13$, $C=18$, $u_m=14$, $u_{m+1}=18$.\\
		Trung vị $M_e=u_m+\dfrac{\dfrac{n}{2}-C}{n_m}\left(u_{m+1}-u_m \right)=14+\dfrac{\dfrac{60}{2}-18}{13}\cdot (18-14)=17{,}7$.\\
		Tứ phân vị thứ hai $Q_2=M_e=17{,}7$.\\
		Tứ phân vị thứ nhất là $\dfrac{1}{2}\left(x_{15}+x_{16} \right)\in [10;14)\Rightarrow Q_1=10+\dfrac{\dfrac{60}{4}-8}{10}\left(14-10\right)=12{,}8$.\\
		Tứ phân vị thứ ba là $\dfrac{1}{2}\left(x_{45}+x_{46}\right)\in[18;22)\Rightarrow Q_3=18+\dfrac{\dfrac{3}{4}.60-31}{16}\left(22-18\right)=21{,}5$.
	}
\end{bt}

\begin{bt}%[Dự án đề cương 3 khối NH24-25 - Đợt 3 - Hải Phụng]%[1D5H2-3]
Thống kê điểm trung bình môn Toán của một số học sinh lớp $11$ được cho ở bảng sau:
	\begin{center}
		\begin{tabular}{|c|c|c|c|c|c|c|c|}
			\hline
			Khoảng điểm	& $\left[6{,}5;\,7\right)$ & $\left[7;\,7{,}5\right)$ & $\left[7{,}5;\,8\right)$ & $\left[8;\,8{,}5\right)$ & $\left[8{,}5;\,9\right)$ & $\left[9;\,9{,}5\right)$ & $\left[9{,}5;\,10\right)$ \\
			\hline
			Tần số	& 8 & 13 & 14 & 19 & 9 & 11 & 10 \\
			\hline
		\end{tabular}
	\end{center}
	Hãy ước lượng số trung bình, tứ phân vị và mốt của mẫu số liệu ghép nhóm trên
	(làm tròn đến phần trăm). 
	\loigiai{
		\begin{center}
			\begin{tabular}{|c|c|c|c|c|c|c|c|}
				\hline
				Khoảng điểm	& $\left[6{,}5;\,7\right)$ & $\left[7;\,7{,}5\right)$ & $\left[7{,}5;\,8\right)$ & $\left[8;\,8{,}5\right)$ & $\left[8{,}5;\,9\right)$ & $\left[9;\,9{,}5\right)$ & $\left[9{,}5;\,10\right)$ \\
				\hline
				Tần số	& 8 & 13 & 14 & 19 & 9 & 11 & 10 \\
				\hline
				Giá trị đại diện	& $6{,}75$ & $7{,}25$ & $7{,}75$ & $8{,}25$ & $8{,}75$ & $9{,}25$ & $9{,}75$ \\
				\hline
			\end{tabular}
		\end{center}
		Số trung bình $\overline{x}=\dfrac{461}{56}\approx8{,}23$.\\
		Nhóm chứa mốt của mẫu số liệu trên là nhóm $\left[8;\,8{,}5\right)$.\\
		Mốt: $M_0=8+\dfrac{19-14}{19-14+19-9}\cdot0{,}5=\dfrac{49}{6}\approx8{,}17$.\\
		Tứ phân vị:\\
		Tứ phân vị thứ hai của dãy số liệu là $\dfrac{1}{2}\left(x_{42}+x_{43}\right)$ thuộc nhóm $\left[8;\,8{,}5\right)$.\\
		$Q_2=M_e=8+\dfrac{\dfrac{84}{2}-35}{19}\cdot0{,}5=\dfrac{311}{38}\approx8{,}18$.\\
		Tứ phân vị thứ nhất của dãy số liệu là $\dfrac{1}{2}\left(x_{21}+x_{22}\right)$.\\ Do $x_{21}\in\left[7;7{,}5\right)$;  $x_{22}\in\left[7{,}5;8\right)$ nên tứ phân vị thứ nhất của mẫu số liệu ghép nhóm là $Q_1=7,5$.\\
		Tứ phân vị thứ ba của dãy số liệu là $\dfrac{1}{2}\left(x_{63}+x_{64}\right)$.\\ Do $x_{63}\in\left[8{,}5;9\right)$;  $x_{64}\in\left[9;9{,}5\right)$ nên tứ phân vị thứ ba của mẫu số liệu ghép nhóm là $Q_3=9$.
	}
\end{bt}
\begin{center}
	\textbf{ĐÁP ÁN}
\end{center}
\inputansbox{12}{ans/ans-TN-ONTAPCHUONG9-DE1}
\inputansbox[2]{2}{ans/ans-DS-ONTAPCHUONG9-DE1}
\inputansbox[3]{5}{ans/ans-KQ-ONTAPCHUONG9-DE1}