\newpage
\def\thoigian{90}%--Thời gian
\de{Đề số 2}{Chương III. Giới hạn. Hàm số liên tục}



\begin{center}
	\textbf{PHẦN 1 - CÂU TRẮC NGHIỆM BỐN PHƯƠNG ÁN}
\end{center}
\Opensolutionfile{ans}[ans/1D3-OTC-D1]
\begin{ex}%[1D3H1-2]
	[Trích đề thi giữa học kỳ 1 lớp 11-THPT Nguyễn Quốc Trình-Hà Nội-Năm học 2024-2025]
	Giá trị của $\lim \limits_{n \rightarrow+\infty} \dfrac{n^{2}+2 n+2024}{2025-2008 n^{2}}$ bằng
	\choice
	{$\dfrac{1}{2025}$}
	{\True $-\dfrac{1}{2008}$}
	{$\dfrac{2024}{2025}$}
	{$0$}
	\loigiai{
		Ta có  $\lim \limits_{n \rightarrow+\infty} \dfrac{n^{2}+2 n+2024}{2025-2008 n^{2}}=\lim \limits_{n \rightarrow+\infty} \dfrac{1+\dfrac{2}{n}+\dfrac{2024}{n^2}}{\dfrac{2025}{n^2}-2008}=-\dfrac{1}{2008}$.
	}
\end{ex}
\begin{ex}%[1D3N1-1]
	[Trích đề thi giữa học kỳ 1 lớp 11-THPT Nguyễn Quốc Trình-Hà Nội-Năm học 2024-2025]
	Giá trị của $\lim \limits_{n \rightarrow+\infty}\left(\dfrac{1}{3}\right)^{n}$ là
	\choice
	{$+\infty$}
	{$-\infty$}
	{\True $0$}
	{$\dfrac{1}{3}$}
	\loigiai{
		Vì $\dfrac{1}{3}<1$ nên $\lim \limits_{n \rightarrow+\infty}\left(\dfrac{1}{3}\right)^{n}=0$.
	}
\end{ex}
\begin{ex}%[1D3N1-2]
	[Trích đề thi học kỳ 1 lớp 11-THPT Phạm Phú Thứ - Đà Nẵng]
	Cho dãy số $\left(u_n\right)$ với $u_n=\dfrac{2 n^2-n}{n^2+1}$. Tính $\lim\limits_{n \rightarrow+\infty} u_n$.
	\choice
	{\True $2$}
	{$\dfrac{1}{2}$}
	{$-1$}
	{$1$}
	\loigiai{
		Ta có
		$$
		\lim\limits_{n \rightarrow+\infty} u_n = \lim _{n \rightarrow+\infty} \dfrac{2 n^2-n}{n^2+1} = \lim _{n \rightarrow+\infty} \dfrac{2 - \dfrac{1}{n}}{1 + \dfrac{1}{n^2}} = \dfrac{2-0}{1+0} = 2.
		$$
		Vậy $\lim\limits _{n \to+\infty} u_n = 2$.
	}
\end{ex}
\begin{ex}%[1D3N1-1]
	[Trích đề thi giữa học kỳ 2 lớp 11-THPT Chuyên Nguyễn Tất Thành - KonTum-Năm học 2023-2024]
	Cho hai dãy số $(u_n)$ và $(v_n)$ thỏa mãn $\lim\limits_{n\to +\infty} u_n=-1$, $\lim\limits_{n\to +\infty} v_n=+\infty$. Giới hạn $\lim\limits_{n\to +\infty}\dfrac{u_n}{v_n}$ bằng
	\choice
	{\True $0$}
	{$+\infty$}
	{$-\infty$}
	{$-1$}
	\loigiai{
		Vì $\lim\limits_{n\to +\infty} u_n=-1$, $\lim\limits_{n\to +\infty} v_n=+\infty$ nên $\lim\limits_{n\to +\infty}\dfrac{u_n}{v_n}=0$.
	}
\end{ex}
\begin{ex}%[1D3N2-1]
	[Trích đề thi học kỳ 1 lớp 11-THPT Chu Văn An- Quãng Nam-Năm học 2023-2024]
	Cho hàm số $y = f(x)$ có giới hạn hữu hạn khi $x$ dần tới $x_0$. Mệnh đề nào sau đây \textbf{sai}?
	\choice
	{$\lim\limits_{x \to x_0} f(x) = L$}
	{$\lim\limits_{x \to x_0} c = c$ (với $c$ là hằng số)}
	{$\lim\limits_{x \to x_0} x = x_0$}
	{\True $\lim\limits_{x \to +\infty} f(x) = L$}
	\loigiai{Nếu $x$ dần tới $x_0$ thì hàm số $y=f(x)$ có giới hạn hữu hạn nên mệnh đề sai là $\lim\limits_{x \to +\infty} f(x) = L$.
	}
\end{ex}

\begin{ex}%[1D3N2-3]
	[Trích đề thi học kỳ 2 lớp 11-THPT Lý Thái Tổ - Bắc Ninh-Năm học 2024-2025]
	Giới hạn $\lim\limits_{x \to-\infty} \dfrac{4 x-5}{1-2x}$ bằng
	\choice
	{$4$}
	{$-4$}
	{\True $-2$}
	{$2$}
	\loigiai
	{
		Ta có $\lim\limits_{x \to-\infty} \dfrac{4 x-5}{1-2x}=\lim \limits_{x \to -\infty} \dfrac{4 -\dfrac{5}{x}}{\dfrac{1}{x}-2}=-2$.
	}
\end{ex}

\begin{ex}%[1D3N2-1]
	[Trích đề thi giữa học kỳ 1 lớp 11-THPT Nguyễn Quốc Trình-Hà Nội-Năm học 2024-2025]
	Cho các giới hạn $\lim \limits_{x \rightarrow x_{0}} f(x)=2$; $\lim \limits_{x \rightarrow x_{0}} g(x)=3$. Tính $\lim \limits_{x \rightarrow x_{0}}\left[3 f(x)+4 g(x)\right]$.
	\choice
	{$-6$}
	{$5$}
	{\True $18$}
	{$17$}
	\loigiai{
		Ta có $\lim \limits_{x \rightarrow x_{0}}\left[3 f(x)+4 g(x)\right]=3\lim \limits_{x \rightarrow x_{0}} f(x)+4\lim \limits_{x \rightarrow x_{0}} g(x)=3\cdot2+4\cdot3=18$.
	}
\end{ex}
\begin{ex}%[1D3H2-1]
	[Trích đề thi học kỳ 1 lớp 11-THPT Kon-Tum -Năm học 2024-2025]
	Nếu $\lim\limits_{x \to 3} f(x) = 5$ thì $\lim\limits_{x \to 3}\left[f(x) - 2\right]$ bằng
	\choice
	{$-1$}
	{\True $3$}
	{$1$}
	{$-3$}
	\loigiai{
		Ta có $\lim\limits_{x \to 3}\left[f(x) - 2\right] = \lim\limits_{x \to 3} f(x) -  \lim\limits_{x \to 3} 2 = 3$.
	}
\end{ex}
\begin{ex}%[1D3N3-3]
	[Trích đề thi học kỳ 1 lớp 11-THPT Chu Văn An- Quãng Nam-Năm học 2023-2024]
	Cho hàm số $y = f(x)$ xác định trên $(m; n)$, $a \in (m; n)$. Phát biểu nào sau đây là đúng?
	\choice
	{\True Hàm số $y = f(x)$ liên tục tại $x = a$ khi và chỉ khi $\lim\limits_{x \to a} f(x) = f(a)$}
	{Hàm số $y = f(x)$ liên tục tại $x = a$ khi và chỉ khi $\lim\limits_{x \to n} f(x) = f(a)$}
	{Hàm số $y = f(x)$ liên tục tại $x = a$ khi và chỉ khi $\lim\limits_{x \to m} f(x) = f(a)$}
	{Hàm số $y = f(x)$ liên tục tại $x = a$ khi và chỉ khi $\lim\limits_{x \to a^+} f(x) = \lim\limits_{x \to a^-} f(x)$}
	\loigiai{Theo định nghĩa, hàm số $y = f(x)$ liên tục tại $x = a$ khi và chỉ khi $\lim\limits_{x \to a} f(x) = f(a)$.
	}
\end{ex}

\begin{ex}%[1D1N4-1]
	[Trích đề thi học kỳ 1 lớp 11-THPT Phạm Phú Thứ - Đà Nẵng-Năm học 2023-2024]
	Trong các hàm số được cho dưới đây, hàm số nào liên tục trên~$\mathbb{R}$?
	\choice
	{$y=\tan x$}
	{$y=\dfrac{1}{x}$}
	{\True $y=x^2+1$}
	{$y=\sqrt{x}$}
	\loigiai{
		\begin{itemize}
			\item $y = \tan x$ có tập xác định $\mathscr{D} = \mathbb{R} \setminus \left\{ \dfrac{\pi}{2} + k\pi \mid k \in \mathbb{Z} \right\}$. \\
			Do đó, hàm số không liên tục trên $\mathbb{R}$.
			\item $y = \dfrac{1}{x}$ có tập xác định $\mathscr{D} = \mathbb{R} \setminus \{0\}$. \\
			Do đó, hàm số không liên tục trên $\mathbb{R}$.
			\item $y = x^2 + 1$ là hàm đa thức, có tập xác định $\mathscr{D} = \mathbb{R}$.\\
			 Hàm đa thức liên tục trên tập xác định của nó, do đó hàm số liên tục trên $\mathbb{R}$.
			\item $y = \sqrt{x}$ có tập xác định $\mathscr{D} = [0; +\infty)$. \\
			Do đó, hàm số không liên tục trên $\mathbb{R}$.
		\end{itemize}
		Vậy hàm số $y=x^2+1$ liên tục trên $\mathbb{R}$.
	}
\end{ex}
\begin{ex}%[1D3N3-1]
	[Trích đề thi học kỳ 1 lớp 11-THPT Phạm Phú Thứ - Đà Nẵng-Năm học 2023-2024]
	Hàm số $y=x^2+\sqrt{x-2}$ gián đoạn tại điểm nào sau đây?
	\choice
	{$x_0=4$}
	{\True $x_0=0$}
	{$x_0=2$}
	{$x_0=3$}
	\loigiai{
		Hàm số $y=x^2$ liên tục trên $\mathbb{R}$.\\
		Hàm số $y=\sqrt{x-2}$ có tập xác định $\mathscr{D}=[2;+\infty)$.\\
		Vậy hàm số $y=x^2+\sqrt{x-2}$ có tập xác định $\mathscr{D}=[2;+\infty)$.\\
		Ta có $\lim\limits_{x\to 2}f(x)=\lim\limits_{x\to 2^+}f(x)=4$.\\
		Nên hàm liên tục tại $x_0=2$.\\
		Vậy hàm số gián đoạn tại $x_0=0$.
	}
\end{ex}
\begin{ex}%[1D3N3-3]
	[Trích đề thi học kỳ 1 lớp 11-THPT Chu Văn An- Quãng Nam-Năm học 2023-2024]
	Cho hàm số $f(x) = \dfrac{x^2 - 6x}{x + 2}$. Hàm số $f(x)$ gián đoạn tại điểm nào dưới đây?
	\choice
	{$x = 1$}
	{\True $x = -2$}
	{$x = 2$}
	{$x = -3$}
	\loigiai{Tập xác định $\mathscr{D}=\mathbb{R}\setminus \left\lbrace -2 \right\rbrace$. \\
		Do đó hàm số $f(x)$ gián đoạn tại điểm $x=-2$.
	}
\end{ex}

\Closesolutionfile{ans}
%\begin{center}
%	\textbf{ĐÁP ÁN}
%	\inputansbox{10}{ans/ans}	
%\end{center}



\begin{center}
	\textbf{PHẦN 2 - CÂU TRẮC NGHIỆM ĐÚNG SAI}
\end{center}

\Opensolutionfile{ans}[ans/1D3-DS-OTC-D1]
\setcounter{ex}{0}

\begin{ex}%[1D3H2-5]
	[Trích đề thi học kỳ 1 lớp 11-THPT Kon-Tum-Năm học 2024-2025]
	Cho hai hàm số $y=f(x)$, $y=g(x)$ thoả mãn $\lim\limits_{x\to 2} f(x) = 3$ và $\lim\limits_{x\to 2} g(x) = -\infty$. 
	\choiceTF
	{$\lim\limits_{x\to 2} [f(x) - g(x)] = -\infty$} 
	{\True $\lim\limits_{x\to 2} [f(x) \cdot g(x)] = -\infty$} 
	{\True $\lim\limits_{x\to 2} \dfrac{f(x)}{g(x)} = 0$} 
	{$\lim\limits_{x\to 2} \dfrac{\sqrt{f(x)+1}-2}{f(x)-3} = 1$} 
	\loigiai{		
		\begin{itemchoice}
			\itemch  $\lim\limits_{x\to 2} [f(x) - g(x)] = +\infty$ vì $\lim\limits_{x\to 2} f(x) = 3$ và $\lim\limits_{x\to 2} g(x) = -\infty$. 
			\itemch  $\lim\limits_{x\to 2} [f(x) \cdot g(x)] = -\infty$ (vì $\lim\limits_{x\to 2} f(x) = 3>0$ và $\lim\limits_{x\to 2} g(x) = -\infty$).
			\itemch $\lim\limits_{x\to 2} \dfrac{f(x)}{g(x)} = \dfrac{\lim\limits_{x\to 2} f(x)}{\lim\limits_{x\to 2} g(x)} = 0$ vì $\lim\limits_{x\to 2} f(x) = 3>0$ và $\lim\limits_{x\to 2} g(x) = -\infty$. 
			\itemch  Đặt $y=f(x)$.\\
			 Khi $x \to 2$ thì $y \to 3$.\\
			  Giới hạn trở thành $\lim\limits_{y\to 3} \dfrac{\sqrt{y+1}-2}{y-3}$.\\
			  Ta có			
			\begin{eqnarray*}
				\lim\limits_{y\to 3} \dfrac{\sqrt{y+1}-2}{y-3}&=&\lim\limits_{y\to 3} \dfrac{(y+1)-4}{(y-3)\left(\sqrt{y+1}+2\right)}\\
				& =& \lim\limits_{y\to 3} \dfrac{y-3}{(y-3)\left(\sqrt{y+1}+2\right)}\\
				&  =& \lim\limits_{y\to 3} \dfrac{1}{\sqrt{y+1}+2}\\
				&  =& \dfrac{1}{\sqrt{3+1}+2}\\
				&  =& \dfrac{1}{4}.
			\end{eqnarray*}
		\end{itemchoice}
	}
\end{ex}
\begin{ex}%[1D3V3-3]
	Cho hàm số $y=f(x)=\heva{&\dfrac{\left|2x^2-7x+6\right|}{x-2} & \text{khi}\ x<2 \\& a+\dfrac{1-x}{2+x} & \text{khi}\ x \ge 2}$. Khi đó
	\choiceTF
	{Khi $a=3$ thì $\lim\limits_{x\to 2^+} f(x)=\dfrac{11}{2}$}
	{\True $\lim\limits_{x\to 2^-} f(x)=-1$}
	{Để hàm số liên tục tại $x_0=2$ thì $a=-\dfrac{1}{2}$}
	{Biết $a$ là giá trị để hàm số $f(x)$ liên tục tại $x_0=2$, thì bất phương trình $-x^2+ax+\dfrac{7}{4}>0$ có $1$ nghiệm nguyên }
	\loigiai{
		\begin{itemchoice}
			\itemch
			Với $a=3$ thì $\lim\limits_{x\to 2^+} f(x)=\lim\limits_{x\to 2^+}\left(3+\dfrac{1-x}{2+x}\right)=3-\dfrac{1}{4}=\dfrac{11}{4}$.
			\itemch
			Tại $x_0=2$, ta có
			\begin{itemize}
				\item $f(2)=a-\dfrac{1}{4}$.
				\item Và
				\begin{eqnarray*}
					\lim\limits_{x\to 2^-} f(x)&=&\lim\limits_{x\to 2^-} \dfrac{\left|2x^2-7x+6\right|}{x-2}=\lim\limits_{x\to 2^-} \dfrac{|(x-2)(2x-3)|}{x-2}\\
					&=&\lim\limits_{x\to 2^-} \dfrac{-(x-2)(2x-3)}{x-2}=-\lim\limits_{x\to 2^-}(2x-3)=-1.
				\end{eqnarray*}
			\end{itemize}
			\itemch
			Ta có $\lim\limits_{x\to 2^+} f(x)=\lim\limits_{x\to 2^+}\left(a+\dfrac{1-x}{2+x}\right)=a-\dfrac{1}{4}$.\\
			Để hàm số liên tục tại $x_0=2$ thì $f(2)=\lim\limits_{x\to 2^+} f(x)=\lim\limits_{x\to 2^-} f(x) \Leftrightarrow a-\dfrac{1}{4}=-1 \Leftrightarrow a=-\dfrac{3}{4}$.
			\itemch
			Với $a=-\dfrac{3}{4}$, xét bất phương trình $-x^2-\dfrac{3}{4} x+\dfrac{7}{4}>0 \Leftrightarrow-\dfrac{7}{4}<x<1$.\\
			Mà $x \in \mathbb{Z}$ nên $x \in\{-1 ; 0\}$.\\
			Vậy bất phương trình đã cho có $2$ nghiệm nguyên.
		\end{itemchoice}
		
	}
\end{ex}

\Closesolutionfile{ans}
%\inputansbox[2]{2}{ans/answer.tex}



\begin{center}
	\textbf{PHẦN 3 - CÂU TRẮC NGHIỆM TRẢ LỜI NGẮN}
\end{center}
\setcounter{ex}{0}
\Opensolutionfile{ans}[ans/1D3-KQ-OTC-D1]
\begin{ex}%[1D3H1-1]
	[Trích đề thi học kỳ 1 lớp 11-THPT Chu Văn An- Quãng Nam -Năm học 2023-2024]
	Cho hai dãy số $(u_n)$ và $(v_n)$ có $u_n = 1 + \dfrac{2024}{n^2}$ và $v_n = \dfrac{1}{n} + 2024$.\\
	Tìm $L = \displaystyle\lim_{n \to \infty} (u_n + v_n)$.
	\par\shortans{2025}
	\loigiai{
		Ta có
		\begin{eqnarray*}
			L &=& \lim_{n \to \infty} (u_n + v_n)\\
			&=& \lim_{n \to \infty} \left( 1 + \dfrac{2024}{n^2} + \frac{1}{n} + 2024 \right) \\
			&=& \lim_{n \to \infty} \left( 2025 + \dfrac{2024}{n^2} + \dfrac{1}{n} \right)\\
			&=& 2025.
		\end{eqnarray*}
		Vậy $L=2025$.
	}
\end{ex}
\begin{ex}%[1D3H2-3]
	[Trích đề thi giữa học kỳ 2 lớp 11-THPT Chuyên Nguyễn Tất Thành - KonTum -Năm học 2023-2024]
	Tính các giới hạn $A=\lim\limits_{x\to 1}\dfrac{x^2-1}{x-1}$.
	\par\shortans{2}
	\loigiai{
			Ta có $A=\lim\limits_{x\to 1}\dfrac{(x-1)(x+1)}{x-1}=\lim\limits_{x\to 1}(x+1)=2$.
	}
\end{ex}
\begin{ex}%[1D3V3-3]
	Cho hàm số
	$f(x)=\heva{&\dfrac{x^3-8}{x^2-4} &\text{ khi }x>2\\& a&\text{ khi }x=2\\&\dfrac{a x^2+b x-3}{x-3} &\text{ khi }x<2}$  liên tục tại $x_0=2$. Giá trị của $a+b$ bằng bao nhiêu?
	\par\shortans{-3}
	\loigiai{
		$f(x)=\heva{&\dfrac{x^3-8}{x^2-4} &\text{ khi }x>2\\& a&\text{ khi }x=2\\&\dfrac{ax^2+bx-3}{x-3} &\text{ khi }x<2}$, với $x_0=2$.\\
		$f\left(x_0\right)=f(2)=a$.\\
		$\lim \limits_{x \to 2^+} f(x)=\lim \limits_{x \to 2^+} \dfrac{x^3-8}{x^2-4}
		=\lim \limits_{x \to 2^+} \dfrac{(x-2)\left(x^2+2x+4\right)}{(x-2)(x+2)}
		=\lim \limits_{x \to 2^+} \dfrac{x^2+2x+4}{x+2}=3$.\\
		$\lim \limits_{x \to 2^-} f(x)=\lim \limits_{x \to 2^-}	\dfrac{ax^2+bx-3}{x-3}=-4a-2b+3$.\\
		Hàm số liên tục tại $x_0=2$ khi \[\lim \limits_{x \to 2^-} f(x)=\lim \limits_{x \to 2^+} f(x)=f(2)\Leftrightarrow -4a-2b+3=3=a \Leftrightarrow \heva{& a=3 \\ & b=-6.}\]
		Suy ra $\heva{& a=3 \\ & b=-6}$ thì hàm số đã cho liên tục tại $x_0=2$.\\
		Vậy $a+b=-3$.
	}
\end{ex}

\begin{ex}%[1D3V2-5]
	Biết rằng giới hạn
	$\lim\limits_{x\to 0}\dfrac{\sqrt{x+9}+\sqrt{x+16}-7}{x}
	=\lim\limits_{x\to 0}\left[ \dfrac{a}{\sqrt{x+9}+b}+\dfrac{c}{\sqrt{x+16}+d} \right]$ với $a$, $b$, $c$, $d$ là các số nguyên dương. Tính tổng các số $a$, $b$, $c$, $d$.
	
	\shortans{9}
	\loigiai{
		\begin{eqnarray*}
			\lim\limits_{x\to 0}\dfrac{\sqrt{x+9}+\sqrt{x+16}-7}{x}
			&=& \lim\limits_{x\to 0}\left( \dfrac{\sqrt{x+9}-3}{x}
			+ \dfrac{\sqrt{x+16}-4}{x} \right)\\
			&=& \lim\limits_{x\to 0}\left[ \dfrac{x+9-9}{x\left( \sqrt{x+9}+3 \right)}
			+ \dfrac{x+16-16}{x\left( \sqrt{x+16}+4 \right)} \right]\\
			&=& \lim\limits_{x\to 0}\left[ \dfrac{1}{\sqrt{x+9}+3}+\dfrac{1}{\sqrt{x+16}+4} \right]\\
			&=& \lim\limits_{x\to 0}\left[ \dfrac{a}{\sqrt{x+9}+b}+\dfrac{c}{\sqrt{x+16}+d} \right].
		\end{eqnarray*}
		Suy ra $a=1$, $b=3$, $c=1$, $d=4$.\\
		Vậy $a+b+c+d=9$.
		
	}
	
\end{ex}
\Closesolutionfile{ans}
\begin{center}
	\textbf{PHẦN 4 - TỰ LUẬN}
\end{center}
\setcounter{ex}{0}
\begin{ex}%[1D3H1-4]
	[Trích đề thi học kỳ 1 lớp 11-THPT Kon-Tum -Năm học 2024-2025]
	Biết rằng $\lim\limits_{n \to +\infty} \dfrac{\sqrt{2\,025n^2 + 1}}{2\,026n+2} = \dfrac{a}{b}$ với $a\in \mathbb{N}, b \in \mathbb{N}^*$ và $\dfrac{a}{b}$ là phân số tối giản. Tính giá trị của $a + b$.
	\loigiai{
		Ta có $ \lim\limits_{n \to +\infty} \dfrac{\sqrt{2\,025n^2 + 1}}{2\,026n+2} = \lim\limits_{n \to +\infty} \dfrac{\sqrt{2\,025 + \dfrac{1}{n^2}}}{2\,026 + \dfrac{2}{n}} = \dfrac{\sqrt{2\,025}}{2\,026} = \dfrac{45}{2\,026}$.\\
		Vậy $a = 45$, $b = 2\,026$, suy ra $a + b = 45 + 2\,026 = 2\,071$.
	}
\end{ex}
\begin{ex}%[1D3V2-5]
	Biết rằng giới hạn
	$\lim\limits_{x\to 2}\dfrac{\sqrt{5x-1}-\sqrt{9x-2}+1}{x-2}
	=\lim\limits_{x\to 2}\left( \dfrac{a}{b\sqrt{5x-1}+3}-\dfrac{c}{4+d\sqrt{9x-2}} \right)$ với $a$, $b$, $c$, $d$ là các số nguyên dương. Tính tổng các số $a$, $b$, $c$, $d$.
	\loigiai{
		\begin{eqnarray*}
			\lim\limits_{x\to 2}\dfrac{\sqrt{5x-1}-\sqrt{9x-2}+1}{x-2}
			&=& \lim\limits_{x\to 2}\dfrac{\sqrt{5x-1}-3+4-\sqrt{9x-2}}{x-2}\\
			&=& \lim\limits_{x\to 2}\left( \dfrac{\sqrt{5x-1}-3}{x-2}+\dfrac{4-\sqrt{9x-2}}{x-2} \right) \\
			&=& \lim\limits_{x\to 2}\left[ \dfrac{5x-1-9}{\left( x-2 \right)\left( \sqrt{5x-1}+3 \right)}+\dfrac{16-9x+2}{\left( x-2 \right)\left( 4+\sqrt{9x-2} \right)} \right]\\
			&=& \lim\limits_{x\to 2}\left[ \dfrac{5\left( x-2 \right)}{\left( x-2 \right)\left( \sqrt{5x-1}+3 \right)}+\dfrac{9\left( 2-x \right)}{\left( x-2 \right)\left( 4+\sqrt{9x-2} \right)} \right]\\
			&=& \lim\limits_{x\to 2}\left( \dfrac{5}{\sqrt{5x-1}+3}-\dfrac{9}{4+\sqrt{9x-2}} \right)\\
			&=& \lim\limits_{x\to 2}\left( \dfrac{a}{b\sqrt{5x-1}+3}-\dfrac{c}{4+d\sqrt{9x-2}} \right).
		\end{eqnarray*}
		Suy ra $a=5$, $b=1$, $c=9$, $d=1$.\\
		Vậy $a+b+c+d=16$.
	}	
\end{ex}

\begin{ex}%[1D3V3-3]
	Định $a$ và $b$ để các hàm số sau liên tục tại $x_0$.
	$f(x)=\heva{&1 &\text{ khi }x=a\\&\dfrac{x^2-a^2}{x-a} &\text{ khi }x>a\\& b-2x&\text{ khi }x<a}$, với $x_0=a$.
	\loigiai{
		$f(x)=\heva{&1 &\text{ khi }x=a\\&\dfrac{x^2-a^2}{x-a} &\text{ khi }x>a\\& b-2x&\text{ khi }x<a}$, với $x_0=a$.\\
		$f\left(x_0\right)=f(a)=1$.\\
		$\lim \limits_{x \to a^+} f(x)=\lim \limits_{x \to a^+} \dfrac{x^2-a^2}{x-a}
		=\lim \limits_{x \to a^+} \dfrac{(x-a)(x+a)}{x-a}
		=\lim \limits_{x \to a^+} (x+a)=2a$.\\
		$\lim \limits_{x \to a^-} f(x)=\lim \limits_{x \to a^-}	(b-2x)=b-2a$.\\
		Hàm số liên tục tại $x_0=a$ khi \[\lim \limits_{x \to a^-} f(x)=\lim \limits_{x \to a^+} f(x)=f(a)\Leftrightarrow 1=2a=b-2a \Leftrightarrow \heva{& a=\dfrac{1}{2} \\ & b=2.}\]
		Vậy $\heva{& a=\dfrac{1}{2} \\ & b=2}$ thì hàm số đã cho liên tục tại $x_0=a$.
	}
\end{ex}