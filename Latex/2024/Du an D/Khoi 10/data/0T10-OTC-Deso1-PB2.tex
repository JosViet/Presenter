\newpage
\section{Ôn tập chương 10}
\def\thoigian{90}%--Thời gian
\de{Đề số 1}{Chương X. Xác suất}


\begin{center}
	\textbf{PHẦN 1 - CÂU TRẮC NGHIỆM BỐN PHƯƠNG ÁN}
\end{center}
\Opensolutionfile{ans}[ans/ans-TN-0D0-DE1]
\begin{ex}%[0D0N1-2]%[Dự án D - đợt 2 NH24-25- Lâm Chính]
	Gieo ngẫu nhiên một đồng xu cân đối và đồng chất hai lần. Kí hiệu mặt ngửa là $N$ và mặt sấp là $S$. Không gian mẫu của phép thử là
	\choice
	{$\Omega=\left\{S,N\right\}$}
	{\True $\Omega=\left\{SS,NN,SN,NS\right\}$}
	{$\Omega=\left\{SN,NS\right\}$}
	{$\Omega=\left\{SS,NN\right\}$}
	\loigiai{
		Gieo ngẫu nhiên một đồng xu cân đối và đồng chất hai lần có $4$ kết quả có thể xảy ra.\\
		Không gian mẫu của phép thử là $\Omega=\left\{SS,NN,SN,NS\right\}$.
	}
\end{ex}
\begin{ex}%[0D0N1-2]%[Dự án D - đợt 2 NH24-25- Lâm Chính]
	Gieo con xúc xắc hai lần. Biến cố $A$ là biến cố sau hai lần gieo có ít nhất một mặt $6$ chấm xuất hiện. Biến cố $A$ là tập nào sau đây?
	\choice
	{$A=\{(1;6),(2;6),(3;6),(4;6),(5;6),(6;6)\}$}
	{\True $A=\{(1;6),(2;6),(3;6),(4;6),(5;6),(6;6),(6;1),(6;2),(6;3),(6;4),(6;5)\}$}
	{$A=\{(1;6),(2;6),(3;6),(4;6),(5;6),(6;1),(6;2),(6;3),(6;4),(6;5)\}$}
	{$A=\{(1;6),(2;6),(3;6),(4;6),(5;6)\}$}
	\loigiai{
		Gieo con xúc xắc hai lần. Biến cố $A$ là biến cố sau hai lần gieo có ít nhất một mặt $6$ chấm xuất hiện là  \[A=\{(1;6),(2;6),(3;6),(4;6),(5;6),(6;6),(6;1),(6;2),(6;3),(6;4),(6;5)\}.\]
	}
\end{ex}
\begin{ex}%[0D0H1-3]%[Dự án D - đợt 2 NH24-25- Lâm Chính]
	Gieo một con xúc xắc liên tiếp hai lần. Gọi $A$ là biến cố ``kết quả hai lần gieo như nhau''. Số phần tử của biến cố $A$ là
	\choice
	{$n(A)=8$}
	{$n(A)=12$}
	{\True $n(A)=6$}
	{$n(A)=1$}
	\loigiai{
		Ta có $A=\{(1;1),(2;2),(3;3),(4;4),(5;5),(6;6)\}$. Suy ra $ n(A)=6$.
	}
\end{ex}

\begin{ex}%[0D0H1-3]%[Dự án D - đợt 2 NH24-25- Lâm Chính]
	Xét phép thử tung con xúc xắc $6$ mặt hai lần. Số phần tử của biến cố $B\colon$ ``Tổng số chấm xuất hiện ở hai lần tung chia hết cho $3$'' là
	\choice
	{$n(B)=14$}
	{$n(B)=15$}
	{$n(B)=13$}
	{\True $n(B)=11$}
	\loigiai{
		Xét các cặp $(i,j)$ với $i,j\in \{1,2,3,4,5,6\}$ thỏa mãn $(i+j)\,\vdots\, 3$.\\
		Ta có các cặp có tổng chia hết cho $3$ là $(1,2)$; $(1,5)$; $(2,4)$; $(3,3)$; $(3,6)$; $(4,5)$.\\
		Hơn nữa mỗi cặp (trừ cặp $(3,3)$) khi hoán vị ta được một cặp thỏa yêu cầu bài toán.\\
		Vậy $n(B)=5\cdot2+1=11$.
	}
\end{ex}
\begin{ex}%[0D0H1-3]%[Dự án D - đợt 2 NH24-25- Lâm Chính]
	Hộp thứ nhất chứa $4$ quả bóng được đánh số từ $1$ đến $4$. Hộp thứ hai chứa $5$ quả bóng được đánh số từ $1$ đến $5$. Chọn ngẫu nhiên từ mỗi hộp $1$ quả bóng. Số kết quả thuận lợi cho biến cố \lq\lq Tổng các số ghi trên hai quả bóng không vượt quá $7$\rq\rq\,là
	\choice
	{\True $17$}
	{$15$}
	{$13$}
	{$16$}
	\loigiai{
		Tổng số các kết quả có thể xảy ra khi chọn bóng là $4 \cdot 5=20$.\\
		Có $3$ kết quả thuận lợi cho biến cố ``Tổng các số ghi trên hai quả bóng lớn hơn 7'' là $\{(3;5);(4;4);(4;5)\}$.\\
		Do đó số các kết quả thuận lợi cho biến cố ``Tổng các số ghi trên hai quả bóng không vượt quá 7'' là $20-3=17$.
	}
\end{ex}
\begin{ex}%[0D0H1-3]%[Dự án D - đợt 2 NH24-25- Lâm Chính]
	Xét phép thử \lq\lq Gieo một xúc xắc hai lần liên tiếp\rq\rq. Gọi $A$ là biến cố \lq\lq Tổng số chấm hai lần gieo không nhỏ hơn $10$\rq\rq. Số kết quả thuận lợi cho $A$ là
	\choice
	{$3$}
	{\True $4$}
	{$5$}
	{$6$}
	\loigiai{
		Ta có $A=\{(5;5);(6;5);(5;6);(6;6)\}$.\\
		Vậy có $4$ kết quả thuận lợi cho biến cố $A$.
	}
\end{ex}
\begin{ex}%[0D0H2-2]%[Dự án D - đợt 2 NH24-25- Lâm Chính]
	Gieo một đồng tiền cân đối và đồng chất bốn lần. Xác suất để cả bốn lần xuất hiện mặt sấp là 
	\choice
	{$\dfrac{4}{16}$}
	{$\dfrac{2}{16}$}
	{\True $\dfrac{1}{16}$}
	{$\dfrac{6}{16}$}
	\loigiai{
		Số phần tử của không gian mẫu là $n(\Omega)=2^4=16$. \\
		Gọi $A$ là biến cố ``Cả bốn lần gieo xuất hiện mặt sấp''.\\
		Ta có $A=\{SSSS\}\Rightarrow n(A)=1$.\\
		Vậy xác suất cần tính là $\mathrm{P}(A)=\dfrac{1}{16}$. 
	}
\end{ex}
\begin{ex}%[0D0H2-2]%[Dự án D - đợt 2 NH24-25- Lâm Chính]
	Gieo ba con xúc xắc. Xác suất để số chấm xuất hiện trên ba con xúc xắc như nhau là
	\choice
	{$\dfrac{12}{216}$}
	{$\dfrac{1}{216}$}
	{\True $\dfrac{6}{216}$}
	{$\dfrac{3}{216}$}
	\loigiai{
		Số phần tử của không gian mẫu là $n(\Omega)=6\cdot  6\cdot 6=216$. \\
		Gọi $A$ là biến cố ``Số chấm xuất hiện trên ba con xúc xắc như nhau''.  \\
		Ta có $A~\{(1 ; 1 ; 1);(2 ; 2 ; 2);(3 ; 3 ; 3); \ldots;(6 ; 6 ; 6)\}$. \\
		Suy ra $n(A)=6$. \\
		Vậy xác suất cần tính $\mathrm{P}(A)=\dfrac{6}{216}$. 
	}
\end{ex}
\begin{ex}%[0D0H2-4]%[Dự án D - đợt 2 NH24-25- Lâm Chính]
	Lớp $11 \mathrm{B}$ có $20$ nam và $25$ nữ. Chọn ngẫu nhiên hai học sinh để làm trực nhật. Xác suất để trong đó có ít nhất một nam là
	\choice
	{$\dfrac{20}{33}$}
	{\True $\dfrac{23}{33}$}
	{$\dfrac{25}{33}$}
	{$\dfrac{1}{20}$}
	\loigiai{
		Lớp $11 \mathrm{B}$ có $20+25=45$ (học sinh). \\
		Chọn $2$ học sinh trong số $45$ học sinh của lớp $11 \mathrm{B}$ thì có $\mathrm{C}_{45}^{2}=990$ (cách).\\
		Chọn $2$ học sinh trong số $45$ học sinh bao gồm $1$ nam và $1$ nữ thì có $\mathrm{C}_{20}^{1} \cdot \mathrm{C}_{25}^{1}=500$ (cách).\\
		Chọn $2$ học sinh nam trong số $20$ học sinh nam thì có $\mathrm{C}_{20}^{2}=190$ (cách).\\
		Xác suất để trong $2$ học sinh được chọn có ít nhất $1$ học sinh nam là $\mathrm{P}=\dfrac{500+190}{990}=\dfrac{23}{33}$.\\
	}
\end{ex}

\begin{ex}%[0D0H2-5]%[Dự án D - đợt 2 NH24-25- Lâm Chính]
	Một bình chứa $6$ viên bi màu, trong đó có $2$ bi xanh, $2$ bi đỏ, $2$ bi trắng. Lấy ngẫu nhiên $2$ viên bi. Xác suất để lấy được $2$ viên bi khác màu là
	\choice
	{$\dfrac{1}{15}$}
	{$\dfrac{2}{15}$}
	{\True $\dfrac{4}{5}$}
	{$\dfrac{4}{15}$}
	\loigiai{
		Lấy $2$ viên bi bất kì trong $6$ viên bi trong bình thì có $\mathrm{C}_{6}^{2}=15$ (cách). \\
		Lấy $2$ viên bi cùng màu thì có $\mathrm{C}_{2}^{2}+\mathrm{C}_{2}^{2}+\mathrm{C}_{2}^{2}=3$ (cách) nên có $15-3=12$ (cách) lấy được 2 viên bi khác màu.  \\
		Xác suất để lấy được $2$ viên bi khác màu trong tổng số 6 viên bi là
		$
		\mathrm{P}=\dfrac{12}{15}=\dfrac{4}{5}.
		$
	}
\end{ex}

\begin{ex}%[0D0H2-5]%[Dự án D - đợt 2 NH24-25- Lâm Chính]
	Một thùng có $7$ sản phẩm, trong đó có $4$ sản phẩm loại I và $3$ sản phẩm loại II. Lấy ngẫu nhiên $2$ sản phẩm. Xác suất để lấy được $2$ sản phẩm cùng loại là
	\choice
	{$\dfrac{1}{7}$}
	{$\dfrac{2}{7}$}
	{\True $\dfrac{3}{7}$}
	{$\dfrac{4}{7}$}
	\loigiai{
		Lấy ngẫu nhiên $2$ sản phẩm trong $7$ sản phẩm thì có $\mathrm{C}_{7}^{2}=21$ (cách). \\
		$2$ sản phẩm được lấy ra đều là sản phẩm loại I có $\mathrm{C}_{4}^{2}=6$ (cách).\\
		$2$ sản phẩm được lấy ra đều là sản phẩm loại II có $\mathrm{C}_{3}^{2}=3$ (cách).\\
		Xác suất để lấy được $2$ sản phẩm cùng loại là
		$
		\mathrm{P}=\dfrac{6+3}{21}=\dfrac{3}{7}.
		$
	}
\end{ex}
\begin{ex}%[0D0H2-7]%[Dự án D - đợt 2 NH24-25- Lâm Chính]
	Gọi $X$ là tập hợp các số tự nhiên gồm $6$ chữ số đôi một khác nhau được tạo thành từ các chữ số $1$, $2$, $3$, $4$, $5$, $6$, $7$, $8$, $9$. Chọn ngẫu nhiên một số từ tập hợp $X$. Xác suất để số được chọn chỉ chứa $3$ chữ số lẻ là
	\choice
	{\True $\dfrac{10}{21}$}
	{$\dfrac{25}{49}$}
	{$\dfrac{1}{3}$}
	{$\dfrac{13}{25}$}
	\loigiai{
		Số phần tử không gian mẫu $n(\Omega)=\mathrm{A}_{9}^{6}$.\\
		Từ bộ sáu chữ số $(1; 3; 5; 2; 4; 6)$ lập được $6!$ số tự nhiên thỏa mãn đề bài.\\
		Tương tự đối với $(1; 3; 5; 2; 4; 8)$, $(1; 3; 5; 2; 6; 8)$, $(1; 3; 5; 4; 6; 8)$ nếu cố định $1; 3; 5$ thì có $6!\cdot 4$ số thỏa mãn.\\
		Tương tự với $(1; 3; 7)$, $(1; 3; 9)$, $(1; 5; 7)$, $(1; 5; 9)$, $(1; 7; 9)$, $(3; 5; 7)$, $(3; 5; 9)$, $(3; 7; 9)$, $(5; 7; 9)$.\\
		Xác suất cần tìm là $\mathrm{P}=\dfrac{6!\cdot4\cdot10}{\mathrm{A}_{9}^{6}}=\dfrac{10}{21}$.
	}
\end{ex}
\Closesolutionfile{ans}
%\begin{center}
%	\textbf{ĐÁP ÁN}
%	\inputansbox{10}{ans/ans-TN-0D0-DE1}	
%\end{center}

\begin{center}
	\textbf{PHẦN 2 - CÂU TRẮC NGHIỆM ĐÚNG SAI}
\end{center}
\setcounter{ex}{0}
\Opensolutionfile{ans}[ans/ans-DS-0D0-DE1]
\begin{ex}%[0D0V2-4]%[Dự án D - đợt 2 NH24-25- Lâm Chính]
	Một tổ trong lớp 10B có $10$ học sinh, trong đó có $7$ học sinh nam và $3$ học sinh nữ. Giáo viên chọn ngẫu nhiên $5$ học sinh trong tổ để tập văn nghệ chào mừng ngày 26/3. 
	\choiceTF
	{Số phần tử của không gian mẫu là $560$}
	{\True Xác suất của biến cố $B \colon$\lq\lq $5$ học sinh được chọn đều là nam\rq\rq\,là $\dfrac{1}{12}$}
	{Xác suất của biến cố $C \colon$\lq\lq Trong $5$ học sinh được chọn có $3$ nam và $2$ nữ\rq\rq\,là $\dfrac{41}{462}$}
	{\True Xác suất của biến cố $D \colon$\lq\lq Trong $5$ học sinh được chọn có ít nhất $2$ nữ\rq\rq\,là $\dfrac{1}{2}$}
	\loigiai{
		\begin{itemchoice}
			\itemch Không gian mẫu là tập tất cả các tập con gồm $5$ học sinh trong $10$ học sinh.\\
			Vậy số phần tử của không gian mẫu là $n(\Omega)=\mathrm{C}_{10}^5=252$.
			\itemch Chọn $5$ học sinh nam từ $7$ học sinh nam, có $\mathrm{C}_7^5=21$ (cách chọn) $\Rightarrow n(B)=21$.\\
			Vậy xác suất của biến cố $B$ là $\mathrm{P}(B)=\dfrac{n(B)}{n(\Omega)}=\dfrac{21}{252}=\dfrac{1}{12}$.
			\itemch Mỗi phần tử của $C$ được hình thành từ $2$ công đoạn:
			\begin{itemize}
				\item Công đoạn $1$:\\
				Chọn $3$ học sinh nam từ $7$ học sinh nam, có $\mathrm{C}_7^3=35$ (cách chọn).
				\item Công đoạn $2$:\\
				Chọn $2$ học sinh nữ từ $3$ học sinh nữ, có $\mathrm{C}_3^2=3$ (cách chọn).
			\end{itemize}
			Theo quy tắc nhân, tập $C$ có $35\cdot 3=105$ (phần tử).\\
			Vậy $n(C)=105\Rightarrow \mathrm{P}(C)=\dfrac{105}{252}=\dfrac{5}{12}$.
			\itemch Trong $5$ học sinh được chọn có ít nhất $2$ học sinh nữ, có $2$ phương án:
			\begin{itemize}
				\item Phương án $1$: \\
				Trong $5$ học sinh được chọn có $3$ nam và $2$ nữ: có $\mathrm{C}_7^3\cdot \mathrm{C}_3^2=105$ (cách chọn).
				\item Phương án $2$:\\
				Trong $5$ học sinh được chọn có $2$ nam và $3$ nữ: có $\mathrm{C}_7^2\cdot \mathrm{C}_3^3=21$ (cách chọn).
			\end{itemize}
			Theo quy tắc cộng, ta có $n(D)=105+21=126$.\\
			Vậy $\mathrm{P}(D)=\dfrac{126}{252}=\dfrac{1}{2}$.
	\end{itemchoice}}
\end{ex}
\begin{ex}%[0D0H2-5]%[Dự án D - đợt 2 NH24-25- Lâm Chính]
	Một hộp có $16$ thẻ được đánh số từ $1$ tới $16$ và không có thẻ nào trùng số nhau. Chọn ngẫu nhiên $2$ thẻ.
	\choiceTF
	{\True Số cách chọn $2$ thẻ là $\mathrm{C}_{16}^2$}
	{Xác suất để chọn được hai thẻ trong đó có $1$ thẻ chia hết cho $9$ là $\dfrac{1}{9}$}
	{Xác suất để chọn được hai thẻ mà tích hai số trên hai thẻ là số lẻ bằng $\dfrac{23}{30}$}
	{\True Xác suất để chọn được hai thẻ đều lẻ là $\dfrac{7}{30}$}
	\loigiai{
		\begin{itemchoice}
			\itemch Số cách chọn $2$ thẻ là $n\left(\Omega\right)=\mathrm{C}_{16}^2$.
			\itemch Từ $1$ đến $16$ có $1$ tấm thẻ mang số chia hết cho $9$.\\
			Gọi $A$ là biến cố ``Chọn được hai thẻ trong đó có $1$ thẻ chia hết cho $9$''.\\
			Khi đó $n(A)=\mathrm{C}_1^1\cdot \mathrm{C}_{15}^1$.\\
			Xác suất để chọn được hai thẻ trong đó có $1$ thẻ chia hết cho $9$ là $\mathrm{P}(A)=\dfrac{\mathrm{C}_1^1\cdot \mathrm{C}_{15}^1}{\mathrm{C}_{16}^2}=\dfrac{1}{8}$.
			\itemch Từ $1$ đến $16$ có $8$ tấm thẻ mang số lẻ.\\
			Để tích hai số trên hai thẻ là số lẻ thì cả hai thẻ đó đều mang số lẻ.\\
			Gọi $B$ là biến cố ``Chọn được hai thẻ mà tích hai số trên hai thẻ là số lẻ''.\\
			Khi đó $n(B)=\mathrm{C}_8^2$.\\
			Xác suất để chọn được hai thẻ mà tích hai số trên hai thẻ là số lẻ là $\mathrm{P}(B)=\dfrac{\mathrm{C}_8^2}{\mathrm{C}_{16}^2}=\dfrac{7}{30}$.
			\itemch Xác suất để chọn được hai thẻ đều lẻ là $\mathrm{P}(B)=\dfrac{\mathrm{C}_8^2}{\mathrm{C}_{16}^2}=\dfrac{7}{30}$.
		\end{itemchoice}	
	}
\end{ex}
\Closesolutionfile{ans}
%\inputansbox[2]{2}{ans/ans-DS-0D0-DE1}



\begin{center}
	\textbf{PHẦN 3 - CÂU TRẮC NGHIỆM TRẢ LỜI NGẮN}
\end{center}
\setcounter{ex}{0}
\Opensolutionfile{ans}[ans/ans-KQ-0D0-DE1]
\begin{ex}%[0D8H2-2]%[Dự án D - đợt 2 NH24-25- Lâm Chính]
	Một hộp gồm $7$ viên bi xanh, $8$ viên bi đỏ và $9$ viên bi trắng. Hỏi có bao nhiêu cách chọn ra $4$ viên bi có đủ ba màu? 
	\par
	\shortans{$5\,292$}
	\loigiai{
		\begin{itemize}
			\item Trường hợp $1$: Chọn được $1$ viên bi xanh, $1$ viên bi đỏ và $2$ viên bi trắng.\\
			Số cách chọn cho trường hợp này là $\mathrm{C}_{7}^1\cdot \mathrm{C}_{8}^1\cdot \mathrm{C}_{9}^2=2\,016$ (cách).
			\item Trường hợp $2$: Chọn được $1$ viên bi xanh, $2$ viên bi đỏ và $1$ viên bi trắng.\\
			Số cách chọn cho trường hợp này là $\mathrm{C}_{7}^1\cdot \mathrm{C}_{8}^2\cdot \mathrm{C}_{9}^1=1\,764$ (cách).
			\item Trường hợp $3$: Chọn được $2$ viên bi xanh, $1$ viên bi đỏ và $1$ viên bi trắng.\\
			Số cách chọn cho trường hợp này là $\mathrm{C}_{7}^2\cdot \mathrm{C}_{8}^1\cdot \mathrm{C}_{9}^1=1\,512$ (cách).
		\end{itemize}
		Vậy có $2\,016+1\,764+1512= 5\,292$ cách chọn.
	}
\end{ex}
\begin{ex}%[0D0H2-7]%[Dự án D - đợt 2 NH24-25- Lâm Chính]
	Gọi $X$ là tập hợp các số tự nhiên gồm $8$ chữ số lấy từ các chữ số $1$, $2$, $3$, $4$, $5$. Chọn ngẫu nhiên $1$ số tự nhiên từ tập X. Tính xác suất để số tự nhiên được chọn có chữ số $5$ xuất hiện đúng $3$ lần (kết quả làm tròn đến hàng phần trăm).	
	\par
	\shortans{$0{,}15$}
	\loigiai{
		Số phần tử của không gian mẫu là $n(\Omega)=5^8$.\\
		Gọi $A$ là biến cố ``Số tự nhiên được chọn có chữ số $5$ xuất hiện đúng $3$ lần''.\\
		Để lập số tự nhiên thỏa mãn yêu cầu bài toán ta thực hiện các công đoạn sau:
		\begin{itemize}
			\item Công đoạn $1$: Chọn $3$ vị trí trong $8$ vị trí để điền số $5$, có $\mathrm{C}_8^3$ cách.
			\item Công đoạn $2$: Trong $5$ vị trí còn lại, mỗi vị trí có $4$ cách điền từ $1$ trong $4$ chữ số $1$, $2$, $3$, $4$ nên có $4^5$ cách điền.
		\end{itemize}		
		Theo quy tắc nhân có $n(A)=\mathrm{C}_8^3 \cdot 4^5=57\,344$.\\
		Suy ra $\mathrm{P}(A)=\dfrac{n(A)}{n(\Omega)}=\dfrac{57\,344}{390\,625}\approx 0{,}15$.
	}
\end{ex}
\begin{ex}%[0D0H2-5]%[Dự án D - đợt 2 NH24-25- Lâm Chính]
	Màu hạt của đậu Hà Lan có hai kiểu hình là màu vàng và màu xanh tương ứng với hai loại gen là gen trội $A$  và gen lặn $a$. Hình dạng hạt của đậu Hà Lan có hai kiểu hình là hạt trơn và hạt nhăn tương ứng với hai loại gen là gen trội $B$  và gen lặn $b$. Biết rằng, cây con lấy ngẫu nhiên một gen từ cây bố và một gen từ cây mẹ.
	Phép thử là cho lai hai loại đậu Hà Lan, trong đó cả cây bố và cây mẹ đều có kiểu gen là $(Aa, Bb)$ và kiểu hình là hạt màu vàng và trơn. Giả sử các kết quả có thể là đồng khả năng. Gọi $T$ là xác suất để cây con có kiểu hình là hạt màu xanh và nhăn. Tính $\dfrac{5}{T}$.
	\par
	\shortans{$80$}
	\loigiai{
		Ta liệt kê được tất cả các kết quả có thể của phép thử bằng cách vẽ bảng như sau:
		\begin{center}
			\begin{tabular}{|c|c|c|c|c|}
				\hline
				\diagbox{Màu hạt}{Dạng hạt} & \textbf{BB} & \textbf{Bb} & \textbf{bB} & \textbf{bb} \\
				\hline
				\textbf{AA} & $(AA,BB)$ & $(AA,Bb)$ & $(AA,bB)$ & $(AA,bb)$ \\
				\hline
				\textbf{Aa} & $(Aa,BB)$ & $(Aa,Bb)$ & $(Aa,bB)$ & $(Aa,bb)$ \\
				\hline
				\textbf{aA} & $(aA,BB)$ & $(aA,Bb)$ & $(aA,bB)$ & $(aA,bb)$ \\
				\hline
				\textbf{aa} & $(aa,BB)$ & $(aa,Bb)$ & $(aa,bB)$ & $(aa,bb)$ \\
				\hline
			\end{tabular}
		\end{center}
		Mỗi ô là một kết quả có thể về kiểu gene của cây con.\\
		Do đó số phần tử của không gian mẫu là $n(\Omega)=16$.\\
		Kết quả thuận lợi cho biến cố \lq\lq Cây con có kiểu hình là hạt màu xanh và nhăn\rq\rq\, là $(aa,bb)$.\\
		Do đó $n(A)=1$.\\
		Xác suất để cây con có kiểu hình là hạt màu xanh và nhăn là $\mathrm{P}(A)=\dfrac{n(A)}{n(\Omega)}=\dfrac{1}{16}$.\\
		Vậy $\dfrac{5}{T}=5\cdot16=80$.
	}
\end{ex}
\begin{ex}%[THPT Nguyễn Trãi - Ba Đình - Hà Nội]%[0D0V2-5]
	Trong một dịp quay xổ số, có $3$ loại giải thưởng: $1\,000\,000$ đồng, $500\,000$ đồng, $100\,000$ đồng. Nơi bán có $100$ tờ vé số, trong đó có $1$ vé trúng thưởng $1\,000\,000$ đồng, $5$ vé trúng thưởng $500\,000$ đồng, $10$ vé trúng thưởng $100\,000$ đồng. Một người mua ngẫu nhiên $3$ vé xổ số. Xác suất để người mua đó trúng thưởng ít nhất $300\,000$ đồng là $\dfrac{a}{b}$ (với $a$, $b\in\mathbb{N}$ và $a<1\,000$). Tính $a+b$.
	\par
	\shortans{6767}
	\loigiai{
		Số phần tử của không gian mẫu là $n(\Omega)=\mathrm{C}_{100}^3=161\,700$.\\
		Gọi ${A}$ là biến cố \lq\lq Người mua đó trúng thưởng ít nhất $300\,000$ đồng\rq\rq\, thì biến cố đối của ${A}$ là ${\overline{A}}\colon$\lq\lq Người mua đó trúng thưởng nhiều nhất $200\,000$ đồng\rq\rq.
		Các khả năng của biến cố $\overline{A}$ là
		\begin{itemize}
			\item Không trúng thưởng: Số khả năng xảy ra là $\mathrm{C}_{84}^3=95\,284$.
			\item Trúng thưởng $100\,000$ đồng: Số khả năng xảy ra là $\mathrm{C}_{84}^2 \cdot \mathrm{C}_{10}^1=34\,860$.
			\item Trúng thưởng $200\,000$ đồng: Số khả năng xảy ra là $\mathrm{C}_{84}^1 \cdot \mathrm{C}_{10}^2=3\,780$.		
		\end{itemize}
		Suy ra xác suất của biến cố $\overline{A}$ là $\mathrm{P}\Big(\overline{A}\Big)=\dfrac{95\,284+34\,860+3\,780}{161\,700}=\dfrac{4\,783}{5\,775}$.\\
		Vậy xác suất của biến cố $A$ là $\mathrm{P}(A)=1-\mathrm{P}\Big(\overline{A}\Big)=1-\dfrac{4\,783}{5\,775}=\dfrac{992}{5\,775}$.\\
		Khi đó $a=992$, $b=5\,775$. Vậy $a+b=6767$.
	}
\end{ex}
\Closesolutionfile{ans}



\begin{center}
	\textbf{PHẦN 4 - TỰ LUẬN}
\end{center}
\begin{ex}%[0D0H2-2]%[Dự án D - đợt 2 NH24-25- Lâm Chính]
	Gieo một đồng xu cân đối đồng chất ba lần liên tiếp. Gọi $A$ là biến cố ``Trong ba lần gieo có ít nhất một lần xuất hiện mặt sấp''. Hãy tính số kết quả thuận lợi cho $A$.
	\loigiai{
		$A$ là biến cố \lq\lq Trong ba lần gieo có ít nhất một lần xuất hiện mặt sấp\rq\rq.\\
		Ta có $A=\{S N N ; N S N ; N N S ; S S N ; S N S ; N S S ; S S S\}$.\\
		Suy ra có $7$ kết quả thuận lợi cho $A$.
	}
\end{ex}
\begin{ex}%[0D0H2-4]%[Dự án D - đợt 2 NH24-25- Lâm Chính]
	Lớp 10A1 có $20$ học sinh nam và $26$ học sinh nữ. Lớp 10A2 có $16$ học sinh nam và $24$ học sinh nữ. Chọn ngẫu nhiên mỗi lớp $3$ học sinh đi dự hội nghị đoàn trường. Tính xác suất của mỗi biến cố sau:
	\begin{enumerate}
		\item Trong $6$ học sinh được chọn đều là học sinh nữ.
		\item Trong $6$ học sinh được chọn có cả học sinh nam và học sinh nữ.
	\end{enumerate}
	\loigiai{
		Chọn ngẫu nhiên mỗi lớp $3$ học sinh, do đó số phần tử của không gian mẫu là $n(\Omega)=\mathrm{C}_{46}^3\cdot \mathrm{C}_{40}^3$.
		\begin{enumerate}
			\item Gọi $A$ là biến cố \lq\lq Trong $6$ học sinh được chọn đều là học sinh nữ\rq\rq.\\
			Ta có $n(A)=\mathrm{C}_{26}^3\cdot\mathrm{C}_{24}^3$.\\
			$\mathrm{P}(A)=\dfrac{n(A)}{n(\Omega)}=\dfrac{\mathrm{C}_{26}^3\cdot\mathrm{C}_{24}^3}{\mathrm{C}_{46}^3\cdot \mathrm{C}_{40}^3}\approx 0{,}02$.
			\item Gọi $B$ là biến cố \lq\lq Trong $6$ học sinh được chọn có cả học sinh nam và học sinh nữ\rq\rq.\\
			Gọi $C$ là biến cố \lq\lq Trong $6$ học sinh được chọn đều là học sinh nam\rq\rq\.\\
			Ta có $n(C)=\mathrm{C}_{20}^3\cdot\mathrm{C}_{16}^3$.\\
			Vậy $n(B)=n(\Omega)-n(A)-n(C)=144\,077\,600$.\\
			Suy ra $\mathrm{P}(B)=\dfrac{n(B)}{n(\Omega)}\approx 0{,}96$.
		\end{enumerate}
	}
\end{ex}
\begin{ex}%[0D0H2-3]%[Dự án D - đợt 2 NH24-25- Lâm Chính]
	Xếp ngẫu nhiên $5$ người $A$, $B$, $C$, $D$, $E$ vào một cái bàn dài có $5$ chỗ ngồi. 
	\begin{enumerate}
		\item Tính xác suất để bạn $A$ ngồi chính giữa.
		\item Tính xác suất để hai bạn $A$ và $B$ ngồi đầu bàn.
	\end{enumerate}
	\loigiai{
		Mỗi cách xếp $5$ người vào bàn $5$ chỗ là một hoán vị của $5$ phần tử nên ta có $n(\Omega)=5!=120$.
		\begin{enumerate}[a)]
			\item Gọi $X$ là biến cố \lq\lq Xếp $5$ người trong đó $A$ ngồi chính giữa\rq\rq.\\
			Suy ra $n(X)=4!=24$.\\
			Xác suất của biến cố $X$ là $\mathrm{P}(X)=\dfrac{24}{120}=\dfrac{1}{5}$.
			\item
			Gọi $Y$ là biến cố \lq\lq Xếp $5$ người trong đó $A$ và $B$ ngồi đầu bàn\rq\rq.\\
			Xếp $5$ người thỏa điều kiện như trên có hai giai đoạn:
			\begin{itemize}
				\item Giai đoạn $1$: Xếp $A$ và $B$ ngồi đầu bàn có $2$ cách.
				\item Giai đoạn $2$: Xếp $3$ người còn lại vào $3$ chỗ có $3!$ cách.
			\end{itemize}
			Do đó số kết quả thuận lợi của biến cố $Y$ là $n(Y)=2\cdot 3!=12.$\\
			Vậy xác suất để hai bạn $A$ và $B$ ngồi đầu bàn là $\mathrm{P}(Y)=\dfrac{12}{120}=0{,}1.$ 
		\end{enumerate}
	}
\end{ex}
