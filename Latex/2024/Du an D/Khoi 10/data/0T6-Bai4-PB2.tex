\newpage
\section{CÁC SỐ ĐẶC TRƯNG ĐO MỨC ĐỘ PHÂN TÁN}
\subsection{LÝ THUYẾT CẦN NHỚ}
\subsubsection{Khoảng biến thiên và khoảng tứ phân vị}
\iconMT \indam{Định nghĩa:}
\begin{boxdn}
	Sắp xếp mẫu số liệu theo thứ tự không giảm, ta được
	$$x_1 \leq x_2 \leq \ldots \leq x_n.$$
	\begin{itemize}
		\item \textbf{Khoảng biến thiên} của một mẫu số liệu, kí hiệu là $R$, là hiệu số giữa giá trị lớn nhất và giá trị nhỏ nhất trong mẫu số liệu, tức là
		$$R=x_n - x_1$$
		\item \textbf{Khoảng tứ phân vị}, kí hiệu là $ \triangle Q $, là hiệu số giữa $Q_3$ và $Q_1$, tức là:
		$ \triangle Q=Q_3-Q_1 $.
	\end{itemize}	
\end{boxdn}

\begin{khung4}{Lưu ý}
	\begin{itemize}
		\item Khoảng biến thiên đặc trưng cho độ phân tán của toàn bộ mẫu số liệu.
		\item Khoảng tứ phân vị đặc trưng cho độ phân tán của một nửa các số liệu, có giá trị thuộc đoạn từ $Q_1$ đến $Q_3$ trong mẫu.
		\item Khoảng tứ phân vị không bị ảnh hưởng bởi các giá trị rất lớn hoặc rất bé trong mẫu.
	\end{itemize}
	
\end{khung4}


\subsubsection{Phương sai và độ lệch chuẩn}
\iconMT \indam{Định nghĩa:}
\begin{boxdn}
	Với mẫu số liệu $x_{1}$, $x_{2}$, $\ldots$, $x_{n}$, nếu gọi số trung bình là $\overline{x}$ thì với mỗi giá trị $x_{i}$, độ lệch của nó so với giá trị trung bình là $x_{i}-\overline{x}$.
	\begin{itemize}
		\item Phương sai là giá trị $S^2=\dfrac{\left(x_{1}-\overline{x}\right)^{2}+\left(x_{2}-\overline{x}\right)^{2}+\ldots+\left(x_{n}-\overline{x}\right)^{2}}{n}$.
		\item Căn bậc hai của phương sai, $s=\sqrt{s^{2}}$, được gọi là độ lệch chuẩn.
	\end{itemize}
\end{boxdn}

\begin{khung4}{Lưu ý}
	\begin{itemize}
		\item Người ta còn sử dụng đại lượng để đo độ phân tán của mẫu số liệu:
		$$\hat{s}^{2}=\dfrac{\left(x_{1}-\overline{x}\right)^{2}+\left(x_{2}-\overline{x}\right)^{2}+\ldots+\left(x_{n}-\overline{x}\right)^{2}}{n-1}.$$
		\item Nếu số liệu càng phân tán thì phương sai và độ lệch chuẩn càng lớn.
	\end{itemize}
\end{khung4}

\subsubsection{Phát hiện số liệu bất thường bằng giá trị ngoại lệ}
%\iconMT %\indam{Tính chất:}
\begin{boxdn}
	\begin{itemize}
		\item  Trong mẫu số liệu thống kê, có khi gặp những giá trị quá lớn 
		hoặc quá nhỏ so với đa số các giá trị khác. Những giá trị này được gọi 
		là \textbf{giá trị bất thường}.
		\item Các số liệu \textbf{lớn hơn} $Q_3+1{,}5 \cdot 
		\Delta_{Q}$ hoặc \textbf{bé hơn} $Q_1-1{,}5 \cdot 
		\Delta_{Q}$ được xem là giá trị bất thường.\\
		Hay các số liệu \textbf{không} thuộc đoạn $\left[Q_1-1{,}5 \cdot \Delta_{Q}; Q_3+1{,}5 \cdot \Delta_{Q} \right]$ là các số liệu bất thường.
	\end{itemize}
\end{boxdn}

%-------------------------------------------------------------------------------------------------------------
\subsection{PHÂN LOẠI VÀ PHƯƠNG PHÁP GIẢI TOÁN}
\begin{dang}{Khoảng biến thiên, khoảng tứ phân vị}
\end{dang}

\begin{vd}%[0D6N4-2]%[Dự án đề cương 3 Khối NH24-25-Dot 2-Nguyễn Hoài Nam]
	Chiều cao (tính theo đơn vị m) của các bạn học sinh trong một lớp học được thống kê và ghi lại trong bảng dưới đây:
	\begin{longtable}{|l|c|c|c|c|c|c|c|}
		\hline
		Chiều cao & 1{,}6 & 1{,}61 & 1{,}62 & 1{,}63 & 1{,}64 & 1{,}65\\
		\hline
		Số lượng & 3 & 5 & 8 & 9 & 7 & 6\\
		\hline
	\end{longtable}
	Hãy tìm khoảng biến thiên của mẫu số liệu trên
	\loigiai{
		Mẫu số liệu có giá trị lớn nhất và giá trị nhỏ nhất lần lượt là $ 1{,}65 $ và $ 1{,}61 $.\\
		Do đó khoảng biến thiên của mẫu số liệu đã cho là $ R=1{,}65-1{,}61=0{,}04 $.
	}
\end{vd}

\begin{vd}%[0D6N4-2]%[Dự án đề cương 3 Khối NH24-25-Dot 2-Nguyễn Hoài Nam]
	Điểm kiểm tra môn Toán của các bạn học sinh Tổ 1 và Tổ 2 lớp 10C như sau:
	\begin{longtable}{p{1.5cm}p{1cm}p{1cm}p{1cm}p{1cm}p{1cm}p{1cm}p{1cm}p{1cm}}
		Tổ 1: & 6 & 9 & 4 & 2 & 7 & 9 & 6 & 10\\
		Tổ 2: & 4 & 5 & 6 & 3 & 9 & 5 & 8 & 4\\
	\end{longtable}
	Hãy tìm các khoảng biến thiên trong hai mẫu số liệu. Căn cứ vào số liệu này, hãy chỉ ra tổ nào học đồng đều hơn.
	\loigiai{
		Mẫu số liệu ``Tổ 1'' có giá trị lớn nhất và giá trị nhỏ nhất lần lượt là $ 10 $ và $ 2 $.\\
		Do đó khoảng biến thiên của mẫu số liệu đã cho là $ R_1=10-2=8 $.\\
		Mẫu số liệu ``Tổ 2'' có giá trị lớn nhất và giá trị nhỏ nhất lần lượt là $ 9 $ và $ 3 $.\\
		Do đó khoảng biến thiên của mẫu số liệu đã cho là $ R_2=9-3=6 $.\\
		Mẫu số liệu ``Tổ 2'' có\\
		Do $R_1>R_2$ nên ta nói các bạn Tổ 2 học đều hơn các bạn Tổ 1.	
	}
\end{vd} 

\begin{vd}%[0D6N4-2]%[Dự án đề cương 3 Khối NH24-25-Dot 2-Nguyễn Hoài Nam]
	Số máy tính bán được trong $7$ tháng liên tiếp của một cửa hàng được ghi lại trong bảng sau:
	\begin{center}
		\begin{tabular}{|c|c|c|c|c|c|c|}
			\hline 
			$83$ & $79$ & $92$ & $71$ & $69$ & $83$ & $74$ \\ 
			\hline 
		\end{tabular} 
	\end{center}
 Tính khoảng biến thiên, khoảng tứ phân vị của mẫu số liệu.
	\loigiai{
		 Các số liệu được sắp xếp lại theo thứ tự không giảm: $69$ $71$ $74$ $79$ $83$ $83$ $92$.\\
			Khoảng biến thiên: $R=92-69=23$.\\
			Tứ phân vị: $Q_1=71$, $Q_2 =(74+79):2 = 76{,}5$, $Q_3=83$.\\
			Khoảng tứ phân vị $\Delta_Q=Q_3-Q_1=83-71=12$.	
	}
\end{vd}


\begin{dang}{Giá trị bất thường của mẫu số liệu}
\end{dang}

\begin{vd}%[0D6H4-3]%[Dự án đề cương 3 Khối NH24-25-Dot 2-Nguyễn Hoài Nam]
	Điểm kiểm tra môn Toán của $10$ học sinh sau:\\
	\hspace*{1cm} $1$ \hspace*{1cm} $7$ \hspace*{1cm} $10$ \hspace*{1cm} $7$ \hspace*{1cm} $7$ \hspace*{1cm} $6$ \hspace*{1cm} $9$ \hspace*{1cm} $8$ \hspace*{1cm} $10$ \hspace*{1cm} $8$\\
	Hãy tìm các số liệu bất thường trong mẫu số liệu trên.
	\loigiai{
		Sắp xếp mẫu số liệu theo thứ tự không giảm từ trái qua phải như sau
		$$1\quad 6\quad 7\quad 7\quad 7\quad 8\quad 8 \quad 9 \quad 10\quad 10$$
		Vì $n=10$ chẵn nên $Q_2$ là trung bình cộng của hai số liệu chính giữa. Ta có $Q_2 = \dfrac{8+7}{2}= 7{,}5$.\\
		Tứ phân vị dưới $Q_1=7$, tứ phân vị trên $Q_3= 9$ và $\Delta_{Q} = Q_3-Q_1 = 2$.\\
		Ta có $\left[Q_1-1{,}5 \cdot  \Delta_{Q}; Q_3+1{,}5 \cdot  \Delta_{Q} \right] =\left[4;12\right]$. \\
		Vì $1\notin \left[4;12\right]$ nên là số liệu bất thường trong mẫu số liệu trên.
		
	}
\end{vd}
\begin{vd}%[0D6H4-3]%[Dự án đề cương 3 Khối NH24-25-Dot 2-Nguyễn Hoài Nam]
	Tuổi thọ của $30$ bóng đèn được thắp thử (đơn vị: giờ) có kết quả như trong bảng sau:
	\begin{center}
		\begin{tabular}{|c|c|c|c|c|c|c|c|c|c|c|c|c|}
			\hline Số giờ 			& $1178$ & $1179$ & $1180$ & $1184$ & $1185$ & $1187$ &$1190$ &$1191$ &$1198$&$1568$ & $1569$ \\
			\hline Số bóng đèn  & $4$   &$ 5$    & $5$    & $3$ 	& $3$ & $2$ &$1$ &$1$ & $4$& $1$& $1$ \\
			\hline
		\end{tabular}
	\end{center} 
	Hãy tìm các số liệu bất thường trong mẫu số liệu trên.
	\loigiai{
		Từ bảng số liệu ta tìm được số trung vị $Q_2=1184$, tứ phân vị dưới $Q_1= 1179$, tứ phân vị trên $Q_3= 1190$ và khoảng tứ phân vị $\Delta_{Q} = 1190 - 1179 = 11$.\\
		Ta có $\left[Q_1-1{,}5 \cdot  \Delta_{Q}; Q_3+1{,}5 \cdot  \Delta_{Q} \right] =\left[1162{,}5;1206{,}5\right]$.\\
		Từ đó ta có các số $1568$ và $1569$ là các số liệu bất thường của mẫu số liệu.		
	}
\end{vd}


\begin{vd}%[0D6H4-3]%[Dự án đề cương 3 Khối NH24-25-Dot 2-Nguyễn Hoài Nam]
	Điều tra thời gian hoàn thành một sản phẩm của $20$ công nhân, người ta thu được mẫu số liệu sau (thời gian tính bằng phút)
	\begin{center}	
		\begin{tabular}{|c|c|c|c|c|c|c|c|c|c|}
			\hline 
			7 & 12 & 13 & 15 & 11 & 13 & 16 & 18 & 19 & 21 \\ 
			\hline 
			23 & 21 & 15 & 17 & 16 & 15 & 20 & 13 & 16 & 29 \\ 
			\hline 
		\end{tabular} 
	\end{center}
	Hãy tìm các số liệu bất thường trong mẫu số liệu trên.
	\loigiai{
		Sắp xếp các số liệu trong mẫu theo thứ tự không giảm ta có
		\begin{center}
			\begin{tabular}{|c|c|c|c|c|c|c|c|c|c|c|c|c|c|c|}
				\hline Số giờ 			& $7$ & $11$ & $12$ & $13$ & $15$ & $16$ &$17$ &$18$ &$19$&$20$ & $21$  & $23$ & $29$ \\
				\hline Số lần xuất hiện  & $1$ &$ 1$ & $1$  & $3$ & $3$ & $3$ 	 &$1$ &$1$ & $1$& $1$& $2$& $1$& $1$ \\
				\hline
			\end{tabular}
		\end{center} 
		Từ bảng số liệu ta tìm được số trung vị $Q_2=16$, tứ phân vị dưới $Q_1= 13$, tứ phân vị trên $Q_3= 19{,}5$ và khoảng tứ phân vị $\Delta_{Q} = 19{,}5 - 13 = 6{,}5$.\\
		Ta có $\left[Q_1-1{,}5 \cdot  \Delta_{Q}; Q_3+1{,}5 \cdot  \Delta_{Q} \right] =\left[3{,}25;29{,}25\right]$.\\
		Từ đó ta có mẫu số liệu trên không có số liệu bất thường.		
	}
\end{vd}



\begin{dang}{Phương sai, độ lệch chuẩn của mẫu số liệu}
\end{dang}

\begin{vd}%[0D6H4-4]%[Dự án đề cương 3 Khối NH24-25-Dot 2-Nguyễn Hoài Nam]
	Sản lượng lúa (đơn vị là tạ) của $40$ thửa ruộng thí nghiệm có cùng diện tích được trình bày trong bảng tần số sau đây
	\begin{center}
		\begin{tabular}{|c|c|c|c|c|c|c|}
			\hline
			Sản lượng $(x)$ & $20$ & $21$ & $22$ & $23$ & $24$ &      \\ \hline
			Tần số $(n)$    & $5$  & $8$  & $11$ & $10$ & $6$  & $n=40$ \\ \hline
		\end{tabular}
	\end{center}
\begin{enumerate}
	\item Tính sản lượng trung bình của $40$ thửa ruộng.
	\item Tính phương sai và độ lệch chuẩn.
\end{enumerate}
	\loigiai{
	\begin{enumerate}
		\item Số trung bình sản lượng của $40$ thửa ruộng là
		$$
		\overline{x}=\dfrac{5\cdot20+821+11\cdot22+10\cdot23+6\cdot24}{40}=22{,}1 \text { (tạ). }
		$$
		\item Tính phương sai:
		\begin{eqnarray*}
			S^{2} &=&\dfrac{1}{40}\left[5\cdot (20-22{,}1)^{2}+8\cdot (21-22{,}1)^{2}+11\cdot (22-22{,}1)^{2}+10\cdot (23-22{,}1)^{2}+6\cdot (24-22{,}1)^{2}\right] \\
			&=& 1{,}54.
		\end{eqnarray*}
		Tính độ lệch chuẩn: $S=\sqrt{S^{2}}=\sqrt{1{,}54}\approx1{,}24$.
	\end{enumerate}	
	}
\end{vd}

\begin{vd}%[0D6H4-4]%[Dự án đề cương 3 Khối NH24-25-Dot 2-Nguyễn Hoài Nam]
	$100$ học sinh tham dự Kỳ thi học sinh giỏi Toán (thang điểm là $20$). Kết quả được cho trong bảng sau:
	\begin{center}
		\begin{tabular}{|c|c|c|c|c|c|c|c|c|c|c|c|c|}
			\hline Điểm & $9$ & $10$ & $11$ & $12$ & $13$ & $14$ & $15$ & $16$ & $17$ & $18$ & $19$ & \\
			\hline Tần số & $1$ & $1$ & $3$ & $5$ & $8$ & $13$ & $19$ & $24$ & $14$ & $10$ & $2$ & $n=100$ \\
			\hline
		\end{tabular}
	\end{center}
\begin{enumerate}
	\item Tính số trung bình.
	\item Tìm phương sai và độ lệch chuẩn.
\end{enumerate}
	\loigiai{ 
		\begin{enumerate}
			\item Tính số trung bình.\\
			$\displaystyle\sum\limits_{i=1}^{11} n_{i} x_{i}=1\cdot9+1\cdot10+3\cdot11+5\cdot12+8\cdot13+13\cdot14+19\cdot15+24\cdot16+14\cdot17+10\cdot18+2\cdot19=1523$\\
			$\Rightarrow$ số trung bình là $\overline{x}=\dfrac{1523}{100}=15{,}23$.
			\item Ta có $\displaystyle\sum\limits_{i=1}^{11} n_{i} x_{i}=1523.$ và $\displaystyle\sum\limits_{i=1}^{11} n_{i} x_{i}^{2}=23591 \Rightarrow$ Phương sai là
			$$S^{2}=\dfrac{1}{n} \sum_{i=1}^{n} n_{i} x_{i}^{2}-\dfrac{1}{n^{2}}\left(\displaystyle\sum\limits_{i=1}^{n} n_{i} x_{i}\right)^{2}=\dfrac{1}{100}\cdot23591-\dfrac{1}{100^{2}}\cdot(1523)^{2}\approx3{,}96.$$
			Độ lệch chuẩn là $S=\sqrt{S^{2}}\approx\sqrt{3{,}96}\approx1{,}99$.	
		\end{enumerate}
	}
\end{vd}

\begin{vd}%[0D6H4-4]%[Dự án đề cương 3 Khối NH24-25-Dot 2-Nguyễn Hoài Nam]
	Một xạ thủ tập bắn, xạ thủ đó đã bắn $30$ viên đạn vào bia. Kết quả được cho trong bảng sau
	\begin{center}
		\begin{tabular}{|l|c|c|c|c|c|c|}
			\hline
			Điểm   & $6$ & $7$ & $8$ & $9$ & $10 $&      \\ \hline
			Tần số & $3$ & $4$ & $8$ & $9$ & $6$  & $n=30$ \\ \hline
		\end{tabular}
	\end{center}
\begin{enumerate}
	\item Tính điểm trung bình của xạ thủ.
	\item Tìm phương sai và độ lệch chuẩn.
\end{enumerate}

	\loigiai{ 
		\begin{enumerate}
			\item Điểm trung bình của xạ thủ.
			$$\overline{x}=\dfrac{1}{30}\left(3\cdot6+4\cdot7+8\cdot8+9\cdot9+6\cdot10 \right)\approx 8{,}37.$$
			\item Phương sai
			$$S^2=\dfrac{1}{30}\left[ 3\cdot{\left( 6-8{,}37 \right)^2}+4\cdot{\left( 7-8{,}37 \right)^2}+8\cdot{\left( 8-8{,}37 \right)^2}+9\cdot{\left( 9-8{,}37 \right)^2}+6\cdot{\left( 10-8{,}37 \right)^2} \right]\approx1{,}50.$$
			Độ lệch chuẩn là $S=\sqrt{S^{2}}\approx\sqrt{1{,}50}\approx1{,}22$.	
		\end{enumerate}
	}
\end{vd}
%-----------------------------------------------------------------------------
\subsection{BÀI TẬP RÈN LUYỆN}
\ind{PHẦN I.} \inden{Câu trắc nghiệm nhiều phương án lựa chọn. Mỗi câu hỏi học sinh chỉ chọn một phương án.}\\
\setcounter{ex}{0}
\Opensolutionfile{ans}[ans/0D6-Bai4-TN]


\begin{ex}%[0D6H3-4]%[Dự án đề cương 3 Khối NH24-25-Dot 2-Nguyễn Hoài Nam]
	[\textit{Trích đề thi HKI - Trường Thị xã Quảng Trị - Năm học 2024-2025}]
	Tứ phân vị thứ ba của mẫu số liệu $8$; $7$; $22$; $20$; $18$; $15$; $19$; $11$; $13$; $13$ là 
	\choice
	{$Q_3 = 18{,}5$}
	{\True $Q_3 = 19$}
	{$Q_3 = 20$}
	{$Q_3 = 19{,}5$}
	\loigiai
	{
		Sắp xếp mẫu số liệu theo thứ tự không giảm ta được 
		\[
		\begin{array}{llllllllll}
			7&8&11&13&13&15&18&19&20&22
		\end{array}
		\]		
		Vì cỡ mẫu $n=10$ nên tứ phân vị thứ ba của mẫu số liệu là trung vị của mẫu
		\[
		\begin{array}{lllll}
			15&18&19&20&22
		\end{array}
		\]	
		Do đó $Q_3=19$.
	}
\end{ex}


\begin{ex}%[0D6H4-4]%[Dự án đề cương 3 Khối NH24-25-Dot 2-Nguyễn Hoài Nam]
	[\textit{Trích đề thi HKI - Trường Thị xã Quảng Trị - Năm học 2024-2025}]
	Phương sai của mẫu số liệu cho bởi bảng tần số sau là
	\begin{center}
		\begin{tabular}{|c|c|c|c|c|c|c|}
			\hline Giá trị & $1$ & $4$ & $6$ & $10$ & $12$ & $17$ \\
			\hline Tần số & $1$ & $3$ & $5$ & $2$ & $4$ & $1$ \\
			\hline
		\end{tabular}
	\end{center}
	\choice
	{$19{,}875$}
	{\True $16{,}875$}
	{$18{,}875$}
	{$17{,}875$}
	\loigiai
	{
		Số trung bình của mẫu số liệu là 
		\begin{eqnarray*}
			\overline{x}&=& \dfrac{n_1 x_1+n_1 x_2+n_3 x_3+n_4 x_4+n_5 x_5+n_6 x_6}{n}\\ 
			&=& \dfrac{1+4\cdot 3+6\cdot 5+10\cdot 2+12\cdot 4+17}{16}\\ 
			&=& 8.
		\end{eqnarray*}	
		Phương sai của mẫu số liệu là 
		\allowdisplaybreaks
		\begin{eqnarray*}
			s^2&=&  \dfrac{1}{n}\left(n_1\cdot x_1^2 +n_2\cdot x_2^2+n_3\cdot x_3^2+n_4\cdot x_4^2+n_5\cdot x_5^2+n_6\cdot x_6^2\right)-\overline{x}^2\\ 
			&=& \dfrac{1}{16}\left(1\cdot 1^2 +3\cdot 4^2+5\cdot 6^2+2\cdot 10^2+4\cdot 12^2+1\cdot 17^2\right)-8^2\\ 
			&=& 16{,}875.
		\end{eqnarray*}
	}
\end{ex}


\begin{ex}%[0D6H4-4]%[Dự án đề cương 3 Khối NH24-25-Dot 2-Nguyễn Hoài Nam]
	[\textit{Trích đề thi HKI - Trường THPT Nguyễn Thượng Hiền - Năm học 2024-2025}]
	Tiền thưởng (triệu đồng) cho cán bộ và nhân viên trong công ty X được trình bày trong bảng tần số sau đây
	\begin{center}		
		\begin{tabular}{|c|r|r|r|r|r|r|}
			\hline
			Tiền thưởng $(x)$ & 2 & 3 & 4 & 5 & 6 & \\
			\hline
			Tần số $(n)$ & 5 & 15 & 10 & 6 & 7 & $n=43$ \\
			\hline
		\end{tabular}
	\end{center}
	Phương sai của bảng số liệu trên là
	\choice
	{$1{,}58$}
	{\True $1{,}59$}
	{$1{,}61$}
	{$1{,}57$}
	\loigiai{
		Giá trị trung bình $\overline{x}=\dfrac{5\cdot 2+15\cdot 3+10\cdot 4+6\cdot 5+7\cdot 6}{43}=\dfrac{167}{43}$.\\
		Phương sai $S^2= \dfrac{5\cdot 2^2+15\cdot 3^2+10\cdot 4^2+6\cdot 5^2+7\cdot 6^2}{43}-\left(\dfrac{167}{43}\right)^2\approx 1{,}59$.
	}
\end{ex}


\begin{ex}%[0D6H4-3]%[Dự án đề cương 3 Khối NH24-25-Dot 2-Nguyễn Hoài Nam]
	Cho mẫu số liệu có các tứ phân vị thứ nhất và thứ ba lần lượt là $Q_1=1{,}5$ và $Q_3=2{,}3$. Giá trị nào sau đây là một giá trị bất thường của mẫu số liệu trên?
	\choice
	{$0{,}5$}
	{\True $3{,}6$}
	{$2{,}1$}
	{$2{,}4$}
	\loigiai{
		Khoảng tứ phân vị $\Delta_Q = Q_3 - Q_1 = 0{,}8$.\\
		Giá trị $x$ là giá trị bất thường của mẫu số liệu nếu $\hoac{&x< Q_1 - 1{,}5\Delta_Q \\ &x> Q_3 + 1{,}5\Delta_Q} \Leftrightarrow \hoac{&x<0{,}3 \\ &x>3{,}5}.$\\
		Vậy $x=3{,}6$ là một giá trị ngoại lệ của mẫu số liệu.
	}
\end{ex}


\begin{ex}%[0D6H4-4]%[Dự án đề cương 3 Khối NH24-25-Dot 2-Nguyễn Hoài Nam]
	[\textit{Trích đề thi HKI - Trường THPT Nguyễn Khuyến - Năm học 2024-2025}]
	Thời gian chay $50$ m của $20$ học sinh được ghi lại trong bảng dưới đây
	\begin{center}
		\begin{tabular}{|>{\centering\arraybackslash}p{2.5cm}|>{\centering\arraybackslash}p{1cm}|>{\centering\arraybackslash}p{1cm}|>{\centering\arraybackslash}p{1cm}|>{\centering\arraybackslash}p{1cm}|>{\centering\arraybackslash}p{1cm}|}
			\hline
			Thời gian (giây)  & $8{,}3$ & 8${,}4$ & $8{,}5$ & $8{,}7$ & $8{,}8$ \\
			\hline
			Tần số            & $2$ & $3$ & $9$ & $5$ & $1$ \\
			\hline		
		\end{tabular}
	\end{center}
	Tính phương sai của mẫu số liệu trên (làm tròn kết quả đến chữ số thập phân thứ hai).
	\choice
	{$0{,}01$}
	{$0{,}2$}
	{\True $0{,}02$}
	{$0{,}11$}
	\loigiai{
		Giá trị trung bình $\overline{x}=\dfrac{8{,}3\cdot 2 +8{,}4\cdot 3+8{,}5\cdot 9+8{,}7\cdot 5 +8{,}8\cdot 1}{2+3+9+5+1}=8{,}53$.\\
		Phương sai của mẫu số liệu 
		$$s^2=\dfrac{(8{,}3-8{,}53)^2\cdot 2 +(8{,}4-8{,}53)^2\cdot 3 +(8{,}5-8{,}53)^2\cdot 9+(8{,}7-8{,}53)^2\cdot 5 +(8{,}8-8{,}53)^2\cdot 1}{2+3+9+5+1}\approx 0{,}02.$$
	}
\end{ex}


\begin{ex}%[0D6N4-2]%[Dự án đề cương 3 Khối NH24-25-Dot 2-Nguyễn Hoài Nam]
	[\textit{Trích đề thi HKI - Trường THPT Phú Nhuận - Năm học 2024-2025}]
	Mẫu số liệu sau ghi rõ số tiền thưởng tết Nguyên Đán của $13$ nhân viên của một công ty (đơn vị: triệu đồng): $10$; $10$; $11$; $12$; $12$; $13$; $14$; $15$; $18$; $20$; $20$; $21$; $28$. Khoảng biến thiên của mẫu số liệu trên là
	\choice
	{$28$}
	{$14$}
	{\True $18$}
	{$10$}
	\loigiai{
		Khoảng biến thiên của mẫu số liệu là $28-10 = 18$.
	}
\end{ex}


\begin{ex}%[0D6N4-2]%[Dự án đề cương 3 Khối NH24-25-Dot 2-Nguyễn Hoài Nam]
	[\textit{Trích đề thi HKI - Trường THPT Bùi Thị Xuân - Năm học 2024-2025}]
	Điểm số của Tổ $1$ được ghi nhận lại như sau:
	$$4 ; 10 ; 7 ; 8 ; 9 ; 7 ; 6 ; 5 ; 9 ; 10.$$
	Khoảng biến thiên của mẫu số liệu trên là
	\choice
	{$3$}
	{$5$}
	{\True $6$}
	{$4$}
	\loigiai{
		Khoảng biến thiên là $10-4=6$.
	}
\end{ex}



\begin{ex}%[0D6N4-2]%[Dự án đề cương 3 Khối NH24-25-Dot 2-Nguyễn Hoài Nam] 
	Cho mẫu số liệu như sau:
	\begin{center}
		\begin{tabular}{|c|c|c|c|c|c|}
			\hline
			Giá trị $x_{i}$ & $10$ & $20$ & $30$ & $40$ & $50$ \\
			\hline
			Tần số $n_{i}$ & $3$ & $4$ & $7$ & $9$ & $1$ \\
			\hline
		\end{tabular}
	\end{center}
	Khoảng tứ phân vị của mẫu số liệu trên bằng 
	\choice
	{\True $20$}
	{$22$}
	{$30$}
	{$40$}
	\loigiai{
		Cỡ mẫu $n=24$.\\
		Ta có trung vị $M_{e}$ là số trung bình cộng của giá trị thứ $12$ và $13$, suy ra $M_{e}=\dfrac{30+30}{2}=30$.\\
		Tứ phân vị thứ nhất $Q_{1}$ là số trung bình cộng của giá trị thứ $6$ và $7$, suy ra $Q_{1}=\dfrac{20+20}{2}=20$.\\
		Tứ phân vị thứ ba $Q_{3}$ là số trung bình cộng của giá trị thứ $18$ và $19$, suy ra $Q_{3}=\dfrac{40+40}{2}=40$.\\
		Vậy khoảng tứ phân vị $\Delta_Q = Q_3 - Q_1 = 40-20=20$.
	}
\end{ex}


\begin{ex}%[0D6N4-4]%[Dự án đề cương 3 Khối NH24-25-Dot 2-Nguyễn Hoài Nam] 
	Cho mẫu số liệu như sau có phương sai bằng $2{,}25$. Độ lệch chuẩn của mẫu số liệu trên bằng
	\choice
	{\True $1{,}5$}
	{$2{,}25$}
	{$0{,}5$}
	{$5{,}0625$}
	\loigiai{
		Độ lệch chuẩn của mẫu số liệu trên bằng $\sqrt{2{,}25} = 1{,}5$.
	}
\end{ex}


\begin{ex}%[0D6N4-4][Dự án đề cương 3 Khối NH24-25-Dot 2-Nguyễn Hoài Nam]
	[\textit{Trích đề thi HKI - Trường THPT Nguyễn Chí Thanh - Năm học 2024-2025}]
	Điểm kiểm tra giữa kỳ $2$ của một học sinh lớp $10$ như sau: $2$, $4$, $6$, $8$, $10$. Phương sai của mẫu số liệu trên là bao nhiêu?
	\choice
	{$9$}
	{\True $8$}
	{$6$}
	{$10$}
	\loigiai{
		Điểm trung bình kiểm tra giữa kỳ $2$ của học sinh là $(2+4+6+8+10): 5=6$. \\
		Phương sai của mẫu số liệu trên là $s^2=\dfrac{1}{5}\left(2^2+4^2+6^2+8^2+10^2\right)-6^2=8$.
	}
\end{ex}

\begin{ex}%[0D6N4-2]%[Dự án đề cương 3 Khối NH24-25-Dot 2-Nguyễn Hoài Nam]
	Số sản phẩm sản xuất mỗi ngày của một phân xưởng trong $9$ ngày liên tiếp được ghi lại như sau: $27$; $26$; $21$; $28$; $25$; $30$; $26$; $23$; $26$. Khoảng biến thiên của mẫu số liệu trên là
	\choice
	{$8$}
	{$5$}
	{$6$}
	{\True $9$}
	\loigiai{
		Khoảng biến thiên $R=30-21=9$.
	}
\end{ex}

\begin{ex}%[0D6N4-2]%[Dự án đề cương 3 Khối NH24-25-Dot 2-Nguyễn Hoài Nam]
	Khoảng biến thiên của mẫu số liệu $27$; $15$; $18$; $30$; $19$; $40$; $100$; $9$; $46$; $10$; $200$ là
	\choice
	{$173$}
	{\True $191$}
	{$91$}
	{$182$}
	\loigiai{
		Ta có $R=200-9=191$.
	}
\end{ex}

\begin{ex}%[0D6N4-2]%[Dự án đề cương 3 Khối NH24-25-Dot 2-Nguyễn Hoài Nam]
	[\textit{Trích đề thi HKI - Trường THPT Hướng Hóa, Quảng Trị - Năm học 2023-2024}]
	Cho mẫu số liệu $152$; $154$; $156$; $158$; $160$. Khoảng tứ phân vị của mẫu số liệu trên bằng
	\choice
	{$6$}
	{$3$}
	{$5$}
	{\True $4$}
	\loigiai{
		Ta có $Q_1=154$, $Q_3=158$.\\
		Khoảng tứ phân vị của mẫu số liệu trên là $\Delta_Q=158-154=4$.
	}
\end{ex}

\begin{ex}%[0D6N4-2]%[Dự án đề cương 3 Khối NH24-25-Dot 2-Nguyễn Hoài Nam]
	Nhiệt độ của thành phố Thanh Hóa (đơn vị: độ C) ghi nhận trong $6$ ngày liên tiếp lần lượt là: $25$; $22$; $30$; $34$; $35$; $27$. Khoảng tứ phân vị của mẫu số liệu trên là
	\choice
	{$12$}
	{$11$}
	{$13$}
	{\True $9$}
	\loigiai{
		Sắp xếp mẫu số liệu theo thứ tự không giảm, ta được 
		$$22 \quad 25 \quad 27 \quad 30 \quad 34 \quad 35.$$
		Do đó $Q_1=25$, $Q_3=34$.\\
		Khoảng tứ phân vị $\Delta_Q=34-25=9$.
	}	
\end{ex}

\begin{ex}%[0D6N4-2]%[Dự án đề cương 3 Khối NH24-25-Dot 2-Nguyễn Hoài Nam]
	[\textit{Trích đề thi HKI - Trường THPT Chuyên Lê Hồng Phong, TP.HCM - Năm học 2024-2025}]
	Cho mẫu số liệu $1$; $3$; $4$; $6$; $8$; $9$; $11$. Phương sai của mẫu số liệu trên bằng
	\choice
	{\True $\dfrac{76}{7}$}
	{$\sqrt{\dfrac{76}{7}}$}
	{$7$}
	{$49$}
	\loigiai{
		Số trung bình của mẫu số liệu là $\overline{x}=\dfrac{1+3+4+6+8+9+11}{7}=6$.\\
		Phương sai của mẫu số liệu là $S^2=\dfrac{1}{7}\left(1^2+3^2+4^2+6^2+8^2+9^2+11^2\right)-6^2=\dfrac{76}{7}$.
	}
\end{ex}

\begin{ex}%[0D6H4-2]%[Dự án đề cương 3 Khối NH24-25-Dot 2-Nguyễn Hoài Nam]
	[\textit{Trích đề thi HKI - Trường THPT Phan Bội Châu, Lâm Đồng - Năm học 2024-2025}]%GV ĐIỀU CHỈNH CHỖ NÀY
	Điểm kiểm tra Toán của một nhóm gồm $7$ học sinh lớp $11$ được ghi lại lần lượt là $2$; $4$; $5$; $6$; $7$; $8$; $10$. Phương sai của mẫu số liệu đã cho là
	\choice
	{$2$}
	{$3$}
	{\True $6$}
	{$5$}
	\loigiai{
		Số trung bình của mẫu số liệu là $\overline{x}=\dfrac{2+4+5+6+7+8+10}{7}=6$.\\
		Phương sai của mẫu số liệu là $S^2=\dfrac{1}{7}\left(2^2+4^2+5^2+6^2+7^2+8^2+10^2\right)-6^2=6.$
	}
\end{ex}

\begin{ex}%[0D6H4-2]%[Dự án đề cương 3 Khối NH24-25-Dot 2-Nguyễn Hoài Nam]
	Điểm số $8$ lượt bắn của một vận động viên bắn súng được ghi lại dưới bảng sau:
	\begin{center}
		\begin{tabular}{|l|c|c|c|c|c|c|c|c|}
			\hline
			Lượt & Lượt $1$ & Lượt $2$ & Lượt $3$ & Lượt $4$ & Lượt $5$ & Lượt $6$ & Lượt $7$ & Lượt $8$ \\
			\hline
			Điểm & $5$ & $7$ & $7$ & $10$ & $9$ & $8$ & $8$ & $6$ \\ 
			\hline
		\end{tabular}
	\end{center}
	Độ lệch chuẩn của mẫu số liệu trên bằng
	\choice
	{\True $\sqrt{2{,}25}$}
	{$\sqrt{2{,}15}$}
	{$\sqrt{1{,}5}$}
	{$\sqrt{2{,}5}$}
	\loigiai{
		Số trung bình của mẫu số liệu là $\overline{x}=\dfrac{5+7+7+10+9+8+8+6}{8}=7{,}5$.\\
		Phương sai của mẫu số liệu là $S^2=\dfrac{1}{8}\left(5^2+7^2+7^2+10^2+9^2+8^2+8^2+6^2\right)-7{,}5^2=2{,}25$.\\
		Độ lệch chuẩn của mẫu số liệu là $S=\sqrt{2{,}25}$.
	}
\end{ex}

\begin{ex}%[0D6H4-2]%[Dự án đề cương 3 Khối NH24-25-Dot 2-Nguyễn Hoài Nam]
	Kết quả kiểm tra của $45$ học sinh lớp $10A$ được ghi lại ở bảng phân bố sau đây:
	\begin{center}
		\begin{tabular}{|c|c|c|c|c|c|c|}
			\hline
			Điểm thi&$5$&$6$&$7$&$8$&$9$&$10$\\
			\hline
			Tần số&$5$&$7$&$12$&$14$&$3$&$4$\\
			\hline
		\end{tabular}
	\end{center}
	Độ lệch chuẩn của bảng số liệu trên gần nhất với giá trị nào sau đây?
	\choice
	{\True $\sqrt{1{,}87}$}
	{$\sqrt{1{,}88}$}
	{$\sqrt{1{,}89}$}
	{$\sqrt{1{,}86}$}
	\loigiai{
		Số trung bình của mẫu số liệu là $\overline{x}=\dfrac{5\cdot 5+6\cdot 7+7\cdot 12+8\cdot 14+9\cdot 3+10\cdot 4}{45}=\dfrac{22}{3}$.\\
		Phương sai của mẫu số liệu là $S^2=\dfrac{1}{45}\left(5\cdot 5^2+7\cdot 6^2+12\cdot 7^2+14\cdot 8^2+3\cdot 9^2+4\cdot 10^2\right)-\left(\dfrac{22}{3}\right)^2=\dfrac{28}{15}\approx 1{,}87.$\\
		Độ lệch chuẩn của mẫu số liệu là $S\approx\sqrt{1{,}87}$. 
	}
\end{ex}

\begin{ex}%[0D6H4-2]%[Dự án đề cương 3 Khối NH24-25-Dot 2-Nguyễn Hoài Nam]
	Điểm trung bình học kỳ $1$ môn Toán của các học sinh ở tổ $1$ lớp $10A$ được ghi lại ở bảng sau:
	\begin{center}
		\begin{tabular}{|c|c|c|c|c|c|c|c|c|c|c|c|}
			\hline
			$8{,}6$&$8{,}2$&$8{,}1$&$8{,}8$&$8{,}8$&$8{,}1$&$8{,}2$&$8{,}0$&$6{,}5$&$9{,}8$&$7{,}8$&$7{,}8$\\
			\hline
		\end{tabular}
	\end{center}
	Giá trị bất thường của mẫu số liệu trên là
	\choice
	{\True $6{,}5$}
	{$9{,}8$}
	{$7{,}8$}
	{$8{,}8$}
	\loigiai{
		Sắp xếp mẫu số liệu theo thứ tự không giảm, ta được 
		$$6{,}5 \ 7{,}8 \  7{,}8 \ 8{,}0 \ 8{,}1 \ 8{,}1 \ 8{,}2 \ 8{,}2 \ 8{,}6 \ 8{,}8 \ 8{,}8 \ 9{,}8.$$
		Do đó $Q_1=7{,}9$, $Q_3=8{,}7$, $\Delta_Q=8{,}7-7{,}9=0{,}8$.\\
		Suy ra $Q_1-1{,}5\cdot\Delta_Q=6{,}7$ và $Q_3+1{,}5\cdot\Delta_Q=9{,}9$.\\
		Vậy mẫu số liệu trên có một giá trị bất thường là $6{,}5$.
	}
\end{ex}
\begin{ex}%[0D6H4-2]%[Dự án đề cương 3 Khối NH24-25-Dot 2-Nguyễn Hoài Nam]
	Mẫu số liệu dưới đây cho biết cân nặng của một số học sinh lớp $10$ tại một trường THPT (đơn vị: kg):
	\begin{center}
		\begin{tabular}{|c|c|c|c|c|c|c|c|c|c|c|c|c|}
			\hline
			$43$&$50$&$43$&$48$&$45$&$45$&$38$&$48$&$35$&$50$&$43$&$45$&$48$\\
			\hline
		\end{tabular}
	\end{center}
	Giá trị bất thường của mẫu số liệu trên là
	\choice
	{\True $35$}
	{$45$}
	{$50$}
	{$45$}
	\loigiai{
		Sắp xếp mẫu số liệu theo thứ tự không giảm, ta được 
		$$35 \ 38 \  43 \ 43 \ 43 \ 45 \ 45 \ 45 \ 48 \ 48 \ 48 \ 50 \ 50.$$
		Do đó $Q_1=43$, $Q_3=48$, $\Delta_Q=48-43=5$.\\
		Suy ra $Q_1-1{,}5\cdot\Delta_Q=35{,}5$ và $Q_3+1{,}5\cdot\Delta_Q=55{,}5$.\\
		Vậy mẫu số liệu trên có một giá trị bất thường là $35$.}
\end{ex}




\Closesolutionfile{ans}

\ind{PHẦN II.} \inden{Câu trắc nghiệm đúng sai. Trong mỗi ý a), b), c), d) ở mỗi câu, học sinh chọn đúng hoặc sai.}\\
\setcounter{ex}{0}
\Opensolutionfile{ans}[ans/0D6-Bai4-DS]

\begin{ex}%[0D6H4-4]%[Dự án đề cương 3 Khối NH24-25-Dot 2-Nguyễn Hoài Nam]
		[\textit{Trích đề thi HKI - Trường THPT Thái Phiên, Hải Phòng - Năm học 2024-2025}]
	Tiến hành đo huyết áp của $8$ người ta thu được kết quả sau 
	$\begin{array}{llllllll}77 & 105 & 117 & 84 & 96 & 72 & 105 & 124.\end{array}$
	\choiceTF
	{ Khoảng biến thiên của mẫu số liệu là $50$}
	{\True Giá trị trung bình của mẫu số liệu là $97{,}5$}
	{ Khoảng tứ phân vị của mẫu số liệu là $\Delta_{Q}=30$}
	{\True Mẫu số liệu không có giá trị bất thường}
	\loigiai{ 
		\begin{itemchoice}
			\itemch Khoảng biến thiên của mẫu số liệu là $R=124-72=52$. 
			\itemch Số trung bình của mẫu số liệu là $\bar{x}=\dfrac{77+105+117+84+96+72+105+124}{8}=97{,}5$.
			\itemch Sắp xếp mẫu số liệu theo thứ tự không giảm  $ \begin{array}{llllllll}72&77 & 84 & 96 & 105 & 105 & 117 & 124\end{array}$\\
			Ta có $Q_2=\dfrac{96+105}{2}=100{,}5 ; \quad Q_1=\dfrac{77+84}{2}=80{,}5 ; \quad Q_3=\dfrac{105+117}{2}=111$.\\
			Khoảng tứ phân vị của mẫu số liệu là $\Delta_Q=Q_3-Q_1=111-80{,}5=30{,}5$. 		
			\itemch Ta có $Q_1-1{,}5 \Delta Q=80{,}5-1{,}5\cdot 30{,}5=34{,}75$ và
			$$
			Q_3+1{,}5 \cdot \Delta Q=111+1{,}5 \cdot 30{,}5=156{,}75.
			$$		
			Tất cả các giá trị của mẫu số liệu đều thuộc đoạn $[34{,}75 ; 156{,}75]$ nên mẫu  số liệu không có giá trị bất thường. 
		\end{itemchoice}
	}
\end{ex}
\begin{ex}%[0D6V4-2]%[Dự án đề cương 3 Khối NH24-25-Dot 2-Nguyễn Hoài Nam]
	[\textit{Trích đề thi HKI - Trường THPT Lê Hồng Phong, Đồng Nai - Năm học 2024-2025}]
	Hai xạ thủ $A$ và $B$ mỗi người bắn $10$ phát đạn. Kết quả được ghi lại trong bảng sau:
	\begin{center}
		\begin{tabular}{|c|c|c|c|c|c|c|c|c|c|c|}
			\hline
			\textbf{Xạ thủ}& & & & & & & & & &  \\
			\hline
			$A$ & $7$ & $9$ & $6$ & $9$ & $8$ & $6$ & $8$ & $7$ & $10$ & $8$ \\
			\hline
			$B$ & $8$ & $7$ & $8$ & $9$ & $6$ & $7$ & $7$ & $9$ & $9$ & $8$ \\
			\hline
		\end{tabular}
	\end{center}
	\choiceTF
	{\True Điểm thấp nhất của xạ thủ $A$ là $6$}
	{Điểm trung bình của xạ thủ $A$ cao hơn điểm trung bình của xạ thủ $B$}
	{\True Độ lệch chuẩn của xạ thủ $A$ lớn hơn độ lệch chuẩn của xạ thủ $B$}
	{Xạ thủ $A$ bắn đều hơn xạ thủ $B$}
	\loigiai{
		\begin{itemchoice}
			\itemch Quan sát bảng, điểm thấp nhất của xạ thủ $A$ là $6$. 
			
			\itemch Tính điểm 
			$$
			\overline{x}_A = \dfrac{7 + 9 + 6 + 9 + 8 + 6 + 8 + 7 + 10 + 8}{10} = 7{,}8.	$$
			$$
			\overline{x}_B = \dfrac{8 + 7 + 8 + 9 + 6 + 7 + 7 + 9 + 9 + 8}{10} = 7{,}8.
			$$
			Hai giá trị bằng nhau.
			
			\itemch Phương sai xạ thủ $A$ là
			$$
			s_A^2 = \dfrac{(x_1 - \overline{x})^2 + \cdots + (x_{10} - \overline{x})^2}{10} = 1{,}56 \Rightarrow s_A \approx \sqrt{1{,}56} \approx 1{,}249.
			$$
			Phương sai xạ thủ $B$ là
			$$
			s_B^2 = 0{,}96 \Rightarrow s_B \approx \sqrt{0{,}96} \approx 0{,}980.
			$$
			Do đó $s_A > s_B$.
			
			\itemch Độ lệch chuẩn của xạ thủ $A$ lớn hơn $B$ $\Rightarrow$ xạ thủ $B$ bắn đều hơn.
		\end{itemchoice} 
	}
\end{ex}

\begin{ex}%[0D6H4-4]%[Dự án đề cương 3 Khối NH24-25-Dot 2-Nguyễn Hoài Nam]
	Cho bảng số liệu điểm kiểm tra môn Toán của $20$ học sinh lớp 10A5 như sau:	
	\begin{center}
		\renewcommand{\arraystretch}{1.2}
		\begin{tabular}{|c|c|c|c|c|c|c|c|c|c|}
			\hline
			\textbf{Điểm} & $4$ & $5$ &$ 6$ & $7$ & $8$ & $9$ & $10$ & Cộng \\
			\hline
			\textbf{Số học sinh} & $1$ & $2$ &$ 3$ & $4$ & $5$ & $4$ & $1$ & $20$ \\
			\hline
		\end{tabular}
	\end{center}
	\choiceTF
	{\True  Khoảng biến thiên mẫu số liệu là $6$}
	{\True Tứ phân vị thứ nhất của mẫu số liệu là $6$}
	{ Khoảng tứ phân vị của mẫu số liệu là là $3$}
	{ Phương sai của mẫu số liệu là $S^2=2$}
	
	\loigiai{
		\begin{itemchoice}
			\itemch Khoảng biến thiên của mẫu số liệu là $\Delta_x=10-4=6$.
			
			\itemch Ta có
			\[
			\dfrac{1}{4} \cdot 20 = 5 \Rightarrow Q_1\text{ là giá trị thứ 5 trong dãy số liệu đã sắp xếp.}
			\]
			Vậy $Q_1=6$.
			\itemch Ta có
			\[
			\dfrac{3}{4} \cdot 20 = 15 \Rightarrow Q_3\text{ là giá trị thứ 15 trong dãy số liệu đã sắp xếp}\Rightarrow Q_3=8
			\]
			Do đó khoảng tứ phân vị là $\Delta_Q=Q_3-Q_1=8-6=2$.
			\itemch Trung bình cộng
			\[
			\overline{x} = \dfrac{1\cdot4 + 2\cdot5 + 3\cdot6 + 4\cdot7 + 5\cdot8 + 4\cdot9 + 1\cdot10}{20}
			= \dfrac{146}{20}=7{,}3.
			\]
			Phương sai $$S^2=\dfrac{\left(4-7{,}3\right)^2\cdot 1+\left(5-7{,}3\right)^2\cdot 2+\left(6-7{,}3\right)^2\cdot 3+\left(7-7{,}3\right)^2\cdot 4+\left(8-7{,}3\right)^2\cdot 5+\left(9-7{,}3\right)^2\cdot 4+\left(10-7{,}3\right)^2\cdot 1}{20}=2{,}41.
			$$
		\end{itemchoice}
	}
\end{ex}

\begin{ex}%[0D6V4-3]%[Dự án đề cương 3 Khối NH24-25-Dot 2-Nguyễn Hoài Nam]
	Một cơ sở chăn nuôi gia cầm tiến hành thử nghiệm giống gà đẻ trứng mới. Khi gà đã cho trứng, họ tiến hành khảo sát với $20$ quả được cân nặng (gam) như sau
	
	\[
	\begin{array}{cccccccccc}
		40 & 42 & 36 & 38 & 40 & 42 & 29 & 48 & 43 & 43 \\
		41 & 41 & 39 & 44 & 45 & 41 & 40 & 39 & 42 & 41
	\end{array}
	\]
	\choiceTF
	{\True  Giá trị nhỏ nhất của mẫu là $29$}
	{\True  Khoảng biến thiên của mẫu số liệu là $19$}
	{ Khoảng tứ phân vị là $\Delta_Q = 2$}
	{\True Các giá trị bất thường là $29$ và $48$}
	
	\loigiai{
		Sắp xếp mẫu số liệu theo thứ tự không giảm
		\[
		29;\ 36;\ 38;\ 39;\ 39;\ 40;\ 40;\ 40;\ 41;\ 41;\ 41;\ 41;\ 42;\ 42;\ 42;\ 43;\ 43;\ 44;\ 45;\ 48
		\]
		\begin{itemchoice}
			\itemch Giá trị nhỏ nhất là $29$.
			\itemch Ta có giá trị lớn nhất là $48$. Do đó $\Delta_x=48-29=19$. 
			
			\itemch Ta có
			\[
			Q_1 = \dfrac{39 + 40}{2} = 39{,}5;
			\quad Q_2 =  \dfrac{41 + 41}{2} = 41;\quad
			Q_3  = \dfrac{42 + 43}{2} = 42{,}5.
			\]
			Do đó khoảng tứ phân vị
			\[
			\Delta_Q = Q_3 - Q_1 = 42{,}5 - 39{,}5 = 3.
			\]
			
			\itemch Xác định giá trị bất thường
			\[
			Q_1 - 1{,}5 \cdot \Delta_Q = 39{,}5 - 1{,}5 \cdot 3 = 35;
			\quad Q_3 + 1{,}5 \cdot \Delta_Q = 42{,}5 + 4{,}5 = 47.
			\]
			Ta có $29 < 35$ và $48 > 47$ nên giá trị bất thường là $29$ và $48$.
		\end{itemchoice}
	}
\end{ex}

\begin{ex}%[0D6H4-4]%[Dự án đề cương 3 Khối NH24-25-Dot 2-Nguyễn Hoài Nam]
	Bạn Hưng và bạn Thịnh thống kê kết quả chiều cao (đơn vị: xăng-ti-mét) của $5$ cây nguyệt quế mà mỗi người trồng sau một thời gian như sau
	
	\begin{center}
		\begin{tabular}{|c|ccccc|}
			\hline
			\textbf{Cây của bạn Hưng} & $35$ & $36$ & $38$ & $36$ & $37$ \\
			\hline
			\textbf{Cây của bạn Thịnh} & $30$ & $35$ & $38$ & $41$ & $33$ \\
			\hline
		\end{tabular}
	\end{center}
	\choiceTF
	{\True  Khoảng tứ phân vị của mẫu số liệu cây của bạn Hưng là $\Delta_{Q_H} = 2$}
	{Khoảng biến thiên của số liệu cây của bạn Thịnh là $\Delta_{x_T} = 8$}
	{ Phương sai của mẫu số liệu cây của bạn Hưng lớn hơn phương sai của mẫu cây bạn Thịnh}
	{\True Các cây nguyệt quế của bạn Hưng phát triển chiều cao đồng đều hơn}
	\loigiai{
		
		\begin{itemchoice}
			\itemch Sắp xếp mẫu số liệu của bạn Hưng theo thứ tự không giảm
			\begin{center}
				\begin{tabular}{ccccc}
					$35$ & $36$ & $36$ & $37$ & $38$ \\
				\end{tabular}
			\end{center}
			Ta thấy $Q_1=\dfrac{35+36}{2}=35{,}5;\ Q_3=\dfrac{37+38}{2}=37{,}5$.\\
			Do đó khoảng tứ phân vị của mẫu số liệu cây của bạn Hưng là $\Delta_{Q_H}=2$.
			\itemch Khoảng biến thiên của số liệu cây của bạn Thịnh là $\Delta_{x_T} = 41-30=11$.
			
			\itemch 
			Tính trung bình cộng của mẫu số liệu cây bạn Hưng
			\[
			\overline{x}_H = \dfrac{35 + 36 + 38 + 36 + 37}{5} = \dfrac{182}{5} = 36{,}4.
			\]
			Trung bình cộng của mẫu số liệu cây bạn Thịnh
			\[
			\overline{x}_T = \dfrac{30 + 35 + 38 + 41 + 33}{5} = \dfrac{177}{5} = 35{,}4.
			\]
			Tính phương sai
			\begin{eqnarray*}
				s_H^2& =& \dfrac{(35 - 36{,}4)^2 + (36 - 36{,}4)^2 + (38 - 36{,}4)^2 + (36 - 36{,}4)^2 + (37 - 36{,}4)^2}{5}\\
				&	= &\dfrac{1{,}96 + 0{,}16 + 2{,}56 + 0{,}16 + 0{,}36}{5} = \dfrac{5{,}2}{5} = 1{,}04.\\
				s_T^2 &=& \dfrac{(30 - 35{,}4)^2 + (35 - 35{,}4)^2 + (38 - 35{,}4)^2 + (41 - 35{,}4)^2 + (33 - 35{,}4)^2}{5}\\
				&	=& \dfrac{29{,}16 + 0{,}16 + 6{,}76 + 31{,}36 + 5{,}76}{5} = \dfrac{73{,}2}{5} = 14{,}64.
			\end{eqnarray*}
			Ta thấy $s_H^2 = 1{,}04 < 14{,}64 = s_T^2$.
			\itemch Vì phương sai bạn Hưng nhỏ hơn nên các cây của bạn Hưng phát triển đồng đều hơn.
		\end{itemchoice}
	}
\end{ex}

\Closesolutionfile{ans}


\ind{PHẦN III.} \inden{Câu trắc nghiệm trả lời ngắn.}\\
\setcounter{ex}{0}
\Opensolutionfile{ans}[ans/0D6-Bai4-TLN]

\begin{ex}%[0D6H4-2]%[Dự án đề cương 3 Khối NH24-25-Dot 2-Nguyễn Hoài Nam]
	[\textit{Trích đề thi HKI - Trường THPT Nguyễn Huệ, Đồng Nai - Năm học 2024-2025}]
	Sản lượng lúa (đơn vị là tạ) của $40$ thửa ruộng thí nghiệm có cùng diện tích được trình bày trong bảng tấn số sau đây
	\begin{center}
		\begin{tabular}{|c|c|c|c|c|c|c|c|c|c|c|c|c|} 
			\hline Sản lượng  &$20$&$21$&$22$&$23$&$24$\\
			\hline
			Số thửa ruộng  &$5$&$8$&$11$&$10$&$6$\\
			\hline
		\end{tabular}
	\end{center}
	Tính khoảng tứ phân vị của mẫu số liệu.
	\shortans[oly]{$2$}
	\loigiai{
		Ta có $Q_1=21$; $Q_3=23$.\\
		Khoảng tứ phân vị $\Delta_Q=23-21=2$.
	}
\end{ex}
\begin{ex}%[0D6V4-4]%[Dự án đề cương 3 Khối NH24-25-Dot 2-Nguyễn Hoài Nam]
	[\textit{Trích đề thi HKI - Trường THPT Đông Du, Đắk Lắk - Năm học 2024-2025}]
	Cho dãy số liệu thống kê $10,\,8,\,6,\,2,\,4$. Độ lệch chuẩn của mẫu số liệu bằng bao nhiêu? (Làm tròn đến hàng chục).
	\shortans[oly]{$2{,}8$}
	\loigiai{
		$\overline{x}=\dfrac{1}{5}(10 \cdot 1+8 \cdot 1+6 \cdot 1+2 \cdot 1+4 \cdot 1)
		=6.$\\
		$			s^{2}=\dfrac{1}{5}\left[1 \cdot(10-6)^{2}+1 \cdot(8-6)^{2}+1 \cdot(6-6)^{2}+1 \cdot(2-6)^{2}+1 \cdot(4-6)^{2}\right]=8.\\$
		Vậy  $s=\sqrt{s^{2}} \approx 2{,}8$.
	}    
\end{ex}
\begin{ex}%[0D6V4-4]%[Dự án đề cương 3 Khối NH24-25-Dot 2-Nguyễn Hoài Nam]
	Bảng sau đây cho ta biết số cuốn sách mà học sinh của một lớp ở trường Trung học phổ thông đã đọc trong năm $2016$. 
	\begin{center}
		\begin{tabular}{|>{\centering\arraybackslash}p{3cm}|>{\centering\arraybackslash}p{1cm}|>{\centering\arraybackslash}p{1cm}|>{\centering\arraybackslash}p{1cm}|>{\centering\arraybackslash}p{1cm}|>{\centering\arraybackslash}p{1cm}|>{\centering\arraybackslash}p{1cm}|>{\centering\arraybackslash}p{2cm}|}
			\hline 
			Số sách & $1$ & $2$ & $3$ & $4$ & $5$ & $6$  & Cộng \\ 
			\hline
			Tần số  & $10$ & $x$ & $8$ & $6$ & $y$ & $3$ & $40$ \\ 
			\hline 
		\end{tabular}  
	\end{center}
	Tính $x \cdot y$, biết rằng phương sai của bảng số liệu $s^2 \approx 2,52$.
	\shortans[oly]{$40$}
	\loigiai{
		Ta lập hệ phương trình hai ẩn $x$ và $y$. Ta có hệ phương trình\\
		$\heva{&4x^2+25y^2+20xy+144x-240y-1632&=0 \\ &x + y&=13} \Leftrightarrow \heva{&x=8\\&y=5.}$\\
		Suy ra $x \cdot y=40$.
	}
\end{ex}
\begin{ex}%[0D6V4-4]%[Dự án đề cương 3 Khối NH24-25-Dot 2-Nguyễn Hoài Nam]
	Tiền thưởng (triệu đồng) cho cán bộ và nhân viên trong công ty được trình bày trong bảng tần số sau đây
	\begin{center}
		\begin{tabular}{|c|c|c|c|c|c|c|}
			\hline
			Tiền thưởng $(x)$ & $2$ & $3$ & $4$ & $5$ & $6$ &      \\ \hline
			Tần số $(n)$    & $5$  & $15$  & $10$ & $6$ & $7$  & $n=43$ \\ \hline
		\end{tabular}
	\end{center}
	Tính phương sai của mẫu số liệu trên. (Làm tròn đến hàng phần trăm).
	\shortans[oly]{$1{,}59$}
	\loigiai{	
		Ta có
		$$\overline{x}=\dfrac{1}{43}(5\cdot2+15\cdot3+10\cdot4+6\cdot5+7\cdot6)\approx 3{,}88.$$
		Phương sai 
		$$s^{2}=\dfrac{1}{43}\left[5(2-3{,}88)^{2}+15 (3-3{,}88)^{2}+10(4-3{,}88)^{2}+6(5-3{,}88)^{2}+7(6-3{,}88)^{2}\right]\approx1{,}59.$$	
	}
\end{ex}
\begin{ex}%[0D6V4-3]%[Dự án đề cương 3 Khối NH24-25-Dot 2-Nguyễn Hoài Nam]
	Một cảnh sát giao thông ghi tốc độ (đơn vị: km/h) của $25$ chiếc xe qua trạm như sau:
	\begin{center}
		\begin{tabular}{ccccccccccccccc}
			$20$ & $41$ & $41$ & $80$ & $40$ & $52$ & $52$ & $52$ & $60$ & $55$ & $60$ & $60$ & $62$  \\  
			$60$ & $65$ & $60$ & $65$ & $135$ & $70$ & $70$ & $65$ & $75$ & $75$ & $70$ & $55$ &  \\  
		\end{tabular} 
	\end{center}
	Hãy tìm  số liệu bất thường trong mẫu số liệu trên.
	\shortans[oly]{$135$}
	\loigiai{
		Sắp xếp các số liệu trong mẫu theo thứ tự không giảm ta có
		\begin{center}
			\begin{tabular}{|c|c|c|c|c|c|c|c|c|c|c|c|c|}
				\hline 
				Tốc độ    & $20$ & $40$& $41$& $52$& $55$& $60$& $62$& $65$& $70$ & $75$& $80$& $135$\\
				\hline 
				Số lần xuất hiện  & $1$ & $1$ & $2 $ & $3$& $2$& $5$& $1$& $3$& $3$& $2$ & $1$& $1$\\
				\hline
			\end{tabular}
		\end{center} 
		Từ bảng số liệu ta tìm được
		Số trung vị $Q_2=60$, tứ phân vị thứ nhất $Q_1= 52$, tứ phân vị thứ 3 $Q_3= 70$.\\
		Khoảng tứ phân vị $\Delta_{Q} = Q_3-Q_1=70 - 52 = 18$.\\
		Ta có $\left[Q_1-1{,}5 \cdot  \Delta_{Q}; Q_3+1{,}5 \cdot  \Delta_{Q} \right] =\left[25;97\right]$.\\
		Số liệu $135$ không thuộc khoảng trên nên nó là số liệu bất thường trong mẫu số liệu.		
	}
\end{ex}

\Closesolutionfile{ans}


\ind{PHẦN IV.} \inden{Tự luận.}\\
\setcounter{ex}{0}
\begin{ex}%[0D6N4-2]%[Dự án đề cương 3 Khối NH24-25-Dot 2-Nguyễn Hoài Nam]
	[\textit{Trích đề thi HKI - Trường THPT Bình Giang, Hải Dương - Năm học 2024-2025}]
	Tìm tứ phân vị thứ ba của mẫu số liệu
	\begin{center}
		\begin{tabular}{|c|c|c|c|c|c|c|}
			\hline
			$5$ & $13$ & $7$ & $5$ & $2$ & $10$ & $3$ \\
			\hline
		\end{tabular}
	\end{center}
	\loigiai{
		Sắp xếp lại mẫu số liệu theo thứ tự không giảm: $2$; $3$; $5$; $5$; $7$; $10$; $13$.\\
		Tứ phân vị thứ ba của mẫu số liệu là trung vị của mẫu số liệu $7$, $10$, $13$ nên $Q_3 = 10$.
	}
\end{ex} 

\begin{ex}%[0D6N4-2]%[Dự án đề cương 3 Khối NH24-25-Dot 2-Nguyễn Hoài Nam]
	Mẫu số liệu thống kê chiều cao (đơn vị: mét) của 15 cây bạch đàn là
	$$6{,}1\quad 6{,}8 \quad7{,}5\quad 8{,}2 \quad8{,}2 \quad7{,}8 \quad7{,}9 \quad9{,}0\quad 8{,}9 \quad7{,}2 \quad7{,}5\quad 8{,}7\quad 7{,}7\quad 8{,}8\quad 7{,}6.$$
	Tính khoảng biến thiên của mẫu số liệu trên.
	\loigiai{
		Trong số liệu, số lớn nhất là $9{,}0$ và số bé nhất là $6{,}1$.\\
		Do đó khoảng biến thiên của mẫu số liệu là $R = 9{,}0 - 6{,}1 = 2{,}9$ (mét).
	}
\end{ex} 
\begin{ex}%[0D6N4-4]%[Dự án đề cương 3 Khối NH24-25-Dot 2-Nguyễn Hoài Nam]
	[\textit{Trích đề thi HKI - Trường THPT Ngô Quyền, Quảng Bình - Năm học 2023-2024}]
	Cho mẫu số liệu  $6$; $7$; $8$; $9$; $9$. Tính phương sai của mẫu số liệu đã cho.
	\loigiai{
		Giá trị trung bình của mẫu số liệu là $\overline{x}=\dfrac{6+7+8+9+9}{5}=7{,}8$.\\
		Phương sai của mẫu số liệu là
		\[s^2=\dfrac{(6-\overline{x})^2+(7-\overline{x})^2+(8-\overline{x})^2+(9-\overline{x})^2+(9-\overline{x})^2}{5} \approx  1{,}36.\]
	}
\end{ex} 

\begin{ex}%[0D6N4-4]%[Dự án đề cương 3 Khối NH24-25-Dot 2-Nguyễn Hoài Nam]
	Kết quả kiểm tra môn toán của 6 bạn trong lớp được ghi lại trong bảng dưới đây:
	\begin{center}
		\begin{tabular}{|l|c|c|c|c|c|c|}
			\hline Tên học sinh & Sơn & Nghĩa & Lan & Huệ & Hồng & Sơn \\
			\hline Điểm & $2$ & $7$ & $6$ & $8$ & $9$ & $10$ \\
			\hline
		\end{tabular}
	\end{center}
	Tìm	độ lệch chuẩn $\delta$ của bảng số liệu trên.
	\loigiai{
		Ta có $\bar{x}=\dfrac{2+7+6+8+9+10}{6}=7$.\\
		Suy ra $\delta^2=\dfrac{1}{6}\left[(2-7)^2+(7-7)^2+(6-7)^2+(8-7)^2+(9-7)^2+(10-7)^2\right]=\dfrac{20}{3}$.\\ 
		Do đó $\delta=\sqrt{\dfrac{20}{3}}$.
	}
\end{ex} 

\begin{ex}%[0D6H4-2]%[Dự án đề cương 3 Khối NH24-25-Dot 2-Nguyễn Hoài Nam]
	Nhiệt độ của thành phố Vinh ghi nhận trong $10$ ngày qua lần lượt là
	\[24\quad 21\quad 30\quad 34\quad 28\quad 35\quad 33\quad 36\quad 25\quad 27
	\]
	Tìm	khoảng tứ phân vị của mẫu số liệu trên.
	\loigiai{
		Ta sắp xếp mẫu số liệu theo thứ tự không giảm
		\[21\quad 24\quad 25\quad 27\quad 28\quad 30\quad 33\quad 34\quad 35 \quad 36\]
		Mẫu số liệu gồm $10$ giá trị nên số trung vị là $Q_2=(28+30):2=29$.\\ Nửa số liệu bên trái là $21$; $24$; $25$; $27$; $28$ gồm $5$ giá trị, số chính giữa là $25$. Khi đó $Q_1=25$.\\ Nửa số liệu bên phải là $30$; $33$; $34$; $35$; $36$ gồm $5$ giá trị, số chính giữa là $34$. Khi đó $Q_3=34$.\\
		Khoảng tứ phân vị của mẫu số liệu bằng $\Delta_Q=Q_3-Q_1=34-25=9$.
	}
\end{ex}
\begin{ex}%[0D6H4-3]%[Dự án đề cương 3 Khối NH24-25-Dot 2-Nguyễn Hoài Nam]
	Mẫu số liệu sau là chiều cao (đơn vị: cm) của các bạn trong tổ của Lan
	\begin{center}
		\begin{tabular}{cccccccccc}
			$105$&$118$&$157$&$162$&$165$&$165$&$179$&$148$&$170$&$204$
		\end{tabular}
	\end{center}
	Tìm tất cả các giá trị bất thường (hay giá trị ngoại lệ) của mẫu số liệu trên.
	\loigiai{
		Từ mẫu số liệu ta tính được $Q_{1}=148$ và $Q_{3}=170$. Do đó, khoảng tứ phân vị là
		\begin{align*}
			\Delta_{Q}=Q_3-Q_1=170-148=22  \ (\text{cm}).
		\end{align*}
		Ta có $Q_{1}-1,5\Delta_{Q}=115$ và $Q_{3}+1,5\Delta_{Q}=203$ nên trong mẫu số liệu trên có $2$ giá trị bất thường là $105$ và $208$.	
	}
\end{ex}
\begin{ex}%[0D6H4-4]%[Dự án đề cương 3 Khối NH24-25-Dot 2-Nguyễn Hoài Nam]
	Sản lượng lúa (đơn vị là tạ) của $40$ thửa ruộng thí nghiệm có cùng diện tích trình bày trong bảng số liệu sau
	\begin{center}
		\begin{tabular}{|c|c|c|c|c|c|c|}
			\hline Sản lượng & $20$ & $21$ & $22$ & $23$ & $24$ & \\
			\hline Tần số& $5$ & $8$ & $11$ & $10$ & $6$ & $N=40$ \\
			\hline
		\end{tabular}
	\end{center}
	Tìm phương sai của mẫu số liệu trên (làm tròn đến hàng phần trăm).
	\loigiai{
		Số trung bình là $\overline{x}=\dfrac{5\cdot20+8\cdot21+11\cdot22+10\cdot23+6\cdot24}{40}=22{,}1$.\\
		Phương sai là $s^2=\dfrac{5\cdot(20-22{,}1)^2+8\cdot(21-22{,}1)^2+11\cdot(22-22{,}1)^2+10\cdot(23-22{,}1)^2+6\cdot(24-22{,}1)^2}{40}=1{,}54$.
	}
\end{ex}


\begin{ex}%[0D6V4-3]%[Dự án đề cương 3 Khối NH24-25-Dot 2-Nguyễn Hoài Nam]
	[\textit{Trích đề thi HKI - Trường THPT Phan Chu Trinh, Phú Yên - Năm học 2024-2025}]
	Bạn Minh ghi lại số tin nhắn điện thoại nhận được từ ngày 1/9 đến ngày 15/9 năm 2024 ở bảng sau
	\begin{center}
		\begin{tabular}{|c|c|c|c|c|c|c|c|c|c|c|c|c|c|c|c|}
			\hline
			Minh & $2$ & $4$ & $3 $& $4$ & $6$ & $2$ & $3$ & $2$ & $4$ & $5$ & $3$ & $4$ & $6$ & $9$ & $3$ \\
			\hline
		\end{tabular}
	\end{center}
	\begin{listEX}[1]
		\item Hãy tìm khoảng biến thiên của mẫu số liệu.
		\item Sau khi bỏ đi các giá trị bất thường (nếu có), hãy tính số lượng tin nhắn trung bình bạn Minh nhận được trong một ngày (làm tròn đến hàng phần chục).
	\end{listEX}
	\loigiai{\begin{listEX}[1]
			\item Khoảng biến thiên bằng $R=9-2=7$.
			\item Ta có $Q_1=3$; $Q_3=5$. Suy ra $\Delta Q=5-3=2$.\\
			Ta có $Q_1-1{,}5\Delta Q=3-1{,}5\cdot 2=0$; $Q_3+1{,}5\Delta Q=5+1{,}5\cdot 2=8$.\\
			Vì $9>8$ nên $9$ là giá trị bất thường.\\
			Số trung bình sau khi bỏ đi giá trị bất thường là
			$$\dfrac{2+4+3+4+6+2+3+2+4+5+3+4+6+3}{14}\approx 3{,}6.$$
	\end{listEX}}
\end{ex}
\begin{ex}%[0D6V4-4]%[Dự án đề cương 3 Khối NH24-25-Dot 2-Nguyễn Hoài Nam]
	[\textit{Trích đề thi HKI - Trường THPT Thị Xã Quảng Trị - Năm học 2024-2025}]
	Kết quả dự báo nhiệt độ cao nhất trong $11$ ngày cuối tháng 12 năm 2023 ở một tỉnh miền núi phía Bắc thu được kết quả sau
	\begin{longtable}{|c|c|c|c|c|c|c|c|c|}
		\hline
		Nhiệt độ ($^\circ$C) & $14$ & $16$ & $17$ & $18$ & $19$ & $20$ & $21$ & $22$\\
		\hline
		Tần số & $1$ & $1$ & $1$ & $2$ & $1$ & $2$ & $2$ & $1$\\
		\hline
	\end{longtable}
	\begin{enumerate}
		\item Hãy tính khoảng biến thiên và độ lệch chuẩn của mẫu số liệu trên.
		\item Hãy tìm các giá trị bất thường (nếu có) của mẫu số liệu trên.
	\end{enumerate}
	\loigiai{
		\begin{enumerate}
			\item Khoảng biến thiên $R=22-14=8$.\\
			Số trung bình của mẫu số liệu
			$$\overline{x}=\dfrac{14\cdot 1+16\cdot 1+17\cdot 1+18\cdot 2+19\cdot 1+20\cdot 2+21\cdot 2+22\cdot 1}{11}=\dfrac{206}{11}.$$
			Phương sai của mẫu số liệu
			$$S^2=\dfrac{14^2\cdot 1+16^2\cdot 1+17^2\cdot 1+18^2\cdot 2+19^2\cdot 1+20^2\cdot 2+21^2\cdot 2+22^2\cdot 1}{10}-\left(\dfrac{206}{11}\right)^2=\dfrac{640}{121}.$$
			Độ lệch chuẩn của mẫu số liệu
			$$S=\sqrt{\dfrac{640}{121}}=\dfrac{8\sqrt{10}}{11}.$$
			\item Xét tứ phân vị thứ nhất
			$Q_1=x_3=17$.\\
			Tứ phân vị thứ ba $Q_3=x_9=21$.\\
			Khoảng tứ phân vị $\Delta_Q=Q_3-Q_1=4$.\\
			Xét $Q_1-1{,}5\Delta_Q=11$ và $Q_3+1{,}5\Delta_Q=27$.\\
			Do đó mẫu số liệu đã cho không có giá trị ngoại lệ.
		\end{enumerate}
	}
\end{ex}

\begin{ex}%[0D6V4-3]%[Dự án đề cương 3 Khối NH24-25-Dot 2-Nguyễn Hoài Nam]
	[\textit{Trích đề thi HKI - Trường THPT Lê Lợi, Quảng Trị - Năm học 2023-2024}]
	Điểm điều tra về chất lượng sản phẩn mới (thang điểm $100$) như sau
	\begin{center}
		\begin{tabular}{ccccccccc}
			$80$ & $65$ & $51$ &$48$ &$45$ &$61$ &$30$ &$35$ & \\
			$87$ & $83$ & $60$ & $58$ & $75$ & $72$& $68$ & $39$& \\
			$41$ & $54$ & $61$ & $72$ & $75$ & $72$ & $61$ &$50$ &$ 65$	
		\end{tabular}
	\end{center}
	Số giá trị bất thường của mẫu số liệu trên là bao nhiêu?
	\loigiai{
		Sắp xếp lại số liệu trên theo thứ tự tăng dần của điểm số.
		\begin{center}
			\begin{tabular}{|c|c|c|c|c|c|c|c|c|c|c|c|c|c|c|c|c|c|c|c|}
				Điểm & $30$&$35$&$39$&$41$&$45$&$48$&$50$&$51$&$54$& $58$& $60$ &$61$& $65$&$ 68$& $72$& $75$& $80$& $83$& $87$ \\
				\hline 
				Tần số & $1$& $1$& $1$&$ 1$&$ 1$&$ 1$& $1$& $1$& $1$&$ 1$& $1$&$ 3$&$ 2$& $1$& $3$& $2$& $1$& $1$ &$1$
			\end{tabular}
		\end{center}
		Vì $n=25$ là số lẻ nên trung vị là số đứng thứ $\dfrac{25+1}{2}=13$ do đó $M_e=Q_2=61$.\\
		Tứ phân vị dưới là $Q_1=\dfrac{50+48}{2}=49$.\\
		Tứ phân vị trên là $Q_3=72$.\\
		Khoảng tứ phân vị của mẫu số liệu trên là $\Delta Q =Q_3-Q_1=72-49=23$.\\
		Ta có $Q_3+1{,}5\Delta Q =72+1{,}5 \cdot 23 =106{,}5$ và $Q_1-1{,}5\Delta Q=49-1{,}5 \cdot 23=14{,}5$.\\
		Nhìn vào bảng ta thấy không có giá trị nào lớn hơn $106{,}5$ hay nhỏ hớn $14{,}5$ nên bảng trên không có giá trị bất thường.
	}
\end{ex}  



