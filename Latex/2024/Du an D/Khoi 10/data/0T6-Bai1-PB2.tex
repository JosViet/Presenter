\section{SỐ GẦN ĐÚNG. SAI SỐ}
\subsection{LÝ THUYẾT CẦN NHỚ}
\subsubsection{Sai số của số gần đúng}
\indam{Sai số tuyệt đối:}
\begin{boxdn}
	Nếu $a$ là số gần đúng của số đúng $\overline{a}$ thì $\Delta_a = |\overline{a} - a|$ được gọi là sai số tuyệt đối của số gần đúng $a$.
\end{boxdn}
\indam{Độ chính xác của một số gần đúng}
\begin{boxdn}
	Ta nói $a$ là số gần đúng của $\overline{a}$ với độ chính xác $d$ nếu $\Delta_a = |\overline{a} - a| \le d$ và quy ước viết gọn là $\overline{a} = a \pm d$.
\end{boxdn}
\indam{Sai số tương đối}
\begin{boxdn}
	Tỉ số $\delta_a = \dfrac{\Delta_a}{|a|}$ được gọi là sai số tương đối của số gần đúng $a$.
\end{boxdn}
\subsubsection{Số quy tròn. Quy tròn số đúng và số gần đúng}
\indam{Số quy tròn}
\begin{boxdn}
	Khi quy tròn một số nguyên hoặc một số thập phân đến một hàng nào đó thì số nhận được gọi là số quy tròn của số ban đầu.
\end{boxdn}
\indam{Quy tròn số đến một hàng cho trước}
\begin{boxdn}
	\begin{itemize}
		\item [$\bullet$] Nếu chữ số ngay sau hàng quy tròn nhỏ hơn $5$ thì ta chỉ việc thay thế chữ số đó và các chữ số bên phải nó bởi $0$.
		\item [$\bullet$] Nếu chữ số ngay sau hàng quy tròn lớn hơn hoặc bằng $5$ thì ta cũng làm như trên nhưng cộng thêm một đơn vị vào chữ số của hàng quy tròn.
	\end{itemize}
	\textbf{Nhận xét:} Ta có thể lấy độ chính xác của số quy tròn bằng nửa đơn vị của hàng quy tròn.
\end{boxdn}
\indam{Quy tròn số gần đúng căn cứ vào độ chính xác cho trước}
\begin{boxdn}
	\textit{Quy ước}: Cho $a$ là số gần đúng với độ chính xác $d$. Giả sử $a$ là số nguyên hoặc số thập phân. Khi được yêu cầu quy tròn số $a$ mà không nói rõ quy tròn đến hàng nào thì ta quy tròn số $a$ đến hàng thấp nhất mà $d$ nhỏ hơn một đơn vị của hàng đó.
\end{boxdn}

%-------------------------------------------------------------------------------------------------------------
\subsection{PHÂN LOẠI VÀ PHƯƠNG PHÁP GIẢI TOÁN}
\begin{dang}{Xác định sai số tuyệt đối, độ chính xác, sai số tương đối của số gần đúng}
	\textbf{Phương pháp giải}
	\\
	Cho số đúng $\overline{a}$ và số gần đúng $a$.
	\begin{itemize}
	    \item \textbf{Sai số tuyệt đối:} $\Delta_a = |\overline{a} - a|$.
	    \item \textbf{Độ chính xác:} Nếu $|\overline{a} - a| \le d$ thì $d$ được gọi là độ chính xác của số gần đúng. Ta viết $\overline{a} = a \pm d$.
	    \item \textbf{Sai số tương đối:} $\delta_a = \dfrac{\Delta_a}{|a|}$. Sai số tương đối càng nhỏ thì chất lượng của phép đo càng cao.
	\end{itemize}
\end{dang}
\begin{vd}%[0D6H1-1]
	Theo Quyết định số 648/QĐ-BCT ngày $20/3/2019$ của Bộ Công Thương, giá bán lẻ điện sinh hoạt từ ngày $20/3/2019$ sẽ dao động trong khoảng từ $1\,678$ đồng đến $2\,927$ đồng mỗi kWh tuỳ bậc thang. Dưới đây là bảng giá bán lẻ điện sinh hoạt (chưa bao gồm thuế VAT)
	\begin{center}
		\begin{tabular}{|l|c|}
			\hline
			\textbf{Mức sử dụng điện trong tháng (kWh)} & \textbf{Đơn giá (đồng/kWh)} \\
			\hline
			$-$ Bậc 1:  Cho kWh từ $0-50$ & $1\,678$ \\
			\hline
			$-$ Bậc 2: Cho kWh từ $51-100$ & $1\,734$ \\
			\hline
			$-$ Bậc 3: Cho kWh từ $101-200$ & $2\,014$ \\
			\hline
			$-$ Bậc 4: Cho kWh từ $201-300$ & $2\,536$ \\
			\hline
			$-$ Bậc 5: Cho kWh từ $301-400$ & $2\,834$ \\
			\hline
			$-$ Bậc 6: Cho kWh từ $401-500$ & $2\,927$ \\
			\hline
		\end{tabular}
	\end{center}
	Biết rằng, nhà bạn Hoa sử dụng điện trong tháng $3$ hết $347$ kWh.
	\begin{enumerate}
		\item Nhà bạn Hoa phải trả bao nhiêu tiền điện (bao gồm thuế VAT)?
		\item Bạn Hoa nói rằng nhà bạn phải trả số tiền điện là $759\,000$ đồng, còn em của bạn Hoa nói rằng phải trả số tiền điện là $758\,800$ đồng. Ai nói chính xác hơn?
	\end{enumerate}
	\loigiai{
		\begin{enumerate}
			\item Số tiền điện nhà bạn Hoa phải trả là\\ 
			$50 \cdot 1\,678 + 50 \cdot 1\,734 + 100 \cdot 2\,014 + 100 \cdot 2\,536 + 47 \cdot 2\,834 = 758\,798$ (đồng).
			\item Gọi $\Delta_{T_1}$, $\Delta_{T_2}$ lần lượt là sai số tuyệt đối của $759\,000$ và $758\,800$ so với số đúng $758\,798$. Ta có \\
			$\Delta_{T_1} = |758\,798 - 759\,000| = 202$, $\Delta_{T_2} = |758\,798 - 758\,800| = 2$.\\
			Vì $\Delta_{T_1} = 202 > 2 = \Delta_{T_2}$ nên em của bạn Hoa nói chính xác hơn.
		\end{enumerate}
	}
\end{vd}
\begin{vd}%[0D6H1-1]
	Một chiếc ti vi có màn hình dạng hình chữ nhật với độ dài đường chéo là $32$ in, tỉ số giữa chiều dài và chiều rộng của màn hình là $16:9$. Tìm một giá trị gần đúng (theo đơn vị inch) của chiều dài màn hình ti vi và tìm độ chính xác, sai số tương đối của số gần đúng đó.
	\loigiai{
		Gọi chiều dài của màn hình ti vi là $x$ (in) với $x > 0$.\\
		Khi đó, chiều rộng màn hình ti vi là $\dfrac{9x}{16}$ (in).\\
		Theo định lí Pythagore, ta có\\
		$x^2 + \left(\dfrac{9x}{16}\right)^2 = 32^2 \Rightarrow 337x^2 = 262\,144 \Rightarrow x = \sqrt{\dfrac{262\,144}{337}} = 27{,}89041719...$\\
		Nếu lấy giá trị gần đúng của $x$ là $27{,}9$ ta có $27{,}89 < x < 27{,}9$.\\
		Suy ra $\Delta_{27{,}9} = |x - 27{,}9| < |27{,}89 - 27{,}9| = 0{,}01$.\\
		Vậy chiều dài màn hình ti vi xấp xỉ $27{,}9$ in và độ chính xác của kết quả tìm được là $0{,}01$ in, hay $x = 27{,}9 \pm 0{,}01$ (in).\\
		Theo đó, ta ước lượng sai số tương đối của $27{,}9$ là
		$$\delta_{27{,}9} = \dfrac{\Delta_{27{,}9}}{|27{,}9|} < \dfrac{0{,}01}{27{,}9} \approx 0{,}036\%.$$
	}
\end{vd}
\begin{dang}{Xác định độ chính xác của số quy tròn và quy tròn số gần đúng căn cứ vào độ chính xác cho trước}
	\textbf{Phương pháp giải}
	\begin{itemize}
	    \item \textbf{Quy tắc làm tròn số:}
	    \begin{itemize}
	        \item Nếu chữ số đầu tiên trong phần bị bỏ đi nhỏ hơn $5$, ta giữ nguyên bộ phận còn lại.
	        \item Nếu chữ số đầu tiên trong phần bị bỏ đi lớn hơn hoặc bằng $5$, ta cộng thêm $1$ vào chữ số cuối cùng của bộ phận còn lại.
	    \end{itemize}
	    \item \textbf{Quy tròn số gần đúng với độ chính xác cho trước:}
	    \\
	    Cho số gần đúng $a$ với độ chính xác $d$. Ta quy tròn số $a$ đến hàng thấp nhất mà $d$ nhỏ hơn một đơn vị của hàng đó.
	    \\
	    \textit{Ví dụ:} Nếu $d=200$, hàng thấp nhất mà $d$ nhỏ hơn một đơn vị của nó là hàng nghìn (vì $200 < 1\,000$). Do đó ta quy tròn $a$ đến hàng nghìn.
	\end{itemize}
\end{dang}
\begin{vd}%[0D6N1-3]
	Quy tròn số $-52{,}3649$ đến hàng phần trăm. Số gần đúng nhận được có độ chính xác là bao nhiêu?
	\loigiai{
		Khi quy tròn số $-52{,}3649$ đến hàng phần trăm ta được số $-52{,}36$. Vì hàng quy tròn là hàng phần trăm nên ta có thể lấy độ chính xác của $-52{,}36$ là $0{,}005$ .}
\end{vd}
\begin{vd}%[0D6H1-3]
	Viết số quy tròn của mỗi số gần đúng sau với độ chính xác $d$:
	\begin{enumerate}
		\item $893{,}275846$ với $d=0{,}007$;
		\item $-12{,}9674507$ với $d=0{,}0005$.
	\end{enumerate}
	\loigiai{
		\begin{enumerate}
			\item Do $0{,}001<d=0{,}007<0{,}01$ nên hàng thấp nhất mà $d$ nhỏ hơn một đơn vị của hàng đó là hàng phần trăm. Vì thế, ta quy tròn số $893{,}275846$ đến hàng phần trăm. Vậy số quy tròn của $893{,}275846$ là $893{,}28$.
			\item Do $0{,}0001<d=0{,}0005<0{,}001$ nên hàng thấp nhất mà $d$ nhỏ hơn một đơn vị của hàng đó là hàng phần nghìn. Vì thế, ta quy tròn số $-12{,}9674507$ đến hàng phần nghìn. Vậy số quy tròn của $-12{,}9674507$ là $-12{,}967$.
		\end{enumerate}
	}
\end{vd}
\subsection{BÀI TẬP RÈN LUYỆN}
\ind{PHẦN I.} \inden{Câu trắc nghiệm nhiều phương án lựa chọn. Mỗi câu hỏi học sinh chỉ chọn một phương án.}\\
\setcounter{ex}{0}
\Opensolutionfile{ans}[ans/2D1-Bai1-TN]%--Đặt tên 2D1-Bai1-Dang1-TN


\begin{ex}%[0D6N1-1]
	Cho $a$ là số gần đúng của số đúng $\overline{a}$. Khi đó $\Delta_a = |\overline{a} - a|$ được gọi là 
	\choice
	{số quy tròn của $\overline{a}$}
	{sai số tương đối của số gần đúng $a$}
	{\True sai số tuyệt đối của số gần đúng $a$}
	{số quy tròn của $a$}
	\loigiai{
		$\Delta_a = |\overline{a} - a|$ được gọi là sai số tuyệt đối của số gần đúng $a$.
	}
\end{ex}

\begin{ex}%[0D6N1-3]
	Cho số $a$ là số gần đúng của số $\overline{a}$. Mệnh đề nào sau đây là mệnh đề đúng?
	\choice
	{$a>\overline{a}$}               
	{$a<\overline{a}$}
	{\True $|\overline{a}-a|>0$}
	{$-a<\overline{a}<a$}
	\loigiai{
		Biết số $a$ là số gần đúng của số $\overline{a}$. Khi đó, $|\overline{a}-a|>0$.
	}
\end{ex}

\begin{ex}%[0D6N1-2]
	Cho số $a$ là số gần đúng của $\overline{a}$ với độ chính xác $d$. Mệnh đề nào sau đây là mệnh đề đúng?
	\choice
	{$\overline{a}=a+d$}
	{$\overline{a}=a-d$}
	{$\overline{a}=a$}	
	{\True $\overline{a}=a\pm d$}
	\loigiai{
		Cho số $a$ là số gần đúng của $\overline{a}$ với độ chính xác $d$. Khi đó, $\overline{a}=a\pm d$.
	}
\end{ex}

\begin{ex}%[0D6N1-3]
	Số quy tròn của $2\,395{,}3$ đến hàng chục là
	\choice
	{$2~390$}
	{$2~395$}
	{\True $2~400$}
	{$2~300$}
	\loigiai{
		Số quy tròn của $2\,395{,}3$ đến hàng chục là $2\,400$.
	}
\end{ex}

\begin{ex}%[0D6N1-3]
	Cho số $a=23\,167$, thực hiện làm tròn số $a$ đến hàng trăm.
	\choice
	{\True $23\,200$}
	{$23\,000$}
	{$23\,170$}
	{$23\,100$}
	\loigiai{
		Làm tròn số $a$ đến hàng trăm là $23\,200$.
	}
\end{ex}

\begin{ex}%[0D6N1-3]
	Số quy tròn của $18{,}693$ đến hàng phần trăm là
	\choice
	{$18{,}6$}
	{$18{,}7$}
	{\True $18{,}69$}
	{$19$}
	\loigiai{
		Số quy tròn của $18{,}693$ đến hàng phần trăm là $18{,}69$.
	}
\end{ex}

\begin{ex}%[0D6N1-3]
	Cho số $b=18{,}062$, thực hiện làm tròn số $b$ đến hàng phần trăm.
	\choice
	{\True $18{,}06$}
	{$18$}
	{$18{,}1$}
	{$18{,}07$}
	\loigiai{
		Làm tròn số $b$ đến hàng phần trăm là $18{,}06$.
	}
\end{ex}

\begin{ex}%[0D6N1-3]
	Cho số đúng $d\in \left[5{,}5;6{,}5\right)$. Số quy tròn của $d$ đến hàng đơn vị là
	\choice
	{$5$}
	{\True $6$}
	{$5{,}5$}
	{$6{,}5$}
	\loigiai{
		Vì $5{,}5\leq d < 6{,}5$ nên số quy tròn của $d$ đến hàng đơn vị là $6$.
	}
\end{ex}

\begin{ex}%[0D6H1-1]
	Cho số đúng $d\in \left[5{,}5;6{,}5\right)$. Sai số tuyệt đối của phép làm tròn $d$ đến hàng đơn vị là
	\choice
	{$1$}
	{$2$}
	{\True $0{,}5$}
	{$1{,}5$}
	\loigiai{
		Vì $5{,}5\leq d < 6{,}5$ nên $|d-6|\leq 0{,}5$. Do đó, sai số tuyệt đối của phép làm tròn $d$ đến hàng đơn vị là $0{,}5$.
	}
\end{ex}

\begin{ex}%[0D6H1-4]
	Cho số gần đúng $a=2{,}52$ với độ chính xác $d=0{,}01$. Số đúng $\overline{a}$ thuộc đoạn
	\choice
	{$[2{,}53;2{,}54]$}
	{$[2{,}52;2{,}54]$}
	{\True $[2{,}51;2{,}53]$}
	{$[2{,}52;2{,}55]$}
	\loigiai{
		Số đúng $\overline{a}$ thuộc đoạn $[2{,}52-0{,}01;2{,}52+0{,}01]=[2{,}51;2{,}53]$.
	}
\end{ex}

\begin{ex}%[0D6H1-3]
	Cho số gần đúng $a=581\,268$ với độ chính xác $d=200$. Số quy tròn của số $a$ là
	\choice
	{$581\,260$}
	{$581\,200$}
	{\True $581\,000$}
	{$581\,270$}
	\loigiai{
		Số quy tròn của số	$a=581\,268$ với độ chính xác $d=200$ là $581\,000$.
	}
\end{ex}

\begin{ex}%[0D6H1-3]
	Số quy tròn của	$11\,251\,900 \pm 300$ là
	\choice
	{$11\,251\,000$}
	{\True $11\,252\,000$}
	{$11\,251\,500$}
	{$11\,251\,600$}
	\loigiai{
		Số quy tròn của	$11\,251\,900 \pm 300$ là $11\,252\,000$.
	}
\end{ex}

\begin{ex}%[0D6H1-3]
	Số quy tròn của	$18{,}2857 \pm 0{,}01$ là
	\choice
	{$18{,}286$}
	{$18{,}28$}
	{\True $18{,}29$}
	{$18{,}3$}
	\loigiai{
		Số quy tròn của	$18{,}2857 \pm 0{,}01$ là $18{,}29$.
	}
\end{ex}

\begin{ex}%[0D6H1-3]
	Kết quả làm tròn số $a=2\,359{,}3$ đến hàng chục là
	\choice
	{$236$}
	{\True $2\,360$}
	{$2\,359$}
	{$235\,960{,}3$}
	\loigiai{
		Kết quả làm tròn số $a=2\,359{,}3$ đến hàng chục là $2\,360$
	}
\end{ex}

\begin{ex}%[0D6H1-3]
	Cho số gần đúng $ x=6\,341\,275 $ với độ chính xác $ d=300 $. Kết quả quy tròn của $ x $ là
	\choice
	{ $6\,341\,300 $}
	{ $ 6\,341\,280 $}
	{\True $ 6\,341\,000  $}
	{ $6\,342\,000 $}
	\loigiai{Vì độ chính xác đến hàng trăm ($ d=300 $) nên ta quy tròn $ a $ đến hàng nghìn theo quy tắc làm tròn.}
\end{ex}

\begin{ex}%[0D6H1-3]
	Số quy tròn của số gần đúng $a = 374\,529$ biết $\overline{a} = 374\,529 \pm 200$ là 
	\choice{$374\,530$}
	{$374\,500$}
	{$374\,000$}
	{\True $375\,000$}
	\loigiai{Vì độ chính xác $d = 200$ nên ta quy tròn đến hàng nghìn. Vậy số quy tròn là $375\,000$. 
	}
\end{ex}

\begin{ex}%[0D6H1-4]
	Cho số gần đúng $a= 174\,325$  và $d = 17$, khi đó số đúng $\overline{a}$ viết dưới dạng  
	\choice{\True $\overline{a} = 174\,325 \pm 17$}
	{$\overline{a} = 174\,325 - 17$}
	{$\overline{a} = 174\,325 + 17$}
	{$\overline{a} = 174\,325$}
	\loigiai{Ta có $\overline{a} = a \pm d = 174\,325 \pm 17$.   
	}
\end{ex}

\begin{ex}%[0D6H1-1]
	Một vật có thể tích $V = 180{,}37 \text{\ cm}^3 \pm 0{,}05 \text{\ cm}^3$. Nếu lấy $180{,}37 \text{\ cm}^3$ làm giá trị gần đúng cho $V$ thì sai số tương đối của giá trị gần đúng đó không vượt quá 
	\choice{\True $0{,}03 \%$}
	{$0{,}01 \%$}
	{$0{,}02 \%$}
	{$0{,}001 \%$}
	\loigiai{Ta có $\delta = \dfrac{\Delta}{|V|} \leq \dfrac{d}{|V|} = \dfrac{0{,}05}{180{,}37} \approx 0{,}03 \%$. 
	}
\end{ex}

\begin{ex}%[0D6H1-4]
	Một phép đo đường kính nhân tế bào cho ta kết quả là $5\pm 0{,}3$ $\mu$m. Đường kính thực của nhân tế bào thuộc đoạn nào?
	\choice
	{$[4{,}7;5]$}
	{$[5;5{,}3]$}
	{\True $[4{,}7;5{,}3]$}
	{$[4{,}8;5{,}2]$}
	\loigiai{
		Giá trị đúng gằm trong đoạn $[5-0{,}3;5+0{,}3]$ hay $[4{,}7;5{,}3]$.
	}
\end{ex}

\begin{ex}%[0D6H1-1]
	Trong một cuộc điều tra dân số, người ta viết dân số của một tỉnh là $3~574~625$ người $\pm 50\,000$ người. Sai số tương đối của số gần đúng xấp xỉ là
	\choice
	{\True $1{,}4\%$}
	{$1{,}4$}
	{$1{,}3\%$}
	{$1{,}3$}
	\loigiai{
		Sai số tương đối là $\delta = \dfrac{50\,000}{3\,574\,625}\approx 1{,}4\%$.
		
	}
\end{ex}

\Closesolutionfile{ans}

\ind{PHẦN II.} \inden{Câu trắc nghiệm đúng sai. Trong mỗi ý a), b), c), d) ở mỗi câu, học sinh chọn đúng hoặc sai.}\\
\setcounter{ex}{0}
\Opensolutionfile{ans}[ans/2D1-Bai1-DS]%--Đặt tên 2D1-Bai1-DS

\begin{ex}%[0D6H1-1]
	Một công ty sử dụng dây chuyền $A$ để đóng gạo vào bao với khối lượng mong muốn là $m=5$ kg. Trên bao bì ghi thông tin khối lượng là  $ 5 \pm 0{,}2 \text{\ kg}$.
	\choiceTF
	{\True Số gần đúng là $5$ kg }
	{\True Độ chính xác là $0{,}2$ kg }
	{ Giá trị đúng thuộc đoạn $[4{,}7; 5{,}1]$}
	{ Sai số tuyệt đối $\delta_m  \approx 3 \%$}
	\loigiai{
		\begin{itemchoice}
			\itemch Vì $ 5 \pm 0{,}2$\,kg nên số gần đúng của  là $5$ kg
			\itemch Vì $ 5 \pm 0{,}2$\,kg nên Độ chính xác là $0{,}2$ kg
			\itemch Giá trị đúng thuộc đoạn $[5-0{,}2 ; 5+0{,}2]$ hay $[4{,}8; 5{,}2]$
			\itemch Sai số tương đối là $\delta_m \leq \dfrac{0{,}2}{502} = 4 \%$.
		\end{itemchoice}
	}
\end{ex}
\begin{ex}%[0D6H1-1]
	Cho hình chữ nhật có các cạnh là $x = 2\text{\ m} \pm 1 \text{\ cm}$ và $y = 5 \text{\ m} \pm 2 \text{\ cm}$. Gọi $S$ là diện tích hính chử nhật này, $\delta_S$ là sai số tương đối của $S$.
	\choiceTF
	{\True Số gần đúng của $x$ là $2$ m }
	{Độ chính xác của $x$ là $1$ m }
	{\True $S = 10{,}0002 \text{\ m}^2 \pm 0{,}09 \text{\ m}^2$}
	{\True $\delta_S  \approx 0{,}9 \%$}
	\loigiai{
		\begin{itemchoice}
			\itemch Vì $x = 2\text{\ m} \pm 1 \text{\ cm}$ nên số gần đúng của $x$ là $2$ m
			\itemch Vì $x = 2\text{\ m} \pm 1 \text{\ cm}$ nên Độ chính xác của $x$ là $1$ cm
			\itemch Ta có $1{,}99 \text{\ m} \leq x \leq 2{,}01 \text{\ m}$ và $4{,}98 \text{\ m} \leq y \leq 5{,}02 \text{\ m}$. \\
			Khi đó $S = x \cdot y \Rightarrow 9{,}9102\text{\ m}^2 \leq S \leq 10{,} 0902 \text{\ m}^2 \Rightarrow S = 10{,}0002 \text{\ m}^2 \pm 0{,}09 \text{\ m}^2$.
			\itemch Sai số tương đối là $\delta_S \leq \dfrac{0{,}09}{10{,}0002} \approx 0{,}9 \%$.
		\end{itemchoice}
	}
\end{ex}
\begin{ex}%[0D6H1-1]
	Cho hình chữ nhật có các cạnh là $x= 7{,}8 \text{\ m} \pm 2 \text{\ cm}$, $y = 25{,}6 \text{\ m} \pm 4 \text{\ cm}$. Gọi $C$ là chu vi hình chữ nhật này, $\delta_S$ là sai số tương đối của $C$.
	\choiceTF
	{\True Số gần đúng của $y$ là $25{,}6$ m }
	{Độ chính xác của $x$ là $4$ m }
	{\True $C  \in [66{,}68 ; 66{,}92]$}
	{\True $C=66\text{\ m} \pm 12\text{\ cm}$}
	\loigiai{
		\begin{itemchoice}
			\itemch Vì $y = 25{,}6 \text{\ m} \pm 4 \text{\ cm}$ nên số gần đúng của $y$ là $25{,}6$ m
			\itemch Vì $x= 7{,}8 \text{\ m} \pm 2 \text{\ cm}$ nên Độ chính xác của $x$ là $2$ cm
			\itemch Ta có $x= 7{,}8 \text{\ m} \pm 2 \text{\ cm} \Rightarrow 7{,}78 \text{\ m} \leq x \leq 7{,}82 \text{\ m} $. \\
			$y = 25{,}6 \text{\ m} \pm 4 \text{\ cm} \Rightarrow 25{,}56 \text{\ m} \leq y  \leq 25{,}64\text{\ m}$. \\
			Do vậy chu vi của khu vườn là $P = 2 (x +y) \in [66{,}68 \text{\ m}; 66{,}92 \text{\ m}]$.
			\itemch Vì $d = 12\text{\ cm} = 0{,}12 \text{\ m} < 0{,}5 = \dfrac{1}{2}$ nên dạng chuẩn là $66\text{\ m} \pm 12\text{\ cm}$.
		\end{itemchoice}
	}
\end{ex}
\begin{ex}%[0D6H1-1]
	Cho số gần đúng $a=9\,981$ với độ chính xác $d=100$.
	\choiceTF
	{ Quy tròn $a$ đến hàng trăm }
	{\True Số quy tròn của $a$ là $10\,000$ }
	{\True Sai số tuyệt đối của số quy tròn là $119$}
	{ Sai số tuyệt đối của số quy tròn là $ 1{,}1 \%$}
	\loigiai{
		\begin{itemchoice}
			\itemch Vì độ chính xác đến hàng trăm nên ta làm tròn $a$ đến hàng nghìn.
			\itemch Số quy tròn của $a$ đến hàng nghìn là $10\,000$
			\itemch Sai số tuyệt đối của số quy tròn là $\Delta_a= |10\,000-9\,981|=119$
			\itemch Sai số tương đối là $\delta_a \leq \dfrac{119}{9\,981} \approx 1{,}2 \%$.
		\end{itemchoice}
	}
\end{ex}
\begin{ex}%[0D6H1-1]
	Dùng thước đo có độ chia nhỏ nhất $1$ cm để đo chiều cao của một học sinh được giá trị là $163$ cm.
	\choiceTF
	{ $163$ cm là giá trị đúng }
	{\True Độ chính xác $d=0{,}5$ cm }
	{\True Sai số tương đối $\Delta_a \leq 0{,}5$ cm}
	{ Sai số tuyệt đối $\delta \approx 0{,}32\%$}
	\loigiai{
		\begin{itemchoice}
			\itemch $163$ cm là số gần đúng
			\itemch Vì độ chia nhỏ nhất của thước là $1$ cm nên độ chính xác $d=0{,}5$ cm.
			\itemch Sai số tương đối $\Delta_a \leq d= 0{,}5$ cm
			\itemch Sai số tương đối $\delta \leq \dfrac{d}{a}=\dfrac{0{,}5}{163}\approx 0{,}31\%$.
		\end{itemchoice}
	}
\end{ex}
\ind{PHẦN III.} \inden{Trả lời ngắn.}\\
\setcounter{ex}{0}
\Opensolutionfile{ans}[ans/2D1-Bai1-DS]%--Đặt tên 2D1-Bai1-DS
\begin{ex}%[0D6N1-4]
	Kết quả đo chiều dài một cây cầu có độ chính xác là $0{,}75$\,m với dụng cụ đo đảm bảo sai số tương đối không vượt quá $0{,}15\%$. Chiều dài gần đúng của cây cầu là bao nhiêu mét?
	\shortans[oly]{500}
	\loigiai{
		Ta có $\delta_a\leq\dfrac{d}{|d|} \Leftrightarrow |a|\leq \dfrac{d}{\delta_a}$.\\
		Do đó chiều dài gần đúng của cây cầu là $l \approx \dfrac{0{,}75}{0{,}15\%}=500$.
	}
\end{ex}
\begin{ex}%[0D6H1-3]
	Cho hình chữ nhật $ABCD$. Gọi $AL$ và $CI$ lần lượt là đường cao của tam giác $ABD$ và $BCD$. Biết $DL=LI=IB=1$. Diện tích hình chữ nhật là bao nhiêu (kết quả chính xác làm tròn đến hàng phần trăm)?
	\shortans[oly]{4,24}
	\loigiai{
		\immini{Ta có $AL^2=BL\cdot LD=2$, do đó $AL=\sqrt{2}$.\\
			Lại có $BD=AL+LI+IB=3$.\\
			Suy ra diện tích hình chữ nhật $ABCD$ là $3\sqrt{2} \approx 4{,}24264\ldots \approx 4{,}24$.}
		{
			\begin{tikzpicture}[scale=1,>=stealth, font=\footnotesize, line join=round, line cap=round]
			\def\d{4} %dài
			\def\r{3} %rộng
			\path 	(0:0) coordinate (B)
			++(0:\d) coordinate (C)
			++(90:\r) coordinate (D)
			++(180:\d) coordinate (A);
			\coordinate (L) at ($(B)!(A)!(D)$);
			\coordinate (I) at ($(B)!(C)!(D)$);
			\draw (A)--(B)--(C)--(D)--cycle 
			(B)--(D);
			\foreach \x/ \goc in {A/135,B/-135,C/-45,D/45,L/-90,I/90} 
			\fill (\x) circle (1pt)
			($(\x)+(\goc:3mm)$) node {$\x$};
			
			\end{tikzpicture}
		}
	}
\end{ex}
\begin{ex}%[0D6N1-1]
	Trong một cuộc điều tra dân số, người ta viết dân số của một tỉnh là: $3\ 574\ 625$ người $\pm 50\ 000$ người. Sai số tương đối của số gần đúng này không vượt quá bao nhiêu phần trăm? ( kết quả làm tròn đến hàng phần mười)
	\shortans[oly]{1,4}
	\loigiai{
		Sai số tương đối của số gần đúng là $\delta\leq\dfrac{d}{|d|}=\dfrac{50\,000}{3\,574\,625} \approx 0{,}013\,987 \approx 1{,}4\%$.
	}
\end{ex}
\begin{ex}%[0D6N1-1]
	Số $\overline{a}$ được cho bởi số gần đúng $a=5{,}7\,824$ với sai số tương đối không vượt quá $0{,}5\%$. Sai số tuyệt đối của số gần đúng $a$ không vượt quá bao nhiêu phần trăm? (kết quả làm tròn đến hàng phần mười)
	\shortans[oly]{2,9}
	\loigiai{
		Ta có $\delta_a=\dfrac{\Delta_a}{|a|}$ suy ra $\Delta_a=\delta_a\cdot|a|$.\\ Do đó $\Delta_a\leq\dfrac{0{,}5}{100}\cdot5{,}7824=0{,}028912 \approx 2{,}9\%$.
	}
\end{ex}
\begin{ex}%[0D6H1-2]
	Một sân bóng đá có dạng hình chữ nhật với chiều dài và chiều rộng của sân lần lượt là $105$\,m và $68$\,m. Khoảng cách xa nhất giữa hai vị trí trên sân đúng bằng độ dài đường chéo của sân. Quy tròn giá trị gần đúng (theo đơn vị mét) của độ dài đường chéo sân đến hàng phần mười. Độ chính xác của số gần đúng đó bằng bao nhiêu?
	\shortans[oly]{0,01}
	\loigiai{
		Gọi $x$ là độ dài đường chéo của sân bóng, Áp dụng định lí Pythagore, ta có
		$$x=\sqrt{105^2+68^2}=\sqrt{15649} \approx 125{,}0\,959\,632.$$
		Quy tròn giá trị gần  đúng của $x$ đến hàng phần mười ta được $125{,}1$.\\
		Ta có $125{,}09<x<125{,}1$. Suy ra $\left|x-125{,}1\right|<\left|125{,}09-125{,}1\right|=0{,}01$.\\
		Vậy độ dài sân bóng có thể lấy bằng $125{,}1$\,m với độ chính xác $d=0{,}01$.
	}
\end{ex}
\Closesolutionfile{ans}
\ind{PHẦN IV.} \inden{Tự luận.}\\
\setcounter{ex}{0}

\begin{ex}%[0D6H1-3]%
	[THPT Thị Xã Quảng Trị  - Quảng Trị - HK1 - NH24-25]
	Viết giá trị gần đúng của số $\sqrt{7}$ chính xác đến hàng phần trăm và viết giá trị gần đúng của số $\sqrt{11}$ chính xác đến hàng phần nghìn.
	\loigiai{
		Ta có $\sqrt{7}=2{,}645751311\ldots$\\
		Do đó giá trị gần đúng của số $\sqrt{7}$ chính xác đến hàng phần trăm là $2{,}65$.\\
		Ta có $\sqrt{11}=3{,}31662479\ldots$\\
		Do đó giá trị gần đúng của số $\sqrt{11}$ chính xác đến hàng phần nghìn là $3{,}317$.
	}
\end{ex}
\begin{ex}%[0D6H1-1]%
	[Tài liệu Thầy Phan Nhật Linh]
	Cho tam giác $ABC$ có độ dài ba cạnh đo được như sau $a=12\,\text{cm}\pm 0{,}2$\,cm; $b=10{,}2\,\text{cm}\pm 0{,}2\,\text{cm}$; $c=8\,\text{cm}\pm 0{,}1$\,cm . Tính chu vi $P$ của tam giác và đánh giá sai số tuyệt đối, sai số tương đối của số gần đúng của chu vi qua phép đo.
	\loigiai{
		Giả sử $a=12+d_1$, $b=10{,}2+d_2$, $c=8+d_3$.\\
		Ta có $P=a+b+c+d_1+d_2+d_3=30{,}2+d_1+d_2+d_3$.\\
		Theo giả thiết, ta có $-0{,}2\le{d_1}\le 0{,}2$, $-0{,}2\le{d_2}\le 0{,}2$; $-0{,}1\le{d_3}\le 0{,}1$ .\\
		Suy ra $0{,}5\le{d_1}+d_2+d_3\le 0{,}5$.\\
		Do đó $P\text=30{,}2\text{cm}\pm 0{,}5$\,cm.\\
		Sai số tuyệt đối $\Delta_P\le 0{,}5$ . 
		Sai số tương đối $\delta_P\le\dfrac{d}{P}\approx 1{,}66\%$.}
\end{ex}

\begin{ex}%[0D6H1-2]%
	[Tài liệu Thầy Phan Nhật Linh]
	Đường kính của một đồng hồ cát là $8{,}52$ m. Dùng giá trị gần đúng của $\pi=3{,}141592654$ chính xác đến hàng phần trăm để tính chu vi của đồng hồ. Kết quả chính xác đến hàng phần chục.
	\loigiai{
		Gọi $d$ là đường kính thì $d=8{,}52$\,m.\\
		Dùng giá trị gần đúng của $\pi=3{,}141592654$ chính xác đến hàng phần trăm là $\pi=3{,}14$.\\
		Chu vi đồng hồ cát: $C=\pi d=3{,}14\cdot 8{,}52=26{,}7528$\,m.\\
		Giá trị gần đúng của chu vi chính xác đến hàng phần chục là $26{,}8$\,m.}
\end{ex}

\begin{ex}%[0D6H1-2]%
	[Tài liệu Thầy Phan Nhật Linh]
	Các nhà khoa học Mỹ đang nghiên cứu liệu một máy bay có thể có tốc độ gấp bảy lần tốc độ ánh sáng. Biết vận tốc ánh sáng là $299\ 792\ 458$\,(m/s). Hỏi máy bay đó trong một ngày (một ngày có $24$ giờ) bay được bao nhiêu km nếu vận tốc ánh sáng được làm tròn đến hàng ngàn (km/s)?
	\loigiai{
		Một ngày có $24$ giờ, một giờ có $60$ phút, một phút có $60$ giây.\\
		Vậy một ngày có $24\cdot 60\cdot 60=86\,400$ giây.\\
		Vận tốc ánh sáng là $v=299\,792\,458$\,(m/s) $=299\,792{,}459$\,(km/s).\\
		Vận tốc ánh sáng làm tròn đến hàng ngàn $v=300\,000$\,(km/s).\\
		Quãng đường máy bay đó đi được trong một ngày: $s=7\cdot 300\,000\cdot 86\,400=1{,}8144\cdot 10^{11}$\,km.}
\end{ex}

\begin{ex}%[0D6N1-2]%
	[Tài liệu Thầy Phan Nhật Linh]
	Một phép đo đường kính nhân tế bào cho kết quả là $5\pm 0{,}3\mu m$ . Đường kính thực của nhân tế bào thuộc đoạn có độ dài bao nhiêu?
\loigiai{
	Đường kính thực của nhân tế bào thuộc đoạn sau: $\left[4{,}7;5{,}3\right]$ .\\
	Khi đó đoạn này có độ dài $5{,}3-4{,}7=0{,}6$.}
\end{ex}

\begin{ex}%[0D6N1-2]%
	[Tài liệu Thầy Phan Nhật Linh]
Trên bao bì của một sản phẩm có ghi "khối lượng tịnh $200\pm 2$\,g". Biết khối lượng đúng của bao bì sản phẩm đó thuộc đoạn $\left[m;n\right]$, với $m$; $n$ là các số tự nhiên. Tính $S=m+n$.
\loigiai{
Khối lượng đúng của bao sản phẩm $\overline{a}$ (tính theo gam) thuộc đoạn $[198; 202]$.\\
Khi đó $\heva{& m=198\\& n=202}\Rightarrow S=400$.}
\end{ex}

\begin{ex}%[0D6H1-4]%
	[Tài liệu Thầy Phan Nhật Linh]
Trong giờ thực hành hình học, bạn Châu đã thực hiện việc đo đạc tính diện tích của một tấm nhôm hình chữ nhật với hai cạnh đo được lần lượt là $17\pm 0{,}01$\,mm và $23\pm 0{,}01$\,mm . Giá trị đúng của diện tích thuộc đoạn có độ dài bằng bao nhiêu? Kết quả làm tròn đến hàng phần chục.
\loigiai{
Ta biểu diễn chiều rộng và chiều dài của hình chữ nhật là $17+d_1$ và $23+d_2$, trong đó $-0{,}01\leq d_1\leq 0{,}01$; $-0{,}01\leq d_2\leq 0{,}01$.\\
Khi đó, ta có $\left(17+d_1\right)\left(23+d_2\right)=391+17d_2+23d_1+d_1d_2$.\\
Vì $-0{,}01 \leq d_1$, $d_2 \leq 0{,}01$ nên $\left|17d_2+23d_1+d_1d_2\right| \leq 17\cdot 0{,}01+23\cdot 0{,}01+0{,}01\cdot 0{,}01=0{,}4001$.\\
Vậy giá trị đúng của diện tích thuộc đoạn $\left[391-0{,}4001;391+0{,}4001\right]$.\\
Khi đó độ dài của đoạn này là $391+0{,}4001-\left(391-0{,}4001\right)=0{,}8$.}
\end{ex}

\begin{ex}%[0D6H1-1]%
	[Tài liệu Thầy Phan Nhật Linh]
Bạn Ngân có một mảnh nhựa với bề mặt hình tròn bán kính $1$\,dm. Bạn ấy thực hiện đo chu vi của mép mảnh nhựa đó bằng cách sử dụng một sợi dây dài không dãn như sau: Cố định một đầu sợi dây trên mép mảnh nhựa, rồi quấn sợi dây quanh mép mảnh nhựa một vòng cho đến khi đầu dây cố định chạm vào thân sợi dây lần đầu tiên, sau đó đo độ dài phần dây chạm vào mép mảnh nhựa và được kết quả là $6$\,dm . Khi đó sai số tương đối trong phép đo không vượt quá bao nhiêu $\%$?
\loigiai{
Ta có: $\Delta_a< 0{,}3$ nên $\dfrac{\Delta_a}{|a|}<\dfrac{0{,}3}{6}=0{,}05=5\%$ .\\
Suy ra sai số tương đối trong phép đo không vượt quá $5\%$.}
\end{ex}
\begin{ex}%[0D6N1-2]%
	[Tài liệu Thầy Phan Nhật Linh]
	Các nhà thiên văn tính được thời gian để Trái đất quay một vòng xung quanh Mặt Trời là $365$ ngày $\pm \dfrac{1}{4}$ ngày, thời gian để Mặt trăng quay một vòng xung quanh Trái đất là $27{,}3$ ngày $\pm \dfrac{1}{50}$ ngày. Trong hai phép đo trên, phép đo nào chính xác hơn?
	\loigiai{
	Phép đo thứ nhất có sai số tương đối là $\delta_1 \leq \dfrac{0{,}25}{365}=\dfrac{1}{1460}=0{,}06849\%$.\\
	Phép đo thứ hai có sai số tương đối là $\delta_2 \leq \dfrac{\dfrac{1}{50}}{27{,}3}\approx 0{,}07326\%$.\\
	Từ đó ta thấy phép đo thứ hai chính xác hơn.	
}
\end{ex}
\begin{ex}%[0D6N1-2]%
	[Tài liệu Thầy Phan Nhật Linh]
	Học sinh thực hành đo chu kỳ dao động của con lắc đơn bằng đồng hồ bấm giây bằng cách đo thời gian thực hiện một dao động toàn phần. Kết quả 3 lần đo như như sau:
	\begin{center}
		\begin{tabular}{|c|c|c|c|}
			\hline
			Lần đo & 1 & 2 & 3 \\
			\hline
			Kết quả & $7{,}391 \pm 0{,}02$ & $7{,}395 \pm 0{,}05$ & $7{,}389 \pm 0{,}06$ \\
			\hline
		\end{tabular}
	\end{center}
	Tính sai số tương đối của mỗi lần đo. Lần nào có sai số tương đối nhỏ nhất.
	\loigiai{
		Sai số tương đối lần $1$: $\delta_1=\dfrac{d}{|a|}=\dfrac{0{,}02}{|7{,}4|}=\dfrac{1}{370}\approx 2{,}7027\cdot 10^{-3}$.\\
		Sai số tương đối lần $2$: $\delta_2=\dfrac{d}{|a|}=\dfrac{0{,}05}{|7{,}4|}=\dfrac{1}{148}\approx 6{,}7568\cdot 10^{-3}$.\\
		Sai số tương đối lần $3$: $\delta_3=\dfrac{d}{|a|}=\dfrac{0{,}06}{|7{,}4|}=\dfrac{3}{370}\approx 8{,}1081\cdot 10^{-3}$.\\
		Vậy sai số tương đối lần $1$ là nhỏ nhất.	
	}
\end{ex}

