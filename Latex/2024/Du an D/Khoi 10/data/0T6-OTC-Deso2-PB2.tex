\newpage
\def\thoigian{90}%--Thời gian
\de{Đề số 2}{Chương VI. Thống kê}

\begin{center}
	\textbf{PHẦN 1 - CÂU TRẮC NGHIỆM BỐN PHƯƠNG ÁN}
\end{center}
\Opensolutionfile{ans}[ans/ans-TN-ONTAPCHUONG-DE2]

\begin{ex}%[0D6N1-1]%[Dự án D - đợt 3 NH24-25- Quang Vinh NT]
	Cho $a$ là số gần đúng của số đúng $\overline{a}$. Khi đó $\Delta_a = |\overline{a} - a|$ được gọi là 
	\choice{số quy tròn của $\overline{a}$}
	{sai số tương đối của số gần đúng $a$}
	{\True sai số tuyệt đối của số gần đúng $a$}
	{số quy tròn của $a$}
	\loigiai{
		$\Delta_a = |\overline{a} - a|$ được gọi là sai số tuyệt đối của số gần đúng $a$.
	}
\end{ex}

\begin{ex}%[0D6N1-1]%[Dự án D - đợt 3 NH24-25- Quang Vinh NT]
	Biết số gần đúng $ a=7975421 $ có độ chính xác $ d=150 $. Hãy ước lượng sai số tương đối của $ a $.
	\choice
	{$ \delta_a\leq0{,}15\% $}
	{\True $ \delta_a\leq0{,}19\% $}
	{$ \delta_a\leq0{,}25\% $}
	{$ \delta_a\leq0{,}21\% $}
	\loigiai{
		Sai số tương đối của $ a $ là $ \delta_a\leq \dfrac{d}{\lvert a\rvert}= \dfrac{150}{7975421}\approx0{,}0019=0{,}19\% $.
	}
\end{ex}
\begin{ex}%[0D6H1-3]%[Dự án D - đợt 3 NH24-25- Quang Vinh NT]
	Số quy tròn của số $27{,}2471$ với độ chính xác $d = 0{,}01$ là
	\choice
	{$27{,}3$}
	{\True $27{,}2$}
	{$27{,}25$}
	{$27{,}24$}
	\loigiai{
		Vì độ chính xác $d=0{,}01$ nên ta quy tròn số $27{,}2471$ đến hàng phần mười. \\
		Suy ra số quy tròn của $27{,}2471$ là $27{,}2$.
	}
\end{ex}
\begin{ex}%[0D6H2-3]%[Dự án D - đợt 3 NH24-25- Quang Vinh NT]
	Một đội $20$ thợ thủ công được chia đều vào $5$ tổ. Trong một ngày, mỗi người thợ làm được $4$ hoặc $5$ sản phẩm. Cuối ngày, đội trưởng thống kê lại số sản phẩm mà mỗi tổ làm được ở bảng sau
	\begin{center}
		\begin{tabular}{|c|c|c|c|c|c|}
			\hline
			Tổ & $1$ & $2$ & $3$ & $4$ & $5$ \\
			\hline
			Số sản phẩm & $17$ & $19$ & $19$ & $21$ & $20$ \\
			\hline
		\end{tabular}
	\end{center}
	Biết rằng trong các tổ có một tổ bị thống kê \textbf{sai}. Hỏi tổ bị thống kê \textbf{sai} là tổ nào?
	\choice
	{$2$}
	{$3$}
	{\True $4$}
	{$5$}
	\loigiai{
		Mỗi tổ có $20:5=4$ người.\\
		Trong một ngày, mỗi người thợ làm được $4$ hoặc $5$ sản phẩm nên mỗi tổ làm được từ $4\cdot 4 = 16$ đến $4\cdot 5=20$ sản phẩm. \\
		Do đó, bảng trên ghi tổ $4$ làm được $21$ sản phẩm là không chính xác. 
	}
\end{ex}
\begin{ex}%[0D6H2-1]%[Dự án D - đợt 3 NH24-25- Quang Vinh NT]
	Một cửa hàng giầy thể thao đã thống kê cỡ giày của một số khách hàng nam vào cửa hàng trong ngày được kết quả như sau $38$; $39$; $39$; $40$; $42$; $40$; $41$; $38$; $39$; $39$; $42$; $41$; $38$; $40$; $41$. Cuối ngày, nhân viên tổng hợp lại trong bảng
	\begin{center}
		\begin{tabular}{|c*{5}{|>{\centering\arraybackslash}p{1.5cm}}|}
			\hline
			Cỡ giày & $38$ & $39$ & $40$ & $41$ & $42$\\
			\hline
			Số lượng khách hàng & $3$ & $4$ & $3$ &  & $2$\\
			\hline
		\end{tabular}
	\end{center}
	Điền số còn thiếu trong bảng trên?
	\choice
	{$5$}
	{$4$}
	{\True $3$}
	{$1$}
	\loigiai{
		Từ dãy số liệu như trên, ta thấy số khách hàng có cỡ giày $41$ là $3$ người.\\
		Vậy số còn thiếu trong bảng trên là $3$.
	}
\end{ex}
\begin{ex}%[0D6H2-2]%[Dự án D - đợt 3 NH24-25- Quang Vinh NT]
	\immini{
		Biểu đồ hình quạt (hình bên) mô tả tỉ lệ về giá trị đạt được của khoáng sản xuất khẩu nước ta. Sản lượng dầu là bao nhiêu?
		\choice
		{$40\%$}
		{$20\%$}
		{$30\%$}
		{\True $60\%$}
	}{
		\begin{tikzpicture}[scale=0.7, font=\footnotesize, line join=round, line cap=round, >=stealth]
			\def\r{3}
			\draw (0,0) -- +(0:\r) arc (0:216:\r)--(0,0) (-0.5,2.5) node {$x\%$};
			\draw[pattern=dots] (0,0) -- +(216:\r) arc (216:306:\r)--(0,0) (0,-3.5) node {$25\%$};
			\draw[pattern=vertical lines] (0,0) -- +(306:\r) arc (306:342:\r)--(0,0) (3,-2) node {$10\%$};
			\draw[pattern=horizontal lines light gray] (0,0) -- +(342:\r) arc (342:360:\r)--(0,0) (3.5,-0.5) node {$5\%$};
			\draw (4.5,1.25) rectangle (5,1.75) (6,1.5) node {Dầu};
			\draw[pattern=dots] (4.5,0.25) rectangle (5,0.75) (6,0.5) node {Than đá};
			\draw[pattern=vertical lines] (4.5,-0.75) rectangle (5,-0.25) (6,-0.5) node {Sắt};
			\draw[pattern=horizontal lines light gray] (4.5,-1.75) rectangle (5,-1.25) (6,-1.5) node {Vàng};
		\end{tikzpicture}
	}
	\loigiai{
		Sản lượng dầu là $100\%-25\%-10\%-5\%=60\%$.
	}
\end{ex}
\begin{ex}%[0D6N3-2]%[Dự án D - đợt 3 NH24-25- Quang Vinh NT]
	Bảng sau cho biết thời gian chạy cự li $100$ m của các bạn trong lớp
	\begin{center}
		\begin{tabular}{|c*{5}{|>{\centering\arraybackslash}p{1.5cm}}|}
			\hline
			Thời gian & $12$ & $13$ & $14$ & $15$ & $16$\\
			\hline
			Số bạn & $4$ & $7$ & $3$ & $18$ & $8$ \\
			\hline
		\end{tabular}
	\end{center}
	Hãy tính thời gian chạy trung bình cự li $100$ m của các bạn trong lớp.
	\choice
	{\True $14{,}475$}
	{$14{,}75$}
	{$14{,}245$}
	{$14{,}094$}
	\loigiai{
		Số bạn học sinh trong lớp là $n=4+7+3+18+8=40$.\\
		Thời gian chạy trung bình cự li $100$ m của các bạn trong lớp là
		$$\overline{x}=\dfrac{4\cdot 12+7\cdot 13+3\cdot 14+18\cdot 15+8\cdot 16}{40}=14{,}475.$$
	}
\end{ex}
\begin{ex}%[0D6N3-4]%[Dự án D - đợt 3 NH24-25- Quang Vinh NT]
	Bảng sau đây cho biết chiều cao của một nhóm học sinh
	\begin{center}
		\begin{tabular}{|c|c|c|c|c|c|c|c|c|}
			\hline
			160 & 178 & 150 & 164 & 168 & 176 & 156 & 172\\
			\hline
		\end{tabular}
	\end{center}
	Các tứ phân vị của mẫu số liệu là
	\choice
	{$Q_1=160$; $Q_2=168$; $Q_3=176$}
	{\True $Q_1=158$; $Q_2=166$; $Q_3=174$}
	{$Q_1=150$; $Q_2=164$; $Q_3=178$}
	{$Q_1=158$; $Q_2=164$; $Q_3=174$}
	\loigiai{
		Sắp xếp các giá trị này theo thứ tự không giảm
		\begin{center}
			\begin{tabular}{|c|c|c|c|c|c|c|c|c|}
				\hline
				$150$ & $156$ & $160$ & $164$ & $168$ & $172$ & $176$ & $178$\\
				\hline
			\end{tabular}
		\end{center}
		Vì $n=8$ là số chẵn nên $Q_2$ là số trung bình cộng của hai giá trị chính giữa là
		$Q_2=\dfrac{164+168}{2}=166$.\\
		Ta tìm $Q_1$ là trung vị của nửa số liệu bên trái $Q_2$
		và tìm được $Q_1=\dfrac{156+160}{2}=158$.\\
		Ta tìm $Q_3$ là trung vị của nửa số liệu bên phải $Q_2$ và tìm được $Q_3=\dfrac{172+176}{2}=174$.}
		\end{ex}
\begin{ex}%[0D6N3-5]%[Dự án D - đợt 3 NH24-25- Quang Vinh NT]
	Giá giày của $8$ vị khách mua như sau
	\begin{center}
		\begin{tabular}{*{8}{>{\centering\arraybackslash}p{0.7cm}}}
			$350$ & $300$ & $650$ & $300$ & $450$ & $600$ & $300$ & $250$
		\end{tabular}
	\end{center}
	Tìm mốt của dãy số liệu trên.
	\choice
	{$350$}
	{\True $300$}
	{$450$}
	{$250$}
	\loigiai{
		Ta có bảng tần số của mẫu số liệu trên như sau
		\begin{center}
			\begin{tabular}{|c|c|c|c|c|c|c|}
				\hline
				Giá giày & $250$ & $300$ & $350$ & $450$ & $600$ & $650$ \\
				\hline
				Tần số & $1$ & $3$ & $1$ & $1$ & $1$ & $1$ \\
				\hline
			\end{tabular}
		\end{center}
		Vậy mốt của dãy số liệu trên là $300$.
	}
\end{ex}
\begin{ex}%[0D6N4-2]%[Dự án D - đợt 3 NH24-25- Quang Vinh NT]
	Số tiền điện phải nộp (đơn vị: nghìn đồng) của một hộ gia đình trong $6$ tháng liên tiếp là $270$; $300$; $350$; $320$; $310$; $280$. Tìm khoảng biến thiên của mẫu số liệu trên.
	\choice
	{\True $80$}
	{$90$}
	{$70$}
	{$40$}
	\loigiai{
		Mẫu số liệu được sắp theo thứ tự không giảm là $270$; $280$; $300$; $310$; $320$; $350$.\\
		Khoảng biến thiên của mẫu số liệu là $350-270=80$.
	}
\end{ex}

\begin{ex}%[0D6N4-4]%[Dự án D - đợt 3 NH24-25- Quang Vinh NT]
	Sản lượng lúa (đơn vị tấn) của $40$ thửa ruộng thí nghiệm có cùng diện tích được trình bày trong bảng tần số sau đây
	\begin{center}
		\begin{tabular}{|c|c|c|c|c|c|}
			\hline Sản lượng & $20$ & $21$ & $22$ & $23$ & $24$ \\
			\hline Tần số & $5$ & $8$ & $11$ & $10$ & $6$ \\
			\hline
		\end{tabular}
	\end{center}
	Độ lệch chuẩn của mẫu số liệu này bằng
	\choice
	{$1{,}54$}
	{$1{,}25$}
	{$1{,}57$}
	{\True $1{,}24$}
	\loigiai{
		Ta có $n = 40$.\\
		Áp dụng vào bảng số liệu, ta có
		\begin{align*}
			\overline{x} &= \dfrac{20 \cdot  5 +21 \cdot  8 + 22 \cdot  11 + 3 \cdot  10 + 24 \cdot  6}{40} = \dfrac{884}{40} = 22{,}1.
		\end{align*}
		Phương sai của mẫu số liệu là
		\[S^2=\dfrac{1}{40}\left[20^2 \cdot  5 +21^2 \cdot  8 + 22^2 \cdot  11 + 23^2 \cdot  10 + 24^2 \cdot  6\right]-22{,}1^2=1{,}54.\]
		Từ đó, ta tính được độ lệch chuẩn
		\[ \sqrt{1{,}54} \approx 1{,}24 .\]
	}
\end{ex}
\begin{ex}%[0D6N4-4]%[Dự án D - đợt 3 NH24-25- Quang Vinh NT]
	Cho dãy số liệu $1$; $3$; $4$; $6$; $8$; $9$; $11$. Phương sai của dãy trên bằng
	\choice
	{$36$}
	{\True $\dfrac{76}{7}$}
	{$6$}
	{$\sqrt{\dfrac{76}{7}}$}
	\loigiai{Số trung bình
		\[\overline{x}=\dfrac{1+3+4+6+8+9+11}{7}=6.\]
		Phương sai
		\[s^2=\dfrac{(1-6)^2+(3-6)^2+(4-6)^2+(6-6)^2+(8-6)^2+(9-6)^2+(11-6)^2}{7}=\dfrac{76}{7}.\]
	}
\end{ex}
\Closesolutionfile{ans}
%\begin{center}
%	\textbf{ĐÁP ÁN}
%	\inputansbox{10}{ans/ans}	
%\end{center}

\begin{center}
	\textbf{PHẦN 2 - CÂU TRẮC NGHIỆM ĐÚNG SAI}
\end{center}
\setcounter{ex}{0}
\Opensolutionfile{ans}[ans/answer-DS-ONTAPCHUONG-DE2]

\begin{ex}%[0D6H3-2]%[Dự án D - đợt 3 NH24-25- Quang Vinh NT]
	Biểu đồ sau cho biết giá cao nhất của cổ phiếu MWG (Thế Giới Di Động) qua các năm
	\begin{center}
		\begin{tikzpicture}[xscale=1.5,yscale=.03]
			\def\a{4mm}
			\begin{scope}[gray!50]
				\foreach \j in {0,50,...,200}
				\draw (0,\j) node[left,black,scale=1]{$\j$}--(8.5,\j);
				\draw(0,0)--(0,200);
				\foreach \i in {0,1,...,9}
				\draw[xshift=2mm] (\i,0)--+(0,-10mm);
			\end{scope}
			\foreach \i/\j/\text in
			{1/89/2016,
				2/102/2017,
				3/127/2018,
				4/153/2019,
				5/105/2020,
				6/135/2021,
				7/162/2022,
				8/128/2023}{
				\fill[blue] (\i,0) rectangle +(\a,\j);
				\draw (\i,0) node[below=8mm,rotate=45,align=left]{\bf \text};
				\draw(\i+0.15,\j) node[above,blue]{$\j$};
			}
			\node[above,scale=1] at (current bounding box.north)
			{Giá ATH của cổ phiếu MWG hàng năm (2016-2023)};
			\node[below=3mm] at (current bounding box.south)
			{\it Năm};
			\draw (-0.5,50) node[above=10mm,rotate=90,align=left]{Giá ATH (ngàn VNĐ)};
		\end{tikzpicture}
	\end{center}	
	\choiceTF
	{\True Giá trị trung bình của giá cổ phiếu MWG là $125{,}125$ nghìn đồng}
	{Trung vị của giá cổ phiếu MWG là $114{,}5$ nghìn đồng}
	{Tứ phân vị thứ nhất của giá cổ phiếu MWG là $105$ nghìn đồng}
	{\True Tứ phân vị thứ ba của giá cổ phiếu MWG là $144$ nghìn đồng}
	\loigiai{
		\begin{itemchoice}
			\itemch  Giá trị trung bình của giá cổ phiếu MWG là
			$$\dfrac{89+102+127+153+105+135+162+128}{8}=125{,}125 \quad\text{(nghìn đồng)}.$$			
			\itemch Sắp xếp dãy số liệu theo thứ tự không giảm ta được
			\begin{center}
				\begin{tabular}{|c|c|c|c|c|c|c|c|}
					\hline
					89 & 102 & 105 & 127 & 128 & 135 & 153 & 162 \\
					\hline
				\end{tabular}
			\end{center}
			Do $n=8$ là số chẵn nên trung vị là
			$$M_\mathrm {e}=\dfrac{x_4+x_5}{2}=\dfrac{127+128}{2}=127{,}5 \quad\text{(nghìn đồng)}.$$
			
			\itemch 
			Ta có $Q_2=M_\mathrm {e}=127{,}5$.\\
			Xét nửa dưới gồm các số			
			\begin{center}
				\begin{tabular}{|c|c|c|c|}
					\hline
					89 & 102 & 105 & 127 \\
					\hline
				\end{tabular}
			\end{center}
			Tứ phân vị thứ nhất $Q_1$ là trung bình cộng của số thứ hai và số thứ ba:
			$$Q_1 = \dfrac{102 + 105}{2} = \dfrac{207}{2} = 103{,}5  \quad\text{(nghìn đồng)}.$$
			\itemch Xét nửa trên gồm các số
			\begin{center}
				\begin{tabular}{|c|c|c|c|}
					\hline
					128 & 135 & 153 & 162 \\
					\hline
				\end{tabular}
			\end{center}
			Tứ phân vị thứ ba $Q_3$ là trung bình cộng của số thứ hai và số thứ ba:
			$$Q_3 = \dfrac{135 + 153}{2} = \dfrac{288}{2} = 144 \quad\text{(nghìn đồng)}.$$
		\end{itemchoice}
	}
\end{ex}
\begin{ex}%[0D6H4-4]%[Dự án D - đợt 3 NH24-25- Quang Vinh NT]
	Có $100$ học sinh lớp $10$ tham dự kì thi học sinh giỏi Toán cấp cụm (thang điểm $20$). Kết quả cho trong bảng sau
	\begin{center}
		\begin{tabular}{|c|c|c|c|c|c|c|c|c|c|c|c|}
			\hline Điểm & $9$ & $10$ & $11$ & $12$ & $13$ & $14$ & $15$ & $16$ & $17$ & $18$& $19$\\
			\hline Tần số & $3$ & $4$ & $5$ & $3$ & $7$ & $15$ & $20$ & $14$ & $19$ & $6$& $4$\\
			\hline
		\end{tabular}
	\end{center}
	\choiceTF
	{\True Điểm số trung bình của của $100$ học sinh là $14{,}9$}
	{Mốt của mẫu số liệu là $20$}
	{Khoảng biến thiên của mẫu số liệu là $1$}
	{\True 	 Độ lệch chuẩn của mẫu số liệu là $2{,}38$}
	\loigiai{
		\begin{itemchoice}
			\itemch Điểm số trung bình là $$\overline{x}=\dfrac{9\cdot3+10\cdot4+11\cdot5+12\cdot3+13\cdot7+14\cdot15+15\cdot20+16\cdot14+17\cdot19+18\cdot6+19\cdot4}{100}=14{,}9.$$
			\itemch Mốt của mẫu số liệu là $15$ vì $15$ điểm có tần số lớn nhất bằng $20$.
			\itemch Khoảng biến thiên của mẫu số liệu là $19-9=10$.
			\itemch	Ta có
			\begin{align*}
				&3(9-14{,}9)^2+4(10-14{,}9)^2+5(11-14{,}9)^2+3(12-14{,}9)^2+7(13-14{,}9)^2+15(14-14{,}9)^2\\
				&\quad +20(15-14{,}9)^2+14(16-14{,}9)^2+19(17-14{,}9)^2+6(18-14{,}9)^2+4(19-14{,}9)^2\\
				&=565.
			\end{align*}
			Phương sai của mẫu số liệu là $s^2=\dfrac{565}{100}=5{,}65$.\\
			Suy ra độ lệch chuẩn của mẫu số liệu là $s=\sqrt{s^2}=\sqrt{5{,}65}\approx 2{,}38$.
		\end{itemchoice}
	}
\end{ex}
\Closesolutionfile{ans}
%\inputansbox[2]{2}{ans/answer.tex}

\begin{center}
\textbf{PHẦN 3 - CÂU TRẮC NGHIỆM TRẢ LỜI NGẮN}
\end{center}
\setcounter{ex}{0}
\Opensolutionfile{ans}[ans-KQ-ONTAPCHUONG-DE2]

\begin{ex}%[0D6H1-2]%[Dự án D - đợt 3 NH24-25- Quang Vinh NT]
	Khi tính diện tích hình tròn bán kính $R=3$ cm, nếu lấy $\pi=3{,}14$ thì độ chính xác là bao nhiêu?
	
	\shortans[oly]{$0{,}09$}
	\loigiai{
		Ta có diện tích hình tròn $S=3{,}14 \cdot 3^2=28{,}26$ cm$^2$ và $\overline{S}=\pi\cdot 3^2=9\pi$ cm$^2$.\\
		Ta có $3{,}14<\pi<3{,}15 \Rightarrow 3{,}14\cdot 3^2 <9\pi< 3{,}15\cdot 3^2 \Rightarrow 28{,}26<\overline{S} <28{,}35$.\\
		Do đó $\overline{S}-S=\overline{S}-28{,}26<28{,}35-28{,}26=0{,}09 \Rightarrow \Delta_S= \left|\overline{S}-S\right|<0{,}09$.\\
		Vậy nếu ta lấy $\pi=3{,}14$ thì diện tích hình tròn là $S = 28{,}26$ cm$^2$ với độ chính xác $d=0{,}09$.
	}
\end{ex}

\begin{ex}%[0D6H3-4]%[Dự án D - đợt 3 NH24-25- Quang Vinh NT]
	Mẫu số liệu thống kê thời gian (đơn vị: phút) đọc hết một cuốn sách của $10$ học sinh lớp $10A$ như sau
	\begin{center}
		\begin{tabular}{|c|c|c|c|c|c|c|c|c|c|}
			\hline
			$120$ & $125$ & $110$ & $101$ & $132$ & $127$ & $133$ & $112$ & $123$ & $130$ \\
			\hline
		\end{tabular}
	\end{center}
	Tìm tứ phân vị thứ ba.
	\shortans[oly]{130}
	\loigiai{
		Sắp xếp mẫu số liệu theo thứ tự không giảm ta có
		\begin{center}
			\begin{tabular}{cccccccccc}
				$101$ & $110$ & $112$ & $120$ & $123$ & $125$ & $127$ & $130$ & $132$ & $133$ \\
			\end{tabular}
		\end{center}
		Ta có cỡ mẫu $n=10$.\\
		Suy ra số trung vị mẫu $M_\mathrm{e}=\dfrac{1}{2}\left(123+125\right)=124$.\\
		Tứ phân vị thứ hai $Q_2=M_\mathrm{e}=124$.\\
		Tứ phân vị thứ nhất là trung vị của $101$; $110$; $112$; $120$; $123$. Do đó $Q_1=112$.\\
		Tứ phân vị thứ ba là trung vị của $125$; $127$; $130$; $132$; $133$. Do đó $Q_3=130$.
	}
\end{ex}
\begin{ex}%[0D6H4-4]%[Dự án D - đợt 3 NH24-25- Quang Vinh NT]
	Thống kê chỉ số IQ của một nhóm gồm $5$ học sinh như sau
	\begin{center}
		\begin{tabular}{|c|c|c|c|c|}
			\hline
			$98$ & $85$ & $86$ & $96$ & $110$ \\
			\hline
		\end{tabular}
	\end{center}
	Phương sai của mẫu số liệu trên bằng bao nhiêu?
	
	\shortans[oly]{83,2}
	\loigiai{
		Số trung bình của mẫu số liệu $\overline{x}=\dfrac{98+85+86+96+110}{5}=95$.\\
		Phương sai của mẫu số liệu
		$$s^2=\dfrac{(98-95)^2+(85-95)^2+(86-95)^2+(96-95)^2+(110-95)^2}{5}=83{,}2.$$
	}
\end{ex}
\begin{ex}%[0D6H4-4]%[Dự án D - đợt 3 NH24-25- Quang Vinh NT]
	Cho mẫu số liệu thống kê $1$; $2$; $3$; $4$; $5$; $6$; $7$; $8$; $9$. Tính độ lệch chuẩn của mẫu số liệu trên?
	
	\shortans[oly]{2,58}
	\loigiai{
		Ta có giá trị trung bình $\overline{x}=\dfrac{1+2+3+4+5+6+7+8+9}{9}=5$.\\
		Do đó độ lệch chuẩn
		\begin{align*}
			s&=\sqrt{\dfrac{\left(1-5\right)^2+\left(2-5\right)^2+\left(3-5\right)^2+\left(4-5\right)^2+\left(5-5\right)^2+\left(6-5\right)^2+\left(7-5\right)^2+\left(8-5\right)^2+\left(9-5\right)^2}{9}}\\
			&=\dfrac{2\sqrt{15}}{3}\approx 2,58.
		\end{align*}
	}
\end{ex}

\Closesolutionfile{ans}

\begin{center}
	\textbf{PHẦN 4 - TỰ LUẬN}
\end{center}
\setcounter{ex}{0}
\begin{ex}%[0D6V3-3]%[Dự án D - đợt 3 NH24-25- Quang Vinh NT]
	Một cửa hàng bán xe đạp thông kê số xe bán được hằng tháng trong năm $2021$ ở bảng sau\\
	\begin{center}
		\begin{tabular}{|c|c|c|c|c|c|c|c|c|c|c|c|c|}
			\hline
			Tháng	& $1$ & $2$ &$3$ & $4$ & $5$ &$6$ & $7$ & $8$ &$9$ & $10$ & $11$ &$12$ \\
			\hline
			Số xe	& $10$ & $8$ & $7$ & $5$ & $8$ &$22$& $28$ & $25$ &$20$& $10$ & $9$ &$7$ \\
			\hline
		\end{tabular}  
	\end{center}
	
	\begin{enumerate}
		\item Hãy tính số xe trung bình cửa hàng bán được mỗi tháng trong năm $2021$.
		\item Hãy so sánh hiệu quả kinh doanh trong quý III của cửa hàng với $6$ tháng đầu năm $2021$ bằng số trung bình.
	\end{enumerate}
	\loigiai
	{
		\begin{enumerate}
			\item Số xe trung bình cửa hàng bán được mỗi tháng trong năm $2021$ là $$\overline{x}=13{,}25.$$
			\item  Số xe trung bình cửa hàng bán được trong quý III là $$\overline{x}_1=\dfrac{28+25+20}{3}=\dfrac{73}{3}.$$
			Số xe trung bình cửa hàng bán được trong $6$ tháng đầu năm $2021$ là $$\overline{x}_2=\dfrac{10+8+7+5+8+22}{3}=10.$$
			Vậy hiệu quả kinh doanh trong quý III của cửa hàng lớn hơn $6$ tháng đầu năm $2021$.
		\end{enumerate}		
	}
\end{ex}
\begin{ex}%[0D6V4-2]%[Dự án D - đợt 3 NH24-25- Quang Vinh NT]
	Thống kê thời gian tự học ở nhà trong một ngày của $40$ học sinh ở một trường THPT được cho bởi bảng số liệu như sau
	\begin{center}
		\begin{tabular}{|c|c|c|c|c|c|c|c|c|c|}
			\hline
			Thời gian (giờ) & $1$ & $1{,}5$ & $2$ & $2{,}5$ & $3$ & $3{,}5$ & $4$ & $4{,}5$ & $5$ \\
			\hline
			Số lượng học sinh & $4$ & $6$ & $8$ & $2$ & $3$ & $2$ & $5$ & $8$ & $2$ \\
			\hline
		\end{tabular}
	\end{center}
	\begin{enumerate}
		\item Tính thời gian trung bình học sinh tự học ở nhà trong mẫu số liệu trên.
		\item Tìm các tứ phân vị $Q_1$, $Q_2$, $Q_3$.
	\end{enumerate}
	
	\loigiai{
		\begin{enumerate}
			\item Thời gian trung bình học sinh tự học ở nhà là 		
			\[\overline{x} = \dfrac{1 \cdot  4 + 1{,}5 \cdot  6 + 2 \cdot  8 + 2{,}5 \cdot  2 + 3 \cdot  3 + 3{,}5 \cdot  2 + 4 \cdot  5 + 4{,}5 \cdot  8 + 5 \cdot  2}{40} = \dfrac{116}{40} = 2{,}9\ \text{(giờ)}.\] 
			\item Sắp xếp mẫu số liệu theo thứ tự không giảm 
			\[1 \quad 1 \quad 1 \quad 1 \quad 1{,}5 \quad 1{,}5 \quad 1{,}5 \quad 1{,}5 \quad 1{,}5 \quad 1{,}5 \quad 2 \quad 2 \quad 2\]
			\[2\quad 2\quad 2\quad 2\quad 2\quad 2{,}5\quad 2{,}5\quad 3\quad3\quad3\quad3{,}5 \quad 3{,}5\quad 4\]
			\[4\quad 4\quad4\quad 4\quad 4{,}5\quad 4{,}5\quad 4{,}5\quad 4{,}5\quad 4{,}5\quad 4{,}5\quad 4{,}5\quad 4{,}5\quad 5\quad 5.\] 
			\begin{itemize}
				\item Tứ phân vị thứ hai $Q_2$ là trung vị của mẫu số liệu.\\ Vì $n=40$ là số chẵn nên $Q_2 = \dfrac{x_{20}+x_{21}}{2} = \dfrac{2{,}5+3}{2} = 2{,}75$. \\
				\item Tứ phân vị thứ nhất $Q_1$ là trung vị của nửa số liệu bên trái $Q_2$. \\
				Dãy này có $20$ số liệu nên $Q_1 = \dfrac{x_{10}+x_{11}}{2} = \dfrac{1{,}5+2}{2} = 1{,}75$. \\
				\item Tứ phân vị thứ ba $Q_3$ là trung vị của nửa số liệu bên phải $Q_2$. \\
				Dãy này có $20$ số liệu nên $Q_3 = \dfrac{x_{30}+x_{31}}{2} = \dfrac{4+4{,}5}{2} = 4{,}25$. 
			\end{itemize}
		\end{enumerate} 
	}
\end{ex}

\begin{ex}%[0D6H4-4]%[Dự án D - đợt 3 NH24-25- Quang Vinh NT]
	Theo Niên giám thống kê tỉnh Ninh Thuận, sản lượng nho (đơn vị: nghìn tấn) của các địa phương thuộc tỉnh Ninh Thuận năm 2022 được cho bởi bảng sau đây
	\begin{center}
		\begin{tabular}{|c|c|c|c|c|c|c|}
			\hline Huyện, thành phố & Phan Rang & Ninh Phước & Ninh Sơn & Ninh Hải & Thuân Bắc & Thuân Nam \\
			\hline Sản lượng & $3{,}4$ & $9{,}8$ & $1{,}8$ & $10$ & $0{,}04$ & $0{,}7$ \\
			\hline
		\end{tabular}
	\end{center}
	\begin{enumerate}
		\item Tính sản lượng nho trung bình của các huyện thành phố trên địa bàn tỉnh Ninh Thuận.
		\item Tính phương sai của mẫu số liệu trên (kết quả làm tròn đến hàng phần trăm).
	\end{enumerate}
	\loigiai{
		\begin{enumerate}
			\item Tính sản lượng nho trung bình của các huyện thành phố trên địa bàn tỉnh Ninh Thuận.\\
			Các số liệu về sản lượng nho (đơn vị: nghìn tấn) là $3{,}4$; $9{,}8$; $1{,}8$; $10$; $0{,}04$; $0{,}7$. \\
			Số lượng các địa phương là $n = 6$. \\
			Sản lượng nho trung bình của các huyện, thành phố là 
			$$\overline{x} = \dfrac{3{,}4 + 9{,}8 + 1{,}8 + 10 + 0{,}04 + 0{,}7}{6}= 4{,}29\,~\text{(nghìn tấn)}.$$
			Vậy sản lượng nho trung bình là $4{,}29$ nghìn tấn.
			\item Tính phương sai của mẫu số liệu trên (kết quả làm tròn đến hàng phần trăm).\\
			\item Phương sai của mẫu số liệu được tính theo công thức $s^2 = \dfrac{1}{n}\left[ (x_1 - \overline{x})^2+ (x_2 - \overline{x})^2+\cdots+ (x_k - \overline{x})^2\right]$. \\
			Ta có các giá trị $(x_i - \overline{x})^2$
			\begin{itemize}
				\item $(3{,}4 - 4{,}29)^2 = (-0{,}89)^2 = 0{,}7921$.
				\item $(9{,}8 - 4{,}29)^2 = (5{,}51)^2 = 30{,}3601$.
				\item $(1{,}8 - 4{,}29)^2 = (-2{,}49)^2 = 6{,}2001$.
				\item $(10 - 4{,}29)^2 = (5{,}71)^2 = 32{,}6041$.
				\item $(0{,}04 - 4{,}29)^2 = (-4{,}25)^2 = 18{,}0625$.
				\item $(0{,}7 - 4{,}29)^2 = (-3{,}59)^2 = 12{,}8881$.
			\end{itemize}
			Tổng các bình phương độ lệch là
			$$0{,}7921 + 30{,}3601 + 6{,}2001 + 32{,}6041 + 18{,}0625 + 12,8881 = 100,907.$$
			Phương sai của mẫu số liệu là
			$$s^2 = \dfrac{100{,}907}{6} = 16{,}817\approx 16{,}82.$$
		\end{enumerate}
	}
\end{ex}

