\newpage
\section{Ôn tập chương 4}
\def\thoigian{90}%--Thời gian
\de{Đề số 1}{Chương IV. Hệ thức lượng trong tam giác}
%[ID]%[Dự án D - Đợt 3 NH24-25 - Lê Hữu Kiệt]
\begin{center}
	\textbf{PHẦN 1 - CÂU TRẮC NGHIỆM BỐN PHƯƠNG ÁN}% 12 câu
\end{center}
\Opensolutionfile{ans}[ans/ans-TN-10-ONTAPCHUONG-IV-DE1]
\begin{ex}%[0H4N1-2]%[Dự án D - Đợt 3 NH24-25 - Lê Hữu Kiệt]
	Giá trị biểu thức $A=\sin 150^\circ+\cos 120^\circ+\tan 30^\circ$ bằng
	\choice
	{$\dfrac{-\sqrt{3}}{3}$}
	{$\dfrac{\sqrt{6}}{2}$}
	{$\sqrt{3}$}
	{\True $\dfrac{\sqrt{3}}{3}$}
	\loigiai{
		$A=\sin 150^\circ+\cos 120^\circ+\tan 30^\circ=\dfrac{1}{2}-\dfrac{1}{2}+\dfrac{\sqrt{3}}{3}=\dfrac{\sqrt{3}}{3}$. 
	}
\end{ex}

\begin{ex}%[0H4N1-1]%[Dự án D - Đợt 3 NH24-25 - Lê Hữu Kiệt]
Cho $\alpha$ là góc tù. Mệnh đề nào đúng trong các mệnh đề sau?
\choice
{$\cos\alpha >0$}
{$\sin\alpha <0$}
{$\cot\alpha >0$}
{\True $\tan\alpha <0$}
\loigiai{
Với $\alpha$ là góc tù thì chỉ $\sin\alpha>0$, còn $\cos\alpha<0$, $\tan\alpha<0$, $\cot\alpha<0$.
}
\end{ex}

\begin{ex}%[0H4N1-2]%[Dự án D - Đợt 3 NH24-25 - Lê Hữu Kiệt]
	Cho góc $\alpha$ thỏa mãn $0^\circ < \alpha < 90^\circ$ với $\cos \alpha=\dfrac{1}{3}$. Giá trị của $\cos(180^\circ-\alpha)$ bằng
	\choice
	{$\dfrac{2 \sqrt{2}}{3}$}
	{\True $-\dfrac{1}{3}$}
	{$-\dfrac{2 \sqrt{2}}{3}$}
	{$\dfrac{1}{3}$}
	\loigiai 
	{
	Ta có $\cos \left(180^\circ-\alpha\right)=-\cos \alpha=-\dfrac{1}{3}$.
	}
\end{ex}

\begin{ex}%[0H4H1-3]%[Dự án D - Đợt 3 NH24-25 - Lê Hữu Kiệt]
	Kết quả thu gọn của biểu thức $A=-\sin(180^\circ-\alpha)+\cos(180^\circ-\alpha)+\cos \alpha$ là
	\choice
	{$2\sin \alpha$}
	{$\cos \alpha$}
	{\True $-\sin \alpha$}
	{$\sin \alpha$}
	\loigiai{
		Ta có $\sin (180^\circ-\alpha)=\sin \alpha$ và $\cos (180^\circ-\alpha)=-\cos \alpha$.\\
		Do đó $A=-\sin \alpha-\cos \alpha+\cos \alpha=-\sin \alpha$.
	}
\end{ex}

\begin{ex}%[0H4H2-1]%[Dự án D - Đợt 3 NH24-25 - Lê Hữu Kiệt]
Cho tam giác $ABC$ có $BC=3$, $AC=4$, $\widehat{C}=60^\circ$. Độ dài cạnh $AB$ bằng
\choice
{$AB=25$ }
{\True $AB=\sqrt{13}$}
{$AB=13$ }
{$AB=5$}
\loigiai{
	Áp dụng định lí côsin trong tam giác ta có
	$$AB^2 = CA^2 + CB^2 - 2CA \cdot CB \cdot \cos C = 3^2 + 4^2 -2\cdot 3\cdot 4\cdot \cos 60^\circ = 13.$$
	Vậy $AB = \sqrt{13}$.
}
\end{ex}

\begin{ex}%[0H4N2-1]%[Dự án D - Đợt 3 NH24-25 - Lê Hữu Kiệt]
Cho tam giác $ABC$ có $AB=4$, $BC=7$, $AC=9$. Giá trị của $\cos A$ bằng
	\choice
	{\True $-\dfrac{2}{3}$}
	{$\dfrac{1}{2}$}
	{$\dfrac{2}{3}$}
	{$\dfrac{1}{3}$}
	\loigiai{Áp dụng hệ quả của định lý côsin, ta được
	$$\cos A =\dfrac{AB^2+AC^2-BC^2}{2AB \cdot AC}=\dfrac{4^2+9^2-7^2}{2 \cdot 4 \cdot 9}=\dfrac{2}{3}.$$}
\end{ex}

\begin{ex}%[0H4H2-1]%[Dự án D - Đợt 3 NH24-25 - Lê Hữu Kiệt]
	Cho tam giác $ABC$ có $BC=a$, $AC=b$, $AB=c$ thoả mãn $b^2+c^2-a^2=-\sqrt{3} bc$. Góc $A$ có số đo bằng
	\choice
	{$120^\circ$}
	{$45^\circ$}
	{\True $150^\circ$}
	{$60^\circ$}
	\loigiai
	{Ta có $\cos A=\dfrac{b^2+c^2-a^2}{2bc}=\dfrac{-\sqrt{3} bc}{2bc}=-\dfrac{\sqrt{3}}{2}$.\\
	Suy ra góc $\widehat{A}=150^\circ$.
	}
\end{ex}

\begin{ex}%[0H4H2-1]%[Dự án D - Đợt 3 NH24-25 - Lê Hữu Kiệt]
\immini[thm]
{Cho tam giác $ABC$ có kích thước và số đo các góc như hình bên. Giá trị của $x$ (làm tròn kết quả đến hàng phần trăm) bằng
\choice[2]
	{$10{,}25$}
	{$7{,}54$}
	{\True $7{,}55$}
	{$10{,}24$}
}
{\begin{tikzpicture}[font=\footnotesize, line join=round, line cap=butt, >=stealth, scale=0.6]
\path (0,0) coordinate (A)+(0:0.1) coordinate (a) 
(120:5) coordinate (B)+(-35:0.1) coordinate (b)
(intersection of A--a and B--b) coordinate (C)
pic[draw, angle radius=3mm, angle eccentricity=2, "$120^\circ$"]{angle=C--A--B}
pic[draw, angle radius=7mm, angle eccentricity=1.8, double, "$25^\circ$"]{angle=A--B--C};
\draw (A)--(B)--(C)--cycle;
\node[left] at ($(A)!1/2!(B)$) {$5$};
\node[right] at ($(B)!1/2!(C)$) {$x$};
\foreach \x/\g in {A/-135, B/90, C/-45}{\fill (\x) circle (1pt)+(\g:0.3)node{$\x$};}
\end{tikzpicture}}
	\loigiai{Ta có $\widehat{A}+\widehat{B}+\widehat{C}=180^\circ$ do đó $\widehat{C}=180^\circ-\widehat{A}-\widehat{B}=180^\circ - 120^\circ -25^\circ=35^\circ$.\\
	Theo định lý sin, ta có $\dfrac{BC}{\sin A}=\dfrac{AB}{\sin C}$.\\
	Suy ra $x=BC=\dfrac{AB \sin A}{\sin C}=\dfrac{5 \sin 120^\circ}{\sin 35^\circ} \approx 7,55$.}
\end{ex}

\begin{ex}%[0H4H2-1]%[Dự án D - Đợt 3 NH24-25 - Lê Hữu Kiệt]
	Cho tam giác $ABC$ có $BC = a$, $\widehat{BAC} = 120^\circ$. Bán kính đường tròn ngoại tiếp tam giác $ABC$ là
	\choice
	{$R = \dfrac{a\sqrt{3}}{2}$}
	{$R = a$}
	{\True $R = \dfrac{a\sqrt{3}}{3}$}
	{$R = \dfrac{a}{2}$}
	\loigiai{
		Xét $\triangle ABC$, áp dụng định lý sin, ta có
		$$ \dfrac{BC}{\sin\widehat{BAC}} = 2R \Rightarrow R = \dfrac{a}{2\cdot \sin 120^\circ} \Rightarrow \dfrac{a}{2\cdot \dfrac{\sqrt{3}}{2}} = \dfrac{a\sqrt{3}}{3}. $$
	}
\end{ex}

\begin{ex}%[0H4H2-1]%[Dự án D - Đợt 3 NH24-25 - Lê Hữu Kiệt]
	Cho tam giác $ABC$ có $BC=2$, $AB=5$, $\widehat{C}=60^\circ$. Số đo của góc $\widehat{A}$ là
	\choice
	{$\widehat{A}\approx 30^\circ$}
	{$\widehat{A}\approx 19^\circ$}
	{\True $\widehat{A}\approx 20^\circ$}
	{$\widehat{A}\approx 21^\circ$}
	\loigiai 
	{
	Áp dụng định lý sin ta có $\dfrac{BC}{\sin A}=\dfrac{AB}{\sin C}$.\\
	Suy ra $\sin A=\dfrac{BC\sin C}{AB}=\dfrac{2\sin60^\circ}{5}=\dfrac{\sqrt3}{5}$.\\
	Do $\sin(180^\circ-A)=\sin\widehat{A}$ nên ta có hai giá trị của góc $A$ sao cho $\sin A=\dfrac{\sqrt3}{5}$ là $\widehat{A}\approx 20^\circ$ và \break$\widehat{A}\approx180^\circ-20^\circ=160^\circ$.\\
	Nếu $\widehat{A}=20^\circ$ thì $\widehat{B}=180^\circ-\widehat{A}-\widehat{C}\approx180^\circ-20^\circ-60^\circ=100^\circ$ (nhận).\\
	Nếu $\widehat{A}=160^\circ$ thì $\widehat{B}=180^\circ-\widehat{A}-\widehat{C}\approx180^\circ-160^\circ-60^\circ=-40^\circ$ (loại).\\
	Vậy số đo của góc $\widehat{A}$ khoảng $20^\circ$.
}
\end{ex}

\begin{ex}%[0H4H2-2]%[Dự án D - Đợt 3 NH24-25 - Lê Hữu Kiệt]
	Tam giác $ABC$ có $AC=4$, $\widehat{A}=30^\circ$, $\widehat{C}=75^\circ$. Diện tích của tam giác $ABC$ bằng
	\choice
	{$8\sqrt{3}$}
	{$4\sqrt{3}$}
	{\True $4$}
	{$8$}
	\loigiai{Ta có $\widehat{B}=180^\circ-\widehat{A}-\widehat{C}=180^\circ-30^\circ-75^\circ=75^\circ$.\\
	Theo định lí sin ta có $\dfrac{BC}{\sin A}=\dfrac{AC}{\sin B}$.\\
	Suy ra $BC=\dfrac{AC\sin A}{\sin B}=\dfrac{4\cdot \sin 30^\circ }{\sin 75^\circ}=2\sqrt{6}-2\sqrt{2}$.\\
	Diện tích tam giác $ABC$ là
	$$S=\dfrac{1}{2}\cdot AB \cdot BC \cdot \sin B = \dfrac{1}{2}\cdot4\cdot\left(2\sqrt6-2\sqrt2\right)\cdot\sin 75^\circ=4.$$
	}
\end{ex}

\begin{ex}%%[0H4H3-1]%[Dự án D - Đợt 3 NH24-25 - Lê Hữu Kiệt]
	Cho tam giác $ABC$ có $AB=8$, $AC=15$, $BC=17$. Diện tích tam giác $ABC$ bằng
	\choice
	{$S_{ABC}=40$}
	{$S_{ABC}=50$}
	{$S_{ABC}=45$}
	{\True $S_{ABC}=60$}
	\loigiai{
	Nửa chi vi của tam giác $ABC$ là $p=\dfrac{AB + AC + BC}{2}=\dfrac{8 + 15 + 17}{2} = 20$. \\
	ÁP dụng công thức Heron ta có
	$$S_{ABC} = \sqrt{p(p-AB)(p-AC)(p-BC)}=\sqrt{20(20-8)(20-15)(20-17)}=60.$$}
\end{ex}

\Closesolutionfile{ans}
\begin{center}
	\textbf{PHẦN 2 - CÂU TRẮC NGHIỆM ĐÚNG SAI}%2 câu
\end{center}
\setcounter{ex}{0}
\Opensolutionfile{ans}[ans/answer-DS-10-ONTAPCHUONG-IV-DE1]
\begin{ex}%[0H4H1-3]%[Dự án D - Đợt 3 NH24-25 - Lê Hữu Kiệt]
	Cho biểu thức $A=\dfrac{\cos(90^\circ-a)+2\sin(180^\circ-a)}{\sin(90^\circ-a)}$ với $0^\circ<a<90^\circ$.
	\choiceTF
	{$\cos(90^\circ-a)=\cos a$}
	{\True $\sin(180^\circ-a)=\sin a$}
	{$\sin(90^\circ-a)=-\sin a$}
	{\True $A=3\tan a$}
	\loigiai{
	\begin{itemchoice}
		\itemch Ta có $\cos(90^\circ-a)=\sin a$.
		\itemch Ta có $\sin(180^\circ-a)=\sin a$.
		\itemch Ta có $\sin(90^\circ-a)=\cos a$.
		\itemch Ta có
		$$\begin{aligned}[t]A&=\dfrac{\cos(90^\circ-a)+2\sin(180^\circ-a)}{\sin(90^\circ-a)}\\
			&=\dfrac{\sin a+2 \sin a}{\cos a}\\
			&=\dfrac{3\sin a}{\cos a}\\
			&=3\tan a.
		\end{aligned}$$
	\end{itemchoice}
	}
\end{ex}

\begin{ex}%[0H4V3-2]%[Dự án D - Đợt 3 NH24-25 - Lê Hữu Kiệt]
	Trong một lần đến tham quan tượng Nữ thần tự do (Ở Newyork, Mỹ), bạn Hiếu muốn ước tính độ cao của tượng. Sau khi quan sát, bạn Hiếu đã minh họa lại kết quả đo đạc như hình bên dưới.
	\begin{center}	
	\begin{tikzpicture}[scale=0.6, font=\footnotesize, line join=round, line cap=butt, >=stealth]
	\clip (-9.5,-5) rectangle (2,2.7);
	\path 
	(-1,2.2) coordinate (C)
	(-9,-4.17) coordinate (A)
	(-6,-4.17) coordinate (B)
	(-1,-4.17) coordinate (H)
	(2,-4.17) coordinate (X)
	pic[draw,angle radius=5mm, angle eccentricity=1.7, "$48^\circ$"]{angle = B--A--C}
	pic[draw, angle radius=5mm, angle eccentricity=1.7, double, "$62^\circ$"]{angle=H--B--C}
	pic[draw,angle radius=2mm]{right angle = C--H--B}
	;
	\draw (0.1,-2.95) node {\includegraphics[scale=0.06]{Images/0H4-OTC-Deso1-Cau2DS}};
	\draw (C)--(H)--(A)--(C)--(B);
	\foreach \x/\g in {A/-90, B/-90, C/150, H/-135}{\fill (\x) circle (1pt)+(\g:0.4)node{$\x$};}
	\node[below] at ($(A)!1/2!(B)$) {$16{,}7$ m};
	\node[left] at ($(C)!1/2!(H)$) {$h$ (m)};
	\draw[dashed] (H)--(X);
	\end{tikzpicture}
	\end{center}
	\choiceTF
	{\True $\widehat{ABC}=118^\circ$}
	{$\widehat{ACB}=70^\circ$}
	{\True $BC=51{,}3$ mét (kết quả được làm tròn đến hàng phần mười)}
	{\True Chiều cao $h$ của tượng Nữ thần tự do là $45{,}3$ mét (kết quả được làm tròn đến hàng phần mười)}
	\loigiai{
	\begin{itemchoice}
		\itemch Ta có $\widehat{ABC}+\widehat{CBH}=180^\circ$ (kề bù) nên $\widehat{ABC}=180^\circ-\widehat{CBH}=180^\circ-62^\circ=118^\circ$.
		\itemch Xét $\triangle ABC$ ta có $\widehat{ABC}+\widehat{BAC}+\widehat{ACB}=180^\circ$.\\
		Suy ra $\widehat{ACB}=180^\circ-\widehat{BAC}-\widehat{ABC}=180^\circ-48^\circ-118^\circ=14^\circ$.
		\itemch Áp dụng định lí sin cho $\triangle ABC$ ta có $\dfrac{BC}{\sin \widehat{CAB}}=\dfrac{AB}{\sin \widehat{ACB}}$.\\
		Suy ra $BC=\dfrac{16{,}7 \sin 48^\circ}{\sin 14^\circ}=51{,}3$ (m).
		\itemch Ta có $\triangle CBH$ vuông tại $H$ nên $\sin \widehat{CBH}=\dfrac{CH}{BC}$.\\
		Suy ra $CH=BC \cdot \sin 62^\circ=\dfrac{16{,}7 \sin 48^\circ}{\sin 14^\circ} \cdot \sin 62^\circ \approx 45{,}3$ (m).\\
		Vậy chiều cao $h$ của tượng Nữ thần tự do là $45{,}3$ mét.
	\end{itemchoice}
	}
\end{ex}
\Closesolutionfile{ans}

\begin{center}
\textbf{PHẦN 3 - CÂU TRẮC NGHIỆM TRẢ LỜI NGẮN}% 4 câu 
\end{center}
\setcounter{ex}{0}
\Opensolutionfile{ans}[ans/ans-KQ-10-ONTAPCHUONG-IV-DE1]
\begin{ex}%[0H4H1-3]%[Dự án D - Đợt 3 NH24-25 - Lê Hữu Kiệt]
Cho $\cot\alpha=\dfrac{1}{3}$. Giá trị của biểu thức $A=\dfrac{3\sin\alpha+4\cos\alpha}{2\sin\alpha- 5\cos\alpha}$ bằng bao nhiêu?
\par\shortans{$13$}
\loigiai{
Từ $\cot\alpha=\dfrac{1}{3}$ suy ra $\sin\alpha\ne 0$.\\
Chia tử và mẫu của biểu thức $A$ cho $\sin\alpha$ ta được
$$A=\dfrac{3+4\cot\alpha}{2-5\cot\alpha}=\dfrac{3+4\cdot\dfrac{1}{3}}{2-5\cdot\dfrac{1}{3}}=13.$$
Vậy $A=13$.
}
\end{ex}

\begin{ex}%[0H4H2-2]%[Dự án D - Đợt 3 NH24-25 - Lê Hữu Kiệt]
Cho hình thoi $ABCD$ có cạnh bằng $10$ m. Góc $\widehat{BAD}=30^\circ$. Diện tích hình thoi $ABCD$ bằng bao nhiêu mét vuông?
\par\shortans{$50$}
\loigiai{
\begin{center}
\begin{tikzpicture}[font=\footnotesize, line join=round, line cap=round, >=stealth, scale=1]
\path (0,0) coordinate (A) (15:3) coordinate (B) (-15:3) coordinate (D) ($(D)-(A)+(B)$) coordinate (C)
pic["$30^\circ$", angle eccentricity=1.3]{angle=D--A--B};
\draw (A)--(B)--(C)--(D)--cycle (B)--(D);
\node[above] at ($(A)!1/2!(B)$) {$10$ m};
\foreach \x/\g in {A/180, B/90, C/0, D/-90}{\fill (\x) circle (1pt)+(\g:0.4)node{$\x$};}
\end{tikzpicture}
\end{center}
Diện tích $\triangle ABD$ là $S_{\triangle ABD}=\dfrac{1}{2}\cdot AB\cdot AD\cdot\sin\widehat{BAD}=\dfrac{1}{2}\cdot10\cdot10\cdot\sin30^\circ=25$ (m$^2$).\\
Do $ABCD$ là hình thoi nên $\triangle ABD=\triangle CBD$.\\
Vậy $S_{ABCD}=2S_{\triangle ABD}=2\cdot25=50$ (m$^2$).
}
\end{ex}

\begin{ex}%[0H4H2-1]%[Dự án D - Đợt 3 NH24-25 - Lê Hữu Kiệt]
Một ô tô muốn đi từ địa điểm $A$ đến địa điểm $B$, nhưng giữa $A$ và $B$ là ngọn núi nên ô tô phải đi thành 2 đoạn từ $A$ đến $C$ và từ $C$ đến $B$. Tam giác $ABC$ (tham khảo hình vẽ) có $AC=7$ km, $BC=8$ km, $\widehat{ACB}=120^\circ$. Nếu người ta đào một đường xuyên núi chạy thẳng từ $A$ đến $B$ thì ô tô chạy trên con đường mới này rút ngắn được bao nhiêu ki-lô-mét?
\begin{center}
		\begin{tikzpicture}[scale=0.7, font=\footnotesize, line join=round, line cap=round]
		\path (0,0) coordinate (C) (-30:8) coordinate (B) (-150:7) coordinate (A)
		($(A)!2/7!(B)$) coordinate (a) ($(A)!5/7!(B)$) coordinate (b)
		pic[draw, angle radius=2mm, angle eccentricity=1.8, "$120^\circ$"]{angle=A--C--B};
		\draw(C)--(A) node[sloped,pos=0.5,above]{$7$ km} (B)--(C)node[sloped,pos=0.5,above]{$8$ km} (B)--(A);
		\draw[pattern=north west lines] (a) .. controls +(70:5) and +(160:3) .. (b);
		\foreach \x/\g in {C/90, A/-150, B/-30}{\fill (\x) circle (1pt)+(\g:0.25)node{$\x$};}
	\end{tikzpicture}
\end{center}
\par\shortans{$2$}
\loigiai{Áp dụng định lý côsin cho tam giác $ABC$, ta được
$$AB^2=CA^2+CB^2-2 \cdot CA \cdot CB \cdot \cos C = 7^2+8^2-2 \cdot 7 \cdot 8 \cdot \cos 120^\circ=169.$$
Suy ra $AB=\sqrt{169}=13$ (km).\\
Vậy độ dài đường xuyên núi chạy thẳng từ $A$ đến $B$ là $13$ km nên ô tô chạy trên con đường mới này rút ngắn được $7+8-13=2$ (km).
}
\end{ex}

\begin{ex}%[0H4V3-2]%[Dự án D - Đợt 3 NH24-25 - Lê Hữu Kiệt]
Hai tàu du lịch xuất phát từ hai thành phố cảng $A$ và $B$ cách nhau $200$ (km) đến đảo $C$ như hình minh họa bên. Biết $\widehat{CAB}=30^\circ$, $\widehat{CBA}=45^\circ$. Hai tàu chuyển động đều, cùng vận tốc $80$ km/h và xuất phát cùng lúc. Biết tàu $2$ đến sớm hơn tàu $1$ là $t$ giờ. Hỏi giá trị của $t$ bằng bao nhiêu \textit{(không làm tròn kết quả các phép tính trung gian, chỉ làm tròn kết quả cuối cùng đến hàng phần trăm)}?
\begin{center}
\begin{tikzpicture}[font=\footnotesize, line join=round, line cap=butt, >=stealth, scale=1]
\path (0,0) coordinate (A)+(30:0.1) coordinate (a)
(5,0) coordinate (B)+(135:0.1) coordinate (b)
(intersection of A--a and B--b) coordinate (C)
pic[draw, angle radius=5mm, angle eccentricity=2, "$30^\circ$"]{angle=B--A--C}
pic[draw, angle radius=5mm, angle eccentricity=1.5, double, "$45^\circ$"]{angle=C--B--A};
\draw (A)--(B)--(C)--cycle;
\node[above] at ($(A)!1/2!(B)$) {$200$ km};
\foreach \x/\g in {A/-135, B/-45, C/90}{\fill (\x) circle (1pt)+(\g:0.25)node{$\x$};}
\end{tikzpicture}
\end{center}
\par\shortans{$0{,}54$}
\loigiai{
Xét tam giác $ABC$, ta có $\widehat{ACB} = 180^\circ - (\widehat{CAB} + \widehat{CBA}) = 180^\circ - (30^\circ + 45^\circ) = 105^\circ$.\\
Áp dụng định lý sin trong tam giác $ABC$, ta có $\dfrac{AC}{\sin\widehat{CBA}} = \dfrac{BC}{\sin\widehat{CAB}} = \dfrac{AB}{\sin\widehat{ACB}}$.\\
Suy ra $AC=\dfrac{AB \cdot \sin\widehat{CBA}}{\sin\widehat{ACB}}$ và $BC=\dfrac{AB \cdot \sin\widehat{CAB}}{\sin\widehat{ACB}}$.\\
Thời gian tàu $1$ đi từ $A$ đến $C$ là $t_1=\dfrac{AC}{80}$ (giờ).\\
Thời gian tàu $2$ đi từ $B$ đến $C$ là $t_2=\dfrac{BC}{80}$ (giờ).\\
Khi đó
$$t=t_1-t_2=\dfrac{AC}{80}-\dfrac{BC}{80}=\dfrac{200\cdot\sin45^\circ}{80\sin105^\circ}-\dfrac{200\cdot\sin30^\circ}{80\cdot\sin105^\circ}\approx0{,}54 \ (\text{giờ}).$$
Vậy $t=0{,}54$. 
}
\end{ex}

\Closesolutionfile{ans}
\begin{center}
	\textbf{PHẦN 4 - TỰ LUẬN} % 3 câu
\end{center}
\setcounter{ex}{0}
\begin{ex}%[0H4H1-3]%[Dự án D - Đợt 3 NH24-25 - Lê Hữu Kiệt]
Tính giá trị của biểu thức
$$A=\cos20^\circ+\cos40^\circ+\cos60^\circ+\cos80^\circ+\cos100^\circ+\cos120^\circ+\cos140^\circ+\cos160^\circ+\cos180^\circ.$$
\loigiai{Ta có $\cos(180^\circ-\alpha)=-\cos\alpha$, suy ra $\cos(180^\circ-\alpha)+\cos\alpha=0$, với mọi $0^\circ<\alpha<180^\circ$.\\
Khi đó
\begin{eqnarray*}
	&A&=\cos20^\circ+\cos40^\circ+\cos60^\circ+\cos80^\circ+\cos100^\circ+\cos120^\circ+\cos140^\circ+\cos160^\circ+\cos180^\circ \\
	&&=(\cos20^\circ+\cos160^\circ)+(\cos40^\circ+\cos140^\circ)+(\cos60^\circ+\cos120^\circ)+(\cos80^\circ+\cos100^\circ)+\cos180^\circ \\
	&&=\cos180^\circ \\
	&&= -1.
\end{eqnarray*}
Vậy $A=-1$.}
\end{ex}

\begin{ex}%[0H4V2-2]%[Dự án D - Đợt 3 NH24-25 - Lê Hữu Kiệt]
	Cho tam giác $ABC$ có $AB=4$, $BC=6$, $\widehat{ABC}=120^\circ$. Tính độ dài cạnh $AC$ và độ dài đường cao $BH$ của tam giác $ABC$.
	\loigiai{
	\immini
	{Ta có $AC^2=AB^2+BC^2-2\cdot AB\cdot BC \cos \widehat{ABC}=4^2+6^2-2\cdot 4\cdot 6\cdot \cos 120^\circ=76$.\\
	Suy ra $AC=2\sqrt{19}$.\\
	Ta có $S_{\triangle ABC}=\dfrac{1}{2}AB\cdot BC\cdot\sin \widehat{ABC}=120^\circ=\dfrac{1}{2}\cdot 4\cdot 6\cdot \sin 120^\circ=6\sqrt{3}$.\\
	Do đó, độ dài đường cao $BH$ là $BH=\dfrac{2S_{\triangle ABC}}{AC}=\dfrac{2\cdot 6\sqrt{3}}{2\sqrt{19}}=\dfrac{6\sqrt{57}}{19}$.}
	{\begin{tikzpicture}[scale=1, line cap=round,line join=round,font=\footnotesize]
	\path
	(0:0) coordinate (A)
	(0:4) coordinate (C)
	(46.5:2) coordinate (B)
	($(A)!(B)!(C)$) coordinate (H)
	pic[draw, angle radius=2mm]{angle=A--B--C}
	pic[draw, angle radius=2mm]{right angle=C--H--B};
	\draw (A)--(B)--(C)--cycle (B)--(H);
	\foreach \x/\g in {A/180,B/90,C/0,H/-90}{\fill (\x) circle (1pt) +(\g:.3) node {$\x$};}
	\end{tikzpicture}}
	}
\end{ex}

\begin{ex}%[0H4V3-2]%[Dự án D - Đợt 3 NH24-25 - Lê Hữu Kiệt]
	\immini[thm]{Hai trạm quan sát ở hai thành phố Đà Nẵng (giả sử là điểm $B$) và Nha Trang (giả sử là điểm $C$) đồng thời nhìn thấy vệ tinh (giả sử là điểm $A$) với góc nâng lần lượt là $75^\circ$ và $60^\circ$. Tính khoảng cách từ vệ tinh đến trạm quan sát tại thành phố Đà Nẵng và khoảng cách từ vệ tinh đến trạm quan sát tại Nha Trang, biết rằng khoảng cách giữa hai trạm quan sát là $520$ km (làm tròn kết quả đến hàng phần mười).
	}
	{	
	\begin{tikzpicture}[line join=round, line cap=round,scale=1,transform shape, font=\footnotesize]
			\clip (-1.8,-1.8) rectangle (4.5,3.8);
			\begin{pgfinterruptboundingbox}
				\tikzset{vetinh/.pic={
						\def\D{ 
							(.94,.77)--(1.2,.94)--(.95,.91)--(.85,.85)--cycle
							;}
						\draw \D;
						\fill[gray] \D;
						\def\V{ 
							(.6,.82)--(1.3,.92)--(1.4,1.05)--(.7,.95)--
							cycle;
						}
						\draw \V;
						\fill[pattern=north east lines] \V;
						\def\T{ 
							(.8,.72)--(.84,.9)
							..controls +(40:0.05) and +(120:0.1) .. (1,.75)--cycle;
						}
						\draw \T;
						\fill[black!50!white] \T;
				}}
				%Vẽ nhà 1
				\tikzset{nha/.pic={
						\draw (-1.05,-.25)coordinate [label=above:] (A)--(.12,3.52)coordinate (C)--(3,-.25)coordinate [label=above:] (B)--cycle;
						%\node at (-.68,0) [above]{\tiny $360$ km};
						\node at (.77,-0.7) [above]{$520$ km};
						%\node at (C) [left]{\tiny $13{,}2^\circ$};
						%\node at ([xshift=0.7cm,yshift=0.3cm]A){\tiny$75^\circ$};
						%\node at ([xshift=-0.7cm,yshift=0.3cm]B){\tiny$60^\circ$};
						%\tkzMarkAngles[size=.5](B,A,C);
						%\tkzMarkAngles[size=.5,double](C,B,A);
						\draw [yellow!40!gray,line width=0.7cm,decorate,decoration={random steps, segment length=1cm}]([yshift=-0.7cm,xshift=-.5]A) -- ([yshift=-0.7cm, xshift=0.3cm]B);
						\node at ([xshift=0.1cm,yshift=-1.25cm]A){\text{Đà Nẵng}};
						\node at ([yshift=-1.25cm]B){\text{Nha Trang}};
						\node at ([xshift=1cm]C){\text{Vệ tinh}};
						\def\N{ 
							(-1.22,-.25)--(-.82,-.25)--(-.82,-.85)--(-.82,-.85)--(-1.22,-.85)--cycle;}
						\draw \N;
						\fill[brown] \N;
				}}
				%ve nha 2
				\tikzset{nhahai/.pic={
						\def\NH{ 
							(-1.22,-.25)--(-.82,-.25)--(-.82,-.85)--(-.82,-.85)--(-1.22,-.85)--cycle;}
						\draw \NH;
						\fill[cyan] \NH;
				}}
				\tikzset{cuaso/.pic={
						\def\C{ 
							(-1.22,-.25)--(-.82,-.25)--(-.82,-.85)--(-.82,-.85)--(-1.22,-.85)--cycle;}
						\draw \C;
						\fill[white] \C;
				}}
				\tikzset{cay/.pic={
						\def\C{ 
							(-1.25,-.85)--(-1.24,-.75)--(-1.22,-.75)--(-1.2,-.85)--cycle
							;}
						\def\L{ 
							(-1.23,-.78)
							..controls +(-40:.13) and +(-160:0.16) .. (-1.27,-.7)
							..controls +(-40:.13) and +(-160:0.14) .. (-1.23,-.6)
							..controls +(120:.0) and +(50:0.11) .. (-1.2,-.7)
							..controls +(60:.1) and +(0:0.2) .. (-1.23,-.78)
							--cycle
							;}
						
						\draw \C;
						\fill[black!50!white] \C;
						\draw \L;
						\fill[green] \L;
				}}
				\path(0,0)pic[scale=1]{nha}
				(0,0)pic[fill=white,scale=.2,,xshift=-4.7cm,,yshift=-1.5cm]{cuaso}
				(0,0)pic[fill=white,scale=.2,,xshift=-4.1cm,,yshift=-1.5cm]{cuaso}
				(0,0)pic[fill=white,scale=.2,,xshift=-3.5cm,,yshift=-1.5cm]{cuaso}
				(0,0)pic[fill=white,scale=.4,,xshift=-1.55cm,,yshift=-1.25cm]{cuaso}
				(0,0)pic[scale=1]{cay}
				(4,0)pic[scale=1]{nhahai}
				(3.25,0)pic[fill=white,scale=.4,,xshift=0.35cm,,yshift=-1.25cm]{cuaso}
				(3.25,0)pic[fill=white,scale=.2,,xshift=-1cm,,yshift=-1.5cm]{cuaso}
				(3.25,0)pic[fill=white,scale=.2,,xshift=-.4cm,,yshift=-1.5cm]{cuaso}
				(3.25,0)pic[fill=white,scale=.2,,xshift=.2cm,,yshift=-1.5cm]{cuaso}
				(-0.5,3)pic[scale=.7]{vetinh}
				(3,0.1)pic[scale=1.3]{cay}
				(2,0.2)pic[scale=1.5]{cay}
			;
			\end{pgfinterruptboundingbox}
			\draw pic [draw,"$60^\circ$", angle eccentricity=1.5,angle radius =6 mm] {angle = C--B--A};
			\draw pic [draw,"$75^\circ$",double, angle eccentricity=1.5,angle radius =6 mm] {angle = B--A--C};
	\end{tikzpicture}
	}
	\loigiai{
	\immini{
	Bài toán đưa về tính cạnh $AB, AC$ của tam giác $ABC$ biết $\widehat{B}=75^\circ$, $\widehat{C}=60^\circ$ và $CB=520$ (km).\\
	Ta có $ \widehat{A}=180^\circ-\widehat{B}-\widehat{C}=180^\circ-75^\circ-60^\circ=45^\circ$.\\
	Áp dụng định lí sin trong tam giác $ABC,$ ta có
	\begin{itemize}
		\item[] $\dfrac{AB}{\sin C}=\dfrac{BC}{\sin A}\Rightarrow AB=\dfrac{BC\cdot \sin C}{\sin A}=\dfrac{520\cdot\sin 60^\circ }{\sin 45^\circ}\approx 636{,}9$ (km).
		\item[] $\dfrac{AC}{\sin B}=\dfrac{BC}{\sin A}\Rightarrow AC=\dfrac{BC\cdot \sin B}{\sin A}=\dfrac{520\cdot\sin 75^\circ }{\sin 45^\circ}\approx 710{,}3$ (km).
	\end{itemize}
	Vậy khoảng cách từ vệ tinh đến trạm quan sát tại thành phố Đà Nẵng và trạm quan sát tại Nha Trang lần lượt là $636{,}9$ (km) và $710{,}3$ (km).}
	{\begin{tikzpicture}[font=\footnotesize, line join=round, line cap=butt, >=stealth, scale=1]
	\path (0,0) coordinate (B)+(75:0.1) coordinate (b)
	(4,0) coordinate (C)+(120:0.1) coordinate (c)
	(intersection of B--b and C--c) coordinate (A)
	pic[draw, angle radius=2mm, angle eccentricity=2.5, double, "$75^\circ$"]{angle=C--B--A}
	pic[draw, angle radius=2mm, angle eccentricity=2.5, "$60^\circ$"]{angle=A--C--B};
	\draw (A)--(B)--(C)--cycle;
	\node[below] at ($(B)!1/2!(C)$) {$520$ km};
	\foreach \x/\g in {B/-135, C/-45, A/90}{\fill (\x) circle (1pt)+(\g:0.25)node{$\x$};}
	\end{tikzpicture}}
	}
\end{ex}
