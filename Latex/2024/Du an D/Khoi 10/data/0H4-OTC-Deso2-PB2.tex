\newpage
\def\thoigian{90}%--Thời gian
\de{Đề số 2}{Chương IV. Hệ thức lượng trong tam giác}

\begin{center}
	\textbf{PHẦN 1 - CÂU TRẮC NGHIỆM BỐN PHƯƠNG ÁN}% 12 câu
\end{center}
\setcounter{ex}{0}
\Opensolutionfile{ans}[ans/ans-TN-10-ONTAPCHUONG-IV-DE2]
%Cau1
\begin{ex}%[0H4N1-1]%[Dự án đề cương 3 Khối NH24-25-Dot 3- Bùi Lương Phúc]
	Giá trị của biểu thức $A=2\cos 45^\circ+\sin 0^\circ-\tan120^\circ$ bằng
	\choice
	{\True $\sqrt{2}+\sqrt{3}$}
	{$\sqrt{2}-\sqrt{3}$}
	{$\sqrt{5}$}
	{$0$}
	\loigiai{
		Ta có $A=2\cos 45^\circ+\sin 0^\circ-\tan120^\circ=2\cdot\dfrac{\sqrt{2}}{2}+0+\sqrt{3}=\sqrt{2}+\sqrt{3}$.
	}
\end{ex}
%Cau2
\begin{ex}%[0H4H1-2]%[Dự án đề cương 3 Khối NH24-25-Dot 3- Bùi Lương Phúc]
	Cho góc $\alpha$ thỏa mãn $0^\circ<\alpha<180^\circ$ và $\cot \alpha=-\dfrac{1}{2}$. Giá trị $\cos \alpha$ bằng
	\choice
	{$\pm \dfrac{\sqrt{5}}{5}$}
	{$-\dfrac{\sqrt{5}}{2}$}
	{\True $-\dfrac{\sqrt{5}}{5}$}
	{$\dfrac{\sqrt{5}}{5}$}
	\loigiai{
		Ta có $ \cot \alpha <0 $ và $ 0^\circ<\alpha<180^\circ\Rightarrow 90^\circ<\alpha<180^\circ \Rightarrow\cos \alpha <0$.\\
		Ta có $ \tan \alpha =\dfrac{1}{\cot \alpha }=-2 $.\\
		$\dfrac{1}{\cos^2\alpha}=\dfrac{\cos^2\alpha+\sin^2\alpha}{\cos^2\alpha}=1+\dfrac{\sin^2\alpha}{\cos^2\alpha}=1+\tan^2\alpha=1+(-2)^2=5 \Rightarrow \cos^2 \alpha= \dfrac{1}{5}\Rightarrow \cos \alpha=- \dfrac{\sqrt{5}}{5}$.
	}
\end{ex}
%Cau3
\begin{ex}%[0H4N1-1]%[Dự án đề cương 3 Khối NH24-25-Dot 3- Bùi Lương Phúc]
	Cho góc $\alpha$ với $90^\circ < \alpha <180^\circ$. Mệnh đề nào sau đây là mệnh đề đúng?
	\choice
	{\True $\cos \alpha < 0$}
	{$\sin \alpha > 0$}
	{$\cot \alpha > 0$}
	{$\tan \alpha > 0$}
	\loigiai{
		Vì $90^\circ < \alpha <180^\circ$ nên $\cos \alpha < 0$.	
	}
\end{ex}
%Cau4
\begin{ex}%[0H4N1-1]%[Dự án đề cương 3 Khối NH24-25-Dot 3- Bùi Lương Phúc]
	Cho góc $\alpha$ tùy ý thỏa mãn $0^\circ < \alpha <180^\circ$. Giá trị $\cos \left(180^\circ-\alpha\right)$ bằng
	\choice
	{\True $-\cos \alpha$}
	{$\cos \alpha$}
	{$\sin \alpha$}
	{$-\sin \alpha$}
	\loigiai{
		Vì $\cos \left(180^\circ-\alpha\right)=-\cos \alpha$.	
	}
\end{ex}
%Cau5
\begin{ex}%[0H4N2-1]%[Dự án đề cương 3 Khối NH24-25-Dot 3- Bùi Lương Phúc]
	Cho tam giác $ABC$. Mệnh đề nào dưới đây \textbf{sai}?
	\choice
	{\True $a^2=b^2+c^2+2 b c \cos A$}
	{$\cos B=\dfrac{a^2+c^2-b^2}{2ac}$}
	{$a^2=b^2+c^2-2 b c \cos A$}
	{$\cos A=\dfrac{b^2+c^2-a^2}{2bc}$}
	\loigiai{
		Ta có $ a^2=b^2+c^2-2 b c \cos A $ nên đẳng thức $a^2=b^2+c^2+2 b c \cos A$ sai.
	}
\end{ex}
%Cau6
\begin{ex}%[0H4N2-1]%[Dự án đề cương 3 Khối NH24-25-Dot 3- Bùi Lương Phúc]
	Cho tam giác $ABC$ có $AB=7$, $AC=8$, $\widehat{BAC}=120^{\circ}$. Độ dài cạnh $BC$ là
	\choice
	{$\sqrt{57}$}
	{$\sqrt{141}$}
	{\True $13$}
	{$169$}
	\loigiai{
		Áp dụng định lý côsin cho tam giác $ABC$, ta có
		\[BC^2=AB^2+AC^2-2\cdot AB \cdot AC \cdot \cos\widehat{BAC}
		= 7^2+8^2-2 \cdot 7 \cdot 8 \cdot \cos 120^{\circ}=169.\]
		Suy ra $BC=13$.
	}
\end{ex}
%Cau7
\begin{ex}%[0H4H2-1]%[Dự án đề cương 3 Khối NH24-25-Dot 3- Bùi Lương Phúc]
	Cho tam giác $ABC$ có $\widehat B=60^\circ$, $BC=8$, $AB=5$. Độ dài cạnh $AC$ bằng
	\choice
	{\True $7$}
	{$129$}
	{$49$}
	{$\sqrt{129}$}
	\loigiai{
		Ta có $b^2 = a^2 + c^2 - 2a\cdot c\cdot \cos B = 8^2 + 5^2 - 2\cdot 8 \cdot 5 \cdot \cos 60^\circ = 49 \Rightarrow b = 7$.
	}
\end{ex}
%Cau8
\begin{ex}%[0H4N2-1]%[Dự án đề cương 3 Khối NH24-25-Dot 3- Bùi Lương Phúc]
	Cho tam giác $ABC$ có $\widehat{ABC} = 45^{\circ}$, $\widehat{ACB} = 60^{\circ}$ và $AB = 3$. Tính $AC$.
	\choice
	{\True $AC = \sqrt{6}$}
	{$AC = 3\sqrt{2}$}
	{$AC = 6$}
	{$AC = 2\sqrt{3}$}
	\loigiai{
		Áp dụng định lí $\sin$ vào tam giác $ABC$ ta có
		\allowdisplaybreaks
		\begin{eqnarray*}
			&&\dfrac{AC}{\sin{\widehat{ABC}}}=\dfrac{AB}{\sin{\widehat{ACB}}}\\
			&\Rightarrow&\dfrac{AC}{\sin 45^{\circ}} = \dfrac{3}{\sin 60^{\circ}} \\
			&\Rightarrow& AC =\sin 45^{\circ}\cdot \dfrac{3}{\sin 60^{\circ}} = \sqrt{6}.
		\end{eqnarray*}
	}
\end{ex}
%Cau9
\begin{ex}%[0H4H2-1]%[Dự án đề cương 3 Khối NH24-25-Dot 3- Bùi Lương Phúc]
Cho tam giác $ ABC$ có $\widehat{A}=30^\circ$, $ BC=5$\,cm. Bán kính $ R$ của đường tròn ngoại tiếp tam giác $ ABC$ là
\choice
{$ R=\dfrac{2}{5}$ (cm)}
{$ R=\dfrac{5}{2}$ (cm)}
{\True $ R=5$ (cm)}
{$ R=10$ (cm)}
\loigiai{
	Ta có $\dfrac{BC}{\sin A}=2R \Rightarrow R=\dfrac{BC}{2\sin A}=\dfrac{5}{2\cdot\sin30^\circ}=\dfrac{5}{2\cdot\dfrac{1}{2}}=5$\,(cm).
	}
\end{ex}
%Cau10
\begin{ex}%[0H4N2-2]%[Dự án đề cương 3 Khối NH24-25-Dot 3- Bùi Lương Phúc]
	Cho tam giác $ABC$ có $BC=a$, $CA=b$, $AB=c$. Diện tích $S$ của tam giác $ABC$ là
	\choice
	{\True $S=\dfrac{1}{2}bc\sin A$}
	{$S=\dfrac{1}{2}ac\sin A$}
	{$S=\dfrac{1}{2}bc\sin B$}
	{$S=\dfrac{1}{2}bc\sin B$}
	\loigiai
	{
		Ta có $S=\dfrac{1}{2}bc\sin A = \dfrac{1}{2}ac\sin B = \dfrac{1}{2}ab\sin C$.
	}
\end{ex}
%Cau11
\begin{ex}%[0H4H3-1]%[Dự án đề cương 3 Khối NH24-25-Dot 3- Bùi Lương Phúc]
	Tam giác $ABC$ có $BC=16{,}8$; $\widehat{B}=56^\circ13'$; $\widehat{C}=71^\circ$. Độ dài cạnh $AB$ gần nhất với giá trị nào dưới đây?
	\choice
	{$29{,}9$}
	{$14{,}1$}
	{$17{,}5$}
	{\True $19{,}9$}
	\loigiai
	{
		Ta có $\widehat{A} = 180^\circ - 56^\circ13' - 71^\circ = 52^\circ47'$.\\
		Áp dụng định lý sin suy ra $AB = \dfrac{BC\sin C}{\sin A} = \dfrac{16{,}8\cdot \sin 71^\circ}{\sin 52^\circ47'} \approx 19{,}9$.
	}
\end{ex}
%Cau12
\begin{ex}%[0H4H2-2]%[Dự án đề cương 3 Khối NH24-25-Dot 3- Bùi Lương Phúc]
	Một tam giác có ba cạnh là $13$, $14$, $15$. Diện tích tam giác đó bằng
	\choice
	{$91$}
	{$7\,056$}
	{$21$}
	{\True $84$}
	\loigiai
	{
		Ta có nửa chu vi của tam giác là $p = \dfrac{13+14+15}{2} = 21$.\\
		Áp dụng công thức Heron ta có
		$$S = \sqrt{21(21-13)(21-14)(21-15)} = \sqrt{21\cdot8\cdot7\cdot6} = \sqrt{7\,056} = 84.$$
	}
\end{ex}
\Closesolutionfile{ans}

\begin{center}
	\textbf{PHẦN 2 - CÂU TRẮC NGHIỆM ĐÚNG SAI}%2 câu
\end{center}
\setcounter{ex}{0}
\Opensolutionfile{ans}[ans/answer-DS-10-ONTAPCHUONG-IV-DE2]
%%Cau1
\begin{ex}%[Dự án đề cương 3 Khối NH24-25-Dot 3- Bùi Lương Phúc]
Cho tam giác $ABC$ có $\tan A=-\sqrt{3}$.
\choiceTF
{$\cos A>0$}
{\True $\cos A=-\dfrac{1}{2}$}
{\True $\sin A=\dfrac{\sqrt{3}}{2}$}
{\True $\cot \dfrac{B+C}{2}=\sqrt{3}$}
\loigiai{
	\begin{itemchoice}
	\itemch Vì $0^\circ<\widehat A<180^\circ$ và $\tan A=-\sqrt{3}$ nên $90^\circ<\widehat A<180^\circ$.\\
	Vậy $\cos A<0$.
	\itemch Ta có $\tan A=-\sqrt{3}\Rightarrow \widehat A=120^\circ\Rightarrow \cos A=\dfrac{1}{2}$.\\
	\textit{Cách khác:}\\
	$\cos A=-\sqrt{\dfrac{1}{1+\tan^2 A}}=-\sqrt{\dfrac{1}{1+\left(-\sqrt{3}\right)^2}}=-\dfrac{1}{2}$.
	\itemch Ta có $\tan A=-\sqrt{3}\Rightarrow \widehat A=120^\circ\Rightarrow \sin A=\dfrac{\sqrt{3}}{2}$.\\
	\textit{Cách khác:}\\
	$\sin A=\sqrt{1-\cos^2 A}=\sqrt{1-\left(-\dfrac{1}{2}\right)^2}=\dfrac{\sqrt{3}}{2}$.
	\itemch $\tan A=-\sqrt{3}\Rightarrow \widehat A=120^\circ$.\\
	Mặt khác, $\widehat B+\widehat C=180^\circ-\widehat A$ nên $\widehat B+\widehat C=180^\circ-120^\circ=60^\circ$.\\
	Suy ra, $\cot \dfrac{B+C}{2}=\cot \dfrac{60^\circ}{2}=\cot 30^\circ=\sqrt{3}$.
	\end{itemchoice}
}
\end{ex}
%%Cau2
\begin{ex}	%[Dự án đề cương 3 Khối NH24-25-Dot 3- Bùi Lương Phúc]
\immini[thm]{
Một người đứng ở vị trí $A$ trên sân thượng của một ngôi nhà cao $4$\,m đang quan sát một cây cao cách ngôi nhà $20$\,m và đo được $\widehat {BAC} = 45^\circ$ (xem hình vẽ bên).
\choiceTF
{\True Độ dài $AB$ bằng $4\sqrt{26}$\,m}
{\True Số đo $\widehat {ABH}$ bằng $11{,}3^\circ$ (kết quả được làm tròn đến hàng phần mười theo đơn vị độ)}
{Số đo $\widehat {ACB}$ lớn hơn $57^\circ$}
{Cây cao không vượt quá $17$\,m}
}{
\begin{tikzpicture}[line join=round, font=\footnotesize, line cap=round, scale=0.85,>=stealth]
\clip (0,-4.5) rectangle (7,1.5);
	\path (6,-4) coordinate (B) 
	(0.42,-4) coordinate (H) 
	($(H)+(0,1.15)$) coordinate (A) 
	(6,1) coordinate (C); 	
	\node at ($(H)!0.5!(B)$) [below]{$20$\,m};
	\node at ($(H)!0.5!(A)$) [rotate=-90,below]{$4$\,m}; 
	\draw pic["$45^\circ$", draw=black, angle eccentricity=1.5, angle radius=0.6cm]{angle=B--A--C};
	\draw pic[draw=black, angle radius=0.2cm]{right angle=A--H--B};
	\def\Than{
		(-.56,0)
		..controls +(-50:.6) and +(40:.5) ..  (-1,-3)
		..controls +(20:.9) and +(-160:.3) ..  (-0.09,-2.7)
		..controls +(-120:.3) and +(60:.3) ..  (-0.26,-3.1)
		..controls +(-30:.2) and +(-140:.3) ..  (.3,-2.8)
		..controls +(-20:.2) and +(-160:.3) ..  (1.14,-2.8)
		..controls +(170:.9) and +(-160:.2) ..  (.7,0.3)
		..controls +(110:0.2) and +(80:1.5) ..  (-.64,0.3)
	}
	\def\La{
		(0,.6)
		..controls +(-100:.5) and +(-60:.4) ..  (-.6,0.5)
		..controls +(-100:.5) and +(-60:.6) ..  (-1.2,0.3)
		..controls +(-120:.5) and +(-110:.1) ..  (-1.8,.4)
		..controls +(-150:0.5) and +(-140:.5) ..  (-2.4,1.4)
		..controls +(-170:.8) and +(-170:.7) ..  (-1.8,2.2)
		..controls +(140:.7) and +(110:.8) ..  (-.74,2.7)
		..controls +(110:.5) and +(80:.6) ..  (-.3,2.6)
		..controls +(80:.6) and +(95:1.6) ..  (1.4,2.2)
		..controls +(80:.4) and +(95:.4) ..  (2.4,2.2)
		..controls +(20:.2) and +(95:.2) ..  (2.6,2)
		..controls +(-20:.8) and +(35:.5) ..  (2.2,.94)
		..controls +(-30:.6) and +(-20:.6) ..  (1.8,0.5)
		..controls +(-120:.6) and +(-60:.4) ..  (.7,0.3)
		..controls +(175:.4) and +(-160:.2) ..  (.4,0.3)
		..controls +(-160:.2) and +(-70:.2) ..  (0,.4)
	}
	\def\LaPhu{
		(-2.3,1.4)
		..controls +(-170:.8) and +(-170:.7) ..  (-1.2,2.2)
		..controls +(140:.7) and +(110:.8) ..  (-.74,2.4)
		..controls +(110:.5) and +(80:.6) ..  (-.3,2.4)
		..controls +(80:.6) and +(95:1.6) ..  (1.1,2.2)
		..controls +(80:.4) and +(95:.4) ..  (1.6,2.2)
		..controls +(20:.2) and +(95:.2) ..  (1.9,2)
		..controls +(-20:.8) and +(35:.5) ..  (2.1,1.2)
		..controls +(-50:1) and +(-85:1.2) ..  (1.2,1.1)
		..controls +(-140:.8) and +(-120:.8) ..  (0,1.1)
		..controls +(-150:.4) and +(-80:.8) ..  (-.86,1.2)
		..controls +(-140:1) and +(-130:.8) ..  (-1.82,1.2)
	}
	\begin{scope}[shift={(6,-1.6)}, xscale=0.35, yscale=0.85]
		\fill[red!60!green!90!blue!65] \Than;
		\fill[red!25!green!75!blue!100] \La;
		\fill[red!0!green!80!blue!80,opacity=0.75] \LaPhu;
	\end{scope}
	\draw (H)--(A)--(B)--cycle (A)--(C);
	\foreach \a/\g in {H/-90,A/90,B/-60,C/90}{
		\draw[fill=black](\a) circle (1pt) +(\g:.3)node{$\a$};
	}
	\end{tikzpicture}
}	
	\loigiai{
		\begin{itemchoice}
			\itemch Trong tam giác $ABH$ vuông tại $H$ có $AB=\sqrt{AH^2+BH^2}=\sqrt{4^2+20^2}=4\sqrt{26}$\,m.
			\itemch Xét tam giác $ABH$ vuông tại $H$ có
			\[\tan \widehat {ABH}=\dfrac{AH}{BH}=\dfrac{4}{20}= \dfrac{1}{5}\Rightarrow \widehat {ABH}\approx11{,}3^\circ.\]				
			\itemch Do $BC\perp BH\Rightarrow\widehat {ABC}=90^\circ-\widehat {ABH}$.\\
			Xét tam giác $ABC$, ta có $\widehat C=180^\circ-\widehat A-\widehat B=180^\circ-45^\circ-\left(90^\circ-\widehat {ABH}\right)=45^\circ+\widehat {ABH}\approx56{,}3^\circ$.\\
			Vậy $\widehat {ACB}\approx 56{,}3^\circ$.
			\itemch Áp dụng định lý sin đối với tam giác $ABC$, ta có
			\[\dfrac{BC}{\sin A}=\dfrac{AB}{\sin C}\Rightarrow BC=\dfrac{AB\cdot \sin A}{\sin C}.
			\] 
			Thay số vào ta được
			\[BC\approx\dfrac{4\sqrt{26}\cdot \sin 45^\circ}{\sin 56{,}3^\circ}\approx17{,}3\ \text{(m)}.
			\]
			Vậy cây cao khoảng $17{,}3$\,m.
		\end{itemchoice}
	}
\end{ex}
\Closesolutionfile{ans}

\begin{center}
\textbf{PHẦN 3 - CÂU TRẮC NGHIỆM TRẢ LỜI NGẮN}% 4 câu 
\end{center}
\setcounter{ex}{0}
\Opensolutionfile{ans}[ans/ans-KQ-10-ONTAPCHUONG-IV-DE2]
%%%Cau1
\begin{ex}%[0H4H2-4]%[Dự án đề cương 3 Khối NH24-25-Dot 3- Bùi Lương Phúc]
Cho góc $\alpha$ thỏa mãn $0^\circ<\alpha<180^\circ$ và $\tan \alpha = -3$. Giá trị biểu thức $A = \dfrac{\sin \alpha - 2\cos \alpha}{2\sin \alpha+\cos \alpha }$ bằng bao nhiêu?
	\par\shortans{$1$}
	\loigiai{
		Ta có $\tan \alpha = -3\Rightarrow \cos \alpha\neq 0$ và $\tan \alpha = -3\Rightarrow \dfrac{\sin \alpha}{\cos \alpha}=-3 \Rightarrow \sin \alpha =-3\cos \alpha$.\\
		Suy ra, $A = \dfrac{\sin \alpha - 2 \cos \alpha}{2 \sin \alpha+ \cos \alpha } = \dfrac{-3\cos \alpha - 2 \cos \alpha}{2\cdot (-3\cos \alpha) + \cos \alpha}== \dfrac{-5\cos \alpha}{-5\cos \alpha}=1$.
	}
\end{ex}
%%%Cau2
\begin{ex}%[0H4H2-4]%[Dự án đề cương 3 Khối NH24-25-Dot 3- Bùi Lương Phúc]
	Cho tam giác $ABC$ có $BC=a$, $CA=b$, $AB=c$ thỏa mãn $b^2 + c^2 - a^2 = \sqrt{3}bc$. Số đo góc $A$ bằng bao nhiêu độ?
	\par\shortans{$30$}
	\loigiai{
		Ta có $\cos A = \dfrac{b^2 + c^2 - a^2}{2bc} = \dfrac{\sqrt{3}bc}{2bc} = \dfrac{\sqrt{3}}{2} \Rightarrow \widehat A = 30^\circ$.
	}
\end{ex}
%%%Cau 3
\begin{ex}%[0H4V2-1]%[Dự án đề cương 3 Khối NH24-25-Dot 3- Bùi Lương Phúc]
	Cho tam giác $ABC$ có $AB = 5$, $AC = 8$, $\widehat{A} = 60^\circ$. Gọi $D$ là điểm nằm trên cạnh $BC$ sao cho $\widehat {BAD}=30^\circ$. Độ dài đoạn thẳng $AD$ bằng bao nhiêu (làm tròn kết quả đến hàng phần mười)?
	\par\shortans{$5{,}3$}
	\loigiai{
		\begin{center}
			\begin{tikzpicture}[scale=1,line cap=butt,line join=round,font=\footnotesize,>=Stealth]
			\path 
			coordinate (A) at (0,0)
			coordinate (C) at ($(A)+(4,0)$)
			coordinate (B) at ($(A)+({2.5*cos(60)},{2.5*sin(60)})$)
			coordinate (D) at ($(A)+({2.66*cos(30)},{2.66*sin(30)})$);
			\draw(A)--(B)--(C)--cycle (A)--(D);
			\fill (A) circle (1pt) node [below]{$A$};
			\fill (B) circle (1pt) node [above]{$B$};
			\fill (C) circle (1pt) node [below]{$C$};
			\fill (D) circle (1pt) node [above right]{$D$};
			\draw pic[draw=black,"$\scriptstyle 30^\circ$",double,angle eccentricity=1.8, angle radius=0.4cm] {angle = D--A--B};
			\draw pic[draw=black,pic text=$\scriptstyle 60^\circ$,
			pic text options={xshift=1pt, yshift=-5pt},angle eccentricity=2, angle radius=0.3cm] {angle = C--A--B};			
			\node [left] at ($(A)!0.5!(B)$){$5$};
			\node [below] at ($(A)!0.5!(C)$){$8$};
			\end{tikzpicture}
		\end{center}
		Ta có $S_{\triangle ABC}=S_{\triangle ABD}+S_{\triangle CBD}$.\\
		Áp dụng công thức tính diện tích tam giác ta được
		\begin{align*}
			& \dfrac{1}{2}AB\cdot AC\cdot \sin \widehat {CAB}=\dfrac{1}{2}AB\cdot AD\cdot \sin \widehat {DAB}+\dfrac{1}{2}AC\cdot AD\cdot \sin \widehat {CAD}\\
			\Rightarrow 
			& AB\cdot AC\cdot \sin \widehat {CAB}=AB\cdot AD\cdot \sin \widehat {DAB}+AC\cdot AD\cdot \sin \widehat {CAD}\\
			\Rightarrow 
			& AD=\dfrac{AB\cdot AC\cdot \sin \widehat {CAB}}{AB\cdot \sin \widehat {DAB}+AC\cdot \sin \widehat {CAD}}.
		\end{align*}
	Lại có $\widehat {CAD}=\widehat {CAB}-\widehat {DAB}=60^\circ-30^\circ=30^\circ$.\\ Thay số vào ta được
	\[AD=\dfrac{5\cdot 8\cdot \sin 60^\circ}{5\cdot \sin 30^\circ+8\cdot \sin 30^\circ}= \dfrac{40 \cdot \dfrac{\sqrt{3}}{2}}{\dfrac{13}{2}}
	= \dfrac{40\sqrt{3}}{13} \approx 5{,}33.
	\]
	Vậy độ dài đoạn thẳng $AD$ là khoảng $5{,}3$.
	}
\end{ex}
%%%Cau 4
\begin{ex}%[0H4V2-1]%[Dự án đề cương 3 Khối NH24-25-Dot 3- Bùi Lương Phúc]
\immini[thm]{
Bạn Minh đứng ở bờ sông muốn đo độ rộng của khúc sông chỗ chảy qua vị trí đang đứng. Giả thiết rằng, khúc sông tương đối thẳng và hai bờ sông song song với nhau.	Từ vị trí đang đứng $A$, Minh đo được góc nghiêng $\alpha = 35^\circ$ so với bờ sông tới một vị trí $C$ quan sát được ở phía bờ bên kia. Sau đó Minh di chuyển dọc bờ sông đến vị trí $B$ cách $A$ một khoảng $50$\,m và tiếp tục đo được góc nghiêng $\beta = 65^\circ$ so với bờ sông tới vị trí $C$ đã chọn (xem hình vẽ bên). \\
Hỏi độ rộng của con sông chỗ chảy qua vị trí của bạn Minh đang đứng là bao nhiêu mét (\textit{không làm tròn các phép tính trung gian, chỉ làm tròn kết quả cuối cùng đến hàng đơn vị})?}{
	\begin{tikzpicture}[scale=1, line join=round, line cap=round, font=\footnotesize, >=stealth]
	\def\a{2.5}
	\pgfmathsetmacro{\yC}{\a * sin(20) * 2.5} 
	\pgfmathsetmacro{\ystep}{0.64}
	\pgfmathsetmacro{\amp}{0.5} 
	\path
	coordinate (A) at (0,0)
	coordinate (B) at (\a,0)
	coordinate (D) at ($(\a,0)+(2,0)$)
	coordinate (C1) at ($(A)+({\a*cos(35)},{\a*sin(35)})$)
	coordinate (C) at ($(A)!1.81!(C1)$);
	\draw ($(A)+(-1,0)$) -- ($(A)+(5,0)$);
	\draw ($(C)+(-4.6,0)$) -- ($(C)+(1.4,0)$);
	\fill (A) circle (1pt) node[below] {$A$};
	\fill (B) circle (1pt) node[below] {$B$};
	\fill (C) circle (1pt) node[above] {$C$};
	\draw(A) -- (C) -- (B);
	\node[below] at ($(A)!0.5!(B)$) {$50$\,m};
	\draw pic["$35^\circ$", draw=black, angle eccentricity=1.7, angle radius=0.4cm]{angle=B--A--C};
	\draw pic["$65^\circ$", double, draw=black, angle eccentricity=2, angle radius=0.3cm]{angle=D--B--C};
	\foreach \x in {-0.5,0.5,...,4.5} {
		\foreach \y in {0,\ystep,...,\yC} {
			\draw[dashed,green!30!blue]
			({\x - \amp}, \y)
			.. controls (\x, {\y + 0.25}) and (\x, {\y + 0.45})
			.. ({\x + \amp}, {\y + \ystep});
		}
	}		
\end{tikzpicture}
}
	\par\shortans{$52$}
	\loigiai{
\begin{center}
		\begin{tikzpicture}[scale=1, line join=round, line cap=round, font=\footnotesize, >=stealth]
	\def\a{2.5}
	\path
	coordinate (A) at (0,0)
	coordinate (B) at (\a,0)
	coordinate (D) at ($(\a,0)+(2,0)$)
	coordinate (C1) at ($(A)+({\a*cos(35)},{\a*sin(35)})$)
	coordinate (C) at ($(A)!1.81!(C1)$);
	\coordinate (H) at ($(A)!(C)!(B)$);
	\draw ($(A)+(-1,0)$) -- ($(A)+(5,0)$);
	\draw ($(C)+(-4.6,0)$) -- ($(C)+(1.4,0)$);
	\fill (A) circle (1pt) node[below] {$A$};
	\fill (B) circle (1pt) node[below] {$B$};
	\fill (C) circle (1pt) node[above] {$C$};
	\fill (H) circle (1pt) node[below] {$H$};
	\draw(A) -- (C) -- (B);
	\draw[<->,dashed](H) -- (C);
	\node[below] at ($(A)!0.5!(B)$) {$50$\,m};
	\draw pic["$35^\circ$", draw=black, angle eccentricity=1.7, angle radius=0.4cm]{angle=B--A--C};
	\draw pic["$65^\circ$", double, draw=black, angle eccentricity=2, angle radius=0.3cm]{angle=D--B--C};
	\draw pic[draw=black, angle radius=0.2cm]{right angle=C--H--B};
		\end{tikzpicture}
\end{center}
		Xét tam giác $ABC$ có $\widehat{ABC} = 180^\circ - 65^\circ = 115^\circ$.\\
		Suy ra
		\[\widehat{C} = 180^\circ - \widehat{A} - \widehat{ABC} = 180^\circ - 35^\circ - 115^\circ = 30^\circ.
		\]		
		Áp dụng định lí sin cho tam giác $ABC$ ta có
		\[
		\dfrac{BC}{\sin A} = \dfrac{AB}{\sin C} \implies BC = \dfrac{AB \sin A}{\sin C} = \dfrac{50 \sin 35^\circ}{\sin 30^\circ} = 100 \sin 35^\circ.
		\]		
		Gọi $H$ là hình chiếu vuông góc của $C$ lên đường thẳng $AB$, ta có
		\[
		CH = BC \sin 65^\circ = 100 \sin 35^\circ \cdot \sin 65^\circ \approx 52\ \text{(m)}.\]		
		Vậy độ rộng của con sông là khoảng $52$ mét.
	}
\end{ex}
\Closesolutionfile{ans}

\begin{center}
	\textbf{PHẦN 4 - TỰ LUẬN} % 3 câu
\end{center}
\setcounter{ex}{0}
\Opensolutionfile{ans}[ans/ans-TL-10-ONTAPCHUONG-IV-DE2]
%%%%Cau1
\begin{ex}%[0H4V2-1]%[Dự án đề cương 3 Khối NH24-25-Dot 3- Bùi Lương Phúc]
	Thực hiện phép tính (không sử dụng MTCT)
	\[A=\sin 10^\circ + \sin 20^\circ+ \sin 30^\circ+ \sin 40^\circ+\cos 100^\circ +\cos 110^\circ+\cos 120^\circ +\cos 130^\circ.\]
	\dapso{$A=0$}
\loigiai{
	Ta có \\
	$\cos 100^\circ=\cos \left(180^\circ-80^\circ\right)=-\cos 80^\circ=-\cos \left(90^\circ-10^\circ\right)=-\sin10^\circ$.\\
	$\cos 110^\circ=\cos \left(180^\circ-70^\circ\right)=-\cos 70^\circ=-\cos \left(90^\circ-20^\circ\right)=-\sin20^\circ$.\\
	$\cos 120^\circ=\cos \left(180^\circ-60^\circ\right)=-\cos 60^\circ=-\cos \left(90^\circ-30^\circ\right)=-\sin30^\circ$.\\	
	$\cos 130^\circ=\cos \left(180^\circ-50^\circ\right)=-\cos 50^\circ=-\cos \left(90^\circ-40^\circ\right)=-\sin40^\circ$.\\
	Cộng vế với vế 4 đẳng thức trên, sau đó thế vào biểu thức $A$ ta được\\
	$A=\sin 10^\circ + \sin 20^\circ+ \sin 30^\circ+ \sin 40^\circ-\left(\sin 10^\circ + \sin 20^\circ+ \sin 30^\circ+ \sin 40^\circ\right)=0$.\\
	Vậy $A=0$.
}	
\end{ex}
%%%%Cau2
\begin{ex}%[0H4H2-2]%[Dự án đề cương 3 Khối NH24-25-Dot 3- Bùi Lương Phúc]
	Cho tam giác $ABC$ có $AB=5$, $AC=4$, diện tích $S = 5\sqrt{3}$ và $\widehat{A}$ là góc tù. Tính bán kính đường tròn ngoại tiếp tam giác $ABC$.	
	\dapso{$\dfrac{\sqrt{183}}{3}$}
	\loigiai{Ta có\\
		$\bullet$	$\sin A = \dfrac{2\cdot S_{\triangle ABC}}{AB\cdot AC} = \dfrac{2\cdot 5\sqrt{3}}{5\cdot 4} = \dfrac{\sqrt{3}}{2}$.\\
		$\bullet$	$\cos A = -\sqrt{1-\sin^2A} = -\sqrt{1-\dfrac{3}{4}} = -\dfrac{1}{2}$.\\
		$\bullet$	$BC^2 = AB^2 + AC^2 - 2AB\cdot AC\cdot \cos A = 61\Rightarrow BC = \sqrt{61}$.\\
		$\bullet$	$\dfrac{BC}{\sin A} = 2R \Leftrightarrow \dfrac{\sqrt{61}}{\dfrac{\sqrt{3}}{2}} = 2R \Rightarrow R = \dfrac{\sqrt{183}}{3}$.\\
		Vậy bán kính đường tròn ngoại tiếp tam giác $ABC$ bằng $\dfrac{\sqrt{183}}{3}$.
	}
\end{ex}
%%%%Cau3
\begin{ex}%[0H4H2-2]%[Dự án đề cương 3 Khối NH24-25-Dot 3- Bùi Lương Phúc]
	Lúc 6 giờ sáng, bạn An đi xe đạp từ nhà (điểm $A$) đến trường (điểm $B$) phải leo lên và xuống một con dốc với đỉnh dốc là điểm $C$. Cho biết đoạn thẳng $AB$ dài $762$\,m, $\widehat{A} = 6^\circ$, $\widehat{B} = 4^\circ$ như hình vẽ dưới đây.
	\begin{enumerate}
		\item Tính chiều cao của con dốc theo đơn vị mét (làm tròn kết quả đến hàng đơn vị).
		\item Hỏi bạn An đến trường lúc mấy giờ? Biết rằng tốc độ trung bình lên dốc là $4$\,km/h và tốc độ trung bình khi xuống dốc là $19$\,km/h.
	\end{enumerate}
	\begin{center}
				\begin{tikzpicture}[scale=1, line join=round, line cap=round, font=\footnotesize, >=stealth]
			\def\a{1.5}
			\path
			coordinate (A) at (0,0)
			coordinate (B) at (7.6*\a,0)
			coordinate (D) at ($(8*\a,0)+(2,0)$)
			coordinate (C1) at ($(A)+({\a*cos(6)},{\a*sin(6)})$)
			coordinate (D1) at ($(B)+({\a*cos(176)},{\a*sin(176)})$)
			coordinate (C) at ($(A)!3.1!(C1)$);
			\coordinate (H) at ($(A)!(C)!(B)$);
			\fill (A) circle (1pt) node[below] {$A$};
			\fill (B) circle (1pt) node[below] {$B$};
			\fill (C) circle (1pt) node[above] {$C$};
			\fill (H) circle (1pt) node[below] {$H$};
			\fill [gray!60,opacity=0.5] (C)--(A)--(B)--cycle;
			\draw (A)++(-1,0)--(A) -- (C) -- (B)--($ (B) + (1,0) $) (H) -- (C);
			\draw [dashed] (A)--(B) ;
			\draw[|<->|] (A)++(0,-0.5)--($ (B) + (0, -0.5) $) ;
			\node[below,yshift=-15pt] at ($(A)!0.5!(B)$) {$762$\,m};
			\draw pic["\tiny $6^\circ$", draw=black, angle eccentricity=1.3, angle radius=1.5cm]{angle=B--A--C};
			\draw pic["\tiny $4^\circ$", double, draw=black, angle eccentricity=1.3, angle radius=2cm]{angle=C--B--A};
			\draw pic[draw=black, angle radius=0.15cm]{right angle=C--H--B};
		\end{tikzpicture}
	\end{center}
\dapso{$6$\,giờ $6$\,phút}
\loigiai{
		\begin{enumerate}
	\item[a)] 
	Xét tam giác $ABC$ có $\widehat{ACB} = 180^\circ - 6^\circ - 4^\circ = 170^\circ$.\\
	Áp dụng định lí sin
	\[\dfrac{AB}{\sin {C}} = \dfrac{AC}{\sin {B}} 
	\Rightarrow AC = \dfrac{AB \cdot \sin {B}}{\sin {C}} 
	= \dfrac{762 \cdot \sin 4^\circ}{\sin 170^\circ} \approx 306\ \text{(m)}.
	\]	
	Xét tam giác vuông $AHC$, ta có
	\[CH = AC \cdot \sin {A} = 306 \cdot \sin 6^\circ \approx 32\ \text{(m)}.
	\]
	\item[b)] Áp dụng định lí sin 
	\[\dfrac{BC}{\sin {A}} = \dfrac{AB}{\sin {C}} 
	\Rightarrow BC = \dfrac{762 \cdot \sin 6^\circ}{\sin 170^\circ} \approx 459\ \text{(m)}.
	\]	
	Ta có $AC \approx 306\ \text{m} = 0{,}306\ \text{km},\quad CB \approx 459\ \text{m} = 0{,}459\ \text{km}$.\\
	Thời gian An đi từ nhà đến trường
	\[t = \dfrac{AC}{4} + \dfrac{CB}{19} 
	\approx \dfrac{0{,}306}{4} + \dfrac{0{,}459}{19} \approx 0{,}1\ \text{giờ} = 6\ \text{phút}.
	\]	
	Vậy bạn An đến trường lúc khoảng $6$\,giờ $6$\,phút.
	\end{enumerate}
}
\end{ex}
\Closesolutionfile{ans}