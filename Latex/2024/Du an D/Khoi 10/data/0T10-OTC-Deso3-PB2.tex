\newpage
\def\thoigian{90}%--Thời gian
\de{Đề số 3}{Chương X. Xác suất}

\ind{PHẦN I.} \inden{Câu trắc nghiệm nhiều phương án lựa chọn. Mỗi câu hỏi học sinh chỉ chọn một phương án.}\\
\setcounter{ex}{0}
\Opensolutionfile{ans}[ans/0D1-Bai1-TN]
\begin{ex}%[0D0N1-2]]%[Dự án dề cương 3 Khối NH24-25-Đợt 4-Võ Thị Thùy Trang]
	Tung một đồng xu cân đối và đồng chất hai lần liên tiếp. Xác định biến cố $A\colon$\lq\lq Hai lần xuất hiện mặt giống nhau\rq\rq.
	\choice
	{\True $A = \{SS; NN\}$}
	{ $A = \{NS; SS\}$}
	{$A = \{NS; SN\}$}
	{$A = \{NN; NS\}$}
	\loigiai{
		Biến cố $A$ xảy ra khi cả hai lần tung đều là mặt sấp $SS$ hoặc cả hai lần đều là mặt ngửa $NN$.
	}
\end{ex}
\begin{ex}%[0D0N1-2]]%[Dự án dề cương 3 Khối NH24-25-Đợt 4-Võ Thị Thùy Trang]
	Gieo một đồng xu cân đối và đồng chất $3$ lần liên tiếp. Phép thử ngẫu nhiên này có không gian mẫu là	
	\choice
	{\True $\{NNN, SSS, NNS, SSN, NSN, SNS, NSS, SNN\}$}
	{$\{NN, NS, SN, SS\}$}
	{$\{NNN, SSS, NNS, SSN, NSN, SNS\}$}
	{$\{NNN, SSS, NNS, SSN, NSS, SNN\}$}
	\loigiai{
		Không gian mẫu của phép thử là $\Omega=\{NNN, SSS, NNS, SSN, NSN, SNS, NSS, SNN\}$.
	}
\end{ex}
\begin{ex}%[0D0N1-3]%[Dự án dề cương 3 Khối NH24-25-Đợt 4-Võ Thị Thùy Trang]
	Gieo ngẫu nhiên một con xúc xắc $6$ mặt cân đối và đồng chất liên tiếp $3$ lần thì số phần tử của không gian mẫu là
	\choice 
	{\True $216$}
	{$18$}
	{$6$}
	{$36$}
	\loigiai{
		Gieo ngẫu nhiên một con xúc xắc $6$ mặt cân đối và đồng chất liên tiếp $3$ lần thì số phần tử của không gian mẫu là $n(\Omega)=6^3=216$.
	}
\end{ex}
\begin{ex}%[0D0N1-3]%[Dự án dề cương 3 Khối NH24-25-Đợt 4-Võ Thị Thùy Trang]
	Gieo một con xúc xắc cân đối và đồng chất hai lần. Có bao nhiêu kết quả thuận lợi cho biến cố $A\colon$\lq\lq Số chấm xuất hiện ở hai lần bằng nhau\rq \rq?
	\choice
	{$5$}
	{$3$}
	{\True $6$}
	{$4$}
	\loigiai{Có $6$ kết quả thuận lợi cho biến cố $A\colon$\lq\lq Số chấm xuất hiện ở hai lần bằng nhau\rq \rq. Vì có $6$ cặp kết quả $(1;1)$, $(2;2)$, $(3;3)$, $(4;4)$, $(5;5)$, $(6;6)$.
	}
\end{ex}
\begin{ex}%[0D0N1-3]%[Dự án dề cương 3 Khối NH24-25-Đợt 4-Võ Thị Thùy Trang]
	Một hộp đựng $40$ thẻ được đánh số từ $1$ đến $40$. Rút ngẫu nhiên cùng lúc hai thẻ. Số phần tử của không gian mẫu là
	\choice
	{$1\,560$}
	{$40$}
	{$2$}
	{\True $780$}
	\loigiai{ Số phần tử của không gian mẫu là $n(\Omega)=\mathrm{C}_{40}^2=780$.
	}
\end{ex}
\begin{ex}%[0D0H1-2]%[Dự án dề cương 3 Khối NH24-25-Đợt 4-Võ Thị Thùy Trang]
	Gieo $2$ đồng tiền đồng chất một lần. Quan tâm đến tính Sấp, Ngửa của nó. Xác định biến cố $M \colon$\lq\lq Hai đồng tiền xuất hiện các mặt không giống nhau\rq\rq.
	\choice
	{$\mathrm{M}=\{SS, NS\}$}
	{$\mathrm{M}=\{NN, SS\}$}
	{$\mathrm{M}=\{NS, NN\}$}
	{\True $\mathrm{M}=\{NS, SN\}$}
	\loigiai{
		Biến cố $M \colon$\lq\lq Hai đồng tiền xuất hiện các mặt không giống nhau\rq\rq\,có các kết quả $M=\{NS, SN\}$.
	}
\end{ex}
\begin{ex}%[0D0N2-1]%[Dự án dề cương 3 Khối NH24-25-Đợt 4-Võ Thị Thùy Trang]
	Giả sử một phép thử có không gian mẫu $\Omega$ gồm hữu hạn các kết quả có cùng khả năng xảy ra và $A$ là một biến cố. Mệnh đề nào sau đây là đúng?
	\choice
	{$\mathrm{P}(A) = \dfrac{n(\Omega)}{n(A)}$}
	{\True $0 \leq \mathrm{P}(A) \leq 1$}
	{$\mathrm{P}(\Omega) = 0$}
	{$\mathrm{P}(A)>1$}
	\loigiai{Ta thấy $0 \leq \mathrm{P}(A) \leq 1$ là mệnh đề đúng.
	}
\end{ex}
\begin{ex}%[0D0N2-2]
	Gieo ngẫu nhiên một con súc sắc đồng chất. Xác suất để mặt $3$ chấm xuất hiện là
	\choice
	{$\dfrac{1}{2}$}
	{\True $\dfrac{1}{6}$}
	{$\dfrac{5}{6}$}
	{$\dfrac{1}{3}$}
	\loigiai{
		Số phần tử của không gian mẫu là $n(\Omega)=6$.\\
		Số phần tử của biến cố \lq\lq Mặt $3$ chấm xuất hiện\rq\rq \, là $1$.\\
		Vậy xác suất của biến cố \lq\lq Mặt $3$ chấm xuất hiện\rq\rq \, là $P= \dfrac{1}{6}$.
	}
\end{ex}
\begin{ex}%[0D0N2-2]%[Dự án dề cương 3 Khối NH24-25-Đợt 4-Võ Thị Thùy Trang]
	Gieo một con xúc xắc cân đối và đồng chất một lần. Xác suất để xuất hiện mặt hai chấm là	
	\choice
	{$\dfrac{1}{3}$}
	{$\dfrac{1}{2}$}
	{$\dfrac{1}{4}$}
	{\True $\dfrac{1}{6}$}
	\loigiai{
		Gọi $A$ là biến cố \lq\lq Xuất hiện mặt hai chấm\rq\rq.\\
		Số phần tử của không gian mẫu $n(\Omega)=6$.\\
		Xác suất để xuất hiện mặt hai chấm là $\mathrm{P}(A)=\dfrac{n(A)}{n(\Omega)}=\dfrac{1}{6}$.
	}
\end{ex}
\begin{ex}%[0D0H2-2]%[Dự án dề cương 3 Khối NH24-25-Đợt 4-Võ Thị Thùy Trang]
	Gieo một đồng tiền liên tiếp $3$ lần. Tính xác suất biến cố $A$: \lq\lq Mặt sấp xuất hiện đúng $2$ lần\rq\rq.
	\choice
	{$\dfrac{1}{4}$}
	{\True $\dfrac{3}{8}$}
	{$\dfrac{1}{2}$}
	{$\dfrac{7}{8}$}
	\loigiai{
		Số phần tử của không gian mẫu là $n(\Omega)=2 \cdot 2 \cdot 2=8$.\\
		Số phần tử của biến cố $A$ là $n(A)=3$.\\
		Vậy xác suất của biến cố $A$ là $P(A) = \dfrac{3}{8}$.
	}
\end{ex}

\begin{ex}%[0D0H2-5]%[Dự án dề cương 3 Khối NH24-25-Đợt 4-Võ Thị Thùy Trang]
	Từ một hộp chứa ba quả cầu trắng và hai quả cầu đen lấy ngẫu nhiên hai quả. Xác suất để lấy được cả hai quả trắng là
	\choice 
	{\True $\dfrac{3}{10}$}
	{$\dfrac{1}{3}$}
	{$\dfrac{2}{5}$}
	{$\dfrac{1}{5}$}
	\loigiai{
		Số phần tử không gian mẫu $\mathrm{C}^2_5=10$.\\
		Số cách lấy được cả $2$ quả cầu trắng là $\mathrm{C}^2_3=3$.\\
		Xác suất để lấy được cả hai quả trắng là $\dfrac{3}{10}$.
	}
\end{ex}
\begin{ex}%[0D0H2-2]%[Dự án dề cương 3 Khối NH24-25-Đợt 4-Võ Thị Thùy Trang]
	Gieo một con xúc xắc cân đối và đồng chất hai lần. Xác suất để ít nhất một lần xuất hiện mặt một chấm là
	\choice
	{$\dfrac{8}{36}$}
	{$\dfrac{12}{36}$}
	{\True $\dfrac{11}{36}$}
	{$\dfrac{6}{36}$}
	\loigiai{
		Gọi $A$ là biến cố ít nhất một lần xuất hiện mặt một chấm.\\
		Suy ra $A=\{(1,1); (1,2); (1,3); (1,4); (1,5); (1,6); (2,1); (3,1); (4,1); (5,1); (6,1)\}$.\\
		Do đó $n(A)=11$.\\
		Mặt khác $n(\Omega)=6\cdot 6=36$ nên xác suất của biến cố $A$ là $\mathrm{P}(A)=\dfrac{11}{36}$.
	}
\end{ex}
\Closesolutionfile{ans}
\ind{PHẦN II.} \inden{Câu trắc nghiệm đúng sai. Trong mỗi ý a), b), c), d) ở mỗi câu, học sinh chọn đúng hoặc sai.}\\
\setcounter{ex}{0}
\Opensolutionfile{ans}[ans/0D1-Bai1-DS]%--Đặt tên 2D1-Bai1-DS
\begin{ex}%[0D0H2-4]%[Dự án dề cương 3 Khối NH24-25-Đợt 4-Võ Thị Thùy Trang]
	Lớp $10$A có $25$ học sinh nam và $20$ học sinh nữ.
	\choiceTF
	{\True Có $45$ cách chọn $1$ bạn học sinh để trực nhật}
	{Cô giáo muốn lập một nhóm gồm $1$ bạn học sinh nam và $1$ bạn học sinh nữ để tham gia cuộc thi hùng biện Tiếng Anh. Số cách lập nhóm là $450$ cách}
	{Xác suất để lập một nhóm học tập gồm $6$ học sinh, trong đó có đúng $3$ học sinh nữ là $0{,}23$ (làm tròn đến chữ số thập phân thứ hai)}
	{\True Xác suất để lập một đội văn nghệ gồm $6$ học sinh, trong đó có ít nhất hai học sinh nữ là $0{,}85$ (làm tròn đến chữ số thập phân thứ hai)}
	\loigiai{
		\begin{itemchoice}
			\itemch \textbf{Đúng.}\\ Theo quy tắc cộng, có $20+25=45$ cách chọn $1$ bạn học sinh để trực nhật.
			\itemch \textbf{Sai.} \\Số cách chọn $1$ bạn nam là $25$, số cách chọn $1$ bạn nữ là $20$. Do đó theo quy tắc nhân, số cách chọn ra một bạn nam và một bạn nữ là $25\cdot 20=500$.
			\itemch \textbf{Sai.} \\Số phần tử của không gian mẫu là $n(\Omega)=\mathrm{C}_{45}^6$.\\
			Gọi $A$ là biến cố \lq\lq lập nhóm gồm $6$ học sinh trong đó có đúng $3$ học sinh nữ\rq\rq.\\
			Ta có $n(A)=\mathrm{C}_{25}^3\cdot \mathrm{C}_{20}^3$.\\
			$\mathrm{P}(A)=\dfrac{\mathrm{C}_{25}^3\cdot \mathrm{C}_{20}^3}{\mathrm{C}_{45}^6} \approx 0{,}32$.
			\itemch \textbf{Đúng.}\\ Số phần tử của không gian mẫu là $n(\Omega)=\mathrm{C}_{45}^6$.\\
			Gọi $B$ là biến cố \lq\lq lập nhóm gồm $6$ học sinh trong đó có ít nhất $2$ học sinh nữ\rq\rq.\\
			Khi đó $\overline{B}$ là biến cố \lq\lq lập nhóm gồm $6$ học sinh trong đó nhiều nhất $1$ học sinh nữ\rq\rq.\\
			Ta có $n(\overline{B})= \mathrm{C}_{25}^{6}+\mathrm{C}_{20}^{1}\cdot \mathrm{C}_{25}^{5}$.\\
			Vậy $\mathrm{P}(B)=1-\mathrm{P}(\overline{B})=1-\dfrac{n(\overline{B})}{n(\Omega)}=1-\dfrac{\mathrm{C}_{25}^{6}+\mathrm{C}_{20}^{1}\cdot \mathrm{C}_{25}^{5}}{\mathrm{C}_{45}^6}\approx 0{,}85$.
		\end{itemchoice}
	}
\end{ex}
\begin{ex}%[0D0V2-4]%[Dự án dề cương 3 Khối NH24-25-Đợt 4-Võ Thị Thùy Trang]
	Một lớp $10$ có $35$ học sinh trong đó có $25$ học sinh học giỏi Toán, $20$ học sinh học giỏi Lý, $15$ học sinh học giỏi cả hai môn Toán và Lý. Chọn ngẫu nhiên $1$ học sinh của lớp, gọi $A$ là biến cố học sinh không học giỏi Toán và không học giỏi Lý.
	\choiceTF
	{Số học sinh chỉ học giỏi toán là $15$ học sinh}
	{\True Số học sinh chỉ học giỏi Lý là $5$ học sinh}
	{Số cách chọn ngẫu nhiên một học sinh là $30$}
	{\True Xác suất của biến cố $A$ là $\mathrm{P}(A)=\dfrac{1}{7}$}
	\loigiai{
		\begin{itemchoice}
			\itemch \textbf{Sai.}\\Số học sinh chỉ giỏi Toán là $25-15=10$ (học sinh).
			\itemch \textbf{Đúng.}\\Số học sinh chỉ giỏi Lý là $20-15=5$ (học sinh).
			\itemch \textbf{Sai.}\\Lớp học có $35$ học sinh. Vì lớp có $35$ học sinh nên số cách chọn ngẫu nhiên một học sinh là $35$.
			\itemch \textbf{Đúng.}\\Số học sinh học giỏi ít nhất một môn là $25+20-15=30$.\\
			Vậy số học sinh không học giỏi Toán và Lý là $35-30=5$ học sinh.\\
			Suy ra $n(A)=5$.\\
			Mà $n(\Omega)=35$.\\
			Khi đó xác suất của biến cố $A$ là $\mathrm{P}(A)=\dfrac{5}{35}=\dfrac{1}{7}$.
		\end{itemchoice}
	}
\end{ex}
\Closesolutionfile{ans}
\ind{PHẦN III.} \inden{Câu trắc nghiệm trả lời ngắn}\\
\setcounter{ex}{0}
\Opensolutionfile{ans}[ans/0D1-Bai1-TLN]
\begin{ex}%[0D0H1-3]%[Dự án dề cương 3 Khối NH24-25-Đợt 4-Võ Thị Thùy Trang]
	Trong một chiếc hộp đựng $6$ viên bi đỏ, $8$ viên bi xanh, $10$ viên bi trắng. Lấy ngẫu nhiên $3$ viên bi. Tính số phần tử của biến cố $B\colon$\lq\lq $3$ viên bi lấy ra có đúng $1$ bi màu đỏ\rq\rq ?
	\shortans{$918$}
	\loigiai{
		Ta có $n(B)=\mathrm{C}_{18}^2\cdot \mathrm{C}_{6}^1=918$.
	}
\end{ex}
\begin{ex}%[0D0V2-2]%[Dự án dề cương 3 Khối NH24-25-Đợt 4-Võ Thị Thùy Trang]
	Gieo ngẫu nhiên một đồng xu cân đối và đồng chất $3$ lần. Tính xác suất của biến cố \lq\lq Có không quá $2$ lần xuất hiện mặt sấp\rq\rq \,là bao nhiêu? ({\itshape{làm tròn kết quả đến chữ số thập phân thứ hai}}).
	\shortans[oly]{$0{,}88$}
	\loigiai{
		$n(\Omega)=2^3=8$.\\
		Gọi $A$ là biến cố \lq\lq Có không quá $2$ lần xuất hiện mặt sấp\rq\rq.\\
		Khi đó $A=\left\{ NNN;NNS;NSN;SNN;SSN;SNS;NSS\right\} \Rightarrow n(A)=7$.\\
		Vậy xác suất của biến cố $A$ là
		$\mathrm{P}(A)=\dfrac{n(A)}{n(\Omega)}=\dfrac{7}{8}=0{,}875\approx 0{,}88$.}
\end{ex}
\begin{ex}%[0D0V2-2]%[Dự án dề cương 3 Khối NH24-25-Đợt 4-Võ Thị Thùy Trang]
	Gieo bốn con xúc xắc cân đối và đồng chất. Tính xác suất để tích số chấm xuất hiện trên bốn con xúc xắc chia hết cho $6$ (làm tròn kết quả đến hàng phần trăm).
	\par
	\shortans[oly]{$0{,}75$}
	\loigiai{
		Không gian mẫu $n(\Omega)=6^4=1\,296$.\\
		Gọi $A$ là biến cố \lq\lq Tích số chấm xuất hiện trên bốn con xúc xắc chia hết cho $6$\rq\rq.\\
		Suy ra $\overline{A}$ là biến cố \lq\lq Tích số chấm xuất hiện trên bốn con xúc xắc không chia hết cho $6$\rq\rq.\\
		Khi đó, ta cần đếm các kết quả sao cho tích số chấm xuất hiện trên bốn con xúc xắc không chia hết cho $2$ hoặc không chia hết cho $3$.\\
		Gọi $C$ là tập hợp các kết quả không chia hết cho $2$.\\
		Ta chọn từ các số chấm lẻ $1$, $3$, $5$.\\
		Mỗi lần gieo xúc xắc có $3$ lựa chọn nên $n(C)=3^4=81$. \\
		Gọi $D$ là tập hợp các kết quả không chia hết cho $3$.\\
		Ta chọn từ các số chấm $1$, $2$, $4$, $5$.\\
		Mỗi lần gieo xúc xắc có $4$ lựa chọn nên $n(D)=4^4=256$.\\
		Ta có $C\cap D$ là tập hợp các kết quả không chia hết cho cả $2$ và $3$.\\
		Ta chọn từ các số chấm $1$, $5$.\\
		Mỗi lần gieo xúc xắc có $2$ lựa chọn nên $n(C\cap D)=2^4=16$.\\
		Suy ra số kết quả không chia hết cho $6$ là \[n\left(\overline{A}\right)=n(C\cup D)=n(C)+n(D)-n(C\cap D)=321.\]
		Xác suất để tích số chấm xuất hiện trên bốn con xúc xắc chia hết cho $6$ là \[\mathrm{P}(A)=1-\mathrm{P}\left(\overline{A}\right)=1-\dfrac{321}{1\,296}=\dfrac{975}{1\,296}\approx 0{,}75.\]
	}
\end{ex}
\begin{ex}%[0D0C2-5]
	Trong một dịp quay xổ số, có $3$ loại giải thưởng: $1\,000\,000$ đồng, $500\,000$ đồng, $100\,000$ đồng. Nơi bán có $100$ tờ vé số, trong đó có $1$ vé trúng thưởng $1\,000\,000$ đồng, $5$ vé trúng thưởng $500\,000$ đồng, $10$ vé trúng thưởng $100\,000$ đồng. Một người mua ngẫu nhiên $3$ vé. Tính xác suất của biến cố \lq\lq Người mua đó trúng thưởng ít nhất $300\,000$ đồng\rq\rq\, (kết quả làm tròn đến hàng phần trăm).
	\shortans[0]{$0{,}17$}
	\loigiai{
		Số cách chọn mua $3$ vé là $\mathrm{C}_{100}^3=161\,700$.\\
		Gọi $A$ là biến cố: \lq\lq Người mua đó trúng thưởng ít nhất $300\,000$ đồng\rq\rq\; thì \\ biến cố đối của $A$ là $\overline{A}$: \lq\lq Người mua đó trúng thưởng nhiều nhất $200\,000$ đồng\rq\rq.\\
\textbf{\textbf{Tìm $n(\overline{A})$}}
		\begin{itemize}
			\item \textbf{Phương án 1}: Không trúng thưởng. Có $\mathrm{C}_{84}^3=95284$ cách .\qquad (1)
			\item \textbf{Phương án 2}:Trúng thưởng $100\,000$ đồng. Có $\mathrm{C}_{84}^2 \cdot \mathrm{C}_{10}^1=34860$ cách.\qquad (2)
			\item \textbf{Phương án 3}:Trúng thưởng $200\,000$ đồng. Có $\mathrm{C}_{84}^1 \cdot \mathrm{C}_{10}^2=3780$ cách.\qquad (3)
			\end{itemize}
		Từ $(1)$, $(2)$ và $(3)$ theo quy tắc cộng ta có				$n(\overline{A})=95284+34860+3780=133924$.\\
				Suy ra xác suất của biến cố $\bar{A}$ là $P(\overline{A})=\dfrac{95284+34860+3780}{161700}=\dfrac{4783}{5775}$.\\
		Vậy xác suất của biến cố $A$ là $\mathrm{P}(A)=1-P(\overline{A})=1-\dfrac{4783}{5775}=\dfrac{992}{5775}\approx 0{,}17$.}
\end{ex}

\Closesolutionfile{ans}
\ind{PHẦN IV.} \inden{Tự luận.}
\setcounter{ex}{0}
\begin{ex}%[0D0H2-5]%[Dự án đề kiểm tra toán khối 10 HKII NH24-25 - Đợt 12 - LamNguyen]%[Đồng Nai]
	Một hộp có chứa $8$ quả cầu xanh và $7$ quả cầu đỏ. Lấy ngẫu nhiên $6$ quả cầu từ hộp.
	\begin{enumerate}[a)]
		\item Xác định số phần tử không gian mẫu.
		\item Tính xác suất của biến cố \lq\lq $6$ quả cầu lấy ra có nhiều nhất $4$ quả màu xanh\rq\rq.
	\end{enumerate}
	\loigiai{
		\begin{enumerate}[a)]
			\item Số phần tử không gian mẫu $ n(\Omega)=\mathrm{C}_{15}^6=5\,005$.
			\item Gọi $A$ là biến cố \lq\lq $6$ quả cầu lấy ra có nhiều nhất $4$ quả màu xanh\rq\rq\\
			$\Rightarrow\overline{A}\colon$\lq\lq $6$ quả cầu lấy ra có ít nhất $5$ quả màu xanh\rq\rq.
			\begin{itemize}
				\item Lấy được $5$ quả cầu xanh và $1$ quả cầu  đỏ có $\mathrm{C}_8^5 \cdot \mathrm{C}_7^1=392$ (cách chọn).
				\item Lấy được $6$ quả cầu xanh có $\mathrm{C}_8^6=28$ (cách chọn).\\
				Suy ra $n\left(\overline{A}\right)=392+28=420$.\\
				$\Rightarrow n(A)=n\left(\Omega\right)-n\left(\overline{A}\right)=5\,005-420=4\,585$.\\
				$\Rightarrow \mathrm{P}(A)=\dfrac{n(A)}{n(\Omega)}=\dfrac{4\,585}{5\,005}=\dfrac{131}{143}\approx 0{,}92$.
			\end{itemize}
	\end{enumerate}}
\end{ex}
\begin{ex}%[0D0V2-5]%[Dự án dề cương 3 Khối NH24-25-Đợt 4-Võ Thị Thùy Trang]
	Trong một cuộc thi giải toán động đội, nhóm $A$ gồm $3$ bạn Dung, Nam, Quỳnh xuất sắc vượt qua các nhóm còn lại để giành giải vô địch. Sau khi nhận cúp, nhóm $A$ được tặng thêm một phần quà bất ngờ đến từ nhà tài trợ. Một chiếc hộp với $20$ tấm thẻ được đánh số từ $1$ đến $20$ (các tấm thẻ có kích thước và khối lượng bằng nhau). Mỗi bạn Dung, Nam, Quỳnh lần lượt bốc $1$ tấm thẻ trong hộp. Biết rằng
	\begin{itemize}
		\item Phần quà $1$: Nếu cả ba bạn đều bốc được tấm thẻ cùng đánh số chẵn hoặc cùng đánh số lẻ thì mỗi bạn sẽ nhận được phần thưởng là một chiếc TV $32$ inch.
		\item Phần quà $2$: Nếu cả ba bạn bốc được ba tấm thẻ mà tổng các số ghi trên ba tấm thẻ chia hết cho $3$ thì mỗi bạn sẽ nhận được phần thưởng là một chiếc xe máy.
		\item Các trường hợp còn lại không được nhận thưởng.
	\end{itemize}
	\begin{enumerate}[a)]
		\item Tính xác suất để các bạn nhóm A nhận được chiếc TV.
		\item Tính xác suất để các bạn nhóm A nhận được chiếc xe máy.
	\end{enumerate}
	\loigiai{
		\begin{enumerate}[a)]
			\item Số phần tử không gian mẫu $ n(\Omega)=\mathrm{A}_{20}^3=6\,840$.\\
			Gọi $A$ là biến cố \lq\lq Cả ba bạn đều bốc được tấm thẻ đánh số chẵn\rq\rq.\\
			Trong $20$ quả cầu được đánh số từ $1$ đến $20$ có $10$ quả cầu được đánh số chẵn và $10$ quả cầu được đánh số lẻ.\\
			$ n(A)=\mathrm{A}_{10}^3\cdot 2=1\,440
			\Rightarrow \mathrm{P}(A)=\dfrac{n(A)}{n(\Omega)}=\dfrac{1\,440}{6\,840}=\dfrac{4}{19}\approx 0{,}21$.\\
			Vậy xác suất để các bạn nhóm $A$ nhận được chiếc TV là $0{,}21$.
			\item Gọi $B$ là biến cố \lq\lq Cả ba bạn bốc được ba tấm thẻ mà tổng các số ghi trên ba tấm thẻ chia hết cho $3$\rq\rq.\\
			Trong $20$ quả cầu được đánh số từ $1$ đến $20$ có
			\begin{itemize}
			\item $6$ quả cầu được đánh số chia hết cho $3$ là $3$; $6$; $9$; $12$; $15$; $18$.
			\item $7$ quả cầu được đánh số chia $3$ dư $1$ là $1$; $4$; $7$; $10$; $13$; $16$; $19$.
			\item $7$ quả cầu được đánh số chia $3$ dư $2$ là $2$; $5$; $8$; $11$; $14$; $17$; $20$.
			\end{itemize}
			Ta xét các trường hợp sau
			\begin{itemize}
				\item \textbf{TH1}: $3$ thẻ lấy ra đều được đánh số chia hết cho $3$ có $ \mathrm{A}_6^3=120$ (cách lấy).
				\item \textbf{TH2}: $3$ thẻ lấy ra đều được đánh số chia $3$ dư $1$ có $ \mathrm{A}_7^3=210$ (cách lấy).
				\item \textbf{TH3}: $3$ thẻ lấy ra đều được đánh số chia $3$ dư $2$ có $\mathrm{A}_7^3=210$ (cách lấy).
				\item \textbf{TH4}: $3$ thẻ lấy ra có $1$ thẻ được đánh số chia hết cho $3$, $1$ thẻ được đánh số chia $3$ dư $1$ và $1$ thẻ được đánh số chia $3$ dư $2$ có $6\cdot7\cdot7=294$ (cách lấy).
			\end{itemize}
			Suy ra $n(B)=120+210+210+294=834
			\Rightarrow \mathrm{P}(B)=\dfrac{n(B)}{n(\Omega)}=\dfrac{834}{6\,840}=\dfrac{139}{1\,140}\approx 0{,}12$.\\
			Vậy xác suất để các bạn nhóm $A$ nhận được chiếc xe máy là $0{,}12$.
	\end{enumerate}}
\end{ex}	
\begin{ex}%[0D0V2-4]%[Dự án dề cương 3 Khối NH24-25-Đợt 4-Võ Thị Thùy Trang]
	Một lớp có $15$ học sinh nam và $20$ học sinh nữ. Chọn ngẫu nhiên $5$ học sinh tham gia lao động. Tính xác suất sao cho
	\begin{enumerate}
		\item Chọn $5$ học sinh có đúng $3$ nam và $2$ nữ.
		\item Chọn $5$ học sinh sao cho có ít nhất $1$ nam.
	\end{enumerate}
	\loigiai{
		\begin{enumerate}
			\item 	$n(\Omega)=\mathrm{C}_{35}^5=324\,632$. \\
			Gọi $A$ là biến cố chọn $5$ học sinh có đúng $3$ nam và $2$ nữ.\\
			$n(A)=\mathrm{C}_{15}^3\cdot \mathrm{C}_{20}^2=86\,450$.\\
			Xác suất của biến cố $A$ là  $\mathrm{P}(A)=\dfrac{n(A)}{n(\Omega)}=\dfrac{86\,450}{324\,632} \approx 0{,}266$.
			\item Gọi $B$ là biến cố chọn $5$ học sinh sao cho có ít nhất $1$ nam. \\
			Khi đó $\overline{B}$ là biến cố chọn được $5$ học sinh nữ nên $n(\overline{B})=\mathrm{C}_{20}^5=15\,504$.\\
			$\mathrm{P}(B)=1-\mathrm{P}(\overline{B})=1-\dfrac{\mathrm{C}_{20}^5}{\mathrm{C}_{35}^5} \approx 0{,}95$.
		\end{enumerate}
	}
\end{ex}