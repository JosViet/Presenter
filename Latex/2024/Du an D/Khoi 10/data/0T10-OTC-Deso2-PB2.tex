\newpage
\def\thoigian{90}%--Thời gian
\de{Đề số 2}{Chương X. Xác suất}


\ind{PHẦN I.} \inden{Câu trắc nghiệm nhiều phương án lựa chọn. Mỗi câu hỏi học sinh chỉ chọn một phương án.}\\
\setcounter{ex}{0}
\Opensolutionfile{ans}[ans/0T10-OTC-Deso2-TN]

%Câu 1
\begin{ex}%[0D0H1-3]%[Dự án D - đợt 2 NH24-25- Thy Nguyen Vo Diem]
	Xét phép thử \lq\lq Gieo một xúc xắc hai lần liên tiếp\rq\rq. Biến cố nào dưới đây là biến cố không?
	\choice
	{\True Tổng số chấm ở hai lần gieo nhỏ hơn hoặc bằng $ 1 $}
	{Cả hai lần gieo đều xuất hiện số chấm lẻ}
	{Số chấm xuất hiện ở hai lần gieo đều chia hết cho $ 5  $}
	{Số chấm ở lần gieo thứ nhất nhỏ hơn số chấm ở lần gieo thứ hai}
	\loigiai{
		Biến cố \lq\lq Tổng số chấm ở hai lần gieo nhỏ hơn hoặc bằng $ 1  $\rq\rq~ là biến cố không.
	}
\end{ex}
%Câu 2
\begin{ex}%[0D0H1-3]%[Dự án D - đợt 2 NH24-25- Thy Nguyen Vo Diem]
	Xét phép thử \lq\lq Tung một đồng xu hai lần liên tiếp\rq\rq. Biến cố nào dưới đây là biến cố chắc chắn?
	\choice
	{Mặt sấp chỉ xuất hiện 1 lần}
	{Lần thứ hai xuất hiện mặt ngửa}
	{\True Lần thứ nhất xuất hiện mặt sấp hoặc mặt ngửa}
	{Cả hai lần tung đều xuất hiện mặt sấp}
	\loigiai{
		Biến cố \lq\lq Lần thứ nhất xuất hiện mặt sấp hoặc mặt ngửa\rq\rq ~là biến cố chắc chắn.
	}
\end{ex}
%Câu 3
\begin{ex}%[0D0H2-7]%[Dự án D - đợt 2 NH24-25- Thy Nguyen Vo Diem]
	Cho tập hợp $A$ gồm $2022$ số nguyên dương liên tiếp $1,2,3, \ldots, 2022$. Chọn ngẫu nhiên $ 2 $ số thuộc tập hợp $A$. Xác suất của biến cố \lq\lq Tích $2$ số được chọn là số chẵn\rq\rq~ là 
	\choice
	{$\dfrac{\mathrm{C}_{1011}^2}{\mathrm{C}_{2022}^2}$}
	{\True $1-\dfrac{\mathrm{C}_{1011}^2}{\mathrm{C}_{2022}^2}$}
	{$\dfrac{1}{2}$}
	{$1-\dfrac{\mathrm{C}_{2022}^2}{\mathrm{C}_{4044}^2}$}
	\loigiai{
		Số phần tử của không gian mẫu là $ n(\Omega) =\mathrm{C}^2_{2022}$.\\
		Trong $ 2022 $ số đã cho có $ 1011 $ số chẵn và $ 1011 $ số lẻ.\\
		Tích của hai số là số lẻ khi chỉ khi hai số được chọn đều lẻ.\\
		Xác suất lấy hai số mà tích hai số được chọn là lẻ $ \dfrac{\mathrm{C}^2_{1011}}{\mathrm{C}^2_{2022}} $.\\
			
			Tích của hai số được chọn là số chẵn có hai trường hợp
			\begin{enumerate}
					\item[TH1:] Cả hai số được chọn là số chẵn khi đó có $ \mathrm{C}^2_{1011}$.
					\item[TH2:] Trong hai số được chọn có một số chẵn và một số lẻ khi đó có  $ \mathrm{C}^1_{1011}\cdot \mathrm{C}^1_{1011}$. 
				\end{enumerate}
		Vậy xác suất cần tìm là $ \mathrm{P} = 1-\dfrac{\mathrm{C}_{1011}^2}{\mathrm{C}_{2022}^2} $.
	}
\end{ex}

%Câu 4
\begin{ex}%[0D0H1-2] %[Dự án D - đợt 2 NH24-25- Thy Nguyen Vo Diem]
	Một hộp có bốn loại bi: bi xanh, bi đỏ, bi trắng và bi vàng. Lấy ngẫu nhiên ra một viên bi. Gọi $E$ là biến cố  \lq\lq Lấy được viên bi đỏ\rq\rq. Biến cố đối của $E$ là biến cố 
	\choice
	{Lấy được viên bi xanh}
	{Lấy được viên bi vàng hoặc bi trắng}
	{Lấy được viên bi trắng}
	{\True Lấy được viên bi vàng hoặc bi trắng hoặc bi xanh}
	\loigiai{
		Biến cố đối của $E$ là biến cố \lq\lq Lấy được viên bi vàng hoặc bi trắng hoặc bi xanh\rq\rq.
	}
\end{ex}
%Câu 5
\begin{ex}%[0D0H2-7]%[Dự án D - đợt 2 NH24-25- Thy Nguyen Vo Diem]
	Rút ngẫu nhiên ra một thẻ từ một hộp có $ 30 $ tấm thẻ được đánh số từ $ 1 $ đến $ 30 $. Xác suất để số  trên tấm thẻ được rút ra chia hết cho $ 5 $ là
	\choice
	{$ \dfrac{1}{30} $}
	{\True $ \dfrac{1}{5} $}
	{$ \dfrac{1}{3} $}
	{$ \dfrac{2}{5} $}
	\loigiai{
		Ta có $ n(\Omega)=30 $.\\
		Số chia hết cho $ 5 $ từ $ 1 $ đến $ 30 $ thuộc tập $ A=\{5;10;15;20;25;30\}\Rightarrow n(A)=6 $.\\
		Vậy $ \mathrm{P}(A)=\dfrac{n(A)}{n(\Omega)}=\dfrac{1}{5} $.
	}
\end{ex}
%Câu 6
\begin{ex}%[0D0H2-2]%[Dự án D - đợt 2 NH24-25- Thy Nguyen Vo Diem]
	Gieo hai con xúc xắc cân đối. Xác suất để tổng số chấm xuất hiện trên hai con xúc xắc không lớn hơn $ 4 $ là 
	\choice
	{$ \dfrac{1}{7} $}
	{\True $ \dfrac{1}{6} $}
	{$ \dfrac{1}{8} $}
	{$ \dfrac{2}{9} $}
	\loigiai{
		Ta có $ n(\Omega)=6\cdot 6=36 $.\\	
		Gọi $A$	là biến cố: \lq\lq tổng số chấm xuất hiện trên hai con xúc xắc không lớn hơn $ 4 $\rq\rq.\\
		Ta có $A=\{(1,1); (1,2); (1,3); (2,1); (2,2); (3,1)\}\Rightarrow n(A)=6$.\\
		Vậy $ \mathrm{P}(A)=\dfrac{n(A)}{n(\Omega)}=\dfrac{1}{6} $.
	}
\end{ex}
%Câu 7
\begin{ex}%[0D0H2-4]%[Dự án D - đợt 2 NH24-25- Thy Nguyen Vo Diem]
	Một tổ trong lớp $ 10\mathrm{T} $	có $ 4 $ bạn nữ và $ 3 $ bạn nam. Giáo viên chọn ngẫu nhiên hai bạn trong tổ đó tham gia đội làm báo của lớp. Xác suất để hai bạn được chọn có một bạn nam và một bạn nữ là 
	\choice
	{\True $ \dfrac{4}{7} $}
	{$ \dfrac{2}{7} $}
	{$ \dfrac{1}{6} $}
	{$ \dfrac{2}{21} $}
	\loigiai{
		Ta có $ n(\Omega)=\mathrm{C}_7^2=21 $.\\	
		Gọi $A$	là biến cố: \lq\lq hai bạn được chọn có một bạn nam và một bạn nữ\rq\rq.\\
		Ta có $ n(A)=\mathrm{C}_4^1\cdot \mathrm{C}_3^1=12$.\\
		Vậy $ \mathrm{P}(A)=\dfrac{n(\mathrm{A})}{n(\Omega)}=\dfrac{4}{7} $.	
	}
\end{ex}
%Câu 8
\begin{ex}%[0D0H2-5]%[Dự án D - đợt 2 NH24-25- Thy Nguyen Vo Diem]
	Một hộp có $ 4 $ viên bi xanh, $ 5 $ viên bi đỏ có kích thước và khối lượng như nhau. Lấy ra ngẫu nhiên đồng thời $ 2 $ viên bi. Xác suất của biến cố \lq\lq $ 2 $ viên bi lấy ra đều là bi xanh\rq\rq \,là
	\choice
	{$ \dfrac{1}{2} $}
	{$ \dfrac{1}{3} $}
	{$ \dfrac{1}{5} $}
	{\True $ \dfrac{1}{6} $}
	\loigiai{
		Ta có $ n(\Omega)=\mathrm{C}_9^2=36 $.\\	
		Gọi $A$	là biến cố: \lq\lq $ 2 $ viên bi lấy ra đều là bi xanh\rq\rq.\\
		Ta có $ n(A)=\mathrm{C}_4^2=6$.\\
		Vậy $ \mathrm{P}(A)=\dfrac{n(A)}{n(\Omega)}=\dfrac{1}{6} $.		
	}
\end{ex}
%Câu 9
\begin{ex}%[0D0H2-2]%[Dự án D - đợt 2 NH24-25- Thy Nguyen Vo Diem]
	Gieo hai con xúc xắc cân đối và đồng chất. Xác suất để tích số chấm xuất hiện bằng $ 7 $ là
	\choice
	{\True $ 0 $}
	{$ \dfrac{1}{36} $}
	{$ \dfrac{1}{7} $}
	{$ \dfrac{1}{6} $}
	\loigiai{
		Gọi $A$	là biến cố \lq\lq tích số chấm xuất hiện bằng $ 7 $ \rq\rq.\\
		Nếu tích số chấm xuất hiện bằng $ 7 $ thì phải có một số bằng $ 7 $ mà không có mặt nào của con xúc xắc ghi số  $ 7 $ nên $A$ là biến cố không thể.\\
		Vậy $ \mathrm{P}(A)=0 $.
	}
\end{ex}
%Câu 10
\begin{ex}%[0D0H2-2]%[Dự án D - đợt 2 NH24-25- Thy Nguyen Vo Diem]
	Tung $ 3 $ đồng xu cân đối và đồng chất. Xác suất để có ít nhất một đồng xu xuất hiện mặt sấp là
	\choice
	{$ \dfrac{1}{2} $}
	{\True $ \dfrac{7}{8} $}
	{$ \dfrac{1}{3} $}
	{$ \dfrac{1}{4} $}
	\loigiai{
		Ta có không gian mẫu là $ \Omega=\{SNS; SNN; SSN; SSS; NSN; NSS; NNS; NNN \} \Rightarrow n(\Omega)=8$.\\
		Gọi $A$	là biến cố: \lq\lq có ít nhất một đồng xu xuất hiện mặt sấp\rq\rq.\\
		Ta có 	$A=\{SNS; SNN; SSN; SSS; NSN; NSS; NNS \} \Rightarrow n(A)=7$.\\
		Vậy $ \mathrm{P}(A)=\dfrac{n(A)}{n(\Omega)}=\dfrac{7}{8} $.		
	}
\end{ex}
%Câu 11
\begin{ex}%[0D0N2-5]%[Dự án D - đợt 2 NH24-25- Thy Nguyen Vo Diem]
	Một hộp chứa $ 2 $	loại bi xanh và đỏ. Lấy ra ngẫu nhiên từ hộp  $ 1 $ viên bi. Biết xác suất lấy được bi đỏ là $ 0{,}3 $. Xác suất lấy được bi xanh là
	\choice
	{$ 0{,}3 $}
	{$ 0{,}5 $}
	{\True $ 0{,}7 $}
	{$ 0{,}09 $}
	\loigiai{
		Vì xác suất lấy được bi đỏ là $ 0{,}3 $ nên xác suất lấy được bi xanh là $ 1-0{,}3=0{,}7 $.
	}
\end{ex}
%Câu 12
\begin{ex}%[0D0H2-2]%[Dự án D - đợt 2 NH24-25- Thy Nguyen Vo Diem]
	Gieo một con xúc xắc bốn mặt cân đối và đồng chất ba lần. Xác suất xảy ra biến cố \lq\lq Có ít nhất một lần xuất hiện đỉnh ghi số $ 4 $\rq\rq \,là
	\choice
	{$ \dfrac{1}{4} $}
	{$ \dfrac{27}{64} $}
	{\True $ \dfrac{37}{64} $}
	{$ \dfrac{3}{4} $}
	\loigiai{
		Ta có $ n(\Omega)=4\cdot 4\cdot 4=64 $.\\	
		Gọi $A$	là biến cố: \lq\lq Có ít nhất một lần xuất hiện đỉnh ghi số $ 4 $\rq\rq.\\
		Suy ra $ n\left(\overline{A}\right)=3\cdot 3\cdot 3=27 \Rightarrow n(A)=64-27=37$.\\
		Vậy $ \mathrm{P}(A)=\dfrac{n(A)}{n(\Omega)}=\dfrac{37}{64}$.				
	}
\end{ex}


\Closesolutionfile{ans}
\ind{PHẦN II.} \inden{Câu trắc nghiệm đúng sai. Trong mỗi ý a), b), c), d) ở mỗi câu, học sinh chọn đúng hoặc sai.}\\
\setcounter{ex}{0}
\Opensolutionfile{ans}[ans/0T10-OTC-Deso2-DS]
\begin{ex}%[0D0H2-4]%[Dự án D - đợt 2 NH24-25- Thy Nguyen Vo Diem]
	Trong một ban tổ chức gồm $5$ nhân viên đến từ Việt Nam, $7$ nhân viên đến từ Hoa Kỳ và $6$ nhân viên đến từ Anh.
	\choiceTF
	{\True Có $210$ cách chọn ra $3$ nhân viên, mỗi người một quốc gia khác nhau}
	{\True Có $\mathrm{C}_7^2$ cách chọn ra $2$ nhân viên từ Hoa  Kỳ} 
	{Chọn ngẫu nhiên $2$ nhân viên từ ban tổ chức, xác suất để chọn được $2$ nhân viên từ hai quốc gia khác nhau là $\dfrac{203}{272}$}
	{Chọn ngẫu nhiên $3$ nhân viên từ ban tổ chức, xác suất để chọn được $3$ nhân viên cùng một quốc gia là $\dfrac{35}{816}$}
	\loigiai{\begin{itemchoice}
			\itemch Chọn $3$ nhân viên trong đó mỗi quốc gia có một người có $\mathrm{C}_5^1\cdot\mathrm{C}_7^1\cdot\mathrm{C}_6^1=210$ (cách).
			\itemch Chọn $2$ nhân viên Hoa Kỳ từ $7$ nhân viên có $\mathrm{C}_7^2$ (cách).
			\itemch Số phần tử của không gian mẫu là $n(\Omega)=\mathrm{C}_{18}^2=153$.\\
			Gọi $A$ là biến cố \lq\lq Chọn $2$ nhân viên từ hai quốc gia khác nhau\rq\rq.\\
			Để chọn ra $2$ nhân viên thuận lợi cho biến cố $A$ ta có $3$ khả năng.
			\begin{enumerate}[\it Khả năng 1:]
				\item $2$ nhân viên chọn ra có $1$ nhân viên đến từ Việt Nam và $1$ nhân viên đến từ Hoa Kỳ có $\mathrm{C}_5^1\cdot\mathrm{C}_7^1$ khả năng.
				\item $2$ nhân viên chọn ra có $1$ nhân viên đến từ Việt Nam và $1$ nhân viên đến từ Anh có $\mathrm{C}_5^1\cdot\mathrm{C}_6^1$.
				\item $2$ nhân viên chọn ra có $1$ nhân viên đến từ Hoa Kỳ và $1$ nhân viên đến từ Anh có $\mathrm{C}_6^1\cdot\mathrm{C}_7^1$.
			\end{enumerate}
			Vậy số phần tử của biến cố $A$ là $n(A)=\mathrm{C}_5^1\cdot\mathrm{C}_7^1+\mathrm{C}_5^1\cdot\mathrm{C}_6^1+\mathrm{C}_6^1\cdot\mathrm{C}_7^1=107$.\\
			Xác suất của biến cố $A$ là $\mathrm{P}(A)=\dfrac{n(A)}{n(\Omega)}=\dfrac{107}{153}$.
			\itemch Số phần tử của không gian mẫu là $n(\Omega)=\mathrm{C}_{18}^3=816$.\\
			Gọi $B$ là biến cố \lq\lq Chọn $3$ nhân viên cùng một quốc gia\rq\rq.\\
			Để chọn ra $3$ nhân viên thuận lợi cho biến cố $B$ ta có $3$ khả năng.
			\begin{enumerate}[\it Khả năng 1:]
				\item $3$ nhân viên chọn ra đều đến từ Việt Nam có $\mathrm{C}_5^3$ khả năng.
				\item $3$ nhân viên chọn ra đều đến từ Hoa Kỳ có $\mathrm{C}_7^3$ khả năng.
				\item $3$ nhân viên chọn ra đều đến từ Anh có $\mathrm{C}_6^3$ khả năng.
			\end{enumerate}
			Vậy số phần tử của biến cố $B$ là $n(B)=\mathrm{C}_5^3+\mathrm{C}_7^3+\mathrm{C}_6^3=65$.\\
			Xác suất của biến cố $B$ là $\mathrm{P}(B)=\dfrac{n(B)}{n(\Omega)}=\dfrac{65}{816}$.
		\end{itemchoice}
	}  
\end{ex}
\begin{ex}%[0D0V2-5]%[Dự án D - đợt 2 NH24-25- Thy Nguyen Vo Diem]
	Một hộp đựng bảy thẻ màu xanh đánh số từ $1$ đến $7$; năm thẻ màu đỏ đánh số từ $1$ đến $5$ và hai thẻ màu vàng đánh số từ $1$ đến $2$. Rút ngẫu nhiên ra một tấm thẻ. 
	\choiceTF
	{\True Số phần tử không gian mẫu là $14$}
	{Xác suất để thẻ được rút ra được thẻ màu đỏ hoặc màu vàng bằng $\dfrac{1}{14}$}
	{\True Xác xuất để thẻ được rút ra đánh số chia hết cho $3$ là $\dfrac{3}{14}$}    
	{Xác suất để thẻ được rút ra mang số 1 là $\dfrac{5}{14}$}
	\loigiai{
		\begin{itemchoice}
			\itemch Hộp có $14$ thẻ nên có $14$ cách lấy. Suy ra số phần tử của không gian mẫu là $14$.
			\itemch Có tổng cộng $7$ thẻ màu đỏ và vàng nên xác suất để rút được thẻ màu vàng là $\dfrac{7}{14}=\dfrac{1}{2}$.
			\itemch Thẻ màu xanh chia hết cho $3$ có $2$ thẻ, màu đỏ chia hết cho $3$ có $1$ thẻ và không có thẻ vàng chia hết cho $3$. Suy ra, xác xuất để thẻ được rút ra đánh số chia hết cho $3$ là $\dfrac{3}{14}$.
			\itemch Có ba thẻ mang số $1$ nên xác suất để thẻ được rút ra mang số 1 là $\dfrac{3}{14}$.
		\end{itemchoice}
	}
\end{ex}
\Closesolutionfile{ans}

\ind{PHẦN III.} \inden{Câu trắc nghiệm trả lời ngắn}\\
\setcounter{ex}{0}
\Opensolutionfile{ans}[ans/0T10-OTC-Deso2-TLN]
\begin{ex}%[0D0H1-3]%[Dự án D - đợt 2 NH24-25- Thy Nguyen Vo Diem]
	Cho hai đường thẳng song song $a$ và $b$. Trên đường thẳng $a$ lấy $6$ điểm phân biệt, trên đường thẳng $b$ lấy $5$ điểm phân biệt. Chọn ngẫu nhiên ba điểm trong các điểm đã cho trên hai đường thẳng $a$ và $b$. Số phần tử không gian mẫu bằng
	\par\shortans[]{165}
	\loigiai{
		Chọn ngẫu nhiên ba điểm trong 11 điểm đã cho trên hai đường thẳng $a$ và $b$ có $\mathrm{C}_{11}^3=165$ cách.
	}
\end{ex}
\begin{ex}%[0D0V2-4]%[Dự án D - đợt 2 NH24-25- Thy Nguyen Vo Diem]
	Một nhóm học sinh gồm có $3$ học sinh lớp $A$, $4$ học sinh lớp $B$ và $5$ học sinh lớp $C$. Chọn ngẫu nhiên $2$ học sinh tham gia câu lạc bộ Toán học. Xác suất sao cho $2$ học sinh được chọn nếu có học sinh lớp $B$ thì không có học sinh lớp $C$ có dạng $T=\dfrac{a}{b}$, $a$, $b \in \mathbb{N}$ và $\dfrac{a}{b}$ là phân số tối giản. Tính $a+b$.
	\par\shortans[]{56}
	\loigiai{
		Không gian mẫu là $n(\Omega) = \mathrm{C}^2_{12} = 66$.\\
		Gọi $A$ là biến cố \lq\lq $2$ học sinh được chọn nếu có học sinh lớp $B$ thì không có học sinh lớp $C$\rq\rq.\\
		Khi đó $\overline{A}$ là biến cố \lq\lq $2$ học sinh được chọn là $1$ học sinh lớp $B$ và $1$ học sinh lớp $C$\rq\rq.\\
		Ta có $n\left( \overline{A} \right) = 4\times 5 = 20\Rightarrow n(A)=66-20=46$ nên $\mathrm{P}(A) = \dfrac{n(A)}{n(\Omega)} =  \dfrac{46}{66} = \dfrac{23}{33}$.
	}
\end{ex}

\begin{ex}%[0D0V2-5]%[Dự án D - đợt 2 NH24-25- Thy Nguyen Vo Diem]
	An và Bình cùng chơi một trò chơi, mỗi lượt chơi một bạn đặt úp năm tấm thẻ, trong đó có hai thẻ ghi số $2$, hai thẻ ghi số $3$ và một thẻ ghi số $4$, bạn còn lại chọn ngẫu nhiên ba thẻ trong năm tấm thẻ đó. Người chọn thẻ thắng lượt chơi nếu tổng các số trên ba tấm thẻ được chọn bằng $8$, ngược lại người kia sẽ thắng. Xác suất để An thắng lượt chơi khi An là người chọn thẻ bằng
	\par\shortans[]{0,3}
	\loigiai{
		Số cách chọn $3$ thẻ trong $5$ tấm thẻ là $n(\Omega)=\mathrm{C}_{5}^{3}=10$.\\
		Gọi $A$ là biến cố để An thắng lượt chơi.\\
		Số các trường hợp xảy ra cho $A$ là
		\begin{itemize}
			\item $2$ thẻ số $2$ và một thẻ số $4$ có $1$ cách.
			\item $2$ thẻ số $3$ và $1$ thẻ số $2$ có $2$ cách.
		\end{itemize}
		Suy ra số các trường hợp xảy ra cho A là $n(A)=3$.\\
		Vậy $\mathrm{P}(A)=\dfrac{n(A)}{n(\Omega)}=\dfrac{3}{10}$.
	}
\end{ex}

\begin{ex}%[Dự án D - đợt 2 NH24-25- Thy Nguyen Vo Diem]%[0D0V2-9]
	Trong đợt ủng hộ sách giáo khoa cho những học sinh bị ảnh hưởng do trận lũ lụt vừa qua, lớp 10A nhận được $20$ cuốn sách gồm $5$ cuốn sách toán học, $7$ cuốn sách Vật lí, $8$ cuốn sách Hóa học, các sách cùng môn học là giống nhau. Số sách này được chia đều cho $10$ học sinh, mỗi học sinh chỉ được nhận đúng $2$ cuốn sách khác môn học. Trong số $10$ học sinh nhận sách đợt này có bạn Hưng và bạn Thành. Tính xác suất để $2$ cuốn sách mà bạn Hưng nhận đươc giống $2$ cuốn sách của bạn Thành (làm tròn kết quả đến hàng phần trăm).
\par\shortans[]{0,31}
	\loigiai{
		Gọi $x$ là số học sinh nhận sách Toán và Lí;\\
		$y$ là số học sinh nhận sách Lí và Hóa;\\
		$z$ là số học sinh nhận sách Hóa và Toán.\\
		Theo bài ra ta có hệ
		$$\heva{&x+y=7\\&y+z=8\\&z+x=5} \Leftrightarrow \heva{&x=2\\&y=5\\&z=3.}$$
		\begin{itemize}
			\item Số phần tử của không gian mẫu: $n(\Omega )=\mathrm{C}_{10}^5\cdot \mathrm{C}_5^3\cdot \mathrm{C}_2^2=2520$.
			\item Gọi $A$ là biến cố \lq\lq Hưng nhận $2$ cuốn sách giống bạn Thành\rq\rq.
			\begin{itemize}
				\item[TH1:] Thành và Hưng nhận sách Toán và Lí có $1\cdot \mathrm{C}_8^5\cdot \mathrm{C}_3^3=56$ cách.
				\item[TH2:] Thành và Hưng nhận sách Hóa và Lí có $1\cdot \mathrm{C}_8^3\cdot \mathrm{C}_5^3\cdot \mathrm{C}_2^2=560$ cách.
				\item[TH3:] Thành và Hưng nhận sách Toán và Hóa có $1\cdot \mathrm{C}_8^1\cdot \mathrm{C}_7^5\cdot \mathrm{C}_2^2=168$ cách.
			\end{itemize}
			$\Rightarrow n(A)=56+560+168=784$.
			\item Xác suất của biến cố $A$ là $\mathrm{P}(A)=\dfrac{784}{2520}\approx 0{,}31$.
		\end{itemize}
	}
\end{ex}
\Closesolutionfile{ans}
\ind{PHẦN IV.} \inden{Tự luận.}
\setcounter{bt}{0}
\begin{bt}%[0D0H2-5]%[Dự án D - đợt 2 NH24-25- Thy Nguyen Vo Diem]
	Một hộp có $4$ tấm bìa cùng loại, mỗi tấm bìa được ghi một trong các số $1,2,3,4$; hai tấm bìa khác nhau thì ghi hai số khác nhau. Rút ngẫu nhiên đồng thời $3$ tấm bìa từ trong hộp.
	\begin{enumEX}[a)]{1}
		\item Tính số phần tử của không gian mẫu.
		\item Xác định các biến cố sau:\\
		$A\colon$ \lq\lq Tổng các số trên ba tấm bìa bằng $9$\rq\rq;\\
		$B\colon$ \lq\lq Các số trên ba tấm bìa là ba số tự nhiên liên tiếp\rq\rq.
		\item Tính $\mathrm{P}(A)$, $\mathrm{P}(B)$.
	\end{enumEX}
	\loigiai{
		\begin{enumEX}[a)]{1}
			\item Số phần tử của không gian mẫu là $n(\Omega)=\mathrm{C}_4^3=4$.
			\item $A\colon$ \lq\lq Tổng các số trên ba tấm bìa bằng 9\rq\rq.\\
			Các kết quả thuận lợi cho biến cố $A$ là $\{(2;3;4)\}$.\\
			$B\colon$ \lq\lq Các số trên ba tấm bìa là ba số tự nhiên liên tiếp\rq\rq.\\
			Các kết quả thuận lợi cho biến cố $B$ là $\{(1;2;3);(2;3;4)\}$.
			\item  $\mathrm{P}(A)=\dfrac{n(A)}{n(\Omega)}=\dfrac{1}{4}$, $\mathrm{P}(B)=\dfrac{n(B)}{n(\Omega)}=\dfrac{1}{2}$.
		\end{enumEX}
	}
\end{bt}
\begin{bt}%[0D0H2-5]%[Dự án D - đợt 2 NH24-25- Thy Nguyen Vo Diem]
	Có ba chiếc hộp. Hộp thứ nhất chứa $5$ tấm thẻ đánh số từ $1$ đến $5$. Hộp thứ hai chứa $6$ tấm thẻ đánh số từ $1$ đến $6$. Hộp thứ ba chứa $7$ tấm thẻ đánh số từ $1$ đến $7$. Từ mỗi hộp rút ngẫu nhiên một tấm thẻ. Tính xác suất để tổng ba số ghi trên ba tấm thẻ bằng $15$.
	\loigiai
	{
		Không gian mẫu $\Omega= \{(a,b,c) \text{ với } 1\le a\le 5,\, 1 \le b\le 6,\, 1\le c\le 7\}$. Suy ra $n(\Omega)= 5\cdot 6\cdot 7 = 210$.\\
		Gọi $A$ là biến cố \lq\lq Tổng ba số ghi trên ba tấm thẻ bằng $15$\rq\rq.\\
		Ta có $A=\{(2,6,7); (3,6,6); (3,5,7); (4,6,5); (4,5,6); (4,4,7); (5,3,7); (5,4,6); (5,5,5); (5,6,4)\}$.\\
		Do đó $n(A)=10$.\\
		Vậy xác suất cần tìm là $\mathrm{P}(A)= \dfrac{10}{210}= \dfrac{1}{21}$.
	}
\end{bt}
\begin{bt}%[0D0V2-3]%[Dự án D - đợt 2 NH24-25- Thy Nguyen Vo Diem]
	Hai bạn nam Dũng, Huy và hai bạn nữ Hoa, Thảo được xếp ngồi ngẫu nhiên vào bốn ghế xếp thành hai dãy đối diện nhau. Tính xác suất xếp được
	\begin{enumEX}[a)]{2}
		\item Nam, nữ ngồi đối diện nhau;
		\item Nữ ngồi đối diện nhau.
	\end{enumEX}
	\loigiai{
		\begin{enumEX}[a)]{1}
			\item Mỗi cách xếp $4$ bạn vào $4$ vị trí ngồi là hoán vị của $4$ phần tử. Do đó số phần tử của không gian mẫu là $n(\Omega)=4!=24$.\\
			Gọi $A$ là biến cố \lq\lq Nam, nữ ngồi đối diện nhau\rq\rq.\\
			Dãy $I$ ta đánh số vị trí $1$ và $2$, dãy $II$ ta đánh số vị trí $3$ và $4$.
			\begin{itemize}
				\item Xếp $1$ bạn nam vào vị trí $1$ có $2$ cách.
				\item Xếp $1$ bạn nữ vào vị trí $3$ có $2$ cách.
				\item $2$ bạn còn lại vào vị trí $2, 4$ có $2!$ cách.
				\item Đổi chỗ ta có $2!$ cách.
			\end{itemize}
			Do đó $n(A)=2\cdot2\cdot2!\cdot2!=16$ phần tử.\\
			Xác suất của biến cố $A$ là $\mathrm{P}(A)=\dfrac{n(A)}{n(\Omega)}=\dfrac{16}{24}=\dfrac{2}{3}$.
			\item Ta có $2$ nữ ngồi đối diện nhau là biến cố đối của biến cố $A$.\\
			Do đó $\mathrm{P}\left( \overline{A}\right) =1-\mathrm{P}(A)=1-\dfrac{2}{3}=\dfrac{1}{3}$.
		\end{enumEX}
	}
\end{bt}
