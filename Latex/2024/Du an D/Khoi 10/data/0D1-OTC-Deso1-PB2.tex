\newpage
\section{Ôn tập Chương 1}
\def\thoigian{90}%--Thời gian
\de{Đề số 1}{Chương I. Mệnh đề và Tập hợp}

\begin{center}
	\textbf{PHẦN 1 - Câu trắc nghiệm nhiều phương án lựa chọn.}
\end{center}
\setcounter{ex}{0}

\Opensolutionfile{ans}[ans-ABCD]

\begin{ex}%[0D1N1-1]%[Hieu Phan]
	Trong các câu dưới đây có bao nhiêu câu là mệnh đề?
	\begin{itemize}
		\item  Số $5$ là số nguyên tố.
		\item  Lớp $10$A hát hay không?
		\item  $2\,024$ là số lẻ.
		\item  Tháng $10$ dương lịch có $26$ ngày.
	\end{itemize}
	\choice
	{$2$}
	{$1$}
	{$4$}
	{\True $3$}
	\loigiai{
		\begin{itemize}
			\item  \lq\lq Số $5$ là số nguyên tố\rq\rq \, là mệnh đề đúng.
			\item  \lq\lq Lớp $10$A hát hay không?\rq\rq\, là câu hỏi, không là mệnh đề.
			\item  \lq\lq $2\,024$ là số lẻ\rq\rq\, là mệnh đề sai.
			\item  \lq\lq Tháng $10$ dương lịch có $26$ ngày\rq\rq\, là mệnh đề sai vì tháng $10$ có $31$ ngày.
		\end{itemize}
		Vậy có $3$ mệnh đề.
	}
\end{ex}
\begin{ex}%[0D1N1-1]%[Hieu Phan]
	Cho mệnh đề $P:$ \lq\lq $\exists x \in \mathbb{R},\, x^2+1>2 x$\rq\rq. Mệnh đề nào sau đây là mệnh đề phủ định của mệnh đề $P$?
	\choice
	{\True $\overline{P}: $ \lq\lq $ \forall x \in \mathbb{R},\, x^2+1 \leq 2 x $\rq\rq}
	{$\overline{P}: $ \lq\lq $ \forall x \in \mathbb{R},\, x^2+1 \neq 2 x $\rq\rq}
	{$\overline{P}: $ \lq\lq $ \exists x \in \mathbb{R},\, x^2+1 \neq 2 x $\rq\rq}
	{$\overline{P}: $ \lq\lq $ \exists x \in \mathbb{R},\, x^2+1<2 x $\rq\rq}
	\loigiai
	{
		Ta có mệnh đề phủ định của mệnh đề $P$ là $\overline{P}: $ \lq\lq $ \forall x \in \mathbb{R},\, x^2+1 \leq 2 x $\rq\rq.
	}
\end{ex}
\begin{ex}%[0D1H1-1]%[Hieu Phan]
	Viết mệnh đề sau bằng cách sử dụng kí hiệu $\forall$ hoặc $\exists$: \lq\lq Có một số nguyên bằng bình phương của chính nó\rq\rq.
	\choice
	{\True $\exists x \in \mathbb{Z},\, x=x^{2}$}
	{$\forall x \in \mathbb{Z},\, x^{2}=x$}
	{$\forall x \in \mathbb{R},\, x=x^{2}$}
	{$\exists x \in \mathbb{R},\, x^{2}-x=0$}
	\loigiai{Mệnh đề: \lq\lq Có một số nguyên bằng bình phương của chính nó\rq\rq  được viết dưới dạng kí hiệu là
		$$\exists x \in \mathbb{Z},\, x=x^{2}.$$
	}
\end{ex}
\begin{ex}%[0D1N1-2]%[Hieu Phan]
	Trong các mệnh đề sau, mệnh đề nào đúng?
	\choice
	{$\forall n \in \mathbb{N},\, n>1$}
	{\True$\exists n \in \mathbb{Q},\, n^2=n$}
	{$\exists n \in \mathbb{Z},\, n(n+1)$ là một số lẻ}
	{$\forall n \in \mathbb{N},\, n^2>n$}
	\loigiai
	{
		Do $n^2=n\Rightarrow \hoac{&n=1\\&n=0}$ nên  $\exists n \in \mathbb{Q},\, n^2=n$.
	}
\end{ex}
\begin{ex}%[0D1N2-1]%[Hieu Phan]
	Cho tập hợp $A=\{x\in\mathbb{N}\mid x\leq6\}$. Tập $A$ được viết dưới dạng liệt kê các phần tử là
	\choice
	{$A=\{0;1;2;3;4;5\}$}
	{\True $A=\{0;1;2;3;4;5;6\}$}
	{$A=\{0;1;2;4;5;6\}$}
	{$A=\{1;2;3;4;5;6\}$}
	\loigiai{
		Ta có $A=\{x\in\mathbb{N}\mid x\leq 6\}=\{0;1;2;3;4;5;6\}$.
	}
\end{ex}
\begin{ex}%[0D1N2-1]%[Hieu Phan]
Cho tập hợp $A=\{x\in \mathbb{R}|x^2+x-12=0\}$. Số phần tử của tập $A$ là
\choice
{\True $2$}
{$\varnothing$}
{$1$}
{$0$}
\loigiai{
	Ta có $x^2+x-12=0\Rightarrow \hoac{&x=-4\\&x=3.}$ Vậy tập hợp $A$ có hai phần tử.	
}
\end{ex}

\begin{ex}%[0D1N2-3]%[Hieu Phan]
	Cho tập hợp $A= \left\{x \in \mathbb{Z} \mid -3<x<1\right\}$. Khẳng định nào sau đây đúng?
	\choice
	{$A=\left(-3;1\right)$}
	{\True $A=\left\{-2;-1;0\right\}$}
	{$A=[-3 ;1)$}
	{$A=[-3;1]$}
	\loigiai{ Ta có $A= \left\{x \in \mathbb{Z} \mid -3<x<1\right\}=\left\{-2;-1;0\right\}$ (đúng).
	}
\end{ex}
\begin{ex}%[0D1H2-2]%[Hieu Phan]
	Cho tập hợp $A = \{ 1;2 \}$ và $B = \{1;2;3;4;5\}$. Có tất cả bao nhiêu tập $X$ thỏa mãn $A \subset X \subset B$?
	\choice
	{$5$}
	{$6$}
	{$7$}
	{\True $8$}
	\loigiai{
		Các tập hợp $X$ thỏa mãn $A \subset X \subset B$ là 
		\begin{itemize}
			\item Gồm hai phần tử là $\{ 1;2 \}$.
			\item Gồm ba phần tử là $\{ 1;2;3 \}$, $\{ 1;2;4 \}$, $\{ 1;2;5 \}$.
			\item Gồm bốn phần tử là $\{ 1;2;3;4 \}$, $\{ 1;2;3;5 \}$, $\{ 1;2;4;5 \}$.
			\item Gồm năm phần tử  là $\{ 1;2;3;4;5 \}$.
		\end{itemize}
		Vậy có tất cả $8$ tập $X$ thoả mãn yêu cầu bài toán.	
	}
\end{ex}
\begin{ex}%[0D1N3-2]%[Hieu Phan]
	Cho tập hợp $A=(2023;+\infty)$. Khi đó tập $C_{\mathbb{R}}A$ là
	\choice
	{$(-\infty;2022]$}
	{$[2023;+\infty)$}
	{$(2023;+\infty)$}
	{\True $(-\infty;2023]$}
	\loigiai{
		Ta có $C_{\mathbb{R}}A=\mathbb{R}\setminus(2023;+\infty)=(-\infty;2023]$.
	}
\end{ex}
\begin{ex}%[0D1N3-3]%[Hieu Phan]
	Cho tập hợp $A=(-\infty;-1]$ và tập $B=(-2;+\infty)$. Khi đó $A \cup B$ là  
	\choice
	{$(-2;+\infty)$}
	{$(-2;-1]$}
	{\True $\mathbb{R}$}
	{$\varnothing$}
	\loigiai{
		$A \cup B=(-\infty;-1] \cup (-2;+\infty) = \mathbb{R}$.
	}
\end{ex}
\begin{ex}%[0D1H3-4]
	Cho hai tập hợp $A= [-1;3)$ và $B=(2;5]$. Khẳng định nào sau đây {\bf sai}?
	\choice
	{$A \cap B=(2;3)$}
	{$A \cup B=[-1;5]$}
	{\True $B \setminus A= (3;5]$}
	{$A \setminus B= [-1;2]$}
	\loigiai{ Cho hai tập hợp $A= [-1;3)$ và $B=(2;5]$.\\
		Khi đó
		$A \cap B=(2;3)$; $A \cup B=[-1;5]$; $B \setminus A= [3;5]$; $A \setminus B= [-1;2]$.\\
		Khẳng định \textbf{sai} là $B \setminus A= (3;5]$.
	}
\end{ex}
\begin{ex}%[0D1H3-1]
	Cho hai tập hợp $A = \left\lbrace 0;2\right\rbrace $ và $B = \left\lbrace 0; 1; 2; 3; 4 \right\rbrace$. Có bao nhiêu tập hợp $X$ thỏa mãn $A \cup X = B$?
	\choice
	{$2$}
	{$3$}
	{\True $4$}
	{$5$}
	\loigiai{
		Các tập hợp $X$ thỏa mãn $A \cup X = B$ là $\left\lbrace 1;3;4 \right\rbrace $, $\left\lbrace 0;1;3;4 \right\rbrace $, $\left\lbrace 1;2;3;4 \right\rbrace$, $\left\lbrace 0;1;2;3;4 \right\rbrace $.\\
		Vậy có $4$ tập hợp $X$ thỏa mãn yêu cầu đề bài.
	}
\end{ex}
\Closesolutionfile{ans}

\begin{center}
	\textbf{PHẦN 2 - Câu trắc nghiệm đúng sai. Trong mỗi ý a,b,c,d ở mỗi câu, thí sinh chọn đúng hoặc sai}
\end{center}

\setcounter{ex}{0}

\Opensolutionfile{ans}[ans-DS]

\begin{ex}%[0D1H1-3]%[Hieu Phan]
	Cho mệnh đề $P(x):$ \lq\lq $x^2-5 x+4 \leq 0$\rq\rq và mệnh đề $Q(x):$ \lq\lq $x^2-2 x+1 \leq 0$\rq\rq.
	\choiceTF{\True Với $x=2$, ta được $P(2)$ là một mệnh đề đúng}
	{Mệnh đề phủ định của $Q(x)$ là $\overline{Q(x)}:$ \lq\lq$x^2-2 x+1 \geq 0$\rq\rq}
	{Có $3$ giá trị nguyên của $x$ để $P(x)$ trở thành mệnh đề đúng}
	{\True Tồn tại giá trị thực của $x$ để $P(x)$ và $Q(x)$ đồng thời trở thành mệnh đề đúng}
	\loigiai{
		
		\begin{itemchoice}
			\itemch Mệnh đề $P(2):$ \lq\lq $-2\le 0$ \rq\rq là mệnh đề đúng.
			\itemch Mệnh đề phủ định của $Q(x)$ là $\overline{Q(x)}:$ \lq\lq $x^2-2 x+1 > 0$\rq\rq.
			\itemch Ta có $x^2-5 x+4 \leq 0\Rightarrow x\in[1;4]$. Với $x\in\mathbb{Z}$ thì $x\in\{1;2;3;4\}$ có $4$ giá trị.
			\itemch Ta có $x^2-2 x+1 \leq 0\Rightarrow (x-1)^2\le 0\Rightarrow x=1$. Vậy tồn tại $x=1$ để $P(x)$, $Q(x)$ cùng đúng.
		\end{itemchoice}
	}	
\end{ex}
\begin{ex}%[0D1H3-4]%[Hieu Phan]
	Cho $A=(1; 9]$, $B=[3;+\infty)$. 
	\choiceTF
	{$A\cap B=(1; 3]$}
	{\True $A\cup B=(1;+\infty)$}
	{$B\setminus A=(1; 3)$}
	{$A\setminus B$ có chứa $2$ số nguyên}
	\loigiai{
		\begin{itemchoice}
			\itemch 
			Vì $A\cap B=[3;9]$.
			\itemch 
			Vì $A\cup B=(1;+\infty)$.
			\itemch 
			Vì $B\setminus A=(9;+\infty)$.
			\itemch 
			$A\setminus B=(1;3)$. Suy ra $A\setminus B$ chứa $1$ số nguyên là $2$.
		\end{itemchoice}
	}
\end{ex}
\Closesolutionfile{ans}

\begin{center}
	\textbf{PHẦN 3 - Câu trắc nghiệm trả lời ngắn}
\end{center}
\setcounter{ex}{0}

\Opensolutionfile{ans}[ans-KQ]
\begin{ex}%[0D1H2-2]%[Hieu Phan]
	Tập hợp $A = \left\{0;1;2;3;4;5\right\}$ có bao nhiêu tập con gồm một phần tử?
	\shortans[oly]{$6$}
	\loigiai{
		Các tập con có một phần tử của $A$ là $\{0\}$; $\{1\}$; $\{2\}$; $\{3\}$; $\{4\}$; $\{5\}$.\\
		Vậy có tất cả $6$ tập con thỏa yêu cầu.}
\end{ex}
\begin{ex}%[0D1V2-2]%[Hieu Phan]
	Cho hai tập hợp con, khác rỗng của $\mathbb{R}$ là $S=\left[-6;24\right)$ và $T=\left(a-2;20\right)$. Có tất cả bao nhiêu giá trị nguyên âm của tham số $a$ để $T\subset S$?
	\shortans{$4$}
	\loigiai{
		Điều kiện $a-2<20\Rightarrow a<22$.\\
		Để $T\subset S$ thì $a-2\ge -6\Rightarrow a\ge -4$.\\
		Kết hợp điều kiện suy ra $-4\le a<22$.\\
		Chọn $a$ là số nguyên âm nên $a\in \left\{-4;-3;-2;-1\right\}$.\\
		Vậy có $4$ giá trị thỏa đề.}
\end{ex}
\begin{ex}%[0D1H3-1]%[Hieu Phan]
Cho hai tập hợp $A=\left\{x\in \mathbb{R}\mid x^2-3x+2=0\right\}$; $B=\{n \in \mathbb{N}\mid n\leq 5\}$. Tìm số tập con của tập $A\cap B$.
\shortans[oly]{$4$}
\loigiai{
	Ta có $x^2-3x+2=0\Rightarrow\hoac{& x=1 \\ & x=2}$ nên $A=\{1;2\}$.\\
	Tiếp đến, $B=\{0;1;2;3;4;5\}$.\\
	Do đó $A\cap B=\{1;2\}$.\\
	Vậy có tất cả $2^2=4$ tập con của tập $A\cap B$.
}
\end{ex}
\begin{ex}%[0D1V3-5]%[Hieu Phan]
	Một cuộc khảo sát về khách du lịch thăm vịnh Hạ Long cho thấy trong $1\,410$ khách du lịch được phỏng vấn có $800$ khách du lịch đến thăm động Thiên Cung, $990$ khách du lịch đến đảo Titop. Biết rằng toàn bộ khách được phỏng vấn đã đến ít nhất một trong hai địa điểm trên. Hỏi có bao nhiêu khách du lịch vừa đến thăm động Thiên Cung vừa đến thăm đảo Titop ở vịnh Hạ Long?
	\shortans[]{$380$}
	\loigiai
	{
		Gọi $A$, $B$ lần lượt là tập hợp khách du lịch đến thăm động Thiên Cung và đảo Titop.\\
		Ta có $n(A)=800$, $n(B)=990$ và $n(A\cup B)=1\,410$.\\
		Số lượng khách vừa đến thăm động Thiên Cung vừa đến thăm đảo Titop là 
		$$n(A\cap B)=n(A)+n(B)-n(A\cup B)=800+990-1\,410=380.$$
	}
\end{ex}
\Closesolutionfile{ans}


\begin{center}
	\textbf{PHẦN 4 - Phần tự luận}
\end{center}
\setcounter{ex}{0}

\Opensolutionfile{ans}[ans-KQ]
\begin{ex}%[0D1H1-5]%[Hieu Phan]
	Xét tính đúng sai và viết mệnh đề phủ định của mệnh đề $P:$  \lq\lq $\forall x \in \mathbb{R}$, $x^2+8>4 x$\rq\rq.
	\loigiai{
		Ta có $x^2+8>4x\Rightarrow x^2-4x+8>0\Rightarrow (x-2)^2+4>0$ luôn đúng $\forall x \in \mathbb{R}$.\\
		Do đó mệnh đề $P$ là mệnh đề đúng.\\
		Mệnh đề $\overline{P}: $\lq\lq $\exists x \in \mathbb{R}, x^2+8\le 4x$\rq\rq.
	}
\end{ex}
\begin{ex}%[0D1V2-2]%[Hieu Phan]
	Gọi $A$ là tập nghiệm của phương trình $(x-m^2)(x+m-3)=0$ và $B=\{1;4\}$. Tìm tất cả giá trị của $m$ để $A=B$.
	\loigiai{
		Ta có 
		$$(x-m^2)(x+m-3)=0 \Rightarrow \hoac{&x-m^2=0 \\ &x+m-3=0} \Rightarrow \hoac{&x=m^2 \\ &x=3-m.}$$
		Suy ra $A=\{ m^2;3-m\}$, nên $A=B$ khi và chỉ khi 
		$$\hoac{&\heva{&m^2=1 \\ &3-m=4} \\ &\heva{&m^2=4 \\ &3-m=1}}
		\Rightarrow \hoac{&\heva{&m=\pm 1 \\ &m=-1} \\ &\heva{&m=\pm 2 \\ &m=2}}
		\Rightarrow \hoac{&m=-1 \\ &m=2.}$$
		Vậy $m=-1$ hoặc $m=2$.
	}
\end{ex}
\begin{ex}%[0D1V3-5]%[Hieu Phan]
	Người ta tiến hành khảo sát $ 100 $ người về hai bộ phim A và B đã được khởi chiếu trong tuần qua và ghi nhận được kết quả như sau
	\begin{itemize}
		\item Có $ 64 $ người đã xem phim A.
		\item $ 52 $ người đã xem phim B.
		\item $ 12 $ người chưa xem phim nào.
	\end{itemize}
	Hỏi trong $ 100 $ người được khảo sát đó, có bao nhiêu người đã xem cả hai phim và bao nhiêu người chỉ xem đúng một phim A?
	\loigiai{
		Số người xem ít nhất một trong hai phim là $100-12=88$ người.\\
		Gọi số người xem cả hai phim là $x$ $\left( x\in \mathbb{N};\,x\le 52 \right)$.
		\begin{itemize}
			\item Số người chỉ xem đúng một phim A là $64-x$ (người).
			\item Số người chỉ xem đúng một phim B là $52-x$ (người).
		\end{itemize}
		Vì số người xem ít nhất một trong hai phim là $88$ người nên $$(64-x)+(52-x)+x=88\Rightarrow x=28.$$
		Vậy có $28$ người đã xem cả hai phim và có $64-28=36$ người chỉ xem đúng một phim A.
	}
\end{ex}