\newpage
\def\thoigian{90}%--Thời gian
\de{Đề số 3}{Chương VI. Thống kê}


\begin{center}
	\textbf{PHẦN 1 - CÂU TRẮC NGHIỆM BỐN PHƯƠNG ÁN}
\end{center}
\Opensolutionfile{ans}[ans/ans-TN-0D6-DE1]

\begin{ex}%[0D6N1-3]
Khi sử dụng máy tính bỏ túi ta được $\sqrt{5}=2{,}236067977\ldots$. Giá trị gần đúng của $\sqrt{5}$ quy tròn đến hàng phần trăm là
\choice
	{$2{,}23$}
	{$2{,}2$}
	{$2{,}236$}
	{\True $2{,}24$}
	\loigiai{
		Giá trị gần đúng của $\sqrt{5}$ quy tròn đến hàng phần trăm là $2{,}24$.
	}
\end{ex}

\begin{ex}%[0D6N1-3]
	Cho số gần đúng $a=326\,819$ với độ chính xác $d=200$. Số quy tròn của $a$ là
	\choice
	{$326\,000$}
	{$32\,620$}
	{$32\,600$}
	{\True $327\,000$} 
	\loigiai
	{ Hàng lớn nhất của độ chính xác là hàng trăm nên ta quy tròn đến hàng nghìn.\\
Số quy tròn của số $a$ là $327\,000$.}
\end{ex}
\begin{ex}%[0D6N1-1]
	Số $a=-6{,}2$ là số gần đúng của số đúng $\overline{a}$ với sai số tuyệt đối $\Delta_a\le 0{,}31$. Sai số tương đối tối đa là
	\choice
	{$6{,}2\%$}
	{$0{,}31\%$}
	{$0{,}5\%$}
	{\True $5\%$}
	\loigiai{
		Sai số tương đối tối đa là $\delta_a=\dfrac{\Delta_a}{|a|}=\dfrac{0{,}31}{|-6{,}2|}=0{,}05=5\%$.
	}
\end{ex}

\begin{ex}
	Biểu đồ dưới đây biểu diễn số áo phông và áo sơ mi một cửa hàng bán được theo bốn mùa trong năm.
	\begin{center}
		\begin{tikzpicture}[>=stealth,line join=round,line cap=round,font=\footnotesize,scale=0.6]
		\def\aa{0.7}\def\bb{1.2}\def\cc{0.5}
		\draw[gray!50](-0.5,0) rectangle(19,10);	
		\foreach \i in {0,1,2,...,8}{\draw[gray!50](-0.5,\i)--++(0:19.5);}
		%A
		\draw[fill=blue!70,draw=blue] (1,0) rectangle (1+\aa,5);
		\draw[pattern=north east lines,pattern color=magenta,draw=magenta] (2,0) rectangle (2+\aa,2);
		%B
		\draw[fill=blue!70,draw=blue] (6,0) rectangle (6+\aa,3);
		\draw[pattern=north east lines,pattern color=magenta,draw=magenta] (7,0) rectangle (7+\aa,7);
		%C
		\draw[fill=blue!70,draw=blue] (11,0) rectangle (11+\aa,2);
		\draw[pattern=north east lines,pattern color=magenta,draw=magenta] (12,0) rectangle (12+\aa,3);
		%D
		\draw[fill=blue!70,draw=blue] (16,0) rectangle (16+\aa,3);
		\draw[pattern=north east lines,pattern color=magenta,draw=magenta] (17,0) rectangle (17+\aa,1);
		\path(1.7,-0.2)node[below]{Xuân}
		(6.7,-0.2)node[below]{Hạ}
		(11.7,-0.2)node[below]{Thu}
		(16.7,-0.2)node[below]{Đông};
		\draw[fill=blue!70,draw=none](2.5,-\bb) rectangle (2.5+\cc,-\bb-\cc)node[right,yshift=0.1cm]{Áo sơ mi};
		\draw[pattern=north east lines,pattern color=magenta,draw=magenta](10,-\bb) rectangle (10+\cc,-\bb-\cc)node[right,yshift=0.125cm]{Áo Phông};
		\foreach \x/\y/\z in {-1/0/200,-1/1/250,-1/2/300,-1/3/350,-1/4/400,-1/5/450,-1/6/500,-1/7/550,-1/8/600}
		\path(\x,\y) node[xshift=0.2cm,left]{\scriptsize $\z$};
		\node[below] at (current bounding box.north){\indamm{Số lượng áo phông và áo sơ mi bán được trong năm}};
		
		\end{tikzpicture}
	\end{center}
Vào mùa Xuân, số áo sơ mi bán được nhiều gấp bao nhiêu lần số áo phông?
	\choice
	{$2$}
	{$1{,}5$}
	{$3$}
	{$2{,}5$}
	\loigiai{
		Vào mùa Xuân, số lượng áo sơ mi bán được là $450$ chiếc và số áo phông bán được là $300$ chiếc nên số áo sơ mi bán được nhiều gấp $1{,}5$ lần số áo phông.}
\end{ex}
\begin{ex}
	Một đội $20$ thợ thủ công được chia đều vào $5$ tổ. Trong một ngày, mỗi người thợ làm được $4$ hoặc $5$ sản phẩm. Cuối ngày, đội trưởng thống kê lại số sản phẩm mà mỗi tổ làm được ở bảng sau:
	\begin{center}
		\begin{tikzpicture}
		\matrix[matrix of nodes,nodes in empty cells,
		row sep=-\pgflinewidth,column sep=-\pgflinewidth,
		nodes={minimum height=8mm,minimum width=20mm,draw=cyan,anchor=center},
		column 1/.style={nodes={minimum width=34mm,color=black,fill=cyan!20,draw=cyan}},
		row 1/.style={nodes={fill=cyan!10}},
		row 2/.style={nodes={minimum height=13mm}},
		]{
			Tổ &1&2&3&4&5\\ 
			\node[align=center]{Số sản phẩm}; &17&19&19&21&20\\
		};
		\end{tikzpicture}
	\end{center}	
	Đội trưởng đã thống kê sai số sản phẩm làm được của các tổ, hãy tìm tổ mà đội trưởng đã thống kê \textbf{sai}.
	\choice
	{\True Tổ $4$}
	{Tổ $5$}
	{Tổ $1$}
	{Tổ $2$ và tổ $3$}
	\loigiai{
		Mỗi tổ có $20:5=4$ người.\\
		Trong một ngày, mỗi người thợ làm được $4$ hoặc $5$ sản phẩm nên mỗi tổ làm được từ $16$ đến $20$ sản phẩm.\\
		Do đó, bảng trên ghi Tổ $4$ làm được $21$ sản phẩm là không chính xác.\\
		Vậy đội trưởng thống kê chưa đúng.
	}
\end{ex}
\begin{ex}
	\immini[thm]{Số lượng khách đến tham quan tại Đà Nẵng trong $12$ tháng được cho bởi biểu đồ như hình bên. Tính số tháng mà số người tham quan không dưới $400$ người.
	\choice
{$5$}
{$6$}
{$4$}
{$8$}}{
		\begin{tikzpicture}[scale=0.6,>=stealth]
		\draw(0,0) circle (3cm);
		\draw[pattern=north east lines,pattern color=orange] (0,0) -- +(0:3) arc (0:60:3)--(0,0) (1.5,0.75) node{\footnotesize $16,67\%$};
		\draw[pattern=north west lines,pattern color=cyan] (0,0) -- +(60:3) arc (60:180:3)--(0,0) (-0.5,1.5) node{\footnotesize $33,33\%$ };
		\draw[pattern=dots,pattern color=magenta] (0,0) -- +(180:3) arc (180:270:3)--(0,0)(-1.5,-1.5)node{\footnotesize $25\%$ }
		(1.5,-1.5)node{\footnotesize $25\%$};
		\draw[pattern=north east lines,pattern color=orange] (3.5,1.5) rectangle (4,2) (5.5,1.75) node{\footnotesize $[200;300)$};
		\draw[pattern=north west lines,pattern color=cyan] (3.5,0.5) rectangle (4,1) (5.5,0.75) node{\footnotesize $[300;400)$};
		\draw[pattern=dots,pattern color=magenta] (3.5,-0.5) rectangle (4,0) (5.5,-0.25) node{\footnotesize $[400;500)$};
		\draw (3.5,-1.5) rectangle (4,-1) (5.5,-1.25) node{\footnotesize $[500;600]$};
		\end{tikzpicture}}
	\loigiai{
		Dựa vào biểu đồ, ta có số người tham quan không dưới $400$ người chiếm tỉ lệ phần trăm là: $25+25=50$ \%.\\
		Vậy số tháng mà số người thăm quan trên $400$ người là $\dfrac{50\cdot 12}{100}=6$ tháng.
	}
\end{ex}

\begin{ex}%[0D6H3-2]
	Dựa vào biểu đồ giá cổ phiếu của ngân hàng Công thương Việt Nam (Viettinbank), mã cổ phiếu CTG qua các năm
	\begin{center}
		\begin{tikzpicture}[xscale=1.5,yscale=.3]
		\def\a{4mm}
		\begin{scope}[gray!50]
		\foreach \j in {20,25,...,40}
		\draw (0,\j-20) node[left,black,scale=1]{$\j$}--(7.5,\j-20); 
		\draw(0,0)--(0,20);
		\foreach \i in {0,1,...,7}
		\draw[xshift=2mm] (\i,0)--+(0,-10mm);
		\end{scope}
		
		\foreach \i/\j/\text in 
		{1/25.3/2018, 
			2/24.6/2019,
			3/27.4/2020,
			4/35.5/2021,
			5/32.8/2022,
			6/37.1/2023}{
			\fill[red] (\i,0) rectangle +(\a,\j-20);
			\draw (\i,0) node[below=8mm,rotate=45,align=left]{\bf \text};
			\draw(\i+0.15,\j-20) node[above,blue]{$\j$};
		}
		\node[above,scale=1] at (current bounding box.north)
		{Giá cổ phiếu CTG qua các năm (2018-2023)}; 
		\node[below=3mm] at (current bounding box.south)
		{\it Năm};
		\draw (-0.5,5) node[above=10mm,rotate=90,align=left]{Giá cổ phiếu (ngàn VNĐ)}; 
		\end{tikzpicture}
	\end{center}
	Giá trị trung bình của giá cổ phiếu CTG từ năm $2018$ đến năm $2023$ là
	\choice
	{$29{,}95$}
	{$28{,}83$}
	{\True $30{,}45$}
	{$30{,}15$}
	\loigiai{
		Giá trị trung bình của giá cổ phiếu CTG từ năm $2018$ đến năm $2023$ là
		$$\dfrac{25{,}3+24{,}6+27{,}4+35{,}5+32{,}8+37{,}1}{6}=30{,}45.$$
	}
\end{ex}

\begin{ex}%[0D6H3-3]
	Cho bảng số liệu ghi nhận chiều cao (cm) của 11 cây cà phê sau 30 ngày như sau: 
	\begin{center}
		$8$; $6$; $1$; $6$; $10$; $3$; $8$; $2$; $11$; $15$; $12$.
	\end{center} 
	Tìm trung vị của mẫu số liệu trên
	\choice
	{$M_e=6$}
	{\True $M_e=8$}
	{$M_e=15$}
	{$M_e=1$}
	\loigiai{
		Sắp xếp mẫu số liệu tăng dần $1$; $2$; $3$; $6$; $6$; $8$;  $8$; $10$; $11$; $15$; $12$.\\
		Vì mẫu số liệu có 11 giá trị nên trung vị là giá trị thứ 6 (nằm chính giữa) của mẫu số liệu đã sắp xếp trên.\\
		Vậy $M_e=8$.
	}
\end{ex}
\begin{ex}%[0D6H4-1]
	Hình dưới là biểu đồ biểu diễn số lượng xuất khẩu gạo (đơn vị: triệu tấn) của Việt Nam giai đoạn 1989-2015.
	\begin{center}
		\begin{tikzpicture}[font=\footnotesize, line join=round, line cap=round, >=stealth]
		\def\l{10}
		\def\xx{0.65}
		\def\r{0.35*\xx}
		\foreach \trai/\mon [count=\x] in {1.37/1989, 1.48/1990, 1.02/1991, 1.96/1992, 1.62/1993, 1.93/1994, 2.02/1995, 3.05/1996, 3.68/1997, 3.79/1998, 4.56/1999, 3.39/2000, 3.53/2001, 3.25/2002, 3.92/2003, 4.06/2004, 5.21/2005, 4.69/2006, 4.53/2007, 4.68/2008, 6.05/2009, 6.75/2010, 7.13/2011, 7.72/2012, 6.68/2013, 6.32/2014, 6.57/2015} {
			% Làm tròn số tới 2 chữ số thập phân
			\pgfmathparse{round(100*\trai)/100} \let\formatted\pgfmathresult
			
			% Tách phần nguyên và phần thập phân
			\pgfmathsetmacro{\ntrai}{int(\formatted)}
			\pgfmathsetmacro{\ttrai}{int(round(100*(\formatted-\ntrai)))}
			
			% Thêm số 0 nếu phần thập phân nhỏ hơn 10
			\pgfmathparse{\ttrai < 10 ? "0" : ""} \let\leadingzero\pgfmathresult
			\edef\finalformatted{\ntrai,\leadingzero\ttrai}
			
			% Vẽ cột và gắn nhãn
			\fill[pattern=dots, draw] (\x*\xx, 0) rectangle ({\x*\xx-\r}, \trai);
			\path({\x*\xx-0.5*\r}, \trai) node[above] {\color{magenta}$\finalformatted$};
			\path({\x*\xx-0.5*\r-0.3}, -0.3) node[below, text width=1.75cm, align=center, rotate=60] {\color{magenta}\textbf{\mon}};
		}
		
		% Trục y
		\draw[->] (0, 0) -- (0, 9.3);
		
		% Trục x
		\draw[->] (0, 0) -- ({28*\xx}, 0);
		
		% Các vạch chia trên trục y
		\foreach \i in {1, 2, ..., 9}
		{
			\draw[fill=black] (0, \i) circle(1pt) node[left] {$\i$};
			\draw[color=gray!40] (0, \i)--(28*\xx, \i);
		}
		
		\end{tikzpicture}
	\end{center}
	Tứ phân vị thứ nhất của mẫu số liệu cho bởi biểu đồ trên thuộc khoảng nào sau đây?
	\choice
	{$[1; 2)$}
	{$[4; 5)$}
	{$[3; 4)$}
	{\True $[2; 3)$}
	\loigiai{
		Sắp xếp mẫu số liệu theo thứ tự không giảm, ta được
		\begin{center}
			\begin{tabular}{|c|c|c|c|c|c|c|c|c|c|c|c|c|c|}
				\hline $1{,}02$ & $1{,}37$ & $1{,}48$ & $1{,}62$ & $1{,}93$ & $1{,}96$ & $2{,}02$ & $3{,}05$ & $3{,}25$ & $3{,}39$ & $3{,}53$& $3{,}68$& $3{,}79$& $3{,}92$\\
				\hline
				$4{,}06$& $4{,}53$& $4{,}56$& $4{,}68$& $4{,}69$& $5{,}21$& $6{,}05$& $6{,}32$& $6{,}57$& $6{,}68$& $6{,}75$& $7{,}13$& $7{,}72$&\\
				\hline 
			\end{tabular}
		\end{center}
		Cỡ mẫu $n=27$.\\
		Trung vị của mẫu số liệu là số hạng thứ $\dfrac{27+1}{2}=14$, tức là $M_e=3{,}92$.\\
		Tứ phân vị thứ nhất của mẫu số liệu ban đầu chính là trung vị của mẫu số liệu
		\begin{center}
			\begin{tabular}{|c|c|c|c|c|c|c|c|c|c|c|c|c|}
				\hline $1{,}02$ & $1{,}37$ & $1{,}48$ & $1{,}62$ & $1{,}93$ & $1{,}96$ & $2{,}02$ & $3{,}05$ & $3{,}25$ & $3{,}39$ & $3{,}53$& $3{,}68$& $3{,}79$\\
				\hline 
			\end{tabular}
		\end{center}
		Tứ phân vị thứ nhất của mẫu số liệu ban đầu là $Q_1=2{,}02$.
	}
\end{ex}
\begin{ex}%[0D6N4-2]
	Cho dãy số liệu thống kê $1$; $2$; $3$; $4$; $5$; $6$; $7$. Khoảng biến thiên là
	\choice
	{$2$}
	{\True $6$}
	{$3$}
	{$1$}
	\loigiai{
		Ta có $7-1=6$.
	}
\end{ex}

\begin{ex}%[0D6H4-4]
	Theo dõi thời gian làm một bài toán (tính bằng phút) của $40$ học sinh, giáo viên lập được bảng sau
	\begin{center}
		\begin{tabular}{|c|c|c|c|c|c|c|c|c|c|c|}
			\hline Thời gian $(x)$& $4$ & $5$ & $6$ & $7$ & $8$ & $9$ & $10$ & $11$ & $12$ & \\
			\hline Tần số $(n)$& $6$ & $3$ & $4$ & $2$ & $7$ & $5$ & $5$ & $7$ & $1$ & $N=40$\\
			\hline
		\end{tabular}
	\end{center}
	Phương sai của mẫu số liệu trên gần với số nào nhất?
	\choice
	{\True $6$}
	{$12$}
	{$40$}
	{$9$}
	\loigiai{
		Ta có giá trị trung bình của mẫu số liệu là
		\[ \overline{x} = \dfrac{1}{N}\left( n_1x_1+n_2x_2+ \cdots + n_kx_k\right) =\dfrac{317}{40}.\]
		Suy ra phương sai của mẫu số liệu là
		\[ s^2 = \dfrac{1}{N}\left[ n_1\left(x_1-\overline{x}\right)^2 + n_2\left(x_2-\overline{x}\right)^2 + \cdots + n_k\left(x_k-\overline{x}\right)^2\right]  \approx 6.\]
	}
\end{ex}
\begin{ex}%[0D6N4-4]
	Cho phương sai của các số liệu bằng $4$. Tìm độ lệch chuẩn.
	\choice
	{$8$}
	{\True $2$}
	{$4$}
	{$16$}
	\loigiai{
		Ta có độ lệch chuẩn là căn bậc hai của phương sai. Suy ra \[ s_x=\sqrt{s_x^2}=\sqrt{4}=2.\]
	}
\end{ex}
\Closesolutionfile{ans}

%\begin{center}
%	\textbf{ĐÁP ÁN}
%	\inputansbox{10}{ans/ans}	
%\end{center}

\begin{center}
	\textbf{PHẦN 2 - CÂU TRẮC NGHIỆM ĐÚNG SAI}
\end{center}
\setcounter{ex}{0}
\Opensolutionfile{ans}[ans/answer-DS-ONTAPCHUONG-DE3]
\begin{ex}%[0D6H3-5]
	Cho mẫu số liệu về tổng thời gian của $9$ học sinh của lớp $10A$ tự học tại nhà trong một tuần (đơn vị: giờ) là $6; 5; 6; 9; 8; 4; 6; 6; 4$.
	\choiceTF
	{Mốt của mẫu số liệu là $M_o=5$}
	{\True Số trung bình của mẫu số liệu là $\overline{x}=6$}
	{\True Trung vị của mẫu số liệu là $M_e=6$}
	{Tứ phân vị thứ nhất của mẫu số liệu là $Q_1=4$}
	\loigiai{\begin{itemchoice}
			\itemch \textbf{Sai.}\\
			Mốt của mẫu số liệu là $M_o=6$.
			\itemch \textbf{Đúng.}\\
			 Số trung bình của mẫu số liệu là $\overline{x}=\dfrac{4\cdot2+5+ 6\cdot4+8+9}{9}=6$.
			\itemch \textbf{Đúng.}\\
			 Sắp xếp lại mẫu số liệu, ta được $4;4;5;6;6;6;6;8;9$.\\
			Vì $n=9$ nên trung vị ở vị trí $\dfrac{n+1}{2}=\dfrac{9+1}{2}=5$.\\
			Vậy $M_e=6$.
			\itemch \textbf{Sai.}\\
			 Tứ phân vị thứ nhất là trung vị của $4;4;5;6$.\\
			Vì $n_1=4$ nên $Q_1=\dfrac{4+5}{2}=4{,}5$.
	\end{itemchoice}}
\end{ex}
\begin{ex}%[0D6H4-4]
	Cho mẫu số liệu điểm thi khảo sát của $10$ học sinh như sau 
	$$10 \quad 5 \quad 8 \quad 8 \quad 8 \quad 7 \quad 7 \quad 7 \quad 9 \quad 6$$
	\choiceTF
	{\True Mẫu số liệu có hai mốt là $8$ và $7$}
	{\True Số trung bình của mẫu số liệu là $\overline{x}=7{,}5$}
	{\True Tứ phân vị của mẫu số liệu là $Q_1=7$, $Q_2=7{,}5$, $Q_3=8$}
	{Phương sai của mẫu số liệu trên (làm tròn đến hàng phần trăm) bằng $2{,}06$}
	\loigiai{
		\begin{itemchoice}
			\itemch \textbf{Đúng}. Mẫu số liệu có điểm $8$ và điểm $7$ có tần số cao nhất nên mẫu số liệu có hai mốt là $8$ và $7$.
			\itemch \textbf{Đúng}. Ta có số trung bình của mẫu số liệu là $\overline{x}=\dfrac{5+6+7+7+7+8+8+8+9+10}{10}=7{,}5$.
			\itemch \textbf{Đúng}.\\
			Sắp xếp lại mẫu số liệu ta được $$5 \quad 6 \quad 7 \quad 7 \quad 7 \quad 8 \quad 8 \quad 8 \quad 9 \quad 10$$
			Tứ phân vị thứ hai $Q_2=\dfrac{1}{2}\cdot(7+8)=7{,}5$.\\
			Tứ phân vị thứ nhất $Q_1=x_3=7$.\\
			Tứ phân vị thứ ba $Q_3=x_8=8$.\\
			\itemch \textbf{Sai}.\\
			Ta có số trung bình của mẫu số liệu là $\overline{x}=\dfrac{5+6+7+7+7+8+8+8+9+10}{10}=7{,}5$.\\
			Phương sai $$s^2=\dfrac{(5-7{,}5)^2+(6-7{,}5)^2+3\cdot(7-7{,}5)^2+3\cdot(8-7{,}5)^2+(9-7{,}5)^2+(10-7{,}5)^2}{10}=1{,}85.$$
		\end{itemchoice}
	}
\end{ex}
\Closesolutionfile{ans}

%\inputansbox[2]{2}{ans/answer.tex}

\begin{center}
	\textbf{PHẦN 3 - CÂU TRẮC NGHIỆM TRẢ LỜI NGẮN}
\end{center}
\setcounter{ex}{0}
\Opensolutionfile{ans}[ans-KQ-ONTAPCHUONG-DE3]
\begin{ex}%[0D6N1-1]
	Trong một cuộc điều tra dân số, người ta viết dân số của một tỉnh là: $3\ 574\ 625$ người $\pm 50\,000$ người. Sai số tương đối của số gần đúng này không vượt quá bao nhiêu phần trăm? ( kết quả làm tròn đến hàng phần mười)\\
	\shortans[oly]{$1{,}4$}
	\loigiai{
		Sai số tương đối của số gần đúng là $\delta\leq\dfrac{d}{|d|}=\dfrac{50\,000}{3\,574\,625} \approx 0{,}013987 \approx 1{,}4\%$.
	}
\end{ex}

\begin{ex}%[0D6H3-4]
	Số ô tô đi qua một cây cầu trong một tuần đếm được như sau
	\[80; \quad 73; \quad 71; \quad 79; \quad 85; \quad 67 \quad 91.\]
	Gọi $x$, $y$, $z$ lần lượt là giá trị trung bình, tứ phân vị thứ nhất và tứ phân vị thứ hai. Tính giá trị $x+y+z$.
	\shortans[oly]{$228$}
	\loigiai{
		Giá trị trung bình là $x=\dfrac{80+73+71+79+85+67+91}{7}=78$.\\
		Sắp xếp mẫu số liệu theo thứ tự không giảm, ta được \[67; \quad 71; \quad 73; \quad 79; \quad 80; \quad 85; \quad 91.\]
		Ta có $\dfrac{n}{2}=\dfrac{7}{2}=3{,}5$.\\
		Tứ phân vị thứ hai là $z=x_4=79$.\\
		Trung vị của mẫu số liệu $67$; $71$; $73$ chính là tứ phân vị thứ nhất.\\
		Tứ phân vị thứ nhất là $y=71$.\\
		Vậy $x+y+z=78+71+79=228$.
	}
\end{ex}
\begin{ex}%[0D6H4-4]
	Điểm thi của lớp 10C của một trường Trung học Phổ Thông được trình bày ở bảng phân bố tần số sau
	\begin{center}
		\begin{tabular}{|c|c|c|c|c|c|c|c|}
			\hline Điểm thi	& $5$ & $6$ & $7$ & $8$ & $9$ & $10$ & \\
			\hline Tần số	& $7$ & $5$ & $10$ & $12$ & $4$ & $2$ & $n=40$\\
			\hline
		\end{tabular}
	\end{center}
	Tìm phương sai của bảng phân bố tần số đã cho. (làm tròn đến hàng phần trăm)
	\shortans[oly]{$1{,}94$}
	\loigiai{
		Trong dãy số liệu về điểm thi của lớp 10C ta có
		\[ \overline{x} = \dfrac{1}{n}\left( n_1x_1+n_2x_2+ \cdots + n_6x_6\right) =\dfrac{1}{40}\left(7\cdot 5+5\cdot 6+10\cdot 7+12\cdot 8+4\cdot 9+2\cdot 10\right)=7{,}175.\]
		Suy ra phương sai là
		\begin{align*}
		s^2 &= \dfrac{1}{n}\left[ n_1\left(x_1-\overline{x}\right)^2 + n_2\left(x_2-\overline{x}\right)^2 + \cdots + n_6\left(x_6-\overline{x}\right)^2\right]
		\\ &=  \dfrac{1}{40}\left[ 7\left(5-7{,}175\right)^2 + 5\left(6-7{,}175\right)^2 + 10\left(7-7{,}175\right)^2 + 12\left(8-7{,}175\right)^2 \right.
		\\ & \quad \left. + 4\left(9-7{,}175\right)^2 + 2\left(10-7{,}175\right)^2 \right]
		\\ & \approx 1{,}94.
		\end{align*}
	}
\end{ex}

\begin{ex}%[0D6H4-3]
	Bà Hoa kiểm tra chiều cao của $30$ cây trồng trong vườn và ghi lại bảng số liệu như sau
	\begin{center}
		\begin{tabular}{|l|c|c|c|c|c|c|}
			\hline
			Chiều cao (m) & $0{,}5$ & $0{,}8$ & $1{,}0$ & $1{,}2$ & $1{,}5$ & $2{,}0$ \\
			\hline
			Số cây & $2$ & $3$ & $15$ & $6$ & $3$ & $1$ \\
			\hline
		\end{tabular}
	\end{center}
	Gọi $S$ là tập hợp các giá trị bất thường của mẫu số liệu trên. Hãy xác định tổng tất cả các phần tử của tập hợp $S$.
		\shortans[oly]{$2{,}5$}
	\loigiai{
		Tập dữ liệu có $30$ số liệu, nên tứ phân vị thứ nhất là số liệu thứ $8$ và tứ phân vị thứ $3$ là số liệu thứ $23$.\\
		Dựa vào bảng số liệu, tứ phân vị tứ nhất là $Q_{1} = 1{,}0$ và tứ phân vị thứ ba là $Q_{3} = 1{,}2$.\\
		Ta có khoảng tứ phân vị là $\Delta_Q = Q_{3} - Q_{1}= 1{,}2 - 1{,}0 = 0{,}2$.\\
		Ta có $Q_{1} - 1,5\cdot\Delta_Q  = 1{,}0 - 1{,}5\cdot 0{,}2= 0{,}7$ và $Q_{3} + 1{,}5\cdot\Delta_Q = 1{,}2 + 1{,}5\cdot 0{,}2= 1{,}5$. \\
		Ta thấy có $2$ số liệu $0{,}5$ nhỏ hơn $0{,}7$ và có $1$ số liệu $2{,}0$ lớn hơn $1{,}5$. \\
		Vậy có các giá trị $0{,}5$ và $2{,}0$ là hai giá trị bất thường, tức là $S=\left\{0{,}5;2{,}0\right\}$.\\
		Suy ra tổng các giá trị của $S$ là $0{,}5 + 2{,}0 = 2{,}5$.
	}
\end{ex}
\Closesolutionfile{ans}

\begin{center}
	\textbf{PHẦN 4 - TỰ LUẬN}
\end{center}
\setcounter{ex}{0}
%Từ ngân hàng. Câu 476. File gốc: C:\Users\X1 gen7\Desktop\SP-DU-AN\NGAN-HANG-24-25\EXTRACT\0D6N3-2_TL.tex
\begin{ex}%[0D6N3-2]
	Cho bảng số liệu về điểm kiểm tra thường xuyên của 12 học sinh tổ 1 lớp 10C1:
	\begin{center}
		\begin{tabular}{p{1cm}p{1cm}p{1cm}p{1cm}p{1cm}p{1cm}p{1cm}p{1cm}p{1cm}p{1cm}p{1cm}p{1cm}}
			9 & 9 & 8 & 10 & 8 & 7 & 9 & 5 & 9 & 10 & 6 & 8
		\end{tabular} 
	\end{center}
	Giả sử số điểm kiểm tra trung bình của các học sinh tổ 1 lớp 10C1 là $\dfrac{a}{b}$ (với $\dfrac{a}{b}$ tối giản). Tính $T=a-3 b$.
	
	\loigiai{Điểm trung bình của các học sinh tổ 1 lớp 10C1 là \[\overline{x}=\dfrac{9+9+8+10+8+7+9+5+9+10+6+8}{12}=\dfrac{49}{6}.\]
		Suy ra $ a=49 $ và $ b=6 $. Do đó $ T=31 $.}
\end{ex}
\begin{ex}
	Điểm học kì một của một học sinh được cho bởi  bảng số liệu sau (Đơn vị: điểm)
	\begin{center}
		\begin{tabular}{|c|c|c|c|c|c|c|c|c|}
			\hline
			5& 6 &6&7& 7 &8 &8& 8,5&9\\
			\hline
		\end{tabular}
	\end{center}
	\begin{listEX}
		\item Tìm số trung vị của mẫu số liệu trên.
		\item Tìm các tứ phân vị của mẫu số liệu trên.
	\end{listEX}
	\loigiai{
		\begin{listEX}
			\item Bảng trên có $9$ số liệu và đã xếp theo thự tự không giảm. Ta có $n=9$ là số lẻ. Áp dụng cách tìm số trung vị, ta được số trung vị là số thứ $\dfrac{n+1}{2}$ tương ứng là số thứ năm trong dãy. Do đó số trung vị là 
			$M_e = 7$.
			\item Ta có $Q_2=7$. 
			\begin{itemize}
				\item [$\bullet$] Xét nửa số liệu bên trái $Q_2$ là
				$5 \quad 6 \quad 6 \quad 7$.\\
				Trung vị của nửa số liệu này là $Q_1=\dfrac{6+6}{2}=6$.
				\item [$\bullet$] Xét nửa số liệu bên phải $Q_2$ là
				$8 \quad 8 \quad 8{,}5 \quad 9$. \\
				Trung vị của nửa số liệu này là $Q_3=\dfrac{8+8{,}5}{2}=8{,}25$.
			\end{itemize}
			Vậy, các tứ phân vị của mẫu số liệu trên là $Q_1=6$, $Q_2=7$, $Q_3=8{,}25$.
		\end{listEX} 
	}
\end{ex}

\begin{ex}%[0D6H4-4]
	Kết quả dự báo nhiệt độ cao nhất trong $11$ ngày cuối tháng $12$ năm $2023$ ở một tỉnh miền núi phía Bắc thu được kết quả sau
	\begin{longtable}{|c|c|c|c|c|c|c|c|c|}
		\hline
		Nhiệt độ ($^\circ$C) & $14$ & $16$ & $17$ & $18$ & $19$ & $20$ & $21$ & $22$\\
		\hline
		Tần số & $1$ & $1$ & $1$ & $2$ & $1$ & $2$ & $2$ & $1$\\
		\hline
	\end{longtable}
	\begin{enumerate}
		\item Hãy tính khoảng biến thiên và độ lệch chuẩn của mẫu số liệu trên.
		\item Hãy tìm các giá trị bất thường (nếu có) của mẫu số liệu trên.
	\end{enumerate}
	\loigiai{
		\begin{enumerate}
			\item Khoảng biến thiên $R=22-14=8$.\\
			Số trung bình của mẫu số liệu
			$$\overline{x}=\dfrac{14\cdot 1+16\cdot 1+17\cdot 1+18\cdot 2+19\cdot 1+20\cdot 2+21\cdot 2+22\cdot 1}{11}=\dfrac{206}{11}.$$
			Phương sai của mẫu số liệu
			$$S^2=\dfrac{14^2\cdot 1+16^2\cdot 1+17^2\cdot 1+18^2\cdot 2+19^2\cdot 1+20^2\cdot 2+21^2\cdot 2+22^2\cdot 1}{10}-\left(\dfrac{206}{11}\right)^2=\dfrac{640}{121}.$$
			Độ lệch chuẩn của mẫu số liệu
			$$S=\sqrt{\dfrac{640}{121}}=\dfrac{8\sqrt{10}}{11}.$$
			\item Xét tứ phân vị thứ nhất
			$Q_1=x_3=17$.\\
			Tứ phân vị thứ ba $Q_3=x_9=21$.\\
			Khoảng tứ phân vị $\Delta_Q=Q_3-Q_1=4$.\\
			Xét $Q_1-1{,}5\Delta_Q=11$ và $Q_3+1{,}5\Delta_Q=27$.\\
			Do đó mẫu số liệu đã cho không có giá trị ngoại lệ.
		\end{enumerate}
	}
\end{ex}
