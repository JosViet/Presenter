\newpage
\section{NHỊ THỨC NEWTON}
\subsection{LÝ THUYẾT CẦN NHỚ}
\subsubsection{Công thức khai triển nhị thức Newton}
\indam{Định nghĩa:}
\begin{boxdn}
	\begin{eqnarray*}
		(a+b)^4 &=&\mathrm{C}_4^0a^4+\mathrm{C}_4^1a^3b+\mathrm{C}_4^2a^2b^2+\mathrm{C}_4^3ab^3+\mathrm{C}_4^4b^4 \\
		&=&a^4+4 a^3b+6 a^2 b^2+4 a b^3+b^4; \\
		(a+b)^5&=&\mathrm{C}_5^0a^5+\mathrm{C}_5^1a^4b+\mathrm{C}_5^2a^3b^2+\mathrm{C}_5^3a^2b^3+\mathrm{C}_5^4ab^4+\mathrm{C}_5^5b^5 \\
		&=&a^5+5a^4 b+10a^3b^2+10a^2b^3+5ab^4+b^5.
	\end{eqnarray*}
	Hai công thức trên gọi là công thức nhị thức Newton (gọi tắt là nhị thức Newton) $(a+b)^n$ ứng với $n=4$ và $n=5$.
\end{boxdn}
\subsubsection{Tam giác Pascal}
	Các hệ số trong khai triển nhị thức Newton $(a+b)^n$ với $n=0;\,1;\,2;\,3;\ldots$ được viết thành từng hàng và xếp thành bảng số như bên. Bảng số này có quy luật: số đầu tiên và số cuối cùng của mỗi hàng đều là 1; tổng của hai số liên tiếp cùng hàng bằng số của hàng kế dưới ở vị trí giữa hai số đó.\\
	Bảng số trên được gọi là \textbf{\textit{tam giác Pascal}} (đặt theo tên của nhà toán học, vật lí học, triết học người Pháp Blaise Pascal, 1623 - 1662).
	\begin{center}
		\begin{tabular}{llccccccccccccccc}
			$n=0$&&&&&&&&&$\color{blue}\circled{1}$&&&&&&&\\
			$n=1$&&&&&&&&$\color{red}\circled{1}$&&$\color{red}\circled{1}$&&&&&&\\
			$n=2$&&&&&&&$\color{blue}\circled{1}$&&$\color{blue}\circled{2}$&&$\color{blue}\circled{1}$&&&&&\\
			$n=3$&&&&&&$\color{red}\circled{1}$&&$\color{red}\circled{3}$&&$\color{red}\circled{3}$&&$\color{red}\circled{1}$&&&&\\
			$n=4$&&&&&$\color{blue}\circled{1}$&&$\color{blue}\circled{4}$&&$\color{blue}\circled{6}$&&$\color{blue}\circled{4}$&&$\color{blue}\circled{1}$&&&\\
			$n=5$&&&&$\color{red}\circled{1}$&&$\color{red}\circled{5}$&&$\color{red}\circled{10}$&&$\color{red}\circled{10}$&&$\color{red}\circled{5}$&&$\color{red}\circled{1}$&&\\
			$n=6$&&&$\color{blue}\circled{1}$&&$\color{blue}\circled{6}$&&$\color{blue}\circled{15}$&&$\color{blue}\circled{20}$&&$\color{blue}\circled{15}$&&$\color{blue}\circled{6}$&&$\color{blue}\circled{1}$&\\
			$n=7$&&$\color{red}\circled{1}$&&$\color{red}\circled{7}$&&$\color{red}\circled{21}$&&$\color{red}\circled{35}$&&$\color{red}\circled{35}$&&$\color{red}\circled{21}$&&$\color{red}\circled{7}$&&$\color{red}\circled{1}$\\
		\end{tabular}
	\end{center}
%-------------------------------------------------------------------------------------------------------------
\subsection{PHÂN LOẠI VÀ PHƯƠNG PHÁP GIẢI TOÁN}
\begin{dang}{Khai triển nhị thức Newton $(a+b)^n$ với $n=4$, $n=5$}
\end{dang}

\begin{vd}%[0D8N3-2]%[Hieu Phan]
	Sử dụng công thức nhị thức Newton, hãy khai triển các biểu thức sau
	\begin{enumEX}{2}
		\item $(x+3)^4$;
		\item $(1-x)^5$.
	\end{enumEX}
	\loigiai{
		\begin{enumerate}
			\item Theo công thức nhị thức Newton, ta có
			\begin{eqnarray*}
				(x+3)^4 &=&1 \cdot x^4+4 \cdot x^3 \cdot 3+6 \cdot x^2 \cdot 3^2+4 \cdot x \cdot 3^3+1 \cdot 3^4 \\
				&=&x^4+4 \cdot 3 \cdot x^3+6 \cdot 9 \cdot x^2+4 \cdot 27 \cdot x+81 \\
				&=&x^4+12 x^3+54 x^2+108 x+81.
			\end{eqnarray*}
			\item Theo công thức nhị thức Newton, ta có
			\begin{eqnarray*}
				(1-x)^5&=&1+5 \cdot(-x)+10 \cdot(-x)^2+10 \cdot(-x)^3+5 \cdot(-x)^4+1 \cdot(-x)^5\\
				&=&1-5 x+10 x^2-10 x^3+5 x^4-x^5.
			\end{eqnarray*}
		\end{enumerate}
	}
\end{vd}
\begin{vd}%[0D8N3-2]%[Hieu Phan]
	Sử dụng công thức nhị thức Newton, hãy khai triển các biểu thức sau
	\begin{enumEX}{2}
		\item $(x-2y)^4$;
		\item $(3x-y)^5$.
	\end{enumEX}
	\loigiai{
		\begin{enumerate}
		
			\item Theo công thức nhị thức Newton, ta có
			\begin{eqnarray*}
				(x-2y)^4&=&[x+(-2y)]^4=x^4+4x^3(-2y)+6x^2(-2y)^2+4x(-2y)^3+(-2y)^4\\
				&=&x^4-8x^3y+24x^2y^2-32xy^3+16y^4.
			\end{eqnarray*}
			\item Theo công thức nhị thức Newton, ta có
			\begin{eqnarray*}
				(3x-y)^5&=&[3x+(-y)]^5\\
				&=&(3x)^5+5(3x)^4(-y)+10(3x)^3(-y)^2+10(3x)^2(-y)^3+5(3x)(-y)^4+(-y)^5\\
				&=&243x^5-405x^4y+270x^3y^2-90x^2y^3+15xy^4-y^5.
			\end{eqnarray*}
		\end{enumerate}
	}
\end{vd}
\begin{vd}%[0D8N3-2]%[Hieu Phan]
	Khai triển và rút gọn các biểu thức sau
	\begin{listEX}[2]
		\item $\left(1-\sqrt{3}\right)^5$;
		\item $\left(1+\sqrt{2}\right)^5+\left(1-\sqrt{2}\right)^5$.
	\end{listEX}
	\loigiai{
		\begin{listEX}
			\item
			Theo công thức nhị thức Newton, ta có
			\begin{eqnarray*}
				\left(1-\sqrt{3}\right)^5&=& 1^5+5\cdot 1^4\cdot(-\sqrt{3})+10\cdot 1^3\cdot(-\sqrt{3})^2+10\cdot 1^2\cdot(-\sqrt{3})^3+5\cdot 1\cdot(-\sqrt{3})^4+1\cdot(-\sqrt{3})^5\\
				&=& 1^5+5\cdot 1^4\cdot\left(-\sqrt{3}\right)+10\cdot 1^3\cdot 3+10\cdot 1^2\cdot(-3\sqrt{3})+5\cdot 1\cdot 9+1\cdot(-9\sqrt{3})\\
				&=& 76-44\sqrt{3}.
			\end{eqnarray*}
			\item
			Áp dụng công thức nhị thức Newton, ta có
			\begin{eqnarray*}
				(1+\sqrt{2})^5&=&1+5 \cdot \sqrt{2}+10 \cdot\left(\sqrt{2}\right)^2+10 \cdot\left(\sqrt{2}\right)^3+5 \cdot\left(\sqrt{2}\right)^4+1 \cdot\left(\sqrt{2}\right)^5 \\
				(1-\sqrt{2})^5&=&1+5 \cdot\left(-\sqrt{2}\right)+10 \cdot\left(-\sqrt{2}\right)^2+10 \cdot\left(-\sqrt{2}\right)^3+5 \cdot\left(-\sqrt{2}\right)^4+1 \cdot\left(-\sqrt{2}\right)^5.
			\end{eqnarray*}
			Từ đó,
			$$ 	\left(1+\sqrt{2}\right)^5+(1-\sqrt{2})^5=2\left[1+10 \cdot\left(\sqrt{2}\right)^2+5 \cdot\left(\sqrt{2}\right)^4\right]=2(1+10 \cdot 2+5 \cdot 4)=82.$$
		\end{listEX}
	}
\end{vd}

\begin{dang}{Xác định hệ số của $x^k$ trong khai triển $(ax+b)^n$ với $n=4$, $n=5$}
\end{dang}
\setcounter{vd}{0}
\begin{vd}%[0D8H3-4]%[Hieu Phan]
	Xác định hệ số của $x^2$ trong khai triển biểu thức $(2x-3)^4$.
	\loigiai{
		Ta có
		\begin{align*}
			\left(2x-3\right)^4&=\mathrm{C}_4^0\left(2x\right)^4+\mathrm{C}_4^1\left(2x\right)^3\left(-3\right)+\mathrm{C}_4^2\left(2x\right)^2\left(-3\right)^2+\mathrm{C}_4^32x\left(-3\right)^3+\mathrm{C}_4^4\left(-3\right)^4\\
			&=16x^4-96x^3+216x^2-216x+81.
		\end{align*}
		Vậy hệ số của $x^2$ trong khai triển biểu thức $(2x-3)^4$ là $216$.
	}
	\end{vd}
	\begin{vd}%[0D8H3-3]%[Hieu Phan]
		Tìm hệ số của $x^4$ trong khai triển biểu thức $(2+3x)^5$. 
		\loigiai{
			Ta có
			$$\begin{aligned}
				\left(2+3x\right)^5
				&=\mathrm{C}_5^0\cdot2^5+\mathrm{C}_5^1\cdot2^4\cdot(3x)+\cdots+\mathrm{C}_5^4\cdot2^1\cdot(3x)^4+\mathrm{C}_5^5\cdot(3x)^5 \\
				&=243x^{5} + 810x^{4} + 1080x^{3} + 720x^{2} + 240x + 32.
			\end{aligned}$$
			Vậy hệ số của $x^4$ trong khai triển $\left(2+3x\right)^5$ là $810$.
		}
	\end{vd}
\begin{vd}%[0D8H3-4]
	Tìm hệ số của $x^3y^2$ trong khai triển thành đa thức của biểu thức $\left(x-2y\right)^5$.
	\loigiai{
		Ta có
		\begin{eqnarray*}
		\left(x-2y\right)^5&=&\mathrm{C}_5^0x^5+\mathrm{C}_5^1x^4\cdot(-2y)+\mathrm{C}_5^2x^3\cdot(-2y)^2+\mathrm{C}_5^3x^2\cdot(-2y)^3+\mathrm{C}_5^4x\cdot(-2y)^4+\mathrm{C}_5^5\cdot(-2y)^5\\
		&=&x^5-10x^4y+40x^3y^2-80x^2y^3+80x{y^4}-32y^5.
		\end{eqnarray*} 
		Vậy hệ số của $x^3y^2$ là $40$.
	}
\end{vd}
\begin{dang}{Chứng minh đẳng thức tổ hợp, tính tổng bằng cách sử dụng khai triển nhị thức Newton}
\end{dang}
\setcounter{vd}{0}
\begin{vd}%[0D8H3-5]%[Hieu Phan]
	Sử dụng công thức nhị thức Newton, chứng tỏ rằng:
	\begin{enumEX}{2}
		\item $\mathrm{C}_4^0+2\mathrm{C}_4^1+2^2\mathrm{C}_4^2+2^3\mathrm{C}_4^3+2^4\mathrm{C}_4^4=81$;
		\item $\mathrm{C}_4^0-2\mathrm{C}_4^1+2^2\mathrm{C}_4^2-2^3\mathrm{C}_4^3+2^4\mathrm{C}_4^4=1$.
	\end{enumEX}
	\loigiai{
		Sử dụng công thức nhị thức Newton, ta có
		$$
		(1+x)^4=\mathrm{C}_4^0+\mathrm{C}_4^1x+\mathrm{C}_4^2x^2+\mathrm{C}_4^3x^3+\mathrm{C}_4^4x^4 .
		$$
		\begin{enumerate}
			\item Thay $x=2$ vào công thức trên, nhận được
			$$
			\mathrm{C}_4^0+2\mathrm{C}_4^1+2^2\mathrm{C}_4^2+2^3\mathrm{C}_4^3+2^4\mathrm{C}_4^4=(1+2)^4=3^4=81.
			$$
			\item Thay $x=-2$ vào công thức trên, nhận được
			$$
			\mathrm{C}_4^0-2\mathrm{C}_4^1+2^2\mathrm{C}_4^2-2^3\mathrm{C}_4^3+2^4\mathrm{C}_4^4=(1-2)^4=(-1)^4=1.
			$$
		\end{enumerate}
	}
\end{vd}
\begin{vd}%[0D8H3-5]%[Hieu Phan]
	Cho $\left(2x-\dfrac{1}{3}\right)^4=a_0+a_1x+a_2x^2+a_3x^3+a_4x^4$. Tính \begin{enumEX}{2}
		\item $a_2$;
		\item $a_0+a_1+a_2+a_3+a_4$.
	\end{enumEX}
	\loigiai{
		\begin{enumEX}{1}
			\item Ta có số hạng chứa $x^2$ trong khai triển biểu thức $\left(2x-\dfrac{1}{3}\right)^4$ là $6\cdot(2x)^2\cdot\left(-\dfrac{1}{3}\right)^2=\dfrac{8}{3}x^2$.\\
			Suy ra $a_2=\dfrac{8}{3}$.
			\item Ta xét $f(x)=\left(2x-\dfrac{1}{3}\right)^4=a_0+a_1x+a_2x^2+a_3x^3+a_4x^4$. \\
			Khi đó $f(1)=a_0+a_1+a_2+a_3+a_4=\left(2\cdot 1-\dfrac{1}{3}\right)^4=\dfrac{625}{81}$.\\
			Vậy $a_0+a_1+a_2+a_3+a_4=\dfrac{625}{81}$.
		\end{enumEX}
	}
\end{vd}
\begin{vd}%[0D8H3-5]%[Hieu Phan]
Cho $\left(\dfrac{3}{5}x+\dfrac{1}{2}\right)^5=a_0+a_1x+a_2x^2+a_3x^3+a_4x^4+a_5x^5$. Tính \begin{enumEX}{2}
	\item $a_3$;
	\item $a_0+a_1+a_2+a_3+a_4+a_5$.
\end{enumEX}
\loigiai{
	\begin{enumEX}{1}
		\item Số hạng chứa $x^3$ trong khai triển biểu thức $\left(\dfrac{3}{5}x+\dfrac{1}{2}\right)^5$ là $10\cdot\left(\dfrac{3}{5}x\right)^3\cdot\left(\dfrac{1}{2}\right)^2=\dfrac{27}{50}x^3$.\\
		Vậy $a_3=\dfrac{27}{50}$.
		\item Xét $f(x)=\left(\dfrac{3}{5}x+\dfrac{1}{2}\right)^5=a_0+a_1x+a_2x^2+a_3x^3+a_4x^4+a_5x^5$.\\
		Khi đó $a_0+a_1+a_2+a_3+a_4+a_5=f(1)=\left(\dfrac{3}{5}\cdot 1+\dfrac{1}{2}\right)^5=\dfrac{161051}{100000}$.
	\end{enumEX}
}
\end{vd}
%-----------------------------------------------------------------------------
\subsection{Bài tập rèn luyện}
\ind{PHẦN I.} \inden{Câu trắc nghiệm nhiều phương án lựa chọn. Mỗi câu hỏi học sinh chỉ chọn một phương án.}\\
\setcounter{ex}{0}
\Opensolutionfile{ans}[ans/2D1-Bai1-TN]
\begin{ex}%[0D8N3-2]%[Hieu Phan]
	Trong các phát biểu sau, phát biểu nào \textbf{sai}?
	\choice
	{$(a+b)^4=a^4+4a^3b+6a^2b^2+4ab^3+b^4$}
	{$(a-b)^4=a^4-4a^3b+6a^2b^2-4ab^3+b^4$}
	{$(a+b)^4=b^4+4b^3a+6b^2a^2+4ba^3+a^4$}
	{\True $(a+b)^4=a^4+b^4$}
	\loigiai{
		Theo công thức nhị thức Newton thì khẳng định $(a+b)^4=a^4+b^4$ là sai.
	}
\end{ex}
\begin{ex}[Trích đề thi HKII-NGUYỄN THÁI BÌNH-Năm học 23-24]%[0D8N3-2]
	Khai triển nhị thức $(x+3y)^4$ thu được kết quả là
	\choice
	{$x^4-4x^3 y+18x^2y^2-36 xy^3+27y^4$}
	{\True $x^4+12x^3y+54x^2y^2+108xy^3+81y^4$}
	{$x^4+4x^3y+1 x^2y^2+36xy^3+27y^4$}
	{$x^4-12x^3y+54x^2y^2-108xy^3+81y^4$}
	\loigiai{
		Ta có $(x+3y)^4=x^4+12x^3y+54x^2y^2+108xy^3+81y^4$.
	}
\end{ex}
\begin{ex}%[0D8N3-2]%[Hieu Phan]
	Trong các phát biểu sau, phát biểu nào đúng?
	\choice
	{\True $(a+b)^5=a^5+5a^4b+10a^3b^2+10a^2b^3+5ab^4+b^5$}
	{$(a-b)^5=a^5-5a^4b+10a^3b^2+10a^2b^3+5ab^4-b^5$}
	{$(a+b)^5=a^5+b^5$}
	{$(a-b)^5=a^5-b^5$}
	\loigiai{
		Theo công thức nhị thức Newton, ta có $(a+b)^5=a^5+5a^4b+10a^3b^2+10a^2b^3+5ab^4+b^5$.
	}
\end{ex}
\begin{ex}%[0D8H3-4]%[Hieu Phan]
	Hệ số của $x^3$ trong khai triển biểu thức $(2x-1)^4$ là
	\choice
	{$32$}
	{\True $-32$}
	{$8$}
	{$-8$}
	\loigiai{
		Số hạng chứa $x^3$ trong khai triển biểu thức $(2x-1)^4$ là $4\cdot (2x)^3\cdot(-1)=-32x^3$.\\
		Vậy hệ số của $x^3$ trong khai triển biểu thức $(2x-1)^4$ là $-32$.
	}
\end{ex}
%%==========Câu 6
\begin{ex}%[0D8H3-4]%[Hieu Phan]
	Hệ số của $x$ trong khai triển biểu thức $(x-2)^5$ là
	\choice
	{$32$}
	{$-32$}
	{\True $80$}
	{$-80$}
	\loigiai{
		Số hạng chứa $x$ trong khai triển biểu thức $(x-2)^5$ là $5\cdot x\cdot(-2)^4=80x$.\\
		Vậy hệ số của $x$ trong khai triển biểu thức $(x-2)^5$ là $80$.
	}
\end{ex}
\begin{ex}[Trích đề thi GKII-THPT Chuyên Lê Quý Đôn-Năm học 23-24]%[0D8H3-2]%[Hieu Phan]
Khai triển Newton biểu thức $(x+2)^5$ ta được	
\choice
{\True $(x+2)^5=\mathrm{C}_5^0 x^5 + 2\mathrm{C}_5^1x^4+ 2^2\mathrm{C}_5^2x^3+2^3\mathrm{C}_5^3x^2+2^4\mathrm{C}_5^4x+2^5\mathrm{C}_5^5$}
{$(x+2)^5=\mathrm{C}_5^0 x^5 - 2\mathrm{C}_5^1x^4+ 2^2\mathrm{C}_5^2x^3-2^3\mathrm{C}_5^3x^2+2^4\mathrm{C}_5^4x-2^5\mathrm{C}_5^5$}
{$(x+2)^5=\mathrm{C}_5^0 x^5 + 2\mathrm{C}_5^1x^4+ 2^2\mathrm{C}_5^2x^3-2^3\mathrm{C}_5^3x^2+2^4\mathrm{C}_5^4x-2^5\mathrm{C}_5^5$}
{$(x+2)^5=2\mathrm{C}_5^0 x^5 + 2^2\mathrm{C}_5^1x^4+ 2^3\mathrm{C}_5^2x^3+2^4\mathrm{C}_5^3x^2+2^5\mathrm{C}_5^4x+2^6\mathrm{C}_5^5$}
\loigiai{Khai triển Newton biểu thức $(x+2)^5$ ta được	$$(x+2)^5=\mathrm{C}_5^0x^5 + 2\mathrm{C}_5^1x^4+ 2^2\mathrm{C}_5^2x^3+2^3\mathrm{C}_5^3x^2+2^4\mathrm{C}_5^4x+2^5\mathrm{C}_5^5.$$ }
\end{ex}
%%==========Câu 7
\begin{ex}%[0D8N3-2]%[Hieu Phan]
	Đa thức $P(x)=32x^5-80x^4+80x^3-40x^2+10x-1$ là khai triển của nhị thức nào dưới đây?
	\choice
	{\True ${\left(2x-1\right)}^5$}
	{${\left(x-1\right)}^5$}
	{${\left(1-2x\right)}^5$}
	{${\left(1+2x\right)}^5$}
	\loigiai{
		Nhận thấy $P(x)$ có dấu đan xen nên loại đáp án ${\left(1+2x\right)}^5$.\\
		Hệ số của $x^5$ bằng $32$ nên loại đáp án ${\left(x-1\right)}^5$ và còn lại hai đáp án ${\left(1-2x\right)}^5$ và ${\left(2x-1\right)}^5$ thì chỉ có ${\left(2x-1\right)}^5$ phù hợp (vì khai triển số hạng đầu tiên của đáp án ${\left(2x-1\right)}^5$ là $32x^5.$)
	}
\end{ex}
\begin{ex}[Trích đề thi GKII-THPT Chuyên Nguyễn Đình Chiểu-Năm học 24-25]%[0D8N3-2]%[Hieu Phan]
	Đa thức $P(x)=x^5+5x^4+10x^3+10x^2+5x+1$ là khai triển của nhị thức nào dưới đây?
	\choice
	{$(1-x)^5$}
	{$(1+2x)^5$}
	{$(x+2)^5$}
	{\True $(x+1)^5$}
	\loigiai{
		Ta có $P(x)=x^5+5x^4+10x^3+10x^2+5x+1=(x+1)^5$.
	}
\end{ex}
\begin{ex}[Trích đề thi HKII-THPT Trần Phú-Năm học 23-24]%[0D8N3-2]%[Hieu Phan]
	Trong khai triển theo công thức nhị thức Newton của $(a-b)^5$ có bao nhiêu số hạng?
	\choice
	{$5$}
	{$3$}
	{$4$}
	{\True $6$}
	\loigiai{
		Trong khai triển theo công thức nhị thức Newton của $(a-b)^5$ có $6$ số hạng.
	}
\end{ex}
\begin{ex}[Trích đề thi GKII-THPT Nam Lý-Năm học 23-24]%[0D8H3-2]%[Hieu Phan]
	Khai triển nhị thức Newton $(2x-1)^5$ bằng
	\choice
	{$32x^5+80x^4-80x^3+40x^2-10x+1$}
	{$32x^5+80x^4+80x^3+40x^2+10x+1$}
	{\True $32x^5-80x^4+80x^3-40x^2+10x-1$}
	{$2x^5-80x^4+80x^3-40x^2+10x-1$}
	\loigiai{
		Ta có 
		\begin{eqnarray*}
			(2x-1)^5&=&\mathrm{C}^0_5\cdot (2x)^5\cdot (-1)^0+\mathrm{C}^1_5\cdot (2x)^4\cdot (-1)^1+\cdots+\mathrm{C}^5_5\cdot (2x)^0\cdot (-1)^5\\
			&=&32x^5-80x^4+80x^3-40x^2+10x-1.
		\end{eqnarray*} 
	}
\end{ex}
\begin{ex}[Trích đề thi GKII-THPT Trần Phú-Năm học 23-24]%[0D8H3-2]%[Hieu Phan]
	Số hạng không chứa $x$ trong khai triển nhị thức Newton của $\left(x^2-\dfrac{1}{x^2}\right)^4$ là 
	\choice
	{$4 $}
	{$0 $}
	{\True $6 $}
	{$-4$}
	\loigiai{
		Theo khai triển nhị thức Newton ta có $\left(x^2-\dfrac{1}{x^2}\right)^4=x^8-4x^4+6-\dfrac{4}{x^4}+\dfrac{1}{x^8}.$\\
		Do đó số hạng không chứa $x$ là $6$.
	}
\end{ex}
\begin{ex}[Trích đề thi GKII-THPT Trần Phú-Năm học 23-24]%[0D8H3-2]%[Hieu Phan]
	Hệ số của $x^3$ trong khai triển $(1+2 x)^4$ là 
	\choice
	{$18 $}
	{$24$ }
	{$28$ }
	{\True $32 $}
	\loigiai{
		Theo khai triển nhị thức ta có $(1+2 x)^4=1+8x+24x^2+32x^3+16x^4$.\\
		Hệ số của $x^3$ trong khai triển là $32$.
	}
\end{ex}
\begin{ex}%[0D8H3-3]%[Hieu Phan]
	Tìm hệ số của $x^4$ trong khai triển biểu thức $(2+3x)^5$. 
	\choice
	{$162$}
	{\True $810$}
	{$3125$}
	{$7776$}
	\loigiai{
		Ta có
		$$\begin{aligned}
			\left(2+3x\right)^5
			&=\mathrm{C}_5^0\cdot2^5+\mathrm{C}_5^1\cdot2^4\cdot(3x)+\cdots+\mathrm{C}_5^4\cdot2^1\cdot(3x)^4+\mathrm{C}_5^5\cdot(3x)^5 \\
			&=243x^5 + 810x^4 + 1080x^3 + 720x^2 + 240x + 32.
		\end{aligned}$$
		Vậy hệ số của $x^4$ trong khai triển $\left(3+3x\right)^5$ là $810$.
	}
\end{ex}
\begin{ex}[Trích đề thi HKII-THPT Nguyễn Thượng Hiền-Năm học 23-24]%[0D8N3-1]%[Hieu Phan]
	Trong khai triển $(2023x-2024)^5$ có bao nhiêu số hạng?
	\choice
	{$4$}
	{$5$}
	{\True $6$}
	{$3$}
	\loigiai{
		Số các số hạng là $5+1=6$.}
\end{ex}
\begin{ex}%[0D8H3-4]%[Hieu Phan]
	Tính tổng các hệ số trong khai triển biểu thức $(2x-3)^4$.
	\choice
	{$-81$}
	{$81$}
	{$-1$}
	{\True $1$}
	\loigiai{
		Ta có khai triển 
		\[(2x-3)^4=\mathrm{C}_4^0(2x)^4+\mathrm{C}_4^1(2x)^3\cdot(-3)+\mathrm{C}_4^2(2x)^2\cdot(-3)^2+\mathrm{C}_4^3(2x)\cdot (-3)^3+\mathrm{C}_4^4(-3)^4.\]
		Với $x=1$ ta có
		\[16\mathrm{C}_4^0-24\mathrm{C}_4^1 +36\mathrm{C}_4^2 -54\mathrm{C}_4^3 +81\mathrm{C}_4^4=(2\cdot 1-3)^4=1.\]
	}
\end{ex}
\begin{ex}%[0D8H3-4]%[Hieu Phan]
	Tìm hệ số của $x^2y^3$ trong khai triển thành đa thức của biểu thức $\left(x-2y\right)^5$.
	\choice
	{$40$}
	{$-40$}
	{\True $-80$}
	{$80$}
	\loigiai{
		Ta có
	\begin{eqnarray*}
	\left(x-2y\right)^5&=&\mathrm{C}_5^0x^5+\mathrm{C}_5^1x^4\cdot(-2y)+\mathrm{C}_5^2x^3\cdot(-2y)^2+\mathrm{C}_5^3x^2\cdot(-2y)^3+\mathrm{C}_5^4x\cdot(-2y)^4+\mathrm{C}_5^5\cdot(-2y)^5\\
	 &=&x^5-10x^4y+40x^3y^2-80x^2y^3+80x{y^4}-32y^5.
	\end{eqnarray*}
		Vậy hệ số của $x^2y^3$ là $-80$.
	}
\end{ex}
\begin{ex}[Trích đề thi HKII-THPT Nguyễn Khuyến-Năm học 23-24]%[0D8H3-4]%[Hieu Phan]
	Tìm số hạng chứa $x^3$ trong khai triển $(4-3x)^5$
	\choice
	{$4320$}
	{$4320x^3$}
	{\True $-4320x^3$}
	{$-4320$}
	\loigiai{
		Ta có \allowdisplaybreaks
		\begin{eqnarray*}
			(4-3x)^5&=&4^5+5\cdot4^4\cdot(-3x)+10\cdot 4^3\cdot (-3x)^2+10\cdot 4^2\cdot (-3x)^3+5\cdot 4\cdot (-3x)^4+(-3x)^5\\
			&=&1024-3840x+5760x^2-4320x^3+1620x^4-243x^5.
		\end{eqnarray*}
		Vậy số hạng chứa $x^3$ là $-4320x^3$.
	}
\end{ex}
\begin{ex}[Trích đề thi HKII-THPT Thiệu Hóa-Năm học 23-24]%[0D8H3-4]%[Hieu Phan]
	Trong khai triển nhị thức Niu-tơn của $(1+3x)^4$ thành đa thức, số hạng đứng chính giữa trong khai triển là
	\choice
	{$108x$}
	{\True $54x^2$}
	{$1$}
	{$12x$}
	\loigiai{
		Ta có 
		\begin{eqnarray*}
			(1+3x)^4&=&1^4+4\cdot 1^3\cdot (3x)+6\cdot 1^2\cdot (3x)^2+4\cdot 1 \cdot (3x)^3+(3x)^4\\
			&=&1+12x+54x^2+108x^3+81x^4
		\end{eqnarray*}
		Suy ra số hạng đứng chính giữa trong khai triển là $54x^2$.
	}
\end{ex}
\begin{ex}[Trích đề thi HKII-THPT Trần Phú-Năm học 23-24]%[0D8H3-4]%[Hieu Phan]
	Hệ số của $a^2b^2$ trong khai triển $(6a-b)^4$ là
	\choice
	{$144$}
	{\True $216$}
	{$36$}
	{$24$}
	\loigiai{
		Ta có $(6a-b)^4=\mathrm{C}_4^0(6a)^4+\mathrm{C}_4^1(6a)^3(-b)+\mathrm{C}_4^2(6a)^2(-b)^2+\mathrm{C}_4^3(6a)(-b)^3+\mathrm{C}_4^4(-b)^4$.\\
		Trong khai triển trên, số hạng chứa $a^2b^2$ là $\mathrm{C}_4^2(6a)^2(-b)^2=216a^2b^2$.
	}
\end{ex}
\begin{ex}[Trích đề thi HKII-Sở GD Bắc Giang-Năm học 23-24]%[0D8H3-5]%[Hieu Phan]
	Tổng tất cả các hệ số của các số hạng trong khai triển $(2+x)^5$ thành đa thức bằng
	\choice
	{$234$}
	{\True $243$}
	{$-1$}
	{$1$}
	\loigiai{
		Ta có
		\allowdisplaybreaks
		\begin{eqnarray*}
			(2+x)^5&=&\mathrm{C}_5^0 2^5+\mathrm{C}_5^1 2^4x+\mathrm{C}_5^2 2^3x^2+\mathrm{C}_5^3 2^2x^3+\mathrm{C}_5^4 2^1x^4+\mathrm{C}_5^5x^5
			\\
			&=&32+80x+80x^2+40x^3+10x^4+x^5.
		\end{eqnarray*}
		Tổng các hệ số trong khai triển là $32+80+80+40+10+1=243$.
	}
\end{ex}

\Closesolutionfile{ans}

\ind{PHẦN II.} \inden{Câu trắc nghiệm đúng sai. Trong mỗi ý a), b), c), d) ở mỗi câu, học sinh chọn đúng hoặc sai.}\\
\setcounter{ex}{0}
\Opensolutionfile{ans}[ans/2D1-Bai1-DS]%--Đặt tên 2D1-Bai1-DS
\begin{ex}%[0D8H3-2]%[0D8H3-4]%[Hieu Phan]
	Khai triển $(x^2-3xy)^4 = a_1\cdot x^8 + a_2\cdot x^7y + a_3\cdot x^6y^2 + a_4\cdot x^5y^3 + a_5\cdot x^4y^4$. 
	\choiceTF
	{\True $a_1=1, a_2=-12$}
	{\True Hệ số của số hạng chứa $x^6y^2$ trong khai triển $(x^2-3xy)^4$ là $54$}
	{Số hạng chứa $x^7y$ trong khai triển $(x^2-3xy)^4$ là $12x^7y$}
	{Tổng hệ số của các số hạng mà lũy thừa của $x$ nhỏ hơn $7$ là $243$}
	\loigiai{
		Số hạng tổng quát $T_{k+1} = \mathrm{C}_4^k(x^2)^{4-k}(-3xy)^k = \mathrm{C}_4^k(-3)^k x^{8-k}y^k$ với $k=0,\ldots, 4$.\\
		Khai triển đầy đủ: $x^8 - 12x^7y + 54x^6y^2 - 108x^5y^3 + 81x^4y^4$.
		\begin{itemchoice}
			\itemch $k=0 \Rightarrow T_1 = x^8 \Rightarrow a_1 = 1$. \\
			$k=1 \Rightarrow T_2 = -12x^7y \Rightarrow a_2 = -12$.
			\itemch Số hạng chứa $x^6y^2$ ứng với $k=2 \Rightarrow T_3 = 54x^6y^2 \Rightarrow$ Hệ số là $54$.
			\itemch Số hạng chứa $x^7y$ là $T_2 = -12x^7y$.
			\itemch Các số hạng có bậc $x < 7$ là $T_3$, $T_4$, $T_5$. \\
			Tổng hệ số là $54 + (-108) + 81 = 27$.
		\end{itemchoice}
	}
\end{ex}
\begin{ex}[Trích đề thi HKII-THPT Như Văn Lang-Năm học 23-24]%[0D8H3-2]%[Hieu Phan]
	Khai triển $\left(x+\dfrac{1}{x}\right)^4$.
	\choiceTF
	{\True Số hạng không chứa $x$ là $6$}
	{Hệ số của $x^2$ là $\dfrac{1}{4}$}
	{\True Hệ số của $x^4$ là $1$}
	{\True Sau khi khai triển, biểu thức có $5$ số hạng}
	\loigiai{Ta có $\left(x+\dfrac{1}{x}\right)^4=x^4+4x^2+6+\dfrac{4}{x^2}+\dfrac{1}{x^4}$.
		\begin{itemchoice}
			\itemch Số hạng không chứa $x$ là $6$.
			\itemch Hệ số của $x^2$ là $4$.
			\itemch Hệ số của $x^4$ là $1$.
			\itemch Sau khi khai triển, biểu thức có $5$ số hạng.
		\end{itemchoice}
	}
\end{ex}
\begin{ex}[Trích đề thi HKII-THPT Lương Thế Vinh-Năm học 23-24]%[0D8H3-4]%[Hieu Phan]
	Cho khai triển $(3x+1)^4$.
	\choiceTF
	{Khai triển có $4$ số hạng}
	{\True Tổng các hệ số của khai triển là $256$}
	{Tổng hệ số của số hạng đầu và số hạng cuối trong khai triển là $81$}
	{\True Hệ số của số hạng chính giữa trong khai triển là $54$}
	\loigiai{
		\begin{itemchoice}
			\itemch Khai triển có 5 số hạng.
			\itemch Tổng các hệ số của khai triển là $(3 \cdot 1+1)^4=256$
			\itemch Tổng hệ số của số hạng đầu và số hạng cuối là $\mathrm{C}_4^0 \cdot 3^0+\mathrm{C}_4^4 \cdot 3^4=82$.
			\itemch Hệ số của số hạng chính giữa trong khai triển là $\mathrm{C}_4^2 \cdot 3^2=54$.
		\end{itemchoice}	
	}
\end{ex}
\begin{ex}%[0D8H3-4]%[Hieu Phan]
	Cho nhị thức $(3x-2)^5$.
	\choiceTF
	{\True  Số hạng chứa $x^5$ trong khai triển Newton của nhị thức trên là $243 x^5$}
	{Hệ số của số hạng chứa $x^2$ trong khai triển Newton của nhị thức trên là $720$}
	{\True Số hạng không chứa $x$ trong khai triển Newton của nhị thức trên là $-32$}
	{\True Tổng các hệ số trong khai triển Newton của nhị thức trên bằng $1$}
	\loigiai{Ta có $(3 x-2)^5=243x^5-810x^4+1080x^3-720x^2+240x-32$.\qquad (1)
		\begin{itemchoice}
			\itemch 
			Số hạng chứa $x^5$ trong khai triển Newton của nhị thức trên là $243x^5$.
			\itemch Hệ số của số hạng chứa $x^2$ trong khai triển Newton của nhị thức trên là $-720$.
			\itemch Số hạng không chứa $x$ trong khai triển Newton của nhị thức trên là $-32$.
			\itemch Thay $x=1$ vào (1) ta được bên phải là tổng của các hệ số và bên phải là $(3\cdot 1-2)^5=1$.\\
			Tổng các hệ số trong khai triển Newton của nhị thức trên bằng $1$.	
		\end{itemchoice}
	}
\end{ex}
\begin{ex}[Trích đề thi HKII-THPT Lương Thế Vinh-Năm học 24-25]%[0D8H3-4]%[Hieu Phan]
	Cho khai triển $\left(2x+\dfrac{1}{x}\right)^4$.
	\choiceTF
	{\True Sau khi khai triển biểu thức có $5$ số hạng}
	{Hệ số của $x^2$ trong khai triển là $8$}
	{Số hạng không chứa $x$ bằng $24$}
	{Tổng các hệ số của khai triển bằng $16$}
	\loigiai{
		Ta có 
		\begin{align*}
			\left(2x+\dfrac{1}{x}\right)^4&=\mathrm{C}_4^0 (2x)^4 \cdot \left( \dfrac{1}{x} \right)^0+\mathrm{C}_4^1 (2x)^3 \cdot \left( \dfrac{1}{x} \right)^1+\mathrm{C}_4^2 (2x)^2 \cdot \left( \dfrac{1}{x} \right)^2+\mathrm{C}_4^3 (2x)^1 \cdot \left( \dfrac{1}{x} \right)^3+\mathrm{C}_4^4 (2x)^0 \cdot \left( \dfrac{1}{x} \right)^4\\
			&=1 \cdot 16x^4 \cdot 1+4 \cdot 8x^3 \cdot \dfrac{1}{x}+6 \cdot 4x^2 \cdot \dfrac{1}{x^2}+4 \cdot 2x \cdot \dfrac{1}{x^3}+1 \cdot 1 \cdot \dfrac{1}{x^4}\\
			&=16x^4+32x^2+24+\dfrac{8}{x^2}+\dfrac{1}{x^4}.
		\end{align*}
		\begin{itemchoice}
			\itemch Khai triển với $ n=4 $ có $ 5 $ số hạng.
			\itemch Hệ số của $x^2$ trong khai triển là $32$
			\itemch Số hạng không chứa $x$ bằng $24$
			\itemch Tổng các hệ số của khai triển bằng $81$.
		\end{itemchoice}
	}
\end{ex}
\Closesolutionfile{ans}

\ind{PHẦN III.} \inden{Trả lời ngắn}\\
\setcounter{ex}{0}
\Opensolutionfile{ans}[ans/2D1-Bai1-DS]%--Đặt tên 2D1-Bai1-DS
\begin{ex}[Trích đề thi HKII-THPT Chuyên Lê Quý Đôn-Năm học 24-25]%[0D8H3-4]%[Hieu Phan]
	Tìm số hạng không chứa $x$ trong khai triển nhị thức $\left(2x-\dfrac{3}{x}\right)^4$.
	\shortans{$216$}
	\loigiai{
		Ta có 
		\begin{eqnarray*}
			\left(2x-\dfrac{3}{x}\right)^4&=&\mathrm{C}_4^0\cdot (2x)^4+\mathrm{C}_4^1 \cdot (2x)^3 \cdot \left(-\dfrac{3}{x}\right)^1+\mathrm{C}_4^2\cdot (2x)^2\cdot \left(-\dfrac{3}{x}\right)^2+\mathrm{C}_4^3\cdot (2x)\cdot \left(-\dfrac{3}{x}\right)^3+\mathrm{C}_4^4\cdot \left(-\dfrac{3}{x}\right)^4\\
			&=& 16x^4-96x^2+216-\dfrac{216}{x^2}+\dfrac{81}{x^4}.
		\end{eqnarray*}
		Số hạng không chứa $x$ trong khai triển là $216$.
	}
\end{ex}
\begin{ex}[Trích đề thi HKII-THPT Lương Thế Vinh-Năm học 23-24]%[0D8H3-2]%[Hieu Phan]
	Khai triển $(1+\sqrt{3})^4$ được viết dưới dạng $a+b\sqrt{3}$ với $a$, $b$ là số nguyên. Tính $a+b$.
	\shortans{$44$}
	\loigiai{
		Ta có
		\begin{eqnarray*}
			(1+\sqrt{3})^4&=& \mathrm{C}_4^0\cdot 1^4+\mathrm{C}_4^1\cdot	1^3\cdot \left(\sqrt{3}\right)+\mathrm{C}_4^2\cdot 1^2\cdot \left(\sqrt{3}\right)^2+\mathrm{C}_4^3\cdot 1^3\cdot (\sqrt{3})^3+\mathrm{C}_4^4(\sqrt{3})^4\\
			&=& 1+4\sqrt{3}+18+12\sqrt{3}+9\\
			&=&28+16\sqrt{3}.
		\end{eqnarray*}
		Vậy $a+b=44$.
	}
\end{ex}

\begin{ex}%[0D8H3-3]%[Hieu Phan]
	Tìm hệ số của số hạng không chứa $x$ trong khai triển $\left(\dfrac{x}{2}-\dfrac{4}{x}\right)^4$ với $x \neq 0$.\par
	\shortans{$24$}
	\loigiai{
		Theo khai triển nhị thức Newton ta có 
		\begin{eqnarray*}
			\left(\dfrac{x}{2}-\dfrac{4}{x}\right)^4&=&\mathrm{C}^0_4 \left(\dfrac{x}{2}\right)^4-\mathrm{C}^1_4 \left(\dfrac{x}{2}\right)^3 \left(\dfrac{4}{x}\right)+ \mathrm{C}^2_4 \left(\dfrac{x}{2}\right)^2 \left(\dfrac{4}{x}\right)^2 -\mathrm{C}^3_4 \left(\dfrac{x}{2}\right) \left(\dfrac{4}{x}\right)^3+\mathrm{C}^4_4  \left(\dfrac{4}{x}\right)^4\\
			&=&  \left(\dfrac{x}{2}\right)^4-4 \left(\dfrac{x}{2}\right)^3 \left(\dfrac{4}{x}\right)+ 24-4 \left(\dfrac{x}{2}\right) \left(\dfrac{4}{x}\right)^3+ \left(\dfrac{4}{x}\right)^4\\
		\end{eqnarray*} 
		Số hạng không chứa $x$ trong khai triển là $24$.
	}
\end{ex}
\begin{ex}[Trích đề thi HKII-THPT Nguyễn Khuyến-Năm học 23-24]%[0D8V3-4]%[Hieu Phan]
	Trong khai triển $(2+x)^n$, biết $\mathrm{C}_n^1+\mathrm{C}_n^2=10$. Tìm hệ số của $x^3$.
	\shortans[]{$8$}
	\loigiai
	{
		Điều kiện $n\in \mathbb{N}$ và $n\ge 2$.\\
		Ta có \allowdisplaybreaks
		\begin{eqnarray*}
			&&\mathrm{C}_n^1+\mathrm{C}_n^2=10\Leftrightarrow \dfrac{n!}{(n-1)!\cdot 1!}+\dfrac{n!}{(n-2)!\cdot 2!}=10\\
			&\Leftrightarrow& n+\dfrac{n(n-1)}{2}=10\Leftrightarrow n^2+n-20=0\Leftrightarrow \hoac{& n=4 \,\text{ (nhận)}\\ & n=-5\,\text{ (loại)}}\Rightarrow n=4.
		\end{eqnarray*}
		Khi đó $(2+x)^4=2^4+4\cdot 2^3\cdot x+6\cdot 2^2\cdot x^2+4\cdot 2\cdot x^3+x^4=16+32x^3+24x^2+8x^3+x^4$.\\
		Vậy hệ số của $x^3$ là $8$.
	}
\end{ex}
\begin{ex}[Trích đề thi HKII-THPT Chuyên Nguyễn Đình Chiểu-Năm học 24-25]%[0D8V3-5]%[Hieu Phan]
	Bạn Minh có $5$ cái bánh khác loại, hỏi Minh có bao nhiêu cách chọn ra một số cái bánh (tính cả trường hợp không chọn cái nào) để mang theo trong buổi cắm trại?
	\shortans[]{$32$}
	\loigiai{
		Số cách chọn bánh của Minh là $\mathrm{C}_5^0 + \mathrm{C}_5^1 + \ldots + \mathrm{C}_5^5$.\\
		Xét $(1+x)^5=\mathrm{C}_5^0 + \mathrm{C}_5^1 x + \ldots + \mathrm{C}_5^5 x^5$. Thay $x=1$ vào hai vế ta được
		$$\mathrm{C}_5^0 + \mathrm{C}_5^1 + \ldots + \mathrm{C}_5^5 = (1+1)^5=2^5=32.$$
		Vậy Minh có $32$ cách chọn.
	}
\end{ex}
\Closesolutionfile{ans}


\ind{PHẦN IV.} \inden{Tự luận.}\\
\setcounter{ex}{0}
\begin{ex}%[0D8H3-3]%[Hiếu Phan]
	Tìm hệ số của số hạng chứa $x^3$ trong khai triển nhị thức Newton $\left(3-2x\right)^5$.
	\loigiai{
		Khai triển $\left(3-2x\right)^5$ có số hạng tổng quát là
		$$\mathrm{C}_5^k3^{5-k}\left(-2x\right)^k=\mathrm{C}_5^k3^{5-k}\left(-2\right)^kx^k.$$
		Số hạng chứa $x^3$ ứng với $x^k=x^3\Rightarrow x=3.$\\
		Hệ số cần tìm là $\mathrm{C}_5^33^{5-3}\left(-2\right)^3=-720$.
	}
\end{ex}
\begin{ex}%[0D8H3-3]%[Hiếu Phan]
	Xác định số hạng không chứa $x$ trong khai triển của $\left(x+\dfrac{2}{x}\right)^4$.
	\loigiai{Ta có 
		\begin{eqnarray*}
			\left(x+\dfrac{2}{x}\right)^4&=&x^4+4x^3\cdot \dfrac{2}{x}+6x^2\cdot\left(\dfrac{2}{x}\right)^2+4x\cdot \left(\dfrac{2}{x}\right)^3+\left(\dfrac{2}{x}\right)^4\\&=&x^4+8x^2+24+\dfrac{32}{x^2}+\dfrac{16}{x^4}.
		\end{eqnarray*}	
		Vậy số hạng không chứa $x$ là $24$.
	}
\end{ex}

\begin{ex}%[0D8H3-3]%[Hiếu Phan]
	Trong khai triển của $(5x-2)^5$, số mũ của $x$ được sắp xếp theo lũy thừa tăng dần, hãy tìm số hạng thứ hai.
	\loigiai{
		Áp dụng công thức khai triển của $(a+b)^5$ với $a=5x;\ b=-2$ ta có
		\allowdisplaybreaks{\begin{eqnarray*}
				(5x-2)^2&=&(5x)^5+5\cdot (5x)^4\cdot(-2)+10\cdot(5x)^3\cdot(-2)^2+10\cdot(5x)^2\cdot(-2)^3+5\cdot 5x\cdot(-2)^4+(-2)^5\\
				&=&-32+400x-2000x^2+5000x^3-6250x^4+3125x^5.
			\end{eqnarray*}
		}
	}
\end{ex}
\begin{ex}%[0D8V3-5]%[Hieu Phan]
	Cho $\left(\dfrac{3}{5}x+\dfrac{1}{2}\right)^5=a_0+a_1x+a_2x^2+a_3x^3+a_4x^4+a_5x^5$. Tính 
	\begin{enumEX}{2}
		\item $a_3$;
		\item $a_0+a_1+a_2+a_3+a_4+a_5$.
	\end{enumEX}
	\loigiai{
		\begin{enumEX}{1}
			\item Số hạng chứa $x^3$ trong khai triển biểu thức $\left(\dfrac{3}{5}x+\dfrac{1}{2}\right)^5$ là $10\cdot\left(\dfrac{3}{5}x\right)^3\cdot\left(\dfrac{1}{2}\right)^2=\dfrac{27}{50}x^3$.\\
			Vậy $a_3=\dfrac{27}{50}$.
			\item Xét $f(x)=\left(\dfrac{3}{5}x+\dfrac{1}{2}\right)^5$.\\
			Khi đó $a_0+a_1+a_2+a_3+a_4+a_5=f(1)=\left(\dfrac{3}{5}\cdot 1+\dfrac{1}{2}\right)^5=\dfrac{161051}{100000}$.
		\end{enumEX}
	}
\end{ex}
\begin{ex}%[0D8V3-4]
	Giả sử hệ số của $x$ trong khai triển của $\left(x^2+\dfrac{r}{x}\right)^5$ bằng $640$. Xác định giá trị của $r$.
	\loigiai{
		Áp dụng công thức khai triển $(a+b)^5$ cho $a=x^2$, $b=\dfrac{r}{x}$ ta được
		\allowdisplaybreaks
		\begin{eqnarray*}
			\left(x^2+\dfrac{r}{x}\right)^5
			&=& (x^2)^5 + 5(x^2)^4\cdot \dfrac{r}{x} + 10(x^2)^3 \cdot \left(\dfrac{r}{x}\right)^2 + 10(x^2)^2 \cdot \left( \dfrac{r}{x} \right)^3 + 5x^2\cdot \left( \dfrac{r}{x} \right)^4 + \left( \dfrac{r}{x} \right)^5 \\
			&=& x^10 + 5rx^7 + 10r^2x^4 + 10r^3x + \dfrac{5r^4}{x^2} + \dfrac{r^5}{x^5}.
		\end{eqnarray*}
		Do vậy, $10r^3=640$, hay $r^3=64$, suy ra $r=4$.
	}
\end{ex}
\begin{ex}[Trích đề thi HKII-THPT Gia Định-Năm học 23-24]%[0D8H3-2]
	Tìm hệ số của số hạng chứa $x^3$ trong khai triển $(2x-3)(3x+1)^4$.
	\loigiai{
		Ta có
		\allowdisplaybreaks
		$\begin{aligned}[t]
			(2x-3)(3x+1)^4&=(2x-3)\left(81x^4+108x^3+54x^2+12x+1\right)\\
			&=162x^5-27x^4-216x^3-138x^2-34x-3.
		\end{aligned}$\\
		Vậy hệ số của số hạng chứa $x^3$ trong khai triển là $-216$.
	}
\end{ex}
\begin{ex}[Trích đề thi HKII-THPT Trần Phú-Năm học 23-24]%[0D8H3-4]%[0D8H3-5]%[Hieu Phan]
	Trong khai triển của $P(x)=4x^3+x^2(x-2)^4$.
	\begin{enumerate}[1)]
		\item{Tìm số hạng chứa $x^2$ trong khai triển đó.}
		\item{Tính tổng các hệ số của $P(x)$.}
	\end{enumerate}
	\loigiai{
		\begin{enumerate}[1)]
			\item{
				Ta có: 
				\begin{eqnarray*}
					(x-2)^4&=&\mathrm{C}_4^0x^4-\mathrm{C}_4^1x^3\cdot 2+\mathrm{C}_4^2 x^2 \cdot 2^2 +\mathrm{C}_4^3 x \cdot 2^3 +\mathrm{C}_4^4 2^4\\
					&=& x^4-8x^3+24x^2-32x+16
				\end{eqnarray*}
				Suy ra $x^2(x-2)^4=x^6-8x^5+24x^4-32x^3 +16x^2$.\\
				Do đó $P(x)=x^6-8x^5+24x^4-28x^3 +16x^2$.\\
				Vậy số hạng chứa $x^2$ là $16x^2$.
			}
			\item{
				Tổng các hệ số của $P(x)$ trong khai triển chính là giá trị của $P(x)$ tại $x=1$.\\
				Ta có $P(1)=3$.\\
				Vậy tổng các hệ số của $P(x)$ bằng $3$.
			}
		\end{enumerate}
	}
\end{ex}
\begin{ex}[Trích đề thi HKII-THPT Chuyên Nguyễn Đình Chiểu-Năm học 24-25]%[0D8H3-2]%[Hieu Phan]
	Tìm số hạng có hệ số lớn nhất trong khai triển $(1+3x^2)^4$.
	\loigiai{
		Ta có $\begin{aligned}[t]
			(1+3x^2)^4 &= \mathrm{C}_4^0 + \mathrm{C}_4^1(3x^2) + \mathrm{C}_4^2(3x^2)^2 + \mathrm{C}_4^3(3x^2)^3 + \mathrm{C}_4^4(3x^2)^4\\
			&=1 + 12x^2 + 54x^4 + 108x^6 + 81x^8.
		\end{aligned}$\\
		Vậy số hạng có hệ số lớn nhất là $108x^6$.
	}
\end{ex}

\begin{ex}[Trích đề thi HKII-THPT Đặng Huy Trứ-Năm học 23-24]%[0D8V3-4]
Cho $n$ là số nguyên dương thỏa mãn $\mathrm{C}^1_n+\mathrm{C}^2_n=15$. Tìm số hạng không chứa $x$ trong khai triển $\left(x^3-\dfrac{1}{x^2}\right)^n$. 
\loigiai{
	\begin{eqnarray*}
		\mathrm{C}^1_n+\mathrm{C}^2_n=15 &\Leftrightarrow& \dfrac{n!}{(n-1)!\cdot 1!}+\dfrac{n!}{(n-2)!\cdot 2!}=15\\
		&\Leftrightarrow& n+\dfrac{n(n-1)}{2}=15\\
		&\Leftrightarrow& 2n+n(n-1)=30\\
		&\Leftrightarrow& n^2+n-30=0\Leftrightarrow n=5 \vee n=-6.  
	\end{eqnarray*}
	Vì $n$ nguyên dương nên ta nhận $n=5$.\\
	$\left(x^3-\dfrac{1}{x^2}\right)^5=\mathrm{C}^0_5\left( x^3\right)^5 \left(-\dfrac{1}{x^2}\right)^0+ \mathrm{C}^1_5\left( x^3\right)^4 \left(-\dfrac{1}{x^2}\right)^1+\mathrm{C}^2_5\left( x^3\right)^3 \left(-\dfrac{1}{x^2}\right)^2+\mathrm{C}^3_5\left( x^3\right)^2 \left(-\dfrac{1}{x^2}\right)^3+\mathrm{C}^4_5\left( x^3\right) \left(-\dfrac{1}{x^2}\right)^4+\mathrm{C}^5_5\left( x^3\right)^0 \left(-\dfrac{1}{x^2}\right)^5$. \\
	$\left(x^3-\dfrac{1}{x^2}\right)^5=x^{15}-5x^{10}+10x^{5}-10+\dfrac{5}{x^5}-\dfrac{1}{x^{10}}$.\\
	Vậy số hạng không chứa $x$ trong khai triển là $-10$.
	
}
\end{ex}
\begin{ex}%[0D8H3-4]%[Hieu Phan]
	Xác định hệ số của $x^3$ trong khai triển của biểu thức $\left(\dfrac{2}{3}x+\dfrac{1}{4}\right)^4$.
	\loigiai{
		Số hạng chứa $x^3$ trong khai triển của biểu thức $\left(\dfrac{2}{3}x+\dfrac{1}{4}\right)^4$ là $4\cdot \left(\dfrac{2}{3}x\right)^3\cdot\dfrac{1}{4}=\dfrac{8}{27}x^3$.\\
		Vậy hệ số của $x^3$ trong khai triển biểu thức $\left(\dfrac{2}{3}x+\dfrac{1}{4}\right)^4$ là $\dfrac{8}{27}$.
	}
\end{ex}

