\newpage
\def\thoigian{90}%--Thời gian
\de{Đề số 2}{Chương VIII. Đại số tổ hợp}



\begin{center}
	\textbf{PHẦN 1 - CÂU TRẮC NGHIỆM BỐN PHƯƠNG ÁN}
\end{center}
\Opensolutionfile{ans}[ans/ans-TN-ONTAPCHUONG-DE1]

\begin{ex}%[0D8N2-2]%[Dự án D - đợt 3 NH24-25- Nguyễn Hữu Đức]
	Trong giỏ có $5$ hoa hồng và $3$ hoa lan. Số cách chọn $1$ bông hoa từ giỏ là
	\choice
	{\True $8$}
	{$10$}
	{$60$}
	{$15$}
	\loigiai{Số cách chọn $1$ bông hoa hồng là $5$ cách.\\
		Số cách chọn $1$ bông hoa lan là $3$ cách.\\
		Theo quy tắc cộng có $5+3=8$ cách chọn một bông hoa.
		%Số cách chọn $1$ bông hoa từ giỏ là $\mathrm{C}_8^1=8$.
	}
\end{ex}
\begin{ex}%[0D8N1-2]%[Dự án D - đợt 3 NH24-25- Nguyễn Hữu Đức]
	Từ thành phố $A$ đến thành phố $B$ có $5$ con đường đi, từ thành phố $B$ đến thành phố $C$ có $4$ con đường đi. Số cách đi từ thành phố $A$ qua thành phố $B$ rồi đến thành phố $C$ bằng
	\choice
	{$32$ cách}
	{\True $20$ cách}
	{$80$ cách}
	{$9$ cách}
	\loigiai{
		Số cách để đi từ thành phố $A$ đến thành phố $C$ phải qua thành phố $B$ là $5\cdot 4 = 20$ cách.
	}
\end{ex}

\begin{ex}%[0D8H2-1]%[Dự án D - đợt 3 NH24-25- Nguyễn Hữu Đức]
	Cho các số nguyên dương $k$ và $n$, $(k<n)$. Mệnh đề nào trong các mệnh đề sau đây \textbf{sai}?
	\choice
	{$\mathrm{P}_n=n!$}
	{$\mathrm{C}_n^k=\dfrac{n!}{k!(n-k)!}$}
	{\True $\mathrm{A}_n^k=n!\mathrm{C}_n^k$}
	{$\mathrm{A}_n^k=k!\mathrm{C}_n^k$}
	\loigiai{Do $\mathrm{P}_n=n!$, $\mathrm{C}_n^k=\dfrac{n!}{k!(n-k)!}$, $\mathrm{A}_n^k=\dfrac{n!}{(n-k)!}$ nên $\mathrm{A}_n^k=k!\mathrm{C}_n^k$.\\ Vậy mệnh đề sai là $\mathrm{A}_n^k=n!\mathrm{C}_n^k$.}
\end{ex}
\begin{ex}%[0D8H3-2]%[Dự án D - đợt 3 NH24-25- Nguyễn Hữu Đức]
	Khai triển đa thức $(3x-2y)^4$ bằng
	\choice
	{\True $81 x^4-216 x^3y+216 x^2 y^2-96x y^3+16 y^4$}
	{$81 x^4+216 x^3y+216 x^2 y^2+96x y^3+16 y^4$}
	{$81 y^4-216x y^3+216 x^2 y^2-96 x^3y+16 x^4$}
	{$81 y^4+216x y^3+216 x^2 y^2+96 x^3y+16 x^4$}
	\loigiai{
		Ta có
		\begin{eqnarray*}
			(3x-2y)^4&=&\mathrm{C}_4^0(3x)^4+\mathrm{C}_4^1(3x)^3\cdot (-2y)+\mathrm{C}_4^2(3x)^2\cdot (-2y)^2+\mathrm{C}_4^3(3x)\cdot (-2y)^3+\mathrm{C}_4^4(-2y)^4\\
			&=&81 x^4-216 x^3y+216 x^2 y^2-96x y^3+16 y^4.
		\end{eqnarray*}
	}
\end{ex}
\begin{ex}%[0D8N2-1]%[Dự án D - đợt 3 NH24-25- Nguyễn Hữu Đức]
	Số hoán vị của một tập có $n$ phần tử là
	\choice
	{$\mathrm{P}_n=(n-1)!$}
	{$\mathrm{P}_n=\dfrac{n!}{(n-k)!k!}$}
	{\True $\mathrm{P}_n=n!$}
	{$\mathrm{P}_n=\dfrac{n!}{(n-k)!}$}
	\loigiai{Số hoán vị của một tập có $n$ phần tử là $\mathrm{P}_n=n!$.
	}
\end{ex}

\begin{ex}%[0D8N2-2]%[Dự án D - đợt 3 NH24-25- Nguyễn Hữu Đức]
	Số chỉnh hợp chập $3$ của $8$ phần tử bằng
	\choice
	{$56$}
	{\True $336$}
	{$24$}
	{$512$}
	\loigiai{
		Số chỉnh hợp chập $3$ của $8$ là $\mathrm{A}_8^3 = 336$.
	}
\end{ex}
\begin{ex}%[0D8H1-2]%[Dự án D - đợt 3 NH24-25- Nguyễn Hữu Đức]
	Có $10$ cặp vợ chồng đi dự tiệc. Số cách chọn một người đàn ông và một người phụ nữ trong bữa tiệc phát biểu ý kiến sao cho hai người đó \textbf{không} là vợ chồng là
	\choice
	{$100$}
	{$91$}
	{$10$}
	{\True $90$}
	\loigiai{
		\begin{itemize}
			\item Chọn một người đàn ông tùy ý có $10$ cách;
			\item Chọn một người phụ nữ không là vợ của người đàn ông đã chọn có $9$ cách.
		\end{itemize}
		Theo quy tắc nhân, có $10\cdot 9=90$ cách chọn một người đàn ông và một người phụ nữ trong bữa tiệc phát biểu ý kiến sao cho hai người đó không là vợ chồng.
	}
\end{ex}
\begin{ex}%[0D8H2-3]%[Dự án D - đợt 3 NH24-25- Nguyễn Hữu Đức]
	Số tự nhiên có $5$ chữ số trong đó các chữ số cách đều chữ số đứng giữa thì giống nhau bằng
	\choice
	{\True $900$}
	{$9\,000$}
	{$90\,000$}
	{$27\,216$}
	\loigiai{
		Gọi số tự nhiên có $5$ chữ số trong đó các chữ số cách đều chữ số đứng giữa thì giống nhau có dạng $n=\overline{abcba}$. 
		\begin{itemize}
			\item Chọn $a$ có $9$ cách chọn $(a \ne 0)$;
			\item Chọn $b$ có $10$ cách chọn;
			\item Chọn $c$ có $10$ cách chọn.
		\end{itemize}
		Theo quy tắc nhân, có $9 \cdot 10 \cdot 10 = 900$ số thỏa yêu cầu bài toán.
	}
\end{ex}
\begin{ex}%[0D8H2-3]%[Dự án D - đợt 3 NH24-25- Nguyễn Hữu Đức]
	Trong hộp có $15$ tấm thẻ được đánh số từ $1$ đến $15$. Lấy ngẫu nhiên từ trong hộp ra $2$ tấm thẻ. Số cách lấy ra hai thẻ sao cho tổng ghi trên hai thẻ một số chẵn là
	\choice
	{$35$}
	{\True $49$}
	{$28$}
	{$21$}
	\loigiai{
		%Gọi $A$ là biến cố \lq\lq Hai thẻ lấy ra có tổng là một số chẵn\rq\rq.\\
		Ta thấy từ $1$ đến $15$ có $8$ thẻ lẻ và $7$ thẻ chẵn.\\
		Do hai thẻ lấy ra có tổng là một số chẵn nên có $2$ trường hợp xảy ra
		\begin{enumerate}[\it TH 1.]
			\item Cả hai thẻ đều chẵn có $\mathrm{C}_7^2=21$ cách.
			\item Cả hai thẻ đều lẻ có $\mathrm{C}_8^2=28$ cách.
		\end{enumerate}
		Theo quy tắc cộng, ta có $21+28=49$ cách.
	} 
\end{ex}
\begin{ex}%[0D8H2-6]%[Dự án D - đợt 3 NH24-25- Nguyễn Hữu Đức]
	Cho hai đường thẳng song song $d_1$ và $d_2$. Trên $d_1$ lấy $17$ điểm phân biệt, trên $d_2$ lấy $20$ điểm phân biệt. Số tam giác mà các đỉnh được chọn từ $37$ điểm này bằng
	\choice
	{$5\,690$}
	{$5\,960$}
	{\True $5\,950$}
	{$5\,590$}
	\loigiai{
		Số cách chọn $3$ điểm bất kì từ $37$ điểm là $\mathrm{C}_{37}^3=7\,770$ cách.\\
		Trong đó, số cách chọn ra $3$ điểm thẳng hàng là $\mathrm{C}_{17}^3+\mathrm{C}_{20}^3=1\,820$ cách.\\
		Do đó số cách chọn $3$ điểm tạo thành tam giác là $7\,770-1\,820=5\,950$.
	} 
\end{ex}

\begin{ex}%[0D8H2-4]%[Dự án D - đợt 3 NH24-25- Nguyễn Hữu Đức]
	Một lớp có $26$ học sinh nam, $15$ học sinh nữ. Có bao nhiêu cách chọn ra $3$ bạn trong lớp để một bạn làm lớp trưởng, một bạn làm lớp phó và một bạn làm bí thư chi đoàn?
	\choice
	{$\mathrm{C}_{41}^3$}
	{$26+15$}
	{$26 \cdot 15$}
	{\True $\mathrm{A}_{41}^3$}
	\loigiai{Để chọn ra $3$ bạn trong lớp để một bạn làm lớp trưởng, một bạn làm lớp phó và một bạn làm bí thư chi đoàn là một chỉnh hợp chập $3$ của $41$ nên ta có $\mathrm{A}_{41}^3$.}
\end{ex}
\begin{ex}%[0D8N3-4]%[Dự án D - đợt 3 NH24-25- Nguyễn Hữu Đức]
	Số hạng không chứa $x$ trong khai triển nhị thức Newton $\left( x-\dfrac{3}{x} \right)^{4}$ với $x\ne 0$ là
	\choice
	{$16$}
	{$-54$}
	{\True $54$}
	{$48$}
	\loigiai{
		Theo công thức về khai triển nhị thức Newton, ta có 
		\begin{eqnarray*}
			\left( x-\dfrac{3}{x} \right)^{4}&=&\mathrm{C}_4^0 x^4+\mathrm{C}_4^1 x^3\cdot\left(-\dfrac{3}{x}\right)^1+\mathrm{C}_4^2 x^2\cdot\left(-\dfrac{3}{x}\right)^2+\mathrm{C}_4^	3 x^1\cdot\left(-\dfrac{3}{x}\right)^3+\mathrm{C}_4^4 \cdot\left(-\dfrac{3}{x}\right)^4\\
			&=& x^4+4x^3\cdot\dfrac{-3}{x}+6\cdot x^2\cdot\dfrac{9}{x^2}+4\cdot x\cdot\dfrac{-27}{x^3}+\dfrac{81}{x^4}\\
			&=& x^4-12 x^2+54-\dfrac{108}{x^2}+\dfrac{81}{x^4}.
		\end{eqnarray*}
		Suy ra số hạng không chứa $x$ trong khai triển là $54$.
	}
\end{ex}

\Closesolutionfile{ans}
%\begin{center}
%	\textbf{ĐÁP ÁN}
%	\inputansbox{10}{ans/ans}	
%\end{center}



\begin{center}
	\textbf{PHẦN 2 - CÂU TRẮC NGHIỆM ĐÚNG SAI}
\end{center}

\Opensolutionfile{ans}[ans/answer-DS-ONTAPCHUONG-DE1]
\setcounter{ex}{0}
\begin{ex}%[0D8H3-4]%[Dự án D - đợt 3 NH24-25- Nguyễn Hữu Đức]
	Xét khai triển của biểu thức $(x+1)^4$, các số hạng được viết theo thứ tự số mũ của $x$ giảm dần. 
	\choiceTF
	{\True Khai triển có $5$ số hạng}
	{Số hạng chứa $x^3$ trong khai triển là $5x^3$}
	{\True Số hạng chính giữa trong khai triển là $6x^2$}
	{\True Hệ số của số hạng thứ hai và thứ tư trong khai triển bằng nhau}
	\loigiai{Ta có $(x+1)^4=x^4+4x^3+6x^2+4x+1$.
		\begin{itemchoice}
			\itemch 
			Khai triển ta được $5$ số hạng.
			\itemch 
			Số hạng chứa $x^3$ là $4x^3$.
			\itemch 
			Số hạng chính giữa là $6x^2$.
			\itemch 
			Hệ số của số hạng thứ hai và số hạng thứ tư đều là $4$.
		\end{itemchoice}
	}
\end{ex}
\begin{ex}%[0D8V2-7]%[Dự án D - đợt 3 NH24-25- Nguyễn Hữu Đức]
	An và Bình cùng $7$ bạn khác rủ nhau đi xem bóng đá. Cả $9$ bạn được xếp vào $9$ ghế theo hàng ngang.
	\choiceTF
	{\True Có $9!$ cách xếp chỗ ngồi tùy ý}
	{Có $5\,040$ cách xếp để An và Bình ngồi $2$ đầu dãy ghế}
	{\True Có $40\,320$ cách xếp để An ngồi chính giữa}
	{\True Có $282\,240$ cách xếp để An và Bình không ngồi cạnh nhau}
	\loigiai{
		\begin{itemchoice}
			\itemch Vì $9$ bạn được xếp vào $9$ ghế theo hàng ngang là việc hoán đổi vị trí của $9$ bạn cho nhau.
			\itemch Vì có $2$ cách sắp xếp cho An và Bình ngồi $2$ đầu dãy ghế.\\
			Có $7!$ cách sắp xếp chỗ cho $7$  bạn còn lại.\\
			Do đó có $2\cdot 7!=10\,080$ cách.
			\itemch Có $1$ cách sắp xếp An ngồi chính giữa.\\
			Có $8!$ cách sắp xếp chỗ cho $8$ bạn còn lại.\\
			Do đó có $8!=40\,320$ cách.
			\itemch Vì số cách sắp xếp An, Bình ngồi cạnh nhau là $2!\cdot 8!=80\,640$ cách.\\
			Số cách xếp để An và Bình không ngồi cạnh nhau là $9!-80\,640=282\,240$ cách.	
		\end{itemchoice}
	}
\end{ex}
\Closesolutionfile{ans}
%\inputansbox[2]{2}{ans/answer.tex}



\begin{center}
\textbf{PHẦN 3 - CÂU TRẮC NGHIỆM TRẢ LỜI NGẮN}
\end{center}
\setcounter{ex}{0}
\Opensolutionfile{ans}[ans-KQ-ONTAPCHUONG-DE1]

\begin{ex}%[0D8V2-7]%[Dự án D - đợt 3 NH24-25- Nguyễn Hữu Đức]
	Một nhóm gồm $4$ học sinh nam và $3$ học sinh nữ chụp ảnh kỉ yếu. Nhóm muốn trong bức ảnh $3$ bạn nữ ngồi ở hàng đầu và $4$ bạn nam đứng ở hàng sau. Có bao nhiêu cách sắp xếp vị trí chụp ảnh như vậy? \par \shortans{$144$}
	\loigiai{
		Xếp $3$ bạn nữ ngồi hàng đầu có $3!$ cách.\\
		Mỗi cách xếp các bạn nữ ta có $4!$ cách xếp các bạn nam đứng ở hàng sau.\\
		Do vậy số cách xếp thỏa yêu cầu là $3!\cdot 4! = 144$ cách.
	}
\end{ex}

\begin{ex}%[0D8H3-4]%[Dự án D - đợt 3 NH24-25- Nguyễn Hữu Đức]
	Hệ số của $x^2$ trong khai triển $(2x + 3)^4$ bằng bao nhiêu?
	\shortans{$216$}
	\loigiai{
		Ta có khai triển
		\allowdisplaybreaks
		\begin{eqnarray*}
			(2x + 3)^4 &=& (2x)^4 + 4\cdot (2x)^3\cdot 3 + 6\cdot (2x)^2\cdot 3^2 + 4\cdot (2x)\cdot 3^3 + 3^4 \\
			&=& 16x^4 + 96x^3 + 216x^2 + 216x + 81.
		\end{eqnarray*}
		Vậy hệ số của $x^2$ là $216$.
	}
\end{ex}
\begin{ex}%[0D8V2-5]%[Dự án D - đợt 3 NH24-25- Nguyễn Hữu Đức]
	Có $5$ tem thư khác nhau và $6$ bì thư cũng khác nhau. Người ta muốn chọn từ đó $3$ tem thư, $3$ bì thư và dán $3$ tem thư đó ấy lên $3$ bì thư đã chọn, mỗi bì thư chỉ dán một tem thư. Hỏi có bao nhiêu cách làm như vậy?
	\par \shortans[]{$1200$} %Chuẩn hóa ko quá 4 ký tự
	\loigiai{
		Chọn $3$ bì thư có $\mathrm{C}_6^3$.\\
		Chọn $3$ tem thư và dán nó vào $3$ bì thư có $\mathrm{A}_5^3$.\\
		Số cách chọn cần tìm là  $\mathrm{C}_6^3\cdot \mathrm{A}_5^3=1\,200$.}
\end{ex}

\begin{ex}%[0D8H2-4]%[Dự án D - đợt 3 NH24-25- Nguyễn Hữu Đức]
	Một đội văn nghệ gồm $6$ nam và $4$ nữ. Có bao nhiêu cách chọn ra $4$ người biểu diễn sao cho số nam nhiều hơn số nữ?
	
	\shortans[]{$95$}
	\loigiai{Để chọn ra $4$ người biểu diễn sao cho số nam nhiều hơn số nữ, ta chia thành các phương án sau
		\begin{itemize}
			\item Chọn ra $4$ nam, có $\mathrm{C}_6^4=15$ cách.
			\item Chọn ra $3$ nam và $1$ nữ, có $\mathrm{C}_6^3\cdot \mathrm{C}_4^1=80$ cách.
		\end{itemize}
		Theo quy tắc cộng, có $15+80=95$ cách chọn ra $4$ người biểu diễn sao cho số nam nhiều hơn số nữ.}
\end{ex}

\Closesolutionfile{ans}
\begin{center}
	\textbf{PHẦN 4 - TỰ LUẬN}
\end{center}
\setcounter{ex}{0}
\begin{ex}%[0D8H3-2]%[Dự án D - đợt 3 NH24-25- Nguyễn Hữu Đức]
	Tìm hệ số của số hạng chứa $x^5$ trong khai triển đa thức $(2-3x)^5$.
	\loigiai{
		Ta có \allowdisplaybreaks
		\begin{eqnarray*}
			\left(2-3x\right)^5&=&\mathrm{C}_5^02^5+\mathrm{C}_5^12^4(-3x)+\mathrm{C}_5^22^3(-3x)^2+\mathrm{C}_5^32^2(-3x)^3+\mathrm{C}_5^42(-3x)^4+\mathrm{C}_5^5(-3x)^5\\
			&=&32+5\cdot 16\cdot (-3x)+10\cdot 8\cdot 9x^2+10\cdot 4\cdot \left(-27x^3\right)+5\cdot 2\cdot 81x^4-243x^5\\
			&=&32-240x+720x^2-1080x^3+810x^4-243x^5.	
		\end{eqnarray*}	
	Vậy hệ số của số hạng chứa $x^5$ là $-243$.
	}
\end{ex}
\begin{ex}%[0D8V2-3]%[Dự án D - đợt 3 NH24-25- Nguyễn Hữu Đức]
	Cho tập  hợp $A$ gồm các chữ số $1$, $2$, $3$, $4$, $5$, $6$. Từ các chữ số của tâp hợp $A$ lập được bao nhiêu số tự nhiên có $8$ chữ số trong đó chữ số $1$ và chữ số $5$ xuất hiện hai lần và các chữ số khác xuất hiện đúng một lần
	\loigiai{
		Chọn vị trí cho hai chữ số $1$ có $\mathrm{C}_8^2=28$ cách, với mỗi cách đó
			\begin{itemize}
				\item Có $\mathrm{C}_6^2=15$ cách chọn vị trí cho $2$ chữ số $5$.
				\item Có $\mathrm{P}_4=24$ cách chọn vị trí cho $4$ chữ số còn lại.
			\end{itemize}
			Vậy, có thể lập được $28\cdot 15\cdot 24=10\,080$ số tự nhiên có $8$ chữ số trong đó chữ số $1$ và chữ số $5$ xuất hiện hai lần và các chữ số khác xuất hiện đúng một lần.
			%\textbf{Nhận xét.} Mỗi cách sắp xếp vị trí cho các chữ số $1$, $1$, $5$, $5$, $2$, $3$, $4$, $6$ được gọi là một hoán vị lặp của $8$ phần tử. Số các số thỏa mãn yêu cầu là số các hoán vị lặp và bằng $\dfrac{8!}{2!2!1!1!1!1!}=10\,080$.
	} 
\end{ex}
\begin{ex}%[0D8H2-5]%[Dự án D - đợt 3 NH24-25- Nguyễn Hữu Đức]
	Thực đơn tại một quán cơm văn phòng có $6$ món mặn, $5$ món rau và $3$ món canh. Một nhóm khách vào quán muốn chọn thực đơn cho bữa trưa gồm: $2$ món mặn, $2$ món rau và $1$ món canh. Hỏi nhóm khách có bao nhiêu cách chọn món ăn?
	\loigiai{
Có $6$ cách chọn $1$ món mặn thứ nhất và có $5$ cách chọn $1$ món mặn thứ hai. Nhưng để chọn hai món này ta không cần chọn thứ tự nên thực tế có $\dfrac{6\cdot5}{2}=15$ (cách).\\
Có $5$ cách chọn $1$ món rau thứ nhất và có $4$ cách chọn $1$ món rau thứ hai. Nhưng để chọn hai món này ta không cần chọn thứ tự nên thực tế có $\dfrac{5\cdot4}{2}=10$ (cách).\\
Có $3$ cách chọn $1$ món canh.\\
Vậy, nhóm khách có 
$15\cdot 10\cdot 3=450$ cách chọn món ăn.\\
\textbf{Cách khác:} Có $\mathrm{C}_6^2$ cách chọn $2$ món mặn, với mỗi cách đó có $\mathrm{C}_5^2$ cách chọn $2$ món rau, với mỗi cách đó có $\mathrm{C}_3^1$ cách chọn $1$ món canh.\\
Vậy, nhóm khách có 
$\mathrm{C}_6^2\cdot\mathrm{C}_5^2\cdot\mathrm{C}_3^1=15\cdot 10\cdot 3=450$ cách chọn món ăn.
	}
\end{ex}

