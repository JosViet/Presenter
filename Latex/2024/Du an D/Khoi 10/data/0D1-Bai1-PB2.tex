\section{MỆNH ĐỀ}
\subsection{LÝ THUYẾT CẦN NHỚ}
\subsubsection{Mệnh đề}
\iconMT\indam{Định nghĩa:}
\begin{boxdn}
	\begin{itemize}
		\item Mệnh đề là một khẳng định \textbf{đúng} hoặc \textbf{sai}.
		\item Một khẳng định đúng gọi là \textbf{mệnh đề đúng}.
		\item Một khẳng định sai gọi là \textbf{mệnh đề sai}.
		\item  Một mệnh đề không thể vừa đúng vừa sai.
	\end{itemize}
\end{boxdn}	
\begin{note}
	\begin{itemize}
		\item Người ta thường sử dụng các chữ cái in hoa $P$, $Q$, $R$, \ldots  để kí hiệu mệnh đề.
		\item Những mệnh đề liên quan đến toán học còn được gọi là \textbf{mệnh đề toán học}.
	\end{itemize}
\end{note}
\subsubsection{Mệnh đề chứa biến}
\iconMT\indam{Định nghĩa:}
\begin{boxdn}
	Những khẳng định mà tính đúng, sai của chúng phụ thuộc vào giá trị của biến gọi là \textbf{mệnh đề chứa biến}.
\end{boxdn}	
\subsubsection{Mệnh đề phủ định}
\iconMT\indam{Định nghĩa:}
\begin{boxdn}
	Mỗi mệnh đề $P$ có \textbf{mệnh đề phủ định}, kí hiệu là $\overline{P}$.\\
	Mệnh đề $P$ và mệnh đề phủ định $\overline{P}$ của nó có tính đúng sai trái ngược nhau. Nghĩa là khi $P$ đúng thì $\overline{P}$ sai, khi $P$ sai thì $\overline{P}$ đúng.
\end{boxdn}	
\subsubsection{Mệnh đề kéo theo}
\iconMT\indam{Định nghĩa:}
\begin{boxdn}
	Cho hai mệnh đề $P$ và $Q$. Mệnh đề \lq\lq Nếu $P$ thì $Q$\rq\rq\ được gọi là mệnh đề kéo theo.\\
	Mệnh đề $P\Rightarrow Q$ chỉ sai khi $P$ đúng và $Q$ sai.
\end{boxdn}	

\begin{nx}
	\begin{enumerate}[a)]
		\item Mệnh đề $P\Rightarrow Q$ còn được phát biểu là \lq\lq $P$ kéo theo $Q$\rq\rq, \lq\lq $P$ suy ra $Q$\rq\rq.
		\item Để xét tính đúng sai của mệnh đề $P \Rightarrow Q$, ta chỉ cần xét trường hợp $P$ đúng. Khi đó, nếu $Q$ đúng thì mệnh đề đúng, nếu $Q$ sai thì mệnh đề sai.
	\end{enumerate}	
\end{nx}
\begin{note}
	Trong toán học, định lí là một mệnh đề đúng, thường có dạng $P\Rightarrow Q$.
	Khi đó ta nói 
	\begin{enumerate}[•]
		\item $P$ là giả thiết, $Q$ là kết luận của định lí.
		\item $P$ là \textbf{điều kiện đủ}  để có $Q$, còn $Q$ là  \textbf{điều kiện cần}  để có $P$.
	\end{enumerate}
\end{note}
\subsubsection{Mệnh đề đảo. Hai mệnh đề tương đương}
\begin{enumerate}[\iconMT]
	\item \indam{Định nghĩa:}
	\begin{boxdn}
		Mệnh đề $Q\Rightarrow P$ được gọi là \textbf{mệnh đề đảo} của mệnh đề $P\Rightarrow Q$.
	\end{boxdn}
	\begin{note}
		Mệnh đề đảo của một mệnh đề đúng không nhất thiết là đúng.
	\end{note}
	\item \indam{Định nghĩa:}
	\begin{boxdn}
		Nếu cả hai mệnh đề $P \Rightarrow Q$ và $Q \Rightarrow P$ đều đúng thì ta nói $P$ và $Q$ là hai \textbf{mệnh đề tương đương}, kí hiệu là $P \Leftrightarrow Q$ (đọc là \lq\lq $P$ tương đương $Q$\rq\rq~hoặc \lq\lq $P$ khi và chỉ khi $Q$\rq\rq).\\	
		Khi đó, ta cũng nói $P$ là \textbf{điều kiện cần và đủ} để có $Q$ (hay $Q$ là điều kiện cần và đủ để có $P$).
	\end{boxdn}
	\begin{nx}
		Hai mệnh đề $P$ và $Q$ tương đương khi chúng cùng đúng hoặc cùng sai.
	\end{nx}
\end{enumerate}
\subsubsection{Mệnh đề chứa kí hiệu $\forall$ và $\exists$}
\iconMT\indam{Định nghĩa:}
\begin{boxdn}
	Mệnh đề \lq\lq $\forall x\in M, P(x)$\rq\rq~ đúng nếu với mọi $x_0\in M$, $P\left(x_0\right)$ là mệnh đề đúng.\\	
	Mệnh đề \lq\lq $\exists x\in M, P(x)$\rq\rq~ đúng nếu có $x_0\in M$ sao cho $P\left(x_0\right)$ là mệnh đề đúng.
\end{boxdn}
%-------------------------------------------------------------------------------------------------------------
\subsection{PHÂN LOẠI VÀ PHƯƠNG PHÁP GIẢI TOÁN}
\begin{dang}{Xác định mệnh đề phủ định và xét tính đúng sai của một mệnh đề}
	 \textbf{Phương pháp giải:}
	\begin{enumerate}
		\item \textbf{Bước 1: Tìm mệnh đề phủ định.}
		\begin{itemize}
			\item Để phủ định một mệnh đề, ta thường thêm (hoặc bớt) từ \lq\lq không\rq\rq\ hoặc \lq\lq không phải\rq\rq\ vào trước vị ngữ.
			\item Phủ định của mệnh đề \lq\lq $\forall x, P(x)$ \rq\rq\ là \lq\lq $\exists x, \text{ không } P(x)$\rq\rq.
			\item Phủ định của mệnh đề \lq\lq $\exists x, P(x)$ \rq\rq\ là \lq\lq $\forall x, \text{ không } P(x)$\rq\rq.
		\end{itemize}
		\item \textbf{Bước 2: Xét tính đúng sai của mệnh đề.}
		\begin{itemize}
			\item Một mệnh đề và mệnh đề phủ định của nó có tính đúng sai trái ngược nhau.
			\item Nếu mệnh đề ban đầu đúng thì mệnh đề phủ định sai và ngược lại.
		\end{itemize}
	\end{enumerate}
\end{dang}
%%%=============VD_1=============%%%
\begin{vd}%[0D1H1-3]%[Dự án dề cương 3 Khối NH24-25-Đợt 1-Võ Thị Thùy Trang]
	Phát biểu mệnh đề phủ định của mỗi mệnh đề sau:
	\begin{multicols}{2}
		\begin{enumerate}
			\item $25$ là số chính phương;
			\item Hình chữ nhật không phải là hình vuông.
		\end{enumerate}
	\end{multicols}
	\loigiai{
		\begin{multicols}{2}
			\begin{enumerate}
				\item $25$ không phải là số chính phương;
				\item Hình chữ nhật là hình vuông.
			\end{enumerate}
		\end{multicols}
	}
\end{vd}
%%%=============================%%%

%%%=============VD_2=============%%%
\begin{vd}%[0D1H1-3]%[Dự án dề cương 3 Khối NH24-25-Đợt 1-Võ Thị Thùy Trang]
	Nêu mệnh đề phủ định của mỗi mệnh đề sau và xét tính đúng sai của mỗi mệnh đề phủ định đó:
	\begin{enumerate}
		\item $A\colon$ \lq\lq $\dfrac{1{,}2}{5}$ là một phân số\rq\rq;
		\item $B\colon$ \lq\lq Phương trình $x^2+3x+2=0$ có nghiệm\rq\rq;
		\item $C\colon$ \lq\lq $2^2+2^3=2^{2+3}$\rq\rq ;
		\item $D\colon$ \lq\lq Số $2025$ chia hết cho $15$\rq\rq.
	\end{enumerate}
	\loigiai{
		\begin{enumerate}[a)]
			\item Mệnh đề phủ định của mệnh đề $A$ là $\overline{A}\colon$ \lq\lq $\dfrac{1{,}2}{5}$ không là phân số\rq\rq.\\ Mệnh đề $\overline{A}$ đúng vì $1{,}2$ không là số nguyên.
			\item Mệnh đề phủ định của mệnh đề $B$ là $\overline{B}\colon$ \lq\lq Phương trình $x^2+3x+2=0$ không có nghiệm\rq\rq.\\ Mệnh đề $\overline{B}$ sai vì phương trình $x^2+3x+2=0$ có hai nghiệm là $x=-1$, $x=-2$.
			\item Mệnh đề phủ định của mệnh đề $C$ là $\overline{C}\colon$ \lq\lq $2^2+2^3\neq 2^{2+3}$\rq\rq.\\ Mệnh đề $\overline{C}$ đúng vì $2^2+2^3=12$ và $2^{2+3}=32$.
			\item Mệnh đề phủ định của mệnh đề $D$ là $\overline{D}\colon$ \lq\lq Số $2025$ không chia hết cho $15$\rq\rq.\\ Mệnh đề $\overline{D}$ sai vì $2025$ chia hết cho $15$.
		\end{enumerate}
	}
\end{vd}
%%%=============================%%%

%%%=============VD_3=============%%%
\begin{vd}%[0D1H1-2]%[Dự án dề cương 3 Khối NH24-25-Đợt 1-Võ Thị Thùy Trang]
	Xét tính đúng sai của các mệnh đề sau và phát biểu mệnh đề phủ định của chúng.
	\begin{enumerate}[a)]
		\item \lq\lq$2019$ chia hết cho $3$\rq\rq;
		\item \lq\lq$\pi<3{,}15$\rq\rq;
		\item \lq\lq Tam giác có hai góc bằng $45^{\circ}$ là tam giác vuông cân\rq\rq.
	\end{enumerate}
	\loigiai{
		\begin{enumerate}[a)]
			\item \lq\lq$2019$ chia hết cho $3$\rq\rq\ là mệnh đề đúng.\\
			Mệnh đề phủ định là \lq\lq$2019$ không chia hết cho $3$\rq\rq.
			\item \lq\lq$\pi<3{,}15$\rq\rq\ là mệnh đề đúng.\\
			Mệnh đề phủ định là \lq\lq $\pi \geq 3{,}15$\rq\rq.
			\item \lq\lq Tam giác có hai góc bằng $45^{\circ}$ là tam giác vuông cân\rq\rq\ là mệnh đề đúng.\\
			Mệnh đề phủ định là \lq\lq Tam giác có hai góc bằng $45^{\circ}$ không là tam giác vuông cân\rq\rq.
		\end{enumerate}
	}
\end{vd}
%%%=============================%%%
\begin{dang}{Xác định mệnh đề kéo theo, mệnh đề đảo, mệnh đề tương đương}
	\textbf{Phương pháp giải:}
	Cho hai mệnh đề $P$ và $Q$.
	\begin{itemize}
		\item \textbf{Mệnh đề kéo theo:} có dạng \lq\lq Nếu $P$ thì $Q$\rq\rq\, ký hiệu là $P \Rightarrow Q$.
		\begin{itemize}
			\item Mệnh đề này chỉ sai khi $P$ đúng và $Q$ sai.
		\end{itemize}
		\item \textbf{Mệnh đề đảo:} là mệnh đề \lq\lq Nếu $Q$ thì $P$\rq\rq ký hiệu là $Q \Rightarrow P$.
		\begin{itemize}
			\item Lưu ý: Tính đúng sai của mệnh đề đảo ($Q \Rightarrow P$) không phụ thuộc vào tính đúng sai của mệnh đề ban đầu ($P \Rightarrow Q$).
		\end{itemize}
		\item \textbf{Mệnh đề tương đương:} có dạng \lq\lq $P$ nếu và chỉ nếu $Q$\rq\rq, ký hiệu là $P \Leftrightarrow Q$.
		\begin{itemize}
			\item Mệnh đề này đúng khi cả hai mệnh đề kéo theo thuận ($P \Rightarrow Q$) và đảo ($Q \Rightarrow P$) đều đúng.
		\end{itemize}
	\end{itemize}
\end{dang}
\setcounter{vd}{0}
%%%=============VD_1=============%%%
\begin{vd}%[0D1N1-4]%[Dự án dề cương 3 Khối NH24-25-Đợt 1-Võ Thị Thùy Trang]
	Cho tam giác $ABC$. Xét hai mệnh đề:\\
	$P$: \lq \lq Tam giác $ABC$ có hai góc bằng $60^\circ$\rq \rq .\\
	$Q$: \lq \lq Tam giác $ABC$ đều\rq \rq. \\
	Hãy phát biểu mệnh đề $P \Rightarrow Q$ và nhận xét tính đúng sai của mệnh đề đó.
	\loigiai{
		$P \Rightarrow Q$: \lq \lq Nếu tam giác $ABC$ có hai góc bằng $60^\circ$ thì tam giác $ABC$ đều\rq \rq.\\
		Mệnh đề kéo theo này là mệnh đề đúng.
	}
\end{vd}
%%%=============================%%%

%%%=============VD_2=============%%%
\begin{vd}%[0D1N1-4]%[Dự án dề cương 3 Khối NH24-25-Đợt 1-Võ Thị Thùy Trang]
	Cho $n$ là số tự nhiên. Xét các mệnh đề\\
	$P\colon$ \lq\lq $n$ là một số tự nhiên chia hết cho $16$\rq\rq,\\
	$Q\colon$ \lq\lq $n$ là một số tự nhiên chia hết cho $8$\rq\rq.
	\begin{enumerate}[a)]
		\item Phát biểu mệnh đề $P\Rightarrow Q$. Nhận xét tính đúng sai của mệnh đề đó.
		\item Phát biểu mệnh đề đảo của mệnh đề $P\Rightarrow Q$. Nhận xét tính đúng sai của mệnh đề đó.
	\end{enumerate}
	\loigiai{
		\begin{enumerate}[a)]
			\item Mệnh đề $P\Rightarrow Q\colon$ \lq\lq Nếu số tự nhiên $n$ chia hết cho $16$ thì $n$ chia hết cho $8$\rq\rq. Đây là mệnh đề đúng vì $8$ là ước của $16$.
			\item Mệnh đề đảo của mệnh đề $P\Rightarrow Q$ là mệnh đề $Q\Rightarrow P\colon$ \lq\lq Nếu số tự nhiên $n$ chia hết cho $8$ thì $n$ chia hết cho $16$\rq\rq. Đây là mệnh đề sai vì với $n=8$, $n$ chia hết cho $8$ nhưng không chia hết cho $16$.
		\end{enumerate}
	}
\end{vd}
%%%=============================%%%

%%%=============VD_3=============%%%
\begin{vd}%[0D1H1-4]%[Dự án dề cương 3 Khối NH24-25-Đợt 1-Võ Thị Thùy Trang]
	Xét hai mệnh đề:\\
	$P\colon$ \lq\lq Tứ giác $A B C D$ là hình bình hành\rq\rq;\\
	$Q\colon$ \lq\lq Tứ giác $A B C D$ có hai đường chéo cắt nhau tại trung điểm của mỗi đường\rq\rq.
	\begin{enumerate}[a)]
		\item Phát biểu mệnh đề $P \Rightarrow Q$ và xét tính đúng sai của nó.
		\item Phát biểu mệnh đề đảo của mệnh đề $P \Rightarrow Q$.
	\end{enumerate}
	\loigiai{
		\begin{enumerate}[a)]
			\item $P \Rightarrow Q\colon$ \lq\lq Nếu tứ giác $ABCD$ là hình bình hành thì tứ giác $ABCD$ có hai đường chéo cắt nhau tại trung điểm mỗi đường\rq\rq.\\
			Mệnh đề $P \Rightarrow Q$ đúng.
			\item $Q \Rightarrow P\colon$ \lq\lq Nếu tứ giác $ABCD$ có hai đường chéo cắt nhau tại trung điểm của mỗi đường thì tứ giác $ABCD$ là hình bình hành\rq\rq.
		\end{enumerate}
	}
\end{vd}
%%%=============================%%%
\begin{dang}{Xác định mệnh đề phủ định của mệnh đề chứa kí hiệu $\forall$ và $\exists$}
	\begin{itemize}
		\item Ta đọc $\forall x\in X$, $\exists x\in X$ lần lượt là với mọi $x$ thuộc $X$, tồn tại (có ít nhất) $x$ thuộc $X$.
		\item Phủ định của mệnh đề \lq\lq$\forall x\in X,P(x )$\rq\rq\,là mệnh đề \lq\lq$\exists x\in X,\overline{P(x)}$\rq\rq.
		\item Phủ định của mệnh đề \lq\lq$\exists x\in X,P(x )$\rq\rq\,là mệnh đề \lq\lq$\forall x\in X,\overline{P(x)}$\rq\rq.
	\end{itemize}
\end{dang}
\setcounter{vd}{0}
%%%=============VD_1=============%%%
\begin{vd}%[0D1H1-5]%[Dự án dề cương 3 Khối NH24-25-Đợt 1-Võ Thị Thùy Trang]
	Lập mệnh đề phủ định của mỗi mệnh đề sau và xét tính đúng sai của mỗi mệnh đề phủ định đó:
	\begin{multicols}{2}
		\begin{enumerate}[a)]
			\item $\forall x \in \mathbb{R},~ x^2 \neq 2x-2$;
			\item $\forall x \in \mathbb{R},~ x^2\leq 2x-1$;
			\item $\exists x \in \mathbb{R},~ x+\dfrac{1}{x} \geq 2$;
			\item $\exists x \in \mathbb{R},~ x^2-x+1< 0$.
		\end{enumerate}
	\end{multicols}
	\loigiai{
		\begin{enumerate}[a)]
			\item Phủ định của mệnh đề \lq\lq $\forall x \in \mathbb{R}, x^2\neq 2x-2$\rq\rq~là mệnh đề \lq\lq $\exists x \in \mathbb{R}, x^2=2x-2$\rq\rq. Mệnh đề phủ định sai vì phương trình $x^2=2x-2$ vô nghiệm nên không có giá trị nào của $x$ thoả mãn $x^2=2x-2$.
			\item Phủ định của mệnh đề \lq\lq $\forall x \in \mathbb{R}, x^2\leq 2x-1$\rq\rq~là mệnh đề \lq\lq $\exists x \in \mathbb{R}, x^2> 2x-1$ \rq\rq. Mệnh đề phủ định đúng vì với $x=2$, ta có $2^2> 2\cdot 2-1$.
			\item Phủ định của mệnh đề \lq\lq $\exists x \in \mathbb{R}, x+\dfrac{1}{x} \geq 2$\rq\rq~là mệnh đề \lq\lq $\forall x \in \mathbb{R}, x+\dfrac{1}{x} < 2$ \rq\rq. Mệnh đề phủ định sai vì với $x=2$, ta có $2+\dfrac{1}{2} > 2$.
			\item Phủ định của mệnh đề \lq\lq $\exists x \in \mathbb{R}, x^2-x+1< 0$\rq\rq~là mệnh đề \lq\lq $\forall x \in \mathbb{R}, x^2-x+1\geq 0$\rq\rq.
			Mệnh đề phủ định đúng vì $x^2-x+1=\left(x-\dfrac{1}{2}\right)^2+\dfrac{3}{4} > 0, \forall x \in \mathbb{R}$.
		\end{enumerate}
	}
\end{vd}
%%%=============================%%%

%%%=============VD_2=============%%%
\begin{vd}%[0D1H1-5]%[Dự án dề cương 3 Khối NH24-25-Đợt 1-Võ Thị Thùy Trang]
	Cho các mệnh đề sau:\\
	$P\colon$ \lq\lq Giá trị tuyệt đối của mọi số thực đều lớn hơn hoặc bằng chính nó\rq\rq;\\
	$Q\colon$ \lq\lq Có số tự nhiên sao cho bình phương của nó bằng $10$ \rq\rq;\\
	$R\colon$ \lq\lq Có số thực $x$ sao cho $x^2+2x-1=0$\rq\rq.
	\begin{enumerate}[a)]
		\item Xét tính đúng sai của mỗi mệnh đề trên.
		\item Sử dụng kí hiệu $\forall$, $\exists$ để viết lại các mệnh đề đã cho.
	\end{enumerate}
	\loigiai{
		\begin{enumerate}[a)]
			\item $P\colon$ Đúng, $Q\colon$ Sai, $R\colon$ Đúng.
			\item $P\colon$ \lq\lq$\forall x\in\mathbb{R}, |x|\geq x$\rq\rq; $Q\colon$ \lq\lq$\exists x\in\mathbb{N}, x^2=10$\rq\rq; $R\colon$ \lq\lq$\exists x\in\mathbb{R}, x^2+2x-1=0$\rq\rq.
		\end{enumerate}
	}
\end{vd}
%%%=============================%%%

%%%=============VD_3=============%%%
\begin{vd}%[0D1H1-5]%[Dự án dề cương 3 Khối NH24-25-Đợt 1-Võ Thị Thùy Trang]
	Xét tính đúng sai và viết mệnh đề phủ định của các mệnh đề sau:
	\begin{multicols}{2}
		\begin{enumerate}[a)]
			\item \lq\lq$\forall x\in\mathbb{R}, x^2+2x+2>0$\rq\rq.
			\item \lq\lq$\exists x\in\mathbb{R}, x^2+3x+4=0$\rq\rq.
		\end{enumerate}
	\end{multicols}
	\loigiai{
		\begin{enumerate}[a)]
			\item Mệnh đề đúng, vì $x^2+2x+2=\left(x^2+2x+1\right)+1=(x+1)^2+1>0$ với mọi số thực $x$.\\
			Mệnh đề phủ định của mệnh đề này là \lq\lq$\exists x\in\mathbb{R}, x^2+2x+2\leq 0$\rq\rq.
			\item Mệnh đề sai, vì phương trình $x^2+3x+4=0$ vô nghiệm $(\Delta=-7<0)$.\\
			Mệnh đề phủ định của mệnh đề này là \lq\lq$\forall x\in\mathbb{R}, x^2+3x+4\neq 0$\rq\rq.
		\end{enumerate}
	}
\end{vd}
%%%=============================%%%
\subsection{Bài tập rèn luyện}
\ind{PHẦN I.} \inden{Câu trắc nghiệm nhiều phương án lựa chọn. Mỗi câu hỏi học sinh chỉ chọn một phương án.}\\
\setcounter{ex}{0}
\Opensolutionfile{ans}[ans/0D1-Bai1-TN]
%%%=============EX_1=============%%%
\begin{ex}[Trích đề ôn tập giữa học kì 1 - Chuyên Lê Quý Đôn Ninh Thuận - Năm học 2024-2025]%[0D1N1-3]%[Dự án dề cương 3 Khối NH24-25-Đợt 1-Võ Thị Thùy Trang]
	Phủ định của mệnh đề \lq\lq $\exists x \in \mathbb{R}, x^2-x+1<0$\rq\rq \,là
	\choice
	{ $\forall x \notin \mathbb{R}, x^2-x+1 \geq 0$}
	{$\exists x \in \mathbb{R}, x^2-x+1 \geq 0$}
	{\True $\forall x \in \mathbb{R}, x^2-x+1 \geq 0$}
	{$\forall x \in \mathbb{R}, x^2-x+1>0$}
	\loigiai{
		Phủ định của mệnh đề \lq\lq $\exists x \in \mathbb{R}, x^2-x+1<0$\rq\rq\,là \lq\lq$\forall x \in \mathbb{R}, x^2-x+1 \geq 0$\rq\rq.
	}
\end{ex}
%%%=============================%%%

%%%=============EX_2=============%%%
\begin{ex}[Trích đề học kì 1 - THPT Nguyễn Thị Minh Khai TpHCM Năm học 24-25]%[0D1N1-3]%[Dự án dề cương 3 Khối NH24-25-Đợt 1-Võ Thị Thùy Trang]
	Mệnh đề phủ định của mệnh đề \lq\lq$\forall n \in \mathbb{N}$, $2^n \geq n+1$\rq\rq\ là
	\choice
	{\True $\exists n \in \mathbb{N}$, $2^n < n+1$}
	{$\forall n \in \mathbb{N}$, $2^n < n+1$}
	{$\exists n \in \mathbb{N}$, $2^n \leq n+1$}
	{$\forall n \in \mathbb{N}$, $2^n \leq n+1$}
	\loigiai{
		Mệnh đề phủ định của mệnh đề \lq\lq$\forall n \in \mathbb{N}$, $2^n \geq n+1$\rq\rq\ là \lq\lq$\exists n \in \mathbb{N}$, $2^n < n+1$\rq\rq.
	}
\end{ex}
%%%=============================%%%

%%%=============EX_3=============%%%
\begin{ex}%[0D1N1-3]%[Dự án dề cương 3 Khối NH24-25-Đợt 1-Võ Thị Thùy Trang]
	Cho mệnh đề $A\colon$ \lq\lq Nghiệm của phương trình $x^2-5=0$ là số hữu tỉ\rq\rq. Mệnh đề phủ định của mệnh đề trên là
	\choice
	{\True \lq\lq Nghiệm của phương trình $x^2-5=0$ không là số hữu tỉ\rq\rq}
	{\lq\lq Nghiệm của phương trình $x^2-5=0$ không là số vô tỉ\rq\rq}
	{\lq\lq Phương trình $x^2-5=0$ vô nghiệm\rq\rq}
	{\lq\lq Nghiệm của phương trình $x^2-5=0$ không là số nguyên\rq\rq}
	\loigiai{
		Mệnh đề phủ định của mệnh đề \lq\lq Nghiệm của phương trình $x^2-5=0$ là số hữu tỉ\rq\rq \ là \lq\lq Nghiệm của phương trình $x^2-5=0$ không là số hữu tỉ\rq\rq.
	}
\end{ex}
%%%=============================%%%

%%%=============EX_4=============%%%
\begin{ex}[Trích đề ôn tập giữa học kì 1 - Chuyên Lê Quý Đôn Ninh Thuận - Năm học 2024-2025]%[0D1N1-2]%[Dự án dề cương 3 Khối NH24-25-Đợt 1-Võ Thị Thùy Trang]
	Cho mệnh đề \lq\lq Nếu hai tam giác bằng nhau thì diện tích chúng bằng nhau\rq\rq. Phát biểu nào sau đây đúng?
	\choice
	{Hai tam giác bằng nhau là điều kiện cần để diện tích chúng bằng nhau}
	{Hai tam giác bằng nhau là điều kiện cần và đủ để chúng có diện tích bằng nhau}
	{Hai tam giác có diện tích bằng nhau là điều kiện đủ để chúng bằng nhau}
	{\True Hai tam giác bằng nhau là điều kiện đủ để diện tích chúng bằng nhau}
	\loigiai{Hai tam giác bằng nhau là điều kiện đủ để diện tích chúng bằng nhau.
	}
\end{ex}
%%%=============================%%%

%%%=============EX_5=============%%%
\begin{ex}[Trích đề ôn tập giữa học kì 1 - Chuyên Lê Quý Đôn Ninh Thuận - Năm học 2024-2025]%[0D1N1-5]%[Dự án dề cương 3 Khối NH24-25-Đợt 1-Võ Thị Thùy Trang]
	Trong các mệnh đề dưới đây, mệnh đề nào đúng?
	\choice
	{$\forall x \in \mathbb{R},-x^2<0$}
	{$\forall x \in \mathbb{N}, x\,\vdots\,3$}
	{\True $\exists x \in \mathbb{R}, x>x^2$}
	{$\exists x \in \mathbb{R}, x^2-x+1 \leq 0$}
	\loigiai{
		Với $x=\dfrac{1}{2}$, ta có $x^2=\dfrac{1}{4}$, khi đó $x> x^2$.}
\end{ex}
%%%=============================%%%

%%%=============EX_6=============%%%
\begin{ex}%[0D1N1-5]%[Dự án dề cương 3 Khối NH24-25-Đợt 1-Võ Thị Thùy Trang]
	Cho tứ giác $ABCD$. Xét mệnh đề \lq\lq Nếu tứ giác $ABCD$ là hình chữ nhật thì tứ giác $ABCD$ có hai đường chéo bằng nhau\rq\rq. Mệnh đề đảo của mệnh đề đó là
	\choice
	{\lq\lq Nếu tứ giác $ABCD$ là hình chữ nhật thì tứ giác $ABCD$ không có hai đường chéo bằng nhau\rq\rq}
	{\lq\lq Nếu tứ giác $ABCD$ không có hai đường chéo bằng nhau thì tứ giác $ABCD$ không là hình chữ nhật\rq\rq}
	{\lq\lq Nếu tứ giác $ABCD$ có hai đường chéo bằng nhau thì tứ giác $ABCD$ không là hình chữ nhật\rq\rq}
	{\True \lq\lq Nếu tứ giác $ABCD$ có hai đường chéo bằng nhau thì tứ giác $ABCD$ là hình chữ nhật\rq\rq}
	\loigiai{
		Mệnh đề đảo cần tìm là \lq\lq Nếu tứ giác $ABCD$ có hai đường chéo bằng nhau thì tứ giác $ABCD$ là hình chữ nhật\rq\rq.
	}
\end{ex}
%%%=============================%%%

%%%=============EX_7=============%%%
\begin{ex}%[0D1N1-5]%[Dự án dề cương 3 Khối NH24-25-Đợt 1-Võ Thị Thùy Trang]
	Phủ định của mệnh đề \lq\lq $\exists x \in \mathbb{R}, x^2-x+1<0$\rq\rq \ là mệnh đề
	\choice
	{\True \lq\lq $\forall x \in \mathbb{R}, x^2-x+1 \geq 0$\rq\rq}
	{\lq\lq $\forall x \in \mathbb{R}, x^2-x+1<0$\rq\rq}
	{\lq\lq $\forall x \in \mathbb{R}, x^2-x+1>0$\rq\rq}
	{\lq\lq $\exists x \in \mathbb{R}, x^2-x+1 \geq 0$\rq\rq}
	\loigiai{
		Phủ định của mệnh đề \lq\lq $\exists x \in \mathbb{R}, x^2-x+1<0$\rq\rq \ là mệnh đề \lq\lq $\forall x \in \mathbb{R}, x^2-x+1 \geq 0$\rq\rq.
	}
\end{ex}
%%%=============================%%%

%%%=============EX_8=============%%%
\begin{ex}%[0D1N1-5]%[Dự án dề cương 3 Khối NH24-25-Đợt 1-Võ Thị Thùy Trang]
	Phủ định của mệnh đề \lq\lq$\exists x \in \mathbb{Q}, x=\dfrac{1}{x}$\rq\rq~là mệnh đề
	\choice
	{\lq\lq $\exists x \in \mathbb{Q}, x \neq \dfrac{1}{x}$\rq\rq}
	{\lq\lq $\forall x \in \mathbb{Q}, x=\dfrac{1}{x}$\rq\rq}
	{\lq\lq $\forall x \notin \mathbb{Q}, x \neq \dfrac{1}{x}$\rq\rq}
	{\True \lq\lq $\forall x \in \mathbb{Q}, x \neq \dfrac{1}{x}$\rq\rq}
	\loigiai{
		Phủ định của mệnh đề \lq\lq$\exists x \in \mathbb{Q}, x=\dfrac{1}{x}$\rq\rq \ là mệnh đề \lq\lq $\forall x \in \mathbb{Q}, x \neq \dfrac{1}{x}$\rq\rq.
	}
\end{ex}
%%%=============================%%%

%%%=============EX_9=============%%%
\begin{ex}%[0D1N1-5]%[Dự án dề cương 3 Khối NH24-25-Đợt 1-Võ Thị Thùy Trang]
	Phủ định của mệnh đề \lq\lq $\forall x \in \mathbb{R}, x^2 \geq 0$\rq\rq \ là mệnh đề
	\choice
	{\lq\lq $\exists x \in \mathbb{R}, x^2 \geq 0$\rq\rq}
	{\lq\lq $\exists x \in \mathbb{R}, x^2>0$\rq\rq}
	{\lq\lq $\exists x \in \mathbb{R}, x^2 \leq 0$\rq\rq}
	{\True \lq\lq $\exists x \in \mathbb{R}, x^2<0$\rq\rq}
	\loigiai{
		Phủ định của mệnh đề \lq\lq $\forall x \in \mathbb{R}, x^2 \geq 0$\rq\rq \ là mệnh đề \lq\lq $\exists x \in \mathbb{R}, x^2<0$\rq\rq.
	}
\end{ex}
%%%=============================%%%

%%%=============EX_10=============%%%
\begin{ex}%[0D1N1-4]%[Dự án dề cương 3 Khối NH24-25-Đợt 1-Võ Thị Thùy Trang]
	Cho số tự nhiên $n$. Xét mệnh đề \lq\lq Nếu số tự nhiên $n$ chia hết cho $4$ thì $n$ chia hết cho $2$\rq\rq. Mệnh đề đảo của mệnh đề đó là
	\choice
	{\lq\lq Nếu số tự nhiên $n$ chia hết cho $2$ thì $n$ không chia hết cho $4$\rq\rq}
	{\lq\lq Nếu số tự nhiên $n$ chia hết cho $4$ thì $n$ không chia hết cho $2$\rq\rq}
	{\True \lq\lq Nếu số tự nhiên $n$ chia hết cho $2$ thì $n$ chia hết cho $4$\rq\rq}
	{\lq\lq Nếu số tự nhiên $n$ không chia hết cho $2$ thì $n$ không chia hết cho 4\rq\rq}
	\loigiai{
		Mệnh đề đảo của mệnh đề \lq\lq Nếu số tự nhiên $n$ chia hết cho $4$ thì $n$ chia hết cho $2$\rq\rq \ là \lq\lq Nếu số tự nhiên $n$ chia hết cho $2$ thì $n$ chia hết cho $4$\rq\rq.
	}
\end{ex}
%%%=============================%%%

%%%=============EX_11=============%%%
\begin{ex}[Trích đề thi giữa học kì 1 - THPT MARIE CURIE - TPHCM - Năm học 2024-2025]%[0D1N1-1]%[Dự án dề cương 3 Khối NH24-25-Đợt 1-Võ Thị Thùy Trang]
	Trong các câu sau, câu nào là mệnh đề?
	\choice
	{Hôm nay là thứ mấy?}
	{Hôm nay trời đẹp quá!}
	{\True Việt Nam là một quốc gia có bờ biển}
	{Hoa hồng đẹp nhất trong các loài hoa}
	\loigiai{
		\lq\lq Việt Nam là một quốc gia có bờ biển\rq\rq\,là một mệnh đề.
	}
\end{ex}
%%%=============================%%%

%%%=============EX_12=============%%%
\begin{ex}[Trích đề ôn tập giữa học kì 1 - Chuyên Lê Quý Đôn Ninh Thuận - Năm học 2024-2025]%[0D1N1-1]%[Dự án dề cương 3 Khối NH24-25-Đợt 1-Võ Thị Thùy Trang]
	Trong các câu sau, câu nào \textbf{không phải} là một mệnh đề toán học?
	\choice
	{\True $16$ có phải là một số chính phương không?}
	{$2+6=8$.}
	{$18$ chia hết cho $5$.}
	{Tam giác đều có $3$ cạnh bằng nhau.}
	\loigiai{\lq\lq$16$ có phải là một số chính phương không?\rq\rq\ là câu hỏi nên không phải mệnh đề.}
\end{ex}
%%%=============================%%%

%%%=============EX_13=============%%%
\begin{ex}[Trích đề giữa học kì 1 - THPT TÂN BÌNH - Tp HCM - Năm học 24-25]%[0D1N1-1]%[Dự án dề cương 3 Khối NH24-25-Đợt 1-Võ Thị Thùy Trang]
	Cho mệnh đề chứa biến \lq\lq$P(n) \colon 3 - n > 0$\rq\rq \, với $n$ là số tự nhiên. Mệnh đề nào sau đây đúng?
	\choice
	{\True $P(2)$}
	{$P(4)$}
	{$P(5)$}
	{$P(3)$}
	\loigiai{
		Thay $n = 2$ vào \lq\lq$P(n) \colon 3 - n > 0$\rq\rq \, ta được $P(2)\colon 3 - 2 > 0 \Rightarrow$ $P(2)$ đúng.
	}
\end{ex}
%%%=============================%%%

%%%=============EX_14=============%%%
\begin{ex}[Trích đề GHKI - THPT Tây Thạnh - NH24-25]%[0D1N1-1]%[Dự án dề cương 3 Khối NH24-25-Đợt 1-Võ Thị Thùy Trang]
	Trong các câu sau, câu nào \textbf{không phải} là mệnh đề?
	\choice
	{$2017$ là số lẻ}
	{Hôm nay là thứ hai}
	{\True Hôm nay trời đẹp quá!}
	{Hình bình hành là đa giác có $3$ cạnh}
	\loigiai{
		Phát biểu \lq\lq Hôm nay trời đẹp quá!\rq\rq\ không phải là một mệnh đề.
	}
\end{ex}
%%%=============================%%%

%%%=============EX_15=============%%%
\begin{ex}%[0D1N1-1]%[Dự án dề cương 3 Khối NH24-25-Đợt 1-Võ Thị Thùy Trang]
	Phát biểu nào sau đây là một mệnh đề?
	\def\dotEX{}
	\choice
	{Đề thi môn Toán khó quá!}
	{\True Hà Nội là thủ đô của Việt Nam.}
	{Bạn có đi học không?}
	{Mùa thu Hà Nội đẹp quá!}
	\loigiai
	{Mệnh đề là \lq\lq Hà Nội là thủ đô của Việt Nam\rq\rq.\\
		Đây là một mệnh đề đúng.}
\end{ex}
%%%=============================%%%

%%%=============EX_16=============%%%
\begin{ex}[Trích đề giữa học kì 1 - THPT TÂN BÌNH - Tp HCM - Năm học 24-25]%[0D1H1-4]%[Dự án dề cương 3 Khối NH24-25-Đợt 1-Võ Thị Thùy Trang]
	Biết rằng $P \Rightarrow Q$ là mệnh đề đúng. Mệnh đề nào sau đây đúng?
	\choice
	{\True $Q$ là điều kiện cần để có $P$}
	{$Q$ là điều kiện đủ để có $P$}
	{$P$ là điều kiện cần để có $Q$}
	{$Q$ là điều kiện cần và đủ để có $P$}
	\loigiai{
		Mệnh đề $Q$ là điều kiện cần và đủ để có $P$ là phát biểu đúng nhất trong trường hợp $P \Rightarrow Q$ đúng.
	}
\end{ex}
%%%=============================%%%

%%%=============EX_17=============%%%
\begin{ex}[Trích đề giữa học kì 1 - THPT TÂN BÌNH - Tp HCM - Năm học 24-25]%[0D1H1-1]%[Dự án dề cương 3 Khối NH24-25-Đợt 1-Võ Thị Thùy Trang]
	Trong các câu sau có bao nhiêu câu là mệnh đề?\\
	(1)$\colon$ Số $3$ là một số chẵn.\\
	(2)$\colon$ $2x+1=3$.\\
	(3)$\colon$ Các em hãy cố gắng làm bài thi cho tốt nhé!\\
	(4)$\colon$ $1<5\Rightarrow 8<6$.
	\choice
	{\True$2$}
	{$3$}
	{$1$}
	{$4$}
	\loigiai{
		Các ý (1) và (4) là mệnh đề.}
\end{ex}
%%%=============================%%%

%%%=============EX_18=============%%%
\begin{ex}[Trích đề học kì 1 - THPT Nguyễn Thị Minh Khai TpHCM Năm học 24-25]%[0D1H1-2]%[Dự án dề cương 3 Khối NH24-25-Đợt 1-Võ Thị Thùy Trang]
	Xét các khẳng định sau
	\begin{enumerate}[1)]
		\item $2024$ chia hết cho $3$.
		\item $\pi$ là số vô tỉ.
		\item Phương trình $3x^2-4x+1=0$ có nghiệm nguyên.
		\item Nếu hai tam giác có diện tích bằng nhau thì hai tam giác đó bằng nhau.
	\end{enumerate}
	Trong các khẳng định trên, hỏi có tất cả bao nhiêu mệnh đề đúng?
	\choice
	{$1$}
	{$3$}
	{\True $2$}
	{$4$}
	\loigiai{
		\begin{enumerate}
			\item $2024$ chia hết cho $3$ là khẳng định sai vì $2024:3=674$ dư $2$.
			\item $\pi$ là số vô tỉ là khẳng định đúng.
			\item Phương trình $3x^2-4x+1=0$ có nghiệm $x=1$ và $x=\dfrac{1}{3}$ nên phương trình $3x^2-4x+1=0$ có nghiệm nguyên là khẳng định đúng.
			\item Nếu hai tam giác có diện tích bằng nhau thì hai tam giác đó bằng nhau là khẳng định sai.
		\end{enumerate}
		Vậy có $2$ khẳng định đúng.
	}
\end{ex}
%%%=============================%%%

%%%=============EX_19=============%%%
\begin{ex}%[0D1H1-2]%[Dự án dề cương 3 Khối NH24-25-Đợt 1-Võ Thị Thùy Trang]
	Cho mệnh đề $P(n)\colon$\lq\lq$n^2+n+1$ là số chia hết cho $3$\rq\rq~($n\in\mathbb{N}$). Mệnh đề nào dưới đây đúng?
	\choice
	{\True $P(1)$}
	{$P(2)$}
	{$P(3)$}
	{$P(2)$}
	\loigiai
	{Ta có $P(1)\colon$\lq\lq$3$ chia hết cho $3$\rq\rq~là mệnh đề đúng.}
\end{ex}
%%%=============================%%%

%%%=============EX_20=============%%%
\begin{ex}%[0D1H1-2]%[Dự án dề cương 3 Khối NH24-25-Đợt 1-Võ Thị Thùy Trang]
	Cho mệnh đề chứa biến $P(x) \colon$\lq\lq$2x^2-1<0$\rq\rq. Mệnh đề đúng là
	\choice
	{$P(-1)$}
	{\True $P(0)$}
	{$P(-2)$}
	{$P(1)$}
	\loigiai
	{
		\begin{itemize}
			\item $P(-1) \colon$\lq\lq$2\cdot(-1)^2-1<0$\rq\rq~là mệnh đề sai.
			\item $P(0) \colon$\lq\lq$2\cdot 0^2-1<0$\rq\rq~là mệnh đề đúng.
			\item $P(-2) \colon$\lq\lq$2\cdot (-2)^2-1<0$\rq\rq~là mệnh đề sai.
			\item $P(1) \colon$\lq\lq$2\cdot 1^2-1<0$\rq\rq~là mệnh đề sai.
		\end{itemize}
	}
\end{ex}
%%%=============================%%%
\Closesolutionfile{ans}
\ind{PHẦN II.} \inden{Câu trắc nghiệm đúng sai. Trong mỗi ý a), b), c), d) ở mỗi câu, học sinh chọn đúng hoặc sai.}\\
\setcounter{ex}{0}
\Opensolutionfile{ans}[ans/0D1-Bai1-DS]%--Đặt tên 2D1-Bai1-DS
%%%=============EX_1=============%%%
\begin{ex}%[0D1H1-2]%[Dự án dề cương 3 Khối NH24-25-Đợt 1-Võ Thị Thùy Trang]
	Xét các câu sau đây
	\begin{itemize}
		\item Ở đây đẹp quá!
		\item Phương trình $x^2-3x+1=0$ vô nghiệm.
		\item $16$ không là số nguyên tố.
		\item Số $\pi $ có lớn hơn $3$ hay không?
	\end{itemize}
	\choiceTF
	{Trong các câu trên có $3$ mệnh đề}
	{\True Trong các câu trên có $2$ câu không phải là mệnh đề}
	{\True $16$ không là số nguyên tố là mệnh đề}
	{\True Số $\pi $ có lớn hơn $3$ hay không? không phải là mệnh đề}
	\loigiai{
		\begin{itemchoice}
			\itemch Vì \lq\lq Phương trình $x^2-3x+1=0$ vô nghiệm\rq\rq\, là mệnh đề và \lq\lq $16$ không là số nguyên tố\rq\rq\, là mệnh đề nên trong các câu trên có $2$ mệnh đề.
			\itemch Vì \lq\lq Ở đây đẹp quá!\rq\rq\, không là mệnh đề và \lq\lq Số $\pi $ có lớn hơn $3$ hay không?\rq\rq\,không là mệnh đề nên trong các câu trên có $2$ câu không phải là mệnh đề.
			\itemch \lq\lq$16$ không là số nguyên tố\rq\rq là mệnh đề đúng.
			\itemch Câu hỏi không phải là mệnh đề.
		\end{itemchoice}
	}
\end{ex}
%%%=============================%%%

%%%=============EX_2=============%%%
\begin{ex}%[0D1H1-2]%[Dự án dề cương 3 Khối NH24-25-Đợt 1-Võ Thị Thùy Trang]
	Cho các phát biểu sau
	\begin{multicols}{2}
		\begin{itemize}
			\item $x\in\mathbb{Z},~ 2x<3 \quad(1)$
			\item $x\in\mathbb{Z},~ x^4-x^2<0\quad(2)$.
		\end{itemize}
	\end{multicols}
	\choiceTF
	{\True $(1)$ là mệnh đề chứa biến}
	{\True Khi $x=1$ thì $(1)$ trở thành mệnh đề đúng}
	{Khi $x=-2$ thì $(2)$ trở thành mệnh đề đúng}
	{\True Không có số nguyên $x$ nào để cả $(1)$ và $(2)$ trở thành các mệnh đề đúng}
	\loigiai{
		\begin{itemchoice}
			\itemch $(1)$ là mệnh đề chứa biến.
			\itemch Vì $2\cdot 1<3$ đúng.
			\itemch Vì $(-2)^4-(-2)^2<0$ sai.
			\itemch Ta có $x^4-x^2<0\Leftrightarrow{x^2}\left(x^2-1\right)<0\Leftrightarrow \heva{& x\neq 0\\
				&x^2<1\\
			}\Leftrightarrow\heva{& x\neq 0\\
				&-1<x<1.\\
			}$\\
			Mà $x\in\mathbb{Z}$ nên không có giá trị nào của $x$ làm cho đúng do đó \lq\lq Không có số nguyên $x$ nào để cả $(1)$ và $(2)$ trở thành các mệnh đề đúng\rq\rq.
		\end{itemchoice}
	}
\end{ex}
%%%=============================%%%

%%%=============EX_3=============%%%
\begin{ex}%[0D1H1-2]%[Dự án dề cương 3 Khối NH24-25-Đợt 1-Võ Thị Thùy Trang]
	Cho $P(n)=n^2-6n+10$ với $n$ là số tự nhiên.
	\choiceTF
	{$P(1)$ chia hết cho $3$}
	{$P(2)$ là số lẻ}
	{$P(2n)>P(n)-1$ với $n=1$}
	{\True Tồn tại số tự nhiên $n$ thỏa mãn điều kiện $\dfrac{2P(n)-1}{n-3}$ là số nguyên}
	\loigiai{
		\begin{itemchoice}
			\itemch Vì $P(1)=5$ không chia hết cho $3$.
			\itemch Vì $P(2)=2$ là số chẵn.
			\itemch Vì
			\begin{itemize}
				\item
				$P(2n)>P(n)-1\Leftrightarrow 4n^2-12n+10>n^2-6n+10-1\Leftrightarrow 3n^2-6n+1>0$.
				\item Khi $n=1\Rightarrow$ VT$=3\cdot 1^2-6\cdot 1+1=-2<0$.
			\end{itemize}
			\itemch Vì
			$\dfrac{2P(n)-1}{n-3}=\dfrac{2\left(n^2-6n+10\right)-1}{n-3}=\dfrac{2\left(n-3\right)^2+1}{n-3}=2\left(n-3\right)+\dfrac{1}{n-3}$.\\
			Suy ra $\dfrac{2P(n)-1}{n-3}\in\mathbb{N}\Leftrightarrow\left(n-3\right)$ là ước của $1\Leftrightarrow n-3=1\Leftrightarrow n=4$.
		\end{itemchoice}
	}
\end{ex}
%%%=============================%%%

%%%=============EX_4=============%%%
\begin{ex}%[0D1V1-2]%[Dự án dề cương 3 Khối NH24-25-Đợt 1-Võ Thị Thùy Trang]
	Cho hai số thực $a$ và $b$.
	\choiceTF
	{$a^2>b^2\Leftrightarrow a>b$}
	{\True $a^3>b^3\Leftrightarrow a>b$}
	{ $a^2+b^2$ chia hết cho $3$ khi và chỉ khi cả hai số $a$ và $b$ cùng chia hết cho $3$}
	{$a+b>2$ khi và chỉ khi ít nhất một trong hai số $a$, $b$ lớn hơn $1$}
	\loigiai{
		\begin{itemchoice}
			\itemch Vì $-2>-3$ nhưng $(-2)^2<(-3)^2$.
			\itemch Vì $a^3-b^3=(a-b)(a^2+ab+b^2)$ và $a^2+ab+b^2=\left(a+\dfrac{b}{2}\right)^2+\dfrac{3b^2}{4}>0$ với $a\ne b$ nên nếu $a>b$ thì $a^3>b^3$ và ngược lại.
			\itemch Vì $\left(\sqrt{7}\right)^2+\left(\sqrt{2}\right)^2=9$ chia hết cho $3$ nhưng $\sqrt{7}$ và $\sqrt{2}$ không chia hết cho $3$.
			\itemch Vì với $a=3$, $b=-1$ thì $a+b=2$.
		\end{itemchoice}
	}
\end{ex}
%%%=============================%%%

%%%=============EX_5=============%%%
\begin{ex}%[0D1V1-3]%[Dự án dề cương 3 Khối NH24-25-Đợt 1-Võ Thị Thùy Trang]
	Một số nguyên dương $n$ được gọi là {\bf{\lq\lq số hoàn hảo\rq\rq}} nếu số đó bằng tổng các ước nguyên dương thực sự của nó. Ví dụ số $6$ là một số hoàn hảo vì các ước nguyên dương thực sự của $6$ là $1$; $2$; $3$ và $6=1+2+3$.
	\choiceTF
	{Không có số hoàn hảo nào nhỏ hơn $10$}
	{\True Số $10$ là một số không hoàn hảo}
	{\True Tất cả các số nguyên tố đều là các số không hoàn hảo}
	{\True Số $2\,020$ không phải là một số hoàn hảo}
	\loigiai{
		\begin{itemchoice}
			\itemch Vì số $6$ là một số hoàn hảo vì các ước nguyên dương thực sự của $6$ là $1$; $2$; $3$ và $6=1+2+3$.
			\itemch Số $10$ là một số không hoàn hảo vì các ước nguyên dương thực sự của $10$ là $1$; $2$; $5$ và $1+2+5=8$.
			\itemch Vì tất cả số nguyên tố có ước nguyên dương thực sự là $1$.
			\itemch Ta thấy các ước nguyên dương thực sự của số $2\,020=2^2\cdot 5\cdot 101$ là $1$; $2$; $4$; $5$; $10$; $20$; $101$; $202$;	$404$; $505$; $1\,010$ và đồng thời \[1+2+4+5+10+20+101+202+404+505+1010 > 2020.\] nên số $2\,020$ không phải là số hoàn hảo.
		\end{itemchoice}
	}
\end{ex}
%%%=============================%%%
\Closesolutionfile{ans}
\ind{PHẦN III.} \inden{Câu trắc nghiệm trả lời ngắn}\\
\setcounter{ex}{0}
\Opensolutionfile{ans}[ans/0D1-Bai1-TLN]
%%%=============EX_1=============%%%
\begin{ex}%[0D1V1-5]%[Dự án dề cương 3 Khối NH24-25-Đợt 1-Võ Thị Thùy Trang]
	Xác định số mệnh đề \textbf{sai} trong các mệnh đề sau?
	\begin{multicols}{2}
		\begin{itemize}
			\item \lq\lq$\exists x\in\mathbb{R}\colon x^2-3x+2=0$\rq\rq.
			\item \lq\lq$\forall x\in\mathbb{R}\colon x^2\geq 0$\rq\rq.
			\item \lq\lq$\exists n\in\mathbb{N}\colon n^2=n$\rq\rq.
			\item \lq\lq$\forall n\in\mathbb{N}\colon n<2n$\rq\rq.
		\end{itemize}
	\end{multicols}
	\par
	\shortans[oly]{1}
	\loigiai{
		\begin{itemize}
			\item Ta có
			\[x^2-3x+2=0\Leftrightarrow \hoac{&x=2\\
				&x=1.
			}\]
			nên mệnh đề \lq\lq$\exists x\in\mathbb{R}\colon x^2-3x+2=0$\rq\rq~là mệnh đề đúng.
			\item $x^2\geq 0$ với mọi $x\in\mathbb{R}$ nên mệnh đề \lq\lq$\forall x\in\mathbb{R}\colon x^2\geq 0$\rq\rq~là mệnh đề đúng.
			\item Với $n=0$ thì $n^2=n$ nên mệnh đề \lq\lq$\exists n\in\mathbb{N}\colon n^2=n$\rq\rq~là mệnh đề đúng.
			\item Với $n=0$ thì $n=2n$ nên mệnh đề \lq\lq$\forall n\in\mathbb{N}\colon n<2n$\rq\rq~là mệnh đề sai.
		\end{itemize}
	}
\end{ex}
%%%=============================%%%

%%%=============EX_2=============%%%
\begin{ex}%[0D1V1-1]%[Dự án dề cương 3 Khối NH24-25-Đợt 1-Võ Thị Thùy Trang]
	Với giá trị nào của $x$ thì \lq\lq$x\in\mathbb{N}, x^2-1=0$\rq\rq~là mệnh đề đúng?
	\par
	\shortans[oly]{1}
	\loigiai
	{Vì $x\in\mathbb{N}$ mà $x^2-1=0$ nên $x=1$.}
\end{ex}
%%%=============================%%%

%%%=============EX_3=============%%%
\begin{ex}%[0D1V1-5]%[Dự án dề cương 3 Khối NH24-25-Đợt 1-Võ Thị Thùy Trang]
	Xác định số mệnh đề đúng trong các mệnh đề sau?
	\begin{multicols}{2}
		\begin{itemize}
			\item \lq\lq$\exists n\in \mathbb{N}$, $n^3-n$ không chia hết cho $3$\rq\rq.
			\item \lq\lq$\forall x\in \mathbb{R}$, $x<4\Leftrightarrow x^2<16$\rq\rq.
			\item \lq\lq$\exists k\in \mathbb{Z}$, $k^2+k+1$ là một số chẵn\rq\rq.
			\item \lq\lq$\forall x\in \mathbb{Z},\,\dfrac{2x^3-6x^2+x-3}{2x^2+1}\in \mathbb{Z}$\rq\rq.
		\end{itemize}
	\end{multicols}
	\par
	\shortans[oly]{1}	
	\loigiai
	{\begin{itemize}
			\item Với $n\in \mathbb{N}$ thì $n^3-n=(n-1)n(n+1)$ là tích của ba số tự nhiên liên tiếp nên luôn chia hết cho $3$ nên mệnh đề \lq\lq$\exists n\in \mathbb{N}$, $n^3-n$ không chia hết cho $3$\rq\rq~là mệnh đề sai.
			\item Với $x=-5$ thì $x<4$ nhưng $x^2=25>16$ nên mệnh đề \lq\lq$\forall x\in \mathbb{R}$, $x<4\Leftrightarrow x^2<16$\rq\rq~là một mệnh đề sai.
			\item Với $k\in \mathbb{Z}$ thì $k^2+k=k(k+1)$ là tích của hai số nguyên liên tiếp nên $k^2+k$ là số chẵn, suy ra $k^2+k+1$ là số lẻ với mọi $k\in \mathbb{Z}$. Vậy mệnh đề \lq\lq$\exists k\in \mathbb{Z}$, $k^2+k+1$ là một số chẵn\rq\rq~là một mệnh đề sai.
			\item Ta có $\dfrac{2x^3-6x^2+x-3}{2x^2+1}=\dfrac{2x^2(x-3)+(x-3)}{2x^2+1}=\dfrac{(x-3)(2x^2+1)}{2x^2+1}=x-3$.\\
			Vậy $x\in \mathbb{Z}$ thì $x-3\in \mathbb{Z}$ hay $\dfrac{2x^3-6x^2+x-3}{2x^2+1}\in \mathbb{Z}$ là mệnh đề đúng.
	\end{itemize}	}
\end{ex}
%%%=============================%%%

%%%=============EX_4=============%%%
\begin{ex}%[0D1V1-5]%[Dự án dề cương 3 Khối NH24-25-Đợt 1-Võ Thị Thùy Trang]
	Xác định số mệnh đề \textbf{sai} trong các mệnh đề sau?
	\begin{multicols}{2}
		\begin{itemize}
			\item \lq\lq$\exists x \in\mathbb{Q}, 9x^2-1=0$\rq\rq.
			\item \lq\lq$\forall x \in \mathbb{N}, x < \dfrac{1}{x}$\rq\rq.
			\item \lq\lq$\forall x \in \mathbb{R}, x^2+2>0$\rq\rq.
			\item \lq\lq$\exists x\in \mathbb{Z}, x^2-3x+2=0$\rq\rq.
		\end{itemize}
	\end{multicols}
	\par
	\shortans[oly]{1}	
	\loigiai
	{\begin{itemize}
			\item Ta có $9x^2-1=0$ có nghiệm $x = \pm \dfrac{1}{3}$.\\
			Vì $\pm \dfrac{1}{3} \in \mathbb{Q}$
			nên mệnh đề \lq\lq$\exists x \in\mathbb{Q}, 9x^2-1=0 $\rq\rq~là mệnh đề đúng.
			\item Do $x=0$ không thỏa mãn điều kiện xác định của bất phương trình $x<\dfrac{1}{x}$ nên mệnh đề \lq\lq$\forall x \in \mathbb{N}, x < \dfrac{1}{x}$\rq\rq~là mệnh đề sai.
			\item Ta có $x^2 \ge 0$ với mọi $x \in \mathbb{R}$.\\
			Suy ra $x^2+2>0$ với mọi $x \in \mathbb{R}$.\\
			Do đó mệnh đề \lq\lq$\forall x \in \mathbb{R}, x^2+2>0$\rq\rq~là mệnh đề đúng.
			\item Ta có $x^2-3x+2=0$ có hai nghiệm phân biệt $x=1$, $x=2$.\\
			Vì $1\in \mathbb{Z}, 2 \in \mathbb{Z}$ nên mệnh đề \lq\lq$\exists x\in \mathbb{Z}, x^2-3x+2=0$\rq\rq~là mệnh đề đúng.
		\end{itemize}
	}
\end{ex}
%%%=============================%%%

%%%=============EX_5=============%%%
\begin{ex}%[0D1V1-5]%[Dự án dề cương 3 Khối NH24-25-Đợt 1-Võ Thị Thùy Trang]
	Xác định số mệnh đề đúng trong các mệnh đề sau?
	\begin{multicols}{2}
		\begin{itemize}
			\item \lq\lq$\forall n \in \mathbb{N}\colon n \leq 2n$\rq\rq.
			\item \lq\lq$\exists n \in \mathbb{N} \colon n^2=n$\rq\rq.
			\item \lq\lq$\forall x \in \mathbb{R}\colon x^2>0$\rq\rq.
			\item \lq\lq$\exists x \in \mathbb{R}\colon x<1$\rq\rq.
		\end{itemize}
	\end{multicols}
	\par
	\shortans[oly]{3}
	\loigiai
	{
		Mệnh đề \lq\lq$\forall x \in \mathbb{R}\colon x^2>0$\rq\rq~là mệnh đề sai chẳng hạn với $x=0$.
	}
\end{ex}
%%%=============================%%%
\Closesolutionfile{ans}
\ind{PHẦN IV.} \inden{Tự luận.}
\setcounter{ex}{0}
%%%=============EX_1=============%%%
\begin{ex}%[0D1N1-3]%[Dự án dề cương 3 Khối NH24-25-Đợt 1-Võ Thị Thùy Trang]
	Phát biểu mệnh đề phủ định của các mệnh đề sau:
	\begin{enumerate}[a)]
		\item $P\colon$ \lq\lq Tháng $12$ dương lịch có $30$ ngày\rq\rq;
		\item $Q\colon$ \lq\lq $9^{10}\ge 10^9$\rq\rq;
		\item $R\colon$ \lq\lq Phương trình $x^2+1=0$ có nghiệm\rq\rq.
	\end{enumerate}
	\loigiai{Mệnh đề phủ định của các mệnh đề trên là
		\begin{enumerate}[a)]
			\item $\overline P\colon$ \lq\lq Không phải tháng $12$ dương lịch có $30$ ngày\rq\rq;
			\item $\overline Q\colon$ \lq\lq $9^{10}<10^9$\rq\rq;
			\item $\overline R\colon$ \lq\lq Phương trình $x^2+1=0$ vô nghiệm\rq\rq.
		\end{enumerate}
	}
\end{ex}
%%%=============================%%%

%%%=============EX_2=============%%%
\begin{ex}%[0D1N1-1]%[Dự án dề cương 3 Khối NH24-25-Đợt 1-Võ Thị Thùy Trang]
	Trong các câu sau đây, câu nào là mệnh đề?
	\begin{multicols}{2}
		\begin{enumerate}[a)]
			\item \lq\lq$3$ là số lẻ\rq\rq;
			\item \lq\lq$1+2>3$\rq\rq;
			\item \lq\lq$\pi$ là số vô tỉ phải không?\rq\rq
			\item \lq\lq$0{,}0001$ là số rất bé\rq\rq;
			\item \lq\lq Đến năm $2050$, con người sẽ đặt chân lên Sao Hỏa\rq\rq.
		\end{enumerate}
	\end{multicols}
	\loigiai{
		\begin{enumerate}[a)]
			\item \lq\lq $3$ là số lẻ\rq\rq~là mệnh đề (là mệnh đề đúng).
			\item \lq\lq$1+2>3$\rq\rq~là mệnh đề (là mệnh đề sai).
			\item \lq\lq$\pi$ là một số vô tỉ phải không?\rq\rq~là câu hỏi, không phải mệnh đề.
			\item Câu \lq\lq $0{,}0001$ là số rất bé\rq\rq~không có tính hoặc đúng hoặc sai (do không đưa ra tiêu chí thế nào là số rất bé). Do đó, nó không phải là mệnh đề.
			\item \lq\lq Đến năm $2050$, con người sẽ đặt chân lên Sao Hỏa\rq\rq ~là một khẳng định chưa thể chắc chắn là đúng hay sai. Tuy nhiên, nó chắc chắn chỉ có thể hoặc đúng hoặc sai. Do đó, nó là một mệnh đề.
		\end{enumerate}
	}
\end{ex}
%%%=============================%%%

%%%=============EX_3=============%%%
\begin{ex}%[0D1N1-1]%[Dự án dề cương 3 Khối NH24-25-Đợt 1-Võ Thị Thùy Trang]
	Trong các khẳng định sau, khẳng định nào là mệnh đề, khẳng định nào là mệnh đề chứa biến?
	\begin{multicols}{4}
		\begin{enumerate}[a)]
			\item \lq\lq$3+2>5$\rq\rq;
			\item \lq\lq$1-2x=0$\rq\rq;
			\item \lq\lq$x-y=2$\rq\rq;
			\item \lq\lq$1-\sqrt 2<0$\rq\rq.
		\end{enumerate}
	\end{multicols}
	\loigiai{
		\begin{enumerate}[a)]
			\item \lq\lq$3+2>5$\rq\rq là mệnh đề.
			\item \lq\lq$1-2x=0$\rq\rq là mệnh đề chứa biến.
			\item \lq\lq$x-y=2$\rq\rq là mệnh đề chứa biến.
			\item \lq\lq$1-\sqrt 2<0$\rq\rq là mệnh đề.
		\end{enumerate}
	}
\end{ex}
%%%=============================%%%

%%%=============EX_4=============%%%
\begin{ex}%[0D1H1-5]%[Dự án dề cương 3 Khối NH24-25-Đợt 1-Võ Thị Thùy Trang]
	Xét tính đúng sai và viết mệnh đề phủ định của các mệnh đề sau đây:
	\begin{multicols}{3}
		\begin{enumerate}[a)]
			\item \lq\lq$\exists x\in\mathbb{N}, x+3=0$\rq\rq;
			\item \lq\lq$\forall x\in\mathbb{R}, x^2+1\geq 2x$\rq\rq;
			\item \lq\lq$\forall a\in\mathbb{R},\sqrt{a^2}=a$\rq\rq.
		\end{enumerate}
	\end{multicols}
	\loigiai{
		\begin{enumerate}[a)]
			\item \lq\lq$\exists x\in\mathbb{N}, x+3=0$\rq\rq ~là mệnh đề sai. Mệnh đề phủ định là \lq\lq$\forall x\in\mathbb{N}, x+3\neq 0$\rq\rq.
			\item \lq\lq$\forall x\in\mathbb{R}, x^2+1\geq 2x$\rq\rq~là mệnh đề đúng. Mệnh đề phủ định là \lq\lq$\exists x\in\mathbb{R}, x^2+1<2x$\rq\rq.
			\item \lq\lq$\forall a\in\mathbb{R},\sqrt{a^2}=a$\rq\rq~là mệnh đề sai. Mệnh đề phủ định là \lq\lq$\exists a\in\mathbb{R},\sqrt{a^2}\neq a$\rq\rq.
		\end{enumerate}
	}
\end{ex}
%%%=============================%%%

%%%=============EX_5=============%%%
\begin{ex}%[0D1H1-4]%[Dự án dề cương 3 Khối NH24-25-Đợt 1-Võ Thị Thùy Trang]
	Sử dụng các thuật ngữ \lq\lq điều kiện cần\rq\rq, \lq\lq điều kiện đủ\rq\rq~để phát biểu lại định lí: \lq\lq Nếu tứ giác $ABCD$ là hình chữ nhật thì hai đường chéo bằng nhau\rq\rq.
	\loigiai{
		Ta có thể phát biểu lại định lí đã cho như sau:\\
		\lq\lq Tứ giác $ABCD$ có hai đường chéo bằng nhau là điều kiện cần để nó là hình chữ nhật\rq\rq~hoặc \lq\lq Tứ giác $ABCD$ là hình chữ nhật là điều kiện đủ để hai đường chéo bằng nhau\rq\rq.
	}
\end{ex}
%%%=============================%%%

%%%=============EX_6=============%%%
\begin{ex}%[0D1H1-4]%[Dự án dề cương 3 Khối NH24-25-Đợt 1-Võ Thị Thùy Trang]
	Xét hai mệnh đề:\\
	$P\colon$\lq\lq Tam giác $ABC$ vuông tại $A$\rq\rq;\\
	$Q\colon$\lq\lq Tam giác $ABC$ có $AB^{2}+AC^{2}=BC^{2}$\rq\rq.\\
	Hai mệnh đề $P$ và $Q$ có tương đương không? Nếu có, hãy phát biểu một định lí thể hiện điều này, trong đó có sử dụng thuật ngữ \lq\lq khi và chỉ khi\rq\rq~hoặc \lq\lq điều kiện cần và đủ\rq\rq.
	\loigiai{
		Theo định lí Pythagore, hai mệnh đề $P\Rightarrow Q$ và $Q \Rightarrow P$ đều đúng. Do đó, $P$ và $Q$ là hai mệnh đề tương đương. Ta có thể phát biểu thành định lí như sau:\\
		\lq\lq Tam giác $A B C$ vuông tại $A$ khi và chỉ khi $AB^{2}+AC^{2}=BC^{2}$\rq\rq\\
		hoặc \lq\lq Để tam giác $A B C$ vuông tại $A$, điều kiện cần và đủ là $AB^2+AC^2=B C^2$\rq\rq.
	}
\end{ex}
%%%=============================%%%

%%%=============EX_7=============%%%
\begin{ex}%[0D1H1-4]%[Dự án dề cương 3 Khối NH24-25-Đợt 1-Võ Thị Thùy Trang]
	Cho các định lí:\\
	$P\colon$ \lq\lq Nếu hai tam giác bằng nhau thì diện tích của chúng bằng nhau\rq\rq;\\
	$Q\colon$ \lq\lq Nếu $a<b$ thì $a+c<b+c$ \rq\rq~($a$, $b$, $c \in \mathbb{R}$).
	\begin{enumerate}[a)]
		\item Chỉ ra giả thiết và kết luận của mỗi định lí.
		\item Phát biểu lại mỗi định lí đã cho, sử dụng thuật ngữ \lq\lq điều kiện cần\rq\rq~ hoặc \lq\lq ~điều kiện đủ\rq\rq.
		\item Mệnh đề đảo của mỗi định lí đó có là định lí không?
	\end{enumerate}
	\loigiai{
		\begin{enumerate}[a)]
			\item Mệnh đề $P$.\\
			Giả thiết: hai tam giác bằng nhau.\\
			Kết luận: diện tích của chúng bằng nhau.\\
			Mệnh đề $Q$.\\
			Giả thiết: $a<b$.\\
			Kết luận: $a+c<b+c$ ($a$, $b$, $c \in \mathbb{R}$).
			\item Phát biểu định lí:
			\begin{itemize}
				\item $P\colon$\lq\lq Hai tam giác bằng nhau là điều kiện đủ để diện tích của chúng bằng nhau\rq\rq~hoặc~\lq\lq Diện tích của hai tam giác bằng nhau là điều kiện cần để hai tam giác bằng nhau\rq\rq.
				\item $Q\colon$ \lq\lq $a<b$ là điều kiện đủ để $a+c<b+c$ ($a$, $b$, $c \in \mathbb{R}$)\rq\rq~hoặc~\lq\lq $a+c<b+c$ ($a$, $b$, $c \in \mathbb{R}$)\rq\rq~là điều kiện cần để \lq\lq$a<b$\rq\rq.
			\end{itemize}
			\item Mệnh đề đảo của mỗi định lí là mệnh đề sai nên không là định lí.
		\end{enumerate}
		
	}
\end{ex}
%%%=============================%%%

%%%=============EX_8=============%%%
\begin{ex}%[0D1H1-2]%[Dự án dề cương 3 Khối NH24-25-Đợt 1-Võ Thị Thùy Trang]
	Cho các mệnh đề chứa biến:
	\begin{enumerate}[a)]
		\item $P(x)\colon$\lq\lq $2x=1$\rq\rq;
		\item $R(x,y)\colon$\lq\lq $2x+y=3 $ \rq\rq~(mệnh đề này chứa hai biến $x$ và $y$);
		\item $T(n)\colon$\lq\lq $2n+1$ là số chẵn\rq\rq~($n$ là số tự nhiên).
	\end{enumerate}
	Với mỗi mệnh đề chứa biến trên, tìm những giá trị của biến để nhận được một mệnh đề đúng và một mệnh đề sai.
	\loigiai{
		\begin{enumerate}[a)]
			\item Với $x=\dfrac{1}{2}$ thì $P\left(\dfrac{1}{2}\right)\colon$
			\lq\lq$2\cdot\dfrac{1}{2}=1$\rq\rq~là mệnh đề đúng.\\
			Với $x=1$ thì $P(1)\colon$\lq\lq $2\cdot1=1$\rq\rq~là mệnh đề sai.
			\item Với $x=1$, $y=1$ thì $R(1,1)\colon$\lq\lq $2\cdot1+1=3$\rq\rq~là mệnh đề đúng.\\
			Với $x=1$, $y=1$ thì $R(1,2)\colon$\lq\lq $2\cdot1+2=3$\rq\rq~là mệnh đề sai.
			\item Lấy số tự nhiên $n_0$ bất kì ta đều được $2n_0+1$ là một số lẻ, nghĩa là $T\left(n_0\right)\colon$\lq\lq $2n_0+1$ là số chẵn \rq\rq~ là mệnh đề sai. Do đó, không có giá trị n$_0$ của $n$ để $T\left(n_0\right)$ là mệnh đề đúng. $T\left(n_0\right)$ là mệnh đề sai với số tự nhiên $n_0 $ bất kì.
		\end{enumerate}
	}
\end{ex}
%%%=============================%%%

%%%=============EX_9=============%%%
\begin{ex}%[0D1H1-2]%[Dự án dề cương 3 Khối NH24-25-Đợt 1-Võ Thị Thùy Trang]
	Xét tính đúng sai của các mệnh đề sau:
	\begin{enumerate}[a)]
		\item $R\colon$\lq\lq Nếu tam giác $ABC$ có hai góc bằng $60^{\circ}$ thì nó là tam giác đều\rq\rq.
		\item $T\colon$\lq\lq Từ $-3<-2$ suy ra $(-3)^{2}<(-2)^{2}$\rq\rq.
	\end{enumerate}
	\loigiai{
		\begin{enumerate}[a)]
			\item $R$ là mệnh đề có dạng $P \Rightarrow Q$, với $P\colon$\lq\lq Tam giác $ABC$ có hai góc bằng $60^{\circ}$ \rq\rq~và $Q\colon$\lq\lq Tam giác $ABC$ là tam giác đều\rq\rq. Ta thấy khi $P$ đúng thì $Q$ cũng đúng. Do đó, $P\Rightarrow Q$ đúng hay $R$ đúng.
			\item $T$ là mệnh đề có dạng $P\Rightarrow Q$, với $P\colon$\lq\lq $-3<-2$\rq\rq~và $Q\colon$\lq\lq $(-3)^{2}<(-2)^{2}$\rq\rq~(hay \lq\lq $9<4$\rq\rq). Ta thấy mệnh đề $P$ đúng, còn mệnh đề $Q$ sai. Do đó, $P \Rightarrow Q$ sai.\\
			Vậy $T$ là mệnh đề sai.
		\end{enumerate}
	}
\end{ex}
%%%=============================%%%

%%%=============EX_10=============%%%
\begin{ex}%[0D1V1-4]%[Dự án dề cương 3 Khối NH24-25-Đợt 1-Võ Thị Thùy Trang]
	Sử dụng thuật ngữ \lq\lq điều kiện cần và đủ\rq\rq, phát biểu lại các định lí sau:
	\begin{enumerate}[a)]
		\item Một phương trình bậc hai có hai nghiệm phân biệt khi và chỉ khi biệt thức của nó dương;
		\item Một hình bình hành là hình thoi thì nó có hai đường chéo vuông góc với nhau và ngược lại.
	\end{enumerate}
	\loigiai{
		\begin{enumerate}[a)]
			\item Biệt thức của phương trình bậc hai dương là điều kiện cần và đủ để phương trình có hai nghiệm phân biệt.
			\item Hình bình hành có hai đường chéo vuông góc với nhau là điều kiện cần và đủ để hình bình hành là hình thoi.
		\end{enumerate}
	}
\end{ex}
%%%=============================%%%