\section{THU THẬP VÀ PHÂN LOẠI DỮ LIỆU} % Tên bài
\subsection{Thu thập dữ liệu}
\subsubsection{Kiến thức trọng tâm}
\begin{tomtat}
	\begin{itemize}
	\item Có nhiều cách để thu thập dữ liệu, chẳng hạn như
	\begin{itemize}
		\item Thu thập từ các nguồn có sẵn.
		\item Phỏng vấn.
		\item Lập phiếu câu hỏi.
		\item Quan sát.
		\item Làm thí nghiệm.
	\end{itemize}
	\item Chúng ta cần tìm phương pháp phù hợp với lĩnh vực, mục đích cần thu thập dữ liệu.
\end{itemize}
\end{tomtat}
\setcounter{vd}{0}
%%=====Ví dụ 1
	\begin{vd}%[Dự án EX-9-Đề Cương Toán 9]%[Lam Dang]%[8T6N1-1]
	Em hãy đề xuất phương pháp thu thập dữ liệu cho các vấn đề sau
	\begin{enumerate}
		\item 	Một nhà nghiên cứu muốn biết sản lượng gạo xuất khẩu của Việt Nam trong mười năm gần nhất.
		\item 	Ban giám hiệu Nhà trường muốn biết ý kiến của học sinh về chất lượng bữa ăn bán trú.
		\item 	Một người mẹ muốn tìm hiểu sự phát triển chiều cao của con mình theo thời gian.
		\item 	Lãnh đạo một bệnh viện muốn biết mức độ hài lòng của bệnh nhân khi đến khám chữa bệnh tại bệnh viện đó.
	\end{enumerate}
	\loigiai{
		\begin{enumerate} 	
			\item Thu thập từ các nguồn có sẵn.
			\item Phỏng vấn, lập phiếu khảo sát học sinh khối $8$ về chất lượng bữa ăn bán trú.
			\item Quan sát, thu thập từ các nguồn có sẵn: Đưa con đi khám sức khỏe định kỳ tại các cơ sở y tế để nhận được các số liệu chính xác về chiều cao và sự phát triển của trẻ.
			\item Phỏng vấn, quan sát, lập phiếu khảo sát.
		\end{enumerate}	
	}
\end{vd}
%%=====Ví dụ 2
\begin{vd}%[Dự án EX-9-Đề Cương Toán 9]%[Lam Dang]%[8T6N1-1]
	Bác An đội trưởng muốn thu thập thông tin về các loại rau, củ theo tiêu chuẩn Vietgap được các gia đình xã mình phụ trách ưa trồng. Theo em, bác An có thể thu thập những thông tin đó bằng cách nào?
	\loigiai{
		Bác An có thể thu thập những thông tin đó bằng cách lập phiếu hỏi như sau:
		\begin{center}
			\begin{tabular}{|c|c|}
				\hline
				\textbf{Loại rau, củ} & \textbf{Ưa trồng} \\
				\hline
				Cà rốt &  \\
				\hline
				Súp lơ xanh &  \\
				\hline
				Khoai tây &  \\
				\hline
				$\ldots$ &  \\
				\hline
			\end{tabular}
		\end{center}
	}
\end{vd}
\subsubsection{Bài tập}
\setcounter{bt}{0}
%%%=============BT_1=============%%%
\begin{bt}%[Dự án EX-9-Đề Cương Toán 9]%[Lam Dang]%[8T6N1-1]
	Em hãy đề xuất phương pháp thu thập dữ liệu cho các vấn đề sau
	\begin{enumerate}
		\item Tỉ số giữa số lần xuất hiện mặt sấp và số lần xuất hiện mặt ngửa khi tung một đồng xu lần.
		\item Ý kiến của học sinh lớp về việc có tham gia Hội diễn văn nghệ của trường hay không.
		\item So sánh số ngày nghỉ lễ mỗi năm của người lao động của các quốc gia trong khu vực ASEAN.
	\end{enumerate}
	\loigiai{
		\begin{enumerate}
			\item Làm thí nghiệm.
			\item Lập phiếu câu hỏi.
			\item Thu thập từ các nguồn có sẵn.
		\end{enumerate}
	}
\end{bt}

%%%=============BT_2=============%%%
\begin{bt}%[Dự án EX-9-Đề Cương Toán 9]%[Lam Dang]%[8T6H1-1]
	Hãy sử dụng phương pháp thích hợp để thu thập dữ liệu và lập bảng thống kê dân số các tỉnh khu vực miền Đông Nam Bộ của Việt Nam.
	\loigiai{
		Thu thập dữ liệu dựa vào Internet.
		\begin{center}
			\begin{tabular}{|c|c|}
				\hline
				\textbf{Các tỉnh khu vực Nam Bộ }& \textbf{Dân số (người)} \\
				\hline
				Hồ Chí Minh & $9\,\,411\,\, 805$ \\
				\hline
				Bà Rịa-Vũng Tàu & $1\,\,181\,\,302$ \\
				\hline
				Bình Dương & $2\,\,678\,\,220$ \\
				\hline
				Bình Phước & $1\,\,020\,\,839$ \\
				\hline
				Đồng Nai & $3\,\,236\,\,248$ \\
				\hline
				Tây Ninh  &  $1\,\,190\,\,852$\\
				\hline
			\end{tabular}
		\end{center}
		\begin{flushright}
			\href{https://vi.wikipedia.org/wiki/Đông_Nam_Bộ}{https://vi.wikipedia.org/wiki/Đông-Nam-Bộ}
		\end{flushright}
	}
\end{bt}

%%%=============BT_3=============%%%
\begin{bt}%[Dự án EX-9-Đề Cương Toán 9]%[Lam Dang]%[8T6H1-1]
	Em hãy đề xuất phương pháp thích hợp cho một cửa hàng bán kem thu thập thông tin để tìm hiểu về loại kem yêu thích của các khách hàng trong buổi sáng chủ nhật.
	\loigiai{
		Vì cửa hàng cần thu thập thông tin của các khách hàng trong buổi sáng Chủ nhật nên có thể sử phiếu hỏi hoặc tiến hành phỏng vấn trực tiếp.\\
		Mẫu phiếu hỏi:
		\begin{center}
			\begin{tabular}{|c|c|}
				\hline Vị kem & Ưa thích \\
				\hline Vani & \\
				\hline Socola & \\
				\hline Matcha & \\
				\hline Khoai môn & \\
				\hline ..... & \\
				\hline
			\end{tabular}
		\end{center}
	}
\end{bt}

\subsection{Phân loại dữ liệu}
\subsubsection{Kiến thức trọng tâm}
\begin{tomtat}
		Việc sắp xếp thông tin theo những tiêu chí nhất định gọi là phân loại dữ liệu. Việc phân loại dữ liệu thống kê phụ thuộc vào những tiêu chí đưa ra, tức phụ thuộc vào mục đích phân loại.
	\begin{itemize}
		\item Dữ liệu định tính được biểu diễn bằng từ, chữ cái, kí hiệu, $\ldots$ và được chia thành hai loại
		\begin{itemize}
			\item Dữ liệu định danh là dữ liệu thể hiện cách gọi tên (ví dụ: giới tính, màu sắc, nơi ở, $\ldots$).
			\item Dữ liệu biểu thị thứ bậc là dữ liệu thể hiện sự hơn kém (ví dụ: mức độ hài lòng, trình độ tay nghề, khối lớp, $\ldots$).
		\end{itemize}
		\item Dữ liệu định lượng nhận giá trị là các số thực và được chia thành hai loại
		\begin{itemize}
			\item Loại rời rạc là dữ liệu chỉ nhận hữu hạn giá trị hoặc biểu thị số đếm (ví dụ: cỡ giày, số học sinh, số ngày công, $\ldots$).
			\item Loại liên tục là dữ liệu có thể nhận mọi giá trị trong một khoảng nào đó (ví dụ: chiều dài, khối lượng, thu nhập, $\ldots$).
		\end{itemize}
	\end{itemize}
\end{tomtat}
\setcounter{vd}{0}
%%=====Ví dụ 1
	\begin{vd}%[Dự án EX-9-Đề Cương Toán 9]%[Lam Dang]%[8T6H1-2]
	Thống kê về các loại lồng đèn mà các bạn học sinh lớp $8A1$ làm được để trao tặng cho các trẻ em khuyết tật nhân dịp Tết trung thu được cho trong bảng dữ liệu sau
	\begin{center}
		\begin{tabular}{|c|c|c|c|c|c|}
			\hline
			STT & Tên lồng đèn & Loại & Số lượng & Màu sắc & Khối lượng \\
			\hline
			$1$ & Con cá & Lớn & $2$ & Vàng & $250$ \\
			\hline
			$2$ & Thiên nga & Vừa & $6$ & Cam & $200$ \\
			\hline
			$3$ & Con thỏ & Nhỏ & $10$ & Xanh & $180$ \\
			\hline
			$4$ & Ngôi sao & Lớn & $2$ & Đỏ & $260$ \\
			\hline
			$5$ & Đèn kéo quân & Nhỏ & $15$ & Đỏ & $185$ \\
			\hline
		\end{tabular}
	\end{center}
	\begin{enumerate}
		\item Tìm dữ liệu định tính và dữ liệu định lượng trong bảng dữ liệu trên.
		\item Trong số các dữ liệu định tính tìm được, dữ liệu nào có thể so sánh hơn kém?
		\item Trong số các dữ liệu định lượng tìm được, dữ liệu nào là rời rạc?
	\end{enumerate}
	\loigiai{
		\begin{enumerate}	
			\item Dữ liệu định tính: Tên lồng đèn, Loại lồng đèn, Màu sắc lồng đèn.\\
			Dữ liệu định lượng: Số lượng, khối lượng.
			\item Dữ liệu định tính có thể so sánh hơn kém.\\
			Loại (Lớn, Vừa, Nhỏ): Dữ liệu này biểu thị thứ bậc, cho thấy sự khác biệt về kích thước.
			\item Dữ liệu định lượng rời rạc.\\
			Số lượng ($2$, $6$, $10$, $2$, $15$): Dữ liệu này là số đếm và chỉ nhận hữu hạn giá trị.
		\end{enumerate}
	}
\end{vd}
%%=====Ví dụ 2
\begin{vd}%[Dự án EX-9-Đề Cương Toán 9]%[Lam Dang]%[8T6H1-2]
	Sau khi tìm hiểu về danh sách quốc gia thành viên ASEAN từ trang web \href{https://vi.wikipedia.org}{https://vi.wikipedia.org}, bạn Na thu được những dữ liệu thống kê sau:
	\begin{itemize}
		\item Mười quốc gia là: Brunei; Campuchia; Indonesia; Lào; Malaysia; Myanmar; Philippines; Singapore; Thái Lan; Việt Nam.
		\item Diện tích (đơn vị: $\mathrm{km}^{2}$ ) của mười quốc gia đó lần lượt là: $5765$; $181 035$; $1904 569$; $236 800$; $329 847$; $676 578$; $300 000$; $719{,}2$; $513 120$; $331210$.
	\end{itemize}
	Hãy phân loại các dữ liệu đó dựa trên tiêu chí định tính, định lượng.
	\loigiai{
		\begin{itemize}
			\item Tên mười quốc gia thành viên ASEAN là dữ liệu định tính.
			\item Diện tích (đơn vị: $\mathrm{km}^{2}$) của mười quốc gia đó là dữ liệu định lượng.
		\end{itemize}
	}
\end{vd}

\subsubsection{Bài tập}
\setcounter{bt}{0}
%%%=============BT_1=============%%%
\begin{bt}%[Dự án EX-9-Đề Cương Toán 9]%[Lam Dang]%[8T6H1-2]
	Dữ liệu thu được trong mỗi câu hỏi sau thuộc loại dữ liệu nào?
	\begin{enumerate}
		\item Bạn nặng bao nhiêu?
		\item Bạn đang học trường nào?
		\item Lớp bạn có bao nhiêu học sinh?
		\item Kết quả xếp loại rèn luyện của bạn trong năm học trước của bạn là gì?
	\end{enumerate}
	\loigiai{
		\begin{enumerate}
			\item Đây là dữ liệu định lượng liên tục vì trọng lượng của bạn có thể được biểu thị bằng một số thực.
			\item Đây là dữ liệu định danh vì câu trả lời là tên của một định danh cụ thể (tên trường).
			\item Đây là dữ liệu định lượng rời rạc vì số lượng học sinh trong lớp là một số nguyên và hữu hạn.
			\item Đây là dữ liệu biểu thị thứ bậc vì câu trả lời biểu thị mức độ xếp loại từ cao đến thấp.
		\end{enumerate}
	}
\end{bt}

%%%=============BT_2=============%%%
\begin{bt}%[Dự án EX-9-Đề Cương Toán 9]%[Lam Dang]%[8T6H1-2]
	Thông tin về $5$ bạn học sinh của một trường trung học cơ sở tham gia Hội khỏe Phù Đổng được cho bởi bảng thống kê sau
	\begin{center}
		\begin{tabular}{|c|p{2cm}|c|c|p{2.5cm}|}
			\hline Hơ và tên & Cân nặng (kg) & Môn bơi sở trường & Kĩ thuật bơi & Số nội dung thi đấu \\
			\hline Nguyễn Kình Ngư & $60$ & Bơi ếch & Tôt & $2$ \\
			\hline Trần Văn Mạnh & $58$ & Bơi sải & Khá & $1$ \\
			\hline Lê Hoàng Phi & $45$ & Bơi bướm & Tốt & $2$ \\
			\hline Nguyễn Ánh Vân & $50$ & Bơi ếch & Xuất sắc & $3$ \\
			\hline Đỗ Hải Hà & $48$ & Bơi tự do & Tốt & $3$ \\
			\hline
		\end{tabular}
	\end{center}
	\begin{enumerate}
		\item Phân loại các dữ liệu trong bảng thống kê trên dựa trên hai tiêu chí định tính và định lượng.
		\item Trong số các dữ liệu định tính tìm được, dữ liệu nào có thể so sánh hơn kém?
		\item Trong số các dữ liệu định lượng tìm được, dữ liệu nào là liên tục?
	\end{enumerate}
	\loigiai{
		\begin{enumerate}
			\item  Phân loại dữ liệu định tính:
			\begin{itemize}
				\item  Họ và tên: Là dữ liệu định danh, thể hiện cách gọi tên của từng học sinh.
				\item  Môn bơi sở trường: Là dữ liệu định danh, thể hiện môn bơi mà mỗi học sinh giỏi nhất.
				\item  Kĩ thuật bơi: Là dữ liệu định danh, thể hiện khả năng bơi của mỗi học sinh (Tốt, Khá, Xuất sắc).
			\end{itemize}
			Phân loại dữ liệu định lượng:
			\begin{itemize}
				\item  Cân nặng (kg): Là dữ liệu định lượng loại rời rạc, thể hiện trọng lượng cơ thể của mỗi học sinh.
				\item  Số nội dung thi đấu: Là dữ liệu định lượng loại rời rạc, thể hiện số nội dung bơi mà mỗi học sinh tham gia.
			\end{itemize}
			\item Trong số các dữ liệu định tính tìm được, dữ liệu \lq\lq Kĩ thuật bơi\rq\rq\, có thể so sánh hơn kém. Cụ thể
			\begin{itemize}
				\item  Họ và tên: Dữ liệu này chỉ thể hiện cách gọi tên của từng học sinh, không thể so sánh hơn kém.
				\item  Môn bơi sở trường: Dữ liệu này chỉ thể hiện môn bơi mà mỗi học sinh giỏi nhất, không thể so sánh hơn kém.
				\item  Kĩ thuật bơi: Dữ liệu này thể hiện khả năng bơi của mỗi học sinh với các mức độ khác nhau (Tốt, Khá, Xuất sắc). Do đó, có thể so sánh hơn kém giữa các mức độ này.
			\end{itemize}
			\item Trong số các dữ liệu định lượng tìm được, dữ liệu \lq\lq Cân nặng (kg)\rq\rq\, là liên tục.
		\end{enumerate}		
	}
\end{bt}

%%%=============BT_3=============%%%
\begin{bt}%[Dự án EX-9-Đề Cương Toán 9]%[Lam Dang]%[8T6C1-2]
	Kết quả khảo sát ý kiến khách hàng trong một buổi sáng ở một cửa hàng cắt tóc được cho bởi bảng thống kê sau
	\begin{center}
		\begin{tabular}{|c|c|c|p{4cm}|p{4cm}|}
			\hline Tên khách hàng & Tuổi & Giới tính & Thời gian chờ để được cắt tóc (phút) & Mức độ hài lòng về dịch vụ \\
			\hline Dũng & $25$ & Nam & $15$ & Hài lòng \\
			\hline Cường & $30$ & Nam & $10$ & Hài lòng \\
			\hline Quân & $31$ & Nam & $8$ & Rất hài lòng \\
			\hline Khánh & $29$ & Nam & $5$ & Rất hài lòng \\
			\hline Xuân & $32$ & Nư & $9$ & Rất hài lòng \\
			\hline
		\end{tabular}
	\end{center}
	\begin{enumerate}
		\item  Phân loại các dữ liệu trong bảng thống kê trên dựa trên hai tiêu chí định tính và định lượng.
		\item  Trong số các dữ liệu định tính tìm được, dữ liệu nào có thể so sánh hơn kém?
		\item  Trong số các dữ liệu định lượng tìm được, dữ liệu nào là rời rạc?
	\end{enumerate}
	\loigiai{
		\begin{enumerate}
			\item  Phân loại dữ liệu định tính:
			\begin{itemize}
				\item  Tên khách hàng: Là dữ liệu định danh, thể hiện tên của từng khách hàng.
				\item  Giới tính: Là dữ liệu định danh, thể hiện giới tính của mỗi khách hàng (Nam, Nữ).
				\item  Mức độ hài lòng về dịch vụ: Là dữ liệu định danh, thể hiện mức độ hài lòng của mỗi khách hàng về dịch vụ cắt tóc (Hài lòng, Rất hài lòng).
			\end{itemize}
			Phân loại dữ liệu định lượng:
			\begin{itemize}
				\item  Tuổi: Là dữ liệu định lượng loại rời rạc, thể hiện độ tuổi của mỗi khách hàng.
				\item  Thời gian chờ để được cắt tóc (phút): Là dữ liệu định lượng loại rời rạc, thể hiện thời gian chờ đợi của mỗi khách hàng để được cắt tóc.
			\end{itemize}
			\item Trong số các dữ liệu định tính tìm được, dữ liệu \lq\lq Mức độ hài lòng về dịch vụ\rq\rq có thể so sánh hơn kém. Cụ thể
			\begin{itemize}
				\item  Tên khách hàng: Dữ liệu này chỉ thể hiện tên của từng khách hàng, không thể so sánh hơn kém.
				\item  Giới tính: Dữ liệu này chỉ thể hiện giới tính của mỗi khách hàng, không thể so sánh hơn kém.
				\item  Mức độ hài lòng về dịch vụ: Dữ liệu này thể hiện mức độ hài lòng của mỗi khách hàng về dịch vụ cắt tóc với các mức độ khác nhau (Hài lòng, Rất hài lòng). Do đó, có thể so sánh hơn kém giữa các mức độ này.
			\end{itemize}
			\item Trong số các dữ liệu định lượng tìm được, dữ liệu  \lq\lq Thời gian chờ để được cắt tóc (phút)\rq\rq\, là liên tục.
		\end{enumerate}
	}
\end{bt}
%%%=============BT_4=============%%%
\begin{bt}%[Dự án EX-9-Đề Cương Toán 9]%[Lam Dang]%[8T6V1-2]
	Bác Toàn đã trồng trong vườn các loại cây sau: cây ăn quả (cây cam, cây mít, cây nhãn, cây ổi, cây na); cây lấy tinh dầu (cây dừa, cây tràm, cây gấc); cây lấy gỗ nhanh, giá trị kinh tế cao (cây gỗ sưa, cây lim xanh, cây xoan đào, cây gỗ cẩm lai). Hãy giúp bác Toàn phân loại những cây đã trồng trong vườn theo những tiêu chí sau:
	\begin{center}
		\begin{tabular}{|l|c|}
			\hline
			\multicolumn{1}{|c|}{\textbf{Loại cây}} & \textbf{Tên cây} \\
			\hline
			Cây ăn quả & $?$ \\
			\hline
			Cây lấy tinh dầu & $?$ \\
			\hline
			Cây lấy gỗ nhanh, giá trị kinh tế cao & $?$ \\
			\hline
		\end{tabular}
	\end{center}
	\loigiai{
		\begin{center}
			\begin{tabular}{|l|l|}
				\hline
				\multicolumn{1}{|c|}{\textbf{Loại cây}} & \multicolumn{1}{c|}{\textbf{Tên cây}} \\
				\hline
				Cây ăn quả & Cây cam, cây mít, cây nhãn, cây ổi, cây na \\
				\hline
				Cây lấy tinh dầu & Cây dừa, cây tràm, cây gấc \\
				\hline
				Cây lấy gỗ nhanh, giá trị  kinh tế cao &  Cây gỗ sưa, cây lim xanh, cây xoan đào, \\
				&cây gỗ cẩm lai \\
				\hline
			\end{tabular}
		\end{center}
	}
\end{bt}

\subsection{Tính hợp lí của dữ liệu}
\subsubsection{Kiến thức trọng tâm}
\begin{tomtat}
	\begin{itemize}
	\item Có thể kiểm tra định dạng của dữ liệu hoặc mối liên hệ toán học đơn giản giữa các số liệu thống kê để nhận biết tính hợp lí của dữ liệu và các kết luận dựa trên các dữ liệu thống kê đó.
	\item Một số đặc điểm của một bảng thống kê đúng:
	\begin{itemize}
		\item Dữ liệu phải đúng định dạng.
		\item Dữ liệu phải nằm trong phạm vi dự kiến.
		\item Dữ liệu phải có tính đại diện đối với vấn đề cần thống kê.
		\item Tổng tất cả các số liệu thành phần phải bằng số liệu của toàn thể.
		\item Số lượng của bộ phận phải nhỏ hơn số lượng của toàn thể.
	\end{itemize}
\end{itemize}
\end{tomtat}
\setcounter{vd}{0}
	%%=====Ví dụ 1
\begin{vd}%[Dự án EX-9-Đề Cương Toán 9]%[Lam Dang]%[8T6H1-3]
	Bảng thống kê sau cho biết dữ liệu về hoạt động trong giờ chơi của các học sinh lớp $8A2$ (biết rằng mỗi học sinh chỉ thực hiện duy nhất một hoạt động và số học sinh của lớp là ít hơn $50$ bạn):
	\begin{center}
		\begin{tabular}{|c|c|}
			\hline
			Hoạt động & Số học sinh thực hiện \\
			\hline
			Ôn bài & $55$ \\
			\hline
			Đọc sách & $3$ \\
			\hline
			Nghe nhạc & $3$ \\
			\hline
			Trò chuyện & Tất cả các bạn tổ $1$ \\
			\hline
			Chơi trò chơi trên điện thoại & $10$ \\
			\hline
			Chơi đá cầu & $5$ \\
			\hline
			Chơi cầu lông & $4$ \\
			\hline
		\end{tabular}
	\end{center}
	Hãy nêu nhận xét về tính hợp lí của các dữ liệu trong bảng thống kê trên.
	\loigiai{
		\begin{enumerate}
			\item  Kiểm tra định dạng của dữ liệu:\\						
			Dữ liệu trong cột \lq\lq Hoạt động\rq\rq\, được biểu diễn bằng từ ngữ, thể hiện đúng định dạng dữ liệu định tính.
			Dữ liệu trong cột \lq\lq Số học sinh thực hiện\rq\rq\, được biểu diễn bằng số nguyên, thể hiện đúng định dạng dữ liệu định lượng.
			\item  Kiểm tra xem dữ liệu có nằm trong phạm vi dự kiến không:						
			\begin{itemize}
				\item  Hoạt động trong giờ chơi (Ôn bài, Đọc sách, Nghe nhạc, Trò chuyện, Chơi trò chơi trên điện thoại, Chơi đá cầu, Chơi cầu lông) là hợp lý và nằm trong phạm vi dự kiến cho một lớp học.
				\item  Số học sinh thực hiện các hoạt động này phải nhỏ hơn $50$, nhưng có một số liệu không hợp lý:
				Có $55$ học sinh ôn bài, trong khi tổng số học sinh của lớp ít hơn $50$. Điều này không hợp lý.
				\item  Cột \lq\lq Trò chuyện\rq\rq\, ghi \lq\lq Tất cả các bạn tổ $1$\rq\rq, đây là mô tả không rõ ràng và không thể kiểm tra được con số chính xác, nên cần làm rõ số lượng học sinh cụ thể.
			\end{itemize}
			\item  Kiểm tra tính đại diện của dữ liệu đối với vấn đề cần thống kê:\\						
			Dữ liệu này nhằm mục đích thống kê các hoạt động của học sinh trong giờ chơi, do đó nó có tính đại diện.
			\item  Kiểm tra tổng số liệu thành phần so với số liệu của toàn thể:\\						
			Tổng số học sinh thực hiện các hoạt động ($55+3+3+$ tất cả các bạn tổ $1$ $+10+5+4$) phải bằng hoặc nhỏ hơn tổng số học sinh của lớp (dưới $50)$. Tuy nhiên, tổng số học sinh ôn bài đã vượt quá $50$, điều này không hợp lý.
			\item  Kiểm tra số lượng của bộ phận so với số lượng của toàn thể:\\
			Số lượng học sinh thực hiện mỗi hoạt động phải nhỏ hơn hoặc bằng số lượng học sinh toàn lớp (dưới $50$), nhưng đã có dữ liệu vượt quá giới hạn này.
		\end{enumerate}
		Kết luận: Bảng dữ liệu không hợp lý vì có số liệu về số học sinh ôn bài $(55)$ vượt quá tổng số học sinh của lớp (dưới $50$). Đồng thời, mục \lq\lq Trò chuyện\rq\rq\, cần làm rõ số lượng cụ thể của các bạn tổ $1$ để có thể kiểm tra tính hợp lý chính xác hơn.
	}
\end{vd}
%%=====Ví dụ 2
\begin{vd}%[Dự án EX-9-Đề Cương Toán 9]%[Lam Dang]%[8T6H1-3]
	Thị phần của một sản phẩm là phần thị trường tiêu thụ mà sản phẩm đó chiếm lĩnh so với tổng sản phẩm tiêu thụ của thị trường. Bảng thống kê sau cho biết thị phần của $4$ loại bút trên thị trường:
	\begin{center}
		\begin{tabular}{|l|c|c|c|c|}
			\hline
			Loại bút & $A$ & $B$ & $C$ & $D$ \\
			\hline
			Thị phần (\%) & $15$ & $10$ & $40$ & $35$ \\
			\hline
		\end{tabular}
	\end{center}
	Xét tính hợp lí của các quảng cáo sau đây đối với nhãn hiệu bút $D$:
	\begin{enumerate}
		\item Là loại bút được đa số mọi người lựa chọn.
		\item Là loại bút chiếm thị phần nhiều nhất.
		\item Là một trong những loại bút chiếm thị phần cao nhất.
	\end{enumerate}
	\loigiai{
		\begin{enumerate}
			\item  Là loại bút được đa số mọi người lựa chọn.\\						
			Thị phần của bút $D$ là $35\%$, nghĩa là không phải đa số (hơn $50\%$). Thị phần cao nhất là của bút $C$ với $40\%$. Vì vậy, tuy bút $D$ có thị phần lớn nhưng không thể nói rằng nó được đa số mọi người lựa chọn.\\
			Vậy quảng cáo này không hợp lý.
			\item  Là loại bút chiếm thị phần nhiều nhất.\\						
			Bút $C$ chiếm thị phần nhiều nhất với $40\%$, còn bút $D$ chỉ chiếm $35\%$. Do đó, bút $D$ không phải là loại bút chiếm thị phần nhiều nhất.
			Vậy quảng cáo này không hợp lý.
			\item  Là một trong những loại bút chiếm thị phần cao nhất.
			Thị phần của bút $D$ là $35\%$, cao thứ hai sau bút $C$ $(40\%$). Điều này hợp lý khi nói rằng bút $D$ là một trong những loại bút chiếm thị phần cao nhất.
			Vậy quảng cáo này hợp lý.
		\end{enumerate}
	}
\end{vd}
%%=====Ví dụ 3
\begin{vd}%[Dự án EX-9-Đề Cương Toán 9]%[Lam Dang]%[8T6H1-3]
	Một công ty du lịch đã hỏi ý kiến của $205$ khách quốc tế về $13$ danh lam thắng cảnh nổi tiếng ở Hà Nội khi đi du lịch Việt Nam và nhận được kết quả là có $82$ khách muốn đến Văn Miếu Quốc Tử Giám. Từ đó, công ty đưa ra kết luận rằng có $40 \%$ số khách quốc tế đến Việt Nam thích tham quan Văn Miếu Quốc Tử Giám. Theo em, công ty du lịch đưa ra kết luận như vậy có hợp lí không? Vì sao?
	\loigiai{
		Kết luận mà công ty du lịch đưa ra là không hợp lí vì $205$ khách quốc tế đó chỉ được hỏi về $13$ danh lam thắng cảnh nổi tiếng ở Hà Nội không đảm bảo đại diện cho toàn bộ các danh lam thắng cảnh ở Việt Nam.
	}
\end{vd}
\subsubsection{Bài tập}
\setcounter{bt}{0}
%%%=============BT_1=============%%%
\begin{bt}%[Dự án EX-9-Đề Cương Toán 9]%[Lam Dang]%[8T6H1-3]
	Lớp $8A4$ có $39$ học sinh được chia thành $5$ tổ, mỗi tổ có không quá $8$ học sinh. Bảng thống kê số học sinh ở mỗi tổ như sau
	\begin{center}
		\begin{tabular}{|c|c|c|c|c|c|}
			\hline
			Tổ & $1$ & $2$ & $3$ & $4$ & $5$ \\
			\hline
			Số học sinh & $8$ & $7$ & $9$ & $8$ & $7$ \\
			\hline
		\end{tabular}
	\end{center}
	Theo em thì số liệu đã cho trong bảng thống kê trên có hợp lí không? Vì sao?
	\loigiai{
		Dựa vào các phương pháp kiểm tra tính hợp lý của dữ liệu, có thể nhận thấy bảng thống kê trên chưa hợp lý. Vì số học sinh của tổ $3$ lớn hơn số học sinh tối đa cho phép của mỗi tổ ($8$ học sinh).
	}
\end{bt}

%%%=============BT_2=============%%%
\begin{bt}%[Dự án EX-9-Đề Cương Toán 9]%[Lam Dang]%[8T6H1-3]
	Xét tính hợp lí của các dữ liệu trong bảng thống kê kết quả rèn luyện của học sinh lớp $8A5$ dưới đây:
	\begin{center}
		\begin{tabular}{|c|c|c|c|c|}
			\hline
			Học lực & Tốt & Khá & Đạt & Chưa đạt \\
			\hline
			Tỉ lệ (\%) & $75$ & $15$ & $10$ & $5$ \\
			\hline
		\end{tabular}
	\end{center}
	\loigiai{
		Dựa vào các phương pháp kiểm tra tính hợp lý của dữ liệu, có thể nhận thấy bảng thống kê trên chưa hợp lý. Vì tổng tỉ lệ phần trăm kết quả rèn luyện của học sinh lớp $8A5$ lớn hơn $100\%$ ($75+15+10+5=105\%$).
	}
\end{bt}


%%%=============BT_3=============%%%
\begin{bt}%[Dự án EX-9-Đề Cương Toán 9]%[Lam Dang]%[8T6V1-3]
	Để chuẩn bị đưa ra thị trường một mẫu xe ô tô mới, một hãng sản xuất ô tô tiến hành thăm dò màu sơn mà mọi người yêu thích. Bộ phận nghiên cứu thị trường đã hỏi ý kiến của $100$ người mua xe ở độ tuổi $30$ đến $40$ và nhận được kết quả là $55$ người thích màu đen, $30$ người thích màu đỏ và $15$ người thích màu trắng. Từ đó, bộ phận này đã viết báo cáo cho rằng $55\%$ số người mua xe thích màu đen, $30\%$ số người mua xe thích màu đỏ và $15\%$ người mua xe thích màu trắng. Theo em thì kết luận của bộ phận nghiên cứu thị trường như vậy có hợp lí không? Vì sao?
	\loigiai{
		Ta thấy, hãng sản xuất xe chỉ hỏi ý kiến của $100$ người mua xe ở độ tuổi từ $30$ đến $40$, trong khi khách hàng mua xe có thể ở nhiều độ tuổi khác nhau. Do đó dữ liệu hãng sản xuất xe thu được không có tính đại diện.\\		
		Vậy hãng sản xuất xe đưa ra kết luận trong quảng cáo là $45\%$ số người mua xe chọn màu đen, $20\%$ số người mua chọn xe màu trắng là chưa hợp lí.
	}
\end{bt}

%%%=============BT_4=============%%%
\begin{bt}%[Dự án EX-9-Đề Cương Toán 9]%[Lam Dang]%[8T6H1-3]
	\immini{
		Bạn Nam vẽ biểu đồ hình quạt tròn như Hình 1 để biểu diễn tỉ lệ các loại hoa được trồng trên một thửa ruộng của gia đình gồm: hoa hồng, hoa ly, hoa cúc. Số liệu mà bạn Nam nêu ra trong biểu đồ hình quạt tròn ở Hình 1 là đúng hay sai? Vì sao?
	}{
		\begin{tikzpicture}
			\def\r{2}
			\def\gocxp{90}
			\coordinate (A) at (90:\r);
			\foreach \val/\freq/\col/\pattern[count=\i from 0] in{
				Hoa hồng/62/red/vertical lines,
				Hoa ly/29/blue/north east lines,
				Hoa cúc/10/magenta/dots}{
				\pgfmathsetmacro\gockt{-(\freq*3.6-\gocxp)}
				\pgfmathsetmacro\gocnode{\gocxp+\gockt}
				\draw[gray!50,pattern = \pattern,pattern color=\col] (0,0)--(A) arc(\gocxp:\gockt:\r) coordinate(A)--cycle;
				\fill[pattern = \pattern,pattern color=\col,draw] (\r+1,\r-.75*\i) --++(0:.5)--++(-90:.5) node[pos=.5,right,black]{\val}--++(180:.5)--cycle;
				\path ($(0,0)+(\gocnode/2:1.1)$) node[fill=white,inner sep=0pt,circle]{\color{black} $\freq\%$};
				\global\let\gocxp=\gockt
			}
			\path (current bounding box.south) node[below=2mm]{Hình 1};
		\end{tikzpicture}
	}
	\loigiai{
		Số liệu mà bạn Nam nêu ra trong biểu đồ hình quạt tròn ở Hình 1 là sai, vì $62 \%+29 \%+10 \%>100 \%$.
	}
\end{bt}

%%%=============BT_5=============%%%
\begin{bt}%[Dự án EX-9-Đề Cương Toán 9]%[Lam Dang]%[8T6V1-3]
	Bảng $1$ thống kê số lượng xe máy bán được (loại có giá chưa đến 50 triệu đồng/xe) và doanh thu mỗi ngày trong 4 ngày cuối tuần của một cửa hàng điện máy.
	\begin{center}
		\begin{tabular}{|c|c|c|}
			\hline
			\textbf{Ngày} & \textbf{Số xe} & \textbf{Doanh thu} \\
			\hline
			Thứ Tư & 8 & 230 triệu đồng \\
			\hline
			Thứ Năm & 7 & 300 triệu đồng \\
			\hline
			Thứ Sáu & 6 & 320 triệu đồng \\
			\hline
			Thứ Bảy & 10 & 480 triệu đồng \\
			\hline
		\end{tabular}
		
		Bảng 1
	\end{center}
	Theo em, các số liệu về doanh thu của cửa hàng trong ngày thứ Sáu nêu ra ở Bảng 1 đã chính xác chưa? Vì sao?
	\loigiai{
		Ngày thứ Sáu, cửa hàng bán được $6$ xe, mỗi xe có giá chưa đến 50 triệu đồng, do đó doanh thu cửa hàng bán trong ngày thứ Sáu chưa đến 300 triệu đồng. Vậy trong hai số liệu của ngày thứ Sáu là số xe 6 và doanh thu của cửa hàng 320 triệu đồng nêu ra ở Bảng 1 có ít nhất một số liệu là không hợp lí.
	}
\end{bt}
%%%=============BT_6=============%%%
\begin{bt}%[Dự án EX-9-Đề Cương Toán 9]%[Lam Dang]%[8T6H1-2]
	\immini{
		Tổ 1, Tổ 2, Tổ 3, Tổ 4 của lớp $8 \mathrm{A}$ đều có tổng số học sinh tham gia câu lạc bộ đàn piano ở Học kì I và Học kì II không ít hơn 8 học sinh. Anh Long phụ trách câu lạc bộ đã lập biểu đồ cột kép ở Hình 2 biểu diễn số học sinh tham gia câu lạc bộ đàn piano ở Học kì I và Học kì II của mỗi tổ. Anh Long đã ghi nhầm số liệu của một tổ. Hỏi anh Long đã ghi nhầm số liệu của tổ nào trong biểu đồ cột kép ở Hình 2?
	}{
		\begin{tikzpicture}[>=stealth,scale=1,font=\scriptsize,yscale=0.7]
			\def\hoanh{7};
			\def\tung{9};
			\def\mau{red};
			\draw[->] (0,0)--(\hoanh,0) node[below]{Tổ};
			\draw[->] (0,0)node[below left]{$O$}--(0,\tung) node[above right]{Số học sinh};
			\foreach \x/\y in{1/3,2.5/2,4/6,5.5/1}{
				\fill[pattern = north east lines,pattern color=blue,draw] (\x-0.25,0) rectangle (\x+0.25,\y);
				\draw[dashed] (\x,\y)node[above]{$\y$}--(0,\y);}
			\foreach \x/\y in{1.5/5,3/4,4.5/6,6/8}{
				\fill[pattern = dots,pattern color=violet,draw] (\x-0.25,0) rectangle (\x+0.25,\y);
				\draw[dashed] (\x,\y)node[above]{$\y$}--(0,\y);}
			\foreach \x/\p in {1.25/Tổ 1,2.75/Tổ 2,4.25/Tổ 3,5.75/Tổ 4}{
				\node[below] at (\x,0){\p};
			}
			\foreach \y in {1,2,...,8}{
				\fill (0,\y) circle (1pt)node[left]{$\y$};
			}
			\fill[pattern = north east lines,pattern color=blue,draw] (3,8.75) rectangle +(45:0.5) +(0.25,0.3)node[midway, right=2mm]{Học kì I};
			\fill[pattern = dots,pattern color=violet,draw] (3,8.25) rectangle +(45:0.5) +(0.25,0.3)node[midway, right=2mm]{Học kì II};
			\path (current bounding box.south) node[below=2mm]{Hình 2};
		\end{tikzpicture}
	}
	\loigiai{
		Từ biểu đồ cột kép ở Hình 2 ta thấy tổng số học sinh tham gia câu lạc bộ đàn piano ở Học kì I và Học kì II của Tổ 1, Tổ 2, Tổ 3, Tổ 4 lần lượt là: 8 học sinh, 6 học sinh, 12 học sinh, 9 học sinh. Vậy trong biểu đồ cột kép ở Hình 2, anh Long đã ghi nhầm số liệu của Tổ 2 .
	}
\end{bt}

%%%=============BT_7=============%%%
\begin{bt}%[Dự án EX-9-Đề Cương Toán 9]%[Lam Dang]%[8T6V1-2]
	\immini{
		Bốn lớp $8 \mathrm{A}$, $8 \mathrm{B}$, $8 \mathrm{C}$, $8 \mathrm{D}$ đều có tỉ số phần trăm số học sinh đạt mức Khá so với số học sinh đạt mức Tốt là lớn hơn $52 \%$. Anh Linh khối trưởng đã lập biểu đồ cột kép ở Hình 3 thống kê số học sinh đạt mức Tốt và Khá của từng lớp $8 \mathrm{A}$, $8 \mathrm{B}$, $8 \mathrm{C}$, $8 \mathrm{D}$. Anh Linh đã ghi nhầm số liệu của một lớp trong biểu đồ cột kép ở Hình 3. Hỏi anh Linh đã ghi nhầm số liệu của lớp nào? Vì sao?
	}{
		\begin{tikzpicture}[>=stealth,scale=1,font=\scriptsize,yscale=0.7]
			\def\hoanh{7};
			\def\tung{10};
			\def\mau{red};
			\draw[->] (0,0)--(\hoanh,0) node[below]{Lớp};
			\draw[->] (0,0)node[below left]{$O$}--(0,\tung) node[above right]{Số học sinh};
			\foreach \x/\y in{1/10,2.5/17,4/14,5.5/12}{
				\fill[pattern = north east lines,pattern color=blue,draw] (\x-0.25,0) rectangle (\x+0.25,\y/2);
				\draw[dashed] (\x,\y/2)node[above]{$\y$}--(0,\y/2);}
			\foreach \x/\y in{1.5/15,3/8,4.5/10,6/8}{
				\fill[pattern = dots,pattern color=violet,draw] (\x-0.25,0) rectangle (\x+0.25,\y/2);
				\draw[dashed] (\x,\y/2)node[above]{$\y$}--(0,\y/2);}
			\foreach \x/\p in {1.25/8A,2.75/8B,4.25/8C,5.75/8D}{
				\node[below] at (\x,0){\p};
			}
			\foreach \y in {2,4,...,18}{
				\fill (0,\y/2) circle (1pt)node[left]{$\y$};
			}
			\fill[pattern = north east lines,pattern color=blue,draw] (3,9.75) rectangle +(45:0.5) +(0.25,0.3)node[midway, right=2mm]{Tốt};
			\fill[pattern = dots,pattern color=violet,draw] (3,9.25) rectangle +(45:0.5) +(0.25,0.3)node[midway, right=2mm]{Khá};
			\path (current bounding box.south) node[below=2mm]{Hình 3};
		\end{tikzpicture}
	}
	\loigiai{
		Anh Linh đã ghi nhầm số liệu của lớp 8B vì tỉ số phần trăm số học sinh đạt mức Khá so với số học sinh đạt mức Tốt ở lớp 8B là lớn hơn $\dfrac{8}{17}\cdot 100\%<52 \%$.
	}
\end{bt}
