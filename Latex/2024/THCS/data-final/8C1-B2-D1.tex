\section{ĐA THỨC NHIỀU BIẾN} % Tên bài
\subsection{Khái niệm đa thức}
\subsubsection{Kiến thức trọng tâm}
\begin{tomtat}
	\begin{dn}
		\begin{itemize}
			\item \textit{Đa thức} là một tổng của những đơn thức. Mỗi đơn thức trong tổng gọi là một \textit{hạng tử} của đa thức đó.
			\item Đa thức thu gọn là đa thức không chứa hai hạng tử nào đồng dạng.
			\item  Bậc của hạng tử có bậc cao nhất trong dạng thu gọn của đa thức gọi là bậc của đa thức đó.
		\end{itemize}	
	\end{dn}
	\begin{luuy}
			\begin{itemize}
			\item Các đơn thức cũng là đa thức.
			\item Mỗi số khác $0$ là đa thức có bậc là $0$. Số $0$ là đa thức không, đa thức này không có bậc.
			\item  Biến đổi một đa thức thành đa thức thu gọn gọi là thu gọn đa thức đó.
			\item  Để thu gọn một đa thức, ta nhóm các hạng tử đồng dạng với nhau và cộng các hạng tử đồng dạng đó với nhau.
			\end{itemize}
	\end{luuy}
\end{tomtat}
\begin{vd}%[Dự án EX-9-Đề Cương Toán 8-L1]%[Trần Chiến]%[8D1N1-2] 
	Chỉ ra các đa thức trong các biểu thức sau
	\begin{multicols}{3}
		\begin{enumerate}
			\item $7$;
			\item $x^5-3x^2+1$;
			\item $\dfrac{2}{y}+3$;
			\item $3ab^2-2a^2b+1$;
			\item $\sqrt{z}+z^4$;
			\item $x^2y^3+xy^4-xy$;
			\item $\dfrac{1}{x+y}+x$;
			\item $m^3n^2-2mn+\sqrt{5}$;
			\item $p^2 q^3+pq+1$;
			\item $x+y+z$;
			\item $a^4-a^2b+b^3$;
			\item $2x^2-x +\dfrac{1}{x}$.
		\end{enumerate}
	\end{multicols}
	\loigiai{Các đa thức là $(1)$, $(2)$, $(4)$, $(6)$, $(8)$, $(9)$, $(10)$ và $(11)$. \\
	Các biểu thức không phải đa thức là $(3)$, $(5)$, $(7)$ và $(12)$.}
\end{vd}

\begin{vd}%[Dự án EX-9-Đề Cương Toán 8-L1]%[Trần Chiến]%[8D1H1-8]   
	Thu gọn rồi xác định hệ số và bậc của từng hạng tử trong đa thức sau
	\begin{enumerate}
		\item $A=-2xy+\dfrac{3}{2}xy^2+\dfrac{1}{2}xy^2+xy$;
		\item $B=xy^2z+2xy^2z-xyz-3xy^2z+xy^2z$;
		\item $C=4x^2y^3+x^4-2x^2+6x^4-x^2y^3$;
		\item $D=\dfrac{3}{4}xy^2-2xy-\dfrac{1}{2}xy^2+3xy$;
		\item $E=2x^2-3y^3-z^4-4x^2+2y^3+3z^4$;
		\item $F=3xy^2z+xy^2z-xyz+2xy^2z-3xyz$.
	\end{enumerate}
	\loigiai{
		\begin{enumerate}
			\item $A=(-2xy+xy)+\left(\dfrac{3}{2}xy^2+\dfrac{1}{2}xy^2\right)=-xy+2xy^2$.\\
			Hạng tử $-xy$ có hệ số là $-1$, bậc là $2$; hạng tử $2xy^2$ có hệ số là $2$, bậc là $3$.
			\item $B=(xy^2z+2xy^2z-3xy^2z+xy^2z)+(-xyz)=xy^2z-xyz$.\\
			Hạng tử $xy^2z$ có hệ số là $1$, bậc là $4$; hạng tử  $-xyz$ có hệ số là $-1$, bậc là $3$.
			\item $C=(4x^2y^3-x^2y^3)+(x^4+6x^4)+(-2x^2)=3x^2y^3+7x^4-2x^2$.\\
			Hạng tử $3x^2y^3$ có hệ số là $3$, bậc là $5$; hạng tử  $7x^4$ có hệ số là $7$, bậc là $4$; hạng tử $-2x^2$ có hệ số là $-2$, bậc là $2$.
			\item $D=\left(\dfrac{3}{4}xy^2-\dfrac{1}{2}xy^2\right)+(-2xy+3xy)=\dfrac{1}{4}xy^2+xy$.\\
			Hạng tử $\dfrac{1}{4}xy^2$ có hệ số là $\dfrac{1}{4}$, bậc là $3$; hạng tử  $xy$ có hệ số là $1$, bậc là $2$.
			\item $E=(2x^2-4x^2)+(-3y^3+2y^3)+(-z^4+3z^4)=-2x^2-y^3+2z^4$.\\
			Hạng tử $-2x^2$ có hệ số là $-2$, bậc là $2$; hạng tử  $-y^3$ có hệ số là $-1$, bậc là $3$; hạng tử $2z^4$ có hệ số là $2$, bậc là $4$.
			\item $F=(3xy^2z+xy^2z+2xy^2z)+(-xyz-3xyz)=6xy^2z-4xyz$.\\
			Hạng tử $6xy^2z$ có hệ số là $6$, bậc là $4$; hạng tử $-4xyz$ có hệ số là $-4$, bậc là $3$.
		\end{enumerate}	}
\end{vd}
\begin{vd}%[Dự án EX-9-Đề Cương Toán 8-L1]%[Trần Chiến]%[8D1N1-7]
	Tính giá trị của các đa thức sau
	\begin{enumerate}
		\item $A=2x^2y-3xy+y^2$ tại $x=2$, $y=-1$;
		\item $B=x^2y-yz^2+2xz$ tại $x=1$, $y=2$ và $z=-1$.
	\end{enumerate}
	\loigiai{
		\begin{enumerate}
			\item Thay $x=2$, $y=-1$ vào đa thức $A$ ta được
			\[A= 2\cdot 2^2\cdot (-1) - 3\cdot 2\cdot (-1)+(-1)^2=-8+6+1=-1.\]
			\item Thay $x=1$, $y=2$ và $z=-1$ vào đa thức $B$ ta được
			\[B= 1^2\cdot 2-2\cdot(-1)^2+2\cdot 1\cdot (-1)=-2.\]
		\end{enumerate}
	}
\end{vd}
\subsubsection{Bài tập}

\begin{bt}%[Dự án EX-9-Đề Cương Toán 8-L1]%[Trần Chiến]%[8D1N1-2] 
	 Chỉ ra các đa thức trong các biểu thức sau
	 \begin{multicols}{3}
	\begin{enumerate}
		\item $\sqrt{5}+1$;
		\item $x^3-\sqrt{5}x^2+5$;
		\item $2x^2y-3xy^2+\dfrac{6}{5}y$;
		\item $\dfrac{1}{2}t^4-3t^2+7$;
		\item $\dfrac{\sqrt{3}}{x}+3$;
		\item $\sqrt{x}+0{,}3x^2$;
		\item $\dfrac{1}{x+y^2}+z$;
		\item $x^{2027}+\sqrt{8}$;
		\item $\dfrac{3}{y^2}-1{,}9y$;
		\item $\dfrac{1}{m^2n}+2mn+2025$;
		\item $\dfrac{5}{p}+\dfrac{4}{9}p^2q-4x$;
		\item $\dfrac{x+y^5}{6}-2026$.
	\end{enumerate}
\end{multicols}
	\loigiai{Các đa thức là $(1)$, $(2)$, $(3)$, $(4)$, $(8)$ và $(12)$.}
\end{bt}


\begin{bt}%[Dự án EX-9-Đề Cương Toán 8-L1]%[Trần Chiến]%[8D1H1-8]
	 Thu gọn rồi tìm bậc của mỗi đa thức sau
	\begin{enumerate}
		\item $A=3{,}5x^2y-\dfrac{5}{4}xy^2+2x^2y+0{,}8xy^2-7x$; 
		\item $B=5x^3y^2-\dfrac{5}{2}x^3y^2+1{,}5x^2y-4x+6$;
		\item $C=\dfrac{1}{2}x^2y^2-0{,}75xy^2+\dfrac{5}{4}xy^2-2x^2y^2$;
		\item $D=4{,}2a^2b-\dfrac{7}{2}ab^2+2a^2b+2{,}5ab^2-7$;
		\item $E=2{,}5m^2n^2-\dfrac{7}{2}mn+1{,}5m^2n^2+2mn-4$;
		\item $F=\dfrac{5}{2}pqr-1{,}5pq+\dfrac{7}{2}pq-2pqr+5$;
		\item $G=1{,}5x^2-2xy+\dfrac{7}{2}y^2+2{,}5x^2y-\dfrac{6}{5}xy^2+5$;
		\item $H=2{,}5x^3y-\dfrac{3}{2}x^2y^2+4xy-0{,}5x^3y+2x^2y^2+5xy-7$.
	\end{enumerate}
	\loigiai{
		\begin{enumerate}
			\item $A=(3{,}5+2)x^2y+\left(-\dfrac{5}{4}+0{,}8\right)xy^2-7x=5{,}5x^2y-\dfrac{9}{20}xy^2-7x$. \\ 
			Bậc của đa thức là $3$.
			\item $B=\left(5-\dfrac{5}{2}\right)x^3y^2+1{,}5x^2y-4x+6=\dfrac{5}{2}x^3y^2+1{,}5x^2y-4x+6$. \\
			Bậc của đa thức là $5$.
			\item $C=\left(\dfrac{1}{2}-2\right)x^2y^2+(-0{,}75+\dfrac{5}{4})xy^2=-\dfrac{3}{2}x^2y^2+\dfrac{1}{2}xy^2$. \\
			Bậc của đa thức là $4$.
			\item $D=(4{,}2+2)a^2b+\left(-\dfrac{7}{2}+2{,}5\right)ab^2-7=6{,}2a^2b-ab^2-7$. \\
			Bậc của đa thức là $3$.
			\item $E=(2{,}5+1{,}5)m^2n^2+\left(-\dfrac{7}{2}+2\right)mn-4=4m^2n^2-\dfrac{3}{2}mn-4$. \\
			Bậc của đa thức là $4$.
			\item $F=\left(\dfrac{5}{2}-2\right)pqr+\left(-1{,}5+\dfrac{7}{2}\right)pq+5=\dfrac{1}{2}pqr+2pq+5$. \\
			Bậc của đa thức là $3$.
			\item $G=1{,}5x^2-2xy+\dfrac{7}{2}y^2+2{,}5x^2y-\dfrac{6}{5}xy^2+5$. \\
			Bậc của đa thức là $3$.
			\item $H=(2{,}5-0{,}5)x^3y+\left(-\dfrac{3}{2}+2\right)x^2y^2+(4+5)xy-7=2x^3y+\dfrac{1}{2}x^2y^2+9xy-7$. \\
			Bậc của đa thức là $4$.
		\end{enumerate}
	}
\end{bt}

\begin{bt}%[Dự án EX-9-Đề Cương Toán 8-L1]%[Trần Chiến]%[8D1H1-7] 
	Tính giá trị của các đa thức sau
	\begin{enumerate}
		\item $A=3x^3-\dfrac{1}{2}x^2+5{,}5x-4$ tại $x=-1$;
		\item $B=-x^4+2x^2-\dfrac{1}{2}x+1{,}25$ tại $x=\dfrac{1}{2}$;
		\item $C=2x^2y-\dfrac{3}{2}xy^2+\dfrac{5}{2}y$ tại $x=1$, $y=-2$;
		\item $D=-x^2y+4{,}5xy^2-2y$ tại $x=0$, $y=3$;
		\item $E=x^3-2xy+\dfrac{3}{4}y^2$ tại $x=-2$, $y=1$;
		\item $F=\dfrac{1}{2}xyz-2xz^2+y^2z$ tại $x=1$, $y=-1$, $z=2$;
		\item $G=-x^2z+0{,}5y^3-3xyz$ tại $x=2$, $y=0$, $z=-1$;
		\item $H=2x^2y^2-3xyz-2{,}5z^3$ tại $x=-1$, $y=1$, $z=-2$.
	\end{enumerate}
\loigiai{
\begin{enumerate}
	\item Thay $x=-1$ vào đa thức $A$ ta được
	\[A=3\cdot(-1)^3-\dfrac{1}{2}\cdot(-1)^2+5{,}5\cdot(-1)-4=-13.\]
	\item Thay $x=\dfrac{1}{2}$ vào đa thức $B$ ta được
	\[B=-\left(\dfrac{1}{2}\right)^4+2\cdot\left(\dfrac{1}{2}\right)^2-\dfrac{1}{2}\cdot\dfrac{1}{2}+1{,}25=\dfrac{23}{16}.\]	
	\item Thay $x=1$, $y=-2$ vào đa thức $C$ ta được
	\[C=2\cdot1^2\cdot(-2)-\dfrac{3}{2}\cdot1\cdot(-2)^2+\dfrac{5}{2}\cdot(-2)=-4-6-5=-15.\]	
	\item Thay $x=0$, $y=3$ vào đa thức $D$ ta được
	\[D=-0^2\cdot3+4{,}5\cdot0\cdot3^2-2\cdot3=0+0-6=-6.\]
	\item Thay $x=-2$, $y=1$ vào đa thức $E$ ta được
	\[E=(-2)^3-2\cdot(-2)\cdot1+\dfrac{3}{4}\cdot1^2=-8+4+\dfrac{3}{4}=-\dfrac{13}{4}.\]
	\item Thay $x=1$, $y=-1$, $z=2$ vào đa thức $F$ ta được
	\[F=\dfrac{1}{2}\cdot1\cdot(-1)\cdot2-2\cdot1\cdot2^2+(-1)^2\cdot2=-1-8+2=-7.\]
	\item Thay $x=2$, $y=0$, $z=-1$ vào đa thức $G$ ta được
	\[G=-(2)^2\cdot(-1)+0{,}5\cdot0^3-3\cdot2\cdot0\cdot(-1)=4+0+0=4.\]
	\item Thay $x=-1$, $y=1$, $z=-2$ vào đa thức $H$ ta được
	\[H=2\cdot(-1)^2\cdot1^2-3\cdot(-1)\cdot1\cdot(-2)-2{,}5\cdot(-2)^3=2-6+20=16.\]
\end{enumerate}}
\end{bt}


\begin{bt}%[Dự án EX-9-Đề Cương Toán 8-L1]%[Trần Chiến]%[8D1H1-7] 
	Thu gọn và tính giá trị của các đa thức sau
	\begin{enumerate}
		\item $A=6x^4y^3-4x^3y^2+3x^4y^3-5x^3y^2+2xy^4$ tại $x=2$, $y=-1$;
		\item $B=7x^5-3x^5+4x^3y-2x^3y+xy^2$ tại $x=1$, $y=3$;
		\item $C=5x^3y^2-6x^2y^3+4x^3y^2-3x^2y^3$ tại $x=-1$, $y=2$;
		\item $D=9xy^4-7x^2y^3+5xy^4+4x^2y^3$ tại $x=2$, $y=1$;
		\item $E=3x^4y^2z-5x^3y^2z+2x^4y^2z-4x^3yz^2+xy^2z^3+6x^3yz^2$ tại $x=1$, $y=2$ và $z=-1$;
		\item $F=7x^5y^3z^2-3x^5y^3z^2+xy^2z^4+5xyz^3-2xyz^3$ tại $x=2$, $y=-1$ và $z=3$;
		\item $G=4xyz^3-6x^2y^2z+3x^2y^2z-5xyz^3$ tại $x=-1$, $y=3$ và $z=2$;
		\item $H=5x^3y^3z^2-2xy^3z^2+7x^3y^3z^2-3xy^3z^2$ tại $x=1$, $y=2$ và $z=-2$.
	\end{enumerate}
	\loigiai{
		\begin{enumerate}
			\item Ta có $A=(6x^4y^3+3x^4y^3)+(-4x^3y^2-5x^3y^2)+2xy^4=9x^4y^3-9x^3y^2+2xy^4$.\\
			Thay $x=2$, $y=-1$ vào $A$ ta được
			\[A=9\cdot2^4\cdot(-1)^3-9\cdot2^3\cdot(-1)^2+2\cdot2\cdot(-1)^4=9\cdot16\cdot(-1)-9\cdot8\cdot 1+4=-212.\]
			\item Ta có $B=(7x^5-3x^5)+(4x^3y-2x^3y)+xy^2=4x^5+2x^3y+xy^2$.\\
			Thay $x=1$, $y=3$ vào $B$ ta được
			\[B=4\cdot1^5+2\cdot1^3\cdot3+1\cdot3^2=4+6+9=19.\]
			\item Ta có $C=(5x^3y^2+4x^3y^2)+(-6x^2y^3-3x^2y^3)=9x^3y^2-9x^2y^3$.\\
			Thay $x=-1$, $y=2$ vào ta được
			\[C=9\cdot(-1)^3\cdot2^2-9\cdot(-1)^2\cdot2^3=-36-72=-108.\]
			\item Ta có $D=(9xy^4+5xy^4)+(-7x^2y^3+4x^2y^3)=14xy^4-3x^2y^3$.\\
			Thay $x=2$, $y=1$ vào $D$ ta được
			\[D=14\cdot2\cdot1^4-3\cdot2^2\cdot1^3=28-12=16.\]
			\item Ta có $E=(3x^4y^2z+2x^4y^2z)+(-5x^3y^2z)+(-4x^3yz^2+6x^3yz^2)+xy^2z^3=5x^4y^2z-5x^3y^2z+2x^3yz^2+xy^2z^3$.\\
			Thay $x=1$, $y=2$ và $z=-1$ vào $E$ ta được
			\[E=5\cdot1^4\cdot2^2\cdot(-1)-5\cdot1^3\cdot2^2\cdot(-1)+2\cdot1^3\cdot2\cdot(-1)^2+1\cdot2^2\cdot(-1)^3=-20+20+4-4=0.\]
			\item Ta có $F=(7x^5y^3z^2-3x^5y^3z^2)+xy^2z^4+(5xyz^3-2xyz^3)=4x^5y^3z^2+xy^2z^4+3xyz^3$.\\
			Thay $x=2$, $y=-1$, $z=3$ vào $F$ ta được
			\[F=4\cdot2^5\cdot(-1)^3\cdot3^2+2\cdot(-1)^2\cdot3^4+3\cdot2\cdot(-1)\cdot3^3=-1\,152.\]
			\item Ta có $G=(4xyz^3-5xyz^3)+(-6x^2y^2z+3x^2y^2z)=-xyz^3-3x^2y^2z$.\\
			Thay $x=-1$, $y=3$ và $z=2$ vào $G$ ta được
			\[G=-(-1)\cdot3\cdot2^3-3\cdot(-1)^2\cdot3^2\cdot2=24-54=-30.\]
			\item Ta có $H=(5x^3y^3z^2+7x^3y^3z^2)+(-2xy^3z^2-3xy^3z^2)=12x^3y^3z^2-5xy^3z^2$.\\
			Thay $x=1$, $y=2$ và $z=-2$ vào $H$ ta được
			\[H=12\cdot1^3\cdot2^3\cdot(-2)^2-5\cdot1\cdot2^3\cdot(-2)^2=224.\]
	\end{enumerate}}
\end{bt}

\subsection{Cộng, trừ hai đa thức }
\subsubsection{Kiến thức trọng tâm}
\begin{tomtat}
	Để cộng, trừ hai đa thức ta thực hiện các bước
	\begin{itemize}
	\item Bỏ dấu ngoặc (sử dụng quy tắc dấu ngoặc).
	\item Nhóm các đơn thức đồng dạng (sử dụng tính chất giao hoán và kết hợp).
	\item Cộng, trừ các đơn thức đồng dạng.
	\end{itemize}
\end{tomtat}

\begin{vd}%[Dự án EX-9-Đề Cương Toán 8-L1]%[Trần Chiến]%[8D1H2-2] 
	Cho hai đa thức  $A=1+3xy-2x^2y^2$ và $B=x-xy+x^2y^2$.
	\begin{multicols}{2}
	\begin{enumerate}
		\item Tìm tổng $A+B$;
		\item Tìm hiệu $A-B$.
	\end{enumerate}
	\end{multicols}
	\loigiai{
		\begin{enumerate}
			\item $A+B=(1+3xy-2x^2y^2)+(x-xy+x^2y^2)= 1+x+(3xy-xy)+(-2x^2y^2+x^2y^2)=1+x+2xy-x^2y^2$.  
			\item $A-B=(1+3xy-2x^2y^2)-(x-xy+x^2y^2)=1-x+(3xy+xy)+(-2x^2y^2-x^2y^2)=1-x+4xy-3x^2y^2$.
	\end{enumerate}}
\end{vd}


\subsubsection{Bài tập}

\begin{bt}%[Dự án EX-9-Đề Cương Toán 8-L1]%[Trần Chiến]%[8D1H2-2] 
  Thực hiện $A+B$ và $A-B$ trong mỗi trường hợp sau
	\begin{enumerate}
		\item $A=4x^2y^2-3xy^3+5x^3y-xy+2x-3$ và $B=-4x^2y^2-4xy^2-x^3y+xy+y+1$;
		\item $A=2x^3y-3x^2y^2+4x^4-5$ và $B=-2x^3y+5x^2y^2-3x^4+7$;
		\item $A=3x^4y^2-2x^2y^3+x-6$ và $B=-3x^4y^2+4x^2y^3-2x+4$;
		\item $A=5x^3y^2-2xy^4+3x^2-1$ và $B=-5x^3y^2+3xy^4-4x^2+2$;
		\item $A=2x^3y^2z-3xy^4z^2+4x^2yz^3-5$ và $B=-2x^3y^2z+5xy^4z^2-3x^2yz^3+1$;
		\item $A=4x^2y^3z-2x^3yz^2+3xyz^4-z$ và $B=-4x^2y^3z+3x^3yz^2-5xyz^4+2z$;
		\item $A=6x^2y^3z-3xyz^4+2x^4-1$ và $B=-6x^2y^3z+5xyz^4-4x^4+3$;
		\item $A=3x^4y^2-2y^3z^2+5xz^3-7$ và $B=-3x^4y^2+4y^3z^2-6xz^3+2$.
	\end{enumerate}
	\loigiai{
		\begin{enumerate}
			\item Ta có	\begin{eqnarray*}
				A+B &=& (4x^2y^2-3xy^3+5x^3y-xy+2x-3)+(-4x^2y^2-4xy^2-x^3y+xy+y+1) \\
				&=& (4x^2y^2-4x^2y^2)+(-3xy^3)+ (5x^3y-x^3y)+(-xy+xy)+2x+(-4xy^2)+y+(-3+1) \\
				&=& 4x^3y-3xy^3-4xy^2+2x+y-2.
			\end{eqnarray*}
			\begin{eqnarray*}
				A-B &=& (4x^2y^2-3xy^3+5x^3y-xy+2x-3)-(-4x^2y^2-4xy^2-x^3y+xy+y+1) \\
				&=& 4x^2y^2-3xy^3+5x^3y-xy+2x-3+4x^2y^2+4xy^2+x^3y-xy-y-1 \\
				&=& (4x^2y^2+4x^2y^2)+(5x^3y+x^3y)+(-3xy^3)+(-xy-xy)+2x+4xy^2+(-y)+(-3-1) \\
				&=& 8x^2y^2+6x^3y-3xy^3-2xy+2x+4xy^2-y-4.
			\end{eqnarray*}
			\item Ta có
			\begin{eqnarray*}
				A+B &=& (2x^3y-3x^2y^2+4x^4-5)+(-2x^3y+5x^2y^2-3x^4+7) \\
				&=& (2x^3y-2x^3y)+(-3x^2y^2+5x^2y^2)+(4x^4-3x^4)+(-5+7) \\
				&=& 2x^2y^2+x^4+2.
			\end{eqnarray*}
			\begin{eqnarray*}
				A-B &=& (2x^3y-3x^2y^2+4x^4-5)-(-2x^3y+5x^2y^2-3x^4+7) \\
				&=& 2x^3y-3x^2y^2+4x^4-5+2x^3y-5x^2y^2+3x^4-7 \\
				&=& (2x^3y+2x^3y)+(-3x^2y^2-5x^2y^2)+(4x^4+3x^4)+(-5-7) \\
				&=& 4x^3y-8x^2y^2+7x^4-12.
			\end{eqnarray*}
			\item Ta có
			\begin{eqnarray*}
				A+B &=& (3x^4y^2-2x^2y^3+x-6)+(-3x^4y^2+4x^2y^3-2x+4) \\
				&=& (3x^4y^2-3x^4y^2)+(-2x^2y^3+4x^2y^3)+(x-2x)+(-6+4) \\
				&=& 2x^2y^3-x-2.
			\end{eqnarray*}
			\begin{eqnarray*}
				A-B &=& (3x^4y^2-2x^2y^3+x-6)-(-3x^4y^2+4x^2y^3-2x+4) \\
				&=& 3x^4y^2-2x^2y^3+x-6+3x^4y^2-4x^2y^3+2x-4 \\
				&=& (3x^4y^2+3x^4y^2)+(-2x^2y^3-4x^2y^3)+(x+2x)+(-6-4) \\
				&=& 6x^4y^2-6x^2y^3+3x-10.
			\end{eqnarray*}
			\item Ta có
			\begin{eqnarray*}
				A+B &=& (5x^3y^2-2xy^4+3x^2-1)+(-5x^3y^2+3xy^4-4x^2+2) \\
				&=& (5x^3y^2-5x^3y^2)+(-2xy^4+3xy^4)+(3x^2-4x^2)+(-1+2) \\
				&=& xy^4-x^2+1.
			\end{eqnarray*}
			\begin{eqnarray*}
				A-B &=& (5x^3y^2-2xy^4+3x^2-1)-(-5x^3y^2+3xy^4-4x^2+2) \\
				&=& 5x^3y^2-2xy^4+3x^2-1+5x^3y^2-3xy^4+4x^2-2 \\
				&=& (5x^3y^2+5x^3y^2)+(-2xy^4-3xy^4)+(3x^2+4x^2)+(-1-2) \\
				&=& 10x^3y^2-5xy^4+7x^2-3.
			\end{eqnarray*}
			\item Ta có
			\begin{eqnarray*}
				A+B &=& (2x^3y^2z-3xy^4z^2+4x^2yz^3-5)+(-2x^3y^2z+5xy^4z^2-3x^2yz^3+1) \\
				&=& (2x^3y^2z-2x^3y^2z)+(-3xy^4z^2+5xy^4z^2)+(4x^2yz^3-3x^2yz^3)+(-5+1) \\
				&=& 2xy^4z^2+x^2yz^3-4.
			\end{eqnarray*}
			\begin{eqnarray*}
				A-B &=& (2x^3y^2z-3xy^4z^2+4x^2yz^3-5)-(-2x^3y^2z+5xy^4z^2-3x^2yz^3+1) \\
				&=& 2x^3y^2z-3xy^4z^2+4x^2yz^3-5+2x^3y^2z-5xy^4z^2+3x^2yz^3-1 \\
				&=& (2x^3y^2z+2x^3y^2z)+(-3xy^4z^2-5xy^4z^2)+(4x^2yz^3+3x^2yz^3)+(-5-1) \\
				&=& 4x^3y^2z-8xy^4z^2+7x^2yz^3-6.
			\end{eqnarray*}
			\item Ta có
			\begin{eqnarray*}
				A+B &=& (4x^2y^3z-2x^3yz^2+3xyz^4-z)+(-4x^2y^3z+3x^3yz^2-5xyz^4+2z) \\
				&=& (4x^2y^3z-4x^2y^3z)+(-2x^3yz^2+3x^3yz^2)+(3xyz^4-5xyz^4)+(-z+2z) \\
				&=& x^3yz^2-2xyz^4+z.
			\end{eqnarray*}
			\begin{eqnarray*}
				A-B &=& (4x^2y^3z-2x^3yz^2+3xyz^4-z)-(-4x^2y^3z+3x^3yz^2-5xyz^4+2z) \\
				&=& 4x^2y^3z-2x^3yz^2+3xyz^4-z+4x^2y^3z-3x^3yz^2+5xyz^4-2z \\
				&=& (4x^2y^3z+4x^2y^3z)+(-2x^3yz^2-3x^3yz^2)+(3xyz^4+5xyz^4)+(-z-2z) \\
				&=& 8x^2y^3z-5x^3yz^2+8xyz^4-3z.
			\end{eqnarray*}
			\item Ta có
			\begin{eqnarray*}
				A+B &=& (6x^2y^3z-3xyz^4+2x^4-1)+(-6x^2y^3z+5xyz^4-4x^4+3) \\
				&=& (6x^2y^3z-6x^2y^3z)+(-3xyz^4+5xyz^4)+(2x^4-4x^4)+(-1+3) \\
				&=& 2xyz^4-2x^4+2.
			\end{eqnarray*}
			\begin{eqnarray*}
				A-B &=& (6x^2y^3z-3xyz^4+2x^4-1)-(-6x^2y^3z+5xyz^4-4x^4+3) \\
				&=& 6x^2y^3z-3xyz^4+2x^4-1+6x^2y^3z-5xyz^4+4x^4-3 \\
				&=& (6x^2y^3z+6x^2y^3z)+(-3xyz^4-5xyz^4)+(2x^4+4x^4)+(-1-3) \\
				&=& 12x^2y^3z-8xyz^4+6x^4-4.
			\end{eqnarray*}
			\item Ta có
			\begin{eqnarray*}
				A+B &=& (3x^4y^2-2y^3z^2+5xz^3-7)+(-3x^4y^2+4y^3z^2-6xz^3+2) \\
				&=& (3x^4y^2-3x^4y^2)+(-2y^3z^2+4y^3z^2)+(5xz^3-6xz^3)+(-7+2) \\
				&=& 2y^3z^2-xz^3-5.
			\end{eqnarray*}
			\begin{eqnarray*}
				A-B &=& (3x^4y^2-2y^3z^2+5xz^3-7)-(-3x^4y^2+4y^3z^2-6xz^3+2) \\
				&=& 3x^4y^2-2y^3z^2+5xz^3-7+3x^4y^2-4y^3z^2+6xz^3-2 \\
				&=& (3x^4y^2+3x^4y^2)+(-2y^3z^2-4y^3z^2)+(5xz^3+6xz^3)+(-7-2) \\
				&=& 6x^4y^2-6y^3z^2+11xz^3-9.
			\end{eqnarray*}
		\end{enumerate}
	}
\end{bt}
\begin{bt}%[Dự án EX-9-Đề Cương Toán 8-L1]%[Trần Chiến]%[8D1H2-2] 
	 Cho hai đa thức $A=7xyz^2-5xy^2z+3x^2yz-xyz+1$ và $B=7x^2yz-5xy^2z+3xyz^2-2$.
	\begin{enumerate}
		\item Tìm đa thức $C$ sao cho $A-C=B$;
		\item Tìm đa thức $D$ sao cho $A+D=B$;
		\item Tìm đa thức $E$ sao cho $E-A=B$.
	\end{enumerate}
	\loigiai{
		\begin{enumerate}
			\item Ta có
			\begin{eqnarray*}
				C=A-B &=& (7xyz^2-5xy^2z+3x^2yz-xyz+1)-(7x^2yz-5xy^2z+3xyz^2-2) \\
				&=& 7xyz^2-5xy^2z+3x^2yz-xyz+1-7x^2yz+5xy^2z-3xyz^2+2 \\
				&=& (7xyz^2-3xyz^2)+(-5xy^2z+5xy^2z)+(3x^2yz-7x^2yz)-xyz+(1+2) \\
				&=& 4xyz^2-4x^2yz-xyz+3.
			\end{eqnarray*}
			\item  Ta có
			\begin{eqnarray*}
				D=B-A &=& (7x^2yz-5xy^2z+3xyz^2-2)-(7xyz^2-5xy^2z+3x^2yz-xyz+1) \\
				&=& 7x^2yz-5xy^2z+3xyz^2-2-7xyz^2+5xy^2z-3x^2yz+xyz-1 \\
				&=& (7x^2yz-3x^2yz)+(-5xy^2z+5xy^2z)+(3xyz^2-7xyz^2)+xyz+(-2-1) \\
				&=& 4x^2yz-4xyz^2+xyz-3.
			\end{eqnarray*}			
			\item Ta có
			\begin{eqnarray*}
				E=A+B &=& (7xyz^2-5xy^2z+3x^2yz-xyz+1)+(7x^2yz-5xy^2z+3xyz^2-2) \\
				&=& 7xyz^2-5xy^2z+3x^2yz-xyz+1+7x^2yz-5xy^2z+3xyz^2-2 \\
				&=& (7xyz^2+3xyz^2)+(-5xy^2z-5xy^2z)+(3x^2yz+7x^2yz)-xyz+(1-2) \\
				&=& 10xyz^2-10xy^2z+10x^2yz-xyz-1.
			\end{eqnarray*}
		\end{enumerate}
	}
\end{bt}

\subsection{Nhân hai đa thức}
\subsubsection{Kiến thức trọng tâm}
\begin{tomtat}
\begin{enumerate}
	\item Để nhân đơn thức với đa thức, ta nhân đơn thức đó với từng số hạng của đa thức, rồi cộng các kết quả với nhau.
	\item Để nhân hai đa thức, ta nhân từng số hạng của đa thức này với đa thức kia, rồi cộng các kết quả với nhau.
\end{enumerate}
\end{tomtat}

\begin{vd}%[Dự án EX-9-Đề Cương Toán 8-L1]%[Trần Chiến]%[8D1H2-3] 
	 Thực hiện phép nhân đa thức
	\begin{enumerate}
		\item $x^2y(2x-3y+xy)$;
		\item $\left(-\dfrac{1}{4}xy\right)(4x^2-8xy+y^2)$;
		\item $(2x-y)(x+3y)$;
		\item $(2x-3y)(x^2+2xy-4y^2)$. 
	\end{enumerate}
	\loigiai{
		\begin{enumerate}
			\item $x^2y(2x-3y+xy)=x^2y\cdot 2x+x^2y\cdot(-3y)+x^2y\cdot xy =2x^3y-3x^2y^2+x^3y^2$.
			\item $\left(-\dfrac{1}{4}xy\right)(4x^2-8xy+y^2)=\left(-\dfrac{1}{4}xy\right)\cdot 4x^2+\left(-\dfrac{1}{4}xy\right)\cdot(-8xy)+\left(-\dfrac{1}{4}xy\right)\cdot y^2=-x^3y+2x^2y^2-\dfrac{1}{4}xy^3$.
			\item $(2x-y)(x+3y)=2x\cdot x+2x\cdot3y+(-y)\cdot x+(-y)\cdot3y=2x^2+6xy-xy-3y^2=2x^2+5xy-3y^2$.
			\item $(2x-3y)(x^2+2xy-4y^2)=2x\cdot x^2+2x\cdot 2xy+2x\cdot(-4y^2)+(-3y)\cdot x^2+(-3y)\cdot2xy+(-3y)\cdot(-4y^2)=2x^3+4x^2y-8xy^2-3x^2y-6xy^2+12y^3=2x^3+x^2y-14xy^2+12y^3$.
		\end{enumerate}
		
	}
\end{vd}

\subsubsection{Bài tập}

\begin{bt}%[Dự án EX-9-Đề Cương Toán 8-L1]%[Trần Chiến]%[8D1H2-3] 
	  Thực hiện phép nhân
	\begin{multicols}{2}
	\begin{enumerate}
		\item $x^2\left(5x^3-x-\dfrac{1}{2}\right)$;
		\item $(3xy-x^2+y)\cdot\dfrac{2}{3}x^2y$;
		\item $(4x^3-5xy+2x)\left(-\dfrac{1}{2}xy\right)$;
		\item $-2x^3y(2x^2-3y+5yz)$;
		\item $\dfrac{2}{5}xy(x^2y-5x+10y)$;
		\item $\dfrac{2}{3}x^2y(3xy-x^2+y)$.
	\end{enumerate}
	\end{multicols}
	\loigiai{
		\begin{enumerate}
			\item $x^2\left(5x^3-x-\dfrac{1}{2}\right)=x^2\cdot 5x^3+x^2\cdot(-x)+x^2\cdot\left(-\dfrac{1}{2}\right)=5x^5-x^3-\dfrac{1}{2}x^2$.
			\item $(3xy-x^2+y)\cdot\dfrac{2}{3}x^2y=3xy\cdot\dfrac{2}{3}x^2y+(-x^2)\cdot\dfrac{2}{3}x^2y+y\cdot\dfrac{2}{3}x^2y=2x^3y^2-\dfrac{2}{3}x^4y+\dfrac{2}{3}x^2y^2$.
			\item $(4x^3-5xy+2x)\cdot\left(-\dfrac{1}{2}xy\right)=4x^3\cdot\left(-\dfrac{1}{2}xy\right)-5xy\cdot\left(-\dfrac{1}{2}xy\right)+2x\cdot\left(-\dfrac{1}{2}xy\right)=-2x^4y+\dfrac{5}{2}x^2y^2-x^2y$.
			\item $(-2x^3y)(2x^2-3y+5yz)=-2x^3y\cdot2x^2-2x^3y\cdot(-3y)-2x^3y\cdot(5yz)=-4x^5y+6x^3y^2-10x^3y^2z$.
			\item $\dfrac{2}{5}xy(x^2y-5x+10y)=\dfrac{2}{5}xy\cdot x^2y+\dfrac{2}{5}xy\cdot (-5x)+\dfrac{2}{5}xy\cdot 10y=\dfrac{2}{5}x^3y^2-2x^2y+4xy^2$.
			\item $\dfrac{2}{3}x^2y(3xy-x^2+y)=\dfrac{2}{3}x^2y\cdot 3xy+\dfrac{2}{3}x^2y\cdot (-x^2)+\dfrac{2}{3}x^2y\cdot y=2x^3y^2-\dfrac{2}{3}x^4y+\dfrac{2}{3}x^2y^2$.
	\end{enumerate}}
\end{bt}


\begin{bt}%[Dự án EX-9-Đề Cương Toán 8-L1]%[Trần Chiến]%[8D1H2-3] 
	 Thực hiện phép nhân
	\begin{multicols}{2}		
		\begin{enumerate}
			\item $(x^2-1)(x^2+2x)$;
			\item $(x-2y)(x^2y^2-xy+2y)$;
			\item $(x+3)(x^2+3x-5)$;
			\item $(x+1)(x^2-x+1)$;
			\item $(2x-3y)(x^2+y+4)$;
			\item $\left(\dfrac{1}{2}xy-1\right)(x^3-2x-6)$;
			\item $(x+y+1)(x^2+y^2+1)$;
			\item $(2x-y+3)(x^2-xy+2)$.
		\end{enumerate}
	\end{multicols}
	\loigiai{
		\begin{enumerate}
			\item \begin{eqnarray*}
				(x^2-1)(x^2+2x) &=& x^2(x^2+2x)-(x^2+2x) \\
				&=& x^4+2x^3-x^2-2x.
			\end{eqnarray*}
			\item \begin{eqnarray*}
				(x-2y)(x^2y^2-xy+2y)&=& x(x^2y^2-xy+2y)-2y(x^2y^2-xy+2y) \\
				&=& x^3y^2-x^2y+2xy-2x^2y^3+2xy^2-4y^2.
			\end{eqnarray*}
			\item \begin{eqnarray*}
				(x+3)(x^2+3x-5)&=& x(x^2+3x-5)+3(x^2+3x-5) \\
				&=& x^3+3x^2-5x+3x^2+9x-15 \\
				&=& x^3+6x^2+4x-15.
			\end{eqnarray*}
			\item \begin{eqnarray*}
				(x+1)(x^2-x+1) &=& x(x^2-x+1)+1(x^2-x+1) \\
				&=& x^3-x^2+x+x^2-x+1 \\
				&=& x^3+1.
			\end{eqnarray*}
			\item \begin{eqnarray*}
				(2x-3y)(x^2+y+4)&=& 2x(x^2+y+4)-3y(x^2+y+4) \\
				&=& 2x^3+2xy+8x-3x^2y-3y^2-12y\\
				&=& 2x^3-3x^2y+2xy+8x-3y^2-12y.
			\end{eqnarray*}
			\item \begin{eqnarray*}
				\left(\dfrac{1}{2}xy-1\right)(x^3-2x-6) &=& \dfrac{1}{2}xy(x^3-2x-6)-(x^3-2x-6) \\
				&=& \dfrac{1}{2}x^4y-x^2y-3xy-x^3+2x+6.
			\end{eqnarray*}
	\end{enumerate}}
\end{bt}

\begin{bt}%[Dự án EX-9-Đề Cương Toán 8-L1]%[Trần Chiến]%[8D1H2-5] 
	Rút gọn các biểu thức sau
	\begin{enumerate}
		\item $A=(3x^3-2x^2+4)+(5x^2-7)-(2x^3-3x+1)+x(2x^3-5x^2+3)$;
		\item $B=(2x^3y^2-3x^2y+4y^3)-(x^3y^2-2x^2y+5y^2)+xy(3x^2-2y^2)$;
		\item $C=(3x^2y-2y^3+4x^3)-(2x^2y-y^3+5x^2)+y(x^3-3x^2+2)$;
		\item $D=(x^2+3x^3)(2x^2-5)-(x^2-1)(x^2+2x)$;
		\item $E=(x^2+y^2)(2x^2-y)-(x^2-y)(x^2+2y^2)$.
	\end{enumerate}
	\loigiai{
		\begin{enumerate}
			\item Ta có
			\begin{eqnarray*}
				A &=& (3x^3-2x^2+4)+(5x^2-7)-(2x^3-3x+1)+x(2x^3-5x^2+3) \\
				&=& (3x^3-2x^3)+(-2x^2+5x^2)+(4-7-1)+(3x+3x)+2x^4-5x^3\\
				&=& x^3+3x^2-4+2x^4-5x^3+6x \\
				&=& 2x^4-4x^3+3x^2+6x-4.
			\end{eqnarray*}
			\item Ta có
			\begin{eqnarray*}
				B &=& (2x^3y^2-3x^2y+4y^3)-(x^3y^2-2x^2y+5y^2)+(3x^3y-2xy^3) \\
				&=& (2x^3y^2-x^3y^2)+(-3x^2y+2x^2y)+4y^3-5y^2+3x^3y-2xy^3 \\
				&=& x^3y^2-x^2y-5y^2+4y^3+3x^3y-2xy^3.
			\end{eqnarray*}
			\item Ta có
			\begin{eqnarray*}
				C &=& (3x^2y-2y^3+4x^3)-(2x^2y-y^3+5x^2)+(x^3y-3x^2y+2y) \\
				&=& (3x^2y-2x^2y-3x^2y)+(-2y^3+y^3)+4x^3-5x^2+x^3y+2y \\
				&=& -2x^2y-y^3+4x^3-5x^2+x^3y+2y.
			\end{eqnarray*}
			\item Ta có
			\begin{eqnarray*}
				D &=& (x^2+3x^3)(2x^2-5)-(x^2-1)(x^2+2x) \\
				&=& (2x^4-5x^2+6x^5-15x^3)-(x^4+2x^3-x^2-2x) \\
				&=& (6x^5+2x^4-15x^3-5x^2)-(x^4+2x^3-x^2-2x) \\
				&=& 6x^5+(2x^4-x^4)+(-15x^3-2x^3)+(-5x^2+x^2)+2x \\
				&=& 6x^5+x^4-17x^3-4x^2+2x.
			\end{eqnarray*}
			\item Ta có
			\begin{eqnarray*}
				E &=& (x^2+y^2)(2x^2-y)-(x^2-y)(x^2+2y^2) \\
				&=& (2x^4-x^2y+2x^2y^2-y^3)-(x^4+2x^2y^2-x^2y-2y^3) \\
				&=& 2x^4-x^2y+2x^2y^2-y^3-x^4-2x^2y^2+x^2y+2y^3 \\
				&=& (2x^4-x^4)+( -x^2y+x^2y)+(2x^2y^2-2x^2y^2)+(-y^3+2y^3) \\
				&=& x^4+y^3.
			\end{eqnarray*}
		\end{enumerate}
	}
\end{bt}

\begin{bt}%[Dự án EX-9-Đề Cương Toán 8-L1]%[Trần Chiến]%[8D1V2-5]
	  Chứng minh rằng giá trị của biểu thức sau không phụ thuộc vào biến $x$.
	\begin{enumerate}
		\item $A=(3x+7)(2x+3)-(3x-5)(2x+11)$;
		\item $B=(x^2-2)(x^2+x-1)-x(x^3+x^2-3x-2)$;
		\item $C=x(x^3+x^2-3x-2)-(x^2-2)(x^2+x-1)$;
		\item $D=7y(4y^2x-3yx^2+1)-7xy(4y^2-2yx)+7x^2y^2$;
		\item $E=(x^2y-3y)(x+y)-(x^3y+x^2y^2-3xy-3y^2)$;
		\item $F=(xy+2)(y^2-1)-y(y^2x+y-x-2)$.
	\end{enumerate}
	\loigiai{
		\begin{enumerate}
			\item Ta có
			\begin{eqnarray*}
				A &=& (3x+7)(2x+3)-(3x-5)(2x+11) \\
				&=& (6x^2+23x+21)-(6x^2+23x-55) \\
				&=& 76.
			\end{eqnarray*}
			Vậy $A=76$, không phụ thuộc vào $x$.
			\item Ta có
			\begin{eqnarray*}
				B &=& (x^2-2)(x^2+x-1)-x(x^3+x^2-3x-2) \\
				&=& (x^4+x^3-3x^2-2x+2)-(x^4+x^3-3x^2-2x) \\
				&=& 2.
			\end{eqnarray*}
			Vậy $B=2$, không phụ thuộc vào $x$.		
			\item Ta có
			\begin{eqnarray*}
				C &=& x(x^3+x^2-3x-2)-(x^2-2)(x^2+x-1) \\
				&=& (x^4+x^3-3x^2-2x)-(x^4+x^3-3x^2-2x+2) \\
				&=& -2.
			\end{eqnarray*}
			Vậy $C=-2$, không phụ thuộc vào $x$.		
			\item Ta có
			\begin{eqnarray*}
				D &=& 7y(2y^2x-5yx^2+1)-7yx(2y^2-4yx)+7x^2y^2 \\
				&=& 14xy^3-35x^2y^2+7y-14xy^3+28x^2y^2+7x^2y^2 \\
				&=& (14xy^3-14xy^3)+(-35x^2y^2+28x^2y^2+7x^2y^2)+7y \\
				&=& 7y.
			\end{eqnarray*}
			Vậy $D=7y$, không phụ thuộc vào $x$.		
			\item Ta có
			\begin{eqnarray*}
				E &=& (x^2y-3y)(x+y)-(x^3y+x^2y^2-3xy-3y^2) \\
				&=& (x^3y+x^2y^2-3xy-3y^2)-(x^3y+x^2y^2-3xy-3y^2) \\
				&=& 0.
			\end{eqnarray*}
			Vậy $E=0$, không phụ thuộc vào $x$.		
			\item Ta có
			\begin{eqnarray*}
				F &=& (xy+2)(y^2-1)-y(y^2x+y-x-2) \\
				&=& (xy^3-xy+2y^2-2)-(xy^3+y^2-xy-2y) \\
				&=& xy^3-xy+2y^2-2-xy^3-y^2+xy+2y \\
				&=& (xy^3-xy^3)+(-xy+xy)+(2y^2-y^2)+2y-2 \\
				&=& y^2+2y-2.
			\end{eqnarray*}
			Vậy $F=y^2+2y-2$, không phụ thuộc vào $x$.
		\end{enumerate}
	}
\end{bt}