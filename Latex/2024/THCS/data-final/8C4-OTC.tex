\section*{BÀI TẬP CUỐI CHƯƠNG 4}
\subsection{Câu hỏi trắc nghiệm}
\Opensolutionfile{ans}[ans/ans-8C4-OTC]
\begin{ex}%%[8C6N1-2]
	Cho bảng dữ liệu sau
	\begin{center}
		\renewcommand{\arraystretch}{1.5}
		\begin{tabular}{|c|l|c|c|c|c|}
			\hline
			$1$ & \textbf{Xếp loại học tập} & \textbf{Tốt} & \textbf{Khá} & \textbf{Đạt} & \textbf{Chưa đạt} \\
			\hline
			$2$ & Số học sinh              & $10$         & $15$        & $10$         & $5$              \\ \hline
			$3$ & Tỉ lệ phần trăm          & $25\%$       & $38\%$      & $25\%$       & $12\%$           \\
			\hline
		\end{tabular}
	\end{center}
	Dựa vào bảng thống kê trên, hãy cho biết dữ liệu ở dòng nào thuộc loại dữ liệu định tính?
	\choice
	{$2$}
	{\True$1$}
	{$3$}
	{ $2$ và $3$}
	\loigiai{
		Dựa vào bảng thống kê trên, ta thấy dữ liệu ở dòng $1$ (Xếp loại học tập) là dữ liệu định tính.
	}
\end{ex}
\begin{ex}%[8C6H1-2]
	Dữ liệu nào sau đây là số liệu liên tục?
	\choice
	{Dữ liệu về tên các vận động viên Việt Nam tham dự SEA Games $31$}
	{Dữ liệu về kết quả đánh giá hiệu quả của chương trình dạy học trên truyền hình}
	{Dữ liệu về cân nặng của $200$ con cá chép sau $6$ tháng nuôi}
	{\True Dữ liệu về số người bị mắc Covid-$19$ trong gia đình của các bạn trong lớp}
	\loigiai{}
\end{ex}
\begin{ex}%[8C6N1-2]
	An đứng từ xa và ghi lại xem bạn nào đi sang đường sử dụng cầu vượt khi tan trường. Phương pháp An thu thập dữ liệu là
	\choice
	{Từ nguồn có sẵn}
	{\True Quan sát}
	{Lập bảng hỏi}
	{Phỏng vấn}
	\loigiai{}
\end{ex}
\begin{ex}%[8C6N2-1]
	Trong biểu đồ cột với gốc trục đứng không bắt đầu từ 0, khẳng định nào sau đây không đúng?
	\choice 
	{Cột cao hơn biểu diễn số liệu lớn hơn}
	{Hai cột cao bằng nhau biểu diễn số liệu bằng nhau}
	{Cột thấp hơn biểu diễn số liệu bé hơn}
	{\True Tỉ lệ chiều cao của hai cột bằng tỉ lệ hai số liệu được biểu diễn}
	\loigiai{}
\end{ex}
\begin{ex}%[8T6N2-3]
	Để biểu diễn tỉ lệ của các phần trăm trong tổng thể ta dùng biểu đồ nào sau đây?
	\choice 
	{Biểu đồ tranh}
	{Biểu đồ cột}
	{Biểu đồ đoạn thẳng}
	{\True Biểu đồ hình quạt tròn}
	\loigiai{}
\end{ex}
\begin{ex}%[8C6N2-2]
	Để biểu diễn sự thay đổi của một đại lượng theo thời gian ta dùng biểu đồ nào sau đây?
	\choice
	{Biểu đồ cột kép}
	{Biểu đồ tranh}
	{\True Biểu đồ đoạn thẳng}
	{Biểu đồ hình quạt tròn}
	\loigiai{}
\end{ex}
\begin{ex}%[8C6H2-1]
	Để so sánh hai tập dữ liệu với nhau ta dùng biểu đồ nào sau đây?
	\choice
	{\True Biểu đồ cột kép}
	{Biểu đồ tranh}
	{Biểu đồ đoạn thẳng}
	{Biểu đồ hình quạt tròn}
	\loigiai{}
\end{ex}
\begin{ex}%[8C6N1-3]
	Phương pháp nào là phù hợp để thống kê dữ liệu về số huy chương của một đoàn thể thao trong một kì Olympic?
	\choice 
	{Làm thí nghiệm}
	{\True Thu thập từ nguồn có sẵn như sách báo, Internet}
	{Phỏng vấn}
	{Quan sát trực tiếp}
	\loigiai{
		Thu thập từ nguồn có sẵn như sách báo, Internet.
	}
\end{ex}
\begin{ex}%[8C6H1-2]
	Cho bảng thống kê sau. Dữ liệu ở dòng nào thuộc loại định lượng và có thể lập tỉ số?
	\begin{center}
		Thống kê xếp loại học tập của học sinh lớp 8A1
		\begin{tabular}{|c|l|c|c|c|c|}
			\hline $1$ & Xếp loại học tập & Tốt & Khá & Đạt & Chưa đạt \\
			\hline $2$ & Số học sinh & $10$ & $15$ & $10$ & $5$ \\
			\hline $3$ & Tỉ lệ phần trăm & $25 \%$ & $38 \%$ & $25 \%$ & $12 \%$ \\
			\hline
		\end{tabular}	
	\end{center}
	\choice
	{$2$ và $3$}
	{\True $2$}
	{$3$}
	{$1$}	
	\loigiai{
		Dữ liệu ở dòng {\bf số học sinh} thuộc loại định lượng và có thể lập tỉ số.	
	}
\end{ex}
\begin{ex}%[8T6H2-3]
	Cho bảng thống kê sau. Loại biểu đồ nào là thích hợp đề biểu diễn dữ liệu ở dòng $3$?
	\begin{center}
		Thống kê xếp loại học tập của học sinh lớp $8A1$
		\begin{tabular}{|c|l|c|c|c|c|}
			\hline $1$ & Xếp loại học tập & Tốt & Khá & Đạt & Chưa đạt \\
			\hline $2$ & Số học sinh & $10$ & $15$ & $10$ & $5$ \\
			\hline $3$ & Tỉ lệ phần trăm & $25 \%$ & $38 \%$ & $25 \%$ & $12 \%$ \\
			\hline
		\end{tabular}	
	\end{center}
	\choice
	{Biểu đồ tranh}
	{Biểu đồ đoạn thẳng}
	{Biểu đồ cột kép}
	{\True Biểu đồ hình quạt tròn}	
	\loigiai{
		Biểu đồ hình quạt tròn.
	}
\end{ex}
%%=====Bài 5
\begin{ex}%[8C6H2-5]
	Cho bảng dữ liệu sau. Loại biểu đồ nào thích hợp để so sánh số lượng ba loại huy chương Vàng, Bạc, Đồng của hai đoàn Việt Nam và Thái Lan?
	\begin{center}
		\begin{tabular}{|l|c|c|c|c|}
			\hline \multicolumn{5}{|c|}{Thống kê huy chương SEA Games $31$}\\
			\hline &Vàng &Bạc & Đồng& Tổng số\\
			\hline Việt Nam & $205$ & $125$ & $116$ & $446$ \\
			Thái Lan & $92$ & $103$ & $137$ & $332$ \\
			Indonesia & $69$ & $91$ & $81$ & $241$ \\
			Philippines & $52$ & $70$ & $105$ & $227$ \\
			Singapore & $47$ & $46$ & $73$ & $166$ \\
			Malaysia & $39$ & $45$ & $90$ & $174$ \\
			Myanmar & $9$ & $18$ & $35$ & $62$ \\
			Campuchia & $9$ & $13$ & $41$ & $63$ \\
			Lào & $2$ & $7$ & $33$ & $42$ \\
			Brunei Darussalam & $1$ & $1$ & $1$ & $3$ \\
			Timor Leste & $0$ & $3$ & $2$ & $5$ \\
			\hline
		\end{tabular}
	\end{center}
	\hfill(Nguồn: https://seagames2021.com)
	\choice
	{Biểu đồ hình quạt tròn}
	{Biểu đồ cột}
	{\True Biểu đồ cột kép}
	{Biểu đồ đoạn thẳng}	
	\loigiai{Biểu đồ cột kép.	
	}
\end{ex}
%%=====Bài 6
\begin{ex}%[8T6H2-3]
	Cho bảng dữ liệu sau. Biểu đồ nào thích hợp để biểu diễn tỉ lệ phần trăm số huy chương vàng của mỗi đoàn so với tổng số huy chương vàng đã trao trong đại hội?
	\begin{center}
		\begin{tabular}{|l|c|c|c|c|}
			\hline \multicolumn{5}{|c|}{Thống kê huy chương SEA Games $31$}\\
			\hline &Vàng &Bạc & Đồng& Tổng số\\
			\hline Việt Nam & $205$ & $125$ & $116$ & $446$ \\
			Thái Lan & $92$ & $103$ & $137$ & $332$ \\
			Indonesia & $69$ & $91$ & $81$ & $241$ \\
			Philippines & $52$ & $70$ & $105$ & $227$ \\
			Singapore & $47$ & $46$ & $73$ & $166$ \\
			Malaysia & $39$ & $45$ & $90$ & $174$ \\
			Myanmar & $9$ & $18$ & $35$ & $62$ \\
			Campuchia & $9$ & $13$ & $41$ & $63$ \\
			Lào & $2$ & $7$ & $33$ & $42$ \\
			Brunei Darussalam & $1$ & $1$ & $1$ & $3$ \\
			Timor Leste & $0$ & $3$ & $2$ & $5$ \\
			\hline
		\end{tabular}
	\end{center}
	\hfill(Nguồn: https://seagames2021.com)
	\choice
	{\True Biểu đồ hình quạt tròn}
	{Biểu đồ cột}
	{Biểu đồ tranh}
	{Biểu đồ đoạn thẳng}	
	\loigiai{
		Biểu đồ hình quạt tròn.	
	}
\end{ex}
\Closesolutionfile{ans}
%-----------
\subsection{Bài tập tự luận}
%%==========Bài 1
\begin{bt}%[8C6H2-3]
	Cho biểu đồ
	\begin{center}
		\begin{tikzpicture}[>=stealth,line join=round,line cap=round,font=\footnotesize,y=1cm, x=1cm,scale=1]
			\draw (-0.75,4.9) node[right,black,scale=0.8]{\large\bf Số sản phẩm bán được theo tháng};
			\def\kc{0.5}
			\foreach \ii in {0,...,4}{	
				\pgfmathsetmacro{\so}{int(\ii*2)} 
				\draw[cyan] (-2pt,\ii) node[left,black]{$\so$}--++(13*\kc,0);
			}
			\draw[cyan] (0,0)--(0,4.25);
			%---
			\def\thoi{node[scale=0.5,diamond, fill=blue]{}}
			\draw[blue,line width=2pt] 
			(1*\kc,1.5) \thoi
			-- (2*\kc,1) \thoi
			-- (3*\kc,1.5) \thoi
			-- (4*\kc,2) \thoi
			-- (5*\kc,2.5) \thoi
			-- (6*\kc,2) \thoi
			-- (7*\kc,3) \thoi
			-- (8*\kc,3.5) \thoi
			-- (9*\kc,4) \thoi
			-- (10*\kc,3.5) \thoi
			-- (11*\kc,3) \thoi
			-- (12*\kc,4) \thoi
			;
			%---
			\foreach \ii in {1,...,12} \draw[cyan] (0.5*\ii,2pt)--++(0,-4pt)node[below,black]{\ii};
			\draw (3,-0.5) node[below,blue]{\bf Tháng};
			\draw (-0.5,2.5) node[rotate=90,above,blue]{\bf Số lượng (nghìn)};
			\draw[cyan] (-1.25,-1.25) rectangle (14*\kc,5.5);
			\draw (0.5*13*\kc,-1.25) node[below=1mm]{\bf a)};
		\end{tikzpicture}
		\hspace*{0.5cm}
		\begin{tikzpicture}[>=stealth,line join=round,line cap=round,font=\footnotesize,y=1cm, x=1cm,scale=1]
			\draw (-0.75,4.9) node[right,black,scale=0.8]{\large\bf Số sản phẩm bán được theo tháng};
			\def\kc{0.5}
			\foreach \ii in {0,...,8}{	
				\pgfmathsetmacro{\so}{int(\ii*2+2)} 
				\draw[cyan] (-2pt,0.5*\ii) node[left,black]{$\so$}--++(13*\kc,0);
			}
			\draw[cyan] (0,0)--(0,4.25);
			%---
			\def\thoi{node[scale=0.5,diamond, fill=blue]{}}
			\draw[blue,line width=2pt] 
			(1*\kc,0.25) \thoi
			-- (2*\kc,0) \thoi
			-- (3*\kc,0.25) \thoi
			-- (4*\kc,0.5) \thoi
			-- (5*\kc,0.75) \thoi
			-- (6*\kc,0.5) \thoi
			-- (7*\kc,1) \thoi
			-- (8*\kc,1.25) \thoi
			-- (9*\kc,1.5) \thoi
			-- (10*\kc,1.25) \thoi
			-- (11*\kc,1) \thoi
			-- (12*\kc,1.5) \thoi
			;
			%---
			\foreach \ii in {1,...,12} \draw[cyan] (0.5*\ii,2pt)--++(0,-4pt)node[below,black]{\ii};
			\draw (3,-0.5) node[below,blue]{\bf Tháng};
			\draw (-0.75,2.5) node[rotate=90,above,blue]{\bf Số lượng (nghìn)};
			\draw[cyan] (-1.5,-1.25) rectangle (14*\kc-0.25,5.5);
			\draw (0.5*13*\kc,-1.25) node[below=1mm]{\bf b)};
		\end{tikzpicture}
	\end{center}
	\begin{enumerate}
		\item Lập bảng thống kê cho dữ liệu được biểu diễn trong mỗi biểu đồ.
		\item Dữ liệu biểu diễn trên hai biểu đồ có như nhau không? Giải thích tại sao hình dạng hai đường gấp khúc trên hai biểu đồ lại khác nhau.
	\end{enumerate}
	\loigiai{
		\begin{enumerate}
			\item Bảng thống kê cho dữ liệu được biểu diễn trong mỗi biểu đồ
			\begin{center}
				{\renewcommand\arraystretch{1.25}\tabcolsep=5mm 
					\begin{tabular}{|c|c|c|c|c|c|c|c|c|c|c|c|c|}
						\hline 
						Tháng &$1$&$2$&$3$&$4$&$5$&$6$&$7$&$8$&$9$&$10$&$11$&$12$\\
						\hline 
						Số lượng &$3$&$2$&$3$&$4$&$5$&$4$&$6$&$7$&$8$&$7$&$6$&$8$ \\
						\hline
				\end{tabular}}
			\end{center}
			\item Dữ liệu biểu diễn trên hai biểu đồ là như nhau. Hình dạng hai đường gấp khúc trên hai biểu đồ khác nhau là do gốc trục đứng khác nhau, tỉ lệ trên trục ngang của $2$ biểu đồ cũng khác nhau.
		\end{enumerate}
	}
\end{bt}
%%==========Bài 2
\begin{bt}%[8C6H1-2]
	Khối $8$ tổ chức giải bóng đá với $5$ đội tham dự là các đội bóng của các lớp A, B, C, D, E. Trước khi giải đấu diễn ra, Bình muốn thực hiện khảo sát dự đoán của các bạn về đội bóng vô địch giải đấu.
	\begin{enumerate}
		\item Theo em Bình có thể thực hiện khảo sát theo những cách nào?
		\item Dữ liệu Bình thu được thuộc loại nào?
	\end{enumerate}
	\loigiai{
		\begin{enumerate}
			\item Bình có thể thực hiện khảo sát theo cách: lập bảng hỏi, phỏng vấn,
			\item Dữ liệu Bình thu được thuộc loại dữ liệu không là số, không thể sắp thứ tự.
		\end{enumerate}
	}
\end{bt}
%%==========Bài 3
\begin{bt}%[8C6V2-5]
	Khối $8$ tổ chức giải bóng đá với $5$ đội tham dự là các đội bóng của các lớp A, B, C, D, E. Trước khi giải đấu diễn ra, Bình muốn thực hiện khảo sát dự đoán của các bạn về đội bóng vô địch giải đấu. Giả sử Bình thu được kết quả như sau: \\
	\centerline{\bf A, B, A, A, A, A, A, B, D, B, A, A, B, D, D, A, A, B, D.}\\
	Lập bảng thống kê về số lượng dự đoán vô địch cho mỗi đội.
	\begin{enumerate}
		\item Có thể dùng biểu đồ nào để biểu diễn dữ liệu trong bảng thống kê thu được.
		\item Nếu muốn biểu diễn tỉ lệ các bạn được hỏi dự đoán mỗi đội vô địch thì nên dùng biểu đổ nào?
	\end{enumerate}
	\loigiai{
		Bảng thống kê
		\begin{center}
			{\renewcommand\arraystretch{1.25}\tabcolsep=5mm 
				\begin{tabular}{|c|c|c|c|c|}
					\hline 
					Đội bóng & A & B & C & D \\
					\hline 
					Số lượng dự đoán &$10$&$5$&$0$&$4$ \\
					\hline
			\end{tabular}}
			\begin{enumerate}
				\item Có thể dùng biểu đồ tranh, biểu đồ cột, biểu đồ đoạn thẳng, biểu đồ hình quạt tròn để biểu diễn dữ liệu trong bảng thống kê thu được ở trên.
				\item Nếu muốn biểu diễn tỉ lệ các bạn được hỏi dự đoán mỗi đội vô địch thì nên dùng biểu đổ hình quạt tròn.
			\end{enumerate}
		\end{center}
	}
\end{bt}
%%==========Bài 4
\begin{bt}%[8C6V2-1]%[8T6V2-3]
	Bảng thống kê sau cho biết số lượng học sinh của các lớp khối $8$ tham gia các câu lạc bộ Thể thao và Nghệ thuật của trường.
	\begin{center}
		{\renewcommand\arraystretch{1.25}\tabcolsep=5mm
			\begin{tabular}{|c|c|c|c|c|}
				\hline & $8$A & $8$B & $8$C & $8$D \\
				\hline CLB thể thao & $8$ & $12$ & $10$ & $5$ \\
				\hline CLB Nghệ thuật & $16$ & $4$ & $8$ & $8$ \\
				\hline
		\end{tabular}}
	\end{center}
	\begin{enumerate}
		\item Lựa chọn và vẽ biểu đồ để so sánh số lượng học sinh tham gia hai câu lạc bộ này ở từng lớp.
		\item Lựa chọn và vẽ biểu đồ biểu diễn tỉ lệ học sinh các lớp tham gia hai câu lạc bộ trong số các học sinh khối $8$ tham gia hai câu lạc bộ này.
	\end{enumerate}
	\loigiai{
		\begin{enumerate}
			\item Chọn biểu đồ hình cột kép để so sánh số lượng học sinh tham gia hai câu lạc bộ này ở từng lớp.\\
			Biểu đồ
			\begin{center}
				\begin{tikzpicture}[>=stealth,line join=round,line cap=round,font=\footnotesize,y=0.7cm, x=1cm,scale=1,cyan]
					\foreach \ii in {0,...,8}{	
						\pgfmathsetmacro{\so}{int(\ii*10)} 
						\draw (-2pt,\ii) node[left]{$\so$}--++(13,0);
					}
					\draw (0,0)--(0,8.5);
					\fill (1,0) rectangle (2,4) node[black,above,xshift=-0.5cm]{$8$};
					\fill[red] (2,0) rectangle (3,8) node[black,above,xshift=-0.5cm]{$16$};
					%--
					\fill (4,0) rectangle (5,6) node[black,above,xshift=-0.5cm]{$12$};
					\fill[red] (5,0) rectangle (6,2)node[black,above,xshift=-0.5cm]{$4$};
					%--
					\fill (7,0) rectangle (8,5) node[black,above,xshift=-0.5cm]{$10$};
					\fill[red] (8,0) rectangle (9,4) node[black,above,xshift=-0.5cm]{$4$};
					%--
					\fill (10,0) rectangle (11,2.5) node[black,above,xshift=-0.5cm]{$2.5$};
					\fill[red] (11,0) rectangle (12,4) node[black,above,xshift=-0.5cm]{$4$};
					%--
					\foreach \ii/\nam in {2/8A,5/8B,8/8C,11/8D} \draw (\ii,2pt)--(\ii,-2pt)node[below]{\nam};
					\draw (6.5,-0.75) node[below,black]{\bf Lớp};
					\draw (-0.75,4) node[rotate=90,above,black]{\bf Số lượng};
					%---
					\def\a{0.4}
					\coordinate (goc) at (13.5,2);
					\fill[cyan] (goc) rectangle +(\a,\a) node[black,right=1mm, yshift=-0.5*\a cm]{CLB thể thao};
					\fill[red] ([yshift=0.8cm]goc) rectangle +(\a,\a) node[black,right=1mm, yshift=-0.5*\a cm]{CLB nghệ thuật};
				\end{tikzpicture}
			\end{center}
			\item Chọn biểu đồ hình quạt tròn để biểu diễn tỉ lệ học sinh các lớp tham gia hai câu lạc bộ trong số các học sinh khối 8 tham gia hai câu lạc bộ này. \\
			Biểu đồ
			\begin{center}
				\begin{tikzpicture}[>=stealth,scale=1, line join = round, line cap = round]
					\def\R{2}
					\draw[pattern = dots] (0,0) circle (\R cm);
					%--
					\def\mot{33.8}\def\hai{22.5}\def\ba{19.7}
					\pgfmathsetmacro{\bon}{\fpeval{round(100-\mot-\hai-\ba,2)}} 
					\draw[pattern = bricks] (0,0)--(\R,0) arc(0:\mot*3.6:\R)--cycle;
					%--
					\draw[pattern = grid] (0,0)--(\mot*3.6:\R) arc(\mot*3.6:(\mot+\hai)*3.6:\R)--cycle;
					%--
					\draw[fill=violet!50] (0,0)--(\mot*3.6+\hai*3.6:\R) arc(\mot*3.6+\hai*3.6:(\mot+\hai+\ba)*3.6:\R)--cycle;
					%--
					\draw (3.6*\mot*0.5:0.65*\R) node{$\mot\%$};
					\draw (3.6*\mot+3.6*\hai*0.5:0.65*\R) node{$\hai\%$};
					\draw (3.6*\mot+3.6*\hai+3.6*\ba*0.5:0.65*\R) node{$\ba\%$};
					\draw (-3.6*\bon*0.5:0.65*\R) node{$\bon\%$};
					%----------
					\def\a{0.4}
					\coordinate (goc) at (\R+0.5,-0.5*\a cm-11mm);
					\def\notemot{Lớp 8A}
					\def\notehai{Lớp 8B}
					\def\noteba{Lớp 8C}
					\def\notebon{Lớp 8D}
					\fill[pattern = bricks] (goc) rectangle +(\a,\a) node[black,right=1mm, yshift=-0.5*\a cm]{\notemot};
					\fill[pattern = grid] ([yshift=7mm]goc) rectangle +(\a,\a) node[black,right=1mm, yshift=-0.5*\a cm]{\notehai};
					\fill[violet!50] ([yshift=14mm]goc) rectangle +(\a,\a) node[black,right=1mm, yshift=-0.5*\a cm]{\noteba};
					\fill[pattern = dots] ([yshift=21mm]goc) rectangle +(\a,\a) node[black,right=1mm, yshift=-0.5*\a cm]{\notebon};
					%--tên bđ
					%	\draw (-\R, \R +1) node[black,right]{\bf\large Cấu trúc dân số Việt Nam năm 2010};
					%	\draw (-\R-0.25,\R+1.5) rectangle (3*\R,-\R-0.5);
				\end{tikzpicture}
			\end{center}
		\end{enumerate}
	}
\end{bt}
%%%%%%%%%%%%%%

%%==========Bài 5
\begin{bt}%[8C6H1-2]
	Em hãy đề xuất phương pháp thu thập dữ liệu cho các vấn đề sau:
	\begin{enumerate}
		\item Ý kiến của học sinh về $3$ mẫu logo của trường em.
		\item Tỉ số giữa số lần xuất hiện mặt có số chấm là số chẵn và số lần xuất hiện mặt có số chấm là số lẻ khi gieo một con xúc xắc $20$ lần.
		\item So sánh dân số ba nước Đông Dương.
		\item Lượng mưa trung bình $12$ tháng trong năm của một địa phương.
	\end{enumerate}	
	\loigiai{
		\begin{enumerate}
			\item Phương pháp thu thập dữ liệu về kiến của học sinh về $3$ mẫu logo của trường em là
			\begin{itemize}
				\item Phỏng vấn.
				\item Lập phiếu thăm dò.
			\end{itemize}
			\item Phương pháp thu thập dữ liệu về tỉ số giữa số lần xuất hiện mặt có số chấm là số chẵn và số lần xuất hiện mặt có số chấm là số lẻ khi gieo một con xúc xắc $20$ lần là \textbf{phương pháp làm thí nghiệm}.
			\item Phương pháp thu thập dữ liệu để so sánh dân số ba nước Đông Dương là \textbf{thu thập từ các nguồn có sẵn như sách, báo, internet}.
			\item Phương pháp thu thập dữ liệu lượng mưa trung bình $12$ tháng trong năm của một địa phương là \textbf{thu thập từ các nguồn có sẵn như sách, báo, internet}.
		\end{enumerate}	
	}
\end{bt}

%%==========Bài 6
\begin{bt}%[8T6N1-1]
	Bảng thống kê sau cho biêt sự lựa chọn của $100$ khách hàng mua điện thoại di động.
	\begin{center}
		\begin{tabular}{|c|c|}
			\hline Thương hiệu điện thoại di động & Số khách hàng chọn \\
			\hline N & $38$ \\
			\hline S & $35$ \\
			\hline H & $15$ \\
			\hline I & $12$ \\
			\hline
		\end{tabular}
	\end{center}
	Xét tính hợp lí của các quảng cáo sau đây đối với nhãn hiệu điện thoại $I$
	\begin{enumerate}
		\item Là sự lựa chọn của mọi người dùng điện thoại.
		\item Là sự lựa chọn hàng đầu của người dùng điên thoại.
	\end{enumerate}	
	\loigiai{
		Tính hợp lí của các quảng cáo sau đây đối với nhãn hiệu điện thoại $I$ {\bf Là sự lưa chọn hàng đầu của người dùng điên thoại}	
	}
\end{bt}

%%==========Bài 7
\begin{bt}%[8C6H1-2]
	Sau phỏng vấn thăm dò ý kiến của $100$ bạn học sinh khối $8$ về chủ trương \lq\lq Xin phép mặc đồng phục riêng của lớp khi đi cắm trại\rq\rq , bạn Thoa đã thu được bảng thống kê sau
	\begin{center}
		\begin{tabular}{|l|c|}
			\hline \multicolumn{1}{|c|}{ Ý kiến } & Số học sinh \\
			\hline Đồng ý & $33$ \\
			\hline Không đồng ý & $54$\\
			\hline Không có ý kiến & $13$ \\
			\hline
		\end{tabular}
	\end{center}
	Kết luận nào sau đây có thể đại diện hợp lí cho dữ liệu thống kê trên?
	\begin{enumerate}
		\item Đa số học sinh khối $8$ đồng ý.
		\item Đa số học sinh khối $8$ không đồng ý.
		\item Đa số học sinh khối $8$ không có ý kiến.
	\end{enumerate}
	\loigiai{
		Kết luận có thể đại diện hợp lí cho dữ liệu thống kê trên là \textbf{Đa số học sinh khối $8$ không đồng ý}.}
\end{bt}

%%==========Bài 8
\begin{bt}%[8C6H2-2]
	Thời gian tự học tại nhà của bạn Tú trong một tuần được biểu diễn trong biểu đồ cột sau đây. Em hãy vẽ biểu đồ đoạn thẳng tương ứng.
	\begin{center}
		\begin{tikzpicture}[scale=0.8, font=\footnotesize, line join=round, line cap=round, >=stealth]
			\node[above] at (5.5,7) {\textbf{Thời gian tự học tại nhà của Tú}};
			\draw[->] (0,0)--(12,0)node[below]{(Thứ)};
			\draw[->] (0,0)--(0,8)node[left]{(Số phút)};
			\foreach \x/\y in {1/20,2/40,3/60,4/80,5/100,6/120,7/140} \draw (0.1,\x)--(-0.1,\x) (0,\x) node[left] {$\y$};
			\foreach \x/\y in {1/3, 2.5/4, 4/5,5.5/6,7/4,8.5/4.25,10/2.25} {\draw[green!40!blue!60!white,fill=green!40!blue!60!white] (\x,0) rectangle (\x+0.5,\y);
				\pgfmathsetmacro\z{int(\y*20)} \node[above] at (\x+0.25,\y) {\z};}
			\node[below] at (1+0.25,0) {Thứ 2};
			\node[below] at (2.5+0.25,0) {Thứ 3};
			\node[below] at (4+0.25,0) {Thứ 4};
			\node[below] at (5.5+0.25,0) {Thứ 5};
			\node[below] at (7+0.25,0) {Thứ 6};
			\node[below] at (8.5+0.25,0) {Thứ 7};
			\node[below] at (10+0.35,0) {Chủ nhật};
		\end{tikzpicture}
	\end{center}
	\loigiai{
		\begin{center}
			\begin{tikzpicture}[scale=0.8, font=\footnotesize, line join=round, line cap=round, >=stealth]
				\node[above] at (5.5,7) {\textbf{Thời gian tự học tại nhà của Tú}};
				\draw[->] (0,0)--(12,0)node[below]{(Thứ)};
				\draw[->] (0,0)--(0,8)node[left]{(Số phút)};
				\foreach \x/\y in {1/20,2/40,3/60,4/80,5/100,6/120,7/140} \draw (0.1,\x)--(-0.1,\x) (0,\x) node[left] {$\y$};
				\foreach \x/\y in {1/3, 2.5/4, 4/5,5.5/6,7/4,8.5/4.25,10/2.25} {\draw[line width=2pt] (\x,0)--(\x,\y);
					\draw[dotted] (\x,\y)--(0,\y)
					;
					\pgfmathsetmacro\z{int(\y*20)} \node[above] at (\x,\y) {\z};}
				\node[below] at (1,0) {Thứ 2};
				\node[below] at (2.5,0) {Thứ 3};
				\node[below] at (4,0) {Thứ 4};
				\node[below] at (5.5,0) {Thứ 5};
				\node[below] at (7,0) {Thứ 6};
				\node[below] at (8.5,0) {Thứ 7};
				\node[below] at (10.5,0) {Chủ nhật};
			\end{tikzpicture}
		\end{center}
	}
\end{bt}

%%==========Bài 9
\begin{bt}%[8C6H1-3]
	Bảng số liệu sau cung cấp giá vé xe buýt giữa các địa điểm (đơn vị: đồng).
	\begin{center}
		\begin{tabular}{|c|c|c|c|c|c|}
			\hline Địa diểm & I & II & III & IV & V \\
			\hline I & $-$ & $10\;000$ & $5\;000$ & $15\;000$ & $10\;000$ \\
			\hline II & $10\;000$ & $-$ & $7\;000$ & $25\;000$ & $20\;000$ \\
			\hline III & $5\;000$ & $7\;000$ & $-$ & $20\;000$ & $15\;000$ \\
			\hline IV & $15\;000$ & $25\;000$ & $20\;000$ & $-$ & $10\;000$ \\
			\hline V & $10\;000$ & $20\;000$ & $15\;000$ & $10\;000$ & $-$ \\
			\hline
		\end{tabular}
	\end{center}
	Hãy phân tích dữ liệu từ bảng thống kê trên để trả lời các câu hỏi sau
	\begin{enumerate}
		\item Trong các tuyến đi từ địa điểm IV, tuyến nào có giá vé thấp nhất?
		\item Hành khách từ địa điểm II đi đến địa điểm nào có giá vé cao nhất?
	\end{enumerate}	
	\loigiai{
		\begin{enumerate}
			\item Trong các tuyến đi từ địa điểm IV, tuyến có giá vé thấp nhất là đi từ IV đến V.
			\item Hành khách từ địa điểm II đi đến điểm IV sẽ có giá vé cao nhất.
		\end{enumerate}
	}
\end{bt}

%%==========Bài 10
\begin{bt}%[8C6V3-3]
	Biểu đồ sau đây biểu diễn dữ liệu về hoạt động trong giờ ra chơi của học sinh lớp $8C$.	
	\begin{center}
		\begin{tikzpicture}[scale=1, font=\footnotesize, line join=round, line cap=round, >=stealth]
			\node[above] at (5.5,7) {\textbf{Hoạt động trong giờ ra chơi của học sinh lớp 8C}};
			\draw[->] (0,0)--(13,0)node[right]{Hoạt động};
			\draw[->] (0,0)--(0,7)node[above]{Số phút};
			\foreach \x/\y in {1/2,2/4,3/6,4/8,5/10,6/12} \draw (0.1,\x)--(-0.1,\x) (0,\x) node[left] {$\y$};
			\foreach \x/\y in {1/5, 3/5, 5/2.5, 7/2, 9/4, 11/2.5} {\draw[green!40!blue!60!white,fill=green!40!blue!60!white] (\x,0) rectangle (\x+0.5,\y);
				\pgfmathsetmacro\z{int(\y*2)} \node[above] at (\x+0.25,\y) {\z};}
			\node[below] at (1+0.25,0) {Đọc sách};
			\node[below] at (3+0.25,0) {Ôn bài};
			\node[below] at (5+0.25,0) {Đánh cầu};
			\node[below] at (7+0.25,0) {Đá cầu};
			\node[below] at (9+0.25,0) {Chơi cờ vua};
			\node[below] at (11+0.25,0) {Nhảy dây};
		\end{tikzpicture}
	\end{center}
	\begin{enumerate}
		\item Hãy phân tích dữ liệu từ biểu đồ trên để so sánh số học sinh tham gia hoạt động tại chỗ (đọc sách, ôn bài, chơi cờ vua) và hoạt động vận động (đánh cầu lông, đá cầu, nhảy dây) trong giờ ra chơi.
		\item Theo em các bạn lớp $8C$ nên tăng cường loại hoạt động nào để có lợi cho sức khoẻ?
	\end{enumerate}
	\loigiai{
		\begin{enumerate}
			\item Số học sinh tham gia các hoạt động tại chỗ (đọc sách, ôn bài, chơi cờ vua) nhiều hơn các bạn tham giác các hoạt động vận động (đánh cầu lông, đá cầu, nhảy dây) trong giờ ra chơi.
			\item Theo em các bạn lớp $8C$ nên tăng cường loại hoạt động vận động (đánh cầu lông, đá cầu, nhảy dây) trong giờ ra chơi để có lợi cho sức khoẻ.
		\end{enumerate}	
		\loigiai{
		}	
	}
\end{bt}

%%==========Bài 11
\begin{bt}%[8C6V2-1]
	Giá trị (triệu USD) xuất khẩu cà phê và gạo của Việt Nam trong các năm $2015$; $2018$; $2019$; $2020$ được cho trong bảng thống kê sau
	\begin{center}
		\begin{tabular}{|l|c|c|c|c|}
			\hline \multicolumn{1}{|c|}{ Năm } & $2015$ & $2018$ & $2019$ & $2020$ \\
			\hline Cà phê & $2\;671$ & $3\;536{,}4$ & $2\;863{,}8$ & $2\;742$ \\
			\hline Gạo & $2\;796{,}3$ & $3\;060{,}2$ & $2\;806{,}4$ & $3\;120$ \\
			\hline
		\end{tabular}
	\end{center}
	\hfill(Nguồn: Tổng cục Thống kê)
	\begin{enumerate}
		\item Lựa chọn dạng biểu đồ thích hợp để biểu diễn bảng thống kê trên.
		\item Tìm các năm giá trị xuất khẩu cà phê vượt giá trị xuất khẩu gạo.
	\end{enumerate}	
	\loigiai{
		\begin{enumerate}
			\item Dùng biểu đồ cột để biểu diễn
			\begin{center}
				\begin{tikzpicture}[scale=0.85, font=\footnotesize, line join=round, line cap=round, >=stealth]
					\node[above] at (5,7) {\textbf{Giá trị xuất khẩu gạo và cà phê của Việt Nam qua các năm}};
					\draw[blue!60!green,fill =blue!60!green] (8.5,6.5) rectangle (9,6) (9,6.25)node[right] {Cà phê};
					\draw[white!60!red,fill =white!60!red] (8.5,5.5) rectangle (9,5) (9,5.25)node[right] {Gạo};
					\draw[->] (0,0)--(10,0)node[below]{Năm};
					\draw[->] (0,0)--(0,7)node[left]{Giá trị (triệu USD)};
					\draw[blue!60!green,fill = blue!60!green] (1,0) rectangle (1.5,3.8);
					\node[right,rotate = 90] at (1.25,3.8) {$2671$};
					\draw[white!60!red, fill = white!60!red] (1.5,0) rectangle (2,4.1);
					\node[right,rotate = 90] at (1.75,4.1) {$2796{,}3$};
					\draw[blue!60!green,fill = blue!60!green] (3,0) rectangle (3.5,5.4);
					\node[right,rotate = 90] at (3.25,5.4) {$3536{,}4$};
					\draw[white!60!red, fill = white!60!red] (3.5,0) rectangle (4,4.4);
					\node[right,rotate = 90] at (3.75,4.4) {$3060{,}2$};
					\draw[blue!60!green,fill = blue!60!green] (5,0) rectangle (5.5,4.3);
					\node[right,rotate = 90] at (5.25,4.3) {$2863{,}8$};
					\draw[white!60!red, fill = white!60!red] (5.5,0) rectangle (6,4.1);
					\node[right,rotate = 90] at (5.75,4.1) {$2806{,}4$};
					\draw[blue!60!green,fill = blue!60!green] (7,0) rectangle (7.5,3.9);
					\node[right,rotate = 90] at (7.25,3.9) {$2742$};
					\draw[white!60!red, fill = white!60!red] (7.5,0) rectangle (8,4.6);
					\node[right,rotate = 90] at (7.75,4.6) {$3120$};
					\node[below] at (1.5,0) {$2015$};
					\node[below] at (3.5,0) {$2018$};
					\node[below] at (5.5,0) {$2019$};
					\node[below] at (7.5,0) {$2020$};
				\end{tikzpicture}
			\end{center}
			\item Các năm $2018$ và $2019$ giá trị xuất khẩu cà phê vượt giá trị xuất khẩu gạo.
		\end{enumerate}
	}
\end{bt}

%%==========Bài 12
\begin{bt}
	Quan sát biểu đồ đoạn thẳng dưới đây.
	\begin{center}
		\begin{tikzpicture}[>=stealth,line join=round,line cap=round,font=\footnotesize,scale=0.85]
			\def\y{6}
			\def\x{15}
			\def\a{\y/6}%Đơn vị trục tung
			\def\b{\x/10}%Đơn vị trục hoành
			\draw[<->] (0,\y)--(0,0)--(\x,0);%Vẽ trục
			%Node các đơn vị của trục hoành
			%\foreach \i in{2,3,...,7} \path ({\b*\i},0) node[below]{$\i$};
			\foreach \i/\j in{2/1959,3/1969,4/1979,5/1989,6/1999,7/2009,8/2019} \path ({\b*\i},0) node[below]{$\j$};
			%Node các đơn vị của trục tung
			\foreach \i/\j in{0/0,1/2,2/4,3/6,4/8,5/10} \path (0,{\a*\i}) node[left]{$\j$};
			%Định nghĩa các điểm trên biểu đồ 
			\path
			({\b*2},{\a*(2.98/2)}) coordinate (A2)node[above]{$2{,}98$}
			({\b*3},{\a*(3.63/2)}) coordinate (A3)node[above]{$3{,}63$}
			({\b*4},{\a*(4.38/2)}) coordinate (A4)node[above]{$4{,}38$}
			({\b*5},{\a*(5.24/2)}) coordinate (A5)node[above]{$5{,}24$}
			({\b*6},{\a*(6/2)}) coordinate (A6)node[above]{$6$}
			({\b*7},{\a*(6.87/2)}) coordinate (A7)node[above]{$6{,}87$}
			({\b*8},{\a*(7.71/2)}) coordinate (A8)node[above]{$7{,}71$}
			(\x,0) node[below]{\scriptsize Năm}
			(-0.7,\y) node[above]{\scriptsize Tỉ người};
			\foreach \i in {1,2,...,5} \draw[dashed,gray] (0,{\a*\i})--({\x},{\a*\i});
			\draw[thick] (A2)--(A3)--(A4)--(A5)--(A6)--(A7)--(A8);
			\foreach \i in {2,3,...,8} \fill[black] (A\i) circle (1.5pt);
		\end{tikzpicture}
	\end{center}
	\begin{enumerate}
		\item Từ biểu đồ trên, lập bảng số liệu dân số thế giới theo mẫu sau
		\begin{center}
			\begin{tabular}{|c|c|c|c|c|c|c|c|}
				\hline Năm & $1959$ & $1969$ & $1979$ & $1989$ & $1999$ & $2009$ & $2019$ \\
				\hline Dân số (tỉ người) & $?$ & $?$ & $?$ & $?$ & $?$ & $?$ & $?$ \\
				\hline
			\end{tabular}
		\end{center}
		\item Tính dân số thế giới tăng lên trong mỗi thập kỉ $1960-1969$; $1970-1979$; $\ldots$ ;$2010-2019$.
		\item Trong các thập kỉ trên, thập kỉ nào có dân số thế giới tăng nhiều nhất, ít nhất?
	\end{enumerate}	
	\loigiai{
		\begin{enumerate}
			\item Ta có bảng số liệu
			\begin{center}
				\begin{tabular}{|c|c|c|c|c|c|c|c|}
					\hline Năm & $1959$ & $1969$ & $1979$ & $1989$ & $1999$ & $2009$ & $2019$ \\
					\hline Dân số (tỉ người) & $2{,}98$ & $3{,}63$ & $4{,}38$ & $5{,}24$ & $6$ & $6{,}87$ & $7{,}71$ \\
					\hline
				\end{tabular}
			\end{center}
			\item Dân số thế giới tăng lên trong thập kỉ $1960-1969$ là $3{,}63-2{,}98=0{,}65$ (tỉ người).\\
			Dân số thế giới tăng lên trong thập kỉ $1970-1979$ là $4{,}38 - 3{,}63 = 0{,}75$ (tỉ người).\\
			Dân số thế giới tăng lên trong thập kỉ $1980-1989$ là $5{,}24 - 4{,}38 = 0{,}86$ (tỉ người).\\
			Dân số thế giới tăng lên trong thập kỉ $1990-1999$ là $6 - 5{,}24 = 0{,}76$ (tỉ người).\\
			Dân số thế giới tăng lên trong thập kỉ $2000-2009$ là $6{,}87 - 6 = 0{,}87$(tỉ người).\\
			Dân số thế giới tăng lên trong thập kỉ $2010-2019$ là $7{,}71 - 6{,}87 = 0{,}84$ (tỉ người).
			\item Trong các thập kỉ trên, thập kỉ có dân số thế giới tăng nhiều nhất là thập kỉ $2010-2019$.\\
			Trong các thập kỉ trên, thập kỉ có dân số thế giới tăng ít nhất là thập kỉ $1960-1969$.
		\end{enumerate}	
	}
\end{bt}

\begin{bt}%[HK1 THCS Ngô Tất Tố, 2023-2024]%[Nguyễn Quang Hiệp, 8HK1-2023-2024]%%[8D5N2-3]
	\immini{
		Biểu đồ hình quạt tròn biểu diễn tỉ lệ mỗi loại trái cây bán được của một siêu thị. Biết rằng siêu thị đã bán được tổng cộng $400$ kg các loại trái cây, gồm: ổi, sầu riêng, cam, xoài, mít. Tính số ki-lô-gam sầu riêng siêu thị đã bán được?
	}{
		\begin{tikzpicture}[>=stealth,line join=round,line cap=round,font=\scriptsize,scale=1]
			\def\r{2}
			\def\gocxp{90}
			\coordinate (A) at (90:\r);
			\foreach \val/\freq/\col/\pattern[count=\i from 0] in{
				Cam/18.0/red/vertical lines,
				Xoài/24/blue/north east lines,
				Mita/26/magenta/checkerboard,
				Ổi/12/teal/crosshatch,
				Sầu riêng/20/teal/dots}{
				\pgfmathsetmacro\gockt{-(\freq*3.6-\gocxp)}
				\pgfmathsetmacro\gocnode{\gocxp+\gockt}
				\draw[gray!50,pattern = \pattern,pattern color=\col] (0,0)--(A) arc(\gocxp:\gockt:\r) coordinate(A)--cycle;
				\fill[pattern = \pattern,pattern color=\col] (\r+1,\r-.75*\i) --++(0:.5)--++(-90:.5) node[pos=.5,right,black]{\val}--++(180:.5)--cycle;
				\path ($(0,0)+(\gocnode/2:1.5)$) node[fill=white,inner sep=0pt,circle]{\color{black} $\freq\%$};
				\global\let\gocxp=\gockt		
			}
			\path (current bounding box.north) node[above=.2]{\parbox{5cm}{\centering\bfseries Tỉ lệ phần trăm các loại trái cây bán được của cửa hàng}};
		\end{tikzpicture}	
	}
	\loigiai{
		Từ biểu đồ, ta có số ki-lô-gam sầu riêng bán được là $20\%\cdot 400=80$ kg.
	}
\end{bt}
\Closesolutionfile{ans}
\indapan{6}{ans/ans-8C4-OTC}