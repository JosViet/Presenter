\section{TỨ GIÁC}
\subsubsection{Kiến thức trọng tâm}
\begin{tomtat}
\begin{itemize}
	\item Tứ giác $ABCD$ là hình gồm $4$ đoạn thẳng $AB$, $BC$, $CD$ và $DA$ trong đó bất kì hai đoạn thẳng nào cũng không cùng nằm trên một đường thẳng.\\
	Chẳng hạn,
	\begin{center}
		\begin{tikzpicture}[scale=1, >=stealth, font=\footnotesize, line join=round, line cap=round]
			\coordinate (B) at (1,1);
			\coordinate (C) at (3,0);
			\coordinate (A) at (0,0);
			\coordinate (D) at (1,-1);
			\draw (A)--(B)--(C)--(D)--(A);
			\foreach \x/\g in {A/180,B/90,C/0,D/-90}\fill(\x) circle (1pt) +(\g:3mm) node {$\x$};
			\draw (1,-1.7)node[below]{a)};
		\end{tikzpicture}
		\begin{tikzpicture}[scale=1, >=stealth, font=\footnotesize, line join=round, line cap=round]
			\coordinate (A) at (0,0);
			\coordinate (B) at (1,-1);
			\coordinate (C) at (-1.5,0);
			\coordinate (D) at (-2,-1.5);
			\draw (A)--(B)--(C)--(D)--(A);
			\foreach \x/\g in {A/90,B/-90,C/90,D/-90}\fill(\x) circle (1pt) +(\g:3mm) node {$\x$};
			\draw (-1,-2)node[below]{b)};
		\end{tikzpicture}
		\begin{tikzpicture}[scale=1, >=stealth, font=\footnotesize, line join=round, line cap=round]
			\coordinate (A) at (0,0);
			\coordinate (B) at (-1,-1.5);
			\coordinate (C) at (0,-1.5);
			\coordinate (D) at (1.5,-1.5);
			\draw (A)--(B)--(C)--(D)--(A);
			\foreach \x/\g in {A/90,B/-90,C/-90,D/-90}\fill(\x) circle (1pt) +(\g:3mm) node {$\x$};
			\draw (0,-2)node[below]{c)};
		\end{tikzpicture}
		\begin{tikzpicture}[scale=1, >=stealth, font=\footnotesize, line join=round, line cap=round]
			\coordinate (A) at (0,0);
			\coordinate (B) at (1,-1.5);
			\coordinate (C) at (0,-1);
			\coordinate (D) at (-1.5,-1.5);
			\draw (A)--(B)--(C)--(D)--(A);
			\foreach \x/\g in {A/90,B/-90,C/-90,D/-90}\fill(\x) circle (1pt) +(\g:3mm) node {$\x$};
			\draw (0,-2)node[below]{d)};
		\end{tikzpicture}
	\end{center}
	Trong hình trên, hình a), b), d) là các tứ giác.
	\item \textbf{\textit{Tứ giác lồi}} là tứ giác luôn nằm trong cùng một phần mặt phẳng được phân chia bởi đường thẳng chứa bất kì cạnh nào của tứ giác.
	\begin{center}
		\begin{tikzpicture}[>=stealth,line join=round,line cap=round,font=\footnotesize,scale=1]
			\tikzset{declare function={a=2;b=1.6;c=2.4;d=2.1;goc1=-10;goc2=-130;goc3=160;}}
			\draw 
			(0,0)coordinate (A)--++(goc1:a)coordinate (D)--++(goc2:d)coordinate (C)--++(goc3:c)coordinate (B)--cycle		
			;
			\foreach \p/\pp in {A/D,D/A,A/B,B/A,D/C,C/D,B/C,C/B}{
				\draw[dashed] (\p)--($(\pp)!1.5!(\p)$);
			}
			\foreach \point/\goc in {A/90,D/-90,B/-90,C/-90}{
				\draw[fill=black](\point)circle(.8pt)+(\goc:2mm)node[scale=.8]{$\point$};
			}
		\end{tikzpicture}
	\end{center}
\end{itemize} 
\begin{luuy}
	Trong chương trình, khi nói đến tứ giác mà không chú thích gì thêm thì hiểu đó là tứ giác lồi.
\end{luuy}
\begin{itemize}
\item Trong một tứ giác:
\begin{itemize}
	\item Hai cạnh kề nhau là hai cạnh có chung một đỉnh.
	\item Hai cạnh kề nhau tạo thành một góc của tứ giác.
	\item Hai cạnh đối nhau là hai cạnh không có chung đỉnh nào.
	\item Hai đỉnh đối nhau là hai đỉnh không cùng nằm trên một cạnh.
	\item Đường chéo là đoạn thẳng nối hai đỉnh đối nhau.	
\end{itemize}	
Chẳng hạn, tứ giác $ABCD$ có 
\begin{center}
	\begin{tikzpicture}[>=stealth,line join=round,line cap=round,font=\footnotesize,scale=.8]
		\tikzset{declare function={a=2;b=1.6;c=2.4;d=2.1;goc1=-10;goc2=-130;goc3=160;}}
		\draw 
		(0,0)coordinate (A)--++(goc1:a)coordinate (D)--++(goc2:d)coordinate (C)--++(goc3:c)coordinate (B)--cycle		
		;
		\foreach \point/\goc in {A/90,D/10,B/190,C/-90}{
			\draw[fill=black](\point)circle(.8pt)+(\goc:2mm)node[scale=1]{$\point$};
		}
	\end{tikzpicture}
\end{center}
\begin{itemize}
	\item Hai cạnh kề nhau là hai cạnh có chung một đỉnh: $AB$ và $BC$; $BC$ và $CD$; $CD$ và $AD$; $AD$ và $AB$.
	\item Hai cạnh kề nhau tạo thành một góc của tứ giác: $\widehat{DAB}, \widehat{ABC}, \widehat{BCD}, \widehat{CDA}$.
	\item Các cặp góc đối nhau: $\widehat{DAB}$ và $\widehat{BCD}, \widehat{ABC}$ và $\widehat{CDA}$.
	\item Hai cạnh đối nhau là hai cạnh không có điểm chung: $AB$ và $CD$; $BC$ và $AD$.
	\item Hai đỉnh đối nhau là hai đỉnh không nằm trên cùng một cạnh: $A$ và $C$; $B$ và $D$.
	\item Đường chéo là đoạn thẳng nối hai đỉnh đối nhau: $AC; BD$.
\end{itemize}
\item Tổng số đo các góc của một tứ giác bằng $360^{\circ}$.
\begin{center}
	$ABCD$ là tứ giác, suy ra $ \widehat{A}+\widehat{B}+\widehat{C}+\widehat{D}=360^{\circ}$. 
\end{center}						
\end{itemize}
\end{tomtat}

\begin{vd}%[Dự án EX-9-Đề Cương Toán 9]%[Đặng Thị Thu Thảo]%[8H5N2-1]
	Tìm tứ giác lồi trong các hình sau
	\begin{center}
		\begin{tikzpicture}[scale=1, >=stealth, font=\footnotesize, line join=round, line cap=round]
			\coordinate (B) at (1.5,-1);
			\coordinate (C) at (-0.5,-2);
			\coordinate (A) at (0,0);
			\coordinate (D) at (-1.5,-1);
			\draw[green] (A)--(B)--(C)--(D)--(A);
			\foreach \x/\g in {A/90,B/0,C/-90,D/180}\fill(\x) circle (1pt) +(\g:3mm) node {$\x$};
			\draw (0.5,-3)node[below]{a)};
		\end{tikzpicture}
		\begin{tikzpicture}[scale=1, >=stealth, font=\footnotesize, line join=round, line cap=round]
			\coordinate (H) at (0,-1.5);
			\coordinate (G) at (3,0.5);
			\coordinate (F) at (0.5,-0.5);
			\coordinate (E) at (-1,0);
			\draw[blue] (H)--(G)--(F)--(E)--(H);
			\foreach \x/\g in {H/-90,G/0,F/90,E/90}\fill(\x) circle (1pt) +(\g:3mm) node {$\x$};
			\draw (1,-3)node[below]{b)};
		\end{tikzpicture}
	\end{center}
	\loigiai{
		Kẻ các đường thẳng chứa các cạnh của tứ giác như trong hình bên dưới, ta thấy $ABCD$ là tứ giác lồi còn $EFGH$ không phải tứ giác lồi.	
		\begin{center}
			\begin{tikzpicture}[scale=1, >=stealth, font=\footnotesize, line join=round, line cap=round]
				\coordinate (B) at (1.5,-1);
				\coordinate (C) at (-0.5,-2);
				\coordinate (A) at (0,0);
				\coordinate (D) at (-1.5,-1);
				\draw[green] (A)--(B)--(C)--(D)--(A);
				\draw[dashed,red] ($(A)!-0.5!(B)$)--($(B)!-0.5!(A)$) ($(B)!-0.5!(C)$)--($(C)!-0.5!(B)$)
				($(C)!-0.5!(D)$)--($(D)!-0.5!(C)$)
				($(A)!-0.5!(D)$)--($(D)!-0.5!(A)$);
				\foreach \x/\g in {A/90,B/0,C/-90,D/180}\fill(\x) circle (1pt) +(\g:3mm) node {$\x$};
				\draw (0.5,-3)node[below]{a)};
			\end{tikzpicture}
			\begin{tikzpicture}[scale=1, >=stealth, font=\footnotesize, line join=round, line cap=round]
				\coordinate (H) at (0,-1.5);
				\coordinate (G) at (3,0.5);
				\coordinate (F) at (0.5,-0.5);
				\coordinate (E) at (-1,0);
				\draw[blue] (H)--(G)--(F)--(E)--(H);
				\draw[dashed,red] ($(E)!-0.5!(F)$)--($(F)!-0.9!(E)$); 
				\foreach \x/\g in {H/-90,G/0,F/90,E/90}\fill(\x) circle (1pt) +(\g:3mm) node {$\x$};
				\draw (1,-3)node[below]{b)};
			\end{tikzpicture}
		\end{center}
	}
\end{vd}

\begin{vd}%[Dự án EX-9-Đề Cương Toán 9]%[Đặng Thị Thu Thảo]%[8H5N2-1]
	Xác định các yếu tố của tứ giác (đỉnh, cặp góc đối, hai cạnh đối nhau, hai đỉnh đối nhau, đường chéo) của tứ giác sau
	\begin{center}
		\begin{tikzpicture}[scale=1.5, >=stealth, font=\footnotesize, line join=round, line cap=round]
		\coordinate (Q) at (1.5,-1);
		\coordinate (P) at (0.5,-2);
		\coordinate (M) at (0,0);
		\coordinate (N) at (-1.5,-0.5);
		\draw (M)--(N)--(P)--(Q)--(M);
		\foreach \x/\g in {M/90,N/180,P/-90,Q/0}\fill(\x) circle (1pt) +(\g:3mm) node {$\x$};
	\end{tikzpicture}
	\end{center}
\loigiai{
	Tứ giác $MNPQ$ có
	\begin{itemize}
		\item Đỉnh là $M$, $N$, $P$, $Q$.
		\item Cặp góc đối nhau là $\widehat{M}$ và  $\widehat{P}$; $\widehat{N}$ và $\widehat{Q}$.
		\item Hai cạnh đối nhau là cạnh $MN$ và cạnh $PQ$; cạnh $NP$ và cạnh $MQ$.
		\item Hai đỉnh đối nhau là đỉnh $M$ và đỉnh $P$; đỉnh $N$ và đỉnh $Q$.
		\item Hai đường chéo là $MP$ và $NQ$.
	\end{itemize}	
}
\end{vd}

\begin{vd}%[Dự án EX-9-Đề Cương Toán 9]%[Đặng Thị Thu Thảo]%[8H5N2-2]
	Tìm số đo $x$ ở mỗi tứ giác sau
	\begin{center}
		\begin{tikzpicture}[scale=1, font=\footnotesize, line join=round, line 
			cap=round, >=stealth]
			\def\a{6}
			
			\path (0,0) coordinate (M)  
			($(M)+(0:2)$)coordinate (Q)
			($(Q)+(-100:3.5)$)coordinate (x) 
			($(M)+(-120:1.3)$)coordinate (N)
			($(N)+(-50:3.5)$)coordinate (y) 
			;	
			\path (intersection of  Q--x and N--y) coordinate (P);
			;	
			\foreach \d/\g in {M/90,N/180,Q/45,P/-90}	
			\path[draw,fill=black] (\d) circle(1pt) + (\g:9pt) node {$\d$};
			\draw (M)--(N)--(P)--(Q)--cycle ;
			\draw (0,-3)node[below]{a)};
			\path pic["\scriptsize$120^\circ$", angle eccentricity=2,,angle radius=6pt]{angle= N--M--Q};
			\path pic["\scriptsize$80^\circ$", angle eccentricity=2,angle radius=6pt]{angle= M--Q--P};
			\path pic["\scriptsize$110^\circ$", angle eccentricity=2,angle radius=8pt]{angle= P--N--M};
			\path pic["\scriptsize$x$", angle eccentricity=2,angle radius=8pt]{angle= Q--P--N};
		\end{tikzpicture}
		\begin{tikzpicture}[scale=1, font=\footnotesize, line join=round, line 
			cap=round, >=stealth]
			\def\a{6}
			
			\path (0,0) coordinate (E)  
			($(E)+(0:3.5)$)coordinate (F) ($(F)+(-90:2)$)coordinate (G)
			($(E)+(-90:2)$)coordinate (H)
			;	
			\foreach \d/\g in {E/90,H/180,F/45,G/-90}	
			\path[draw,fill=black] (\d) circle(1pt) + (\g:9pt) node {$\d$};
			\draw (E)--(H)--(G)--(F)--cycle ;
			\draw (2,-3)node[below]{b)}
			(G)node[above left]{$x$};
			\foreach \i/\j/\k in {H/E/F}{
				\draw[black] ($(\j)!6pt!(\i)$)--($(\j)!6pt!(\k)-(\j)+(\j)!6pt!(\i)$)--($(\j)!6pt!(\k)$);}
			\foreach \i/\j/\k in {E/F/G}{
				\draw[black] ($(\j)!6pt!(\i)$)--($(\j)!6pt!(\k)-(\j)+(\j)!6pt!(\i)$)--($(\j)!6pt!(\k)$);}
			\foreach \i/\j/\k in {G/H/E}{
				\draw[black] ($(\j)!6pt!(\i)$)--($(\j)!6pt!(\k)-(\j)+(\j)!6pt!(\i)$)--($(\j)!6pt!(\k)$);}
		\end{tikzpicture}
		\begin{tikzpicture}[scale=1, font=\footnotesize, line join=round, line 
			cap=round, >=stealth]
			\def\a{6}
			\path (0,-3) coordinate (A)  
			($(A)+(0:3.5)$)coordinate (D) ($(D)+(90:2)$)coordinate (x)
			($(A)+(65:3)$)coordinate (B)
			($(B)+(-25:2)$)coordinate (y) 
			;	
			\path (intersection of  B--y and x--D) coordinate (C);
			\foreach \d/\g in {A/180,B/90,D/-90,C/90}	
			\path[draw,fill=black] (\d) circle(1pt) + (\g:9pt) node {$\d$};
			\draw (A)--(B)--(C)--(D)--cycle;
			\draw (2,-3)node[below]{c)}
			(A)node[above right]{$65^\circ$}
			(C)node[below left]{$x$};
			\foreach \i/\j/\k in {A/B/C}{
				\draw[black] ($(\j)!6pt!(\i)$)--($(\j)!6pt!(\k)-(\j)+(\j)!6pt!(\i)$)--($(\j)!6pt!(\k)$);}
			\foreach \i/\j/\k in {A/D/C}{
				\draw[black] ($(\j)!6pt!(\i)$)--($(\j)!6pt!(\k)-(\j)+(\j)!6pt!(\i)$)--($(\j)!6pt!(\k)$);}
		\end{tikzpicture}
	\end{center}
	
	\loigiai{
		Do tổng số đo bốn góc của một tứ giác bằng $360^\circ$ nên ta có:
		\begin{enumerate}
			\item Trong tứ giác $MNPQ$: $x=360^\circ-(120^\circ+110^\circ+80^\circ)$, suy ra $x=50^\circ$.
			\item Trong tứ giác $EFGH$: $x=360^\circ-(90^\circ+90^\circ+90^\circ)$, suy ra $x=90^\circ$.
			\item Trong tứ giác $ABCD$: $x=360^\circ-(90^\circ+65^\circ+90^\circ)$, suy ra $x=115^\circ$.
		\end{enumerate}
	}
\end{vd}

\begin{vd}%[Dự án EX-9-Đề Cương Toán 9]%[Đặng Thị Thu Thảo]%[8H5H2-2]
	Cho tứ giác $ABCD$ có $\widehat{A}=115^{\circ}$, $ \widehat{D}=65^{\circ}$ và góc ngoài tại đỉnh $B$ bằng $70^{\circ}$. Tính số đo góc $\widehat{C}$ của tứ giác. 
	\loigiai
	{
		Ta có 
		\allowdisplaybreaks
		\begin{eqnarray*}
			\widehat{A}+\widehat{B}+\widehat{C}+\widehat{D}&=&360^{\circ}\,\text{(tổng bốn góc một tứ giác)}\\
			115^{\circ}+40^{\circ}+\widehat{C}+65^{\circ}&=&360^{\circ}\\
			\widehat{C}&=&360^{\circ}-115^{\circ}-40^{\circ}-65^{\circ}\\
			\widehat{C}&=&140^{\circ}.
		\end{eqnarray*}
		Vậy $\widehat{C}=140^{\circ}$.
	}
\end{vd}

\subsubsection{Bài tập}

\begin{bt}%[Dự án EX-9-Đề Cương Toán 9]%[Đặng Thị Thu Thảo]%[8H5N2-1]
	Tìm tứ giác lồi trong các hình sau
	\begin{center}
		\begin{tabular}{cc}
			\begin{tikzpicture}[>=stealth,line join=round,line cap=round,font=\footnotesize,scale=1]
				\tikzset{declare function={a=2;goc1=-30;goc2=110;goc3=60;goc4=100;}}
				\path 
				(0,0)coordinate (A)+(goc1:a)coordinate (B)
				($(A)!{sin(goc2)*1.6}!{goc2}:(B)$)coordinate (D)
				($(D)!{sin(goc3)*.2}!{goc3}:(A)$)coordinate (x)
				($(B)!{sin(goc4)*1}!-{goc4}:(A)$)coordinate (y)
				(intersection of D--x and B--y)coordinate (C)
				;
				\foreach \pointo/\pointt in {A/B,A/D,D/C,C/B}{
					\draw[fill=black](\pointo)--(\pointt);
				}
				\foreach \point/\goc in {A/180,D/90,B/-90,C/0}{
					\draw[fill=black](\point)circle(.8pt)+(\goc:2mm)node[scale=.8]{$\point$};
				}
			\end{tikzpicture}&
			\begin{tikzpicture}[>=stealth,line join=round,line cap=round,font=\footnotesize,scale=1]
				\tikzset{declare function={a=2;goc1=10;goc2=120;goc3=40;goc4=20;}}
				\path 
				(0,0)coordinate (M)+(goc1:a)coordinate (N)
				($(N)!{sin(goc2)*2.2}!-{goc2}:(M)$)coordinate (P)
				($(P)!{sin(goc3)*.2}!{goc3}:(N)$)coordinate (x)
				($(M)!{sin(goc4)*1}!-{goc4}:(N)$)coordinate (y)
				(intersection of P--x and M--y)coordinate (Q)
				;
				\foreach \pointo/\pointt in {M/N,N/P,P/Q,Q/M}{
					\draw[fill=black](\pointo)--(\pointt);
				}
				\foreach \point/\goc in {M/180,N/120,P/90,Q/-90}{
					\draw[fill=black](\point)circle(.8pt)+(\goc:2mm)node[scale=.8]{$\point$};
				}
			\end{tikzpicture}\\
			a) & b) \\
		\end{tabular}
	\end{center}	
	\loigiai{
		\begin{center}
			\begin{tabular}{cc}
				\begin{tikzpicture}[>=stealth,line join=round,line cap=round,font=\footnotesize,scale=1]
					\tikzset{declare function={a=2;goc1=-30;goc2=110;goc3=60;goc4=100;}}
					\path 
					(0,0)coordinate (A)+(goc1:a)coordinate (B)
					($(A)!{sin(goc2)*1.6}!{goc2}:(B)$)coordinate (D)
					($(D)!{sin(goc3)*.2}!{goc3}:(A)$)coordinate (x)
					($(B)!{sin(goc4)*1}!-{goc4}:(A)$)coordinate (y)
					(intersection of D--x and B--y)coordinate (C)
					;
					\foreach \p/\pp in {A/B,B/C,C/D,D/A}{
						\draw[dashed]($(\pp)!1.5!(\p)$)--(\p)--($(\p)!1.5!(\pp)$);	
					} 
					
					\foreach \pointo/\pointt in {A/B,A/D,D/C,C/B}{
						\draw[fill=black](\pointo)--(\pointt);
					}
					\foreach \point/\goc in {A/180,D/90,B/-90,C/0}{
						\draw[fill=black](\point)circle(.8pt)+(\goc:2mm)node[scale=.8]{$\point$};
					}
				\end{tikzpicture}&
				\begin{tikzpicture}[>=stealth,line join=round,line cap=round,font=\footnotesize,scale=1]
					\tikzset{declare function={a=2;goc1=10;goc2=120;goc3=40;goc4=20;}}
					\path 
					(0,0)coordinate (M)+(goc1:a)coordinate (N)
					($(N)!{sin(goc2)*2.2}!-{goc2}:(M)$)coordinate (P)
					($(P)!{sin(goc3)*.2}!{goc3}:(N)$)coordinate (x)
					($(M)!{sin(goc4)*1}!-{goc4}:(N)$)coordinate (y)
					(intersection of P--x and M--y)coordinate (Q)
					;
					\path[draw,dashed]
					(M)--($(M)!2.5!(N)$)
					;
					\foreach \pointo/\pointt in {M/N,N/P,P/Q,Q/M}{
						\draw[fill=black](\pointo)--(\pointt);
					}
					\foreach \point/\goc in {M/180,N/120,P/90,Q/-90}{
						\draw[fill=black](\point)circle(.8pt)+(\goc:2mm)node[scale=.8]{$\point$};
					}
				\end{tikzpicture}\\
				a) & b) \\
			\end{tabular}
		\end{center}
		\begin{itemize}
			\item  Tứ giác $ABCD$ luôn nằm trong cùng một phần mặt phẳng được phân chia bởi đường thẳng chứa bất kì cạnh nào của tứ giác nên $ABCD$ là tứ giác lồi. 
			\item  Đường thẳng đi qua cạnh $NQ$ của tứ giác $MNPQ$ chia tứ giác thành hai phần nên $MNPQ$ không phải là tứ giác lồi.		
		\end{itemize}		
	}
\end{bt}

\immini{
	\begin{bt}%[Dự án EX-9-Đề Cương Toán 9]%[Đặng Thị Thu Thảo]%[8H5N2-1]
		Vẽ tứ giác $MNPQ$ và tìm
		\begin{itemize}
			\item Hai đỉnh đối nhau;
			\item Hai đường chéo;
			\item Hai cạnh đối nhau.
		\end{itemize}
		\loigiai{
			\begin{itemize}
				\item Hai đỉnh đối nhau là đỉnh $M$ và đỉnh $P$; đỉnh $N$ và đỉnh $Q$.
				\item Hai đường chéo là $MP$ và $NQ$.
				\item Hai cạnh đối nhau là cạnh $MN$ và cạnh $PQ$; cạnh $NP$ và cạnh $MQ$
			\end{itemize}	
		}
	\end{bt}
}{
	\begin{tikzpicture}[scale=1.5, >=stealth, font=\footnotesize, line join=round, line cap=round]
		\coordinate (Q) at (1.5,-1);
		\coordinate (P) at (0.5,-2);
		\coordinate (M) at (0,0);
		\coordinate (N) at (-1.5,-0.5);
		\draw (M)--(N)--(P)--(Q)--(M);
		\foreach \x/\g in {M/90,N/180,P/-90,Q/0}\fill(\x) circle (1pt) +(\g:3mm) node {$\x$};
	\end{tikzpicture}
}

\begin{bt}%[Dự án EX-9-Đề Cương Toán 9]%[Đặng Thị Thu Thảo]%[8H5H2-1]
	\immini{
		Trên bản đồ, tứ giác $BDNQ$ với các đỉnh là các thành phố Buôn Ma Thuột, Đạt Lạt, Nha Trang, Quy Nhơn.
		\begin{enumerate}
			\item Tìm các cạnh kề và cạnh đối của cạnh $BD$.
			\item Tìm các đường chéo của tứ giác.
		\end{enumerate}
	}{
		\includegraphics[width=0.3\linewidth]{images/8C3-B2-1};
	}
	\loigiai{
		\begin{enumerate}
			\item Xét tứ giác $BDNQ$ ta có
			\begin{itemize}
				\item Cạnh kề với cạnh $BD$ là $BQ$, $DN$;
				\item Cạnh đối với cạnh $BD$ là $QN$.
			\end{itemize}
			\item Đường chéo của tứ giác $BDNQ$ là $DQ$ và $BN$.
		\end{enumerate}
	}
\end{bt}

\begin{bt}%[Dự án EX-9-Đề Cương Toán 9]%[Đặng Thị Thu Thảo]%[8H5N2-2]
	Tìm $x$ trong mỗi tứ giác sau:
	\begin{center}
		\begin{tikzpicture}[scale=1, font=\footnotesize, line join=round, line 
			cap=round, >=stealth]
			\def\a{6}
			\path (0,0) coordinate (Q)  
			($(Q)+(10:3)$)coordinate (R)
			($(R)+(50:3.5)$)coordinate (x) 
			($(Q)+(80:3)$)coordinate (P)
			($(P)+(-20:3.5)$)coordinate (y) 
			;	
			\path (intersection of  R--x and P--y) coordinate (S);
			;	
			\foreach \d/\g in {Q/180,P/180,R/0,S/90}	
			\path[draw,fill=black] (\d) circle(1pt) + (\g:9pt) node {$\d$};
			\draw (P)--(S)--(R)--(Q)--cycle ;
			\draw (1,-1)node[below]{a)}
			(R)node[above left]{$2x$}
			(Q)node[above right]{$x$};
			\path pic["\scriptsize$70^\circ$", angle eccentricity=2,,angle radius=8pt]{angle= P--S--R};
			\path pic["\scriptsize$80^\circ$", angle eccentricity=2,angle radius=8pt]{angle= Q--P--S};
		\end{tikzpicture}
		\begin{tikzpicture}[scale=1, font=\footnotesize, line join=round, line 
			cap=round, >=stealth]
			\def\a{6}
			\path (0,0) coordinate (C)  
			($(C)+(-10:3)$)coordinate (B)
			($(B)+(75:3.5)$)coordinate (x) 
			($(C)+(90:2)$)coordinate (D)
			($(D)+(0:3.5)$)coordinate (y) 
			;	
			\path (intersection of  B--x and D--y) coordinate (A);
			;	
			\foreach \d/\g in {C/-90,D/180,B/-45,A/90}	
			\path[draw,fill=black] (\d) circle(1pt) + (\g:9pt) node {$\d$};
			\draw (C)--(D)--(A)--(B)--cycle ;
			\draw (1.5,-1)node[below]{b)}
			(A)node[below left]{$x$}
			(C)node[above right]{$100^\circ$}
			(B)node[above left]{$95^\circ$};
			\foreach \i/\j/\k in {A/D/C}{
				\draw[black] ($(\j)!6pt!(\i)$)--($(\j)!6pt!(\k)-(\j)+(\j)!6pt!(\i)$)--($(\j)!6pt!(\k)$);}
		\end{tikzpicture}
		\begin{tikzpicture}[scale=1, font=\footnotesize, line join=round, line 
			cap=round, >=stealth]
			\def\a{6}
			\path (0,0) coordinate (H)  
			($(H)+(0:3)$)coordinate (G)
			($(G)+(90:3.5)$)coordinate (x) 
			($(H)+(81:2)$)coordinate (E)
			($(E)+(0:3.5)$)coordinate (y) 
			;	
			\path (intersection of  G--x and E--y) coordinate (F);
			;	
			\foreach \d/\g in {H/180,E/180,G/45,F/60}	
			\path[draw,fill=black] (\d) circle(1pt) + (\g:9pt) node {$\d$};
			\draw (H)--(E)--(F)--(G)--cycle ;
			\draw (1.5,-1)node[below]{c)}
			(E)node[below right]{$99^\circ$}
			(H)node[above right]{$x$};
			\foreach \i/\j/\k in {E/F/G}{
				\draw[black] ($(\j)!6pt!(\i)$)--($(\j)!6pt!(\k)-(\j)+(\j)!6pt!(\i)$)--($(\j)!6pt!(\k)$);}
			\foreach \i/\j/\k in {F/G/H}{
				\draw[black] ($(\j)!6pt!(\i)$)--($(\j)!6pt!(\k)-(\j)+(\j)!6pt!(\i)$)--($(\j)!6pt!(\k)$);}
		\end{tikzpicture}
	\end{center}
	\loigiai{
		Do tổng số đo bốn góc của một tứ giác bằng $360^\circ$ nên ta có
		\begin{enumerate}
			\item Trong tứ giác $MNPQ$: $x+2x+80^\circ+70^\circ=360^\circ$, suy ra $3x+150=360^\circ$. Vậy $x=70^\circ$.
			\item Trong tứ giác $EFGH$: $x+100^\circ+95^\circ+90^\circ=360^\circ$, suy ra $x=75^\circ$. Vậy $x=75^\circ$.
			\item Trong tứ giác $ABCD$: $x+99^\circ+90^\circ+90^\circ=360^\circ$, suy ra $x=81^\circ$. Vậy $x=81^\circ$.
		\end{enumerate}
	}
\end{bt}

\begin{bt}%[Dự án EX-9-Đề Cương Toán 9]%[Đặng Thị Thu Thảo]%[8H5N2-2]
	Cho tứ giác $A B C D$ có các góc $\widehat{A}=30^{\circ}, \widehat{B}=120^{\circ}, \widehat{C}=80^{\circ}$. Tính số đo góc $\widehat{D}$.
	\loigiai{
		Ta có 
		\allowdisplaybreaks
		\begin{eqnarray*}
			\widehat{A}+\widehat{B}+\widehat{C}+\widehat{D}&=&360^{\circ}\,\text{(tổng bốn góc một tứ giác)}\\
			30^{\circ}+120^{\circ}+80^{\circ}+\widehat{D}&=&360^{\circ}\\
			\widehat{D}&=&360^{\circ}-30^{\circ}-120^{\circ}-80^{\circ}\\
			\widehat{D}&=&130^{\circ}.
		\end{eqnarray*}
		Vậy $\widehat{D}=130^{\circ}$.
	}
\end{bt}

\begin{bt}%[Dự án EX-9-Đề Cương Toán 9]%[Đặng Thị Thu Thảo]%[8H5V2-2]
	Tính góc chưa biết của tứ giác trong hình bên dưới.
	Biết rằng $\widehat{H}=\widehat{E}+10^\circ$.
	\begin{center}
		\begin{tikzpicture}[scale=1,line cap=round,line join=round,font=\footnotesize,>=stealth]
			\path (0,0) coordinate (G)++(0:5) coordinate (F)
			++(120:2) coordinate (E)
			(50:.1) coordinate (g)
			($(E)!.1mm!-120:(F)$) coordinate (e)
			(intersection of E--e and G--g) coordinate (H)
			;
			%			\foreach \a/\b/\c in {R/V/U}{
				%				\draw pic[draw,angle radius=2mm]{right angle=\a--\b--\c};
				%			}
			\draw (E)--(F)--(G)--(H)--cycle;
			\foreach \a/\b/\c in {G/15/$50^\circ$,F/150/$60^\circ$}{
				\fill (\a)node[scale=.8,shift={(\b:.6)}]{\c};
			}
			\foreach \a/\b in {H/-70,E/-130}{
				\fill (\a)node[scale=.8,shift={(\b:.4)}]{?};
			}
			\foreach \a/\b in {E/40,F/-40,G/200,H/120}{
				\fill[black] (\a) circle (.8pt)
				($(\a)+(\b:2mm)$)node[scale=.8]{$\a$};
			}
		\end{tikzpicture}
	\end{center}
	\loigiai{
		Ta có $\widehat{H}+\widehat{E}=360-\widehat{G}-\widehat{F}=250^\circ$. \\
		Mà $\widehat{H}=\widehat{E}+10^\circ$ nên ta có $2\widehat{E}+10^\circ=250^\circ$.\\
		 Suy ra $\widehat{E}=120^\circ$; $\widehat{H}=130^\circ$.
	}
\end{bt}

\begin{bt}%[Dự án EX-9-Đề Cương Toán 9]%[Đặng Thị Thu Thảo]%[8H5H2-2]
	Tính các số đo $x$, $y$, $z$ ở các hình sau.
	\begin{enumEX}{3}
		\item 
		\begin{tikzpicture}[line join = round, line cap = round,>=stealth,font=\footnotesize,scale=1]
			\path
			(0,0) coordinate (D)
			($(D)+(.7,0)$) node[above]{$50^\circ$}
			($(D)+(4,0)$) coordinate (C)
			($(C)+(-.5,0)$) node[above]{$80^\circ$}
			($(D)!.85!50:(C)$) coordinate (A)
			($(A)+(.3,0.1)$) node[]{$x$}
			($(C)!1!-80:(D)$) coordinate (b)
			($(A)!1!120:(D)$) coordinate (d)
			(intersection of C--b and A--d) coordinate (B)
			($(B)+(-.35,0)$) node[below]{$120^\circ$}
			;
			\draw[shorten <= 0cm,shorten >= -0.5cm] (D)--(A);
			\draw
			(D)--(C)--(B)--(A)
			;
			\foreach \x/\g in {D/180,C/0,A/180,B/0} \fill (\x) circle (1.5pt) ($(\x)+(\g:3mm)$) node{$\x$};
		\end{tikzpicture}
		\item 
		\begin{tikzpicture}[line join = round, line cap = round,>=stealth,font=\footnotesize,scale=1]
			\path
			(0,0) coordinate (K)
			($(K)+(.35,0)$) node[above]{$y$}
			($(K)+(3,0)$) coordinate (I)
			($(I)+(0,2)$) coordinate (h1)
			($(K)!1!65:(I)$) coordinate (G)
			($(G)!1!90:(K)$) coordinate (h2)
			(intersection of G--h2 and I--h1) coordinate (H)
			($(H)+(.15,.35)$) node[left=.05pt]{$65^\circ$}
			;
			\draw[shorten <= 0cm,shorten >= -0.75cm] (I)--(H);
			\draw
			(I)--(K)--(G)--(H)
			pic[draw,angle radius=.25cm]{right angle=H--I--K}
			pic[draw,angle radius=.25cm]{right angle=K--G--H}
			;
			\foreach \x/\g in {K/180,I/0,G/90,H/0} \fill (\x) circle (1.5pt) ($(\x)+(\g:3mm)$) node{$\x$};
		\end{tikzpicture}
		\item 
		\begin{tikzpicture}[line join = round, line cap = round,>=stealth,font=\footnotesize,scale=1]
			\path
			(0,0) coordinate (Q)
			($(Q)+(.25,.25)$) node{$z$}
			($(Q)+(0,3)$) coordinate (M)
			($(M)+(1,0)$) coordinate (N)
			($(N)+(.5,-.2)$) node{$60^\circ$}
			($(Q)+(3,-.5)$) coordinate (P)
			($(P)+(-.3,-.2)$) node{$130^\circ$}
			;
			\draw[shorten <= 0cm,shorten >= -0.8cm] (M)--(N);
			\draw[shorten <= 0cm,shorten >= -0.75cm] (N)--(P);
			\draw
			(P)--(Q)--(M)
			pic[draw,angle radius=.25cm]{right angle=Q--M--N}
			;
			\foreach \x/\g in {Q/180,P/0,M/90,N/90} \fill (\x) circle (1.5pt) ($(\x)+(\g:3mm)$) node{$\x$};
		\end{tikzpicture}
	\end{enumEX}
	\loigiai{
		\begin{enumerate}
			\item Trong tứ giác $ABCD$ có $\widehat{A} + \widehat{B} + \widehat{C} + \widehat{D}=360^{\circ}$.\\
			Do đó $120^{\circ} + 50^{\circ} + 80^{\circ} + \widehat{A}=360^{\circ}$ hay $250^{\circ} + \widehat{A}=360^{\circ}$.\\
			 Suy ra $\widehat{A}=110^{\circ}$.\\
			Vậy $x=180^\circ-110^\circ=70^\circ$.
			\item Ta có $\widehat{GHI}=180^\circ-65^\circ=115^\circ$.\\
			Trong tứ giác $GHIK$ có $\widehat{G} + \widehat{H} + \widehat{I} + \widehat{K}=360^{\circ}$.\\
			Do đó $90^{\circ} + 115^{\circ} + 90^{\circ} + \widehat{K}=360^{\circ}$ hay $295^{\circ} + y=360^{\circ}$.\\
			 Vậy $y=65^{\circ}$.
			\item Ta có $\widehat{QPN}=180^\circ-130^\circ=50^\circ$.\\
			Ta có $\widehat{PNM}=180^\circ-60^\circ=120^\circ$.\\
			Trong tứ giác $MNPQ$ có $\widehat{M} + \widehat{N} + \widehat{P} + \widehat{Q}=360^{\circ}$.\\
			Do đó $90^{\circ} + 120^{\circ} + 50^{\circ} + \widehat{Q}=360^{\circ}$ hay $260^{\circ} + z=360^{\circ}$.\\
			 Vậy $z=100^{\circ}$.
		\end{enumerate}
	}
\end{bt}
%--------------------------------------
% Ví dụ 3
\begin{bt}%[Dự án EX-9-Đề Cương Toán 9]%[Đặng Thị Thu Thảo]%[8H5H2-2]
	\immini{
		Tìm số đo $x$ trong hình bên.	
	}{
		\begin{tikzpicture}[>=stealth,line join=round,line cap=round,font=\footnotesize,scale=1]
			\tikzset{declare function={a=3;goc1=-90;goc2=90;goc3=-40;goc4=120;}}
			\path 
			(0,0)coordinate (A)+(goc1:a)coordinate (D) ($(D)!{sin(goc2)*1.5}!-goc2:(A)$)coordinate (C)
			($(C)!{sin(goc3)*1}!{goc3}:(D)$)coordinate (b)
			($(A)!{sin(goc4)*.4}!{goc4}:(D)$)coordinate (bb)
			(intersection of A--bb and C--b)coordinate (B)node[right=2mm,scale=.8]{$70^\circ$}
			($(A)!2.2!(B)$)coordinate (x)
			;
			\path[line join=miter,line cap=butt] 
			pic[draw,angle radius=2mm]{right angle=A--D--C}
			;
			\foreach \pointo/\pointt in {A/D,D/C,C/B,B/A,B/x}{
				\draw[fill=black](\pointo)--(\pointt);
			}	
			\foreach \point/\goc in {A/180,B/120,C/-90,D/-90}{
				\draw[fill=black](\point)circle(.6pt)+(\goc:2mm)node[scale=.8]{$\point$};
			}
			\foreach \point/\goc/\name in {A/-25/3x,C/160/x}{
				\draw[fill=black](\point)+(\goc:3 mm)node[scale=.8]{$\name$};
			}
		\end{tikzpicture}	
	}	
	\loigiai{
		Góc ngoài tại đỉnh $B$ có số đo bằng $70^{\circ}$ nên góc trong tại đỉnh $B$ có số đo bằng $110^{\circ}$.\\
		 Xét tứ giác $ABCD$, ta có
		$$
		\begin{aligned}
			&\widehat{A}+\widehat{B}+\widehat{C}+\widehat{D}=360^{\circ} \\
			&3 x+110^{\circ}+x+90^{\circ}=360^{\circ} \\
			&4 x  =160^{\circ} \\
			&x =40^{\circ} .
		\end{aligned}
		$$
		Vậy $x=40^{\circ}$.	
	}
\end{bt}
%--------------------------------------
% Ví dụ 4
\begin{vd}%[Dự án EX-9-Đề Cương Toán 9]%[Đặng Thị Thu Thảo]%[8H5H2-2]
	Cho tứ giác $ABCD$ có $\widehat{A}=30^{\circ}, \widehat{B}=90^{\circ}, \widehat{D}=100^{\circ}$. Tính số đo góc $C$ và góc ngoài tại đỉnh $C$ của tứ giác đó.
	\loigiai{
		\immini{
			Xét tứ giác $ABCD$ ta có
			\[\begin{aligned}
				&\widehat{A}+\widehat{B}+\widehat{C}+\widehat{D}=360^{\circ}\\
				&30^{\circ}+90^{\circ}+\widehat{C}+100^{\circ}=360^{\circ}\\
				&\widehat{C}=360^{\circ}-30^{\circ}-90^{\circ}-100^{\circ}\\
				&\widehat{C}=140^{\circ}.
			\end{aligned}\]
		}{
			\begin{tikzpicture}[>=stealth,line join=round,line cap=round,font=\footnotesize,scale=1]
				\tikzset{declare function={a=4;goc1=0;goc2=100;goc3=-140;goc4=30;}}
				\path 
				(0,0)coordinate (A)node[shift={(10:6mm)},scale=.8]{$30^\circ$}+(goc1:a)coordinate (D)node[shift={(150:3mm)},scale=.8]{$100^\circ$}
				($(D)!{sin(goc2)*.3}!-goc2:(A)$)coordinate (C)
				($(C)!{sin(goc3)*1}!{goc3}:(D)$)coordinate (b)
				($(A)!{sin(goc4)*.4}!{goc4}:(D)$)coordinate (bb)
				(intersection of A--bb and C--b)coordinate (B)
				($(D)!2.2!(C)$)coordinate (x)
				;
				\path[line join=miter,line cap=butt] 
				pic[draw,angle radius=2mm]{right angle=A--B--C}
				;
				\foreach \pointo/\pointt in {A/D,D/C,C/B,B/A,D/x}{
					\draw[fill=black](\pointo)--(\pointt);
				}	
				\foreach \point/\goc in {A/180,B/120,C/0,D/-90}{
					\draw[fill=black](\point)circle(.6pt)+(\goc:2mm)node[scale=.8]{$\point$};
				}
			\end{tikzpicture}
		}	
		Số đo góc ngoài tại đỉnh $C$ của tứ giác $ABCD$ là $180^{\circ}-140^{\circ}=40^{\circ}$.			
	}
\end{vd}

\begin{bt}%[Dự án EX-9-Đề Cương Toán 9]%[Đặng Thị Thu Thảo]%[8H5V2-2]
	Cho tứ giác $ABCD$ có $\widehat{B}+\widehat{C}=200^{\circ}$, $ \widehat{B}+\widehat{D}=180^{\circ}$, $ \widehat{C}+\widehat{D}=120^{\circ}$. Tính số đo các góc của tứ giác. 
	\loigiai{
		Ta có 
		\allowdisplaybreaks
		\begin{eqnarray*}
			\widehat{B}+\widehat{C}+\widehat{B}+\widehat{D}-\widehat{C}-\widehat{D}&=&200^{\circ}+180^{\circ}-120^{\circ}\\
			2\widehat{B}&=&260\\
			\widehat{B}&=&260:2=130^{\circ}.
		\end{eqnarray*}
		\noindent Khi đó 
		\begin{itemize}
			\item $\widehat{C}=200^{\circ}-130^{\circ}=70^{\circ}$.
			\item $\widehat{D}=180^{\circ}-130^{\circ}=50^{\circ}$.
			\item $\widehat{A}=360^{\circ}-\widehat{B}-\widehat{C}-\widehat{D}=360^{\circ}-130^{\circ}-70^{\circ}-50^{\circ}=110^{\circ}$.
		\end{itemize}
	}
\end{bt}

\begin{bt}%[Dự án EX-9-Đề Cương Toán 9]%[Đặng Thị Thu Thảo]%[8H5C2-2]
	\hfill
	\begin{enumerate}
		\item Cho tứ giác $ABCD$ có $AB\parallel CD$, $\widehat{B}=135^\circ$, $\widehat{D}=70^\circ$, $\widehat{ACB}=25^\circ$ (hình $a$). Tính số đo góc $DAC$.
		\item Cho tứ giác $GHIK$ có $\widehat{KGH}=\widehat{K}=90^\circ$, $\widehat{I}=65^\circ$. Trên $HI$ lấy điểm $E$ sao cho $\widehat{EGH}=25^\circ$ (hình $b$). Tính số đo góc $GEI$.
		\item Cho tứ giác $MNPQ$ có $PM$ là tia phân giác của góc $NPQ$, $\widehat{QMN}=110^\circ$, $\widehat{N}=120^\circ$, $\widehat{Q}=60^\circ$ (hình $c$). Tính số đo các góc $NPM$, $MPQ$, $QMP$.
	\end{enumerate}
	\begin{enumEX}{3}
		\item 
		\begin{tikzpicture}[line join = round, line cap = round,>=stealth,font=\footnotesize,scale=1]
			\path
			(0,0) coordinate (D)
			($(D)+(.45,.25)$) node{$70^\circ$}
			($(D)!.5!70:(C)$) coordinate (A)
			%($(A)+(-.4,-.25)$) node{$1$}
			($(D)+(4,0)$) coordinate (C)
			($(C)+(-.85,.75)$) node{$25^\circ$}
			($(C)!1!-25:(A)$) coordinate (b)
			($(A)+(2,0)$) coordinate (b1)
			(intersection of A--b1 and C--b) coordinate (B)
			($(B)+(-.25,-.25)$) node{$135^\circ$}
			
			;
			\draw (A)--(B)--(C)--(D)--cycle
			(A)--(C)
			;
			\foreach \x/\g in {A/180,B/0,D/-90,C/-90} \fill (\x) circle (1.5pt) ($(\x)+(\g:3mm)$) node{$\x$};
		\end{tikzpicture}
		\item 
		\begin{tikzpicture}[line join = round, line cap = round,>=stealth,font=\footnotesize,scale=1]
			\path
			(0,0) coordinate (K)
			($(K)+(0,3)$) coordinate (G)
			($(G)+(.8,-.15)$) node{$25^\circ$}
			($(K)+(3,0)$) coordinate (I)
			($(I)+(-.5,.25)$) node{$65^\circ$}
			($(G)+(1.5,0)$) coordinate (H)
			($(G)!1!-25:(H)$) coordinate (e)
			(intersection of G--e and H--I) coordinate (E)
			;
			\draw
			(K)--(G)--(H)--(I)--cycle (G)--(E)
			pic[draw,angle radius=.25cm]{right angle=I--K--G}
			pic[draw,angle radius=.25cm]{right angle=K--G--H}
			;
			\foreach \x/\g in {K/180,I/0,H/0,G/180,E/0} \fill (\x) circle (1.5pt) ($(\x)+(\g:3mm)$) node{$\x$};
		\end{tikzpicture}
		\item 
		\begin{tikzpicture}[line join = round, line cap = round,>=stealth,font=\footnotesize,scale=1]
			\path
			(0,0) coordinate (Q)
			($(Q)+(.5,.25)$) node{$60^\circ$}
			($(Q)+(4,0)$) coordinate (P)
			($(P)!.9!-35:(Q)$) coordinate (M)
			($(M)+(.25,-.35)$) node[rotate=50]{$110^\circ$}
			($(M)!1!110:(Q)$) coordinate (n)
			($(P)!1!-70:(Q)$) coordinate (n1)
			(intersection of M--n and P--n1) coordinate (N)
			($(N)+(-.35,-.2)$) node{$120^\circ$}
			;
			\draw
			(M)--(N)--(P)--(Q)--cycle (M)--(P)
			pic[draw,angle radius=.5 cm]{angle=M--P--Q}
			pic[draw,angle radius=.75 cm]{angle=N--P--M}
			;
			\foreach \x/\g in {Q/180,M/180,P/0,N/0} \fill (\x) circle (1.5pt) ($(\x)+(\g:3mm)$) node{$\x$};
		\end{tikzpicture}
	\end{enumEX}
	\loigiai{
		\begin{enumerate}
			\item Trong $\triangle ABC$ có $\widehat{BAC}+\widehat{ABC}+\widehat{ACB}=180^\circ$.\\
			Do đó $\widehat{BAC}+135^\circ+25^\circ=180^\circ$. Vậy $\widehat{BAC}=20^\circ$.\\
			Vì $AB\parallel DC$ nên $\widehat{BAC}=\widehat{ACD}=20^\circ$ (so le trong).\\
			Trong $\triangle ADC$ có $\widehat{ADC}+\widehat{DAC}+\widehat{ACD}=180^\circ$.\\
			Do đó $70^\circ+\widehat{DAC}+20^\circ=180^\circ$.\\
			 Vậy $\widehat{DAC}=90^\circ$.
			\item Trong tứ giác $GHIK$ có $\widehat{G} + \widehat{H} + \widehat{I} + \widehat{K}=360^{\circ}$.\\
			Do đó $90^{\circ} + 65^{\circ} + 90^{\circ} + \widehat{H}=360^{\circ}$ hay $245^{\circ} + \widehat{H}=360^{\circ}$.\\
			 Vậy $\widehat{H}=115^\circ$.\\
			Trong $\triangle GHE$ có $\widehat{HGE}+\widehat{GHE}+\widehat{HEG}=180^\circ$.\\
			Do đó $25^\circ+115^\circ+\widehat{HEG}=180^\circ$.\\
			 Vậy $\widehat{HEG}=40^\circ$.\\
			Ta có $\widehat{GEI}=180^\circ-\widehat{HEG}=180^\circ-40^\circ=140^\circ$.
			\item Trong tứ giác $MNPQ$ có $\widehat{M} + \widehat{N} + \widehat{P} + \widehat{Q}=360^{\circ}$.\\
			Do đó $110^{\circ} + 120^{\circ} + 60^{\circ} + \widehat{P}=360^{\circ}$ hay $290^{\circ} + \widehat{P}=360^{\circ}$.\\
			 Vậy $\widehat{P}=70^\circ$.\\
			Vì $PM$ là tia phân giác của góc $NPQ$ suy ra $\widehat{NPM}=\widehat{MPQ}=\widehat{NPQ}:2=70^\circ:2=35^\circ$.\\
			Trong $\triangle MPQ$ có $\widehat{QMP} + \widehat{MPQ} + \widehat{PQM}=180^{\circ}$.\\
			Do đó $60^{\circ} + 35^{\circ} + \widehat{PMQ}=180^{\circ}$ hay $95^{\circ} + \widehat{PMQ}=180^{\circ}$. \\
			Vậy $\widehat{PMQ}=85^\circ$.
		\end{enumerate}
	}
\end{bt}

 \begin{bt}%[Dự án EX-9-Đề Cương Toán 9]%[Đặng Thị Thu Thảo]%[8H5V2-3]
 	\immini{
 		Ta gọi tứ giác $ABCD$ có $AB=AD$, $CB=CD$ là hình \lq\lq cái diều\rq\rq.
 		\begin{enumerate}
 			\item Chứng minh rằng $AC$ là đường trung trực của $BD$.
 			\item Cho biết $\widehat{B}=95^\circ$, $\widehat{C}=35^\circ$. Tính $\widehat{A}$ và $\widehat{D}$.
 		\end{enumerate}
 	}{
 		\begin{tikzpicture}[scale=1, font=\footnotesize, line join=round, line 
 			cap=round, >=stealth]
 			\def\a{6}
 			\path (-6.5,-1) coordinate (C)  
 			($(C)+(35:3)$)coordinate (B)
 			($(B)+(-30:1.5)$)coordinate (x) 
 			($(C)+(0:3)$)coordinate (D)
 			($(D)+(67.5:1.5)$)coordinate (y) 
 			;	
 			\path (intersection of  B--x and D--y) coordinate (A);
 			;	
 			\foreach \d/\g in {C/90,D/0,B/180,A/90}	
 			\path[draw,fill=black] (\d) circle(1pt) + (\g:9pt) node {$\d$};
 			\draw (C)--(D)--(A)--(B)--cycle ;
 			%\draw (-3,-2)node[below]{Hình 13}
 			;
 			\path (B)--(C) node[midway,sloped]{$||$};
 			\path (D)--(C) node[midway,sloped]{$||$};
 			\path (B)--(A) node[midway,sloped]{$|$};
 			\path (D)--(A) node[midway,sloped]{$|$};
 			%\pgftext{\includegraphics[width=150pt]{images/tha-dieu-700.jpg}} at (0,0);
 		\end{tikzpicture}
 	}
 	\loigiai{
 		\begin{enumerate}
 			\item Xét $\triangle ABC$ và $\triangle ADC$ ta có
 			\begin{itemize}
 				\item $AB=AD$
 				\item $BC=BD$
 				\item $AC$ cạnh chung
 			\end{itemize}
 			Suy ra $\triangle ABC =\triangle ADC$ (c.c.c).\\
 			Do đó $\widehat{BAC}=\widehat{DAC}$ nên $AC$ là tia phân giác của $\widehat{BAD}$.\\
 			Xét $\triangle ABD$ cân tại $A$, có $AC$ là tia phân giác nên cũng là đường trung trực.\\
 			Vậy $AC$ là đường trung trực của $BD$.
 			\item Vì $\triangle ABC =\triangle ADC$ nên $\widehat{B}=\widehat{D}$.\\
 			Xét tứ giác $ABCD$ ta có
 			\allowdisplaybreaks
 			\begin{eqnarray*}
 				\widehat{B}+\widehat{D}+60^\circ+130^\circ&=&360^\circ\\
 				2\widehat{B}+190^\circ&=&360^\circ\\
 				\widehat{B}&=&85^\circ.
 			\end{eqnarray*}
 			Suy ra $\widehat{B}=\widehat{D}=85^\circ$.
 		\end{enumerate}
 	}
 \end{bt}

\begin{bt}%[Dự án EX-9-Đề Cương Toán 9]%[Đặng Thị Thu Thảo]%[8H5V2-3]
	\immini
	{
		Thả diều là một trò chơi dân gian của nhiều trẻ em ở Việt Nam cũng như nhiều nước trên thế giới. Một tứ giác $ABCD$ với $AB=AD$, $BC=CD$ gọi là hình \lq\lq Chiếc diều\rq\rq\ (hình bên).
		\begin{enumerate}
			\item So sánh $\widehat{B}$ và $\widehat{D}$.
			\item Tìm mối liên hệ giữa hai đường chéo $AC$ và $BD$.
		\end{enumerate}
	}
	{
		\begin{tikzpicture}[line join = round, line cap = round,>=stealth,font=\footnotesize,scale=1,|| mark/.style={postaction=decorate,decoration={markings,
					mark=at position #1 with {\draw[line cap=round,mark segment] (-1pt,-2pt) -- (-1pt,2pt);
						\draw[line cap=round,mark segment] (1pt,-2pt) -- (1pt,2pt);
			}}},mark segment/.style={thick}, | mark/.style={postaction=decorate,decoration={markings,
					mark=at position #1 with {\draw[line cap=round,mark segment] (-1pt,-2pt) -- (-1pt,2pt);
			}}},mark segment/.style={thick}]
			\path
			(0,0) coordinate (A)
			($(A)+(1,1.5)$) coordinate (B)
			($(A)+(1,-1.5)$) coordinate (D)
			($(A)+(4,0)$) coordinate (C)
			;
			\draw
			(A)--(B)--(C)--(D)--cycle (A)--(C) (B)--(D)
			;
			\path[|| mark=0.5] (D) -- (A);
			\path[|| mark=0.5] (A) -- (B);
			\path[| mark=0.5] (D) -- (C);
			\path[| mark=0.5] (C) -- (B);
			\foreach \x/\g in {A/180,B/90,C/0,D/-90} \fill (\x) circle (1.5pt) ($(\x)+(\g:3mm)$) node{$\x$};
		\end{tikzpicture}
	}
	\loigiai{
		\begin{enumerate}
			\item Xét $\triangle ABC$ và $\triangle ADC$ có\\
			$\heva{& AB=AD \mbox{ (gt)} \\ & CB=CD \mbox{ (gt)} \\& AC \mbox{ chung}}$\\
			Do đó $\triangle ABC=\triangle ADC$ (c.c.c)\\
			Suy ra $\widehat{ABC}=\widehat{ADC}$ (hai góc tương ứng).
			\item Ta có $\heva{& AB=AD \mbox{ (gt)} \\ & CB=CD \mbox{ (gt)}}$, suy ra $AC$ là đường trung trực của $BD$.\\
			Do đó $AC\perp BD$.
		\end{enumerate}
	}
\end{bt}

\begin{bt}%[Dự án EX-9-Đề Cương Toán 9]%[Đặng Thị Thu Thảo]%[8H5V2-2]
	\immini{
		Phần thân của cái diều được vẽ lại như hình bên. Tìm số đo các góc chưa biết trong hình.
	}{
		\begin{tikzpicture}[scale=1, font=\footnotesize, line join=round, line 
			cap=round, >=stealth]
			\def\a{6}
			\path (5,2) coordinate (C)  
			($(C)+(-155:2)$)coordinate (B)
			($(B)+(-60:3.5)$)coordinate (x) 
			($(C)+(-25:2)$)coordinate (D)
			($(D)+(-120:3.5)$)coordinate (y) 
			;	
			\path (intersection of  B--x and D--y) coordinate (A);
			;	
			\foreach \d/\g in {C/90,D/0,B/180,A/-90}	
			\path[draw,fill=black] (\d) circle(1pt) + (\g:9pt) node {$\d$};
			\draw (C)--(D)--(A)--(B)--cycle ;
			%\draw (5,-3)node[below]{b)}
			;
			\path pic["\scriptsize$130^\circ$", angle eccentricity=2,draw,angle radius=8pt]{angle= B--C--D};
			\path pic["\scriptsize$60^\circ$", angle eccentricity=2,draw,angle radius=8pt, double]{angle= D--A--B};
			\path (B)--(C) node[midway,sloped]{$|$};
			\path (D)--(C) node[midway,sloped]{$|$};
			\path (B)--(A) node[midway,sloped]{$||$};
			\path (D)--(A) node[midway,sloped]{$||$};
			%\pgftext{\includegraphics[width=150pt]{images/cai-dieu-hinh-thoi.jpg}} at (0,0);
			%\draw (0,-3)node[below]{a)}
			%;
			%\draw (2.5,-3.5)node[below]{Hình 10}
			%;
		\end{tikzpicture}
	}
	\loigiai{
		Xét $\triangle ABC$ và $\triangle ADC$ ta có
		\begin{itemize}
			\item $BC=DC$
			\item $AB=AD$
			\item $AC$ là cạnh chung
		\end{itemize}
		Suy ra $\triangle ABC =\triangle ADC$ (c.c.c).\\
		Do đó $\widehat{B}=\widehat{D}$.\\
		Xét tứ giác $ABCD$ ta có
		\[\widehat{B}+\widehat{D}+60^\circ+130^\circ=360^\circ \Rightarrow 2\widehat{B}+190^\circ=360^\circ\]
		Suy ra $\widehat{B}=\widehat{D}=85^\circ$.
	}
\end{bt}
