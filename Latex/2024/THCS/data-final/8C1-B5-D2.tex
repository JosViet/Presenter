%\section{CỘNG, TRỪ PHÂN THỨC ĐẠI SỐ} % Tên bài
\subsection{Cộng, trừ hai phân thức có cùng mẫu thức}
\subsubsection{Kiến thức trọng tâm}
\begin{tomtat}
	\begin{itemize}
		\item Muốn cộng (hoặc trừ) hai phân thức có cùng mẫu thức, ta cộng (hoặc trừ) các tử thức với nhau và giữ nguyên mẫu thức.
		$$\dfrac{A}{B}+\dfrac{C}{B}=\dfrac{A+C}{B}(B\ne0);\quad \dfrac{A}{B}-\dfrac{C}{B}=\dfrac{A-C}{B}(B\ne0).$$
		\item Phép cộng phân thức có các tính chất giao hoán, kết hợp tương tự như đối với phân số.
	\end{itemize}
\end{tomtat}

\begin{vd}%%[Dự án EX-9-Đề Cương Toán 9 - đợt 2]%[Lê Hòa Nam]%[8D1N6-1]
	Thực hiện các phép cộng, trừ phân thức sau:
	\begin{multicols}{3}
\begin{enumerate}
	\item $\dfrac{x+y}{xy}+\dfrac{x-y}{xy}$;
	\item $\dfrac{x^2+5x}{x+2}-\dfrac{x-4}{x+2}$;
	\item $\dfrac{3x+2y}{x^2-y^2}-\dfrac{x}{x^2-y^2}$.	
\end{enumerate}
\end{multicols}
	\loigiai{
	\begin{enumerate}
		\item  
		\begin{eqnarray*}
			\dfrac{x+y}{xy}+\dfrac{x-y}{xy}=\dfrac{x+y+x-y}{xy}=\dfrac{2x}{xy}=\dfrac{2}{y}.
		\end{eqnarray*}
		\item 
		\begin{eqnarray*}
		&&\dfrac{x^2+5x}{x+2}-\dfrac{x-4}{x+2}\\
		&=&\dfrac{x^2+5x-(x-4)}{x+2}\\
		&=&\dfrac{x^2+5x-x+4}{x+2}\\
		&=&\dfrac{x^2+4x+4}{x+2}=\dfrac{(x+2)^2}{x+2}\\
		&=& \dfrac{(x+2)(x+2)}{x+2}\\
		&=&x+2.
		\end{eqnarray*}
		 
		\item  \begin{eqnarray*}
			&&\dfrac{3x+2y}{x^2-y^2}-\dfrac{x}{x^2-y^2}\\
			&=&\dfrac{3x+2y-x}{x^2-y^2}=\dfrac{2x+2y}{(x+y)(x-y)}\\
			&=&\dfrac{2(x+y)}{(x+y)(x-y)}\\
			&=&\dfrac{2}{x-y}.
		\end{eqnarray*}
	\end{enumerate}
	}
\end{vd}

\subsection{Cộng, trừ hai phân thức khác mẫu}
\subsubsection{Kiến thức trọng tâm}
\begin{tomtat}
		\textbf{Nhận xét:}\\
	\textit{Quy đồng mẫu thức} hai phân thức là biến đổi hai phân thức đã cho thành hai phân thức mới có cùng mẫu thức và lần lượt bằng hai phân thức đã cho.\\
	Mẫu thức của các phân thức mới đó gọi là \textit{mẫu thức chung} của hai phân thức đã cho.\\
	Cho hai phân thức $\dfrac{A}{B}$ và $\dfrac{C}{D}$.
	\begin{itemize}
		\item 	 Ta có $\dfrac{A}{B}=\dfrac{A \cdot D}{B \cdot D}$ và $\dfrac{C}{D}=\dfrac{B \cdot C}{B \cdot D}$.\\
		Nghĩa là, ta luôn có thể quy đồng hai phân thức đã cho với mẫu thức chung là $B \cdot D$ (tích của hai mẫu thức).
		\item  Nếu $D$ là một nhân tử của $B$ $(B=D\cdot P$ với $P$ là một đa thức) thì lấy mẫu thức chung là $B$. Khi đó, ta quy đồng mẫu thức:
		$$\dfrac{C}{D}=\dfrac{C \cdot P}{D \cdot P}=\dfrac{C \cdot P}{B} \left(\text{giữ nguyên phân thức } \dfrac{A}{B} \right).$$
		(Tương tự cho trường hợp $B$ là một nhân tử của $D$.)
		\item  Nếu $B$ và $D$ có nhân tử chung là $E~(B=E \cdot M, D=E \cdot N$ với $M$ và $N$ là những đa thức) thì lấy mẫu thức chung là $E \cdot M \cdot N$. Khi đó, ta quy đồng mẫu thức:
		$$\dfrac{A}{B}=\dfrac{A \cdot N}{B \cdot N}=\dfrac{A \cdot N}{E \cdot M \cdot N} \text { và } \dfrac{C}{D}=\dfrac{C \cdot M}{D \cdot M}=\dfrac{C \cdot M}{E \cdot N \cdot M}=\dfrac{C \cdot M}{E \cdot M \cdot N}.$$
	\end{itemize}
\end{tomtat}

\begin{vd}%%[Dự án EX-9-Đề Cương Toán 9 - đợt 2]%[Lê Hòa Nam]%[8D1N6-2]
	Quy đồng mẫu thức của các cặp phân thức sau:
	\begin{multicols}{3}
		\begin{enumerate}
			\item  $\dfrac{2a}{a-5}$ và $\dfrac{-a}{a+5}$;
			\item  $\dfrac{1}{3abc}$ và $\dfrac{a+b}{ab^2}$;
			\item  $\dfrac{3}{a^2-4}$ và $\dfrac{a^2}{a+2}$.
		\end{enumerate}
	\end{multicols}
	\loigiai{
		\begin{enumerate}
			\item  Mẫu thức chung là $(a-5)(a+5)=a^2-25$.
			$$\dfrac{2a}{a-5}=\dfrac{2a(a+5)}{(a-5)(a+5)}=\dfrac{2a^2+10a}{a^2-25}; \quad \dfrac{-a}{a+5}=\dfrac{-a(a-5)}{(a+5)(a-5)}=\dfrac{-a^2+5a}{a^2-25}.$$
			\item  Ta có $3abc=ab \cdot 3c$ và $ab^2=ab \cdot b$ nên mẫu thức chung là $ab \cdot 3c \cdot b=3ab^2c$.
			$$\dfrac{1}{3abc}=\dfrac{b}{3abc \cdot b}=\dfrac{b}{3ab^2c}; \quad \dfrac{a+b}{ab^2}=\dfrac{(a+b) \cdot 3c}{ab^2 \cdot 3c}=\dfrac{3ac+3bc}{3ab^2c}.$$
			\item  Ta có $a^2-4=(a+2)(a-2)$. Do đó, mẫu thức chung là $a^2-4$.
			$$\dfrac{a^2}{a+2}=\dfrac{a^2(a-2)}{(a+2)(a-2)}=\dfrac{a^3-2 a^2}{a^2-4}.$$
			Nhờ quy đồng mẫu thức, ta đưa các phép tính cộng, trừ hai phân thức khác mẫu thức về phép tính cộng, trừ hai phân thức cùng mẫu thức.
		\end{enumerate}
	}
\end{vd}
\begin{tomtat}
	Muốn cộng, trừ hai phân thức khác mẫu thức, ta thực hiện các bước
	\begin{itemize}
		\item  Quy đồng mẫu thức;
		\item  Cộng, trừ các phân thức có cùng mẫu thức vừa tìm được. 	
	\end{itemize}
\end{tomtat}

\begin{vd}%%[Dự án EX-9-Đề Cương Toán 9 - đợt 2]%[Lê Hòa Nam]%[8D1N6-2]
	Thực hiện các phép cộng, trừ phân thức sau:
	\begin{multicols}{3}
\begin{enumerate}
	\item  $\dfrac{2}{a}+\dfrac{1}{a-3}$;
	\item $\dfrac{2x}{x^2-4}-\dfrac{1}{x-2}$;
	\item  $\dfrac{x}{xy-y^2}-\dfrac{y}{x^2-xy}$.
\end{enumerate}
\end{multicols}
	
	\loigiai{
		\begin{enumerate}
			\item $\dfrac{2}{a}+\dfrac{1}{a-3}=\dfrac{2(a-3)}{a(a-3)}+\dfrac{a}{a(a-3)}=\dfrac{2 a-6+a}{a(a-3)}=\dfrac{3a-6}{a(a-3)}=\dfrac{3(a-2)}{a(a-3)}.$
			\item 
				$\dfrac{2x}{x^2-4}-\dfrac{1}{x-2}=\dfrac{2x}{(x+2)(x-2)}-\dfrac{(x+2)}{(x+2)(x-2)}
				=\dfrac{2x-(x+2)}{(x+2)(x-2)}=\dfrac{x-2}{(x+2)(x-2)}=\dfrac{1}{x+2}.$
			\item 
			\begin{eqnarray*}
				&&\dfrac{x}{xy-y^2}-\dfrac{y}{x^2-xy} \\ 
				&=&\dfrac{x}{y(x-y)}-\dfrac{y}{x(x-y)}\\
				&=&\dfrac{x^2}{xy(x-y)}-\dfrac{y^2}{xy(x-y)} \\
				&=&\dfrac{x^2-y^2}{xy(x-y)}\\
                &=&\dfrac{(x+y)(x-y)}{xy(x-y)}\\
                &=&\dfrac{x+y}{xy}.
			\end{eqnarray*}
		\end{enumerate}
	}
\end{vd}

\begin{luuy}
	\begin{enumerate}
		\item Phép cộng các phân thức cũng có các tính chất giao hoán, kết hợp:
		$$\dfrac{A}{B}+\dfrac{C}{D}=\dfrac{C}{D}+\dfrac{A}{B} ; \quad \left(\dfrac{A}{B}+\dfrac{C}{D}\right)+\dfrac{E}{F}=\dfrac{A}{B}+\left(\dfrac{C}{D}+\dfrac{E}{F}\right).$$
		Nhờ tính chất kết hợp, trong một dãy phép cộng nhiều phân thức, ta không cần đặt dấu ngoặc.
		\item  Nếu $\dfrac{A}{B}+\dfrac{C}{D}=0$ thì phân thức $\dfrac{C}{D}$ được gọi là phân thức đối của phân thức $\dfrac{A}{B}$, kí hiệu là $-\dfrac{A}{B}$. Tương tự như với phân số, ta có tính chất:
		$$-\dfrac{A}{B}=\dfrac{-A}{B}=\dfrac{A}{-B}.$$
		\item Phép trừ phân thức có thể chuyển thành phép cộng với phân thức đối:
		$$\dfrac{A}{B}-\dfrac{C}{D}=\dfrac{A}{B}+\left(-\dfrac{C}{D}\right).$$
	\end{enumerate}
\end{luuy}

\begin{vd}%%[Dự án EX-9-Đề Cương Toán 9 - đợt 2]%[Lê Hòa Nam]%[8D1H6-2]
	Thực hiện phép tính $\dfrac{2a}{(a+1)^2}+\dfrac{a}{a+1}+\dfrac{1-a}{a^2+2a+1}$.
	\loigiai{
		\begin{eqnarray*}
		&&\dfrac{2a}{(a+1)^2}+\dfrac{a}{a+1}+\dfrac{1-a}{a^2+2 a+1} \\
		&=&\dfrac{2a}{(a+1)^2}+\dfrac{1-a}{(a+1)^2}+\dfrac{a}{a+1}\\
		&=&\dfrac{a+1}{(a+1)^2}+\dfrac{a}{a+1}\\
		&=&\dfrac{1}{a+1}+\dfrac{a}{a+1}=\dfrac{1+a}{a+1}=1.
		\end{eqnarray*}
	}
\end{vd}


\begin{vd}%%[Dự án EX-9-Đề Cương Toán 9 - đợt 2]%[Lê Hòa Nam]%[8D1H6-2]
	Giải các phương trình
	\begin{multicols}{2}
		\begin{enumerate}
			\item $3x(x+7)=0$;
			\item $(x-5)(2x-4)=0$.
		\end{enumerate}
	\end{multicols}
	\loigiai{
		\begin{enumerate}
			\item Ta có
			\allowdisplaybreaks
			\begin{eqnarray*}
				&& 3x(x+7)=0 \\
				&& 3x=0 \text{ hoặc } x+7=0	\\
				&& x=0 \text{ hoặc } x=-7.
			\end{eqnarray*}
			Vậy phương trình đã cho có hai nghiệm là $x=0$; $x=-7$.
			\item Ta có
			\allowdisplaybreaks
			\begin{eqnarray*}
				&& (x-5)(2x-4)=0 \\
				&& x-5=0 \text{ hoặc } 2x-4=0 \\
				&& x=5 \text{ hoặc } x=2.
			\end{eqnarray*}
			Vậy phương trình đã cho có hai nghiệm là $x=5$; $x=2$.
		\end{enumerate}
	}
\end{vd}

\begin{vd}%%[Dự án EX-9-Đề Cương Toán 9 - đợt 2]%[Lê Hòa Nam]%[8D1H6-2]
	Giải các phương trình sau bằng cách đưa về phương trình tích
	\begin{multicols}{2}
		\begin{enumerate}
			\item $x^2+7x=0$;
			\item $(3x+2)^2-4x^2=0$.
		\end{enumerate}
	\end{multicols}
	\loigiai{
		\begin{enumerate}
			\item Ta có
			\allowdisplaybreaks
			\begin{eqnarray*}
				&& x^2+7x=0 \\
				&& x(x+7)=0 \\
				&& x=0 \text{ hoặc } x+7=0 \\
				&& x=0 \text{ hoặc } x=-7.
			\end{eqnarray*}
			Vậy phương trình đã cho có hai nghiệm là $x=0$; $x=-7$.
			\item Ta có
			\allowdisplaybreaks
			\begin{eqnarray*}
				&& (3x+2)^2-4x^2=0 \\
				&& (3x+2+2x)(3x+2-2x)=0 \\
				&& (5x+2)(x+2)=0 \\
				&& 5x+2=0 \text{ hoặc } x+2=0 \\
				&& x=-\dfrac{2}{5} \text{ hoặc } x=-2.
			\end{eqnarray*}
			Vậy phương trình đã cho có hai nghiệm là $x=-\dfrac{2}{5}$; $x=-2$. 
		\end{enumerate}
	}
\end{vd}

\subsubsection{Bài tập}
\begin{bt}%%[Dự án EX-9-Đề Cương Toán 9 - đợt 2]%[Lê Hòa Nam]%[8D1N6-2]%[8D1H6-2]%%[8D1V6-2]
	Thực hiện các phép cộng, trừ phân thức sau:
\begin{multicols}{2}
	\begin{enumerate}
		\item $\dfrac{x}{x+3}+\dfrac{2-x}{x+3}$;
		\item $\dfrac{x^2y}{x-y}-\dfrac{xy^2}{x-y}$;
		\item $\dfrac{2x}{2x-y}+\dfrac{y}{y-2x}$.
		
		\item  $\dfrac{a}{a-3}-\dfrac{3}{a+3}$;
		\item  $\dfrac{1}{2x}+\dfrac{2}{x^2}$;
		\item $\dfrac{4}{x^2-1}-\dfrac{2}{x^2+x}$;
		
		\item  $\dfrac{a-1}{a+1}+\dfrac{3-a}{a+1}$;
		\item  $\dfrac{b}{a-b}+\dfrac{a}{b-a}$;
		\item  $\dfrac{(a+b)^2}{ab}-\dfrac{(a-b)^2}{ab}$;	
		
		\item  $\dfrac{1}{2a}+\dfrac{2}{3b}$;
		\item  $\dfrac{x-1}{x+1}-\dfrac{x+1}{x-1}$;
		\item  $\dfrac{x+y}{xy}-\dfrac{y+z}{yz}$;
		\item  $\dfrac{2}{x-3}-\dfrac{12}{x^2-9}$;
		\item  $\dfrac{1}{x-2}+\dfrac{2}{x^2-4x+4}$;	
		
		\item $\dfrac{x}{x+y}+\dfrac{2xy}{x^2-y^2}-\dfrac{y}{x+y}$.
	\end{enumerate}
\end{multicols}
	\loigiai{
		\begin{enumerate}
				\item $\dfrac{x}{x+3}+\dfrac{2-x}{x+3} = \dfrac{x+2-x}{x+3} = \dfrac{2}{x+3}.$
			\item 
			$\dfrac{x^2y}{x-y}-\dfrac{xy^2}{x-y} = \dfrac{x^2y-xy^2}{x-y} = \dfrac{xy(x-y)}{x-y} = xy.$
			\item 
				$\dfrac{2x}{2x-y}+\dfrac{y}{y-2x} = \dfrac{2x}{2x-y}-\dfrac{y}{2x-y} = \dfrac{2x-y}{2x-y}=1.$
			\item  
				$\dfrac{a}{a-3}-\dfrac{3}{a+3} = \dfrac{a(a+3)}{(a-3)(a+3)} - \dfrac{3(a-3)}{(a-3)(a+3)} = \dfrac{a^2+3a-3a+9}{a^2-9} = \dfrac{a^2+9}{a^2-9}.$
			\item  
				$\dfrac{1}{2x}+\dfrac{2}{x^2} = \dfrac{x}{2x^2} + \dfrac{2\cdot 2}{2x^2} = \dfrac{x+4}{2x^2}.$
			
			\item \begin{eqnarray*}
				&&\dfrac{4}{x^2-1}-\dfrac{2}{x^2+x}\\ 
				&=&  \dfrac{4}{(x-1)(x+1)} - \dfrac{2}{x(x+1)}\\
				& =& \dfrac{4x}{x(x-1)(x+1)} - \dfrac{2(x-1)}{x(x-1)(x+1)}\\
				&=&\dfrac{4x-2x+2}{x(x-1)(x+1)}\\
				&=& \dfrac{2x+2}{x(x-1)(x+1)}\\
				& = &\dfrac{2(x+1)}{x(x-1)(x+1) }\\
				& = &\dfrac{2}{x(x-1)}.
			\end{eqnarray*}
			
			\item  
				$\dfrac{a-1}{a+1}+\dfrac{3-a}{a+1}=\dfrac{a-1+3-a}{a+1} = \dfrac{2}{a+1}.$
			\item  
				$\dfrac{b}{a-b}+\dfrac{a}{b-a}=\dfrac{b}{a-b}-\dfrac{a}{a-b} =\dfrac{b-a}{a-b} = -1.$
			\item  
				$\dfrac{(a+b)^2}{ab}-\dfrac{(a-b)^2}{ab} = \dfrac{(a^2+2ab+b^2)-(a^2-2ab+b^2)}{ab} = \dfrac{4ab}{ab}=4.$
			\item  
				$\dfrac{1}{2a}+\dfrac{2}{3b} = \dfrac{3b}{6ab} + \dfrac{4a}{6ab} =\dfrac{4a+3b}{6ab}.$
			\item  
				$\dfrac{x-1}{x+1}-\dfrac{x+1}{x-1} = \dfrac{(x-1)^2}{(x+1)(x-1)} -\dfrac{(x+1)^2}{(x+1)(x-1)} = \dfrac{(x^2-2x+1) - (x^2+2x+1)}{(x+1)(x-1)}  = \dfrac{-4x}{x^2-1}.$
			\item  
				$\dfrac{x+y}{xy}-\dfrac{y+z}{yz} =  \dfrac{z(x+y)}{xyz} - \dfrac{x(y+z)}{xyz} = \dfrac{xz+yz-xy-xz}{xyz} = \dfrac{y(z-x)}{xyz} = \dfrac{z-x}{xz}.$
			\item  
				$\dfrac{2}{x-3}-\dfrac{12}{x^2-9} = \dfrac{2(x+3)}{(x-3)(x+3)} - \dfrac{12}{(x-3)(x+3)} = \dfrac{2x+6-12}{(x-3)(x+3)} 
				= \dfrac{2(x-3)}{(x-3)(x+3)} = \dfrac{2}{x+3}.$
			\item  
				$\dfrac{1}{x-2}+\dfrac{2}{x^2-4x+4} = \dfrac{1}{x-2} + \dfrac{2}{(x-2)^2} = \dfrac{x-2}{(x-2)^2}+ \dfrac{2}{(x-2)^2} = \dfrac{x-2+2}{(x-2)^2} = \dfrac{x}{(x-2)^2}.$
			\item 
			\begin{eqnarray*}
				&&\dfrac{x}{x+y}+\dfrac{2xy}{x^2-y^2}-\dfrac{y}{x+y}  \\
				&=& \dfrac{x}{x+y} - \dfrac{y}{x+y} + \dfrac{2xy}{(x-y)(x+y)} \\
				&=& \dfrac{x-y}{x+y} + \dfrac{2xy}{(x-y)(x+y)}\\
				&=& \dfrac{(x-y)^2}{(x-y)(x+y)}+ \dfrac{2xy}{(x-y)(x+y)}\\
				& =& \dfrac{x^2-2xy+y^2+2xy}{x^2-y^2} = \dfrac{x^2+y^2}{x^2-y^2}.
			\end{eqnarray*}
		\end{enumerate}
	}
\end{bt}



\begin{bt}%%[Dự án EX-9-Đề Cương Toán 9 - đợt 2]%[Lê Hòa Nam]%[8D1V6-3]
	Tại một cuộc đua thuyền diễn ra trên một khúc sông từ $A$ đến $B$ dài $3$ km. Mỗi đội thực hiện một vòng đua, xuất phát từ $A$ đến $B$, rồi quay về $A$ là đích. Một đội đua đạt tốc độ $(x+1)$ km/h khi xuôi dòng từ $A$ đến $B$ và đạt tốc độ $(x-1)$ km/h khi ngược dòng từ $B$ về $A$.\\
	Viết biểu thức tính tổng thời gian đi và về, chênh lệch thời gian giữa đi và về của đội đua thuyền. Tính giá trị của các đại lượng này khi $v=6$ km/h.
	\loigiai{
		Quãng đường từ $A$ đến $B$ dài $3$ km. Ta có thời gian $= \dfrac{\text{quãng đường}}{\text{vận tốc}}$.\\
		Khi xuôi dòng, tốc độ của đội đua là $(x+1)$ km/h nên thời gian đi (thời gian xuôi dòng) là $\dfrac{3}{x+1}$ h. \\
		Khi ngược dòng, tốc độ của đội đua là $(x-1)$ km/h nên thời gian về (thời gian ngược dòng) là $\dfrac{3}{x-1}$ h. \\
		\begin{itemize}
			\item 	Tổng thời gian đi và về của đội đua thuyền là $\dfrac{3}{x+1} + \dfrac{3}{x-1}$ h. 
			\item Chênh lệch thời gian giữa đi và về là $\dfrac{3}{x-1} - \dfrac{3}{x+1}$ h. 	
			\item Khi $v=6$ km/h tức là $x = 6$. Khi đó
			\begin{itemize}
				\item Tổng thời gian đi và về của đội đua thuyền là $\dfrac{3}{6+1} + \dfrac{3}{6-1} = \dfrac{36}{35}$ h.
				\item Chênh lệch thời gian giữa đi và về là $\dfrac{3}{6-1} - \dfrac{3}{6+1} = \dfrac{6}{35}$ h. \\
				Tức là đi khi đội đua thuyền đi về thời gian lâu hơn khi đội đi là $\dfrac{6}{35}$ giờ.
			\end{itemize}
		\end{itemize}
	}
\end{bt}

\begin{bt}%%[Dự án EX-9-Đề Cương Toán 9 - đợt 2]%[Lê Hòa Nam]%[8D1V6-3]
	Cùng đi từ thành phố $A$ đến thành phố $B$ cách nhau $450$ km, xe khách chạy với tốc độ $x$ km/h; xe tải chạy với tốc độ $y$ km/h $(x>y)$. Nếu xuất phát cùng lúc thì xe khách đến thành phố $B$ sớm hơn xe tải bao nhiêu giờ?
	\loigiai{
		Quãng đường đi từ thành phố $A$ đến thành phố $B$ cách nhau $450$ km. \\
		Xe khách chạy với tốc độ $x$ km/h nên thời gian xe khách đi hết quãng đường $AB$ là $\dfrac{450}{x}$ h.\\
		Xe tải chạy với tốc độ $y$ km/h nên thời gian xe tải đi hết quãng đường $AB$ là $\dfrac{450}{y}$ h.\\	
		Thời gian xe khách đến thành phố $B$ sớm hơn xe tải là $\dfrac{450}{y} - \dfrac{450}{x} = \dfrac{450x-450y}{xy}$ h.
	}
\end{bt}

\begin{bt}%%[Dự án EX-9-Đề Cương Toán 9 - đợt 2]%[Lê Hòa Nam]%[8D1C6-3]
	\immini{Có ba hình hộp chữ nhật $A$, $B$, $C$ có chiều dài, chiều rộng và thể tích được cho như Hình $ 2 $. Hình $B$ và $C$ có các kích thước giống nhau, hình $A$ có cùng chiều rộng với $B$ và $C$.
		\begin{enumerate}
			\item Tính chiều cao của các hình hộp chữ nhật. Biểu thị chúng bằng các phân thức cùng mẫu số.
			\item Tính tổng chiều cao của hình $A$ và $C$, chênh lệch chiều cao của hình $A$ và $B$.
		\end{enumerate}
	}
	{
		\begin{tikzpicture}[>=stealth,line join=round,line cap=round,font=\footnotesize,scale=1]
			\coordinate(O) at (0,0); 
			\coordinate[label=below right:$A$](A) at (0,2);
			\coordinate[label=above:$E$](E) at (3,0); 		
			\coordinate[label=below left:$D$](D) at (3,2.); 		
			\coordinate[label=below right:$C$](C) at (0,3.5);
			\coordinate[label=below left:$G$](G) at (2.5,3.5);		
			\coordinate[label=below left:$H$](H) at (2.5,2);		
			\coordinate[label=below left:$K$](K) at (1,4);
			\coordinate[label=above:$F$](F) at (5.5,0); 	
			\coordinate[label=below left:$I$](I) at (5.5,1.5); 
			\coordinate[label=below left:$J$](J) at (3,1.5);		
			\coordinate[label=below left:$L$](L) at ($(G)-(C)+(K)$);			
			\coordinate[label=below left:$M$](M) at ($(L)-(G)+(H)$);
			\coordinate[label=below left:$N$](N) at ($(M)-(H)+(D)$);		
			\coordinate[label=below left:$P$](P) at ($(N)-(D)+(J)$);		
			\coordinate[label=below left:$Q$](Q) at ($(P)-(J)+(I)$);						
			\coordinate(R) at ($(Q)-(I)+(F)$);	 
			\fill[cyan!20] (A)--(C)--(K)--(L)--(M)--(H)--cycle
			(E)--(F)--(R)--(Q)--(P)--(J)--cycle;
			\fill[pink!30] (O)--(A)--(D)--(E)
			(H)--(D)--(N)--(M)
			(D)--(J)--(P)--(N);
			\draw[red!70] (A)--(O)--(E)--(D) --cycle
			(A)--(C)--(K)--(L)--(G)--(C)     (G)--(L)--(M)--(H)--cycle
			(E)--(F)--(I)--(J) (M)--(N)--(D)--(H)
			(D)--(N)--(P)--(J) (P)--(Q)--(I)
			(I)--(Q)--(R)--(F)
			;		
			\draw (1.5,1) node {$a$ cm$^3$}
			(4.3,0.8) node {$b$ cm$^3$}
			(1.5,-0.3) node {$x$ cm}
			(4.3,-0.3) node {$y$ cm}
			(6.2,0) node {$z$ cm}
			(0.3,1.7) node {A}	
			(0.3,3.2) node {C}	 
			(3.3,1.2) node {B}
			(4.2,2.2) node {?}
			(3,-1) node {Hình 2}		 	 	
			;
		\end{tikzpicture}
	}
	\loigiai{
		\begin{enumerate}
			\item 	Ta có $V= \text{Chiều dài } \cdot \text{ Chiều rộng } \cdot \text{ Chiều cao} \Rightarrow  \text{Chiều cao} = \dfrac{V}{\text{Chiều dài}\cdot \text{Chiều rộng}}$. Dựa vào Hình $2$ ta thấy: 
			\begin{itemize}
				\item 	Khối hộp chữ nhật $A$ có thể tích là $a$ cm$^3$, chiều dài là $x$ cm và chiều rộng là $z$ cm nên chiều cao là $\dfrac{a}{xz} = \dfrac{ay}{xyz}$ cm.
				\item Khối hộp chữ nhật $B$ có thể tích là $b$ cm$^3$, chiều dài là $y$ cm và chiều rộng là $z$ cm nên chiều cao là $\dfrac{b}{yz} = \dfrac{bx}{xyz}$ cm.
				\item Khối hộp chữ nhật $C$ có kích thước giống khối hộp $B$ nên chiều cao là $\dfrac{bx}{xyz}$ cm.
			\end{itemize}
			\item Tổng chiều cao của hình $A$ và $C$ là $\dfrac{ay}{xyz} +\dfrac{bx}{xyz} = \dfrac{ay+bx}{xyz}$ cm.\\
			Ta có $\dfrac{ay}{xyz} - \dfrac{bx}{xyz} = \dfrac{ay-bx}{xyz}$ nên hình $A$ cao hơn hình $B$ là $\dfrac{ay-bx}{xyz}$ cm.
		\end{enumerate}
	}
\end{bt}

\subsection{Nhân phân thức đại số}
\subsubsection{Kiến thức trọng tâm}
\begin{tomtat}
	Muốn nhân hai phân thức, ta nhân các tử thức với nhau, các mẫu thức với nhau.
	\[\dfrac{A}{B} \cdot \dfrac{C}{D}=\dfrac{A \cdot C} {B \cdot D}.\]
	Cūng tương tự phép nhân các phân số, phép nhân các phân thức có các tính chất sau:
	\begin{enumerate}
		\item Tính chất giao hoán:
		\[\dfrac{A}{B} \cdot \dfrac{C}{D}=\dfrac{C}{D} \cdot \dfrac{A}{B}.\]
		\item Tính chất kết hợp:
		\[\left(\dfrac{A}{B} \cdot \dfrac{C}{D}\right) \cdot \dfrac{E}{G}=\dfrac{A}{B} \cdot\left(\dfrac{C}{D} \cdot \dfrac{E}{G}\right).\]
		\item Tính chất phân phối đối với phép cộng:
		\[\dfrac{A}{B}\cdot\left(\dfrac{C}{D}+\dfrac{E}{G}\right)=\dfrac{A}{B} \cdot \dfrac{C}{D}+\dfrac{A}{B} \cdot \dfrac{E}{G}.\]
	\end{enumerate}
\end{tomtat}

%%=====Ví dụ 1
\begin{vd}%%[Dự án EX-9-Đề Cương Toán 9 - đợt 2]%[Lê Hòa Nam]%[8D1N7-1]
	Thực hiện các phép nhân phân thức sau:
	\begin{multicols}{2}
	\begin{enumerate}
		\item $\dfrac{2ac}{3b} \cdot \dfrac{-6b^3}{8a^2c}$;
		\item $\dfrac{x^2-1}{x^2+4x} \cdot \dfrac{2x}{x-1}$.
	\end{enumerate}
\end{multicols}
	\loigiai{
		\begin{enumerate}
			\item 
				$\dfrac{2ac}{3b} \cdot \dfrac{-6b^3}{8a^2c}= \dfrac{2ac \cdot (-6b^3)}{3b \cdot 8a^2c}=\dfrac{2 \cdot (-6) \cdot a \cdot c \cdot b^3}{3 \cdot 8 \cdot b \cdot a^2 \cdot c}=\dfrac{-b^2}{2a}.$
			\item 
				$\dfrac{x^2-1}{x^2+4x} \cdot \dfrac{2x}{x-1}=\dfrac{(x+1) \cdot (x-1) \cdot 2x}{x \cdot (x+4) \cdot (x-1)}=\dfrac{2 \cdot (x+1)}{x+4}.$
		\end{enumerate}
	}
\end{vd}
%%=====Ví dụ 2
\begin{vd}%%[Dự án EX-9-Đề Cương Toán 9 - đợt 2]%[Lê Hòa Nam]%[8D1N7-1]
	Tính:
	\begin{multicols}{2}
	\begin{enumerate}
		\item $\dfrac{x^2-4x+4}{x^2+2x+1} \cdot \dfrac{x+1}{x^2-2x} \cdot \dfrac{6x}{2x+4}$;
		\item $\dfrac{1}{4a}-\dfrac{1}{a+b}\cdot \left( \dfrac{a+b}{4a}-a^2b-ab^2 \right)$.
	\end{enumerate}
\end{multicols}
	\loigiai{
	\begin{enumerate}
		\item
		\begin{eqnarray*}
			&&\dfrac{x^2-4x+4}{x^2+2x+1} \cdot \dfrac{x+1}{x^2-2x} \cdot \dfrac{6x}{2x+4}\\
			&=&\dfrac{(x-2)^2}{(x+1)^2} \cdot \dfrac{x+1}{x \cdot (x-2)} \cdot \dfrac{6x}{2(x+2)}\\
			&=&\dfrac{6x \cdot (x-2)^2 \cdot (x+1)}{2x \cdot (x+1)^2 \cdot (x-2) \cdot (x+2)}\\
			&=&\dfrac{3 \cdot (x-2)}{(x+1) \cdot(x+2)}\\
			&=&\dfrac{3 x-6}{(x+1) \cdot(x+2)}.
		\end{eqnarray*}
		\item 
		\begin{eqnarray*}
			&&\dfrac{1}{4a}-\dfrac{1}{a+b}\cdot \left( \dfrac{a+b}{4a}-a^2b-ab^2 \right)\\
			&=&\dfrac{1}{4a}-\dfrac{1}{a+b}\cdot \dfrac{a+b}{4a}+\dfrac{1}{a+b}\cdot ab \cdot (a+b)\\
			&=&\dfrac{1}{4a}-\dfrac{a+b}{4a \cdot (a+b)}+\dfrac{ab \cdot (a+b)}{a+b}\\
			&=&\dfrac{1}{4a}-\dfrac{1}{4a}+ab\\
			&=&ab.
		\end{eqnarray*}
	\end{enumerate}
	}
\end{vd}

\subsection{Chia phân thức đại số}
\subsubsection{Kiến thức trọng tâm}
\begin{tomtat}
	Tương tự phép chia phân số, phép chia phân thức được thực hiện theo quy tắc sau:\\
	Muốn chia phân thức $\dfrac{A}{B}$ cho phân thức $\dfrac{C}{D}$ ($C$ khác đa 
	thức không), ta nhân phân thức $\dfrac{A}{B}$ với phân thức $\dfrac{D}{C}$.
	\[
	\dfrac{A}{B}:\dfrac{C}{D}=\dfrac{A}{B}\cdot\dfrac{D}{C}.
	\]
	
	\textbf{Nhận xét:} Phân thức $\dfrac{D}{C}$ gọi là phân thức \textit{nghịch đảo} của phân thức 
	$\dfrac{C}{D}\cdot$
\end{tomtat}
%%=====Ví dụ 3
\begin{vd}%%[Dự án EX-9-Đề Cương Toán 9 - đợt 2]%[Lê Hòa Nam]%[8D1N7-2]
	Thực hiện các phép tính sau: 
	\begin{multicols}{2}
	\begin{enumerate}
		\item $\dfrac{x^2-4}{x^2+5x}:\dfrac{x+2}{2x}$;
		\item $\dfrac{x^2}{y}\cdot\dfrac{xz}{y^2}:\dfrac{x^2}{yz}$.
	\end{enumerate}
\end{multicols}
	\loigiai{
		\begin{enumerate}
			\item 
				$\dfrac{x^2-4}{x^2+5x}:\dfrac{x+2}{2x}
				=\dfrac{(x-2)(x+2)}{x(x+5)}\cdot\dfrac{2x}{x+2}
				=\dfrac{2(x-2)}{x+5}.$
			\item  
				$\dfrac{x^2}{y}\cdot\dfrac{xz}{y^2}:\dfrac{x^2}{yz}
				=\left(\dfrac{x^2}{y}\cdot\dfrac{xz}{y^2}\right):\dfrac{x^2}{yz}
				=\dfrac{x^3z}{y^3}\cdot\dfrac{yz}{x^2}=\dfrac{xz^2}{y^2}.$
		\end{enumerate}		
	}
\end{vd}
%%=====Ví dụ 4
\begin{vd}%%[Dự án EX-9-Đề Cương Toán 9 - đợt 2]%[Lê Hòa Nam]%[8D1N7-2]
	Thực hiện  phép tính sau: $\dfrac{1}{x}-\dfrac{1}{x}:\dfrac{1}{x}
	+\dfrac{1}{x}\cdot\dfrac{x^2}{2}\cdot$
	\loigiai{
			$\dfrac{1}{x}-\dfrac{1}{x}:\dfrac{1}{x}
			+\dfrac{1}{x}\cdot\dfrac{x^2}{2}\cdot
			=\dfrac{1}{x}-1+\dfrac{x}{2}
			=\dfrac{2-2x+x^2}{2x}.$
	}
\end{vd}

\subsubsection{Bài tập}

\begin{bt}%%[Dự án EX-9-Đề Cương Toán 9 - đợt 2]%[Lê Hòa Nam]%[8D1N7-1]%[8D1H7-1]%[8D1V7-1]
	Tính:
	\begin{multicols}{2}
	\begin{enumerate}
		\item $\dfrac{3a^2}{10b^3}\cdot \dfrac{15b}{9a^4}$;
		\item $\dfrac{x-3}{x^2}\cdot \dfrac{4x}{x^2-9}$;
		\item $\dfrac{a^2-6a+9}{a^2+3a}\cdot \dfrac{2a+6}{a-3}$;
		\item $\dfrac{x+1}{x}\cdot \left( x+\dfrac{2-x^2}{x^2-1} \right)$;
		\item $\dfrac{4y}{3x^2}\cdot\dfrac{5x^3}{2y^3}$;
		\item $\dfrac{x^2-2x+1}{x^2-1}\cdot\dfrac{x^2+x}{x-1}$;
		\item  $\dfrac{2x+x^2}{x^2-x+1}\cdot\dfrac{3x^3+3}{3x+6}$;
		\item $\dfrac{4x^2+2}{x-2}\cdot\dfrac{3x+2}{x-4}\cdot\dfrac{4-2x}{2x^2+1}$.
	\end{enumerate}
\end{multicols}
	\loigiai{
		\begin{enumerate}
			\item 
				$\dfrac{3a^2}{10b^3}\cdot \dfrac{15b}{9a^4}=\dfrac{3a^2 \cdot 15b}{10b^3 \cdot 9a^4}=\dfrac{1 \cdot 3}{2b^2 \cdot 3a^2}=\dfrac{1}{2a^2b^2}.$
			\item 
				$\dfrac{x-3}{x^2}\cdot \dfrac{4x}{x^2-9}=\dfrac{(x-3) \cdot 4x}{x^2 \cdot (x-3)\cdot (x+3)}=\dfrac{4}{x \cdot (x+3)}.$
			\item 
				$\dfrac{a^2-6a+9}{a^2+3a}\cdot \dfrac{2a+6}{a-3}=\dfrac{(a-3)^2 \cdot 2(a+3)}{a(a+3) \cdot (a-3)}=\dfrac{(a-3) \cdot 2}{a}=\dfrac{2a-6}{a}.$
			\item 
			\begin{eqnarray*}
				&&\dfrac{x+1}{x}\cdot \left( x+\dfrac{2-x^2}{x^2-1} \right)\\
				&=&\dfrac{x+1}{x}\cdot \left( \dfrac{x(x^2-1)+2-x^2}{x^2-1} \right)\\
				&=&\dfrac{x+1}{x}\cdot \left( \dfrac{x^3-x+2-x^2}{(x-1)(x+1)} \right)\\
				&=&\dfrac{1}{x}\cdot \left( \dfrac{x^3-x^2-x+2}{x-1} \right)\\
				&=&\dfrac{x^3-x^2-x+2}{x(x-1)}.
			\end{eqnarray*}
			\item 
				$\dfrac{4y}{3x^2}\cdot\dfrac{5x^3}{2y^3}
				=\dfrac{20x^3y}{6x^2y^3}=\dfrac{10x}{3y^2}.$
			\item 
				$\dfrac{x^2-2x+1}{x^2-1}\cdot\dfrac{x^2+x}{x-1}
				=\dfrac{(x-1)^2}{(x-1)(x+1)}\cdot\dfrac{x(x+1)}{x-1}
				=x.$
			\item 
		 	\begin{eqnarray*}
		 		&&\dfrac{2x+x^2}{x^2-x+1}\cdot\dfrac{3x^3+3}{3x+6}\\
		 		&=&\dfrac{x(x+2)}{x^2-x+1}\cdot\dfrac{3(x^3+1)}{3(x+2)}\\
		 		&=&\dfrac{x(x+2)}{x^2-x+1}\cdot\dfrac{3(x+1)(x^2-x+1)}{3(x+2)}\\
		 		&=&x(x+1)\\
		 		&=&x^2+x.
		 	\end{eqnarray*}
			
			\item
			\begin{eqnarray*}
				&&\dfrac{4x^2+2}{x-2}\cdot\dfrac{3x+2}{x-4}\cdot\dfrac{4-2x}{2x^2+1}\\
				&=&\dfrac{2(2x^2+1)}{x-2}\cdot\dfrac{3x+2}{x-4}\cdot\dfrac{-2(x-2)}{2x^2+1}\\
				&=&\dfrac{-4(3x+2)}{x-4}\\
				&=&\dfrac{-12x-8}{x-4}.
			\end{eqnarray*}
			
		\end{enumerate}
	}
\end{bt}

\begin{bt}%%[Dự án EX-9-Đề Cương Toán 9 - đợt 2]%[Lê Hòa Nam]%[8D1N7-2]%[8D1H7-2]%[8D1V7-2]
	Thực hiện các phép tính sau:
	\begin{multicols}{2}
	\begin{enumerate}
		\item $\dfrac{x^2-9}{x-2}:\dfrac{x-3}{x}\cdot$
		\item $\dfrac{2x+10}{x^3-64}:\dfrac{(x+5)^2}{2x-8}$;
		\item $\dfrac{2x^2}{3 y^3}:\left(-\dfrac{4x^3}{21y^2}\right)$;
		\item $\dfrac{x}{z^2}\cdot\dfrac{xz}{y^3}:\dfrac{x^3}{yz}$;
		\item $\dfrac{2}{x}-\dfrac{2}{x}:\dfrac{1}{x}+\dfrac{4}{x}\cdot
		\dfrac{x^2}{2}$;		
		\item $\left(\dfrac{1-x}{x}+x^2-1\right):\dfrac{x-1}{x}$;
		\item $\dfrac{x+3}{x}\cdot\dfrac{x+2}{x^2+6x+9}:\dfrac{x^2-4}{x^2+3x}$;
		\item $\dfrac{3}{x}-\dfrac{2}{x}:\dfrac{1}{x}+\dfrac{1}{x}
		\cdot\dfrac{x^2}{3}$;
		\item $\dfrac{2x^2+4}{x-3} \cdot \dfrac{3x+1}{x-1}:\dfrac{x^2+2}{6-2x}$;
		\item $\dfrac{1}{x+y}\left(\dfrac{x+y}{xy}-x-y\right)-\dfrac{1}{x^2}:\dfrac{y}{x}$.
	\end{enumerate}
\end{multicols}
	\loigiai{
		\begin{enumerate}
			\item 
				$\dfrac{x^2-9}{x-2}:\dfrac{x-3}{x}
				=\dfrac{(x-3)(x+3)}{x-2}\cdot\dfrac{x}{x-3}=\dfrac{x(x+3)}{x-2}.$
				\item 
					$\dfrac{2x+10}{x^3-64}:\dfrac{(x+5)^2}{2x-8}=\dfrac{2(x+5)}{(x-4)(x^2+4x+16)}\cdot \dfrac{2(x-4)}{(x+5)^2}=\dfrac{4}{(x^2+4x+16)(x+5)}.$
				\item 
					$\dfrac{2x^2}{3 y^3}:\left(-\dfrac{4x^3}{21y^2}\right)=\dfrac{2x^2}{3 y^3}\cdot \dfrac{-21y^2}{4x^3}=\dfrac{-7}{2xy}.$
			\item 
				$\dfrac{x}{z^2}\cdot\dfrac{xz}{y^3}:\dfrac{x^3}{yz}
				=\left(\dfrac{x}{z^2}\cdot\dfrac{xz}{y^3}\right)\cdot\dfrac{yz}{x^3}
				=\dfrac{x^2}{zy^3}\cdot\dfrac{yz}{x^3}
				=\dfrac{1}{xy^2}.$
			\item 
				$\dfrac{2}{x}-\dfrac{2}{x}:\dfrac{1}{x}+\dfrac{4}{x}\cdot
				\dfrac{x^2}{2}
				=\dfrac{2}{x}-2+2x =\dfrac{2-2x+2x^2}{x}.$
			\item 
			\begin{eqnarray*}
				&\left(\dfrac{1-x}{x}+x^2-1\right):\dfrac{x-1}{x}
				&=\left[-\dfrac{x-1}{x}+(x-1)(x+1)\right]\cdot\dfrac{x}{x-1}\\
				&&=-\dfrac{x-1}{x}\cdot\dfrac{x}{x-1}+(x-1)(x+1)\cdot\dfrac{x}{x-1}\\
				&&=-1+ x(x+1)= x^2+x-1.
			\end{eqnarray*}
			\item 
				$\dfrac{x+3}{x}\cdot\dfrac{x+2}{x^2+6x+9}:\dfrac{x^2-4}{x^2+3x}
				=\dfrac{(x+3)}{x}\cdot\dfrac{x+2}{(x+3)^2}\cdot\dfrac{x(x+3)}{(x-2)(x+2)}
				=\dfrac{1}{x-2}.$
			\item
				$\dfrac{3}{x}-\dfrac{2}{x}:\dfrac{1}{x}+\dfrac{1}{x}
				\cdot\dfrac{x^2}{3}
				=\dfrac{3}{x}-2+\dfrac{x}{3}
				=\dfrac{x^2-6x+9}{3x}.$
			\item 
				$\dfrac{2x^2+4}{x-3} \cdot \dfrac{3x+1}{x-1}:\dfrac{x^2+2}{6-2x}=\dfrac{2(x^2+2)\cdot (3x+1)}{(x-3)(x+1)}\cdot \dfrac{2(3-x)}{x^2+2}=\dfrac{-4(3x+1)}{x+1}.$
			\item 
				$\dfrac{1}{x+y}\left(\dfrac{x+y}{xy}-x-y\right)-\dfrac{1}{x^2}:\dfrac{y}{x}=\dfrac{x+y}{xy(x+y)}-\dfrac{x+y}{x+y}-\dfrac{1}{x^2}\cdot \dfrac{x}{y}=\dfrac{1}{xy}-1-\dfrac{1}{xy}=-1.$
		\end{enumerate}		
	}
\end{bt}

\begin{bt}%%[Dự án EX-9-Đề Cương Toán 9 - đợt 2]%[Lê Hòa Nam]%[8D1H7-3]
	Đường sắt và đường bộ đi từ thành phố $A$ đến thành phố $B$ có độ dài bằng nhau và bằng $s$ (km). Thời gian để đi từ $A$ đến $B$ của tàu hỏa là $a$ (giờ), của ô tô khách là $b$ (giờ) ($a<b$). Tốc độ của tàu hỏa gấp bao nhiêu lần tốc độ của ô tô? Tính giá trị này khi $s=350$, $a=5$, $b=7$.
	\loigiai{
		Vận tốc tàu hỏa: $\dfrac{s}{a}$ (km/h).\\
		Vận tốc ô tô: $\dfrac{s}{b}$ (km/h).\\
		Vận tốc tàu hỏa gấp vận tốc ô tô: $\dfrac{s}{a}:\dfrac{s}{b}
		=\dfrac{s}{a}\cdot\dfrac{b}{s}=\dfrac{b}{a}$ (lần).\\
		Giá trị cần tính là $\dfrac{b}{a}=\dfrac{7}{5}.$		
	}
\end{bt}

\begin{bt}%%[Dự án EX-9-Đề Cương Toán 9 - đợt 2]%[Lê Hòa Nam]%[8D1H7-3]
	Máy $A$ xát được $x$ tấn gạo trong $a$ giờ, máy B xát được $y$ tấn gạo trong $b$ giờ.
	\begin{enumerate}
		\item Viết các biểu thức biểu thị số tấn gạo mỗi máy xát được trong $1$ giờ (gọi là công suất của máy).
		\item Công suất của máy A gấp bao nhiêu lần máy B? Viết biểu thức biểu thị số lần này.
		\item Tính giá trị biểu thức ở câu b khi $x =3$, $a=5$, $y=2$, $b=4$.
	\end{enumerate}
	\loigiai{
		\begin{enumerate}
			\item Công suất máy A là $\dfrac{x}{a}$,
			Công suất máy B là $\dfrac{y}{b}$.
			\item Công suất của máy A gấp máy B: $\dfrac{x}{a}:\dfrac{y}{b}
			=\dfrac{x}{a}\cdot\dfrac{b}{y}=\dfrac{xb}{ay}$ (lần).
			\item Tính giá trị biểu thức ở câu b khi $x =3,a=5,y=2,b=4$.
			\[
			\dfrac{xb}{ay}=\dfrac{3\cdot4}{5\cdot2}=\dfrac{6}{5}.
			\]
		\end{enumerate}		
	}
\end{bt}

\begin{bt}%%[Dự án EX-9-Đề Cương Toán 9 - đợt 2]%[Lê Hòa Nam]%[8D1H7-3]
	Tâm đạp xe từ nhà đến câu lạc bộ câu cá có quãng đường dài $15$ km với tốc độ 
	$x$ (km/h). Lượt về thuận chiều gió nên tốc độ nhanh hơn lượt đi $4$ (km/h).
	\begin{enumerate}
		\item Viết biểu thức biểu thị tổng thời gian $T$ hai lượt đi và về.
		\item Viết biểu thức biểu thị hiệu thời gian $t$ lượt đi đối với lượt về.
		\item Tính $T$ và $t$ với $x=10$.
	\end{enumerate}
	\loigiai{
		\begin{enumerate}
			\item Viết biểu thức biểu thị tổng thời gian $T$ hai lượt đi và về.\\
			Thời gian đi $\dfrac{15}{x}$ (h).\\
			Thời gian về $\dfrac{15}{x+4}$ (h).\\
			Vậy $T=\dfrac{15}{x}+\dfrac{15}{x+4}$.
			\item Viết biểu thức biểu thị hiệu thời gian $t$ lượt đi đối với lượt về:\\
			$$t=\dfrac{15}{x}-\dfrac{15}{x+4}.$$
			\item Tính $T$ và $t$ với $x=10$.\\
			\begin{eqnarray*}
				&T&=\dfrac{15}{x}+\dfrac{15}{x+4}=\dfrac{15}{10}+\dfrac{15}{14}=\dfrac{18}{7}.\\
				&t&=\dfrac{15}{x}-\dfrac{15}{x+4}=\dfrac{15}{10}-\dfrac{15}{14}=\dfrac{3}{7}.
			\end{eqnarray*}
		\end{enumerate}
	}
\end{bt}

\begin{bt}  %%[Dự án EX-9-Đề Cương Toán 9 - đợt 2]%[Lê Hòa Nam]%[8D1H7-3]
	Hôm qua, thanh long được bán với giá $a$ đồng mỗi kilôgam. Hôm nay, người ta đã giảm giá 1000 đồng cho mỗi kilôgam thanh long. Với cùng số tiền $b$ đồng thì hôm nay mua được nhiều hơn bao nhiêu kilôgam thanh long so với hôm qua?
	\loigiai{
		Gọi $a-1000$ (đồng) là số tiền mỗi kilôgam thanh long ngày hôm nay.\\
		Số kilôgam thanh long mua được với số tiền $b$ đồng ngày hôm qua là $\dfrac{b}{a}$ kilôgam.\\
		Số kilôgam thanh long mua được với số tiền $b$ đồng ngày hôm nay là $\dfrac{b}{a-1000}$ kilôgam.\\
		Số kilôgam thanh long mua được nhiều hơn so với hôm qua là $\dfrac{b}{a}-\dfrac{b}{a-1000}$ kilôgam.
	}
\end{bt}
\begin{bt}  %%[Dự án EX-9-Đề Cương Toán 9 - đợt 2]%[Lê Hòa Nam]%[8D1H7-3]
	Trên một dòng sông, một con thuyền đi xuôi dòng với tốc độ $(x+3)$ km/h và đi ngược dòng với tốc độ $(x-3)$ km/h $(x>3)$.
	\begin{enumerate}
		\item Xuất phát từ bến $A$, thuyền đi xuôi dòng trong 4 giờ, rồi đi ngược dòng trong 2 giờ. Tính quãng đường thuyền đã đi. Lúc này thuyền cách bến $A$ bao xa?
		\item Xuất phát từ bến $A$, thuyền đi xuôi dòng đến bến $B$ cách bến $A$ $15$ km, nghỉ 30 phút, rồi quay về bến $A$. Sau bao lâu kể từ lúc xuất phát thì thuyền quay về đến bến $A$?
	\end{enumerate}
	\loigiai{
	
			\begin{enumerate}
				\item Quãng đường thuyền đi xuôi dòng trong 4 giờ là $4(x+3)$ (km).\\
				Quãng đường thuyền đi ngược dòng trong 2 giờ là $2(x-3)$ (km).\\
				Quãng đường thuyền đã đi là $4(x+3)+2(x-3)=4x+12+2x-6=6x+6$ (km).\\
				Vì lúc đầu thuyền đi xuôi dòng và quay về đi ngược dòng nên quãng đường còn cách bến $A$ là $ 4(x+3)-2(x-3)=4x+12-2x+6=2x+18 $ (km).
				\item Đổi $30$ (phút) bằng $\dfrac{1}{2}$ (h).\\
				Thời gian thuyền đi xuôi dòng từ $A$ đến $B$ là $\dfrac{15}{x+3}$ (h).\\
				Thời gian thuyền đi ngược dòng từ $B$ đến $A$ là $\dfrac{15}{x-3}$ (h).\\
				Thời gian thuyền đi kể từ lúc xuất phát đến khi thuyền quay về đến bến $A$ là\\
				$\dfrac{15}{x+3}+\dfrac{1}{2}+\dfrac{15}{x-3}$ (h).
			\end{enumerate}
	}
\end{bt}


