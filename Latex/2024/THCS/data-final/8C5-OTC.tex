\section*{BÀI TẬP CUỐI CHƯƠNG 5}
\subsection{Câu hỏi trắc nghiệm}
\Opensolutionfile{ans}[ans/ans-8C5-OTC]

\begin{ex}%[8D3N2-2]
	Cho hàm số $f(x)=3x-2$ có đồ thị $(C)$. Điểm nào sau đây thuộc đồ thị hàm số $(C)$.
	\choice
	{$M(0;1)$}
	{$N(2;3)$}
	{\True $P(-2;-8)$}
	{$Q(-2;0)$}
	\loigiai
	{
		\begin{itemize}
			\item Với $x=0$ thì $f(0)=3\cdot 0-2=-2\neq 1$.
			\item Với $x=2$ thì $f(2)=3\cdot 2-3=4\neq 3$.
			\item Với $x=-2$ thì $f(x)=3\cdot (-2)-2=-8\neq 0$.
		\end{itemize}
		Vậy điểm $P(-2;-8)$ thuộc đồ thị hàm số $(C)$.
	}
\end{ex}
\begin{ex}%[8D3N3-1] 
	Trong các hàm số sau, hàm số nào là hàm số bậc nhất?
	\choice
	{\True $y=0{,}5x-5$}
	{$y=1$}
	{$y=\dfrac{1}{2x}+3$}
	{$y=2x^2+1$}
	\loigiai{	
	Hàm số $y=0{,}5x-5$ là hàm số bậc nhất.}
\end{ex}
\begin{ex}%[8D3N2-2]
	Đường thẳng nào sau đây đi qua điểm $N(1;1)$?
	\choice
	{\True $2x+y-3=0$}
	{$y-3=0$}
	{$4x+2y=0$}
	{$5x+3y-1=0$}
	\loigiai
	{
		Ta thấy $2\cdot 1+1-3=0$ nên đường thẳng $2x+y-3=0$ đi qua điểm $N(1;1)$.
	}
\end{ex}
\begin{ex}%[8D3H2-1] 
	Trong các điểm $M(3;-3)$, $N(4;2)$, $P(-3;-3)$, $Q(-2;1)$, $H(-1;3)$ có bao nhiêu điểm thuộc góc phần tư thứ hai?
	\choice
	{$0$}
	{$1$}
	{$4$}
	{\True $2$}
	\loigiai
	{
		Góc phần tư thứ hai là góc phần tư với $x<0$ và $y>0$ nên $Q(-2;1)$ và $H(-1;3)$ là các điểm nằm trong góc phần tư thứ $2$.
	}
\end{ex}
\begin{ex}%[8D3H2-1] 
	Trên mặt phẳng tọa độ $Oxy$, vẽ các điểm $A(3;1)$, $B(-2;1)$, $C(3;4)$, $D(-2;4)$. Tính diện tích tứ giác $ABCD$.
	\choice
	{\True $15$ cm$^2$}
	{$16$ cm$^2$}
	{$30$ cm$^2$}
	{$40$ cm$^2$}
	\loigiai
	{
		\immini
		{
			Từ đồ thị ta thấy $ABCD$ là hình chữ nhật với $AB=3+2=5$, $AC=4-1=3$.\\
			Diện tích của hình chữ nhật $ABCD$ là $AB\cdot AC=5\cdot 3=15$ cm$^2$.
		}
		{
			\begin{tikzpicture}[line join = round, line cap = round,>=stealth,font=\footnotesize,scale=1]
				\def \xmin{-3};
				\def \xmax{4};
				\def \ymin{-0.5};
				\def \ymax{5};
				\draw[->] (\xmin,0) -- (\xmax,0) node[below] {$x$};
				\draw[->] (0,\ymin) -- (0,0) node[below left] {$O$} -- (0,\ymax) node[left] {$y$};
				\clip (\xmin+0.1,\ymin+0.1) rectangle (\xmax-0.1,\ymax-0.1);
				\path
				(3,1) coordinate (A)
				(-2,1) coordinate (B)
				(3,4) coordinate (C)
				(-2,4) coordinate (D)
				;
				\draw (A) -- (B) -- (C) -- (D) -- (A) -- (C) (B) -- (D);
				\foreach \x/\g in {A/0,B/180,C/0,D/180} \fill[black](\x) circle (1pt)+(\g:3mm) node{$\x$};
				\draw[dashed] (3,0) node[below] {$3$} -- (3,1);
				\foreach \y in {-1,0,...,4}
				{
					\fill (0,\y) circle (1pt);
				}
				\foreach \x in {-2,-1,1,2,3}
				{
					\fill (\x,0) circle (1pt);
				}
				\draw (0,1) node[above right] {$1$};
				\draw (0,4) node[above right] {$4$};
				\draw (-2,0) node[below] {$-2$} -- (-2,1);
			\end{tikzpicture}
		}
	}
\end{ex}

\begin{ex}%[8D3H3-3] 
	Cho hàm số bậc nhất $y=f(x)=ax-a-4$. Biết $f(2)=5$, vậy $f(5)$ bằng
	\choice
	{$2$}
	{$0$}
	{\True $32$}
	{Một đáp án khác}
	\loigiai
	{
		Vì $f(2)=5$ nên $5=2a-a-4$ suy ra $a=9$. Khi đó $f(x)=9x-13$.\\
		Vậy $f(5)=9\cdot 5-13=32$.
	}
\end{ex}

\begin{ex}%[8D3H4-5]
	Cho hàm số $y=(1-m)x+m$. Xác định $m$ để đồ thị hàm số cắt trục hoành tại điểm có hoành độ  $x=-3$.
	\choice
	{$m=\dfrac{1}{2}$}
	{\True $m=\dfrac{3}{4}$}
	{$m=\dfrac{-3}{4}$}
	{$m=\dfrac{4}{5}$}
	\loigiai
	{
		Vì đồ thị hàm số $y=(1-m)x+m$ cắt trục hoành tại điểm có hoành độ $x=-3$ nên nó đi qua điểm $(-3;0)$, do đó
		\allowdisplaybreaks
		\begin{eqnarray*}
		-3(1-m)+m&=&0\\
		-3+3m+m&=&0\\
		-3+4m&=&0\\
		4m&=&3\\
		m&=&\dfrac{3}{4}.	
		\end{eqnarray*}
	}
\end{ex}


\begin{ex}%[8D3H3-2] 
	Giá trị của $m$ để hàm số $y=\dfrac{m+1}{m-2}x+2m-3$ là hàm số bậc nhất?
	\choice
	{$m\neq -1$}
	{$m>-1$}
	{\True $m\neq -1$, $m\neq 2$}
	{$m\neq 2$}
	\loigiai
	{
		Hàm số $y=\dfrac{m+1}{m-2}x+2m-3$ là hàm số bậc nhất khi $m+1\neq 0$ và $m-2\neq 0$ hay $m\neq -1$, $m\neq 2$.
	}
\end{ex}

\begin{ex}%[8D3V4-5]
	Cho đường thẳng $(d)\colon y=(m-3)x+3m+2$. Tìm giá trị nguyên của $m$ để đường thẳng $d$ cắt trục hoành tại điểm có hoành độ nguyên.
	\choice
	{$m=4$}
	{$m=14$}
	{$m=2$}
	{\True $m\in \left\{-8;2;4;14\right\}$}
	\loigiai
	{
		Đường thẳng $(d)\colon y=(m-3)x+3m+2$ cắt trục hoành tại điểm $A\left(-\dfrac{3m+2}{m-3};0\right)$.\\
		Để điểm $A$ có hoành độ nguyên thì $-\dfrac{3m+2}{m-3}=-3-\dfrac{11}{m-3}$ phải là số nguyên.\\
		Khi đó $(m-3)\in \text{Ư}(11)=\{-11;-1;1;11\}$ nên $m\in \{-8;2;4;14\}$.
	}
\end{ex}

\begin{ex}%[8D3H4-3]
	Cho đường thẳng $(d)\colon y=(m+2)x-5$ có hệ số góc là $k=-4$. Tìm $m$.  
	\choice
	{$m=-4$}
	{\True $m=-6$}
	{$m=-5$}
	{$m=-3$}
	\loigiai
	{
		Đường thẳng $(d)\colon y=(m+2)x-5$ có hệ số góc là $k=-4$ khi $m+2=-4$ suy ra $m=-6$.
	}
\end{ex}

\begin{ex}%[8D3V4-3]
	Tìm hệ số góc của đường thẳng $(d)\colon y=(3-m)x+2$ biết nó vuông góc với đường thẳng $(d')\colon x-2y-6=0$. 
	\choice
	{\True $-2$}
	{$3$}
	{$1$}
	{$2$}
	\loigiai
	{
		Đường thẳng $(d)\colon y=(3-m)x+2$ vuông góc với đường thẳng $(d')\colon x-2y-6=0$ hay $y=\dfrac{1}{2}x-3$ khi\\
		$(3-m)\cdot \dfrac{1}{2}=-1$ suy ra $3-m=-2$ hay $m=5$.\\
		Vậy hệ số góc của đường thẳng $d$ là $3-m=3-5=-2$.
	}
\end{ex}

\begin{ex}%[8D3V4-3]
	Tính hệ số góc của đường thẳng $d\colon y=5mx+4m-1$ biết nó song song với đường thẳng $d'\colon x-3y+1=0$.
	\choice
	{\True $\dfrac{1}{3}$}
	{$\dfrac{2}{3}$}
	{$1$}
	{$3$}
	\loigiai
	{
		Đường thẳng $d\colon y=5mx+4m-1$ song song với đường thẳng $d'\colon x-3y+1=0$ hay $y=\dfrac{1}{3}x+\dfrac{1}{3}$ khi
		\begin{align*}
			\heva{& 5m=\dfrac{1}{3} \\ & 4m-1\neq \dfrac{1}{3}} \text{ suy ra } m=\dfrac{1}{15}.
		\end{align*}	
		Khi đó hệ số góc của đường thẳng $d$ là $5\cdot \dfrac{1}{15}=\dfrac{1}{3}$.
	}
\end{ex}
\subsection{Bài tập tự luận}
\begin{bt}%[8D3B1-2]
	Cho hàm số $y=f(x)=-4x+5$
	\begin{enumerate}
		\item Tính $f(2),f(-4)$.
		\item Lập bảng giá trị của hàm số $x$ lần lượt bằng $-3;-2;0;1;3$.
	\end{enumerate}
	\loigiai {
		\begin{enumerate}
			\item Thay $x=2$ và $ x=-4$ vào $f(x)$ ta có\\
			$f(2)=-4\cdot (2)+5=-8+5=-3$.\\
			$f(-4)=-4\cdot (-4)+5=16+5=21$.\\
			\item Cho $x$ lần lượt bằng $-3;-2;0;1;3$, ta có bảng giá trị của hàm số:\\
			\begin{tabular}{|l|c|c|c|c|c|}
				\hline
				\begin{tabular}{l}
					$x$
				\end{tabular} & \begin{tabular}{c}
					$-3$
				\end{tabular} & \begin{tabular}{c}
					$-2$
				\end{tabular} & \begin{tabular}{c}
					$0$
				\end{tabular} & \begin{tabular}{c}
					$1$
				\end{tabular} & \begin{tabular}{c}
					$3$
				\end{tabular} \\
				\hline
				\begin{tabular}{l}
					$y=f(x)=-4x+5$
				\end{tabular} & $17$ & $13$ & $5$ & $1$ & $-7$ \\
				\hline
			\end{tabular}\\
		\end{enumerate}
	}
\end{bt}
\begin{bt}%[8D3H4-2]
	Vẽ đồ thị của các hàm số sau:
	\begin{enumEX}{2}
		\item  $y=2x$;	
		\item  $y=\dfrac{1}{4}x$.
	\end{enumEX}
	\loigiai{
		\begin{enumerate}
			\item  Cho $x=1$ ta có $y=2$. Ta vẽ điểm $M(1;2)$.\\
			Đồ thị của hàm số $y=2x$ là đường thẳng đi qua các điểm $O(0;0)$ và $M(1;2)$ (Hình a).
			\item  Cho $x=4$ ta có $y=1$. Ta vẽ điểm $N(4;1)$.\\		
			Đồ thị hàm số $y=\dfrac{1}{4}x$ là đường thẳng đi qua các điểm $O(0;0)$ và $N(4;1)$ (Hình b).
			\begin{center}
				\begin{tikzpicture}[line cap=butt,line join=miter,>=stealth,scale=1,font=\footnotesize]
					\tikzset{declare function={xmin=-2.5;xmax=2.5;ymin=-2.5;ymax=3;
							f(\x)=2*(\x);
						},
						smooth,samples=450
					}
					%============== Trục Ox =====================
					\draw[->] (xmin,0)--(xmax,0) node[shift={(0:7pt)}]{$ x $};
					\foreach  \x in {-1,1,2}{\draw (\x,1.5pt)--(\x,-1.5pt) node [below]{$\x$};}	% Đánh số trục Ox
					%============== Trục Oy =====================
					\draw[->] (0,ymin)--(0,ymax) node[shift={(90:7pt)}]{$ y $};
					\foreach  \y in {1,2}{\draw (1.5pt,\y)--(-1.5pt,\y) node [left]{$\y$};}		% Đánh số trục Oy
					\foreach  \y in {-2,-1}{\draw (1.5pt,\y)--(-1.5pt,\y) node [right]{$\y$};}	% Đánh số lệch trục Oy
					%=============================================
					\fill (0,0) node[shift={(125:7pt)}]{$ O $};
					\foreach \x/\y/\t in {1/2/M
						%			2/4/B,3/6/M,4/8/N
					}{\draw[dashed] (\x,0)|-(0,\y); \fill[red] (\x,\y) circle (1.5pt) node [right]{$\t$} ;}
					\clip (xmin,ymin) rectangle (xmax,ymax);
					\draw  plot[domain=xmin:xmax] (\x, {f(\x)});
					%============== Tên hàm số =====================
					\node[rotate=60] at (-1.1,-1.7) {$y=2x$};
					\node[rotate=0] at (2,-2.3) {Hình a};
					\draw[fill=black]  (0,0) circle (1.0pt);
				\end{tikzpicture}\hspace{1.6cm}
				\begin{tikzpicture}[line cap=butt,line join=miter,>=stealth,scale=1,font=\footnotesize]
					\tikzset{declare function={xmin=-2.5;xmax=4.5;ymin=-2.5;ymax=3;
							f(\x)=1/4*(\x);
						},
						smooth,samples=450
					}
					%============== Trục Ox =====================
					\draw[->] (xmin,0)--(xmax,0) node[shift={(0:7pt)}]{$ x $};
					\foreach  \x in {1,2,3,4}{\draw (\x,1.5pt)--(\x,-1.5pt) node [below]{$\x$};}	% Đánh số trục Ox
					\foreach  \x in {-2,-1}{\draw (\x,1.5pt)--(\x,-1.5pt) node [above]{$\x$};}		% Đánh số lệch trục Ox
					%============== Trục Oy =====================
					\draw[->] (0,ymin)--(0,ymax) node[shift={(90:7pt)}]{$ y $};
					\foreach  \y in {1,2}{\draw (1.5pt,\y)--(-1.5pt,\y) node [left]{$\y$};}			% Đánh số trục Oy
					\foreach  \y in {-2,-1}{\draw (1.5pt,\y)--(-1.5pt,\y) node [right]{$\y$};}		% Đánh số lệch trục Oy
					%=============================================
					\fill (0,0) node[shift={(125:7pt)}]{$ O $};
					\foreach \x/\y/\t in {4/1/N
						%			2/4/B,3/6/M,4/8/N
					}{\draw[dashed] (\x,0)|-(0,\y); \fill[red] (\x,\y) circle (1.5pt) node [above]{$\t$} ;}
					\clip (xmin,ymin) rectangle (xmax,ymax);
					\draw  plot[domain=xmin:xmax] (\x, {f(\x)});
					%============== Tên hàm số =====================
					\node[rotate=15] at (-2,-1) {$y=\dfrac{1}{4}x$};
					\node[rotate=0] at (2,-2.3) {Hình b};
					\draw[fill=black]  (0,0) circle (1.0pt);
				\end{tikzpicture}
			\end{center}.
		\end{enumerate}
	}
\end{bt}
\begin{bt}%[8D3H4-1]
	Tìm $a$ để hàm số $y=ax$ có đồ thị như trong hình sau
	\begin{center}
		\begin{tikzpicture}[line cap=butt,line join=miter,>=stealth,scale=0.8,font=\footnotesize]
			\tikzset{declare function={xmin=-2.5;xmax=2.5;ymin=-2.5;ymax=3.5;
					f(\x)=-3*(\x);
				},
				smooth,samples=450
			}
			%============== Trục Ox =====================
			\draw[->] (xmin,0)--(xmax,0) node[shift={(0:7pt)}]{$ x $};
			\foreach  \x in {-2,-1,1,2}{\draw (\x,1.5pt)--(\x,-1.5pt) node [below]{$\x$};}	% Đánh số trục Ox
			%		\foreach  \x in {-2,-1}{\draw (\x,1.5pt)--(\x,-1.5pt) node [above]{$\x$};}		% Đánh số lệch trục Ox
			%============== Trục Oy =====================
			\draw[->] (0,ymin)--(0,ymax) node[shift={(90:7pt)}]{$ y $};
			\foreach  \y in {-1,-2}{\draw (1.5pt,\y)--(-1.5pt,\y) node [left]{$\y$};}			% Đánh số trục Oy
			\foreach  \y in {1,2,3}{\draw (1.5pt,\y)--(-1.5pt,\y) node [right]{$\y$};}		% Đánh số lệch trục Oy
			%=============================================
			\fill (0,0) node[shift={(-125:7pt)}]{$ O $};
			\foreach \x/\y/\t in {-1/3/A
				%			2/4/B,3/6/M,4/8/N
			}{\draw[dashed] (\x,0)|-(0,\y); \fill[red] (\x,\y) circle (1.5pt) node [left]{$\t$} ;}
			\clip (xmin,ymin) rectangle (xmax,ymax);
			\draw  plot[domain=xmin:xmax] (\x, {f(\x)});
			%============== Tên hàm số =====================
			%			\node[rotate=15] at (-2,-1) {$y=\dfrac{1}{4}x$};
			\node[rotate=0] at (-1.8,-2.3) {Hình a};
			\draw[fill=black]  (0,0) circle (1.0pt);
		\end{tikzpicture}\hspace{2cm}
		\begin{tikzpicture}[line cap=butt,line join=miter,>=stealth,scale=0.8,font=\footnotesize]
			\tikzset{declare function={xmin=-2.5;xmax=2.5;ymin=-2.5;ymax=4.5;
					f(\x)=4*(\x);
				},
				smooth,samples=450
			}
			%============== Trục Ox =====================
			\draw[->] (xmin,0)--(xmax,0) node[shift={(0:7pt)}]{$ x $};
			\foreach  \x in {-2,-1,1,2}{\draw (\x,1.5pt)--(\x,-1.5pt) node [below]{$\x$};}			% Đánh số trục Ox
			%		\foreach  \x in {-2,-1}{\draw (\x,1.5pt)--(\x,-1.5pt) node [above]{$\x$};}		% Đánh số lệch trục Ox
			%============== Trục Oy =====================
			\draw[->] (0,ymin)--(0,ymax) node[shift={(90:7pt)}]{$ y $};
			\foreach  \y in {-1,-2}{\draw (1.5pt,\y)--(-1.5pt,\y) node [right]{$\y$};}			% Đánh số trục Oy
			\foreach  \y in {1,2,3,4}{\draw (1.5pt,\y)--(-1.5pt,\y) node [left]{$\y$};}			% Đánh số lệch trục Oy
			%=============================================
			\fill (0,0) node[shift={(125:7pt)}]{$ O $};
			\foreach \x/\y/\t in {1/4/B
				%			2/4/B,3/6/M,4/8/N
			}{\draw[dashed] (\x,0)|-(0,\y); \fill[red] (\x,\y) circle (1.5pt) node [right]{$\t$} ;}
			\clip (xmin,ymin) rectangle (xmax,ymax);
			\draw  plot[domain=xmin:xmax] (\x, {f(\x)});
			%============== Tên hàm số =====================
			%			\node[rotate=15] at (-2,-1) {$y=\dfrac{1}{4}x$};
			\node[rotate=0] at (1.8,-2.3) {Hình b};
			\draw[fill=black]  (0,0) circle (1.0pt);
		\end{tikzpicture}
	\end{center}
	\loigiai{
		\begin{enumerate}
			\item  Đường thẳng trong hình a) đi qua các điểm $O(0;0)$ đồ thị hàm số có dạng $y=ax$.\\
			Vì $A(-1;3)$ thuộc đồ thị hàm số nên ta có $3=-a\Rightarrow a=-3$.\\
			Vậy đồ thị ở hình a) là đồ thị của hàm số $y=-3x$.
			\item  Đường thẳng trong hình b) đi qua các điểm $O(0;0)$ nên đồ thị hàm số có dạng $y=ax$.\\ 
			Vì $B(1;4)$ thuộc đồ thị hàm số nên ta có $a=4$.\\		
			Vậy đồ thị ở hình b) là đồ thị của hàm số $y=4x$.
		\end{enumerate}
	}
\end{bt}
\begin{bt}%[8D3V4-1]
	Cho hàm số dạng $y=-2x+3$ có đồ thị là $(d_1)$.
	\begin{enumerate}
		\item Xác định hệ số góc của hàm số trên; góc tạo bởi đường thẳng $d_1$ và trục $Ox$ là loại góc gì? Vì sao?
		\item Vẽ đồ thị hàm số trên.
		\item Tìm hệ số $a$ của đường thẳng $(d_2)\colon y=ax-5$ để $(d_2)$ song song $(d_1)$.
	\end{enumerate}
	\loigiai{
		\begin{enumerate}
			\item Vì $y=-2x+3$ nên hệ số góc $k=-2$. Vì $k=-2<0$ nên góc tạo bởi đường thẳng $(d_1)$ và trục $Ox$ là loại góc tù.
			\item Ta có bảng giá trị
			\begin{center}
				\begin{tabular}{|c|c|c|}
					\hline $x$ & $0$ & $1$ \\
					\hline $y=-2x+3$ & $3$ & $1$\\
					\hline
				\end{tabular}
			\end{center}
			Đồ thị
			\begin{center}
				\begin{tikzpicture}[line join = round, line cap = round,>=stealth,font=\footnotesize,scale=1]
					\draw[step=1,gray, very thin,opacity=0.5]
					(-3.5,-1.5) grid (3.5,6.5);
					\def\xmin{-1.5} \def\xmax{3}
					\def\ymin{-1.5} \def\ymax{6}
					\def\fmot(#1){(-2)*(#1)+3}
					\draw[->] (\xmin,0)--(\xmax,0)node[below]{$x$};
					\draw[->] (0,\ymin)--(0,\ymax)node[right]{$y$};	
					\foreach \x in {-1,1,2}
					\draw (\x,.1)--(\x,-.1) node [below] {$\x$};	
					\foreach \y in {-1,1,3,4,5}
					\draw (.1,\y)--(-.1,\y) node [left] {$\y$};
					\draw (.1,2)--(-.1,2) node [above left] {$2$};
					\draw[fill=black]  (1,1) circle (1.0pt);
					\filldraw (0,0)node[below left]{$O$} circle (.05);
					\clip (\xmin,\ymin) rectangle (\xmax,\ymax);
					\draw[samples=100,domain=\xmin:\xmax,smooth,red] plot (\x, {\fmot(\x)});
					\draw[dashed] (1,0)--(1,1)--(0,1);
				\end{tikzpicture}
			\end{center}
			\item Vì $(d_2)\parallel (d_1)$ nên $a=-2$.
		\end{enumerate}
	}
\end{bt}

\begin{bt}%[8D3C4-5]
	Cho hai hàm số $ y=-\dfrac{1}{3}x+1$ ($d_1$); $y=x-3$ ($d_2$).
	\begin{enumerate}
		\item Vẽ đồ thị của hai hàm số trên cùng một mặt phẳng tọa độ.
		\item Gọi $A$, $B$ lần lượt là giao điểm của hai đường thẳng $d_1$, $d_2$ với trục tung và $C$  giao điểm của hai đường thẳng đó. Tính chu vi và diện tích của tam giác $ABC$ (đơn vị đo trên  các trục là {\it centimét}).
		\item Cho hàm số $y=2x-6$ ($d_3$). Chứng minh rằng $d_1$, $d_2$, $d_3$ đồng quy.
	\end{enumerate}
	\loigiai
	{
		\begin{enumerate}
			\item Bảng giá trị\\
			\begin{tabular}{|c|c|c|}
				\hline
				$x$ & $0$ & $3$ \\
				\hline
				$y=-\dfrac{1}{3}x+1$ & $1$ &$0$\\
				\hline
			\end{tabular}\hspace*{1cm}
			\begin{tabular}{|c|c|c|}
				\hline
				$x$ & $0$ & $3$\\
				\hline
				$y=x-3$ & $-3$ & $0$\\
				\hline
			\end{tabular}
			\\
			
			Đồ thị
			\begin{center}
				\begin{tikzpicture}[line join = round, line cap=round,>=stealth,font=\footnotesize,scale=0.7]
					\draw[->] (-1,0) -- (4.5,0) node[below,scale=0.7] {$x$};
					\draw[->] (0,-4) -- (0,3) node[left,scale=0.7] {$y$};
					\draw (0,0)node[above right,scale=0.7]{$O$};
					\draw[smooth,samples=100,domain=-1:4] plot(\x,{		-	(1/3)*(\x)+1	}) node[below,scale=0.7]{$(d_1)$};
					\draw[smooth,samples=100,domain=-1:4] plot(\x,{		(\x)-3	})node[above,scale=0.7]{$(d_2)$};
					\draw (3,0) node[below,scale=0.7]{$3$}
					(0,-3) node[right,scale=0.7]{$-3$}
					(0,1) node[above right,scale=0.7]{$1$};
					\draw[fill=black]  (3,0) circle (1.0pt) (0,1) circle (1.0pt) (0,-3) circle (1.0pt);
					
				\end{tikzpicture}
			\end{center}
			\item 
			\begin{itemize}
				\item Do $A$ là giao điểm của $(d_1)\colon y=-\dfrac{1}{3}x+1 $ với $Oy$ nên $x_A=0$ và $y_A=-\dfrac{1}{3}\cdot 0+1=1$, suy ra $A(0;1)$.
				\item  Do $B$ là giao điểm của $(d_2)\colon y=x-3$ với $Oy$ nên $x_B=0$ và $y_B=0-3=-3$, suy ra $B(0;-3)$.
			\end{itemize}
			Phương trình hoành độ giao điểm của $(d_1)$ và $(d_2)$	 là
			\begin{eqnarray*}
				-\dfrac{1}{3}x+1&=&x-3\\
				\dfrac{4}{3}x&=&4\\
				x&=&4.
			\end{eqnarray*}
			Với $x=4$ thì $y=0$, suy ra $C(3;0)$.
			\begin{center}
				\begin{tikzpicture}[line join = round, line cap=round,>=stealth,font=\footnotesize,scale=0.7]
					\draw[->] (-1,0) -- (4.5,0) node[below,scale=0.7] {$x$};
					\draw[->] (0,-4) -- (0,3) node[left,scale=0.7] {$y$};
					\draw (0,0)node[below left,scale=0.7]{$O$};
					\draw[smooth,samples=100,domain=-1:4] plot(\x,{		-	(1/3)*(\x)+1	}) node[below,scale=0.7]{$(d_1)$};
					\draw[smooth,samples=100,domain=-1:4] plot(\x,{		(\x)-3	})node[above,scale=0.7]{$(d_2)$};
					\draw (3,0) node[below,scale=0.7]{$C$}node[above,scale=0.7]{$3$}
					(0,-3) node[right,scale=0.7]{$B$}node[left,scale=0.7]{$-3$}
					(0,1) node[above right,scale=0.7]{$A$}node[below left,scale=0.7]{$1$};
					\draw[fill=black]  (3,0) circle (1.0pt) (0,1) circle (1.0pt) (0,-3) circle (1.0pt);
				\end{tikzpicture}	
			\end{center}
			Tam giác $ABC$ có đường cao $CO=3$ cm, $AB=4$ cm nên diện tích tam giác $ABC$ là
			$$S=\dfrac{1}{2}CO\cdot AB=\dfrac{1}{2}\cdot 3\cdot 4=6\ (\text{cm}^2).$$
			ÁP dụng định lí Py-ta-go cho hai tam giác vuông tại $O$ là $AOC$ và $BOC$, ta có
			$$\heva{&AC=\sqrt{OA^2+OC^2}=\sqrt{1^2+3^2}=\sqrt{10}\ (\text{cm})\\&BC=\sqrt{OC^2+OB^2}=\sqrt{3^2+3^2}=3\sqrt{2}\ (\text{cm}).}$$
			Vậy chu vi tam giác $ABC$ là
			$$P=AB+BC+CA=4+3\sqrt{2}+\sqrt{10}\ (\text{cm}).$$
			\item Ta có $0=2\cdot 3-6$ nên  $C(3;0)$ cũng thuộc $d_3\colon y=2x-6$.\\
			Do đó $d_1$, $d_2$, $d_3$ cùng đi qua $C$ hay $d_1$, $d_2$, $d_3$ đồng quy.
		\end{enumerate}	
	}
\end{bt} 
\begin{bt}%[8D3V4-5]
	Cho hàm số $y=(2a-1)x-a+2$.
	\begin{enumerate}
		\item Xác định $a$ để hàm số là hàm số bậc nhất;
		\item Xác định $a$ để đồ thị hàm số đi qua điểm $M(-1;2)$;
		\item Xác định $a$ để đồ thị hàm số cắt trục hoành tại điểm có hoành độ  bằng $1$;
		\item Xác định $a$ để đồ thị hàm số cắt trục tung tại điểm có tung độ  bằng $3$.
	\end{enumerate}	
	
	\loigiai{
		\begin{enumerate}
			\item Hàm số đã cho là hàm số bậc nhất khi $2a-1\ne 0$ hay $a\ne \dfrac{1}{2}$.
			\item Đồ thị hàm số đi qua điểm $M(-1;2)$ khi
			\begin{eqnarray*}
				2&=&(2a-1)\cdot (-1)-a+2\\
				2	&=&-2a+1-a+2\\
				2&=&-3a+3\\
				a&=&\dfrac{1}{3}.
			\end{eqnarray*}
			\item Đồ thị hàm số $y=(2a-1)x-a+2$ cắt trục hoành tại điểm có hoành độ bằng $1$ khi $(1;0)$ thuộc đồ thị, suy ra 
			\begin{eqnarray*}
				0	&=&(2a-1)\cdot 1-a+2\\
				0&=&a+1\\
				a&=&-1.
			\end{eqnarray*}
			\item Đồ thị hàm số $y=(2a-1)x-a+2$ cắt trục tung tại điểm có tung độ  bằng $3$ khi $(0;3)$ thuộc đồ thị, suy ra 
			\begin{eqnarray*}
				3	&=&(2a-1)\cdot 0-a+2\\
				a&=&-1.
			\end{eqnarray*}
		\end{enumerate}
	}
\end{bt}
\begin{bt}%[8D3H4-5]
	Cho hai hàm số $y=mx+3$ và $y=(2m-1)x+5$. Tìm các giá trị của $m$ để
	\begin{enumerate}
		\item Đồ thị hai hàm số là hai đường thẳng song song với nhau;
		\item Đồ thị hai hàm số là hai đường thẳng cắt nhau.
	\end{enumerate}	
	
	\loigiai{
		\begin{enumerate}
			\item  Đồ thị hai hàm số đã cho song song với nhau khi
			$$\heva{&m=2m-1\\&3\ne 5}\text{ suy ra } m=1.$$
			\item  Đồ thị hai hàm số đã cho cắt nhau khi
			$$m\ne2m-1\text{ suy ra } m\ne1.$$
		\end{enumerate}	
	}
\end{bt}
\begin{bt}%[8D3C4-5]
	Cho hàm số $(d)\colon y=x-2$.
	\begin{enumerate}		
		\item Tìm $a$, $b$ để $y=a x+b$ để đường thẳng này đi qua $A(1 ;-5)$ và song song với $(d)$.
		\item Tìm $m$ để $\left(d'\right)\colon  y=(m-3) x+5$ với $m \neq 3$ cắt $(d)$ tại một điểm có tung độ bằng $1$.
		\item Tìm $m$ để đường thẳng $y=(3-m) x-m+5$ cắt đường thẳng $d$ tại 1 điểm trên trục hoành.
	\end{enumerate}	
	\loigiai{
		\begin{enumerate}
			\item Đường thẳng $y=ax+b$ song song với đường thẳng $(d)$ khi $a=1$ và $b\neq -2$.\\ Khi đó $y=x+b$.\\
			Vì đường thẳng $y=x+b$ đi qua điểm $A(1 ;-5)$ nên $-5=1+b$ suy ra $b=-6$.
			\item Vì $(d')$ cắt $(d)$ tại một điểm có tung độ bằng $1$ nên $1=x-2$ suy ra $x=3$, ta được giao điểm của $(d)$ và $(d')$ là $M(3;1)$.\\
			Vì $(d')$ đi qua điểm $M(3;1)$ nên $1=(m-3)\cdot 3+5$ suy ra $m-3=\dfrac{4}{3}$. Do đó $m=\dfrac{13}{3}$.
			\item Đường thẳng $y=(3-m) x-m+5$ cắt đường thẳng $d$ tại 1 điểm trên trục hoành nên $y=0$.\\
			Với $y=0$ thì $0=x-2$ suy ra $x=2$, ta được giao điểm của đồ thị với trục $Ox$ là $B(2;0)$.\\
			Vì đường thẳng $y=(3-m) x-m+5$ đi qua $B(2;0)$ thì 
			\begin{eqnarray*}
				0&=&(m-3)\cdot 2-m+5\\
				0&=&2m-6-m+5\\
				0&=&m-1\\
				m&=&1.
			\end{eqnarray*}
		\end{enumerate}		
	}
	
\end{bt}
\begin{bt}%[8D3C4-5]
	Trong hệ tọa độ $Oxy$ cho đường thẳng $(d)\colon y=2 x+m-5$.
	\begin{enumerate}
		\item Tính diện tích tam giác tạo bởi đường thẳng $(d)$ với hai trục tọa độ khi $m=4$.
		\item Tìm giá trị $m$ để đường thẳng $(d)$ song song với đường thẳng $y=\left(m^2+1\right) x-4$.
		\item Tìm giá trị $m$ để đường thẳng $(d)$ đồng quy với hai đường thẳng $y=4 x-3$ và $y=3 x+4$
	\end{enumerate}	
	\loigiai{
		\begin{enumerate}
			\item Khi $m=4$ thì $y=2x-1$.\\
			Cho $x=0$ thì $y=-1$, ta được giao điểm của đồ thị với trục $Oy$ là $A(0;-1)$. Do đó độ dài đoạn $OA$ bằng $1$.\\
			Cho $y=0$ thì $x=\dfrac{1}{2}$, ta được giao điểm của đồ thị với trục $Ox$ là $B\left(\dfrac{1}{2};0\right)$. Do đó độ dài đoạn $OB$ bằng $\dfrac{1}{2}$.\\
			Khi đó diện tích tam giác tạo bởi đường thẳng $(d)$ với hai trục tọa độ bằng $$\dfrac{1}{2}\cdot OA\cdot OB=\dfrac{1}{2}\cdot 1\cdot \dfrac{1}{2}=\dfrac{1}{4}.$$
			\item Đường thẳng $d$ song song với đường thẳng $y=\left(m^2+1\right) x-4$ khi $2=m^2+1$ và $m-5\neq 4$.\\ Hay $m^2=1$ và $m\neq 9$. Do đó $m=1$ hoặc $m=-1$.\\
			\item Phương trình hoành độ giao điểm của hai đường thẳng  $y=4 x-3$ và $y=3 x+4$ là $4x-3=3x+4$ suy ra $x=7$.\\
			Với $x=7 $ thì $y=25$, ta được giao điểm của đồ thị với trục $Oy$ là $A(7;25)$.\\
			Ba đường thẳng đồng quy nên đường thẳng $(d)$ đi qua $A$.\\
			Khi đó $25=2\cdot 7+m-5$ suy ra $m=16$.
		\end{enumerate}	
	}
\end{bt}
\begin{bt}%[8D3V3-4] 
	Anh Bình làm nhân viên bán hàng cho một công ty và được trả lương theo hình thức sau. Mỗi tháng anh được nhận mức lương cơ bản là $3\,000\,000$ đồng và cứ bán được một sản phẩm anh được nhận $10\,000$ đồng. Gọi $x$ (sản phẩm) là số sản phẩm anh Bình bán được trong một tháng, $y$ (đồng) là tổng số tiền lương anh Bình nhận được trong tháng đó.\\
	Em hãy viết biểu thức biểu thị mối liên hệ giữa hai đại lượng $x$ và $y$ nêu trên.
	\loigiai{
		Biểu thức biểu thị mối liên hệ giữa hai đại lượng $x$ và $y$ là $y=10\,000x+3\,000\,000$.
	}
\end{bt}

\begin{bt}%[8D3V3-4] 
	Một cửa hàng bán giày thể thao nhập một đơn hàng và ngày đầu tiên cửa hàng nhanh chóng bán được $40$ đôi giày. Hôm sau mở cửa, cửa hàng tiếp tục bán giày thể thao; và số đôi giày thể thao bình quân mỗi ngày cửa hàng bán ra được tính theo công thức: $G=kx+m$ và được biểu diễn minh họa bởi biểu đồ bên; trong đó $G$ là số đôi giày cửa hàng bán được và $x$ là số ngày bán.
	\begin{enumerate}
		\item Dựa vào hình bên, xác định hệ số $k$ và $m$.
		\item Nếu lúc đầu cửa hàng nhập về $250$ đôi giày thể thao thì sau $15$ ngày cửa hàng còn lại bao nhiêu đôi?
	\end{enumerate}
	\begin{center}
		\begin{tikzpicture}[>=stealth, line join = round, line cap = round, font=\footnotesize, scale=1]
			\draw[->] (-1,0)--(6,0) node[below]{$x$ (ngày)};
			\draw[->] (0,-1)--(0,3) node[above]{$G$ (đôi)};
			\foreach \x in {1,2,3,4}
			\draw[shift={(\x,0)},color=black] (0pt,2pt) -- (0pt,-2pt) 
			node[below] { $\x$};
			\foreach \y in {40,80}
			\draw[shift={(0,\y/40)},color=black] (2pt,0pt) -- (-2pt,0pt) 
			node[left] {\normalsize $\y$};
			\draw[dashed] (4,0)--(4,2)--(0,2);
			\draw[fill=black] (4,2) circle (2pt);
			\draw (0,1)--(5,2.25);
		\end{tikzpicture} 
	\end{center}
	\loigiai{
	\begin{enumerate}
		\item Từ hình vẽ ta thấy $(0;40)$ thuộc đồ thị hàm số $G=kx+m$ nên ta có $m=40$ suy ra $G=kx+40$.\\
		  Lại có $(4;80)$ thuộc đồ thị hàm số $G=kx+40$ nên ta có $4k+40=80$ nên $k=10$.
		\item Theo câu a ta có $G=10x+40$.\\
		Sau $15$ của hàng ngày bán được $G=10\cdot 15+40=190$ đôi.\\
		Vậy sau $15$ ngày cửa hàng còn lại $250-190=60$ đôi
		
	\end{enumerate}	
	}
\end{bt}

\begin{bt}%[8D3V3-4] 
	Một lò xo có chiều dài ban đầu khi chưa treo vật nặng là $10$ cm. Cho biết khi treo thêm vào lò xo một vật nặng $1$ kg thì chiều dài của lò xo tăng thêm $3$ cm. Tính chiều dài $y$ (cm) của lò xo theo khối lượng $x$ (kg) của vật.
	\loigiai{
		Chiều dài $y$ (cm) của lò xo theo khối lượng $x$ (kg) của vật $y=ax+b$ $(*)$\\
		Thay $x=0$ và $y=10$ vào $(*)$ ta được $b=10$.\\
		Thay $x=1$ và $y=10+3=13$ vào $(*)$ ta được $13=a\cdot 1+10$ suy ra $a=3$.\\
		Vậy $y=3x+10$.	
	}
\end{bt}
\begin{bt}%[8D3V3-4] 
	Quãng đường của một chiếc xe chạy từ $A$ đến $B$ cách nhau $235$ km được xác định bởi hàm số $s=50\cdot t+10$, trong đó $s$ (km) là quãng đường của xe chạy được và $t$ (giờ) là thời gian đi của xe.
	\begin{enumerate}
		\item Hỏi sau $3$ giờ xuất phát thì xe cách $A$ bao nhiêu kilomet?
		\item Thời gian xe chạy hết quãng đường $AB$ là bao nhiêu giờ?
	\end{enumerate}
	\loigiai{
		\begin{enumerate}
			\item 
			$s=50\cdot 3+10=160$.\\
			Vậy sau $3$ giờ xuất phát thì xe cách $A$ là $160$ km.
			\item $235=50\cdot t+10$ ta được $t=4{,}5$.\\
			Vậy thời gian xe chạy hết quãng đường $AB$ là $4{,}5$ giờ.
		\end{enumerate}
	}
\end{bt}
\begin{bt}%[8D3V4-6]
	Cho hàm số $y=-3x+1$ có đồ thị $d_1$ và $y=\dfrac{1}{3}x-2$ có đồ thị $d_2$.
	\begin{enumerate}
		\item Vẽ $d_1$ và $d_2$ trên cùng một mặt phẳng tọa độ.
		\item Xác định hàm số có đồ thị $d_3$, biết $d_3$ là đường thẳng song song với $d_1$ và đi qua điểm $A(2;3)$.
	\end{enumerate}
	\loigiai{
		\begin{enumerate}
			\item 
			\immini{
				Bảng giá trị của $d_1$ 
				\begin{center}
					\begin{tabular}{|c|c|c|}
						\hline
						$x$ & $0$  & $1$  \\
						\hline
						$y=-3x+1$ & $1$ &$-2$  \\
						\hline
					\end{tabular}
				\end{center}
				Bảng giá trị của $d_2$ 
				\begin{center}
					\begin{tabular}{|c|c|c|}
						\hline
						$x$ & $0$  & $3$  \\
						\hline
						$y=\dfrac{1}{3}x-2$ & $-2$ &$-1$  \\
						\hline
					\end{tabular}
			\end{center}}
			{\begin{tikzpicture}[scale=0.7, font=\footnotesize, line join=round, line cap=round, >=stealth]
					\def\xmax{4} \def\ymax{4}
					%			
					\draw[->] (-3,0)--(\xmax,0) node [below]{$x$};
					\draw[->] (0,-3)--(0,\ymax) node [left]{$y$};
					\fill (0,0) circle(1pt) node[below left]{$O$};
					\path (1,-3) node[below]{$y=-3x+1$};
					\path (4,-1) node[below]{$y=\dfrac{1}{3}x-2$};
					\begin{scope}
						\clip (-3,-3) rectangle (4,4);
						\draw[samples=100,smooth,variable=\x] plot (\x,{-3*(\x)+1});
						\draw[samples=100,smooth,variable=\x] plot (\x,{(1/3)*(\x)-2});
					\end{scope}
					\foreach \p/\r in {-3/90,-2/90,-1/90,1/90,2/90,3/90}
					\fill (\p,0) circle (1pt) node[shift={(\r:2mm)}]{$\p$};
					\foreach \p/\r in {1/180,2/150,3/180}
					\fill (0,\p) circle (1pt) node[shift={(\r:2mm)}]{$\p$};
					\foreach \p/\r in {-1/180,-2/160}
					\fill (0,\p) circle (1pt) node[shift={(\r:3mm)}]{$\p$};
					\fill (1,-2) circle (1pt);
					\fill (3,-1) circle (1pt);
					\draw[dashed] 
					(3,0)--(3,-1)--(0,-1)
					(1,0)--(1,-2)--(0,-2)
					;
			\end{tikzpicture}}	
			\item $d_3$ có phương trình $y=ax+b$.\\
			Vì $d_3\parallel d_1$ nên $a=-3$.\\
			Vì $d_3$ đi qua điểm $A(2;3)$ nên $3=(-3)\cdot 2+b$ ta được $b=9$.\\
			Vậy $y=-3x+9$.		
		\end{enumerate}
	}
\end{bt}
% In đáp án trắc nghiệm
\Closesolutionfile{ans}
\indapan{6}{ans/ans-8C5-OTC}