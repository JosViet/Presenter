\section*{BÀI TẬP CUỐI CHƯƠNG 3}

\Opensolutionfile{ans}[ans/ans-8C3-OTC]
\subsection{Câu hỏi trắc nghiệm}

%%%====Câu 1
\begin{ex}%[Dự án EX-8-Đề Cương Toán 8]%[Thầy Nguyễn Đức Tuấn Anh]%[8H2N1-1]
	\immini{Áp dụng định lý Pythagore vào tam giác $ABC$ vuông tại $A$. Chọn câu trả lời đúng
		\choice
		{$AB^2 = BC^2 + AC^2$}
		{\True $BC^2 = AB^2 + AC^2$}
		{$AC^2 = BC^2 + AB^2$}
		{$BC^2 = AB^2 - AC^2$}
	}{
		\begin{tikzpicture}[scale=1, font=\footnotesize, line join=round, line cap=round, >=stealth]
			\path 
			(0,0) coordinate (A)
			(0,2.5) coordinate (B)
			(4,0) coordinate (C)
			;
			\draw 
			(A) -- (B) -- (C) -- (A)
			;
			\fill 
			(A) circle (1pt) node[below]{$A$}
			(B) circle (1pt) node[left]{$B$}
			(C) circle (1pt) node[below]{$C$}
			;
			\draw pic[draw,angle radius=2mm]{right angle = C--A--B};
		\end{tikzpicture}
	}
	\loigiai{
		Xét $\triangle ABC$ vuông tại $A$, theo định lý Pythagore ta có $BC^2 = AB^2 + AC^2$.
	}
\end{ex}

%%%====Câu 2
\begin{ex}%[Dự án EX-8-Đề Cương Toán 8]%[Thầy Nguyễn Đức Tuấn Anh]%[8H2N1-1]
	Trong các phát biểu sau, phát biểu nào là định lý Pythagore?
	\choice
	{Nếu một tam giác có bình phương cạnh huyền bằng hiệu bình phương của hai cạnh góc vuông thì tam giác đó là tam giác vuông}
	{Nếu một tam giác có một cạnh bằng tổng của hai cạnh còn lại thì tam giác đó là tam giác vuông}
	{\True Trong một tam giác vuông, bình phương của cạnh huyền bằng tổng bình phương của hai cạnh góc vuông}
	{Trong một tam giác vuông, bình phương một cạnh bằng tổng bình phương của hai cạnh còn lại}
	\loigiai{
		Định lý Pythagore \lq\lq Trong một tam giác vuông, bình phương của cạnh huyền bằng tổng bình phương của hai cạnh góc vuông\rq\rq.
	}
\end{ex}

%%%====Câu 3
\begin{ex}%[Dự án EX-8-Đề Cương Toán 8]%[Thầy Nguyễn Đức Tuấn Anh]%[8H2N6-1]
	Khẳng định nào sau đây là \textbf{sai}?
	\choice
	{Hình vuông có hai đường chéo bằng nhau}
	{Hình vuông có hai đường chéo vuông góc}
	{Hình thoi có hai đường chéo vuông góc}
	{\True Hình thoi có hai đường chéo bằng nhau}
	\loigiai{
	Trong hình thoi, chưa đủ điều kiện để hai đường chéo bằng nhau.
	}
\end{ex}

%%%====Câu 4
\begin{ex}%[Dự án EX-8-Đề Cương Toán 8]%[Thầy Nguyễn Đức Tuấn Anh]%[8H2N7-1]
	Hình nào sau đây là hình vuông?
	\choice
	{Hình thang cân có một góc vuông}
	{\True Hình chữ nhật có hai cạnh kề bằng nhau}
	{Tứ giác có $3$ góc vuông}
	{Hình bình hành có một góc vuông}
	\loigiai{
		Hình chữ nhật có hai cạnh kề bằng nhau là hình vuông.
	}
\end{ex}

%%%====Câu 5
\begin{ex}%[Dự án EX-8-Đề Cương Toán 8]%[Thầy Nguyễn Đức Tuấn Anh]%[8H2N3-1]
	Hình thang cân là hình thang
	\choice
	{có hai đường chéo vuông góc với nhau}
	{\True có hai đường chéo bằng nhau}
	{có hai đường chéo cắt nhau tại trung điểm mỗi đường}
	{có hai đường chéo cùng vuông góc hai đáy}
	\loigiai{
		Hình thang có hai đường chéo bằng nhau là hình thang cân.
	}
\end{ex}

%%%====Câu 6
\begin{ex}%[Dự án EX-8-Đề Cương Toán 8]%[Thầy Nguyễn Đức Tuấn Anh]%[8H2N7-2]
	Trong các khẳng định sau, khẳng định nào đúng?
	\choice
	{Hình chữ nhật có hai đường chéo bằng nhau là hình vuông}
	{Hình thoi có hai đường chéo vuông góc là hình vuông}
	{\True Hình thoi có một góc vuông là hình vuông}
	{Hình chữ nhật có một góc vuông là hình vuông}
	\loigiai{
		\lq\lq Hình thoi có một góc vuông là hình vuông\rq\rq\ là khẳng định đúng.
	}
\end{ex}

%%%====Câu 7
\begin{ex}%[Dự án EX-8-Đề Cương Toán 8]%[Thầy Nguyễn Đức Tuấn Anh]%[8H2H1-2]
	Bộ ba số đo nào sau đây là độ dài các cạnh của một tam giác vuông?  
	\choice
	{$3$ cm, $4$ cm, $6$ cm}
	{\True $5$ cm, $12$ cm, $13$ cm}
	{$4$ cm, $5$ cm, $6$ cm}
	{$7$ cm, $10$ cm, $12$ cm}
	\loigiai{ 
		Ta có $5^2+12^2=169$ và $13^2=169$ nên $5^2+12^2=13^2$.\\
		Theo định lí Pythagore đảo $5$ cm, $12$ cm, $13$ cm là độ dài các cạnh của một tam giác vuông.
	}
\end{ex}

%%%====Câu 8
\begin{ex}%[Dự án EX-8-Đề Cương Toán 8]%[Thầy Nguyễn Đức Tuấn Anh]%[8H2H2-2]
	Cho tứ giác $ABCD$ có $\widehat{A} = 100^\circ$, $\widehat{B} = 70^\circ$, $\widehat{D} = 50^\circ$. Số đo góc $C$ bằng
	\choice
	{$ 120^\circ$}
	{$ 380^\circ$}
	{$ 220^\circ$}
	{\True $140^\circ$}    
	\loigiai{
		Tổng các góc trong tứ giác là $360^\circ$. Do đó, $\widehat{C} = 360^\circ - (100^\circ + 70^\circ + 50^\circ) = 140^\circ$.
	}
\end{ex}

%%%====Câu 9
\begin{ex}%[Dự án EX-8-Đề Cương Toán 8]%[Thầy Nguyễn Đức Tuấn Anh]%[8H2N4-1]
	\immini{Cho hình bình hành $TQRS$. Chọn câu trả lời \textbf{sai}.
		\choice
		{$TQ=RS$; $QR=TS$}
		{$TQ \parallel RS$; $QR \parallel TS$}
		{$\widehat{T}=\widehat{R}$; $\widehat{Q}=\widehat{S}$}
		{\True $QS \perp RT$ tại $O$}}{
		\begin{tikzpicture}[line join=round, line cap=round,>=stealth,font=\footnotesize,scale=1]
			\path
			(0,0) coordinate (S)
			(0:3) coordinate (R)
			(60:2) coordinate (T)
			($(T)+(R)-(S)$) coordinate (Q)
			($(T)!0.5!(R)$) coordinate (O)
			;
			\draw (S)--(R)--(Q)--(T)--(S)--(Q) (T)--(R);
			\foreach \i/\g in {O/90,T/90,Q/90,S/-90,R/-90}{\draw[fill=black](\i) circle (1pt) ($(\i)+(\g:3mm)$) node[scale=1]{$\i$};}
		\end{tikzpicture}
	}
	\loigiai{
		Do tứ giác $TQRS$ là hình bình hành chứ không phải hình thoi nên không có tính chất hai đường chéo vuông góc.}
\end{ex}

%%%====Câu 10
\begin{ex}%[Dự án EX-8-Đề Cương Toán 8]%[Thầy Nguyễn Đức Tuấn Anh]%[8H2N5-1]
	Cho các khẳng định sau
	\begin{enumerate}
		\item Hình bình hành có hai đường chéo bằng nhau.
		\item Hình thang có hai cạnh bên bằng nhau là hình thang cân.
		\item Trong hình chữ nhật, giao của hai đường chéo cách đều bốn đỉnh của hình chữ nhật.
		\item Hình bình hành có hai cạnh kề bằng nhau là hình chữ nhật.
	\end{enumerate}
	Số các khẳng định đúng là
	\choice
	{$0$}
	{\True $1$}
	{$2$}
	{$3$}
	\loigiai{
		\begin{enumerate}
			\item \textbf{Sai.} Hình bình hành chưa chắc có hai đường chéo bằng nhau.
			\item \textbf{Sai.} Hình thang có hai cạnh bên bằng nhau có thể là hình thang cân hoặc hình bình hành.
			\item \textbf{Đúng.} Trong hình chữ nhật, giao của hai đường chéo là trung điểm của mỗi đường. Mà hai đường chéo bằng nhau. Do đó giao điểm cách đều bốn đỉnh.
			\item \textbf{Sai.} Hình bình hành có hai cạnh kề bằng nhau là hình thoi.
		\end{enumerate}
		Số khẳng định \textbf{đúng} là $1$.
	}
\end{ex}

%%%====Câu 11 
\begin{ex}%[Dự án EX-8-Đề Cương Toán 8]%[Thầy Nguyễn Đức Tuấn Anh]%[8H2V6-3]
	Cho hình thoi $ABCD$ tâm $O$. Độ dài $AC = 8$ cm, $BD = 10$ cm. Độ dài cạnh của hình thoi là
	\choice
	{$3$ cm}
	{$4$ cm}
	{$5$ cm}
	{\True $\sqrt{41}$ cm}
	\loigiai{
	\begin{center}
	\begin{tikzpicture}[scale=1, font=\footnotesize, line join=round, line cap=round, >=stealth]
		\tikzset{
			markx/.pic={\draw (45:.1)--(-135:.1) (-45:.1)--(135:.1);},
			markl/.pic={\draw (90:.1)--(-90:.1);},
			markll/.pic={\draw[shift={(180:.02)}] (90:.1)--(-90:.1); 		\draw[shift={(0:.02)}] (90:.1)--(-90:.1);},}
		\path 
		(0:0) coordinate (O)
		(90:2) coordinate (A)
		(180:2.5) coordinate (B)
		(-90:2) coordinate (C)
		(0:2.5) coordinate (D)
		;
		\draw[]
		(O)--(A)pic[pos =.5,sloped]{markl}
		(O)--(C)pic[pos =.5,sloped]{markl}
		(O)--(B)pic[pos =.5,sloped]{markll}
		(O)--(D)pic[pos =.5,sloped]{markll}
		(A)--(B)--(C)--(D)--(A);
		 \draw pic[draw,angle radius=2mm] {right angle = A--O--B};
		\foreach \x /\goc in {A/90,B/180,C/-90,D/0,O/-45} \fill (\x) circle (1pt) ($(\x)+(\goc:2.5mm)$) node {$\x$};
	\end{tikzpicture}
	\end{center}
		Vì $ABCD$ là hình thoi tâm $O$ nên hai đường chéo $AC$ và $BD$ cắt nhau tại trung điểm mỗi đường, $AC\perp BD$.\\
		Do đó
		$OA = \dfrac{AC}{2} = 4$ cm và $OB = \dfrac{BD}{2} = 5$ cm.\\
		Bên cạnh đó, $\triangle OAB$ vuông tại $O$ nên áp dụng định lí Pythagore, ta có
		$$AB = \sqrt{OA^2 + OB^2} = \sqrt{41}\text{ (cm)}.$$
	}
\end{ex}

%%%====Câu 12 
\begin{ex}%[Dự án EX-8-Đề Cương Toán 8]%[Thầy Nguyễn Đức Tuấn Anh]%[8H2V7-4]
	Cho tam giác $ABC$ vuông tại $A$. Gọi $M$, $N$, $P$ lần lượt là trung điểm của $AB$, $BC$, $CA$. Tam giác $ABC$ cần có thêm điều kiện gì để hình chữ nhật $AMNP$ là hình vuông.
	\choice
	{$AB = \dfrac{1}{2}AC$}
	{\True $AB = AC$}
	{$AC = \dfrac{1}{2}AB$}
	{$\widehat{B} = 60^\circ$}
	\loigiai{
	\begin{center}
	\begin{tikzpicture}[scale=1, font=\footnotesize, line join=round, line cap=round, >=stealth]
		\tikzset{
			markx/.pic={\draw (45:.1)--(-135:.1) (-45:.1)--(135:.1);},
			markl/.pic={\draw (90:.1)--(-90:.1);},
			markll/.pic={\draw[shift={(180:.02)}] (90:.1)--(-90:.1); 		\draw[shift={(0:.02)}] (90:.1)--(-90:.1);},}
		\path 
		(0:0) coordinate (A)
		(90:4) coordinate (B)
		(0:4) coordinate (C)
		;
		\foreach \x/\y/\z in {A/B/M,B/C/N,C/A/P}{\path ($(\x)!.5!(\y)$) coordinate (\z);}
		\draw[]
			(A)--(M)pic[pos =.5,sloped]{markl}
			(M)--(B)pic[pos =.5,sloped]{markl}
			(B)--(N)pic[pos =.5,sloped]{markll}
			(N)--(C)pic[pos =.5,sloped]{markll}
			(C)--(P)pic[pos =.5,sloped]{markx}
			(P)--(A)pic[pos =.5,sloped]{markx}
			(M)--(N)--(P)--(M);
		\foreach \x /\goc in {A/-150,B/90,C/0,M/180,N/60,P/-90} \fill (\x) circle (1pt) ($(\x)+(\goc:2.5mm)$) node {$\x$};
	\end{tikzpicture}
	\end{center}
		Để hình chữ nhật $AMNP$ là hình vuông thì $AM = AP$.\\
		Mà $AM = \dfrac{1}{2}AB$, $AP = \dfrac{1}{2}AC$ nên $AM = AP$ khi $AB = AC$.\\
		Vậy nếu tam giác $ABC$ vuông cân tại $A$ thì hình chữ nhật $AMNP$ là hình vuông.
	}
\end{ex}

\subsection{Bài tập tự luận}
%%%=======Bài 1
\begin{bt}%[Dự án EX-8-Đề Cương Toán 8]%[Thầy Nguyễn Đức Tuấn Anh]%[8H2N1-1]
	\begin{enumerate}
		\item Cho tam giác $ABC$ vuông tại $A$ có $AB=3$ cm, $AC=4$ cm. Tính độ dài cạnh huyền $BC$ của tam giác vuông đó.
		\item Cho tam giác vuông $MNP$ có cạnh huyền $NP= 10$ dm và $MN=6$ dm. Tính độ dài cạnh $MP$.
	\end{enumerate}
	\loigiai{
		\begin{enumerate}
			\item Áp dụng định lí Pythagore vào tam giác $ABC$ vuông tại $A$, ta có
			\begin{eqnarray*}
				BC^2&=&AB^2+AC^2\\
				BC^2&=&3^2+4^2\\
				BC^2&=&9+16\\
				BC^2&=&25\\
				BC&=&\sqrt{25}=5.
			\end{eqnarray*}
			Vậy $BC=5 \mathrm{~cm}$.
			\item Vì $\triangle MNP$ vuông có cạnh huyền là cạnh $NP$ nên $\triangle MNP$ vuông tại $M$.\\
			Áp dụng định lí Pythagore vào tam giác $MNP$ vuông tại $M$, ta có
			\begin{eqnarray*}
				NP^2 &=& MN^2+MP^2\\ 
				MP^2&=&NP^2-MN^2\\
				MP^2&=&10^2-6^2\\
				MP^2&=&64\\
				MP&=&\sqrt{64}=8.
			\end{eqnarray*}
			Vậy $MP=8$ dm.
		\end{enumerate}	
	}
\end{bt}

%%%=======Bài 2
\begin{bt}%[Dự án EX-8-Đề Cương Toán 8]%[Thầy Nguyễn Đức Tuấn Anh]%[8H2N2-2]
	Tìm $x$ trong mỗi tứ giác sau
	\begin{center}
	\begin{multicols}{2}
		\begin{tikzpicture}[scale=0.8, font=\footnotesize, line join=round, line 
			cap=round, >=stealth]
			\def\a{6}
			\path (0,0) coordinate (Q)  
			($(Q)+(10:3)$)coordinate (R)
			($(R)+(50:3.5)$)coordinate (x) 
			($(Q)+(80:3)$)coordinate (P)
			($(P)+(-20:3.5)$)coordinate (y) 
			;	
			\path (intersection of  R--x and P--y) coordinate (S);
			;	
			\foreach \d/\g in {Q/180,P/180,R/0,S/90}	
			\path[draw] (\d) circle(1pt) + (\g:9pt) node {$\d$};
			\draw [](P)--(S)--(R)--(Q)--cycle ;
			\draw (1,-1)node[below]{a)};
			\draw [] (R)node[above left]{$2x$}
			(Q)node[above right]{$x$};
			\path pic["\scriptsize$70^\circ$", angle eccentricity=2,,angle radius=8pt]{angle= P--S--R};
			\path pic["\scriptsize$80^\circ$", angle eccentricity=2,angle radius=8pt]{angle= Q--P--S};
		\end{tikzpicture}
		
		\begin{tikzpicture}[scale=0.8, font=\footnotesize, line join=round, line 
			cap=round, >=stealth]
			\def\a{6}
			\path (0,0) coordinate (H)  
			($(H)+(0:3)$)coordinate (G)
			($(G)+(90:3.5)$)coordinate (x) 
			($(H)+(81:2)$)coordinate (E)
			($(E)+(0:3.5)$)coordinate (y) 
			;	
			\path (intersection of  G--x and E--y) coordinate (F);
			;	
			\foreach \d/\g in {H/180,E/180,G/45,F/60}	
			\path[draw] (\d) circle(1pt) + (\g:9pt) node {$\d$};
			\draw [] (H)--(E)--(F)--(G)--cycle ;
			\draw (1.5,-0.8)node[below]{b)};
			\draw []
			(E)node[below right]{$99^\circ$}
			(H)node[above right]{$x$};
			\foreach \i/\j/\k in {E/F/G}{
				\draw[black,line width = 1pt] ($(\j)!6pt!(\i)$)--($(\j)!6pt!(\k)-(\j)+(\j)!6pt!(\i)$)--($(\j)!6pt!(\k)$);}
			\foreach \i/\j/\k in {F/G/H}{
				\draw[black,line width = 1pt] ($(\j)!6pt!(\i)$)--($(\j)!6pt!(\k)-(\j)+(\j)!6pt!(\i)$)--($(\j)!6pt!(\k)$);}
		\end{tikzpicture}
	\end{multicols}
	\end{center}
	\loigiai{
		Do tổng số đo bốn góc của một tứ giác bằng $360^\circ$ nên ta có
		\begin{enumerate}
			\item Trong tứ giác $PQRS$: $x+2x+80^\circ+70^\circ=360^\circ$, suy ra $x=70^\circ$.
			\item Trong tứ giác $EFGH$: $x+99^\circ+90^\circ+90^\circ=360^\circ$, suy ra $x=81^\circ$.
		\end{enumerate}
	}
\end{bt}

%%%=======Bài 3
\begin{bt}%[Dự án EX-8-Đề Cương Toán 8]%[Thầy Nguyễn Đức Tuấn Anh]%[8H2H1-2]
	Tìm tam giác vuông trong các tam giác sau
	\begin{enumerate}
		\item Tam giác $ABC$ có $AB=3$ cm, $AC=4$ cm, $BC=4$ cm;
		\item Tam giác $OHK$ có $OH=6$ dm, $OK=8$ dm, $KH=10$ dm;
		\item Tam giác $MNP$ có $MN=20$ m, $NP=12$ m, $PM=16$ m.
	\end{enumerate}
	\loigiai{
		\begin{enumerate}
			\item Ta có 
			\begin{itemize}
				\item $BC^2= 4^2=16$.
				\item $AB^2+AC^2=3^2+4^2=25$.
			\end{itemize}
			Suy ra $BC^2\neq AB^2+AC^2$.\\
			Vậy $\triangle EFK$ không phải là tam giác vuông.
			\item Ta có 
			\begin{itemize}
				\item $KH^2= 10^2=100$.
				\item $OH^2+OK^2=6^2+8^2=100$.
			\end{itemize}
			Suy ra $KH^2=OH^2+OK^2$.\\
			Vậy $\triangle OHK$ vuông tại $O$ (định lí Pythagore đảo).
			\item Ta có 
			\begin{itemize}
				\item $MN^2= 20^2=400$.
				\item $PM^2+PN^2=16^2+12^2=400$.
			\end{itemize}
			Suy ra $MN^2=PM^2+PN^2$.\\
			Vậy $\triangle PMN$ vuông tại $P$ (định lí Pythagore đảo).
		\end{enumerate}
	}	
\end{bt}

%%%=======Bài 4
\begin{bt}%[Dự án EX-8-Đề Cương Toán 8]%[Thầy Nguyễn Đức Tuấn Anh]%[8H2H1-4]
	\immini{Một chiếc xuồng máy qua sông từ vị trí $B$ hướng tới vị trí $A$. Tuy nhiên do nước chảy nên khi qua tới bờ, thuyền tới vị trí $C$ cách $A$ một khoảng là $22$ m. Trong suốt quá trình qua sông, vận tốc chuyển động của xuồng là $v=2$ m/s. Biết độ dài quãng đường xuồng đi được cho bởi công thức $s=vt$, với $t$ là thời gian. Tính khoảng cách $AB$ giữa hai bờ sông biết rằng để đi từ $B$ đến $C$ thì xuồng mất khoảng thời gian là $61$ giây.}{
		\begin{tikzpicture}[line join=round, line cap=round, font=\footnotesize, >=stealth, thick,  scale=0.85]
			\def\a{4}
			\path
			(-1.5,0) coordinate (A)
			(-1.5,-\a) coordinate (B)
			(-3,-\a) coordinate (B')
			(1.2,0) coordinate (C)
			;
			\draw (-3,0)--(5,0) (-3,-\a)--(5,-\a);
			\draw (A)--(B)--(C)node[midway,sloped]{$>$};
			\foreach \p/\r in {A/90,B/-90,C/90}
			\filldraw (\p) circle (1.2pt) node[shift={(\r:0.35cm)}]{$\p$};
			\foreach \x/\y/\z in {B/A/C,A/B/B'} \draw pic[draw,angle radius=2mm]
			{right angle= \x--\y--\z};
	\end{tikzpicture}}
	\loigiai{
		Vì thời gian xuồng đi từ $B$ đến $C$ là $61$ giây và vận tốc chuyển động của xuồng là $2$ (m/s) nên ta có khoảng cách giữa $B$ và $C$ là $$BC=2\cdot61=122\text{ (m)}.$$
		Áp dụng định lý Pythagore, ta có khoảng cách giữa $A$ và $B$ là
		$$AB=\sqrt{BC^2-AC^2}=\sqrt{122^2-22^2}=120\text{ (m)}.$$
		Vậy khoảng cách $AB$ giữa hai bờ sông bằng $120$ m.	
	}
\end{bt}

%%%=======Bài 5
\begin{bt}%[Dự án EX-8-Đề Cương Toán 8]%[Thầy Nguyễn Đức Tuấn Anh]%[8H2H3-4]
	Cho hình thang cân $ABCD$ ($AB\parallel CD$). Chứng minh rằng $\widehat{CAD} = \widehat{DBC}$.
	\loigiai{
	\begin{center}
		\begin{tikzpicture}[scale=0.9, font=\footnotesize, line join=round, line cap=round, >=stealth]
			\tikzset{markl/.pic={\draw (90:.1)--(-90:.1);},}
			\path 
			(0:0) coordinate (A)
			++(0:3) coordinate (B)
			++(-70:3.5) coordinate (C)
			(-110:3.5) coordinate (D)
			;
			\draw[]
			(A)--(D)pic[pos =.5,sloped]{markl}
			(B)--(C)pic[pos =.5,sloped]{markl}
			(A)--(B) (C)--(D) (A)--(C) (B)--(D);
			\foreach \x /\goc in {A/150,B/60,C/-60,D/-150} \fill (\x) circle (1pt) 	($(\x)+(\goc:2.5mm)$) node {$\x$};
			\foreach \x/\y/\z in {C/D/A,B/C/D} \draw pic[draw,angle radius=4mm]
			{angle= \x--\y--\z};
		\end{tikzpicture}
	\end{center}
	Vì $ABCD$ là hình thang cân nên $AD=BC$ và  $\widehat{ADC} = \widehat{BCD}$.\\
		Xét $\triangle CAD$ và $\triangle DBC$, ta có
		\begin{itemize}
			\item $AD = BC$ (cmt);
			\item $\widehat{ADC} = \widehat{BCD}$ (cmt);
			\item $CD$ chung.
		\end{itemize}
		Suy ra $\triangle CAD = \triangle DBC$ (c.g.c).\\
		Vậy $\widehat{CAD} = \widehat{DBC}$.
	}
\end{bt}

%%%=======Bài 6
\begin{bt}%[Dự án EX-8-Đề Cương Toán 8]%[Thầy Nguyễn Đức Tuấn Anh]%[8H2H4-4]
	Cho hình bình hành $ABCD$, kẻ $AH$ vuông góc với $BD$ tại $H$ và $CK$ vuông góc với $BD$ tại $K$. 
	\begin{enumerate}
		\item Chứng minh tứ giác $AHCK$ là hình bình hành. 
		\item Gọi $I$ là trung điểm của $HK$. Chứng minh $I$ là trung điểm của $BD$. 
	\end{enumerate}
	\loigiai{
		\begin{center}
			\begin{tikzpicture}[scale=1, font=\footnotesize, line join=round, line cap=round, >=stealth]
				\def\d{5}
				\def\r{3}
				\path (0:0) coordinate (D)
				++(0:\d) coordinate (C)
				++(60:\r) coordinate (B)
				($(B)+(D)-(C)$) coordinate (A)
				($(B)!(A)!(D)$) coordinate (H)
				($(B)!(C)!(D)$) coordinate (K)
				($(A)!0.5!(C)$) coordinate (I)
				;
				\draw(A)--(B)--(C)--(D)--cycle (A)--(K) (C)--(H) (B)--(D) (A)--(H) (C)--(K) (A)--(C);
				\draw pic[draw,angle radius=2mm]{right angle=A--H--B};
				\draw pic[draw,angle radius=2mm]{right angle=C--K--D};
				\foreach \x/ \goc in {A/90,D/-90,C/-90,B/90,H/-110,K/70, I/90} 
				\fill (\x) circle (1.2pt)	
				($(\x)+(\goc:3mm)$) node {$\x$};
			\end{tikzpicture}
		\end{center}
		\begin{enumerate}
			\item Xét $\triangle ADH$ và $\triangle CBK$, ta có
			\begin{itemize}
				\item $AD=BC$ ($ABCD$ là hình bình hành);
				\item $\widehat{AHD}=\widehat{CKB}=90^\circ$ ($AH \perp BD$, $CK \perp BD$);
				\item $\widehat{ADH}=\widehat{CBK}$ (so le trong).
			\end{itemize}
			Vậy $\triangle AHD = \triangle CKB$ (ch-gn).\\
			Suy ra $AH=CK$.\\
			Xét tứ giác $AHCK$, ta có
			\begin{itemize}
				\item $AH \parallel CK$ ($AH \perp BD$, $CK \perp BD$);
				\item $AH=CK$ (cmt).
			\end{itemize}
			Vậy $AHCK$ là hình bình hành.
			\item Ta có
			\begin{itemize}
				\item $AHCK$ là hình bình hành;
				\item $I$ là trung điểm $HK$.
			\end{itemize}
			Suy ra $I$ là trung điểm $AC$.\\
			Tứ giác $ABCD$ là hình bình hành có $I$ là trung điểm $AC$.\\
			Suy ra $I$ là trung điểm $BD$.
		\end{enumerate}
	}
\end{bt}

%%%=======Bài 7
\begin{bt}%[Dự án EX-8-Đề Cương Toán 8]%[Thầy Nguyễn Đức Tuấn Anh]%[8H2H5-4]
	Cho hình chữ nhật $ABCD$, hai đường chéo cắt nhau tại $O$. Trên cạnh $AB$ lấy một điểm $M$. Kẻ $AH\perp CM$. Chứng minh rằng
	\begin{multicols}{2}
		\begin{enumerate}
		\item $OH=OA$;
		\item $HB\perp HD$.
	\end{enumerate}
	\end{multicols}
	\loigiai{
	\begin{center}
		\begin{tikzpicture}[scale=1, font=\footnotesize, line join=round, line cap=round, >=stealth]
			\def\d{5}
			\def\r{3}
			\path (0:0) coordinate (D)
			++(0:\d) coordinate (C)
			++(90:\r) coordinate (B)
			($(B)+(D)-(C)$) coordinate (A)
			(intersection of A--C and B--D) coordinate (O)
			($(A)!0.7!(B)$) coordinate (M)
			($(C)!(A)!(M)$) coordinate (H)
			;
			\draw
				(A)--(B)--(C)--(D)--cycle (B)--(H)--(D)
				(A)--(C) (B)--(D) (C)--(H) (A)--(H)--(O);
			\foreach \x/ \goc in {A/150,D/-150,C/-60,B/60,H/90,M/60,O/-90} 
				\fill (\x) circle (1.2pt) ($(\x)+(\goc:3mm)$) node {$\x$};
			\foreach \x/\y/\z in {A/H/C}
			 \draw pic[draw,angle radius=2mm] {right angle= \x--\y--\z};
		\end{tikzpicture}
	\end{center}
		\begin{enumerate}
			\item Vì $ABCD$ là hình chữ nhật nên $OA=OB=OC=OD$.\\
			Xét $\triangle HAC$ vuông tại $H$, có $HO$ là đường trung tuyến nên $HO=\dfrac{1}{2}AC$.\\ 
			Suy ra $OH=OA$.
			\item Ta có $AC=BD$ nên $HO=\dfrac{1}{2}BD$.\\
			Xét $\triangle BHD$, ta có
			\begin{itemize}
				\item $HO$ là đường trung tuyến;
				\item $HO=\dfrac{1}{2}BD$.
			\end{itemize}
			Suy ra $\triangle BHD$ vuông tại $H$.
		\end{enumerate}
	}
\end{bt}

%%%=======Bài 8
\begin{bt}%[Dự án EX-8-Đề Cương Toán 8]%[Thầy Nguyễn Đức Tuấn Anh]%[8H2H6-2]
	\begin{enumerate}
		\item Trong Hình 1 và Hình 2, hình nào là hình thoi? Vì sao?
		\begin{center}
			\begin{tikzpicture}[font=\footnotesize, line join=round, line cap=round, >=stealth,scale=0.8]
				\path
				(0,0) coordinate (Q)
				(3,2) coordinate (M)
				(3,-2) coordinate (P)
				(6,0) coordinate (N)
				($(M)!.5!(P)$) coordinate (O)
				(8,0) coordinate (B)
				(11,2) coordinate (A)
				(11,-2) coordinate (C)
				(14,0) coordinate (D)
				;
				\draw (A)--(B)node[midway,sloped]{$|$}--(C)node[midway,sloped]{$|$}--(D)node[midway,sloped]{$|$}--(A)node[midway,sloped]{$|$} (M)--(N)node[midway,sloped]{$|$}--(P)node[midway,sloped]{$||$}--(Q)node[midway,sloped]{$|$}--(M)node[midway,sloped]{$||$}--(P) (Q)--(N);
				\foreach \x/\y/\z in {M/O/N} {
					\path 
					($(\y)!5pt!(\z)$) coordinate (1)
					($(\y)!5pt!(\x)$) coordinate (2)
					($(1)+(2)-(\y)$) coordinate (3)
					;
					\draw (1)--(3)--(2);
				}
				\foreach \x/\y in {A/90,B/180,C/-90,D/0,M/90,N/0,P/-90,Q/180,O/-135}
				\draw [fill=black] (\x) circle (0.03cm) + (\y:0.3cm) node {$\x$};
				\draw (P) + (-90:1cm) node {Hình 1};
				\draw (C) + (-90:1cm) node {Hình 2};
			\end{tikzpicture}
		\end{center}
		\item Cho hình thoi $ABCD$ có $\widehat{A}=40^\circ$. Tính số đo $\widehat{ABC}$, $\widehat{BCD}$, $\widehat{CAD}$.
	\end{enumerate}
	\loigiai{
		\begin{enumerate}
			\item 
			Hình 1: $MNPQ$ có $MN=PQ$ và $MQ=NP$ nên $MNPQ$ là hình bình hành. \\
			Khi đó, hình bình hành $MNPQ$ có hai đường chéo $MP \perp NQ$ nên $MNPQ$ là hình thoi.\\
			Hình 2: $ABCD$ là hình thoi vì có 4 cạnh bằng nhau.
			\item Xét hình thoi $ABCD$ ta có 
			\immini{\begin{itemize}
					\item $\widehat{BCD} = \widehat{DAB} = 40^\circ$.
					\item $\widehat{ABC} + \widehat{BAD} = 180^\circ$.\\
					Suy ra $\widehat{ABC} = 180^\circ - 40^\circ = 140^\circ$.
					\item $\widehat{ADC}=\widehat{ABC} = 140^\circ$.
			\end{itemize}}{
				\begin{tikzpicture}[font=\footnotesize, line join=round, line cap=round, >=stealth]
					\path
					(0,0) coordinate (A)
					(6,0) coordinate (C)
					($(A)!1mm!20:(C)$) coordinate (a)
					($(C)!1mm!-20:(A)$) coordinate (c)
					(intersection of A--a and C--c) coordinate (B)
					($(A)!1/2!(C)$) coordinate (O)
					($(B)!2!(O)$) coordinate (D)
					;
					\draw (A)--(B)node[midway,sloped]{$|$}--(C)node[midway,sloped]{$|$}--(D)node[midway,sloped]{$|$}--(A)node[midway,sloped]{$|$};
					
					\foreach \x/\y in {B/90,A/180,D/-90,C/0}
					\draw [fill=black] (\x) circle (0.03cm) + (\y:0.3cm) node {$\x$};
					\path (A) + (0:1cm) node {$40^\circ$};
			\end{tikzpicture}}
		\end{enumerate}
	}
\end{bt}

%%%=======Bài 9
\begin{bt}%[Dự án EX-8-Đề Cương Toán 8]%[Thầy Nguyễn Đức Tuấn Anh]%[8H2V5-4]
	Cho tam giác $ABC$ vuông tại $A$, đường cao $AH$. Từ $H$, kẻ $HM$ vuông góc $AB$ tại $M$ và $HN$ vuông góc $AC$ tại $N$.
	\begin{enumerate}
		\item Chứng minh $AMHN$ là hình chữ nhật.
		\item Trên tia đối của tia $MH$, lấy điểm $D$ sao cho $MH = MD$. Trên tia đối của tia $NH$, lấy điểm $E$ sao cho $NH = NE$. Chứng minh tứ giác $AMNE$ là hình bình hành.
		\item Chứng minh $A$ là trung điểm của $DE$.
	\end{enumerate}
	\loigiai{
		\begin{center}
			\begin{tikzpicture}[font=\footnotesize, line join=round, line cap=round, >=stealth]
				\path
				(0,0) coordinate (A)
				(0,4) coordinate (B)
				(6,0) coordinate (C)
				($(B)!(A)!(C)$) coordinate (H)
				($(A)!(H)!(C)$) coordinate (N)
				($(B)!(H)!(A)$) coordinate (M)
				($(H)!2!(M)$) coordinate (D)
				($(H)!2!(N)$) coordinate (E)
				;
				\draw (B)--(C)--(A)--(B) (A)--(H) (D)--(M)node[midway]{/}--(H)node[midway]{/}--(N)node[midway]{$\times$}--(E)node[midway]{$\times$}--(A) (M)--(N);
				\draw[dashed] (A)--(D);
				\foreach \x/\y in {A/-135,B/90,C/-10,D/120,E/-90,H/45,M/135,N/-45}
				\draw[fill=black] (\x) circle (0.3mm) + (\y:0.3) node {$\x$};
				\foreach \x/\y/\z in {B/A/C,A/H/C,A/M/H,A/N/H} {
					\path 
					($(\y)!5pt!(\z)$) coordinate (1)
					($(\y)!5pt!(\x)$) coordinate (2)
					($(1)+(2)-(\y)$) coordinate (3)
					;
					\draw (1)--(3)--(2);
				}
				\path (0,0) node{\hypersetup{hidelinks}\href{VR5dC6I}{ }};
				\path (0,0) node{\hypersetup{hidelinks}\href{VR5dC6I}{ }};
			\end{tikzpicture}
		\end{center}
		\begin{enumerate}
			\item 
			Xét tứ giác $AMHN$ có  $\heva{&\widehat{HMA}=90^{\circ} & (HM\perp AB),\\&\widehat{HNA}=90^{\circ} & (HN\perp AC),\\&\widehat{MAN}=90^{\circ} & (\text{tam giác } ABC \text{ vuông tại } A).}$\\
			Vậy $AMHN$ là hình chữ nhật.      
			
			\item 
			Ta có $\heva{&MA=HN\quad (\text{hai cạnh đối của hình chữ nhật } AMHN),\\&NE=HN \quad (\text{giả thiết}).}$\\
			Suy ra $MA=NE$.\\
			Lại có $\heva{&MA \parallel HN\quad (\text{hai cạnh đối của hình chữ nhật } AMHN),\\& E \text{ thuộc tia đối của tia } NH.}$\\
			Suy ra $MA \parallel NE$.\\
			Tứ giác $AMNE$ có $MA=NE$, $MA \parallel NE$ nên là hình bình hành.
			\item 
			Theo chứng minh trên, ta có $AMNE$ là hình bình hành.\\
			Suy ra $MN\parallel AE$ và $MN=AE$.\quad (1)\\       
			Hoàn toàn tương tự, ta chứng minh được $ANMD$ là hình bình hành.\\
			Suy ra $MN\parallel AD$ và $MN=AD$.\quad (2)\\
			Từ (1) và (2), suy ra $\heva{&A,\,E,\,D\text{ thẳng hàng}  &\text{(tiên đề Euclid)},\\&AE=AD  &\text{(cùng bằng } MN). }$        \\
			Vậy $A$ là trung điểm của $DE$.
			
		\end{enumerate}
	}
\end{bt}

%%%=======Bài 10
\begin{bt}%[Dự án EX-8-Đề Cương Toán 8]%[Thầy Nguyễn Đức Tuấn Anh]%[8H2C7-4]
	 Cho hình vuông $ABCD$ lấy $M$ trên đường chéo $AC$ ($AM>MC$). Kẻ $MI$ vuông góc với $AD$ ($I\in AD$). Gọi $P$, $N$ lần lượt là điểm đối xứng của $M$ và $A$ qua $I$.
	\begin{enumerate}
		\item Tứ giác $AMNP$ là hình gì? Vì sao?
		\item Chứng minh $BM=PD$.
		\item Gọi $Q$ là giao điểm của $BM$ và $PD$. Chứng minh ba điểm $C$, $Q$, $N$ thẳng hàng.
	\end{enumerate}
	\loigiai{
	\begin{center}
		\begin{tikzpicture}[line join=round, line cap=round, font=\footnotesize, >=stealth, scale=1]
			\def\r{3}
			\path
			(135:\r) coordinate (A)
			(45:\r) coordinate (B)
			(-135:\r) coordinate (D)
			(-45:\r) coordinate (C)
			($(A)!0.8!(C)$) coordinate (M)
			($(A)!(M)!(D)$) coordinate (I)
			($(M)!2!(I)$) coordinate (P)
			($(A)!2!(I)$) coordinate (N)
			(intersection of B--M and P--D) coordinate (Q)
			($(B)!0.5!(D)$) coordinate (O)
			;
			\draw (A)--(B)--(C)--(D)--cycle (A)--(C) (M)--(P) (M)--(N)--(P)--(A) (A)--(N) (P)--(Q)--(B) (A)--(Q)--(I) (B)--(D) (O)--(Q);
			\draw[dashed, red] (C)--(N);
			\draw pic[draw, angle radius=0.25cm]{right angle=A--I--M};
			\foreach \p/\r in {A/135,B/45,C/-45,D/-135,M/0,I/135,P/180,N/-90,Q/-90,O/90}
			\filldraw (\p) circle (1pt) node[shift={(\r:0.35cm)}]{$\p$};
		\end{tikzpicture}
	\end{center}
	\begin{enumerate}
		\item Theo giả thiết, ta có $AN\perp MP$ tại $I$ với $I$ là trung điểm của $AN$ và $MP$. \\
		Suy ra tứ giác $AMNP$ là hình thoi.\quad(1)\\
		Xét hai tam giác $AIP$ và $AIM$ có\\ $\heva{&AI\text{ chung},\\&\widehat{AIP}=\widehat{AIM}=90^\circ,\\&IP=IM\text{ (do $I$ là trung điểm $IM$)}.}$\\
		Suy ra $\triangle AIP=\triangle AIM$ (c.g.c).\\
		Suy ra $\widehat{IAP}=\widehat{IAM}$.\\
		Mặt khác, ta lại có $ABCD$ là hình vuông và $AC$ là đường chéo nên $\widehat{IAM}=45^\circ$.\\
		Từ đó ta suy ra
		$$\widehat{PAM}=\widehat{IAP}+\widehat{IAM}=2\cdot\widehat{IAM}=2\cdot45^\circ=90^\circ.\quad(2)$$
		Từ (1) và (2) suy ra tứ giác $AMNP$ là hình vuông.
		\item Xét hai tam giác $APD$ và $AMB$ có\\
		$\heva{&AP=AM\text{ (do tứ giác $AMNP$ là hình vuông)},\\&\widehat{PAD}=\widehat{MAB}=45^\circ\text{ (cùng bằng $\widehat{IAM}$)},\\&AD=AB\text{ (do tứ giác $ABCD$ là hình vuông)}.}$\\
		Suy ra $\triangle APD=\triangle AMB$ (c.g.c).\\
		Suy ra $PD=BM$.
		\item Xét tứ giác $AMQP$ ta có
		$$\widehat{PAM}+\widehat{AMQ}+\widehat{MQP}+\widehat{QPA}=360^\circ.$$
		Mà ta lại có
		$$\widehat{APQ}+\widehat{AMQ}=\widehat{APQ}+180^\circ-\widehat{AMB}=\widehat{APQ}+180^\circ-\widehat{APQ}=180^\circ.$$
		Bên cạnh đó $\widehat{PAM}=90^\circ$. Suy ra $\widehat{MQP}=90^\circ$.\\
		Suy ra tam giác $PMQ$ vuông tại $Q$. Kết hợp với $I$ là trung điểm $PM$, ta suy ra
		$$IP=IM=IQ=IA=IN.$$
		Từ việc $IQ=IA=IN$, ta suy ra tam giác $AQN$ vuông tại $Q$, hay
		$$\widehat{AQN}=90^\circ.\quad(3)$$
		Gọi $O$ là trung điểm $BD$, ta cũng có $\widehat{BQD}=\widehat{MQP}=90^\circ$.\\
		Bằng chứng minh tương tự, ta chỉ ra rằng
		$$OQ=OB=OD=OA=OC.$$
		Từ đó, suy ra tam giác $AQC$ vuông tại $Q$, hay
		$$\widehat{AQC}=90^\circ.\quad(4)$$
		Từ (3) và (4), suy ra
		$$\widehat{NQC}=\widehat{NQA}+\widehat{AQC}=90^\circ+90^\circ=180^\circ.$$
		Suy ra $C$, $Q$, $N$ thẳng hàng.
	\end{enumerate}
	}
\end{bt}

% In đáp án trắc nghiệm
\Closesolutionfile{ans}
\indapan{6}{ans/ans-8C3-OTC}