\section{XÁC SUẤT CỦA BIẾN CỐ} % Tên bài

\subsection{Kết quả đồng khả năng}
\subsubsection{Kiến thức trọng tâm}
\begin{tomtat}
	Trong một phép thử ngẫu nhiên, hai kết quả được gọi là \textit{đồng khả năng} nếu chúng có khả năng xảy ra như nhau.
	\begin{luuy}
		\begin{enumerate}
			\item Trong phép thử tung đồng xu (hoặc gieo xúc xắc), nếu có giả thiết đồng xu, xúc xắc là cân đối và đồng chất thì các mặt của đồng xu hay xúc xắc sẽ có cùng khả năng xuất hiện.
			\item Trong phép thử lấy vật (quả bóng, viên bi, \ldots), nếu có giả thiết các vật có cùng kích thước và khối lượng thì mỗi vật đều có cùng khả năng được lựa chọn.
		\end{enumerate}
	\end{luuy}
\end{tomtat}

\begin{vd}%[Dự án EX-9-Đề Cương Toán 9]%[Hiep Nguyen Quang]%[9D6N2-1]
	Kết quả của mỗi phép thử sau có đồng khả năng không? Tại sao?
	\begin{enumerate}
		\item Gieo một con xúc xắc cân đối và đồng chất có $6$ mặt từ $1$ đến $6$.
		\item Rút một lá bài từ một hộp chứa $48$ lá bài màu đen và $13$ lá bài màu đỏ.
		\item Chọn ngẫu nhiên lần lượt $3$ quả banh tennis từ một hộp chứa $8$ quả banh tennis có cùng kích thước và khối lượng.
	\end{enumerate}
	\loigiai{
		\begin{enumerate}
			\item Do con xúc xắc cân đối và đồng chất nên các mặt đều có cùng khả năng xuất hiện. Các kết quả của phép thử là đồng khả năng.
			\item Do trong hộp có nhiều lá bài màu đen hơn so với lá bài màu đỏ. Do đó, khả năng xuất hiện của lá bài màu đen và lá bài màu đỏ không như nhau. Các kết quả của phép thử không đồng khả năng.
			\item Do các quả banh tennis có cùng kích thước và khối lượng nên có cùng khả năng được chọn. Các kết quả của phép thử là đồng khả năng.
		\end{enumerate}
	}
\end{vd}

\subsubsection{Bài tập}

\begin{bt}%[Dự án EX-9-Đề Cương Toán 9]%[Hiep Nguyen Quang]%[9D6N2-1]
	Kết quả của mỗi phép thử sau có đồng khả năng không? Tại sao?
	\begin{enumerate}
		\item Một vòng quay may mắn có $8$ ô hình quạt bằng nhau, được đánh số từ $1$ đến $8$. Quay vòng quay một lần.
		\item Một vận động viên bóng rổ ném một quả bóng vào rổ. Các kết quả có thể là \lq\lq ném trúng\rq\rq\ và \lq\lq ném trượt\rq\rq.
		\item Chọn ngẫu nhiên một bạn học sinh từ danh sách của một tổ có $10$ thành viên để làm tổ trưởng.
	\end{enumerate}
	\loigiai{
		\begin{enumerate}
			\item Vì các ô hình quạt có diện tích bằng nhau nên khả năng kim chỉ vào mỗi ô là như nhau. Do đó, các kết quả của phép thử là đồng khả năng.
			\item Khả năng ném trúng hay ném trượt phụ thuộc vào kỹ năng của vận động viên, không phải là ngẫu nhiên hoàn toàn. Một vận động viên giỏi sẽ có khả năng ném trúng cao hơn ném trượt. Do đó, các kết quả của phép thử không đồng khả năng.
			\item Vì mỗi bạn học sinh trong danh sách đều có cơ hội được chọn như nhau. Do đó, các kết quả của phép thử là đồng khả năng.
		\end{enumerate}
	}
\end{bt}

\begin{bt}%[Dự án EX-9-Đề Cương Toán 9]%[Hiep Nguyen Quang]%[9D6H2-1]
	Kết quả của mỗi phép thử sau có đồng khả năng không? Tại sao?
	\begin{enumerate}
		\item Quan sát một cột đèn tín hiệu giao thông tại một ngã tư. Các kết quả là \lq\lq đèn xanh\rq\rq, \lq\lq đèn vàng\rq\rq, \lq\lq đèn đỏ\rq\rq.
		\item Sử dụng một phần mềm tạo số ngẫu nhiên để chọn một số nguyên từ $1$ đến $100$.
		\item Lấy một viên bi từ một chiếc hộp chứa $5$ viên bi xanh và $5$ viên bi đỏ cùng kích thước. Tuy nhiên, các viên bi xanh làm bằng sắt và các viên bi đỏ làm bằng nhựa.
	\end{enumerate}
	\loigiai{
		\begin{enumerate}
			\item Thời gian sáng của mỗi màu đèn là khác nhau (thường đèn xanh sáng lâu nhất, đèn vàng sáng nhanh nhất). Do đó, xác suất quan sát được mỗi màu đèn tại một thời điểm ngẫu nhiên là không bằng nhau. Các kết quả của phép thử không đồng khả năng.
			\item Các phần mềm tạo số ngẫu nhiên được thiết kế để mỗi số trong khoảng cho trước có khả năng được chọn là bằng nhau. Do đó, các kết quả của phép thử là đồng khả năng.
			\item Mặc dù số lượng và kích thước bi như nhau, nhưng khối lượng của chúng khác nhau (bi sắt nặng hơn bi nhựa). Điều này có thể ảnh hưởng đến việc lựa chọn (ví dụ: người lấy có xu hướng cảm nhận và chọn bi nặng hơn hoặc nhẹ hơn). Do đó, các kết quả của phép thử không đồng khả năng.
		\end{enumerate}
	}
\end{bt}

\subsection{Xác suất của biến cố}
\subsubsection{Kiến thức trọng tâm}
\begin{tomtat}
	Giả sử một phép thử có không gian mẫu $\Omega$ gồm hữu hạn các kết quả đồng khả năng và $A$ là một biến cố. Xác suất của biến cố $A$, kí hiệu $\mathrm{P}(A)$, được xác định bởi công thức
	$$\mathrm{P}(A)=\dfrac{n(A)}{n(\Omega)},$$
	trong đó $n(A)$ là số các kết quả thuận lợi cho $A$; $n(\Omega)$ là số các kết quả có thể xảy ra.
	\begin{luuy}
		Để tính xác suất của biến cố $A$, ta thực hiện các bước sau
		\begin{itemize}
			\item \textit{Bước 1:} Xác định $n(\Omega)$ là số các kết quả có thể xảy ra.
			\item \textit{Bước 2:} Kiểm tra tính đồng khả năng của các kết quả.
			\item \textit{Bước 3:} Kiểm đếm số các kết quả thuận lợi cho biến cố $A$.
			\item \textit{Bước 4:} Tính xác suất của biến cố $A$ bằng công thức $\mathrm{P}(A)=\dfrac{n(A)}{n(\Omega)}$.
		\end{itemize}
	\end{luuy}
\end{tomtat}

\begin{vd}%[Dự án EX-9-Đề Cương Toán 9]%[Hiep Nguyen Quang]%[9D6H2-2]
	Trên bàn có một tấm bìa hình tròn được chia thành $10$ hình quạt bằng nhau và được đánh số từ $1$ đến $10$ như hình bên dưới. Thanh quay mũi tên ở tâm và quan sát xem khi dừng lại mũi tên chỉ vào ô số mấy.
	\immini{\begin{enumerate}
			\item Các kết quả của phép thử có đồng khả năng không? Tại sao?
			\item Tính xác suất của mỗi biến cố sau
			\begin{itemize}
				\item $A$: \lq\lq Mũi tên chỉ vào ô ghi số chia hết cho $3$\rq\rq;
				\item $B$: \lq\lq Mũi tên chỉ vào ô ghi số lớn hơn $6$\rq\rq.
			\end{itemize}
	\end{enumerate}}
	{\begin{tikzpicture}[scale=1, font=\footnotesize, line join=round, line cap=round, >=Stealth]
			\foreach\x in{36,108,...,360}
			\draw[fill=gray!20](\x:2)arc(\x:\x+36:2)--(0,0)--cycle;
			\foreach \x [count=\i] in{90,54,...,-234}
			\path(\x:1.6) node[rotate=\x-90]{\bfseries \i};	
			\fill(0,0)circle(.1);
			\draw[line width=.8mm,->](0,0)--(90:1.4);
			\draw[ultra thick] (0,0)circle (2);
			%\path (current bounding box.south) node[below=2mm]{\bfseries \it Hình 1};
	\end{tikzpicture}}
	\loigiai{
		\begin{enumerate}
			\item Do các hình quạt tròn có kích thước bằng nhau. Các kết quả của phép thử là đồng khả năng.
			\item Không gian mẫu của phép thử là $\Omega=\{1;2;3;4;5;6;7;8;9;10\}$.\\
			Số kết quả có thể xảy ra là $n(\Omega)=10$.\\
			Vì các kết quả thuận lợi cho biến cố $A$ là $3$;  $6$; $9$ nên $n(A)=3$.\\
			Xác suất của biến cố $A$ là 
			$\mathrm{P}(A)=\dfrac{n(A)}{n(\Omega)}=\dfrac{3}{10}=0{,}3$.\\
			Vì các kết quả thuận lợi cho biến cố $B$ là $7$; $8$; $9$; $10$ nên $n(B)=4$.\\
			Xác suất của biến cố $B$ là $\mathrm{P}(B)=\dfrac{n(B)}{n(\Omega)}=\dfrac{4}{10}=0{,}4$.\\
		\end{enumerate}
	}
\end{vd}

\begin{vd}%[Dự án EX-9-Đề Cương Toán 9]%[Hiep Nguyen Quang]%[9D6H2-2]
	Cho tập hợp $A=\{1;2\}$ và $B=\{0;3;4\}$. Viết ngẫu nhiên một số tự nhiên có hai chữ số $\overline{ab}$, trong đó $a \in A$ và $b\in B$.
	\begin{enumerate}
		\item Viết tập hợp $\Omega$ gồm các kết quả có thể xảy ra đối với số tự nhiên được viết ra.
		\item Tính xác suất của biến cố $I$: \lq\lq Số tự nhiên được viết ra là ước của $48$\rq\rq.
		\item Tính xác suất của biến cố $K$: \lq\lq Số tự nhiên được viết ra nhỏ hơn $20$\rq\rq.
	\end{enumerate}
	\loigiai{
		\begin{enumerate}
			\item Tập hợp $\Omega$ gồm các kết quả có thể xảy ra đối vối số tự nhiên được viết ra là $\Omega=\{10;13;14;20;23;24\}$.\\
			Do đó, tập hợp $\Omega$ có $6$ phần tử nên $n(\Omega)=6$.
			\item Số tự nhiên được viết ra là ước của $48$ là số $24$.\\
			Do đó có $1$ kết quả thuận lợi cho biến cố $I$ nên $n(I)=1$.\\
			Vậy xác suất của biến cố $I$ là $\mathrm{P}(I)=\dfrac{n(I)}{n(\Omega)}=\dfrac{1}{6}$.
			\item  Các số tự nhiên được viết ra nhỏ hơn $20$ là $10;13;14$.\\
			Do đó có $3$ kết quả thuận lợi cho biến cố $K$ nên $n(K)=3$.\\
			Vậy xác suất của biến cố $K$ là $\mathrm{P}(K)=\dfrac{n(K)}{n(\Omega)}=\dfrac{3}{6}=\dfrac{1}{2}$.
		\end{enumerate}
	}
\end{vd}

\begin{vd}%[Dự án EX-9-Đề Cương Toán 9]%[Hiep Nguyen Quang]%[9D6V2-2]
	Ở một loài thực vật, cho biết tính trạng màu sắc hoa do một gen quy định, trong đó allele $A$ quy định hoa đỏ là trội hoàn toàn so với allele $a$ quy định hoa trắng. Tính trạng chiều cao cây do một gen khác quy định, trong đó allele $B$ quy định thân cao là trội hoàn toàn so với allele $b$ quy định thân thấp. Các cặp gen này phân li độc lập. Cho lai hai cây bố mẹ đều dị hợp về cả hai cặp gen (kiểu gen $AaBb$). Phép thử là lấy ngẫu nhiên một cây ở đời con. Tính xác suất để lấy được một cây có kiểu hình \lq\lq hoa trắng, thân thấp\rq\rq.
	\loigiai{
		Phép lai giữa hai cây bố mẹ là $P: AaBb \times AaBb$.
		Mỗi cây bố mẹ dị hợp $2$ cặp gen sẽ tạo ra $4$ loại giao tử với tỉ lệ bằng nhau là $AB$, $Ab$, $aB$, $ab$.\\		
		Ta có bảng kết quả của phép lai
		\begin{center}
			\begin{tabular}{|c|c|c|c|c|}
				\hline
				Giao tử & $AB$ & $Ab$ & $aB$ & $ab$ \\
				\hline
				$AB$ & $AABB$ & $AABb$ & $AaBB$ & $AaBb$ \\
				\hline
				$Ab$ & $AABb$ & $AAbb$ & $AaBb$ & $Aabb$ \\
				\hline
				$aB$ & $AaBB$ & $AaBb$ & $aaBB$ & $aaBb$ \\
				\hline
				$ab$ & $AaBb$ & $Aabb$ & $aaBb$ & $aabb$ \\
				\hline
			\end{tabular}
		\end{center}
		Không gian mẫu $\Omega$ bao gồm $16$ tổ hợp kiểu gen có khả năng xuất hiện như nhau nên $n(\Omega) = 16$.\\
		Gọi $E$ là biến cố \lq\lq Cây con có kiểu hình hoa trắng, thân thấp\rq\rq\,.\\
		Kiểu hình \lq\lq hoa trắng, thân thấp\rq\rq\, là kiểu hình lặn, tương ứng với chỉ một kiểu gen là $aabb$.\\		
		Có $1$ kết quả thuận lợi duy nhất cho biến cố $E$ nên $n(E)=1$.\\		
		Vậy xác suất của biến cố $E$ là	$\mathrm{P}(E)=\dfrac{n(E)}{n(\Omega)}=\dfrac{1}{16}$.
	}
\end{vd}

\subsubsection{Bài tập}

\begin{bt}%[Tham Khảo HK2 NH 24-25, TPHCM, Tp. Thủ Đức]%[9D6H2-2] 
	Hộp thứ nhất đựng $1$ quả bóng trắng, $1$ quả bóng đỏ. Hộp thứ hai đựng $1$ quả bóng đỏ, $1$ quả bóng vàng. Lấy ra ngẫu nhiên từ mỗi hộp $1$ quả bóng.
	\begin{enumerate}
		\item Xác định không gian mẫu và số kết quả có thể xảy ra của phép thử.
		\item Biết rằng các quả bóng có cùng kích thước và khối lượng. Hãy tính xác suất của biến cố $A$: \lq\lq có đúng $1$ quả bóng màu đỏ trong $2$ quả bóng lấy ra\rq\rq.
	\end{enumerate}
	\loigiai
	{
		\begin{enumerate}
			\item Kí hiệu T là màu trắng, Đ là màu đỏ, V là màu vàng.\\
			Kí hiệu XY là kết quả bóng lấy ra từ hộp thứ nhất có màu X và hộp thứ hai có màu Y.\\
			Không gian mẫu của phép thử là $\Omega=\{$TĐ; TV; ĐĐ; ĐV$\}$.\\
			Số kết quả có thể xảy ra là $n(\Omega)=4$.
			\item Các kết quả thuận lợi cho biến cố $A$ là TĐ và ĐV.
			Do đó $n(A)=2$.\\
			Xác suất của biến cố $A$ là $\mathrm{P}(A)=\dfrac{n(A)}{n(\Omega)}=\dfrac{2}{4}=\dfrac{1}{2}$.
		\end{enumerate}	
	}
\end{bt} 

\begin{bt}%[Dự án EX-9-Đề Cương Toán 9]%[Hiep Nguyen Quang]%[9D6H2-2]
	Trong tủ lạnh nhà bạn Minh có $6$ hộp thịt gà, $4$ hộp thịt heo và $10$ hộp thịt bò. Các hộp này có kích thước và khối lượng bằng nhau. Vì tinh nghịch nên bạn Minh đã xé hết nhãn ghi trên các hộp. Mẹ Minh chọn ngẫu nhiên $1$ trong các hộp thịt trên.
	\begin{enumerate}
		\item Các kết quả của phép thử có đồng khả năng không? Tại sao?
		\item Tính xác suất của mỗi biến cố sau
		\begin{itemize}
			\item $A$: \lq\lq Hộp được chọn là hộp thịt gà\rq\rq;
			\item $B$: \lq\lq Hộp được chọn không phải là hộp thịt heo\rq\rq.
		\end{itemize}
	\end{enumerate}			
	\loigiai{%
		\begin{enumerate}
			\item Do các hộp thịt có cùng kích thước và khối lượng nên chúng có cùng khả năng được chọn. Các kết quả của phép thử là đồng khả năng.
			\item Số kết quả có thể xảy ra là $n(\Omega)=6+4+10=20$.\\
			Vì có $6$ hộp thịt gà nên số kết quả thuận lợi cho biến cố $A$ là $n(A)=6$.\\
			Xác suất của biến cố $A$ là 
			$\mathrm{P}(A)=\dfrac{n(A)}{n(\Omega)}=\dfrac{6}{20}=0{,}3$.\\
			Số hộp thịt không phải là hộp thịt heo là $6+10=16$.\\
			Vậy số kết quả thuận lợi cho biến cố $B$ là $n(B)=16$.\\
			Xác suất của biến cố $B$ là $\mathrm{P}(B)=\dfrac{n(B)}{n(\Omega)}=\dfrac{16}{20}=0{,}8$.
		\end{enumerate}
	}
\end{bt}

\begin{bt}%[Dự án EX-9-Đề Cương Toán 9]%[Hiep Nguyen Quang]%[9D6H2-2]
	Cho tập hợp $A=\{3;5\}$ và $B=\{1;6;8\}$. Viết ngẫu nhiên một số tự nhiên có hai chữ số $\overline{ab}$, trong đó $a\in A$ và $b\in B$.
	\begin{enumerate}
		\item Viết tập hợp không gian mẫu $\Omega$ gồm các kết quả có thể xảy ra.
		\item Tính xác suất của biến cố $M$: \lq\lq Số tự nhiên được viết ra là số nguyên tố\rq\rq.
		\item Tính xác suất của biến cố $N$: \lq\lq Số tự nhiên được viết ra là số chẵn\rq\rq.
	\end{enumerate}
	\loigiai{
		\begin{enumerate}
			\item Các số tự nhiên có hai chữ số $\overline{ab}$ với $a \in A$ và $b \in B$ có thể tạo thành là $31$; $36$; $38$; $51$; $56$; $58$.\\
			Tập hợp $\Omega$ gồm các kết quả có thể xảy ra là $\Omega=\{31;36;38;51;56;58\}$.\\
			Số phần tử của không gian mẫu là $n(\Omega) = 6$.
			\item Trong các số trên, số nguyên tố là $31$.\\
			Do đó, có $1$ kết quả thuận lợi cho biến cố $M$ nên $n(M)=1$.\\
			Vậy xác suất của biến cố $M$ là $\mathrm{P}(M)=\dfrac{n(M)}{n(\Omega)}=\dfrac{1}{6}$.
			\item Trong các số trên, các số chẵn là $36;38;56;58$.\\
			Do đó, có $4$ kết quả thuận lợi cho biến cố $N$ nên $n(N)=4$.\\
			Vậy xác suất của biến cố $N$ là $\mathrm{P}(N)=\dfrac{n(N)}{n(\Omega)}=\dfrac{4}{6}=\dfrac{2}{3}$.
		\end{enumerate}
	}
\end{bt}

\begin{bt}%[Dự án EX-9-Đề Cương Toán 9]%[Hiep Nguyen Quang]%[9D6V2-2]
	Một hộp có $30$ quả bóng với kích thước và khối lượng như nhau. Bạn An viết lên các quả bóng đó các số $1$, $2$, $3$, \dots, $30$; hai quả bóng khác nhau thì viết hai số khác nhau. Lấy ngẫu nhiên một quả bóng trong hộp. Tính xác suất biến cố: \lq\lq Số xuất hiện trên quả bóng được lấy ra chia cho $5$ dư $3$\rq\rq.
	\loigiai{
		Không gian mẫu của phép thử là $\Omega = \{1;2;3;\dots;30\}$.\\
		Số phần tử của không gian mẫu là $n(\Omega) = 30$.\\
		Gọi $A$ là biến cố \lq\lq Số xuất hiện trên quả bóng được lấy ra chia cho $5$ dư $3$\rq\rq.\\
		Các kết quả thuận lợi cho biến cố $A$ là $3$; $8$; $13$; $18$; $23$; $28$.\\
		Có $6$ kết quả thuận lợi cho biến cố $A$ nên $n(A)=6$.\\
		Vậy xác suất của biến cố $A$ là $\mathrm{P}(A) = \dfrac{n(A)}{n(\Omega)} = \dfrac{6}{30} = \dfrac{1}{5}$.
	}
\end{bt}

\begin{bt}%[Dự án EX-9-Đề Cương Toán 9]%[Hiep Nguyen Quang]%[9D6H2-2]
	Một hộp kín chứa $25$ quả bóng giống hệt nhau, được đánh số lần lượt từ $1$ đến $25$. Lấy ngẫu nhiên một quả bóng từ trong hộp. Tính xác suất của biến cố: \lq\lq Số xuất hiện trên quả bóng là số chính phương\rq\rq.
	\loigiai{
		Không gian mẫu của phép thử là tập hợp các số từ $1$ đến $25$
		$$\Omega = \{1; 2; 3; 4; 5; 6; 7; 8; 9; 10; 11; 12; 13; 14; 15; 16; 17; 18; 19; 20; 21; 22; 23; 24; 25\}.$$
		Số phần tử của không gian mẫu là $n(\Omega) = 25$.\\
		Gọi $B$ là biến cố \lq\lq Số xuất hiện trên quả bóng là số chính phương\rq\rq.\\
		Các kết quả thuận lợi cho biến cố $B$ là các số chính phương có trong tập $\Omega$. Đó là các số $1$; $4$; $9$; $16$; $25$.\\
		Có $5$ kết quả thuận lợi cho biến cố $B$ nên $n(B)=5$.\\
		Vậy xác suất của biến cố $B$ là $\mathrm{P}(B) = \dfrac{n(B)}{n(\Omega)} = \dfrac{5}{25} = \dfrac{1}{5}$.
	}
\end{bt}

\begin{bt}%[Dự án EX-9-Đề Cương Toán 9]%[Hiep Nguyen Quang]%[9D6H2-2]
	Một túi đựng $4$ quả bóng xanh được ghi số $1$, $2$, $3$, $4$ và $2$ quả bóng vàng được ghi số $5$, $6$. Chọn ngẫu nhiên đồng thời hai quả bóng từ trong túi.
	\begin{enumerate}
		\item Liệt kê các phần tử của không gian mẫu.
		\item Tính xác suất của biến cố $D$: \lq\lq Tổng các số ghi trên hai quả bóng là một số lẻ\rq\rq.
	\end{enumerate}
	\loigiai{
		\begin{enumerate}
			\item Trong túi có tất cả $6$ quả bóng được ghi số từ $1$ đến $6$. Không gian mẫu gồm các cặp hai số có thể lấy ra là
			\begin{eqnarray*}
				\Omega &=& \{ \{1;2\}; \{1;3\}; \{1;4\}; \{1;5\}; \{1;6\}; \{2;3\}; \{2;4\}; \{2;5\};\\
				&& \{2;6\}; \{3;4\}; \{3;5\}; \{3;6\}; \{4;5\}; \{4;6\}; \{5;6\} \}.
			\end{eqnarray*}
			Tổng cộng có $15$ kết quả có thể xảy ra nên $n(\Omega) = 15$.
			\item Tổng hai số là một số lẻ khi một số là chẵn và một số là lẻ.
			\begin{itemize}
				\item Các số lẻ là $\{1; 3; 5\}$ (có $3$ số).
				\item Các số chẵn là $\{2; 4; 6\}$ (có $3$ số).
			\end{itemize}
			Các kết quả thuận lợi cho biến cố $D$ là các cặp gồm một số lẻ và một số chẵn
			$$ \{1;2\}; \{1;4\}; \{1;6\}; \{2;3\}; \{2;5\}; \{3;4\}; \{3;6\}; \{4;5\}; \{5;6\}. $$
			Có $9$ kết quả thuận lợi cho biến cố $D$ nên $n(D)=9$.\\
			Vậy xác suất của biến cố $D$ là $\mathrm{P}(D)=\dfrac{n(D)}{n(\Omega)}=\dfrac{9}{15}=\dfrac{3}{5}$.
		\end{enumerate}
	}
\end{bt}

\begin{bt}%[Tham Khảo HK2 NH 24-25, TPHCM, Nguyễn Văn Bá - Tp. Thủ Đức]%[9D6H2-2]
	Biểu đồ cột kép bên dưới biểu thị số lượng học sinh nam, nữ của các khối lớp tại một trường trung học cơ sở. Nhà trường cần chọn ra $1$ em bất kỳ để tham dự \lq\lq Diễn đàn lắng nghe tiếng nói học sinh\rq\rq.
	\begin{center}
		\begin{tikzpicture}[>=stealth, scale=0.8,thick,
			join = round, cap = round,
			declare function={
				x=1.5; %Co trục ngang
				y=2; %Co trục đứng
				kcy=1; %Khoảng cách từ trục đứng đến cột 1
				kc=2; %Khoảng cách giữa 2 cột
				dochia=50.0; %Độ chia nhỏ nhất trên trục đứng
			},
			xscale = 1/x, yscale = 1/y, 
			font = \scriptsize
			]
			\foreach \x/\y/\z[count = \i from 0] in {
				Lớp 6/360/315,
				Lớp 7/380/415,
				Lớp 8/355/370,
				Lớp 9/411/400
			}{
				\pgfmathsetmacro{\j}{(kc+2)*\i+kcy+1}
				\draw[pattern = north east lines, pattern color=blue] (\j-1,0) rectangle (\j,\y/dochia);
				\path (\j-.5,\y/dochia) node[above]{$\y$};
				\draw[pattern = dots, pattern color=red] (\j,0) rectangle (\j+1,\z/dochia);
				\path (\j+.5,\z/dochia) node[above]{$\z$};
				\path (\j,-.3*y) node[rotate = 0]{\x};
				\global\let\n=\j
			}
			%Vẽ hệ trục
			\draw[<->] (\n+x+1,0)node[below right]{Khối} -| (0,500/dochia+y/dochia)node[above,align=center, text width=2cm]{Số lượng học sinh};
			\foreach \y in {0,50,...,450}{
				\draw (.05*x,\y/dochia)--(-.05*x,\y/dochia) node[left]{$\y$};
			}
			%Chú thích
			\draw[pattern = north east lines, pattern color=blue] (\n+x,450/dochia) rectangle ++(.5*x,.5*y)++(0,-.25*y) node[right]{Nam};
			\draw[pattern = dots, pattern color=red] (\n+x,450/dochia-.7*y) rectangle ++(.5*x,.5*y)++(0,-.25*y) node[right]{Nữ};
		\end{tikzpicture}
	\end{center}
	\begin{enumerate}
		\item Tìm số phần tử của tập hợp $\Omega$ (Không gian mẫu).
		\item Tính xác suất của biến cố $A$: \lq\lq Học sinh được chọn không nhỏ hơn lớp $8$\rq\rq.
	\end{enumerate}
	\loigiai{
		\begin{enumerate}
			\item Số phần tử của tập hợp $\Omega$ là
			$$n(\Omega) = 360+315+380+415+355+370+411+400 = 3\,006.$$
			\item Số học sinh được chọn không nhỏ hơn lớp $8$ là $355 + 370 + 411 + 400 = 1\,536$ (học sinh).\\
			Do đó, số kết quả thuận lợi cho biến cố $A$ là $n(A) = 1\,536$.\\
			Xác suất của biến cố $A$ là
			$$\mathrm{P}(A) = \dfrac{n(A)}{n(\Omega)} = 
			\dfrac{1\,536}{3\,006} = \dfrac{256}{501}. $$
		\end{enumerate}
	}
\end{bt}

\begin{bt}%[Dự án EX-9-Đề Cương Toán 9]%[Hiep Nguyen Quang]%[9D6H2-2]
	Biểu đồ cột kép ở hình bên dưới biểu diễn số lượng ly trà sữa và nước ép một cửa hàng bán được trong cả tuần.\\
	Chọn ngẫu nhiên một ly nước được bán ra trong tuần đó.\\
	Tính xác suất của mỗi biến cố sau
	\begin{itemize}
		\item $G$: \lq\lq Ly nước được chọn là Nước ép\rq\rq.
		\item $H$: \lq\lq Ly nước được chọn được bán vào một ngày cuối tuần (Thứ Bảy hoặc Chủ Nhật)\rq\rq.
		\item $I$: \lq\lq Ly nước được chọn là Trà sữa và không được bán vào ngày Thứ Hai\rq\rq.
	\end{itemize}
	\begin{center}
		\begin{tikzpicture}[declare function={r=1.2;d=1.8;},x=.65cm, y=.5cm]	
			% Trục và lưới y
			\foreach \i in {0,5,...,40}{
				\draw[thin,gray!30] node[shift={(-0.5*r,\i/2)},color=black]{\i} ({-0.35*r},\i/2)--({8.5*r+6*d},\i/2);}						
			% Dữ liệu các ngày
			% Thứ Hai
			\fill[color=red!60] (r,0) node[shift={(0.25*r,-12pt)}]{\tiny T2} rectangle ({1.5*r},10/2);
			\fill (r,10/2)--({1.5*r},10/2) node[midway,above]{\tiny $10$};
			\fill[color=yellow!60] ({1.5*r},0) rectangle ({2*r},12/2);
			\fill ({1.5*r},12/2)--({2*r},12/2) node[midway,above]{\tiny $12$};
			% Thứ Ba
			\fill[color=red!60] ({1.5*r+d},0) node[shift={(0.25*r,-12pt)}]{\tiny T3} rectangle ({2*r+d},12/2);
			\fill ({1.5*r+d},12/2)--({2*r+d},12/2) node[midway,above]{\tiny $12$};
			\fill[color=yellow!60] ({2*r+d},0) rectangle ({2.5*r+d},15/2);
			\fill ({2*r+d},15/2)--({2.5*r+d},15/2) node[midway,above]{\tiny $15$};
			% Thứ Tư
			\fill[color=red!60] ({2.5*r+2*d},0) node[shift={(0.25*r,-12pt)}]{\tiny T4} rectangle ({3*r+2*d},15/2);
			\fill ({2.5*r+2*d},15/2)--({3*r+2*d},15/2) node[midway,above]{\tiny $15$};
			\fill[color=yellow!60] ({3*r+2*d},0) rectangle ({3.5*r+2*d},18/2);
			\fill ({3*r+2*d},18/2)--({3.5*r+2*d},18/2) node[midway,above]{\tiny $18$};
			% Thứ Năm
			\fill[color=red!60] ({3.5*r+3*d},0) node[shift={(0.25*r,-12pt)}]{\tiny T5} rectangle ({4*r+3*d},18/2);
			\fill ({3.5*r+3*d},18/2)--({4*r+3*d},18/2) node[midway,above]{\tiny $18$};
			\fill[color=yellow!60] ({4*r+3*d},0) rectangle ({4.5*r+3*d},20/2);
			\fill ({4*r+3*d},20/2)--({4.5*r+3*d},20/2) node[midway,above]{\tiny $20$};
			% Thứ Sáu
			\fill[color=red!60] ({4.5*r+4*d},0) node[shift={(0.25*r,-12pt)}]{\tiny T6} rectangle ({5*r+4*d},25/2);
			\fill ({4.5*r+4*d},25/2)--({5*r+4*d},25/2) node[midway,above]{\tiny $25$};
			\fill[color=yellow!60] ({5*r+4*d},0) rectangle ({5.5*r+4*d},22/2);
			\fill ({5*r+4*d},22/2)--({5.5*r+4*d},22/2) node[midway,above]{\tiny $22$};
			% Thứ Bảy
			\fill[color=red!60] ({5.5*r+5*d},0) node[shift={(0.25*r,-12pt)}]{\tiny T7} rectangle ({6*r+5*d},35/2);
			\fill ({5.5*r+5*d},35/2)--({6*r+5*d},35/2) node[midway,above]{\tiny $35$};	
			\fill[color=yellow!60] ({6*r+5*d},0) rectangle ({6.5*r+5*d},30/2);
			\fill ({6*r+5*d},30/2)--({6.5*r+5*d},30/2) node[midway,above]{\tiny $30$};	
			% Chủ Nhật
			\fill[color=red!60] ({6.5*r+6*d},0) node[shift={(0.25*r,-12pt)}]{\tiny CN} rectangle ({7*r+6*d},30/2);
			\fill ({6.5*r+6*d},30/2)--({7*r+6*d},30/2) node[midway,above]{\tiny $30$};	
			\fill[color=yellow!60] ({7*r+6*d},0) rectangle ({7.5*r+6*d},28/2);
			\fill ({7*r+6*d},28/2)--({7.5*r+6*d},28/2) node[midway,above]{\tiny $28$};	
			% Trục toạ độ và chú thích
			\draw[-stealth,thick] (0,0)--(0,22) node[above]{Số ly};
			\draw[-stealth,thick] (0,0)--(7.5*r+7*d,0) node[right]{Ngày};		
			\node[below=0mm] at (current bounding box.south){\it Hình $30$};
			\node[shift = {(5cm,-0.75cm)}] at (current bounding box.north){\it 
				\begin{tikzpicture}
					\fill[red!60] (0,0) rectangle (1,1) node[shift={(-15:1cm)}]{\color{black}Trà sữa} ;
					\fill[yellow!60] (0,-2) rectangle (1,-1) node[shift={(-15:1cm )}]{\color{black}{Nước ép}} ;
			\end{tikzpicture}};
		\end{tikzpicture}
	\end{center}
	\loigiai{
		Dựa vào biểu đồ, ta tính tổng số ly nước đã bán trong cả tuần là
		$$(10+12) + (12+15) + (15+18) + (18+20) + (25+22) + (35+30) + (30+28) = 290 \text{ (ly)}.$$
		Tổng số ly nước ép đã bán trong tuần là
		$$12+15+18+20+22+30+28=145 \text{ (ly)}.$$
		Tổng số ly nước bán vào cuối tuần (Thứ Bảy và Chủ Nhật) là
		$$(35+30) + (30+28) = 65+58=123 \text{ (ly)}.$$
		Tổng số ly trà sữa đã bán trong tuần là
		$$10+12+15+18+25+35+30=145 \text{ (ly)}.$$
		Số ly trà sữa không bán vào ngày Thứ Hai là $145-10=135$ (ly).\\
		Ta tính được xác suất của các biến cố.\\
		Xác suất của biến cố $G$: \lq\lq Ly nước được chọn là Nước ép\rq\rq\, là
		\[\mathrm{P}(G)=\dfrac{145}{290}=\dfrac{1}{2}.\]
		Xác suất của biến cố $H$: \lq\lq Ly nước được chọn được bán vào một ngày cuối tuần\rq\rq\, là
		\[\mathrm{P}(H)=\dfrac{123}{290}.\]
		Xác suất của biến cố $I$: \lq\lq Ly nước được chọn là Trà sữa và không được bán vào ngày Thứ Hai\rq\rq\, là
		\[\mathrm{P}(I)=\dfrac{135}{290}=\dfrac{27}{58}.\]
	}
\end{bt}

\begin{bt}%[Tham Khảo Tuyển Sinh 10 NH25-26, TPHCM, Huyện Củ Chi - Thị Trấn 2]%[9D6V2-2]
	Biểu đồ cột kép biểu diễn số lượng học sinh tham gia giải thi đấu thể thao của một trường trung học cơ sở.
	\begin{center}
		\begin{tikzpicture}[scale=1, font=\footnotesize,line join=round, line cap=round,>=stealth,yscale=.7,x=1.3cm]
			\def\kc{0.05}
			\def\tl{1}
			\foreach \d in {1,...,10}
			{
				\pgfmathsetmacro{\ny}{int(\d*\tl)}
				\draw (0,\d) node[left]{$ \ny $};
				\draw[shift={(0,\d)}] (-1pt,0)--(1pt,0);
				\draw[opacity=.3] (0,\d)--++(0:8);
			}
			\def\mau{{"blue!60","orange!40"}}
			\def\xeploai{{"Nam","Nữ"}}
			\foreach \x/\y/\z [count = \i from 0]  in {7/9/Khối 6,9/7/Khối 7,9/8/Khối 8,9/8/Khối 9}
			{
				\pgfmathsetmacro\col{\mau[1]}
				\pgfmathsetmacro\coll{\mau[0]}
				\pgfmathsetmacro{\xi}{\x/\tl}
				\pgfmathsetmacro{\yi}{\y/\tl}
				\pgfmathsetmacro{\ni}{1+1.75*\i}
				\fill[\coll,draw] ({\ni-\kc},0) rectangle ++(-.4,\xi);
				\draw ({\ni-\kc-.2},0)++(0,\xi)node[above]{\x};
				\fill[\col,draw] ({\ni+\kc},0) rectangle ++(.4,\yi);
				\draw ({\ni+\kc+.2},0)++(0,\yi)node[above]{\y};
				\path (\ni,0)node[below]{\z};
			}
			\foreach \i in {0,1}
			{
				\pgfmathsetmacro\col{\mau[\i]}
				\pgfmathsetmacro\xl{\xeploai[\i]}
				\fill[\col] (1+3*\i,10.5) rectangle ++(.45,.6);
				\draw (1.45+3*\i,10.5)node[ above right] {\xl};
			}
			\draw[->] (0,0)--(0,11) node[left] {Số học sinh}
			;
			\draw [->] (0,0)--(8,0) node[below] {Khối}
			;
			\fill circle (1pt) node[left]{$0$};
			%	\path (current bounding box.west) node[rotate=90,font=\bfseries,scale=.7,above]{Số lượng học sinh (học sinh)};
			%	\path (current bounding box.south) node[font=\bfseries,scale=.7,below]{Môn thể thao yêu thích};
		\end{tikzpicture}
	\end{center}
	Chọn ngẫu nhiên một học sinh tham gia giải thi đấu thể thao của trường đó. Tính xác suất của mỗi biến cố sau
	\begin{itemize}
		\item $A$: \lq\lq Học sinh được chọn là nam\rq\rq;
		\item $B$: \lq\lq Học sinh được chọn thuộc khối $6$\rq\rq;
		\item $C$: \lq\lq Học sinh được chọn là nữ và không thuộc khối $9$\rq\rq.
	\end{itemize}
	\loigiai{
		Từ biểu đồ, ta có
		\begin{itemize}
			\item Tổng số học sinh khối $6$ là $7+9=16$ (học sinh).
			\item Tổng số học sinh khối $7$ là $9+7=16$ (học sinh).
			\item Tổng số học sinh khối $8$ là $9+8=17$ (học sinh).
			\item Tổng số học sinh khối $9$ là $9+8=17$ (học sinh).
		\end{itemize}
		Như vậy, không gian mẫu có tất cả
		$$n(\Omega) = 16 + 16 + 17 + 17 = 66\ \text{(học sinh).}$$
		Số kết quả thuận lợi cho biến cố $A$ là $7+9+9+9=34$.\\
		Do đó $n(A) = 34$.\\
		Xác suất để biến cố $A$ xảy ra là $\mathrm{P}(A) = \dfrac{n(A)}{n(\Omega)} = \dfrac{34}{66}=\dfrac{17}{33}$.\\
		Số kết quả thuận lợi cho biến cố $B$ là $16$. \\
		Do đó $n(B) = 16$.\\
		Xác suất để biến cố $B$ xảy ra là $\mathrm{P}(B) = \dfrac{n(B)}{n(\Omega)} = \dfrac{16}{66}=\dfrac{8}{33}$.\\
		Số kết quả thuận lợi cho biến cố $C$ là $9+7+8=24$. \\
		Do đó $n(C) = 24$.\\
		Xác suất để biến cố $C$ xảy ra là $\mathrm{P}(C) = \dfrac{n(C)}{n(\Omega)} = \dfrac{24}{66}=\dfrac{4}{11}$.
	}
\end{bt}

\begin{bt}%[Tham Khảo Tuyển Sinh 10 NH25-26, TPHCM, Huyện Củ Chi - Tân Tiến]%[9D6H2-2]
	Số điểm $10$ của các bạn lớp $9$B đạt được trong tuần qua được cho bởi biểu đồ sau
	\begin{center}
		\begin{tikzpicture}[line join = round, line cap=round,>=stealth,font=\footnotesize,scale=0.7]
			
			\draw[gray,dashed] (0,0)node[left]{$0$}--(16,0);
			\draw[gray,dashed] (0,1)node[left]{$2$}--(16,1);
			\draw[gray,dashed] (0,2)node[left]{$4$}--(16,2);
			\draw[gray,dashed] (0,3)node[left]{$6$}--(16,3);
			\draw[gray,dashed] (0,4)node[left]{$8$}--(16,4);
			\draw[gray,dashed] (0,5)node[left]{$10$}--(16,5);
			
			\draw[->] (0,0)--(16,0)node[below]{Thứ};
			\draw[->] (0,0)--(0,6)node[above]{Số điểm 10};
			\draw[blue!60,line width=0.5mm] (1.5,3)--(4.5,2.5)--(7.5,4)--(10.5,3.5)--(13.5,4.5);
			\fill[blue!60,thick] (1.5,3)circle(3pt)--(4.5,2.5)circle(3pt)--(7.5,4)circle(3pt)--(10.5,3.5)circle(3pt)--(13.5,4.5)circle(3pt);
			\draw 
			(1.5,0) node[below]{Thứ 2}
			(1.5,3)  node[above]{$6$}
			(4.5,0) node[below]{Thứ 3}
			(4.5,2.5) node[above]{$5$}
			(7.5,0) node[below]{Thứ 4}
			(7.5,4) node[above]{$8$}
			(10.5,0) node[below]{Thứ 5}
			(10.5,3.5) node[above]{$7$}
			(13.5,0) node[below]{Thứ 6}
			(13.5,4.5) node[above]{$9$}		
			;
		\end{tikzpicture}
	\end{center}
	\begin{enumerate}
		\item Trong tuần, học sinh lớp $9$B đạt được nhiều điểm $10$ nhất là thứ mấy?
		\item Chọn ngẫu nhiên một ngày trong tuần. Biết rằng khả năng $5$ ngày được chọn là như nhau. Tính xác suất của các biến cố sau\\
		$A$: \lq\lq Vào ngày được chọn học sinh lớp $9$B đạt $8$ điểm $10$\rq\rq.\\
		$B$: \lq\lq Vào ngày được chọn học sinh lớp $9$B đạt trên $9$ điểm $10$\rq\rq.
	\end{enumerate}
	\loigiai
	{
		\begin{enumerate}
			\item Theo biểu đồ, trong tuần học sinh lớp $9$B đạt được nhiều điểm $10$ nhất là thứ $6$.
			\item Gọi $\Omega$ là không gian mẫu.\\
			Khi đó $\Omega = \left\lbrace \textrm{Thứ $2$}; \textrm{Thứ $3$}; \textrm{Thứ $4$}; \textrm{Thứ $5$}; \textrm{Thứ $6$} \right\rbrace$ suy ra $n(\Omega)=5$.\\
			Trong tuần, học sinh lớp $9$B đạt được $8$ điểm $10$ là thứ $4$ nên $n(A)=1$.\\
			Xác suất của biến cố $A$ là $\mathrm{P}(A) = \dfrac{n(A)}{n(\Omega)} = \dfrac{1}{5}$.\\
			Trong tuần, không có ngày nào học sinh lớp $9$B đạt trên $9$ điểm $10$ nên $n(B)=0$.\\
			Xác suất của biến cố $B$ là $\mathrm{P}(B) = \dfrac{n(B)}{n(\Omega)} = \dfrac{0}{5}=0$.
		\end{enumerate}
	}
\end{bt}

\begin{bt}%[Dự án EX-9-Đề Cương Toán 9]%[Hiep Nguyen Quang]%[9D6H2-2]
	Kết quả bài kiểm tra môn Toán của một nhóm học sinh được cho ở bảng tần số ghép nhóm sau
	\begin{center}
		\begin{tabular}{|c|c|}
			\hline
			Điểm số & Tần số (số học sinh) \\
			\hline
			$[0;5)$ & $5$ \\
			\hline
			$[5;6{,}5)$ & $12$ \\
			\hline
			$[6{,}5;8)$ & $15$ \\
			\hline
			$[8;10]$ & $8$ \\
			\hline
		\end{tabular}
	\end{center}
	Chọn ngẫu nhiên một học sinh trong nhóm. Tính xác suất của các biến cố sau
	\begin{enumerate}
		\item $D$: \lq\lq Học sinh đó có điểm dưới $5$\rq\rq.
		\item $E$: \lq\lq Học sinh đó có điểm từ $6{,}5$ trở lên\rq\rq.
		\item $F$: \lq\lq Học sinh đó đạt yêu cầu (có điểm từ $5$ trở lên)\rq\rq.
	\end{enumerate}
	\loigiai{
		Tổng số học sinh trong nhóm là
		\[ 5 + 12 + 15 + 8 = 40 \text{ (học sinh)}.\]
		Do đó $n(\Omega) = 40$.
		\begin{enumerate}
			\item Số học sinh có điểm dưới $5$ là $5$ học sinh nên $n(D) = 5$.\\
			Xác suất của biến cố $D$ là
			\[ \mathrm{P}(D) = \dfrac{n(D)}{n(\Omega)} = \dfrac{5}{40} = \dfrac{1}{8}. \]
			\item Số học sinh có điểm từ $6,5$ trở lên là $15 + 8 = 23$ học sinh nên $n(E) = 23$.\\
			Xác suất của biến cố $E$ là
			\[ \mathrm{P}(E) = \dfrac{n(E)}{n(\Omega)} = \dfrac{23}{40}. \]
			\item Số học sinh đạt yêu cầu (điểm từ $5$ trở lên) là $40 - 5 = 35$ học sinh nên $n(F) = 35$.\\
			Xác suất của biến cố $F$ là
			\[ \mathrm{P}(F) = \dfrac{n(F)}{n(\Omega)} = \dfrac{35}{40} = \dfrac{7}{8}. \]
		\end{enumerate}
	}
\end{bt}

\begin{bt}%[Dự án EX-9-Đề Cương Toán 9]%[Hiep Nguyen Quang]%[9D6H2-2]
	\immini{
		Gieo một con xúc xắc cân đối đồng chất và có $6$ mặt. Tính xác suất của biến cố gieo được mặt có số chấm là số nguyên tố.	
	}{
		\begin{tikzpicture}[scale=0.7]
			\tikzset{sucsac/.pic={
					\begin{scope}[rounded corners=2mm]
						\def\a{1}
						\path(0,0)coordinate(A)++(-30:0.85*\a)coordinate(B)(1.5*\a,0)coordinate(C)($(A)+(C)-(B)$)coordinate(D);
						\path[shift={(0,\a)}](0,0)coordinate(A')++(-30:0.85*\a)coordinate(B')(1.5*\a,0)coordinate(C')($(A')+(C')-(B')$)coordinate(D');
						\fill[color=gray!30,draw=gray](A)--(A')--(D')--(C')--(C)--(B)--cycle;
						\fill[color=cyan!10,draw=teal](B)--(C)--(C')--(B')--cycle (A')--(B')--(C')--(D')--cycle (A)--(A')--(B')--(B)--cycle;
						\fill[color=teal,yscale=0.5,draw=teal!30]($(A')!0.5!(C')$)circle(0.2*\a)($(A')!0.5!(C')$)circle(0.2*\a);
						\path($(A')!0.5!(B')$)coordinate(M')($(A)!0.5!(B)$)coordinate(M);
						\fill[color=teal,yslant=-0.3,,draw=teal!30]($(M)!1/3!(M')$)circle(0.125*\a)($(M)!2/3!(M')$)circle(0.125*\a);
						\path($(B')!1/3!(C')$)coordinate(N')($(B) !1/3!(C)$)coordinate(N)($(B')!2/3!(C')$)coordinate(P')($(B)!2/3!(C)$)coordinate(P);
						\fill[color=teal,yslant=0.3,draw=teal!30]($(N)!1/3!(N')$)circle(0.1*\a)($(N)!2/3!(N')$)circle(0.1*\a)($(P)!1/3!(P')$)circle(0.1*\a)($(P)!2/3!(P')$)circle(0.1*\a);
					\end{scope}
			}}
			\path(-2,0)pic{sucsac};
		\end{tikzpicture}	
	}
	\loigiai{
		Phép thử có không gian mẫu là $\Omega = \{1;2;3;4;5;6\}$.\\
		Số phần tử của không gian mẫu là $n(\Omega) = 6$.\\
		Gọi $A$ là biến cố \lq\lq Gieo được mặt có số chấm là số nguyên tố\rq\rq.\\
		Các kết quả thuận lợi cho biến cố $A$ là các số nguyên tố trong tập $\Omega$. Đó là $\{2;3;5\}$.\\
		Có $3$ kết quả thuận lợi cho biến cố $A$ nên $n(A) = 3$.\\
		Vậy xác suất của biến cố $A$ là $\mathrm{P}(A) = \dfrac{n(A)}{n(\Omega)} = \dfrac{3}{6} = \dfrac{1}{2}$.
	}
\end{bt}

\begin{bt}%[Dự án EX-9-Đề Cương Toán 9]%[Hiep Nguyen Quang]%[9D6H2-2]
\immini{
	Gieo đồng thời hai con xúc xắc cân đối và đồng chất. Tính xác suất của biến cố: \lq\lq Tổng số chấm xuất hiện trên hai con xúc xắc bằng $7$\rq\rq.
}{
\begin{tikzpicture}[scale=0.7]
	\tikzset{sucsac/.pic={
			\begin{scope}[rounded corners=2mm]
				\def\a{1}
				\path(0,0)coordinate(A)++(-30:0.85*\a)coordinate(B)(1.5*\a,0)coordinate(C)($(A)+(C)-(B)$)coordinate(D);
				\path[shift={(0,\a)}](0,0)coordinate(A')++(-30:0.85*\a)coordinate(B')(1.5*\a,0)coordinate(C')($(A')+(C')-(B')$)coordinate(D');
				\fill[color=gray!30,draw=gray](A)--(A')--(D')--(C')--(C)--(B)--cycle;
				\fill[color=cyan!10,draw=teal](B)--(C)--(C')--(B')--cycle (A')--(B')--(C')--(D')--cycle (A)--(A')--(B')--(B)--cycle;
				\fill[color=teal,yscale=0.5,draw=teal!30]($(A')!0.5!(C')$)circle(0.2*\a)($(A')!0.5!(C')$)circle(0.2*\a);
				\path($(A')!0.5!(B')$)coordinate(M')($(A)!0.5!(B)$)coordinate(M);
				\fill[color=teal,yslant=-0.3,,draw=teal!30]($(M)!1/3!(M')$)circle(0.125*\a)($(M)!2/3!(M')$)circle(0.125*\a);
				\path($(B')!1/3!(C')$)coordinate(N')($(B) !1/3!(C)$)coordinate(N)($(B')!2/3!(C')$)coordinate(P')($(B)!2/3!(C)$)coordinate(P);
				\fill[color=teal,yslant=0.3,draw=teal!30]($(N)!1/3!(N')$)circle(0.1*\a)($(N)!2/3!(N')$)circle(0.1*\a)($(P)!1/3!(P')$)circle(0.1*\a)($(P)!2/3!(P')$)circle(0.1*\a);
			\end{scope}
	}}
	\path(-2,0)pic[rotate=-20]{sucsac}(1,-1)pic[rotate=30]{sucsac};
\end{tikzpicture}	
}
	\loigiai{
		Mỗi kết quả của phép thử là một cặp số $(x;y)$, trong đó $x$ là số chấm xuất hiện trên con xúc xắc thứ nhất và $y$ là số chấm xuất hiện trên con xúc xắc thứ hai.
		Ta có bảng các kết quả có thể xảy ra như sau
		\begin{center}
			\begin{tabular}{|c|c|c|c|c|c|c|}
				\hline
				\diagbox{Xúc xắc $1$}{Xúc xắc $2$} & $1$ & $2$ & $3$ & $4$ & $5$ & $6$ \\
				\hline
				$1$ & $(1;1)$ & $(1;2)$ & $(1;3)$ & $(1;4)$ & $(1;5)$ & $(1;6)$ \\
				\hline
				$2$ & $(2;1)$ & $(2;2)$ & $(2;3)$ & $(2;4)$ & $(2;5)$ & $(2;6)$ \\
				\hline
				$3$ & $(3;1)$ & $(3;2)$ & $(3;3)$ & $(3;4)$ & $(3;5)$ & $(3;6)$ \\
				\hline
				$4$ & $(4;1)$ & $(4;2)$ & $(4;3)$ & $(4;4)$ & $(4;5)$ & $(4;6)$ \\
				\hline
				$5$ & $(5;1)$ & $(5;2)$ & $(5;3)$ & $(5;4)$ & $(5;5)$ & $(5;6)$ \\
				\hline
				$6$ & $(6;1)$ & $(6;2)$ & $(6;3)$ & $(6;4)$ & $(6;5)$ & $(6;6)$ \\
				\hline
			\end{tabular}
		\end{center}
		Không gian mẫu $n(\Omega)= 6\cdot 6 = 36$ phần tử.\\
		Gọi $A$ là biến cố \lq\lq Tổng số chấm xuất hiện trên hai con xúc xắc bằng $7$\rq\rq.\\
		Các kết quả thuận lợi cho biến cố $A$ là $(1;6)$;$(2;5)$;$(3;4)$;$(4;3)$;$(5;2)$;$(6;1)$.\\
		Do đó $n(A)=6$.\\
		Vậy xác suất của biến cố $A$ là $\mathrm{P}(A) = \dfrac{n(A)}{n(\Omega)} = \dfrac{6}{36} = \dfrac{1}{6}$.
	}
\end{bt}

\begin{bt}%[Dự án EX-9-Đề Cương Toán 9]%[Hiep Nguyen Quang]%[9D6V2-2]
	Một túi vải chứa $12$ viên bi màu xanh và một số viên bi màu vàng, tất cả đều có kích thước và khối lượng như nhau. Lấy ngẫu nhiên một viên bi từ trong túi. Biết xác suất của biến cố \lq\lq Lấy được viên bi màu vàng\rq\rq\, là $\dfrac{2}{5}$. Hỏi trong túi có bao nhiêu viên bi màu vàng?
	\loigiai{
		Gọi $x$ là số viên bi màu vàng có trong túi (điều kiện: $x$ là số nguyên dương).\\		
		Tổng số viên bi trong túi là $x + 12$ (viên).\\
		Số cách chọn ra ngẫu nhiên $1$ viên bi từ túi là $x+12$ (viên).\\		
		Số kết quả thuận lợi cho biến cố \lq\lq Lấy được viên bi màu vàng\rq\rq\, là $x$ nên xác suất của biến cố này là
		$$ \mathrm{P} = \dfrac{x}{x+12}. $$
		Theo giả thiết, ta có phương trình
		\allowdisplaybreaks
		\begin{eqnarray*}
			\dfrac{x}{x+12} &=& \dfrac{2}{5}\\
			5x &=& 2(x+12) \\
			5x &=& 2x + 24 \\
			5x - 2x &=& 24 \\
			3x &=& 24 \\
			x &=& 8.
		\end{eqnarray*}
		Giá trị $x=8$ thỏa mãn điều kiện.\\
		Vậy trong túi có $8$ viên bi màu vàng.
	}
\end{bt}

\begin{bt}%[Dự án EX-9-Đề Cương Toán 9]%[Hiep Nguyen Quang]%[9D6H2-2]
	Một đội văn nghệ có bốn bạn, trong đó có hai bạn nữ là Dung và Ánh, hai bạn nam là Minh và Quân. Cô tổng phụ trách chọn ngẫu nhiên hai bạn để hát song ca. Tính xác suất của biến cố: \lq\lq Trong hai bạn được chọn có một bạn là Minh\rq\rq.
	\loigiai{
		Để chọn ngẫu nhiên $2$ bạn từ $4$ bạn, ta có các cặp có thể xảy ra là
		$$\Omega = \left\{(\text{Dung}, \text{Ánh}), (\text{Dung}, \text{Minh}), (\text{Dung}, \text{Quân}), (\text{Ánh}, \text{Minh}), (\text{Ánh}, \text{Quân}), (\text{Minh}, \text{Quân})\right\}.$$
		Vậy có tổng cộng $6$ kết quả có thể xảy ra nên $n(\Omega)=6$.\\		
		Gọi $A$ là biến cố \lq\lq Trong hai bạn được chọn có một bạn là Minh\rq\rq. \\
		Các kết quả thuận lợi cho biến cố $A$ là
		$$\left\{(\text{Dung}, \text{Minh}), (\text{Ánh}, \text{Minh}), (\text{Minh}, \text{Quân})\right\}.$$
		Có $3$ kết quả thuận lợi cho biến cố $A$ nên $n(A)=3$.\\		
		Khi đó
		\[\mathrm{P}(A)=\dfrac{n(A)}{n(\Omega)}=\dfrac{3}{6}=\dfrac{1}{2}. \]
		Vậy xác suất để trong hai bạn được chọn có một bạn là Minh là $\dfrac{1}{2}$.
	}
\end{bt}

\begin{bt}%[Dự án EX-9-Đề Cương Toán 9]%[Hiep Nguyen Quang]%[9D6V2-2]
	Ở một loài thực vật, tính trạng màu sắc hoa do một gen quy định, trong đó allele $A$ quy định hoa đỏ là trội hoàn toàn so với allele $a$ quy định hoa trắng. Tính trạng chiều cao cây do một gen khác quy định, trong đó allele $B$ quy định thân cao là trội hoàn toàn so với allele $b$ quy định thân thấp. Các cặp gen này phân li độc lập.	Cho lai hai cây bố mẹ đều dị hợp về cả hai cặp gen ($P: AaBb \times AaBb$). Tính xác suất của biến cố $F$: \lq\lq Cây con có kiểu hình hoa đỏ, thân cao\rq\rq.
	\loigiai{
		Phép lai giữa hai cây bố mẹ là $P: AaBb \times AaBb$.
		Mỗi cây bố mẹ dị hợp $2$ cặp gen sẽ tạo ra $4$ loại giao tử với tỉ lệ bằng nhau là $AB$, $Ab$, $aB$, $ab$.\\		
		Ta có bảng kết quả của phép lai
		\begin{center}
			\begin{tabular}{|c|c|c|c|c|}
				\hline
				Giao tử & $AB$ & $Ab$ & $aB$ & $ab$ \\
				\hline
				$AB$ & $AABB$ & $AABb$ & $AaBB$ & $AaBb$ \\
				\hline
				$Ab$ & $AABb$ & $AAbb$ & $AaBb$ & $Aabb$ \\
				\hline
				$aB$ & $AaBB$ & $AaBb$ & $aaBB$ & $aaBb$ \\
				\hline
				$ab$ & $AaBb$ & $Aabb$ & $aaBb$ & $aabb$ \\
				\hline
			\end{tabular}
		\end{center}
		Không gian mẫu $\Omega$ bao gồm $16$ tổ hợp kiểu gen có khả năng xuất hiện như nhau nên $n(\Omega)=16$.\\		
		Gọi $A$ là biến cố \lq\lq Cây con có kiểu hình hoa đỏ, thân cao\rq\rq.\\
		Kiểu hình \lq\lq hoa đỏ, thân cao\rq\rq\, là kiểu hình trội về cả hai tính trạng. Nó được quy định bởi các kiểu gen có ít nhất một allele trội $A$ (quy định hoa đỏ) và một allele trội $B$ (quy định thân cao).\\		
		Các kiểu gen quy định kiểu hình này là $AABB, AABb, AaBB, AaBb$.\\
		Đếm trên bảng, ta có
		\begin{itemize}
			\item $1$ tổ hợp $AABB$;
			\item $2$ tổ hợp $AABb$;
			\item $2$ tổ hợp $AaBB$;
			\item $4$ tổ hợp $AaBb$.
		\end{itemize}
		Tổng số kết quả thuận lợi cho biến cố $A$ là $n(A)=1+2+2+4=9$.\\		
		Vậy xác suất của biến cố $A$ là	$\mathrm{P}(A)=\dfrac{n(A)}{n(\Omega)}=\dfrac{9}{16}$.
	}
\end{bt}

