\section*{BÀI TẬP CUỐI CHƯƠNG}
\subsection{Câu hỏi trắc nghiệm}
\Opensolutionfile{ans}[ans/ans-8C1-OTC]
\begin{ex}%[Đề cương Toán THCS - Nguyễn Văn Cường (Cường NV)]%[8D2B1-1]
	Điều kiện xác định của phân thức $\dfrac1{x-3}$ là
	\choice
	{$x-3>0$}
	{$x-3<0$}
	{\True $x-3\ne0$}
	{$x-3=0$}
	\loigiai{
		Điều kiện xác định của phân thức $\dfrac1{x-3}$ là $x-3\ne0$.
	}
\end{ex}

\begin{ex}%[Đề cương Toán THCS - Nguyễn Văn Cường (Cường NV)]%[8D2B1-4]
Rút gọn phân thức $\dfrac{2x^2 + 4x}{2x}$ ta được	
\choice
{\True $x+2$ }
{ $x+1$ }
{ $x$ }
{ $2x+1$ }
\loigiai{
Ta có $\dfrac{2x^2 + 4x}{2x} = \dfrac{2x(x+2)}{2x} = x+2$.
	}
\end{ex}

\begin{ex}%[Đề cương Toán THCS - Nguyễn Văn Cường (Cường NV)]%[8D2B1-6]
Giá trị của phân thức $\dfrac{x^2 - 9}{x-3}$ khi $x = 5$ là bao nhiêu?
\choice
	{ $7$ }
	{\True $8$ }
	{ $6$ }
	{ $10$ }
	\loigiai{
Ta có $\dfrac{x^2 - 9}{x-3} = \dfrac{(x-3)(x+3)}{x-3} = x+3$.\\
Khi $x = 5$ ta được $x + 3 = 5 + 3 = 8$.  
	}
\end{ex}




\begin{ex}%[Đề cương Toán THCS - Nguyễn Văn Cường (Cường NV)]%[8D2H1-6]
	Giá trị của phân thức $M=\dfrac1{3+x}+\dfrac1{3-x}$ tại $x=0{,}5$ là
	\choice
	{$\dfrac{22}{37}$}
	{$\dfrac{22}{35}$}
	{\True $\dfrac{24}{35}$}
	{$\dfrac{24}{37}$}
	\loigiai{
		Tại $x=0{,}5$ ta có
		\allowdisplaybreaks
		\begin{eqnarray*}
			M&=&\dfrac1{3+0{,}5}+\dfrac1{3-0{,}5}=\dfrac1{3{,}5}+\dfrac1{2{,}5}\\
			&=&\dfrac27+\dfrac25=\dfrac{24}{35}.
		\end{eqnarray*}
	}
\end{ex}

\begin{ex}%[Đề cương Toán THCS - Nguyễn Văn Cường (Cường NV)]%[8D2H3-3]
	Thương của phép chia phân thức $\dfrac{y^3-x^3}{6x^3y}$ cho phân thức $\dfrac{x^2+xy+y^2}{2xy}$ là
	\choice
	{$\dfrac{y-x}{3x}$}
	{$\dfrac{x-y}{3x^2}$}
	{$\dfrac{x-y}{3x}$}
	{\True $\dfrac{y-x}{3x^2}$}
	\loigiai{
		Ta có $\dfrac{y^3-x^3}{6x^3y}\colon\dfrac{x^2+xy+y^2}{2xy}=\dfrac{(y-x)(y^2+xy+y^2)}{6x^3y}\cdot\dfrac{2xy}{x^2+xy+y^2}=\dfrac{y-x}{3x^2}$.
	}
\end{ex}

\begin{ex}%[Đề cương Toán THCS - Nguyễn Văn Cường (Cường NV)]%[8D2H1-2]
	Biểu thức nào sau đây không phải là phân thức đại số?
	\choice
	{$2 x+1$}
	{$\sqrt{5}$}
	{$\pi$}
	{\True $\sqrt{x}$}
	\loigiai{
		Theo định nghĩa, biểu thức $\sqrt{x}$ không phải là phân thức đại số.
	}	
\end{ex}

\begin{ex}%[Đề cương Toán THCS - Nguyễn Văn Cường (Cường NV)]%[8D2H1-3]
	Phân thức nào sau đây bằng phân thức $\dfrac{16 x^4-1}{12 x^3-3 x}$?
	\choice
	{$\dfrac{4 x^2-1}{3 x}$}
	{\True $\dfrac{4 x^2+1}{3 x}$}
	{$\dfrac{4 x^2-1}{4 x-3}$}
	{$\dfrac{4 x^2+1}{4-3 x}$}
	\loigiai{
		$\dfrac{16 x^4-1}{12 x^3-3 x}=\dfrac{(4x^2-1)(4x^2+1)}{3x(4x^2-1)}=\dfrac{4x^2+1}{3x}$.
	}
\end{ex}

\begin{ex}%[Đề cương Toán THCS - Nguyễn Văn Cường (Cường NV)]%[8D2H1-5]
	Đa thức nào sau đây không thể chọn làm mẫu thức chung của hai phân thức $\dfrac{x}{3\left(x^2-1\right)(x+2)}$ và $\dfrac{x^3-x+1}{\left(x^2-4\right)\left(x^3+1\right)}$?
	\choice
	{$3\left(x^2-1\right)\left(x^2-4\right)\left(x^2-x+1\right)$}
	{$3\left(x^2-1\right)\left(x^2-4\right)\left(x^3+1\right)$}
	{\True $3\left(x^2-1\right)\left(x^2-4\right)\left(x^2+x+1\right)$}
	{$3\left(x^4-1\right)\left(x^6-1\right)\left(x^6-64\right)$}
	\loigiai{
		Ta có $3\left(x^2-1\right)(x+2)=3(x-1)(x+1)(x+2)$.\\
		$\left(x^2-4\right)\left(x^3+1\right)=(x-2)(x+2)(x+1)(x^2-x+1)$.\\
		Do đó mẫu chung của hai phân thức là $3(x-2)(x+2)(x+1)(x^2-x+1)$.
	}
\end{ex}

\begin{ex}%[Đề cương Toán THCS - Nguyễn Văn Cường (Cường NV)]%[8D2H1-6]
	Giá trị của phân thức $\dfrac{8 x-4}{8 x^3-1}$ tại $x=-0{,}5$ là
	\choice
	{\True $4$}
	{$-4$}
	{$0{,}25$}
	{$-0{,}25$}
	\loigiai{
		Ta có $\dfrac{8 x-4}{8 x^3-1}=\dfrac{4(2x-1)}{(2x-1)(4x^2+2x+1)}=\dfrac{4}{4x^2+2x+1}$.\\
		Với $x=-0{,}5$ ta có $\dfrac{4}{4\cdot(-0{,}5)^2+2\cdot(-0{,}5)+1}=\dfrac{4}{1-1+1}=4$.
	}
\end{ex}

\begin{ex}%[Đề cương Toán THCS - Nguyễn Văn Cường (Cường NV)]%[8D2H1-4]
	Rút gọn biểu thức $\dfrac{x-1}{x^3+1}+\dfrac{1-2 x}{x-1}-\dfrac{3 x+2}{x^3+1}+\dfrac{1-x}{x^3+1}+\dfrac{3 x}{x^3+1}+\dfrac{1-2 x}{1-x}$, ta được kết quả là
	\choice
	{$\dfrac{2}{x-1}$}
	{\True $\dfrac{-2}{x^3+1}$}
	{$\dfrac{2}{x^3+1}$}
	{$\dfrac{2}{x+1}$}
	\loigiai{
		Ta có 
		\allowdisplaybreaks
		\begin{eqnarray*}
			&&\dfrac{x-1}{x^3+1}+\dfrac{1-2 x}{x-1}-\dfrac{3 x+2}{x^3+1}+\dfrac{1-x}{x^3+1}+\dfrac{3 x}{x^3+1}+\dfrac{1-2 x}{1-x}\\
			&=&\left(\dfrac{x-1}{x^3+1}-\dfrac{3 x+2}{x^3+1}+\dfrac{1-x}{x^3+1}+\dfrac{3 x}{x^3+1}\right)+\left(\dfrac{1-2 x}{x-1}+\dfrac{1-2 x}{1-x}\right)\\
			&=&\dfrac{x-1-3x-2+1-x+3x}{x^3+1}+\dfrac{1-2x-1+2x}{x-1}=\dfrac{-2}{x^3+1}.
		\end{eqnarray*}
	}
\end{ex}

\begin{ex}%[Nguyễn Văn Cường (Cường NV) - ]%[8D2H1-3]
	Khẳng định nào sau đây là đúng?
	\choice
	{$\dfrac{(x-1)^2}{x-2}=\dfrac{(1-x)^2}{2-x}$}
	{$\dfrac{3 x}{(x+2)^2}=\dfrac{3 x}{(x-2)^2}$}
	{$\dfrac{3 x}{(x+2)^2}=\dfrac{-3 x}{(x-2)^2}$}
	{\True $\dfrac{3 x}{(x+2)^2}=\dfrac{3 x}{(-x-2)^2}$}
	\loigiai{Khẳng định đúng là $\dfrac{3 x}{(x+2)^2}=\dfrac{3 x}{(-x-2)^2}$.}
\end{ex}


\begin{ex}%%[Đề cương Toán THCS - Nguyễn Văn Cường (Cường NV)]%[8D2H1-3]
	Khẳng định nào sau đây là {\bf sai}?
	\choice
	{$\dfrac{-6 x}{-4 x^2(x+2)^2}=\dfrac{3}{2 x(x+2)^2}$}
	{$\dfrac{-5}{-2}=\dfrac{10 x}{4 x}$}
	{\True $\dfrac{x+1}{x-1}=\dfrac{x^2+x+1}{x^2-x+1}$}
	{$\dfrac{-6 x}{-4(-x)^2(x-2)^2}=\dfrac{3}{2 x(-x+2)^2}$}
	\loigiai{Khẳng định {\bf sai} là khẳng định $\dfrac{x+1}{x-1}=\dfrac{x^2+x+1}{x^2-x+1}$.}
\end{ex}

%%=====Bài 6.38
\begin{ex}%[Đề cương Toán THCS - Nguyễn Văn Cường (Cường NV)]%[8D2H1-3]
	Trong đẳng thức $\dfrac{2 x^2+1}{4 x-1}=\dfrac{8 x^3+4 x}{Q}$; $Q$ là đa thức
	\choice
	{$4 x$}
	{$4 x^2$}
	{$16 x-4$}
	{\True $16 x^2-4 x$}	
	\loigiai{Ta có $\dfrac{2 x^2+1}{4 x-1}= \dfrac{4x (2x^2 + 1)}{4x(4x-1)} = \dfrac{8x^3 + 4x}{16x^2 - 4x}$.\\
		Do đó $Q = 16 x^2-4 x$.}
\end{ex}


\begin{ex}%[Đề cương Toán THCS - Nguyễn Văn Cường (Cường NV)]%[8D2V1-3]
	Nếu $\dfrac{-5 x+5}{2 x y}-\dfrac{-9 x-7}{2 x y}=\dfrac{b x+c}{x y}$ thì $b+c$ bằng?
	\choice 
	{$-4$}
	{$8$}
	{$4$}
	{$-10$}
	\loigiai{Ta có $\dfrac{-5 x+5}{2 x y}-\dfrac{-9 x-7}{2 x y}= \dfrac{-5x+5+9x+7}{2xy} = \dfrac{4x + 12}{2xy} = \dfrac{2(2x +6)}{2xy} = \dfrac{2x +6}{xy}$.\\
		Suy ra $b =2$; $c = 6$, nên $b + c = 8$.}
\end{ex}

%Bài 6.40
\begin{ex}%[Đề cương Toán THCS - Nguyễn Văn Cường (Cường NV)]%[8D2H1-6]
	Một ngân hàng huy động vốn với mức lãi suất một năm (tính theo $\%$) là $x$. Để sau một năm, người gửi được lãi $a$ đồng thì người đó phải gửi vào ngân hàng số tiền là
	\choice{\True$\dfrac{100a}{x}$ đồng}
	{$\dfrac{a}{x+100}$ đồng}
	{$\dfrac{a}{x+1}$ đồng}
	{$\dfrac{100a}{x+100}$}
	\loigiai{
		Số tiền gửi là $a : x\% = \dfrac{a\cdot 100}{x}$.	
	}
\end{ex}
\subsection{Bài tập tự luận}
\setcounter{bt}{0}
\begin{bt}%[Đề cương Toán THCS - Nguyễn Văn Cường (Cường NV)]%[8D2H2-4]
Thực hiện phép tính
\begin{multicols}{2}
	\begin{enumerate}
		\item $\dfrac{x}{xy+y^2}-\dfrac{y}{x^2+xy}$.
		\item $\dfrac{x^2+4}{x^2-4}- \dfrac{x}{x+2}- \dfrac{x}{2-x}$.
		\item $\dfrac{a^2+ab}{b-a}: \dfrac{a+b}{2a^2-2b^2}$.
		\item $\left(\dfrac{2x+1}{2x-1}- \dfrac{2x-1}{2x+1} \right): \dfrac{4x}{10x-5}$.
	\end{enumerate}
\end{multicols}
	\loigiai{
		\begin{enumerate}
			\item Điều kiện xác định $\heva{&xy+y^2 \ne 0\\ &x^2+xy \ne 0} \Rightarrow \heva{&x(x+y) \ne 0\\ &y(x+y) \ne 0} \Rightarrow \heva{&x \ne 0\\ &y \ne 0\\ &x \ne -y.}$
			\allowdisplaybreaks
			\begin{eqnarray*}
				\dfrac{x}{xy+y^2}-\dfrac{y}{x^2+xy} &=& \dfrac{x}{(x+y)y}-\dfrac{y}{x(x+y)}\\
				&=& \dfrac{x^2}{xy(x+y)}- \dfrac{y^2}{xy(x+y)}\\
				&=& \dfrac{x^2-y^2}{xy(x+y)}\\
				&=& \dfrac{(x-y)(x+y)}{xy(x+y)}\\
				&=& \dfrac{x-y}{xy}.
			\end{eqnarray*}
			\item Điều kiện xác định $\heva{&x^2-4 \ne 0\\ &x+2 \ne 0\\ &2-x \ne 0} \Rightarrow \heva{&x \ne \pm 2\\ &x \ne -2\\ &x \ne 2} \Rightarrow x \ne \pm 2.$
			\allowdisplaybreaks
			\begin{eqnarray*}
				\dfrac{x^2+4}{x^2-4}- \dfrac{x}{x+2}- \dfrac{x}{2-x} &=& \dfrac{x^2+4}{(x+2)(x-2)}- \dfrac{x}{x+2}- \dfrac{x}{2-x}\\
				&=& \dfrac{x^2+4}{(x+2)(x-2)}- \dfrac{x(x-2)}{(x+2)(x-2)}+ \dfrac{x(x+2)}{(x+2)(x-2)}\\
				&=& \dfrac{x^2+4-x^2+2x+x^2+2x}{(x+2)(x-2)} = \dfrac{x^2+4x+4}{(x+2)(x-2)} \\
				&=& \dfrac{(x+2)^2}{(x+2)(x-2)} = \dfrac{x+2}{x-2}.
			\end{eqnarray*}
			\item Điều kiện xác định $\heva{&b-a \ne 0\\ &2a^2-2b^2 \ne 0} \Rightarrow \heva{&a \ne b\\ &a \ne \pm b} \Rightarrow a \ne \pm b$.
			\allowdisplaybreaks
			\begin{eqnarray*}
				\dfrac{a^2+ab}{b-a}: \dfrac{a+b}{2a^2-2b^2} &=& -\dfrac{a(a+b)}{a-b}: \dfrac{a+b}{2\left(a^2-b^2\right)}\\
				&=& -\dfrac{a(a+b)}{a-b}: \dfrac{a+b}{2\left(a-b\right)\left(a+b\right)}\\
				&=& -\dfrac{a(a+b)}{a-b}\cdot \dfrac{2\left(a-b\right)\left(a+b\right)}{a+b}\\
				&=& -2a(a+b).
			\end{eqnarray*}
			\item Điều kiện xác định $\heva{&2x-1 \ne 0\\ &2x+1 \ne 0\\ &10x-5 \ne 0} \Rightarrow \heva{&x \ne \dfrac{1}{2}\\ &x\ne -\dfrac{1}{2}\\ &x \ne \dfrac{1}{2}} \Rightarrow x \ne \pm \dfrac{1}{2}$.
			\allowdisplaybreaks
			\begin{eqnarray*}
				\left(\dfrac{2x+1}{2x-1}- \dfrac{2x-1}{2x+1} \right): \dfrac{4x}{10x-5} &=& \left(\dfrac{(2x+1)^2}{(2x-1)(2x+1)}- \dfrac{(2x-1)^2}{(2x-1)(2x+1)} \right): \dfrac{4x}{5(2x-1)}\\
				&=& \dfrac{(2x+1)^2-(2x-1)^2}{(2x-1)(2x+1)}: \dfrac{4x}{5(2x-1)}\\
				&=& \dfrac{8x}{(2x-1)(2x+1)}\cdot \dfrac{5(2x-1)}{4x}\\
				&=& \dfrac{10}{2x+1}.
			\end{eqnarray*}
		\end{enumerate}
	}
\end{bt}

% Bài 3
\begin{bt}%[Đề cương Toán THCS - Nguyễn Văn Cường (Cường NV)]%[8D2H3-4]
Rút gọn biểu thức $P = \left(x-\dfrac{x^2+y^2}{x+y}\right) \cdot\left(\dfrac{2 x}{y}+\dfrac{4 x}{x-y}\right): \dfrac{1}{y}$ $(y \neq 0, y \neq x, y \neq-x)$.	
	\loigiai{
		Ta có \allowdisplaybreaks
		\begin{eqnarray*}
			P&=&\left(x-\dfrac{x^2+y^2}{x+y}\right) \cdot\left(\dfrac{2 x}{y}+\dfrac{4 x}{x-y}\right): \dfrac{1}{y}\\
			&=&\dfrac{x^2+xy-x^2-y^2}{x+y}\cdot \dfrac{2x^2-2xy+4xy}{y(x-y)}\cdot y\\
			&=&\dfrac{y(x-y)}{x+y}\cdot \dfrac{2x^2+2xy}{x-y}\\
			&=&\dfrac{y(x-y)}{x+y}\cdot \dfrac{2x(x+y)}{x-y}\\
			&=&2xy.
		\end{eqnarray*}	
	}
\end{bt}

\begin{bt}%[Đề cương Toán THCS - Nguyễn Văn Cường (Cường NV)]%[8D2H3-4]
	Rút gọn các biểu thức sau
	\begin{multicols}{2}
		\begin{enumerate}
			\item $\dfrac{2}{3x} + \dfrac{x}{x-1} + \dfrac{6x^2-4}{2x(1-x)}$;
			\item $\dfrac{x^3+1}{1-x^3} + \dfrac{x}{x-1} - \dfrac{x+1}{x^2 + x +1}$;
			\item $\left(\dfrac{2}{x+2} - \dfrac{2}{1-x}\right) \cdot \dfrac{x^2-4}{4x^2 - 1}$;
			\item $1 + \dfrac{x^3 - x}{x^2 + 1}\left(\dfrac{1}{1-x} - \dfrac{1}{1+x^2}\right)$. 
		\end{enumerate}
	\end{multicols}
	\loigiai{
		\begin{enumerate}
			\item Ta có
			\allowdisplaybreaks
			\begin{eqnarray*}
				& &\dfrac{2}{3x} + \dfrac{x}{x-1} + \dfrac{6x^2-4}{2x(1-x)}\\
				&= & \dfrac{2}{3x} + \dfrac{x}{x-1} - \dfrac{6x^2-4}{2x(x-1)}\\
				&=&\dfrac{2\cdot 2(x-1)}{6x(x-1)} + \dfrac{6x^2}{6x(x-1)} - \dfrac{18x^2-12}{6x(x-1)}\\
				&=& \dfrac{4x-4 + 6x^2 - 18x^2 + 12}{6x(x-1)}\\
				&=&\dfrac{-12x^2 + 4x + 8}{6x(x-1)}\\
				&=&\dfrac{-4(x-1)\left(3x+2\right)}{6x(x-1)}\\
				&=&\dfrac{-2\left(3x+2\right)}{3x}.
			\end{eqnarray*}
			\item Ta có
			\allowdisplaybreaks
			\begin{eqnarray*}
				&&\dfrac{x^3+1}{1-x^3} + \dfrac{x}{x-1} - \dfrac{x+1}{x^2 + x +1}\\
				&=&\dfrac{x^3+1}{1-x^3}-\dfrac{x}{1-x}-\dfrac{x+1}{x^2+x+1}\\
				&=&\dfrac{x^3+1}{(1-x)(1 + x + x^2)} -\dfrac{x(1+x+x^2)}{(1-x)(1+x+x^2)} - \dfrac{(x+1)(1-x)}{(1-x)(1+x+x^2)}\\
				&=&\dfrac{x^3+1 - x-x^2-x^3 + x^2 -1}{(1-x)(1+x+x^2)}\\
				&=&\dfrac{-x}{(1-x)(1+x+x^2)}
			\end{eqnarray*}
			\item Ta có
			\allowdisplaybreaks
			\begin{eqnarray*}
				&&\left(\dfrac{2}{x+2} - \dfrac{2}{1-x}\right) \cdot \dfrac{x^2-4}{4x^2 - 1}\\
				&=&\dfrac{2(1-x)-2(x+2)}{(x+2)(1-x)}\cdot\dfrac{(x+2)(x-2)}{(2x+1)(2x-1)}\\
				&=&\dfrac{-2(2x+1)}{(x+2)(1-x)}\cdot\dfrac{(x+2)(x-2)}{(2x+1)(2x-1)}\\
				&=&\dfrac{-2(x-2)}{(1-x)(2x-1)}.
			\end{eqnarray*}
			\item Ta có
			\allowdisplaybreaks
			\begin{eqnarray*}
				&&1 + \dfrac{x^3 - x}{x^2 + 1}\left(\dfrac{1}{1-x} - \dfrac{1}{1-x^2}\right)\\
				&=&1+ \dfrac{x(x+1)(x-1)}{x^2+1} \dfrac{1+x-1}{(1-x)(1+x)}\\
				&=&1 + \dfrac{x(x+1)(x-1)}{x^2+1}\cdot \dfrac{-x}{(x-1)(x+1)}\\
				&=& \dfrac{x^2+1}{x^2+1} + \dfrac{-x^2}{x^2+1}\\
				& = &\dfrac{1}{x^2+1}.
			\end{eqnarray*} 
		\end{enumerate}
	}
\end{bt}

\begin{bt}%[Đề cương Toán THCS - Nguyễn Văn Cường (Cường NV)]%[8D2H3-5]
	Cho biểu thức
	$$A= \left(\dfrac{x+1}{2x-2}+ \dfrac{3}{x^2-1}- \dfrac{x+3}{2x+2} \right)\cdot \dfrac{4x^2-4}{5}.$$
	\begin{enumerate}
		\item Viết điều kiện xác định của biểu thức $A$.
		\item Chứng minh giá trị của biểu thức $A$ không phụ thuộc vào giá trị của biến.
	\end{enumerate}
	\loigiai{
		\begin{enumerate}
			\item Điều kiện xác định $\heva{&2x-2 \ne 0\\ &x^2-1 \ne 0\\ &2x+2 \ne 0} \Rightarrow \heva{&x \ne 1\\ &x \ne \pm 1\\ & x \ne -1} \Rightarrow x \ne \pm 1$.
			\item Ta có
			\allowdisplaybreaks
			\begin{eqnarray*}
				A &=& \left(\dfrac{x+1}{2x-2}+ \dfrac{3}{x^2-1}- \dfrac{x+3}{2x+2} \right)\cdot \dfrac{4x^2-4}{5}\\ 
				&=& \left(\dfrac{x+1}{2(x-1)}+ \dfrac{3}{(x-1)(x+1)}- \dfrac{x+3}{2(x+1)} \right)\cdot \dfrac{4(x^2-1)}{5}\\
				&=& \left(\dfrac{(x+1)^2}{2(x-1)(x+1)}+ \dfrac{6}{2(x-1)(x+1)}- \dfrac{(x+3)(x-1)}{2(x-1)(x+1)} \right)\cdot \dfrac{4(x-1)(x+1)}{5}\\
				&=& \dfrac{(x+1)^2+6-(x+3)(x-1)}{2(x-1)(x+1)}\cdot \dfrac{4(x-1)(x+1)}{5}\\
				&=& \dfrac{10}{2(x-1)(x+1)}\cdot \dfrac{4(x-1)(x+1)}{5} = 4.
			\end{eqnarray*}
			Vậy giá trị của biểu thức $A$ không phụ thuộc vào giá trị của biến.
		\end{enumerate}
	}
\end{bt}


\begin{bt}%[Đề cương Toán THCS - Nguyễn Văn Cường (Cường NV)]%[8D2H2-4]
	Tìm đa thức $P$ trong các đẳng thức sau
	\begin{multicols}{2}
		\begin{enumerate}
			\item $P+\dfrac{1}{x+2}=\dfrac{x}{x^2-2 x+4}$;
			\item $P-\dfrac{4(x-2)}{x+2}=\dfrac{16}{x-2}$;
			\item $P \cdot \dfrac{x-2}{x+3}=\dfrac{x^2-4 x+4}{x^2-9}$;
			\item $P: \dfrac{x^2-9}{2 x+4}=\dfrac{x^2-4}{x^2+3 x}$.	
		\end{enumerate}
	\end{multicols}
	\loigiai{
		\begin{enumerate}
			\item Ta có
			{\allowdisplaybreaks
				\begin{eqnarray*}
					&&P+\dfrac{1}{x+2}=\dfrac{x}{x^2-2 x+4}\\
					&\Rightarrow& P =\dfrac{x}{x^2-2 x+4} - \dfrac{1}{x+2}\\
					&\Rightarrow& P = \dfrac{x(x+2)}{(x+2)(x^2-2x+4)} - \dfrac{x^2-2x+4}{(x+2)(x^2-2x+4)}\\
					&\Rightarrow& P = \dfrac{x^2+2x}{(x+2)(x^2-2x+4)} - \dfrac{x^2-2x+4}{(x+2)(x^2-2x+4)}\\
					&\Rightarrow& P = \dfrac{x^2+2x-x^2 + 2x -4}{(x+2)(x^2-2x+4)}\\
					&\Rightarrow& P = \dfrac{4x-4}{(x+2)(x^2-2x+4)}.
			\end{eqnarray*}}
			Vậy $P = \dfrac{4x-4}{(x+2)(x^2-2x+4)}$.
			\item Ta có
			{\allowdisplaybreaks
				\begin{eqnarray*}
					&&P-\dfrac{4(x-2)}{x+2}=\dfrac{16}{x-2}\\
					&\Rightarrow& P=\dfrac{16}{x-2} + \dfrac{4(x-2)}{x+2}\\	
					&\Rightarrow& P=\dfrac{16(x+2)}{(x-2)(x+2)} + \dfrac{4(x-2)^2}{(x-2)(x+2)}\\	
					&\Rightarrow& P=\dfrac{16x+32}{(x-2)(x+2)} + \dfrac{4(x^2-4x + 4)}{(x-2)(x+2)}\\	
					&\Rightarrow& P=\dfrac{16x+32}{(x-2)(x+2)} + \dfrac{4x^2-16x + 16}{(x-2)(x+2)}\\
					&\Rightarrow& P=\dfrac{16x+32 + 4x^2 - 16x + 16}{(x-2)(x+2)}\\
					&\Rightarrow& P=\dfrac{4x^2 + 48}{(x-2)(x+2)}.
			\end{eqnarray*}}
			Vậy $P=\dfrac{4x^2 + 48}{(x-2)(x+2)}$.
			\item Ta có
			{\allowdisplaybreaks
				\begin{eqnarray*}
					&&P \cdot \dfrac{x-2}{x+3}=\dfrac{x^2-4 x+4}{x^2-9}\\	
					&\Rightarrow& P = \dfrac{x^2-4 x+4}{x^2-9} : \dfrac{x-2}{x+3}\\
					&\Rightarrow& P = \dfrac{(x-2)^2}{(x-3)(x+3)} : \dfrac{x-2}{x+3}\\	
					&\Rightarrow& P = \dfrac{(x-2)^2}{(x-3)(x+3)} \cdot \dfrac{x+3}{x-2}\\
					&\Rightarrow& P = \dfrac{(x-2)^2 (x+3)}{(x-3)(x+3)(x-2)}\\
					&\Rightarrow& P = \dfrac{x-2}{x-3}.
			\end{eqnarray*}}
			Vậy $P = \dfrac{x-2}{x-3}$.
			\item Ta có
			{\allowdisplaybreaks
				\begin{eqnarray*}
					&&P: \dfrac{x^2-9}{2 x+4}=\dfrac{x^2-4}{x^2+3 x}\\
					&\Rightarrow&	P=\dfrac{x^2-4}{x^2+3 x} \cdot \dfrac{x^2-9}{2 x+4}\\	
					&\Rightarrow&	P=\dfrac{(x-2)(x+2)}{x(x+3)} \cdot \dfrac{(x-3)(x+3)}{2(x+2)}\\
					&\Rightarrow&	P=\dfrac{(x-2)(x+2)(x-3)(x+3)}{2x(x+2)(x+3)}\\
					&\Rightarrow&	P=\dfrac{(x-2)(x-3)}{2x}.	
			\end{eqnarray*}}
			Vậy $P=\dfrac{(x-2)(x-3)}{2x}$.
	\end{enumerate}}
\end{bt}

\begin{bt}%[Đề cương Toán THCS - Nguyễn Văn Cường (Cường NV)]%[8D2H3-4]
	Rút gọn rồi tính giá trị biểu thức:
	\begin{enumerate}
		\item $A=\left(\dfrac{x^2+y^2}{x^2-y^2}-1\right)\cdot\dfrac{x-y}{2y}$ tại $x=5$; $y=7$;
		\item $B=\dfrac{2x+y}{2x^2-xy}+\dfrac{8y}{y^2-4x^2}+\dfrac{2x-y}{2x^2+xy}$ tại $x=-\dfrac12$; $y=\dfrac32$;
		\item $C=\left(\dfrac{x^2}{y}-\dfrac{y^2}{x}\right)\left(\dfrac{x+y}{x^2+xy+y^2}+\dfrac1{x-y}\right)-\dfrac xy$ tại $x=-15$; $y=5$.
	\end{enumerate}
	\loigiai{
		\begin{enumerate}
			\item $A=\left(\dfrac{x^2+y^2}{x^2-y^2}-1\right)\cdot\dfrac{x-y}{2y}=\dfrac{x^2+y^2-x^2+y^2}{(x-y)(x+y)}\cdot\dfrac{x-y}{2y}=\dfrac{y}{x+y}$.\\
			Tại $x=5$; $y=7$ ta có $A=\dfrac7{12}$.
			\item Ta có 
			\begin{eqnarray*}
				B&=&\dfrac{2x+y}{2x^2-xy}+\dfrac{8y}{y^2-4x^2}+\dfrac{2x-y}{2x^2+xy}=\dfrac{2x+y}{x(2x-y)}-\dfrac{8y}{(2x-y)(2x+y)}+\dfrac{2x-y}{x(2x+y)}\\
				&=&\dfrac{(2x+y)^2-8xy+(2x-y)^2}{x(2x-y)(2x+y)}=\dfrac{2(2x-y)^2}{x(2x-y)(2x+y)}\\
				&=&\dfrac{2(2x-y)}{x(2x+y)}.
			\end{eqnarray*}
			Tại $x=-\dfrac12$; $y=\dfrac32$ ta có $B=20$.
			\item Ta có 
			\begin{eqnarray*}
				C&=&\left(\dfrac{x^2}{y}-\dfrac{y^2}{x}\right)\left(\dfrac{x+y}{x^2+xy+y^2}+\dfrac1{x-y}\right)-\dfrac xy\\
				&=&\dfrac{x^3-y^3}{xy}\cdot\dfrac{(x+y)(x-y)+x^2+xy+y^2}{x^3-y^3}-\dfrac xy\\
				&=&\dfrac{2x^2+xy}{xy}-\dfrac xy=\dfrac{2x^2+xy-x^2}{xy}=\dfrac{x(x+y)}{xy}=\dfrac{x+y}{y}.
			\end{eqnarray*}
			Tại $x=-15$; $y=5$ ta có $C=-2$.
		\end{enumerate}
	}
\end{bt}



\begin{bt}%[Đề cương Toán THCS - Nguyễn Văn Cường (Cường NV)]%[8D2V3-4]
	Cho phân thức $P=\dfrac{2 x+1}{x+1}$.
	\begin{enumerate}
		\item Viết điều kiện xác định của $P$.
		\item Hãy viết $P$ dưới dạng $P=a-\dfrac{b}{x+1}$, trong đó $a$; $b$ là hai số nguyên dương.
		\item Với giá trị nguyên nào của $x$ thì $P$ có giá trị là số nguyên?
	\end{enumerate}	
	\loigiai{
		\begin{enumerate}
			\item ĐKXĐ: $x \ne -1$.
			\item Ta có $P=a-\dfrac{b}{x+1} = 2 - \dfrac{1}{x + 1}$.
			\item Để $P$ nguyên thì $\dfrac{1}{x + 1}$ nguyên.\\
			Do đó $1 \;\vdots\; (x + 1)$ hay $(x+1) \in \text{Ư}(1) = \{-1;1\}$.\\
			Do đó ta có $x \in \{-2; 0\}$.\\
			Vậy với $x \in \{-2; 0\}$ thì $P$ nguyên.
	\end{enumerate}}
\end{bt}


\begin{bt}%[Đề cương Toán THCS - Nguyễn Văn Cường (Cường NV)]%[8D2V1-7]
	Một xe ô tô đi từ Hà Nội đến Vinh với vận tốc trung bình là $60$ km/h và dự kiến sẽ đến Vinh sau $5$ giờ chạy xe. Tuy nhiên, sau $2 \dfrac{2}{3}$ giờ chạy với vận tốc $60$ km/h, xe dừng nghỉ $20$ phút. Sau khi dừng nghỉ, để đến Vinh đúng thời gian dự kiến, xe phải tăng vận tốc so với chặng đầu.
	\begin{enumerate}
		\item Tính độ dài quãng dường Hà Nội - Vinh.
		\item Tính độ dài quãng đường còn lại sau khi dừng nghỉ.
		\item Cho biết ở chặng thứ hai xe tăng vận tốc thêm $x$ (km/h). Hãy viết biểu thức $P$ biểu thị thời gian (tính bằng giờ) thực tế xe chạy hết chặng đường Hà Nội - Vinh.
		\item Tính giá trị của $P$ lần lượt tại $x = 5$; $x= 10$; $x =15$, từ đó cho biết ở chặng thứ hai sau khi xe dừng nghỉ:
		\begin{itemize}
			\item Nếu tăng vận tốc thêm $5$km/h thì xe đến Vinh muộn hơn dự kiến bao nhiêu giờ?
			\item Nếu tăng vận tốc thêm $10$km/h thì xe đến Vinh có đúng thời gian dự kiến không?
			\item Nếu tăng vận tốc thêm $15$km/h thì xe đến Vinh sớm hơn dự kiến bao nhiêu giờ?
		\end{itemize}
	\end{enumerate}
	\loigiai{
		\begin{enumerate}
			\item Độ dài quãng đường Hà Nội  - Vinh: $60\cdot 5 = 300$ km.
			\item Độ dài quãng đường còn lại sau khi dừng nghỉ: $300 - 60\cdot 2\dfrac{2}{3} = 140$ km
			\item $P = 2\dfrac{2}{3} + \dfrac{1}{3} + \dfrac{140}{60+x} = 3+\dfrac{140}{60+x}$. 
			\item Thay $x =5$ vào biểu thức $P$ ta được $P = 3 + \dfrac{140}{65} = \dfrac{67}{13}$. Như vậy xe sẽ đến Vinh muộn hơn $\dfrac{67}{13} - 5 = \dfrac{28}{13}$ giờ nếu tăng vận tốc thêm $5$ km/h.
			\\Thay $x =10$ vào biểu thức $P$ ta được $P = 3 + \dfrac{140}{70} = 5$. Như vậy xe sẽ đến đúng giờ dự kiến nếu tăng vận tốc thêm $10$km/h.
			\\Thay $x =15$ vào biểu thức $P$ ta được $P = 3 + \dfrac{140}{75} = \dfrac{73}{15}$. Như vậy xe sẽ đến Vinh muộn hơn $5-\dfrac{73}{15} = \dfrac{2}{15}$ giờ nếu tăng vận tốc thêm $15$ km/h.
		\end{enumerate}
		
	}
\end{bt}



\begin{bt}%[Đề cương Toán THCS - Nguyễn Văn Cường (Cường NV)]%[8D2V3-4]
	Cho biểu thức
	$$B= \left(\dfrac{5x+2}{x^2-10x}+ \dfrac{5x-2}{x^2+10x} \right)\cdot \dfrac{x^2-100}{x^2+4}.$$
	\begin{enumerate}
		\item Viết điều kiện xác định của biểu thức $B$.
		\item Rút gọn $B$ và tính giá trị của biểu thức $B$ tại $x=0{,}1$.
		\item Tìm số nguyên $x$ để biểu thức $B$ nhận giá trị nguyên.
	\end{enumerate}
	\loigiai{
		\begin{enumerate}
			\item Điều kiện xác định $\heva{&x^2-10x \ne 0\\ &x^2+10x \ne 0} \Rightarrow \heva{&x \ne 0\\ &x \ne 10\\ &x \ne -10} \Rightarrow \heva{&x \ne 0\\ &x \ne \pm 10.}$
			\item Ta có
			\begin{eqnarray*}
				B &=& \left(\dfrac{5x+2}{x^2-10x}+ \dfrac{5x-2}{x^2+10x} \right)\cdot \dfrac{x^2-100}{x^2+4}\\
				&=& \left(\dfrac{5x+2}{x(x-10)}+ \dfrac{5x-2}{x(x+10)} \right)\cdot \dfrac{x^2-100}{x^2+4}\\
				&=& \left(\dfrac{(5x+2)(x+10)}{x(x-10)(x+10)}+ \dfrac{(5x-2)(x-10)}{x(x-10)(x+10)} \right)\cdot \dfrac{x^2-100}{x^2+4}\\
				&=& \dfrac{(5x+2)(x+10)+(5x-2)(x-10)}{x(x-10)(x+10)}\cdot \dfrac{(x-10)(x+10)}{x^2+4}\\
				&=& \dfrac{10\left(x^2+4\right)}{x(x-10)(x+10)}\cdot \dfrac{(x-10)(x+10)}{x^2+4} = \dfrac{10}{x}.
			\end{eqnarray*}
			Thay $x=0{,}1$ vào biểu thức $B$ ta có $B= \dfrac{10}{0{,}1}=100$.
		\end{enumerate}
	}
\end{bt}

\begin{bt}%[Đề cương Toán THCS - Nguyễn Văn Cường (Cường NV)]%[8D2H2-5]
	Hai người thợ cùng sơn một bức tường. Nếu một mình sơn xong bức tường thì người thứ nhất làm xong lâu hơn người thứ hai là $2$ giờ. Gọi $x$ là số giờ mà người thứ nhất một mình sơn xong bức tường. Viết phân thức biểu thị tổng số phần của bức tường sơn được mà người thứ nhất sơn trong $3$ giờ và người thứ hai sơn trong $4$ giờ theo $x$.
	\loigiai{
		Gọi $x$ là số giờ mà người thứ nhất một mình sơn xong bức tường.\\
		Khi đó
		\begin{itemize}
			\item $1$ giờ người thứ nhất sơn được $\dfrac{1}{x}$ bức tường.
			\item $3$ giờ người thứ nhất sơn được $\dfrac{3}{x}$ bức tường.
		\end{itemize}
		Người thứ hai một mình sơn xong bức tường mất $x-2$ giờ.\\
		Do đó
		\begin{itemize}
			\item $1$ giờ người thứ hai sơn được $\dfrac{1}{x-2}$ bức tường.
			\item $4$ giờ người thứ hai sơn được $\dfrac{4}{x-2}$ bức tường.
		\end{itemize}
		Do đó, số phần của bức tường sơn được mà người thứ nhất sơn trong $3$ giờ và người thứ hai sơn trong $4$ giờ là 
		$$\dfrac{3}{x}+\dfrac{4}{x-2}= \dfrac{3(x-2)+4x}{x(x-2)}= \dfrac{7x-6}{x(x-2)}.$$
	}
\end{bt}

\begin{bt}%[Đề cương Toán THCS - Nguyễn Văn Cường (Cường NV)]%[8D2H1-7]
	Số tiền hằng năm $A$ (triệu đô la Mỹ) mà người Mỹ chi cho việc mua đồ ăn, đồ uống khi ra khỏi nhà và dân số $P$ (triệu người) hằng năm của Mỹ từ năm $2000$ đến năm $2006$ lần lượt được cho bằng công thức sau
	\begin{eqnarray*}
		&& A= \dfrac{-8~242{,}58t+348~299{,}6}{-0{,}06t+1} \text{ với } 0 \le t \le 6.\\
		&& P= 2{,}71t+282{,}7 \text{ với } 0 \le t \le 6.
	\end{eqnarray*}
	Trong đó, $t$ là số năm tính từ năm $2000$, $t=0$ tương ứng với năm $2000$.
	\begin{flushright}
		\textit{(Nguồn: U.S. Bureau of Economic Analysis and U.S. Census Bureau)}
	\end{flushright}
	Viết phân thức biểu thị (theo $t$) số tiền bình quân hằng năm mà mỗi người Mỹ đã chi cho việc mua đồ ăn, đồ uống khi ra khỏi nhà.
	\loigiai{
		Số tiền bình quân hằng năm mà mỗi người Mỹ đã chi cho việc mua đồ ăn, đồ uống khi ra khỏi nhà là
		$$\dfrac{A}{P}= \dfrac{-8~242{,}58t+348~299{,}6}{(-0{,}06t+1)\left(2{,}71t+282{,}7\right)} \text{ với } 0 \le t \le 6.$$
	}
\end{bt}

\begin{bt}%[Đề cương Toán THCS - Nguyễn Văn Cường (Cường NV)]%[8D2H1-4]
	Cho phân thức $P=\dfrac{x^2-4 x+3}{x^2-9}$.
	\begin{enumerate}
		\item Viết điều kiện xác định của phân thức. Tìm tập hợp tất cả các giá trị của $x$ không thoả mãn điểu kiện xác định.
		\item Rút gọn phân thức đã cho.
		\item Tìm tập hợp tất cả các giá trị nguyên của $x$ để phân thức $P$ nhận giá trị là số nguyên.		
	\end{enumerate} 	
	\loigiai{
		\begin{enumerate}
			\item Điều kiện xác định là $x^2-9 \neq 0$.\\
			$x$ không thoả mãn điều kiện $x^2-9 \neq 0$ nghĩa là $x^2-9=0$, hay $(x-3)(x+3)=0$, tức là $x-3=0$ hoặc $x+3=0$.\\
			Do đó tập hợp tất cả các giá trị của biến $x$ không điều kiện xác định là $\{3;-3\}$.\\
			\item Ta có $x^2-4 x+3=x^2-4 x+4-1=(x-2)^2-1=(x-2-1)(x-2+1)=(x-3)(x-1)$.\\
			Do đó $P=\dfrac{(x-3)(x-1)}{(x-3)(x+3)}=\dfrac{x-1}{x+3}$.\\
			\item $P=\dfrac{x-1}{x+3}=\dfrac{x-3+2}{x-3}=1+\dfrac{2}{x-3}$ nên $\dfrac{2}{x-3}=P-1$.\\
			Nếu $x, P \in \mathbb{Z}$ thì $x-3$ là ước số nguyên của $2$, do đó $x-3 \in\{1; 2;-1;-2\}$.
			\begin{center}
				\begin{tabular}{|c|c|c|c|c|}
					\hline
					$x-3$ & $1$ & $2$ & $-1$ & $-2$ \\
					\hline
					$x$ & $4$ & $5$ & $2$ & $1$ \\
					\hline
				\end{tabular}
			\end{center}
			Các giá trị tìm được của $x$ đều thoả mãn điều kiện xác định của phân thức. Do đó tập hợp cần tìm là $\{4; 5; 2; 1\}$.		
		\end{enumerate}	
	}
\end{bt}
%--------------------------------------
% Bài 2
\begin{bt}%[Đề cương Toán THCS - Nguyễn Văn Cường (Cường NV)]%[8D2H1-5]
	Cho hai phân thức $P=\dfrac{1}{2 x^2+7 x-15}$ và $Q=\dfrac{1}{x^2+3 x-10}$.
	Có thể quy đồng mẫu thức hai phân thức đã cho với mẫu thức chung là $M=2 x^3+3 x^2-29 x+30$ được không? Vì sao?	
	\loigiai{
		Đặt tính chia $M=2 x^3+3 x^2-29 x+30$ cho $2 x^2+7 x-15$ (mẫu thức của $P$).\\
		Ta thấy thương là $x-2$ và dư bằng $0$.\\
		Do đó $M=2 x^3+3 x^2-29 x+30=\left(2 x^2+7 x-15\right)(x-2)$.\\
		Tương tự, chia $M$ cho mẫu thức của $Q$ ta được thương là $2 x-3$ và dư bằng $0$.\\
		Do đó $M=2 x^3+3 x^2-29 x+30=\left(x^2+3 x-10\right)(2 x-3)$.\\
		Vì vậy $P=\dfrac{x-2}{M}, Q=\dfrac{2 x-3}{M}$.\\
		Do đó có thể quy đồng mẫu thức hai phân thức đã cho với mẫu thức chung là $M$.	
	}
\end{bt}
%--------------------------------------


\begin{bt}%[Đề cương Toán THCS - Nguyễn Văn Cường (Cường NV)]%[8D2H3-5]
	Cho phân thức $P=\dfrac{x^2-y^2}{(x+y)(a y-a x)}(a \neq 0, y \neq x, y \neq-x)$.
	Chứng minh rằng $P$ có giá trị không phụ thuộc vào $x, y$.
	\loigiai{
		Ta có $P=\dfrac{x^2-y^2}{(x+y)(a y-a x)}=\dfrac{(x+y)(x-y)}{(x+y) a(y-x)}=\dfrac{-1}{a}$ không phụ thuộc vào $x, y$.			
	}
\end{bt}


\begin{bt}%[Đề cương Toán THCS - Nguyễn Văn Cường (Cường NV)]%[8D2H3-5]
	Biết $x+y+z=0$ và $x$, $y \neq 0$. Chứng minh phân thức $\dfrac{x y}{x^2+y^2-z^2}$ có giá trị không đổi.	
	\loigiai{
		Từ giả thiết suy ra $z=-(x+y)$ nên $x^2+y^2-z^2=x^2+y^2-(x+y)^2=-2 x y$.\\
		Do đó $\dfrac{x y}{x^2+y^2-z^2}=\dfrac{x y}{-2 x y}=\dfrac{-1}{2}$.	
	}
\end{bt}
%--------------------------------------
% Bài 6
\begin{bt}%[Đề cương Toán THCS - Nguyễn Văn Cường (Cường NV)]%[8D2H2-4]
	Cho $x+y+z=0$ và $x, y, z \neq 0$. Rút gọn biểu thức sau:
	\[\dfrac{x y}{x^2+y^2-z^2}+\dfrac{y z}{y^2+z^2-x^2}+\dfrac{z x}{z^2+x^2-y^2}.\]
	\loigiai{
		Từ giả thiết suy ra $z=-(x+y)$ nên $x^2+y^2-z^2=x^2+y^2-(x+y)^2=-2 x y$.\\
		Do đó $\dfrac{x y}{x^2+y^2-z^2}=\dfrac{x y}{-2 x y}=\dfrac{-1}{2}$.\\
		Tương tự $\dfrac{y z}{y^2+z^2-x^2}=\dfrac{-1}{2}$ và $\dfrac{z x}{z^2+x^2-y^2}=\dfrac{-1}{2}$.\\
		Do đó $\dfrac{x y}{x^2+y^2-z^2}+\dfrac{y z}{y^2+z^2-x^2}+\dfrac{z x}{z^2+x^2-y^2}=\dfrac{-3}{2}$.		
	}
\end{bt}
%--------------------------------------
% Bài 7
\begin{bt}%[Đề cương Toán THCS - Nguyễn Văn Cường (Cường NV)]%[8D2H3-4]
	Cho phân thức $P=\dfrac{4 x^2+2 x+3}{2 x+1}\left(x \neq-\dfrac{1}{2}\right)$.
	\begin{enumerate}
		\item  Tìm thương và dư của phép chia đa thức $4 x^2+2 x+3$ cho đa thức $2 x+1$.
		\item  Sử dụng kết quả của câu a), hãy viết $P$ dưới dạng tổng của một đa thức và một phân thức với tử thức là một hằng số. Dùng kết quả đó để tìm tất cả các giá trị nguyên của $x$ để phân thức đã cho có giá trị cũng là số nguyên.			
	\end{enumerate}
	\loigiai{
		\begin{enumerate}
			\item Đặt tính chia đa thức $4 x^2+2 x+3$ cho đa thức $2 x+1$ ta được $2 x$ và dư là $3$.\\
			Vậy $4 x^2+2 x+3=(2 x+1) 2 x+3$.
			\item Vì $4 x^2+2 x+3=(2 x+1) 2 x+3$ nên
			$P=\dfrac{4 x^2+2 x+3}{2 x+1}=\dfrac{(2 x+1) 2 x+3}{2 x+1}=2 x+\dfrac{3}{2 x+1}$.\\
			Từ đó suy ra $\dfrac{3}{2 x+1}=P-2 x$.
			Nếu $x, P \in \mathbb{Z}$ thì $\dfrac{3}{2 x+1} \in \mathbb{Z}$.\\
			Suy ra $2 x+1$ là một ước số nguyên của $3$.\\
			Do đó $2 x+1 \in\{1; 3;-1;-3\}$ hay $x \in\{0; 1;-1;-2\}$ (các giá trị tìm được của $x$ đều thỏa mãn điều kiện $x \neq \dfrac{-1}{2}$).		
		\end{enumerate}		
	}
\end{bt}

\begin{bt}%[Đề cương Toán THCS - Nguyễn Văn Cường (Cường NV)]%[8D2H3-4]
	\begin{enumerate}
		\item  Rút gọn biểu thức $P=\dfrac{(x+2)^2}{x} \cdot\left(1-\dfrac{x^2}{x+2}\right)-\dfrac{x^2+6 x+4}{x}$.
		\item  Tìm giá trị lớn nhất của $P$.		
	\end{enumerate}
	\loigiai{
		\begin{enumerate}
			\item Ta có \allowdisplaybreaks
			\begin{eqnarray*}
				P&=&\dfrac{(x+2)^2}{x} \cdot\left(1-\dfrac{x^2}{x+2}\right)-\dfrac{x^2+6 x+4}{x}\\
				&=&\dfrac{(x+2)^2}{x}\cdot \dfrac{x+2-x^2}{x+2}-\dfrac{x^2+6x+4}{x}\\
				&=&\dfrac{(x+2)(x+2-x^2)}{x}-\dfrac{x^2+6x+4}{x}\\
				&=&\dfrac{(x+2)^2-x^3-2x^2}{x}-\dfrac{x^2+6x+4}{x}\\
				&=&\dfrac{x^2+4x+4-x^3-2x^2-x^2-6x-4}{x}\\
				&=&\dfrac{-x^3-2x^2-2x}{2}\\
				&=&-x^2-2x-2.
			\end{eqnarray*}	
			\item $P=-x^2-2 x-2=-1-(x+1)^2 \leq-1$.\\
			Giá trị lớn nhất của $P$ là $-1$ (đạt được tại $x=-1$).	
		\end{enumerate}		
	}
\end{bt}
%--------------------------------------
% Bài 9
\begin{bt}%[Đề cương Toán THCS - Nguyễn Văn Cường (Cường NV)]%[8D2?3-4]
	Cho phân thức $P=\dfrac{x^2-4 x+12}{x^2-4 x+10}$. Đặt $t=x-2$, hãy biểu diễn $P$ dưới dạng một phân thức của biến $t$. Từ đó suy ra $P$ luôn nhận giá trị dương.	
	\loigiai{
		Ta có $t=x-2$, suy ra $t^2=(x-2)^2=x^2-4 x+4$.\\
		Do đó $x^2-4 x=t^2-4$ và $P=\dfrac{t^2-4+12}{t^2-4+10}=\dfrac{t^2+8}{t^2+6}>0$.		
	}
\end{bt}


\begin{bt}%[Đề cương Toán THCS - Nguyễn Văn Cường (Cường NV)]%[8D2V3-6]
	Một bể chứa nước có hai vòi thoát. Biết rằng khi bể chứa đầy nước thì thời gian cần thiết để xả hết nước trong bể mà chỉ dùng vòi thứ nhất là $x$ (giờ) và thời gian cần thiết để xả hết nước trong bể mà chỉ dùng vòi thứ hai là $y$ (giờ).
	\begin{enumerate}
		\item Viết phân thức biểu thị thời gian cần thiết để xả hết nước trong bể (khi bể chứa đầy nước) nếu mở cả hai vòi.
		\item Tính thời gian cần thiết để xả hết nước trong bể (khi bể chứa đầy nước) nếu mở cả hai vòi, biết rằng khi chỉ mở một vòi, vòi thứ nhất xả hết nước trong $2$ giờ, vòi thứ hai xả hết nước trong $3$ giờ.			
	\end{enumerate}
	\loigiai{
		\begin{enumerate}
			\item Gọi $t$(giờ) là thời gian cần thiết để xả hết nước trong bể (đầy nước) khi mở cả hai vòi. Như vậy, trong một giờ cả hai vòi cùng mở sẽ xả được $\dfrac{1}{t}$ (bể).\\
			Mặt khác, từ giả thiết suy ra trong một giờ, một mình vòi thứ nhất xả hết $\dfrac{1}{x}$ (bể), một mình vòi thứ hai xả được $\dfrac{1}{y}$ (bể).\\
			Do đó, trong một giờ cả hai vòi cùng mở sẽ xả được $\dfrac{1}{x}+\dfrac{1}{y}=\dfrac{x+y}{x y}$ (bể).\\
			Từ đó suy ra $\dfrac{1}{t}=\dfrac{x+y}{x y}$.\\
			Do $t$ là nghịch đảo của $\dfrac{1}{t}=\dfrac{x+y}{x y}$ nên $t=\dfrac{x y}{x+y}$.
			\item Với $x=2, y=3$ thi $t=\dfrac{2 \cdot 3}{2+3}=1 \dfrac{1}{5}$ (giờ) $=1$ giờ $12$ phút.\\
			Do đó, trong trường hợp khi chỉ mở một vòi, vòi thứ nhất xả hết nước trong $4$ giờ, vòi thứ hai xả hết nước trong $3$ giờ; khi mở cả hai vòi sẽ xả được hết nước trong bể sau $1$ giờ $12$ phút.		
		\end{enumerate}		
	}
\end{bt}

\begin{bt}%[Đề cương Toán THCS - Nguyễn Văn Cường (Cường NV)]%[8D2H1-4]
	Cho biểu thức
	$$D=\left(\dfrac{x+2}{3x}+\dfrac2{x+1}-3\right)\colon\dfrac{2-4x}{x+1}-\dfrac{3x-x^2+1}{3x}.$$
	\begin{enumerate}
		\item Viết điều kiện xác định của biểu thức $D$.
		\item Tính giá trị của biểu thức $D$ tại $x=5~947$.
		\item Tìm giá trị của $x$ để $D$ nhận giá trị nguyên.
	\end{enumerate}
	\loigiai{
		\begin{enumerate}
			\item Điều kiện xác định của biểu thức $D$ là $x\ne0$; $x\ne-1$; $x\ne\dfrac12$.
			\item Ta có
			\begin{eqnarray*}
				D&=&\left(\dfrac{x+2}{3x}+\dfrac2{x+1}-3\right)\colon\dfrac{2-4x}{x+1}-\dfrac{3x-x^2+1}{3x}\\
				&=&\dfrac{(x+2)(x+1)+2\cdot3x-3\cdot3x(x+1)}{3x(x+1)}\cdot\dfrac{x+1}{2-4x}-\dfrac{3x-x^2+1}{3x}\\
				&=&\dfrac{-8x^2+2}{3x\cdot2(1-2x)}-\dfrac{3x-x^2+1}{3x}\\
				&=&\dfrac{2(1-2x)(1+2x)}{3x\cdot2(1-2x)}-\dfrac{3x-x^2+1}{3x}\\
				&=&\dfrac{1+2x-3x+x^2-1}{3x}=\dfrac{x-1}{3}.
			\end{eqnarray*}
			Tại $x=5~947$ ta có $D=1~982$.
			\item Để $D$ nhận giá trị nguyên thì $\dfrac{x-1}{3}$ phải nhận giá trị nguyên. Suy ra $x-1$ chia hết cho $3$, tức là $x-1=3k$ hay $x=3k+1$ với $k\in\mathbb{Z}$ (thỏa mãn điều kiện xác định).
		\end{enumerate}
	}
\end{bt}

%%=====Bài 3
\begin{bt}%[Đề cương Toán THCS - Nguyễn Văn Cường (Cường NV)]%[8D2H1-4]
	Cho biểu thức:
	$$S=\dfrac{(x+2)^2}{x}\cdot\left(1-\dfrac{x^2}{x+2}\right)-\dfrac{x^2+6x+4}{x}.$$
	\begin{enumerate}
		\item Rút gọn rồi tính giá trị của biểu thức $S$ tại $x=0{,}1$.
		\item Tìm giá trị lớn nhất của biểu thức $S$.
	\end{enumerate}
	\loigiai{
		\begin{enumerate}
			\item Điều keiẹn xác định của biểu thức $S$ là $x\ne0$; $x\ne-2$.\\
			Rút gọn biểu thức $S$ ta được $S=-x^2-2x-2$.\\
			Giá trị biểu thức $S$ tại $x=0{,}1$ là $-2{,}21$.
			\item Ta có $S=-x^2-2x-2=-(x^2-2x+1)-1=-(x-1)^2-1$.\\
			Suy ra $S$ đạt giá trị lớn nhất khi $-(x-1)^2-1$ đạt giá trị lớn nhất. Mà với mọi $x$ ta có $(x-1)^2\ge0$ hay $-(x-1)^2-1\le-1$.\\
			Vậy giá trị lớn nhất của $S$ là $-1$ khi $(x-1)^2=0$ hay $x=1$ (thỏa mãn điều kiện xác định).
		\end{enumerate}
	}
\end{bt}

%%=====Bài 4
\begin{bt}%[Dự án THCS - Nguyễn Văn Cường (Cường NV)]%[8D2H2-5]
	Hai ca nô cùng xuất phát đi xuôi dòng từ bến $A$ đến bến $B$ dài $24$ km. Ca nô thứ nhất đến $B$ trước và quay ngược lại thì gặp ca nô thứ hai tại vị trí $C$ cách bến $A$ là $8$ km. Biết tốc độ của dòng nước là $4$ km/h. Gọi $x$ (km/h) là tốc độ của ca nô thứ nhất $(x>4)$. Viết phân thức biểu thị theo $x$:
	\begin{enumerate}
		\item Thời gian ca nô thứ nhất đi từ bến $A$ đến bến $B$;
		\item Thời gian ca nô thứ nhất đi từ bến $B$ đến vị trí $C$;
		\item Tổng thời gian ca nô thứ nhất đi từ bến $A$ đến bến $B$ và từ bến $B$ đến vị trí $C$.
	\end{enumerate}
	\loigiai{
		\begin{enumerate}
			\item Thời gian ca nô thứ nhất đi từ bến $A$ đến bến $B$ là $\dfrac{24}{x+4}$ (giờ).
			\item Thời gian ca nô thứ nhất đi từ bến $B$ đến vị trí $C$ là $\dfrac{16}{x-4}$ (giờ).
			\item Tổng thời gian ca nô thứ nhất đi từ bến $A$ đến bến $B$ và từ bến $B$ đến vị trí $C$ là
			$$\dfrac{24}{x+4}+\dfrac{16}{x-4}=\dfrac{24(x-4)+16(x+4)}{(x+4)(x-4)}=\dfrac{40x-32}{x^2-16}\text{ (giờ).}$$
		\end{enumerate}
	}
\end{bt}

%%=====Bài 5
\begin{bt}%[Đề cương Toán THCS - Nguyễn Văn Cường (Cường NV)]%[8D2H2-5]
	Một tổ sản xuất theo kế hoạch phải may $600$ chiếc khẩu trang trong thời gian quy định. Do tăng năng suất lao động, mỗi giờ tổ sản xuất đó may được nhiều hơn kế hoạch $20$ chiếc. Gọi $x$ là số khẩu trang mà tổ sản xuất phải may trong mỗi giờ theo kế hoạch ($x\in\mathbb{N}^*$, $x<600$). Viết phân thức biểu thị theo $x$:
	\begin{enumerate}
		\item Thời gian tổ sản xuất phải hoàn thành công việc theo kế hoạch;
		\item Thời gian tổ sản xuất đã hoàn thành công việc theo thực tế;
		\item Tỉ số của thời gian tổ sản xuất đã hoàn thành công việc theo thực tế và thời gian tổ sản xuất phải hoàn thành công việc theo kế hoạch.
	\end{enumerate}
	\loigiai{
		\begin{enumerate}
			\item Thời gian tổ sản xuất phải hoàn thành công việc theo kế hoạch là $\dfrac{600}{x}$ (giờ).
			\item Thời gian tổ sản xuất đã hoàn thành công việc theo thực tế là $\dfrac{600}{x+20}$ (giờ).
			\item Tỉ số của thời gian tổ sản xuất đã hoàn thành công việc theo thực tế và thời gian tổ sản xuất phải hoàn thành công việc theo kế hoạch là
			$$\dfrac{600}{x+20}\colon\dfrac{600}{x}=\dfrac{x}{x+20}.$$
		\end{enumerate}
	}
\end{bt}
% In đáp án trắc nghiệm
\Closesolutionfile{ans}
\indapan{6}{ans/ans-8C1-OTC}