\section{PHÂN THỨC ĐẠI SỐ} % Tên bài
\subsection{Phân thức đại số}
\subsubsection{Kiến thức trọng tâm}
\begin{tomtat}
Một phân thức đại số (hay nói gọn là phân thức) là một biểu thức có dạng $\dfrac{A}{B}$, trong đó $A$, $B$ là những đa thức và $B$ khác đa thức không.\\
$A$ được gọi là tử thức (hay tử), $B$ được gọi là mẫu thức (hay mẫu).
\begin{luuy}
Mỗi đa thức được coi là một phân thức với mẫu thức bằng $1$.
\end{luuy}
\end{tomtat}
\begin{vd}%[Dự án EX-9-Đề Cương Toán 9 đợt 2]%[Trần Thanh Phong]%[8D2N1-2]
Chỉ ra các phân thức trong các biểu thức sau đây
\begin{center}
$\dfrac{3x-7}{x+5}$; $\dfrac{x-y}{xy}$; $x^2-4x+4$; $\sqrt{11}$; $5+\dfrac{1}{x}$.
\end{center}
\loigiai{
Trong các biểu thức trên, $\dfrac{3x-7}{x+5}$; $\dfrac{x-y}{xy}$; $x^2-4x+4$; $\sqrt{11}$ là phân thức.\\
Biểu thức $5+\dfrac{1}{x}$ không phải là phân thức.
}
\end{vd}

\begin{tomtat}
Điều kiện xác định của phân thức $\dfrac{A}{B}$ là điều kiện của biến để mẫu thức $B$ khác $0$.\\
Khi thay các biến của phân thức đại số bằng các giá trị nào đó (sao cho phân thức xác định), rồi thực hiện các phép tính thì ta nhận được giá trị của phân thức đại số đó tại các giá trị của biến.
\end{tomtat}

\begin{vd}%[Dự án EX-9-Đề Cương Toán 9 đợt 2]%[Trần Thanh Phong]%[8D2H1-6]
Cho phân thức $Q=\dfrac{x^2-9}{3x-6}$.
\begin{enumerate}
  \item Tính giá trị của phân thức tại $x=0$; $x=1$; $x=3$.
  \item Tại $x=2$ thì phân thức có xác định không? Tại sao?
\end{enumerate}
\loigiai{
\begin{enumerate}
  \item Tại $x=0$, $Q=\dfrac{0^2-9}{3\cdot 0-6}=\dfrac{-9}{-6}=\dfrac{3}{2}$.\\
  Tại $x=1$, $Q=\dfrac{1^2-9}{3\cdot 1-6}=\dfrac{-8}{-3}=\dfrac{8}{3}$.\\
  Tại $x=3$, $Q=\dfrac{3^2-9}{3\cdot 3-6}=\dfrac{9-9}{9-6}=\dfrac{0}{3}=0$.
  \item Với $x=2$, thì giá trị của mẫu thức là $3\cdot 2-6=6-6=0$ nên phân thức không xác định.
\end{enumerate}
}
\end{vd}

\begin{luuy}
Khi xét phân thức mà không nói gì thêm thì ta hiểu các biến chỉ nhận các giá trị làm cho phân thức xác định.  
\end{luuy}

\begin{vd}%[Dự án EX-9-Đề Cương Toán 9 đợt 2]%[Trần Thanh Phong]%[8D2N1-1]
Viết điều kiện xác định của mỗi phân thức sau
\begin{multicols}{2}
	\begin{enumerate}
		\item $\dfrac{2x-7}{5x+10}$;
		\item $\dfrac{x^2}{3x-y}$.
	\end{enumerate}
\end{multicols}
\loigiai{
\begin{enumerate}
  \item Phân thức xác định khi $5x+10\neq 0$ hay $x\neq -2$.
  \item Phân thức xác định khi $3x-y\neq 0$ (nghĩa là tại các giá trị của $x$ và $y$ thoả mãn $3x\neq y$).
\end{enumerate}
}
\end{vd}

\subsubsection{Bài tập}
\begin{bt}%[Dự án EX-9-Đề Cương Toán 9 đợt 2]%[Trần Thanh Phong]%[8D2N1-2]
Trong các biểu thức sau, biểu thức nào là phân thức đại số?
\begin{center}
$\dfrac{5x^2-2x+1}{x-7}$; $\dfrac{x-\sqrt{y}}{2x+y}$; $15$; $\dfrac{1}{x^2+y^2}$; $\dfrac{\sqrt{5}}{x}$.
\end{center}
\loigiai{
Các biểu thức là phân thức đại số là
\begin{center}
$\dfrac{5x^2-2x+1}{x-7}$; $15$; $\dfrac{1}{x^2+y^2}$; $\dfrac{\sqrt{5}}{x}$.
\end{center}
Biểu thức $\dfrac{x-\sqrt{y}}{2x+y}$ không phải là phân thức đại số vì tử thức $x-\sqrt{y}$ không phải là một đa thức.
}
\end{bt}

\begin{bt}%[Dự án EX-9-Đề Cương Toán 9 đợt 2]%[Trần Thanh Phong]%[8D2H1-6]
Tính giá trị của phân thức $P=\dfrac{x^2-4x}{2x-2}$ tại
\begin{multicols}{4}
	\begin{enumerate}
		\item $x=3$;
		\item $x=-1$;
		\item $x=0$;
		\item $x=1$.
	\end{enumerate}
\end{multicols}
\loigiai{
\begin{enumerate}
    \item Tại $x=3$, giá trị của $P$ là $P=\dfrac{3^2-4\cdot 3}{2\cdot 3-2}=\dfrac{9-12}{6-2}=\dfrac{-3}{4}$.
    \item Tại $x=-1$, giá trị của $P$ là $P=\dfrac{(-1)^2-4\cdot (-1)}{2\cdot (-1)-2}=\dfrac{1+4}{-2-2}=\dfrac{5}{-4}=-\dfrac{5}{4}$.
    \item Tại $x=0$, giá trị của $P$ là $P=\dfrac{0^2-4\cdot 0}{2\cdot 0-2}=\dfrac{0}{-2}=0$.
    \item Tại $x=1$, mẫu thức có giá trị $2\cdot 1-2=0$. Vậy tại $x=1$, phân thức $P$ không xác định.
\end{enumerate}
}
\end{bt}

\begin{bt}%[Dự án EX-9-Đề Cương Toán 9 đợt 2]%[Trần Thanh Phong]%[8D2H1-1]
Tìm điều kiện xác định của mỗi phân thức sau
\begin{multicols}{5}
	\begin{enumerate}
		\item $\dfrac{3x-5}{2x+8}$;
		\item $\dfrac{7}{x^2-1}$;
		\item $\dfrac{x-y}{x+y}$;
		\item $\dfrac{2x}{x^2-6x+9}$;
		\item $\dfrac{5}{x^2+4}$;
		\item $\dfrac{xy}{x^2y+xy^2}$;
		\item $\dfrac{x-1}{x^3-1}$;
		\item $\dfrac{3y}{5}$;
		\item $\dfrac{a^2-b^2}{a^2+b^2}$;
		\item $\dfrac{x+2}{x^2-3x+2}$.
	\end{enumerate}
\end{multicols}
\loigiai{
\begin{enumerate}
    \item $\dfrac{3x-5}{2x+8}$.\\
    Điều kiện xác định $2x+8\neq 0$ hay $x\neq -4$.
    \item $\dfrac{7}{x^2-1}$.\\
    Điều kiện xác định $x^2-1\neq 0$ hay $(x-1)(x+1)\neq 0$ hay $x\neq 1$ và $x\neq -1$.
    \item $\dfrac{x-y}{x+y}$.\\
    Điều kiện xác định $x+y\neq 0$ hay $x\neq -y$.
    \item $\dfrac{2x}{x^2-6x+9}$.\\
    Điều kiện xác định $x^2-6x+9\neq 0$ hay $(x-3)^2\neq 0$ hay $x\neq 3$.
    \item $\dfrac{5}{x^2+4}$.\\
    Vì $x^2\ge 0$ với mọi $x$ nên $x^2+4>0$ với mọi $x$. Phân thức xác định với mọi $x\in \mathbb{R}$.
    \item $\dfrac{xy}{x^2y+xy^2}$.\\
    Điều kiện xác định $x^2y+xy^2\neq 0$ hay $xy(x+y)\neq 0$ hay $x\neq 0$, $y\neq 0$ và $x\neq -y$.
    \item $\dfrac{x-1}{x^3-1}$.\\
    Điều kiện xác định $x^3-1\neq 0$ hay $(x-1)(x^2+x+1)\neq 0$ hay $x\neq 1$.
    \item $\dfrac{3y}{5}$.\\
    Mẫu thức là $5\neq 0$. Phân thức xác định với mọi $y$.
    \item $\dfrac{a^2-b^2}{a^2+b^2}$.\\
    Vì $a^2+b^2=0$ chỉ khi $a=b=0$ nên điều kiện xác định là $a$ và $b$ không đồng thời bằng $0$.
    \item $\dfrac{x+2}{x^2-3x+2}$.\\
    Điều kiện xác định $x^2-3x+2\neq 0$ hay $(x-1)(x-2)\neq 0$ hay $x\neq 1$ và $x\neq 2$.
\end{enumerate}
}
\end{bt}

\subsection{Hai phân thức bằng nhau}
\subsubsection{Kiến thức trọng tâm}
\begin{tomtat}
Ta nói hai phân thức $\dfrac{A}{B}$ và $\dfrac{C}{D}$ bằng nhau nếu $A\cdot D=B\cdot C$. Khi đó, ta viết
$$\dfrac{A}{B}=\dfrac{C}{D}.$$
\end{tomtat}

\begin{vd}%[Dự án EX-9-Đề Cương Toán 9 đợt 2]%[Trần Thanh Phong]%[8D2H1-3]
Hai phân thức $A=\dfrac{2x^2+10x}{x^2-25}$ và $B=\dfrac{2x}{x-5}$ có bằng nhau không? Tại sao?
\loigiai{
Ta có $(2x^2+10x)\cdot (x-5)=2x^3-10x^2+10x^2-50x=2x^3-50x$.\\
Và $(x^2-25)\cdot 2x=2x^3-50x$.\\
Vậy $(2x^2+10x)\cdot (x-5)=(x^2-25)\cdot 2x$.\\
Do đó $\dfrac{2x^2+10x}{x^2-25}=\dfrac{2x}{x-5}$, hay $A=B$.
}
\end{vd}

\subsubsection{Bài tập}

\begin{bt}%[Dự án EX-9-Đề Cương Toán 9 đợt 2]%[Trần Thanh Phong]%[8D2H1-3]
Các cặp phân thức sau có bằng nhau không? Tại sao?
\begin{multicols}{2}
	\begin{enumerate}
		\item $\dfrac{x-1}{x^2-1}$ và $\dfrac{1}{x+1}$;
		\item $\dfrac{3x(x-2)}{x-2}$ và $3x$;
		\item $\dfrac{a}{-b}$ và $\dfrac{-a}{b}$;
		\item $\dfrac{y}{y^2+1}$ và $\dfrac{1}{y+1}$.
	\end{enumerate}
\end{multicols}
\loigiai{
\begin{enumerate}
    \item $\dfrac{x-1}{x^2-1}$ và $\dfrac{1}{x+1}$.\\
    Ta có $(x-1)(x+1)=x^2-1$ và $(x^2-1)\cdot 1=x^2-1$.\\
    Vì tích chéo bằng nhau nên hai phân thức bằng nhau.
    \item $\dfrac{3x(x-2)}{x-2}$ và $3x$.\\
    Ta có $3x(x-2)\cdot 1=3x^2-6x$ và $(x-2)\cdot 3x=3x^2-6x$.\\
    Vì tích chéo bằng nhau nên hai phân thức bằng nhau.
    \item $\dfrac{a}{-b}$ và $\dfrac{-a}{b}$.\\
    Ta có $a\cdot b=ab$ và $(-b)\cdot (-a)=ab$.\\
    Vì tích chéo bằng nhau nên hai phân thức bằng nhau.
    \item $\dfrac{y}{y^2+1}$ và $\dfrac{1}{y+1}$.\\
    Ta có $y(y+1)=y^2+y$ và $(y^2+1)\cdot 1=y^2+1$.\\
    Vì $y^2+y\neq y^2+1$ (với $y\neq 1$) nên hai phân thức không bằng nhau.
\end{enumerate}
}
\end{bt}

\subsection{Tính chất cơ bản của phân thức}
\subsubsection{Kiến thức trọng tâm}
\begin{tomtat}
Tương tự như đối với phân số, ta có các tính chất cơ bản của phân thức sau đây
\begin{itemize}
  \item Khi nhân cả tử và mẫu của một phân thức với cùng một đa thức khác đa thức không thì được một phân thức bằng phân thức đã cho.
  $$\dfrac{A}{B}=\dfrac{A\cdot C}{B\cdot C} \quad \text{($C$ là một đa thức khác đa thức không)}.$$
  \item Khi chia cả tử và mẫu của một phân thức cho cùng một nhân tử chung của chúng thì được một phân thức bằng phân thức đã cho.
  $$\dfrac{A}{B}=\dfrac{A:D}{B:D} \quad \text{($D$ là một nhân tử chung)}.$$
\end{itemize}
\end{tomtat}

\begin{vd}%[Dự án EX-9-Đề Cương Toán 9 đợt 2]%[Trần Thanh Phong]%[8D2H1-4]
Biến đổi phân thức bên trái thành phân thức bên phải của mỗi đẳng thức sau
\begin{multicols}{3}
	\begin{enumerate}
		\item $\dfrac{a-3}{9-a^2}=\dfrac{-1}{a+3}$;
		\item $\dfrac{5y}{7-y}=\dfrac{-5y}{y-7}$;
		\item $\dfrac{15x^3y^2z}{-10xy^4}=\dfrac{-3x^2z}{2y^2}$.
	\end{enumerate}
\end{multicols}
\loigiai{
\begin{enumerate}
  \item $\dfrac{a-3}{9-a^2}=\dfrac{-(3-a)}{(3-a)(3+a)}=\dfrac{-1}{a+3}$;
  \item $\dfrac{5y}{7-y}=\dfrac{5y}{-(y-7)}=\dfrac{-5y}{y-7}$;
  \item $\dfrac{15x^3y^2z}{-10xy^4}=\dfrac{3\cdot 5\cdot x^2\cdot x\cdot y^2\cdot z}{-2\cdot 5\cdot x\cdot y^2\cdot y^2}=\dfrac{-3x^2z}{2y^2}$.
\end{enumerate}
}
\end{vd}

\begin{nx}
Ở Ví dụ $6$, các phân thức bên phải đều đơn giản hơn phân thức bên trái. Ta gọi các phép biến đổi ở trên là \textbf{rút gọn phân thức}.  
\end{nx}

\begin{luuy}
Để rút gọn một phân thức, ta thường thực hiện như sau
\begin{itemize}
  \item Phân tích tử và mẫu thành nhân tử (nếu cần) để tìm nhân tử chung.
  \item Chia cả tử và mẫu cho nhân tử chung.
\end{itemize}   
\end{luuy}

\subsubsection{Bài tập}

\begin{bt}%[Dự án EX-9-Đề Cương Toán 9 đợt 2]%[Trần Thanh Phong]%[8D2V1-4]
Rút gọn các phân thức sau
\begin{multicols}{4}
	\begin{enumerate}
		\item $\dfrac{12x^2y}{18xy^2}$;
		\item $\dfrac{5(x-y)}{10(y-x)}$;
		\item $\dfrac{x^2-4}{2x+4}$;
		\item $\dfrac{a^2+2a+1}{a^2-1}$;
		\item $\dfrac{x^3-x}{3x+3}$;
		\item $\dfrac{y^2-xy}{3x^2-3xy}$;
		\item $\dfrac{x^2+5x+6}{x^2+3x+2}$;
		\item $\dfrac{(2x-4)(x+3)}{x^2+x-6}$.
	\end{enumerate}
\end{multicols}
\loigiai{
\begin{enumerate}
    \item $\dfrac{12x^2y}{18xy^2}=\dfrac{2x\cdot 6xy}{3y\cdot 6xy}=\dfrac{2x}{3y}$;
    \item $\dfrac{5(x-y)}{10(y-x)}=\dfrac{5(x-y)}{-10(x-y)}=-\dfrac{5}{10}=-\dfrac{1}{2}$;
    \item $\dfrac{x^2-4}{2x+4}=\dfrac{(x-2)(x+2)}{2(x+2)}=\dfrac{x-2}{2}$;
    \item $\dfrac{a^2+2a+1}{a^2-1}=\dfrac{(a+1)^2}{(a-1)(a+1)}=\dfrac{a+1}{a-1}$;
    \item $\dfrac{x^3-x}{3x+3}=\dfrac{x(x^2-1)}{3(x+1)}=\dfrac{x(x-1)(x+1)}{3(x+1)}=\dfrac{x(x-1)}{3}$;
    \item $\dfrac{y^2-xy}{3x^2-3xy}=\dfrac{y(y-x)}{3x(x-y)}=\dfrac{-y(x-y)}{3x(x-y)}=-\dfrac{y}{3x}$;
    \item $\dfrac{x^2+5x+6}{x^2+3x+2}=\dfrac{(x+2)(x+3)}{(x+1)(x+2)}=\dfrac{x+3}{x+1}$;
    \item $\dfrac{(2x-4)(x+3)}{x^2+x-6}=\dfrac{2(x-2)(x+3)}{(x-2)(x+3)}=2$.
\end{enumerate}
}
\end{bt}

\subsection{Cộng, trừ hai phân thức cùng mẫu}
\subsubsection{Kiến thức trọng tâm}
Muốn cộng (hoặc trừ) hai phân thức có cùng mẫu thức, ta cộng (hoặc trừ) các tử thức với nhau và giữ nguyên mẫu thức.
\begin{center}
$\dfrac{A}{B}+\dfrac{C}{B}=\dfrac{A+C}{B}$; $\dfrac{A}{B}-\dfrac{C}{B}=\dfrac{A-C}{B}$.
\end{center}

\begin{luuy}
Phép cộng phân thức có các tính chất giao hoán, kết hợp tương tự như đối với phân số.  
\end{luuy}

\begin{vd}%[Dự án EX-9-Đề Cương Toán 9 đợt 2]%[Trần Thanh Phong]%[8D2H2-1]
Thực hiện các phép cộng, trừ phân thức sau
\begin{multicols}{3}
	\begin{enumerate}
		\item $\dfrac{a+3b}{ab^2}+\dfrac{a-3b}{ab^2}$;
		\item $\dfrac{x^2-5x}{x-3}-\dfrac{x-9}{x-3}$;
		\item $\dfrac{5a}{a^2-4}-\dfrac{2a+6}{a^2-4}$.
	\end{enumerate}
\end{multicols}
\loigiai{
\begin{enumerate}
    \item $\dfrac{a+3b}{ab^2}+\dfrac{a-3b}{ab^2}=\dfrac{a+3b+a-3b}{ab^2}=\dfrac{2a}{ab^2}=\dfrac{2}{b^2}$;
    \item $\dfrac{x^2-5x}{x-3}-\dfrac{x-9}{x-3}=\dfrac{x^2-5x-(x-9)}{x-3}=\dfrac{x^2-6x+9}{x-3}=\dfrac{(x-3)^2}{x-3}=x-3$;
    \item $\dfrac{5a}{a^2-4}-\dfrac{2a+6}{a^2-4}=\dfrac{5a-(2a+6)}{a^2-4}=\dfrac{3a-6}{a^2-4}=\dfrac{3(a-2)}{(a-2)(a+2)}=\dfrac{3}{a+2}$.
\end{enumerate}
}
\end{vd}

\subsubsection{Bài tập}

\begin{bt}%[Dự án EX-9-Đề Cương Toán 9 đợt 2]%[Trần Thanh Phong]%[8D2H2-3]
Thực hiện các phép tính sau
\begin{multicols}{3}
	\begin{enumerate}
		\item $\dfrac{x+1}{x+3}+\dfrac{2x-1}{x+3}$;
		\item $\dfrac{y-5}{2xy}-\dfrac{3y-5}{2xy}$;
		\item $\dfrac{a^2}{a-5}+\dfrac{25-10a}{a-5}$;
		\item $\dfrac{b^2}{b-c}+\dfrac{c^2}{c-b}$;
		\item $\dfrac{x-y}{z-x}-\dfrac{y-z}{x-z}$;
		\item $\dfrac{3m+1}{m^2-n^2}-\dfrac{3m-1}{m^2-n^2}$;
		\item $\dfrac{x^2-2}{x-2}-\dfrac{x}{x-2}$;
		\item $\dfrac{1}{x^2+x+1}+\dfrac{x^2+x}{x^2+x+1}$.
	\end{enumerate}
\end{multicols}
\loigiai{
\begin{enumerate}
    \item $\dfrac{x+1}{x+3}+\dfrac{2x-1}{x+3}=\dfrac{x+1+2x-1}{x+3}=\dfrac{3x}{x+3}$;
    \item $\dfrac{y-5}{2xy}-\dfrac{3y-5}{2xy}=\dfrac{y-5-(3y-5)}{2xy}=\dfrac{y-5-3y+5}{2xy}=\dfrac{-2y}{2xy}=-\dfrac{1}{x}$;
    \item $\dfrac{a^2}{a-5}+\dfrac{25-10a}{a-5}=\dfrac{a^2-10a+25}{a-5}=\dfrac{(a-5)^2}{a-5}=a-5$;
    \item $\dfrac{b^2}{b-c}+\dfrac{c^2}{c-b}=\dfrac{b^2}{b-c}-\dfrac{c^2}{b-c}=\dfrac{b^2-c^2}{b-c}=\dfrac{(b-c)(b+c)}{b-c}=b+c$;
    \item $\dfrac{x-y}{z-x}-\dfrac{y-z}{x-z}=-\dfrac{x-y}{x-z}-\dfrac{y-z}{x-z}=\dfrac{-(x-y)-(y-z)}{x-z}=\dfrac{-x+y-y+z}{x-z}=\dfrac{z-x}{x-z}=-1$;
    \item $\dfrac{3m+1}{m^2-n^2}-\dfrac{3m-1}{m^2-n^2}=\dfrac{3m+1-(3m-1)}{m^2-n^2}=\dfrac{3m+1-3m+1}{m^2-n^2}=\dfrac{2}{m^2-n^2}$;
    \item $\dfrac{x^2-2}{x-2}-\dfrac{x}{x-2}=\dfrac{x^2-x-2}{x-2}=\dfrac{(x-2)(x+1)}{x-2}=x+1$;
    \item $\dfrac{1}{x^2+x+1}+\dfrac{x^2+x}{x^2+x+1}=\dfrac{1+x^2+x}{x^2+x+1}=1$.
\end{enumerate}
}
\end{bt}