\section{KHÁI NIỆM HÀM SỐ} % Tên bài
\subsection{Khái niệm hàm số}
\subsubsection{Kiến thức trọng tâm}
\begin{tomtat}
	Nếu đại lượng $y$ phụ thuộc vào một đại lượng thay đổi $x$ sao cho với mỗi giá trị của $x$ ta luôn xác định được duy nhất một giá trị tương ứng của $y$ thì $y$ được gọi là \textit{hàm số} của \textit{biến số} $x$.
\end{tomtat}

\begin{vd}%[Dự án EX-8-Đề Cương Toán 8]%[GVSB: Trần Đoàn Hoàng Minh - GVPB1: Nguyễn Thế Duy - GVPB2: Nguyễn Trần Anh Tuấn]%[8D3B1-1]
	Hãy chỉ ra các đại lượng là hàm số và biến số ở các trường hợp sau
	\begin{enumerate}
		\item Nhiệt độ cơ thể $d$ ($^\circ C$) của bệnh nhân theo thời gian $h$ (giờ) trong ngày được ghi trong bảng sau:
		\begin{tabular}{|>{\centering\arraybackslash}m{2cm}|>{\centering\arraybackslash}m{1cm}|>{\centering\arraybackslash}m{1cm}|>{\centering\arraybackslash}m{1cm}|>{\centering\arraybackslash}m{1cm}|>{\centering\arraybackslash}m{1cm}|>{\centering\arraybackslash}m{1cm}|>{\centering\arraybackslash}m{1cm}|>{\centering\arraybackslash}m{1cm}|>{\centering\arraybackslash}m{1cm}|}
			\hline
			$h$ (giờ) & $7$ & $8$ & $9$ & $10$ & $11$ & $12$ & $13$ & $14$ & $15$\\
			\hline
			$d$ ($^\circ C$) & $36$ & $37$ & $36$ & $37$ & $38$ & $37$ & $38$ & $39$ & $39$\\
			\hline
		\end{tabular}
		\item Thời gian $t$ (giờ) để một vật chuyển động đều đi hết quãng đường $180$ km tỉ lệ nghịch với vận tốc $v$ (km/h) của nó theo công thức: $t=\dfrac{180}{v}$.
	\end{enumerate}
	\loigiai{
		\begin{enumerate}
			\item Đại lượng $d$ là hàm số của biến số $h$.
			\item Đại lượng $t$ là hàm số của biến số $v$.
		\end{enumerate}
	}
\end{vd}

\begin{vd}%[Dự án EX-8-Đề Cương Toán 8]%[GVSB: Trần Đoàn Hoàng Minh - GVPB1: Nguyễn Thế Duy - GVPB2: Nguyễn Trần Anh Tuấn]%[8D3B1-1]
	Các giá trị tương ứng của hai đại lượng $x$ và $y$ được cho bởi các bảng sau. Đại lượng $y$ có phải là một hàm số của đại lượng $x$ không?
	\begin{enumerate}
		\item \begin{tabular}{|>{\centering\arraybackslash}m{1.25cm}|>{\centering\arraybackslash}m{1.25cm}|>{\centering\arraybackslash}m{1.25cm}|>{\centering\arraybackslash}m{1.25cm}|>{\centering\arraybackslash}m{1.25cm}|>{\centering\arraybackslash}m{1.25cm}|>{\centering\arraybackslash}m{1.25cm}|}
			\hline
			$x$ & $-4$ & $-3$ & $1$ & $2$ & $3$ & $5$\\
			\hline
			$y$ & $-3$ & $-4$ & $3$ & $4$ & $5$ & $2$\\
			\hline
		\end{tabular}
		\item
		\begin{tabular}{|>{\centering\arraybackslash}m{1.25cm}|>{\centering\arraybackslash}m{1.25cm}|>{\centering\arraybackslash}m{1.25cm}|>{\centering\arraybackslash}m{1.25cm}|>{\centering\arraybackslash}m{1.25cm}|>{\centering\arraybackslash}m{1.25cm}|>{\centering\arraybackslash}m{1.25cm}|}
			\hline
			$x$ & $-3$ & $-2$ & $1$ & $2$ & $3$ & $-3$\\
			\hline
			$y$ & $-2$ & $1$ & $0$ & $4$ & $6$ & $2$\\
			\hline
		\end{tabular}
	\end{enumerate}
	\loigiai{
		\begin{enumerate}
			\item Đại lượng $y$ là hàm số của $x$ vì với mỗi giá trị của $x$ ($x\in\{ -4; -3; 1; 2; 3; 5 \}$), ta luôn xác định được chỉ một giá trị tương ứng của $y$.
			\item Đại lượng $y$ không phải là một hàm số của $x$ vì với $x=-3$, ta xác định được hai giá trị tương ứng của $y$ ($y=-2$ và $y=2$).
		\end{enumerate}
	}
\end{vd}

\begin{vd}%[Dự án EX-8-Đề Cương Toán 8]%[GVSB: Trần Đoàn Hoàng Minh - GVPB1: Nguyễn Thế Duy - GVPB2: Nguyễn Trần Anh Tuấn]%[8D3B1-1]
	Khi đo nhiệt độ, ta có công thức đổi từ đơn vị độ $C$ (Celsius) sang đơn vị độ $F$ (Fahrenheit) như sau: $F=1{,}8C+32$. Theo bạn, $F$ có phải là một hàm số theo biến số $C$ hay không? Giải thích.
	\loigiai{
		$F$ là một hàm số theo biến $C$ vì với mỗi giá trị của $C$ chỉ cho ta duy nhất một giá trị của $F$.
	}
\end{vd}

\subsubsection{Bài tập}

\begin{bt}%[Dự án EX-8-Đề Cương Toán 8]%[GVSB: Trần Đoàn Hoàng Minh - GVPB1: Nguyễn Thế Duy - GVPB2: Nguyễn Trần Anh Tuấn]%[8D3B1-1]
	Hãy chỉ ra các đại lượng là hàm số và biến số ở các trường hợp sau:
	\begin{enumerate}
		\item Nhiệt độ $T$ ($^0C$) ngoài trời trong một ngày mùa hè thay đổi theo thời gian $t$ (giờ) được ghi lại trong bảng sau
		\begin{center}
			\begin{tabular}{|>{\centering\arraybackslash}m{2cm}|>{\centering\arraybackslash}m{1cm}|>{\centering\arraybackslash}m{1cm}|>{\centering\arraybackslash}m{1cm}|>{\centering\arraybackslash}m{1cm}|>{\centering\arraybackslash}m{1cm}|>{\centering\arraybackslash}m{1cm}|}
				\hline
				$t$ (giờ) & $6$ & $9$ & $12$ & $15$ & $18$ & $21$\\
				\hline
				$T$ ($^0C$) & $26$ & $30$ & $34$ & $33$ & $31$ & $28$\\
				\hline
			\end{tabular}
		\end{center}
		\item Tốc độ $v$ (km/h) của một vận động viên chạy marathon được ghi lại sau mỗi km đầu tiên như sau
		\begin{center}
			\begin{tabular}{|>{\centering\arraybackslash}m{4cm}|>{\centering\arraybackslash}m{1cm}|>{\centering\arraybackslash}m{1cm}|>{\centering\arraybackslash}m{1cm}|>{\centering\arraybackslash}m{1cm}|>{\centering\arraybackslash}m{1cm}|>{\centering\arraybackslash}m{1cm}|}
				\hline
				Quãng đường $d$ (km) & $1$ & $2$ & $3$ & $4$ & $5$ & $6$\\
				\hline
				Vận tốc $v$ (km/h) & $14$ & $15$ & $14$ & $13$ & $12$ & $11$\\
				\hline
			\end{tabular}
		\end{center}
		\item Lượng xăng tiêu thụ $L$ (lít) khi xe máy đi được quãng đường $d$ (km) được tính bởi công thức: $L=0{,}04d$.
		\item Số tiền lãi $I$ (nghìn đồng) khi gửi ngân hàng $10$ triệu đồng trong thời gian $t$ (tháng) với lãi suất $0{,}6\%$/tháng được tính bởi công thức: $I=10\;000\cdot 0{,}006\cdot t$.
		\item Độ cao $h$ (m) của một vật rơi tự do sau $t$ (giây) theo công thức: $h=\dfrac{1}{2}gt^2$ với $g\thickapprox 9{,}8$ m/s$^2$.
	\end{enumerate}
	\loigiai{
		\begin{enumerate}
			\item Đại lượng $T$ là hàm số của biến số $t$.
			\item Đại lượng $v$ là hàm số của biến số $d$.
			\item Đại lượng $L$ là hàm số của biến số $d$.
			\item Đại lượng $I$ là hàm số của biến số $t$.
			\item Đại lượng $h$ là hàm số của biến số $t$.
		\end{enumerate}
	}
\end{bt}

\begin{bt}%[Dự án EX-8-Đề Cương Toán 8]%[GVSB: Trần Đoàn Hoàng Minh - GVPB1: Nguyễn Thế Duy - GVPB2: Nguyễn Trần Anh Tuấn]%[8D3B1-1]
	Các giá trị tương ứng của hai đại lượng $x$ và $y$ được cho bởi các bảng sau. Đại lượng $y$ có phải là một hàm số của đại lượng $x$ không?
	\begin{enumerate}
		\item \begin{tabular}{|>{\centering\arraybackslash}m{1.25cm}|>{\centering\arraybackslash}m{1.25cm}|>{\centering\arraybackslash}m{1.25cm}|>{\centering\arraybackslash}m{1.25cm}|>{\centering\arraybackslash}m{1.25cm}|>{\centering\arraybackslash}m{1.25cm}|>{\centering\arraybackslash}m{1.25cm}|>{\centering\arraybackslash}m{1.25cm}|}
			\hline
			$x$ & $-5$ & $-2$ & $0$ & $1$ & $2$ & $3$ & $4$\\
			\hline
			$y$ & $10$ & $4$ & $0$ & $1$ & $4$ & $9$ & $16$\\
			\hline
		\end{tabular}
		\item \begin{tabular}{|>{\centering\arraybackslash}m{1.25cm}|>{\centering\arraybackslash}m{1.25cm}|>{\centering\arraybackslash}m{1.25cm}|>{\centering\arraybackslash}m{1.25cm}|>{\centering\arraybackslash}m{1.25cm}|>{\centering\arraybackslash}m{1.25cm}|>{\centering\arraybackslash}m{1.25cm}|>{\centering\arraybackslash}m{1.25cm}|}
			\hline
			$x$ & $-3$ & $-1$ & $0$ & $1$ & $2$ & $3$ & $-1$\\
			\hline
			$y$ & $9$ & $1$ & $0$ & $1$ & $4$ & $9$ & $5$\\
			\hline
		\end{tabular}
		\item \begin{tabular}{|>{\centering\arraybackslash}m{1.25cm}|>{\centering\arraybackslash}m{1.25cm}|>{\centering\arraybackslash}m{1.25cm}|>{\centering\arraybackslash}m{1.25cm}|>{\centering\arraybackslash}m{1.25cm}|>{\centering\arraybackslash}m{1.25cm}|>{\centering\arraybackslash}m{1.25cm}|>{\centering\arraybackslash}m{1.25cm}|}
			\hline
			$x$ & $0$ & $1$ & $2$ & $3$ & $4$ & $2$ & $5$\\
			\hline
			$y$ & $0$ & $1$ & $4$ & $9$ & $16$ & $-4$ & $25$\\
			\hline
		\end{tabular}
	\end{enumerate}
	\loigiai{
		\begin{enumerate}
			\item Đại lượng $y$ là hàm số của $x$ vì với mỗi giá trị của $x$ ($x \in \{-5; -2; 0; 1; 2; 3; 4\}$), ta luôn xác định được chỉ một giá trị tương ứng của $y$.
			\item Đại lượng $y$ không phải là một hàm số của $x$ vì với $x=-1$, ta xác định được hai giá trị tương ứng của $y$ ($y=1$ và $y=5$).
			\item Đại lượng $y$ không phải là một hàm số của $x$ vì với $x=2$, ta xác định được hai giá trị tương ứng của $y$ ($y=4$ và $y=-4$).
		\end{enumerate}
	}
\end{bt}

\begin{bt}%[Dự án EX-8-Đề Cương Toán 8]%[GVSB: Trần Đoàn Hoàng Minh - GVPB1: Nguyễn Thế Duy - GVPB2: Nguyễn Trần Anh Tuấn]%[8D3B1-1]
	Các giá trị tương ứng của hai đại lượng $x$ và $y$ được cho bởi các bảng sau. Đại lượng $y$ có phải là một hàm số của đại lượng $x$ không?
	\begin{enumerate}
		\item \begin{tabular}{|>{\centering\arraybackslash}m{1.25cm}|>{\centering\arraybackslash}m{1.25cm}|>{\centering\arraybackslash}m{1.25cm}|>{\centering\arraybackslash}m{1.25cm}|>{\centering\arraybackslash}m{1.25cm}|>{\centering\arraybackslash}m{1.25cm}|>{\centering\arraybackslash}m{1.25cm}|>{\centering\arraybackslash}m{1.25cm}|>{\centering\arraybackslash}m{1.25cm}|}
			\hline
			$x$ & $-3$ & $-8$ & $-6$ & $-2$ & $3$ & $-7$ & $-6$ & $9$\\
			\hline
			$y$ & $1$ & $5$ & $10$ & $9$ & $7$ & $6$ & $3$ & $1$\\
			\hline
		\end{tabular}
		\item
		\begin{tabular}{|>{\centering\arraybackslash}m{1.25cm}|>{\centering\arraybackslash}m{1.25cm}|>{\centering\arraybackslash}m{1.25cm}|>{\centering\arraybackslash}m{1.25cm}|>{\centering\arraybackslash}m{1.25cm}|>{\centering\arraybackslash}m{1.25cm}|>{\centering\arraybackslash}m{1.25cm}|>{\centering\arraybackslash}m{1.25cm}|>{\centering\arraybackslash}m{1.25cm}|}
			\hline
			$x$ & $-4$ & $-3$ & $-2$ & $-1$ & $1$ & $2$ & $3$ & $4$ \\
			\hline
			$y$ & $15$ & $13$ & $14$ & $12$ & $1$ & $2$ & $3$ & $4$\\
			\hline
		\end{tabular}
		\item
		\begin{tabular}{|>{\centering\arraybackslash}m{1.25cm}|>{\centering\arraybackslash}m{1.25cm}|>{\centering\arraybackslash}m{1.25cm}|>{\centering\arraybackslash}m{1.25cm}|>{\centering\arraybackslash}m{1.25cm}|>{\centering\arraybackslash}m{1.25cm}|>{\centering\arraybackslash}m{1.25cm}|>{\centering\arraybackslash}m{1.25cm}|>{\centering\arraybackslash}m{1.25cm}|}
			\hline
			$x$ & $-10$ & $-8$ & $-4$ & $-2$ & $0$ & $-2$ & $4$ & $6$ \\
			\hline
			$y$ & $1$ & $3$ & $5$ & $7$ & $7$ & $5$ & $3$ & $1$\\
			\hline
		\end{tabular}
	\end{enumerate}
	\loigiai{
		\begin{enumerate}
			\item Đại lượng $y$ không phải là một hàm số của $x$ vì với $x=-6$, ta xác định được hai giá trị tương ứng của $y$ ($y=10$ và $y=3$).
			\item Đại lượng $y$ là hàm số của $x$ vì với mỗi giá trị của $x$ ($x \in \{-4; -3; -2; -1; 1; 2; 3; 4\}$), ta luôn xác định được chỉ một giá trị tương ứng của $y$.
			\item Đại lượng $y$ không phải là một hàm số của $x$ vì với $x=-2$, ta xác định được hai giá trị tương ứng của $y$ ($y=5$ và $y=7$).
		\end{enumerate}
	}
\end{bt}

\begin{bt}%[Dự án EX-8-Đề Cương Toán 8]%[GVSB: Trần Đoàn Hoàng Minh - GVPB1: Nguyễn Thế Duy - GVPB2: Nguyễn Trần Anh Tuấn]%[8D3B1-1]
	Các giá trị tương ứng của hai đại lượng $x$ và $y$ được cho bởi các bảng sau. Đại lượng $y$ có phải là một hàm số của đại lượng $x$ không?
	\begin{enumerate}
		\item \begin{tabular}{|>{\centering\arraybackslash}m{1.25cm}|>{\centering\arraybackslash}m{1.25cm}|>{\centering\arraybackslash}m{1.25cm}|>{\centering\arraybackslash}m{1.25cm}|>{\centering\arraybackslash}m{1.25cm}|>{\centering\arraybackslash}m{1.25cm}|>{\centering\arraybackslash}m{1.25cm}|}
			\hline
			$x$ & $-3$ & $-2$ & $-1$ & $0$ & $1$ & $2$\\
			\hline
			$y$ & $5$ & $5$ & $5$ & $5$ & $5$ & $5$\\
			\hline
		\end{tabular}
		\item
		\begin{tabular}{|>{\centering\arraybackslash}m{1.25cm}|>{\centering\arraybackslash}m{1.25cm}|>{\centering\arraybackslash}m{1.25cm}|>{\centering\arraybackslash}m{1.25cm}|>{\centering\arraybackslash}m{1.25cm}|>{\centering\arraybackslash}m{1.25cm}|>{\centering\arraybackslash}m{1.25cm}|}
			\hline
			$x$ & $0$ & $1$ & $2$ & $3$ & $4$ & $5$\\
			\hline
			$y$ & $1$ & $2$ & $3$ & $1$ & $2$ & $3$\\
			\hline
		\end{tabular}
	\end{enumerate}
	\loigiai{
		\begin{enumerate}
			\item Đại lượng $y$ là hàm số của $x$ vì với mỗi giá trị của $x$ ($x \in \{-3; -2; -1; 0; 1; 2\}$), ta luôn xác định được chỉ một giá trị tương ứng của $y$.
			\item Đại lượng $y$ là hàm số của $x$ vì với mỗi giá trị của $x$ ($x \in \{0; 1; 2; 3; 4; 5\}$), ta luôn xác định được chỉ một giá trị tương ứng của $y$.
		\end{enumerate}
	}
\end{bt}

\subsection{Giá trị của hàm số}

\subsubsection{Kiến thức trọng tâm}
\begin{luuy}
	{Cách cho một hàm số}\\
	Hàm số có thể được cho bằng bảng, biểu đồ hoặc bằng công thức, \dots.\\
	Nếu $y$ là hàm số của $x$ ta có thể viết $y=f(x)$, $y=g(x)$, \dots. Chẳng hạn, với hàm số được cho bởi công thức $y=4x+1$, ta còn có thể viết $y=f(x)=4x+1$.
\end{luuy}

\begin{tomtat}
	Cho hàm số $y=f(x)$, nếu ứng với $x=a$ ta có $y=f(a)$ thì $f(a)$ được gọi là \textit{giá trị của hàm số} $y=f(x)$ tại $x=a$.\\
	Bảng số liệu sau đây được gọi là một \textit{bảng giá trị của hàm số} $y=f(x)$.
	\begin{center}
		\begin{tabular}{|>{\centering\arraybackslash}m{2cm}|>{\centering\arraybackslash}m{1.25cm}|>{\centering\arraybackslash}m{1.25cm}|>{\centering\arraybackslash}m{1.25cm}|>{\centering\arraybackslash}m{1.25cm}|>{\centering\arraybackslash}m{1.25cm}|}
			\hline
			$x$ & $a$ & $b$ & $c$ & $\dots$ & $\dots$\\
			\hline
			$y=f(x)$ & $f(a)$ & $f(b)$ & $f(c)$ & $\dots$ & $\dots$\\
			\hline
		\end{tabular}
	\end{center}
\end{tomtat}

\begin{vd}%[Dự án EX-8-Đề Cương Toán 8]%[GVSB: Trần Đoàn Hoàng Minh - GVPB1: Nguyễn Thế Duy - GVPB2: Nguyễn Trần Anh Tuấn]%[8D3H1-2]
	Cho hàm số $y=f(x)=-4x+5$
	\begin{enumerate}
		\item Tính $f(2)$; $f(-4)$.
		\item Lập bảng giá trị của hàm số $x$ lần lượt bằng $-3$; $-2$; $0$; $1$; $3$.
	\end{enumerate}
	\loigiai{
		\begin{enumerate}
			\item Thay $x=2$ và $x=-4$ vào $f(x)$, ta có
			\begin{itemize}
				\item $f(2)=-4\cdot 1+5=-3$.
				\item $f(-4)=-4\cdot (-4)+5=21$.
			\end{itemize}
			\item Cho $x$ lần lượt bằng $-3$; $-2$; $0$; $1$; $3$, ta có bảng giá trị của hàm số
			\begin{center}
				\begin{tabular}{|>{\centering\arraybackslash}m{4cm}|>{\centering\arraybackslash}m{1.25cm}|>{\centering\arraybackslash}m{1.25cm}|>{\centering\arraybackslash}m{1.25cm}|>{\centering\arraybackslash}m{1.25cm}|>{\centering\arraybackslash}m{1.25cm}|}
					\hline
					$x$ & $-3$ & $-2$ & $0$ & $1$ & $3$\\
					\hline
					$y=f(x)=-4x+5$ & $17$ & $13$ & $5$ & $1$ & $-7$\\
					\hline
				\end{tabular}
			\end{center}
		\end{enumerate}
	}
\end{vd}

\begin{vd}%[Dự án EX-8-Đề Cương Toán 8]%[GVSB: Trần Đoàn Hoàng Minh - GVPB1: Nguyễn Thế Duy - GVPB2: Nguyễn Trần Anh Tuấn]%[8D3H1-2]
	Cho hàm số $y=f(x)=2x^2-3$
	\begin{enumerate}
		\item Tính $f(-1)$; $f(-3)$.
		\item Lập bảng giá trị của hàm số $x$ lần lượt bằng $-2$; $0$; $1$; $2$; $3$.
	\end{enumerate}
	\loigiai{
		\begin{enumerate}
			\item Thay $x=-1$ và $x=-3$ vào $f(x)$, ta có
			\begin{itemize}
				\item $f(-1)=2\cdot(-1)^2-3=-1$.
				\item $f(-3)=2\cdot(-3)^2-3=15$.
			\end{itemize}
			\item Cho $x$ lần lượt bằng $-2$; $0$; $1$; $2$; $3$, ta có bảng giá trị của hàm số
			\begin{center}
				\begin{tabular}{|>{\centering\arraybackslash}m{4cm}|>{\centering\arraybackslash}m{1.25cm}|>{\centering\arraybackslash}m{1.25cm}|>{\centering\arraybackslash}m{1.25cm}|>{\centering\arraybackslash}m{1.25cm}|>{\centering\arraybackslash}m{1.25cm}|}
					\hline
					$x$ & $-2$ & $0$ & $1$ & $2$ & $3$\\
					\hline
					$y=f(x)=2x^2-3$ & $5$ & $-3$ & $-1$ & $5$ & $15$\\
					\hline
				\end{tabular}
			\end{center}
		\end{enumerate}
	}
\end{vd}

\begin{vd}%[Dự án EX-8-Đề Cương Toán 8]%[GVSB: Trần Đoàn Hoàng Minh - GVPB1: Nguyễn Thế Duy - GVPB2: Nguyễn Trần Anh Tuấn]%[8D3H1-2]
	Gọi $C=f(d)$ là hàm số mô tả mối quan hệ giữa chu vi $C$ và đường kính $d$ của một đường tròn. Tìm công thức $f(d)$ và lập bảng giá trị của hàm số ứng với $d$ lần lượt bằng $1$; $2$; $3$; $4$ (theo đơn vị cm).
	\loigiai{
		Ta có $C=f(d)=\pi d$.\\
		Cho $d$ lần lượt bằng $1$; $2$; $3$; $4$, ta có bảng giá trị của hàm số
		\begin{center}
			\begin{tabular}{|>{\centering\arraybackslash}m{4cm}|>{\centering\arraybackslash}m{1.25cm}|>{\centering\arraybackslash}m{1.25cm}|>{\centering\arraybackslash}m{1.25cm}|>{\centering\arraybackslash}m{1.25cm}|}
				\hline
				$d$ & $1$ & $2$ & $3$ & $4$\\
				\hline
				$C=f(d)=\pi d$ & $\pi$ & $2\pi$ & $3\pi$ & $4\pi$\\
				\hline
			\end{tabular}
		\end{center}
	}
\end{vd}

\subsubsection{Bài tập}

\begin{bt}%[Dự án EX-8-Đề Cương Toán 8]%[GVSB: Trần Đoàn Hoàng Minh - GVPB1: Nguyễn Thế Duy - GVPB2: Nguyễn Trần Anh Tuấn]%[8D3H1-2]
	Cho hàm số $y=f(x)=4x$.
	\begin{enumerate}
		\item Tính $f(1)$; $f(-2)$; $f(5)$; $f\left( \dfrac{3}{4} \right)$; $f\left( \dfrac{-5}{8} \right)$.
		\item Lập bảng các giá trị tương ứng của $y$ khi $x$ lần lượt nhận các giá trị: $-3$; $-2$; $-1$; $0$; $1$; $2$; $3$
	\end{enumerate}
	\loigiai{
		\begin{enumerate}
			\item Thay $x=1$; $x=-2$; $x=5$; $x=\dfrac{3}{4}$ và $x=-\dfrac{5}{8}$ vào $f(x)$, ta có
			\begin{itemize}
				\item $f(1)=4\cdot 1=4$.
				\item $f(-2)=4\cdot (-2)=-8$.
				\item $f(5)=4\cdot 5=20$.
				\item $f\left(\dfrac{3}{4}\right)=4\cdot\dfrac{3}{4}=3$.
				\item $f\left(-\dfrac{5}{8}\right)=4\cdot \left(-\dfrac{5}{8}\right)=-\dfrac{20}{8}=-\dfrac{5}{2}$.
			\end{itemize}
			\item Cho $x$ lần lượt bằng $-3$; $-2$; $-1$; $0$; $1$; $2$; $3$, ta có bảng giá trị của hàm số
			\begin{center}
				\begin{tabular}{|>{\centering\arraybackslash}m{4cm}|*{7}{>{\centering\arraybackslash}m{1.25cm}|}}
					\hline
					$x$ & $-3$ & $-2$ & $-1$ & $0$ & $1$ & $2$ & $3$\\
					\hline
					$y=f(x)=4x$ & $-12$ & $-8$ & $-4$ & $0$ & $4$ & $8$ & $12$\\
					\hline
				\end{tabular}
			\end{center}
		\end{enumerate}
	}
\end{bt}

\begin{bt}%[Dự án EX-8-Đề Cương Toán 8]%[GVSB: Trần Đoàn Hoàng Minh - GVPB1: Nguyễn Thế Duy - GVPB2: Nguyễn Trần Anh Tuấn]%[8D3H1-2]
	Cho hàm số $y=f(x)=3x-7$.
	\begin{enumerate}
		\item Tính $f(1)$; $f(-3)$.
		\item Lập bảng giá trị của hàm số $x$ lần lượt bằng $-2$; $-1$; $0$; $2$; $4$.
	\end{enumerate}
	\loigiai{
		\begin{enumerate}
			\item Thay $x=1$ và $x=-3$ vào $f(x)$, ta có
			\begin{itemize}
				\item $f(1)=3\cdot 1-7=-4$.
				\item $f(-3)=3\cdot (-3)-7=-9-7=-16$.
			\end{itemize}
			\item Cho $x$ lần lượt bằng $-2$; $-1$; $0$; $2$; $4$, ta có bảng giá trị của hàm số
			\begin{center}
				\begin{tabular}{|>{\centering\arraybackslash}m{4cm}|*{5}{>{\centering\arraybackslash}m{1.25cm}|}}
					\hline
					$x$ & $-2$ & $-1$ & $0$ & $2$ & $4$\\
					\hline
					$y=f(x)=3x-7$ & $-13$ & $-10$ & $-7$ & $-1$ & $5$\\
					\hline
				\end{tabular}
			\end{center}
		\end{enumerate}
	}
\end{bt}

\begin{bt}%[Dự án EX-8-Đề Cương Toán 8]%[GVSB: Trần Đoàn Hoàng Minh - GVPB1: Nguyễn Thế Duy - GVPB2: Nguyễn Trần Anh Tuấn]%[8D3H1-2]
	Cho hàm số $y=f(x)=x^2-3$. Tính $f(-3)$; $f(-2)$; $f(-1)$; $f(0)$; $f(2)$; $f(4)$ và lập bảng giá trị của hàm số tương ứng.
	\loigiai{
		Thay $x=-3$; $x=-2$; $x=-1$; $x=0$; $x=2$; $x=4$ vào $f(x)$, ta có
		\begin{itemize}
			\item $f(-3)=(-3)^2-3=9-3=6$.
			\item $f(-2)=(-2)^2-3=4-3=1$.
			\item $f(-1)=(-1)^2-3=1-3=-2$.
			\item $f(0)=0^2-3=-3$.
			\item $f(2)=2^2-3=4-3=1$.
			\item $f(4)=4^2-3=16-3=13$.
		\end{itemize}
		Vậy bảng giá trị của hàm số là
		\begin{center}
			\begin{tabular}{|>{\centering\arraybackslash}m{4cm}|*{6}{>{\centering\arraybackslash}m{1.25cm}|}}
				\hline
				$x$ & $-3$ & $-2$ & $-1$ & $0$ & $2$ & $4$\\
				\hline
				$y=f(x)=x^2-3$ & $6$ & $1$ & $-2$ & $-3$ & $1$ & $13$\\
				\hline
			\end{tabular}
		\end{center}
	}
\end{bt}

\begin{bt}%[Dự án EX-8-Đề Cương Toán 8]%[GVSB: Trần Đoàn Hoàng Minh - GVPB1: Nguyễn Thế Duy - GVPB2: Nguyễn Trần Anh Tuấn]%[8D3H1-3]
	Khối lượng $m$ (g) của một thanh sắt có khối lượng riêng là $7{,}8$ kg/dm$^3$ tỉ lệ thuận với thể tích $V$ (cm$^3$) theo công thức $m=7{,}8\cdot V$. Đại lượng $m$ có phải là hàm số của đại lượng $V$ không? Nếu có, tính $m(10)$; $m(20)$; $m(30)$; $m(35)$; $m(40)$; $m(50)$; $m(60)$.
	\loigiai{
		Với mỗi giá trị của $V$ ta luôn xác định được một giá trị duy nhất của $m$ theo công thức $m=7{,}8\cdot V$, nên $m$ là hàm số của $V$.\\
		Thay $V=10$; $V=20$; $V=30$; $V=35$; $V=40$; $V=50$ và $V=60$ vào $m=7{,}8V$, ta có
		\begin{itemize}
			\item $m(10)=7{,}8\cdot 10=78$ (g).
			\item $m(20)=7{,}8\cdot 20=156$ (g).
			\item $m(30)=7{,}8\cdot 30=234$ (g).
			\item $m(35)=7{,}8\cdot 35=273$ (g).
			\item $m(40)=7{,}8\cdot 40=312$ (g).
			\item $m(50)=7{,}8\cdot 50=390$ (g).
			\item $m(60)=7{,}8\cdot 60=468$ (g).
		\end{itemize}
	}
\end{bt}

\begin{bt}%[Dự án EX-8-Đề Cương Toán 8]%[GVSB: Trần Đoàn Hoàng Minh - GVPB1: Nguyễn Thế Duy - GVPB2: Nguyễn Trần Anh Tuấn]%[8D3H1-3]
	Thể tích $V$ (lít) của một bình chứa nước có khối lượng riêng là $1$ (kg/lít) tỉ lệ thuận với khối lượng $m$ (kg) theo công thức $V=m$.
	\begin{enumerate}
		\item Đại lượng $V$ có phải là hàm số của $m$ không?
		\item Nếu có, hãy tính $V(2)$; $V(5)$; $V(8)$; $V(10)$; $V(12)$.
	\end{enumerate}
	\loigiai{
		\begin{enumerate}
			\item Với mỗi giá trị của $m$ ta luôn xác định được một giá trị duy nhất của $V$ theo công thức $V=m$, nên $V$ là hàm số của $m$.
			\item Thay $m=2$; $m=5$; $m=8$; $m=10$ và $m=12$ vào $V=m$, ta có
			\begin{itemize}
				\item $V(2)=2$ (lít).
				\item $V(5)=5$ (lít).
				\item $V(8)=8$ (lít).
				\item $V(10)=10$ (lít).
				\item $V(12)=12$ (lít).
			\end{itemize}
		\end{enumerate}
	}
\end{bt}

\begin{bt}%[Dự án EX-8-Đề Cương Toán 8]%[GVSB: Trần Đoàn Hoàng Minh - GVPB1: Nguyễn Thế Duy - GVPB2: Nguyễn Trần Anh Tuấn]%[8D3H1-3]
	Khối lượng $m$ (g) của một dây đồng có khối lượng riêng là $8{,}9$ (g/cm$^3$) tỉ lệ thuận với thể tích $V$ (cm$^3$) theo công thức $m=8{,}9V$.
	\begin{enumerate}
		\item Đại lượng $m$ có phải là hàm số của $V$ không?
		\item Nếu có, hãy tính $m(5)$; $m(15)$; $m(25)$; $m(30)$; $m(50)$.
	\end{enumerate}
	\loigiai{
		\begin{enumerate}
			\item Với mỗi giá trị của $V$ ta luôn xác định được một giá trị duy nhất của $m$ theo công thức $m=8{,}9V$, nên $m$ là hàm số của $V$.
			\item Thay $V=5$; $V=15$; $V=25$; $V=30$; $V=50$ vào $m=8{,}9V$, ta có
			\begin{itemize}
				\item $m(5)=8{,}9\cdot 5=44{,}5$ (g).
				\item $m(15)=8{,}9\cdot 15=133{,}5$ (g).
				\item $m(25)=8{,}9\cdot 25=222{,}5$ (g).
				\item $m(30)=8{,}9\cdot 30=267$ (g).
				\item $m(50)=8{,}9\cdot 50=445$ (g).
			\end{itemize}
		\end{enumerate}
	}
\end{bt}

\begin{bt}%[Dự án EX-8-Đề Cương Toán 8]%[GVSB: Trần Đoàn Hoàng Minh - GVPB1: Nguyễn Thế Duy - GVPB2: Nguyễn Trần Anh Tuấn]%[8D3H1-3]
	Thời gian $t$ (giờ) của một vật chuyển động đều trên quãng đường $20$ km tỉ lệ nghịch với tốc độ $v$ (km/h) của nó theo công thức $t=\dfrac{20}{v}$. Tính và lập bảng các giá trị tương ứng của $t$ khi $v$ lần lượt nhận các giá trị $5$; $10$; $20$; $25$; $40$; $80$.
	\loigiai{
		Thay $v=5$; $v=10$; $v=20$; $v=25$;$v=40$ và $v=80$ vào $t=\dfrac{20}{v}$, ta có
		\begin{itemize}
			\item $t(5)=\dfrac{20}{5}=4$ (giờ).
			\item $t(10)=\dfrac{20}{10}=2$ (giờ).
			\item $t(20)=\dfrac{20}{20}=1$ (giờ).
			\item $t(25)=\dfrac{20}{25}=\dfrac{4}{5}$ (giờ).
			\item $t(40)=\dfrac{20}{40}=\dfrac{1}{2}$ (giờ).
			\item $t(80)=\dfrac{20}{80}=\dfrac{1}{4}$ (giờ).
		\end{itemize}
		Vậy bảng giá trị tương ứng là
		\begin{center}
			\begin{tabular}{|>{\centering\arraybackslash}m{4cm}|*{6}{>{\centering\arraybackslash}m{1.25cm}|}}
				\hline
				$v$ (km/h) & $5$ & $10$ & $20$ & $25$ & $40$ & $80$\\
				\hline
				$t=\dfrac{20}{v}$ (giờ) & $4$ & $2$ & $1$ & $\dfrac{4}{5}$ & $\dfrac{1}{2}$ & $\dfrac{1}{4}$\\
				\hline
			\end{tabular}
		\end{center}
	} 
\end{bt}

\begin{bt}%[Dự án EX-8-Đề Cương Toán 8]%[GVSB: Trần Đoàn Hoàng Minh - GVPB1: Nguyễn Thế Duy - GVPB2: Nguyễn Trần Anh Tuấn]%[8D3H1-3]
	Thời gian $t$ (giờ) để một nhóm công nhân hoàn thành công việc tỉ lệ nghịch với số người $n$. Nếu $5$ người làm trong $12$ giờ thì công thức là $t=\dfrac{60}{n}$.  Tính và lập bảng giá trị của $t$ khi $n$ lần lượt nhận các giá trị $5$; $6$; $10$; $12$; $15$.
	\loigiai{
		Thay $n=5$; $n=6$; $n=10$; $n=12$ và $n=15$ vào $t=\dfrac{60}{n}$, ta có
		\begin{itemize}
			\item $t(5)=\dfrac{60}{5}=12$ (giờ).
			\item $t(6)=\dfrac{60}{6}=10$ (giờ).
			\item $t(10)=\dfrac{60}{10}=6$ (giờ).
			\item $t(12)=\dfrac{60}{12}=5$ (giờ).
			\item $t(15)=\dfrac{60}{15}=4$ (giờ).
		\end{itemize}
		Vậy bảng giá trị tương ứng là
		\begin{center}
			\begin{tabular}{|>{\centering\arraybackslash}m{4cm}|*{5}{>{\centering\arraybackslash}m{1.25cm}|}}
				\hline
				$n$ & $5$ & $6$ & $10$ & $12$ & $15$\\
				\hline
				$t=\dfrac{60}{n}$ (giờ) & $12$ & $10$ & $6$ & $5$ & $4$\\
				\hline
			\end{tabular}
		\end{center}
	}
\end{bt}

\begin{bt}%[Dự án EX-8-Đề Cương Toán 8]%[GVSB: Trần Đoàn Hoàng Minh - GVPB1: Nguyễn Thế Duy - GVPB2: Nguyễn Trần Anh Tuấn]%[8D3H1-3]
	Một xưởng cần sản xuất $240$ sản phẩm. Thời gian $t$ (giờ) để hoàn thành công việc tỉ lệ nghịch với năng suất $v$ (sản phẩm/giờ) của máy. Khi đó công thức được tính bằng $t=\dfrac{240}{v}$. Tính và lập bảng giá trị của $t$ khi $v$ lần lượt nhận các giá trị $20$; $30$; $40$; $60$; $120$.
	\loigiai{
		Thay $v=20$; $v=30$; $v=40$; $v=60$ và $v=120$ vào $t=\dfrac{240}{v}$, ta có
		\begin{itemize}
			\item $t(20)=\dfrac{240}{20}=12$ (giờ).
			\item $t(30)=\dfrac{240}{30}=8$ (giờ).
			\item $t(40)=\dfrac{240}{40}=6$ (giờ).
			\item $t(60)=\dfrac{240}{60}=4$ (giờ).
			\item $t(120)=\dfrac{240}{120}=2$ (giờ).
		\end{itemize}
		Vậy bảng giá trị tương ứng là
		\begin{center}
			\begin{tabular}{|>{\centering\arraybackslash}m{4cm}|*{5}{>{\centering\arraybackslash}m{1.25cm}|}}
				\hline
				$v$ (sản phẩm/giờ) & $20$ & $30$ & $40$ & $60$ & $120$\\
				\hline
				$t=\dfrac{240}{v}$ (giờ) & $12$ & $8$ & $6$ & $4$ & $2$\\
				\hline
			\end{tabular}
		\end{center}
	}
\end{bt}

\begin{bt}%[Dự án EX-8-Đề Cương Toán 8]%[GVSB: Trần Đoàn Hoàng Minh - GVPB1: Nguyễn Thế Duy - GVPB2: Nguyễn Trần Anh Tuấn]%[8D3H1-3]
	Một bể nước có dung tích $500$ (m$^3$). Thời gian $t$ (giờ) để bơm đầy bể tỉ lệ nghịch với lưu lượng $q$ (m$^3$/giờ) của máy bơm với công thức $t=\dfrac{500}{q}$. Tính và lập bảng giá trị của $t$ khi $q$ lần lượt nhận các giá trị $50$; $100$; $125$; $250$; $500$.
	\loigiai{
		Thay $q=50$; $q=100$; $q=125$; $q=250$ và $q=500$ vào $t=\dfrac{500}{q}$, ta có
		\begin{itemize}
			\item $t(50)=\dfrac{500}{50}=10$ (giờ).
			\item $t(100)=\dfrac{500}{100}=5$ (giờ).
			\item $t(125)=\dfrac{500}{125}=4$ (giờ).
			\item $t(250)=\dfrac{500}{250}=2$ (giờ).
			\item $t(500)=\dfrac{500}{500}=1$ (giờ).
		\end{itemize}
		Vậy bảng giá trị tương ứng là
		\begin{center}
			\begin{tabular}{|>{\centering\arraybackslash}m{4cm}|*{5}{>{\centering\arraybackslash}m{1.25cm}|}}
				\hline
				$q$ (m$^3$/giờ) & $50$ & $100$ & $125$ & $250$ & $500$\\
				\hline
				$t=\dfrac{500}{q}$ (giờ) & $10$ & $5$ & $4$ & $2$ & $1$\\
				\hline
			\end{tabular}
		\end{center}
	}
\end{bt}