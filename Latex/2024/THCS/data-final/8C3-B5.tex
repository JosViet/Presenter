\section{HÌNH CHỮ NHẬT - HÌNH VUÔNG} % Tên bài
\subsection{Hình chữ nhật}
\subsubsection{Kiến thức trọng tâm}
\begin{tomtat}
	\begin{dn}
	\immini{Hình chữ nhật là tứ giác có bốn góc vuông.}
	{	\begin{tikzpicture}[line join = round, line cap = round,>=stealth,font=\footnotesize,scale=1]
			\path (0,0) coordinate (D) (4,0) coordinate (C) (0,2) coordinate(A) (4,2) coordinate(B);
			\draw (A)--(B)--(C)--(D)--cycle;
			\foreach\i/\j in{D/-90,A/90,B/90,C/-90}\fill[black](\i)circle(1.5pt)node[shift=(\j:2.5mm)]{$\i$};
			\pic[draw, angle radius=8pt]{right angle=D--A--B};
			\pic[draw, angle radius=8pt]{right angle=A--B--C};
			\pic[draw, angle radius=8pt]{right angle=B--C--D};
			\pic[draw, angle radius=8pt]{right angle=A--D--C};
	\end{tikzpicture}}
	\end{dn}
	\begin{tc}
			\immini
		{
			Hình chữ nhật có đầy đủ tính chất của hình bình hành và hình thang cân, ngoài ra còn có tính chất riêng là hai đường chéo bằng nhau.
		}
		{
			\begin{tikzpicture}[line join = round, line cap = round,>=stealth,font=\footnotesize,scale=1]
				\path (0,0) coordinate (D) (4,0) coordinate (C) (0,2) coordinate(A) (4,2) coordinate(B) (intersection of A--C and B--D)coordinate (O);
				\draw (A)--(B)--(C)--(D)--cycle (A)--(C) (B)--(D);
				\foreach\i/\j in{D/-90,A/90,B/90,C/-90,O/-90}\fill[black](\i)circle(1.5pt)node[shift=(\j:3.2mm)]{$\i$};
				\pic[draw, angle radius=8pt]{right angle=D--A--B};
				\pic[draw, angle radius=8pt]{right angle=A--B--C};
				\pic[draw, angle radius=8pt]{right angle=B--C--D};
				\pic[draw, angle radius=8pt]{right angle=A--D--C};
				\path (A)--(O)node[sloped, midway]{\tiny |};
				\path (C)--(O)node[sloped, midway]{\tiny |};
				\path (B)--(O)node[sloped, midway]{\tiny |};
				\path (D)--(O)node[sloped, midway]{\tiny |};
		\end{tikzpicture}}
	\end{tc}
	\begin{dl}
		\begin{itemize}
			\item Tứ giác có bốn góc vuông là hình chữ nhật.
			\item  Hình bình hành có một góc vuông là hình chữ nhật.
			\item   Hình bình hành có hai đường chéo bằng nhau là hình chữ nhật.
		\end{itemize}
	\end{dl}
	
	
	\begin{nx}
		\begin{itemize}
			\item Trong tam giác vuông, đường trung tuyến ứng với cạnh huyền bằng một nửa cạnh huyền.
			\item Nếu tam giác có một đường trung tuyến bằng nửa cạnh tương ứng thì tam giác đó là tam giác vuông.
		\end{itemize}
	\end{nx}
\end{tomtat}
\begin{vd}%[Dự án EX-8-Đề Cương Toán 8]%[Dat Tien Pham]%[8H2V5-1]
	Tìm $x$ ở hình vẽ bên dưới
	\begin{center}
		\begin{tikzpicture}[>=stealth,line join=round,line cap=round,line width=0.6pt,font=\footnotesize,scale=0.6]
			\coordinate[label=below left:$B$](B) at (0,0);
			\coordinate[label=below right:$C$](C) at (10,0);
			\coordinate[label=above left:$A$](A) at (2,4);
			\coordinate[label=below:$M$] (M) at ($(B)!0.5!(C)$);
			\coordinate[label=left:$3$ cm] (E) at ($(A)!0.5!(B)$);
			\coordinate[label=above right:$4$ cm] (F) at ($(A)!0.5!(C)$);
			%	\coordinate[label=below:$\text{Hình 14}$] (H) at (5,-0.5);
			\coordinate[label=left:$x$] (G) at ($(A)!0.5!(M)$);
			\draw (A)--(B)--(C)--cycle;
			\draw(A)--(M);
			\path (B)--(M)node[midway]{$|$} (C)--(M)node[midway]{$|$};
			\draw pic[draw,angle radius=3mm] {right angle = B--A--C};
		\end{tikzpicture}
	\end{center}
	\loigiai{Áp dụng định lý Pythagore cho tam giác $ABC$ vuông tại $A$, ta có
		\begin{eqnarray*}
			BC^2&=&AB^2+AC^2\\
			BC&=&\sqrt{AB^2+AC^2}\\
			BC&=&\sqrt{3^2+4^2}\\
			BC&=&5 \text{ cm}.
		\end{eqnarray*}
		Trong tam giác $ABC$ vuông tại $A$, $AM$ là đường trung tuyến, do đó
		\[AM=\dfrac{BC}{2}=\dfrac{5}{2} \text{ cm}.\]
	}
\end{vd}

\begin{vd}%[Dự án EX-8-Đề Cương Toán 8]%[Dat Tien Pham]%[8H2H5-1]
	Cho tứ giác $MNPQ$ có ba góc $\widehat{M}$, $\widehat{N}$, $\widehat{P}$ vuông. Chứng minh rằng $MNPQ$ là hình chữ nhật.
	\loigiai{
		Trong tứ giác $MNPQ$, ta có $\widehat{M}+\widehat{N}+\widehat{P}+\widehat{Q}=360^\circ$.\\
		Do $\widehat{M}=\widehat{N}=\widehat{P}=90^\circ$, suy ra $\widehat{Q}=90^\circ$.\\
		Tứ giác $MNPQ$ có bốn góc vuông nên là hình chữ nhật.
	}
\end{vd}

\begin{vd}%[Dự án EX-8-Đề Cương Toán 8]%[Dat Tien Pham]%[8H2V5-1]
	\immini
	{Cho tam giác $ABC$ vuông tại $A$, $O$ là trung điểm của $BC$. Lấy điểm $D$ đối xứng với $A$ qua $O$. Chứng minh rằng tứ giác $ABDC$ là hình chữ nhật.			
	}{\begin{tikzpicture}[line join= round, cap = round, thick, font = \small, scale = .7]
			\tikzset{
				markx/.pic={\draw (45:.1)--(-135:.1) (-45:.1)--(135:.1);},
				markl/.pic={\draw (90:.1)--(-90:.1);},
				markll/.pic={
					\draw[shift={(180:.02)}] (90:.1)--(-90:.1);
					\draw[shift={(0:.02)}] (90:.1)--(-90:.1);
				},
				marklll/.pic={
					\draw[shift={(180:.03)}] (90:.1)--(-90:.1);
					\draw (90:.1)--(-90:.1);
					\draw[shift={(0:.03)}] (90:.1)--(-90:.1);
				}
			}
			
			\path 
			(0:0) coordinate (C)
			+(0:5) coordinate (D)
			+(90:3) coordinate (A)
			($(A)+(D)-(C)$) coordinate (B)
			($(B)!.5!(C)$) coordinate (O);
			\draw 
			(A)--(B)--(D)--(C)--cycle
			(B)--(C)
			(A)--(D)
			;
			\path (A)--(O) pic[pos = .5, sloped]{markl}
			(D)--(O) pic[pos = .5, sloped]{markl}
			(O)--(B) pic[pos = .5, sloped]{markll}
			(O)--(C) pic[pos = .5, sloped]{markll};
			
			\foreach \x/\g in {C/-90,D/-90,A/90,B/90,O/-90}
			\fill (\x) circle (1pt)
			+(\g:3mm) node{$\x$};
	\end{tikzpicture}}
	\loigiai{Ta có $D$ đối xứng với $A$ qua $O$, suy ra $OA=OD$, mà $OB=OC$.\\
	Suy ra tứ giác $ABDC$ có hai đường chéo cắt nhau tại trung điểm của mỗi đường nên là hình bình hành.\\
		Ta có $AB\parallel CD$ và $\widehat{BAC}=90^\circ$, suy ra $\widehat{ACD}=90^\circ$.\\
		Do $ABDC$ là hình bình hành nên $\widehat{BDC}=\widehat{BAC}=90^\circ$ và $\widehat{ABD}=\widehat{ACD}=90^\circ$.\\
		Suy ra $ABDC$ là hình chữ nhật. }
\end{vd}
\begin{vd}%[Dự án EX-8-Đề Cương Toán 8]%[Dat Tien Pham]%[8H2V5-1]
	\immini
	{Cho tam giác $ABC$ có điểm $O$ thuộc $BC$ sao cho $OA=OB=OC$. Lấy điểm $D$ đối xứng với $A$ qua $O$. Chứng minh rằng tứ giác $ABDC$ là hình chữ nhật.}{
		\begin{tikzpicture}[line join= round, cap = round, thick, font = \small, scale = 0.7]
			\tikzset{
				markx/.pic={\draw (45:.1)--(-135:.1) (-45:.1)--(135:.1);},
				markl/.pic={\draw (90:.1)--(-90:.1);},
				markll/.pic={
					\draw[shift={(180:.02)}] (90:.1)--(-90:.1);
					\draw[shift={(0:.02)}] (90:.1)--(-90:.1);
				},
				marklll/.pic={
					\draw[shift={(180:.03)}] (90:.1)--(-90:.1);
					\draw (90:.1)--(-90:.1);
					\draw[shift={(0:.03)}] (90:.1)--(-90:.1);
				}
			}
			
			\path 
			(0:0) coordinate (C)
			+(0:5) coordinate (D)
			+(90:3) coordinate (A)
			($(A)+(D)-(C)$) coordinate (B)
			($(B)!.5!(C)$) coordinate (O);
			\draw 
			(A)--(B)--(D)--(C)--cycle
			(B)--(C)
			(A)--(D)
			;
			\path 
			(A)--(O) pic[pos = .5, sloped]{markl}
			(D)--(O) pic[pos = .5, sloped]{markl}
			(O)--(B) pic[pos = .5, sloped]{markl}
			(O)--(C) pic[pos = .5, sloped]{markl};
			\draw pic[draw, angle radius = 12pt, "1", angle eccentricity = 1.5]{ angle = O--A--B};
			\draw pic[draw, angle radius = 12pt, "1", angle eccentricity = 1.5]{ angle = A--B--O};
			\draw pic[draw, angle radius = 15pt, "2", angle eccentricity = 1.5]{ angle = C--A--O};
			\draw pic[draw, angle radius = 12pt, "1", angle eccentricity = 1.5]{ angle = O--C--A};
			
			\foreach \x/\g in {C/-90,D/-90,A/90,B/90,O/-90}
			\fill (\x) circle (1pt)
			+(\g:3mm) node{$\x$};
	\end{tikzpicture}}
	\loigiai{
		Ta có $OA=OB=OC$, suy ra hai tam giác $OAB$ và $OAC$ cân tại $O$.\\
		Ta có $\widehat{A}_1=\widehat{B}_1$; $\widehat{A}_2=\widehat{C}_1$ và $\widehat{A}_1+\widehat{A}_2+\widehat{B}_1+\widehat{C}_1=180^\circ$ (tổng ba góc trong $\triangle ABC$).\\
		Suy ra $\widehat{A}_1+\widehat{A}_2=\widehat{B}_1+\widehat{C}_1=90^\circ$, suy ra $\widehat{BAC}=\widehat{A}_1+\widehat{A}_2=90^\circ$. \quad $(1)$\\
		Lại có $D$ đối xứng với $A$ qua $O$ nên $OD=OA$, kết hợp với $OB=OA$\\
		Suy ra $ABDC$ là hình bình hành.\quad $(2)$.\\
		Từ $(1)$ và $(2)$ suy ra $ABDC$ là hình chữ nhật.}
\end{vd}

\subsubsection{Bài tập}
\begin{bt}%[Dự án EX-8-Đề Cương Toán 8]%[Dat Tien Pham]%[8H2V5-1]
	Tìm $x$ ở hình vẽ bên dưới
	\begin{center}
		\begin{tikzpicture}[>=stealth,line join=round,line cap=round,line width=0.6pt,font=\footnotesize,scale=1]
			\coordinate[label=below left:$B$](B) at (0,0);
			\coordinate[label=below right:$C$](C) at (10,0);
			\coordinate[label=above left:$A$](A) at (2,4);
			\coordinate[label=below:$M$] (M) at ($(B)!0.5!(C)$);
			\coordinate[label=left:$6 \text{ cm}$] (E) at ($(A)!0.5!(B)$);
			\coordinate[label=above right:$8 \text{ cm}$] (F) at ($(A)!0.5!(C)$);
			%	\coordinate[label=below:$\text{Hình 14}$] (H) at (5,-0.5);
			\coordinate[label=left:$x$] (G) at ($(A)!0.5!(M)$);
			\draw (A)--(B)--(C)--cycle;
			\draw(A)--(M);
			\path (B)--(M)node[midway]{$|$} (C)--(M)node[midway]{$|$};
			\draw pic[draw,angle radius=3mm] {right angle = B--A--C};
			\foreach \x in {A,B,C,M}
			\fill (\x) circle (1.5pt);
		\end{tikzpicture}
	\end{center}
	\loigiai{Áp dụng định lý Pythagore cho tam giác vuông $ABC$, ta có
		$$BC=\sqrt{AB^2+AC^2}=\sqrt{6^2+8^2}=10\text{ cm}.$$
		Trong tam giác $ABC$ vuông tại $A$, $AM$ là đường trung tuyến, do đó
		$$AM=\dfrac{BC}{2}=5 \text{ cm}.$$
			}
\end{bt}

\begin{bt}%[Dự án EX-8-Đề Cương Toán 8]%[Dat Tien Pham]%[8H2V5-1]
	Cho tam giác $ABC$ vuông tại $A$ ($AB<AC$). Gọi $D$ là trung điểm của $BC$. Vẽ $DE\parallel AB$, vẽ $DF\parallel AC$ $(E\in AC, F\in AB)$. Chứng minh rằng tứ giác $AEDF$ là hình chữ nhật;
	\loigiai{
		\begin{center}
			\begin{tikzpicture}[>=stealth,line join=round,line cap=round,line width=0.5pt,font=\footnotesize,scale=1]
				\coordinate[label=below left:$B$](B) at (0,0);
				\coordinate[label=below right:$C$](C) at (10,0);
				\coordinate[label=above left:$A$](A) at (2,4);
				\coordinate[label=below :$D$] (D) at ($(B)!0.5!(C)$);
				\coordinate (M) at ($(A)+(D)-(B)$);
				\coordinate (N) at ($(A)+(D)-(C)$);
				\coordinate[label=above:$E$] (E) at (intersection cs:first line={(A)--(C)}, second line={(D)--(M)});
				\coordinate[label=left:$F$] (F) at (intersection cs:first line={(A)--(B)}, second line={(D)--(N)});
				\draw (A)--(B)--(C)--(A)--(D) (F)--(D)--(E)--(F);
				\draw pic[draw,angle radius=3mm] {right angle = B--A--C};
				\path (D)--(B) node[midway]{$|$} (D)--(C) node[midway]{$|$};
				\foreach \x in {A,B,C,D,E,F}
				\fill (\x) circle (1.5pt);
			\end{tikzpicture}
		\end{center}
		Ta có $DE\parallel AF$ và $ DF\parallel AE$, suy ra tứ giác $AEDF$ là hình bình hành.\\
		Mặt khác, tứ giác $AEDF$ có $\widehat{FAE}=90^\circ$, suy ra tứ giác $AEDF$ là hình chữ nhật.
	}
\end{bt}

\begin{bt}%[Dự án EX-8-Đề Cương Toán 8]%[Dat Tien Pham]%[8H2V5-1]
	Cho tam giác $ABC$ có đường cao $AH$. Gọi $I$ là trung điểm của $AC$, trên tia $HI$ lấy điểm $K$ sao cho $I$ là trung điểm của $HK$. Chứng minh tứ giác $AHCK$ là hình chữ nhật.
	\loigiai{
		\begin{center}
			\begin{tikzpicture}[>=stealth,line join=round,line cap=round,line width=0.5pt,font=\footnotesize,scale=1]
				\coordinate[label=below left:$B$](B) at (0,0);
				\coordinate[label=below right:$C$](C) at (8,0);
				\coordinate[label=above left:$A$](A) at (3,4);
				\coordinate[label=above :$I$] (I) at ($(A)!0.5!(C)$);
				\coordinate[label=below :$H$] (H) at ($(B)!(A)!(C)$);
				\coordinate[label=right:$K$] (K) at ($(A)+(C)-(H)$);
				\draw (A)--(B)--(C)--(A)--(H) (H)--(K)--(A) (C)--(K);
				\draw pic[draw,angle radius=3mm] {right angle = A--H--C};
				\path	(A)--(I)node[rotate=-30,midway]{$\textsf{X}$}
				(C)--(I)node[rotate=-30,midway]{$\textsf{X}$};
				\path (H)--(I) node[midway]{$|$} (I)--(K) node[midway]{$|$};
				\foreach \x in {A,B,C,H,K,I}
				\fill (\x) circle (1.5pt);
			\end{tikzpicture}
		\end{center}
		Ta có $I$ là trung điểm của $HK$, do đó $HI=IK$.\\
		Lại có $AI=IC$, suy ra tứ giác $AHCE$ là hình bình hành. Mặt khác $\widehat{AHC}=90^\circ$.\\
		Suy ra tứ giác $AHCK$ là hình chữ nhật.	}
\end{bt}

\begin{bt}%[Dự án EX-8-Đề Cương Toán 8]%[Dat Tien Pham]%[8H2V5-1]
	Cho tam giác $ABC$ có đường cao $AH$. Gọi $I$ là trung điểm của $AC$, $E$ là điểm đối xứng với $H$ qua $I$. Gọi $M$, $N$ lần lượt là trung điểm của $HC$, $CE$. Các đường thẳng $AM$, $AN$ cắt $HE$ tại $G$ và $K$.
	\begin{enumerate}
		\item Chứng minh tứ giác $AHCE$ là hình chữ nhật.
		\item Chứng minh $HG=GK=KE$.
	\end{enumerate}
	\loigiai{
		\begin{center}
			\begin{tikzpicture}[>=stealth,line join=round,line cap=round,line width=0.5pt,font=\footnotesize,scale=1]
				\coordinate[label=below left:$B$](B) at (0,0);
				\coordinate[label=below right:$C$](C) at (8,0);
				\coordinate[label=above left:$A$](A) at (3,4);
				\coordinate[label=above :$I$] (I) at ($(A)!0.5!(C)$);
				\coordinate[label=below :$H$] (H) at ($(B)!(A)!(C)$);
				\coordinate[label=right:$E$] (E) at ($(A)+(C)-(H)$);
				\coordinate[label=below :$M$] (M) at ($(H)!0.5!(C)$);
				\coordinate[label=right :$N$] (N) at ($(E)!0.5!(C)$);
				\coordinate[label=right:$G$] (G) at (intersection cs:first line={(A)--(M)}, second line={(H)--(E)});
				\coordinate[label=above:$K$] (K) at (intersection cs:first line={(A)--(N)}, second line={(H)--(E)});
				\draw (A)--(B)--(C)--(A)--(H) (H)--(E)--(A) (C)--(E) (A)--(M) (A)--(N);
				\draw pic[draw,angle radius=3mm] {right angle = A--H--C};
				\path (M)--(H) node[midway]{$|$} (M)--(C) node[midway]{$|$}
				(N)--(E) node[rotate=70,midway]{$||$} 
				(N)--(C) node[rotate=70,midway]{$||$}
				(A)--(I)node[rotate=-30,midway]{$\textsf{X}$}
				(C)--(I)node[rotate=-30,midway]{$\textsf{X}$};
				\foreach \x in {A,B,C,H,K,G,I,E,M,N}
				\fill (\x) circle (1.5pt);
			\end{tikzpicture}
		\end{center}
		\begin{enumerate}
			\item	Ta có $E$ là điểm đối xứng của $H$ qua $I$, do đó $HI=IE$.\\
			Lại có $AI=IC$, suy ra tứ giác $AHCE$ là hình bình hành. \\
			Mặt khác $\widehat{AHC}=90^\circ$.\\
			Suy ra tứ giác $AHCE$ là hình chữ nhật.	
			\item Xét tam giác $AHC$, ta có $HI$ và $AM$ là hai đường trung tuyến.\\ Do đó, $G$ là trọng tâm của tam giác $AHC$.\\
			Suy ra $\heva{&IG=\dfrac{1}{3}IH\\&IG=\dfrac{1}{2}HG.}$\\
			Xét tam giác $AEC$, ta có $AN$ và $EI$ là hai đường trung tuyến. \\Do đó, $K$ là trọng tâm của tam giác $AEC$.\\
			Suy ra $\heva{&IK=\dfrac{1}{3}IE\\&IK=\dfrac{1}{2}KE.}$\\
			Lại có $IH=IE$ suy ra $IG=IK$. \\
			Do đó, $HG=KE=GI+IK$.\\
			Vậy $HG=GK=KE$.
		\end{enumerate}
	}
\end{bt}

\begin{bt}%[Dự án EX-8-Đề Cương Toán 8]%[Dat Tien Pham]%[8H2V5-1]%[8H2C4-1]
	Cho tam giác $ABC$ cân tại $A$. Gọi $H$, $D$ lần lượt là trung điểm của các cạnh $BC$ và $AB$.
	\begin{enumerate}
		\item Chứng minh rằng tứ giác $ADHC$ là hình thang.
		\item Gọi $E$ là điểm đối xứng với $H$ qua $D$. Chứng minh rằng tứ giác $AHBE$ là hình chữ nhật.
		\item Tia $CD$ cắt $AH$ tại $M$ và cắt $BE$ tại $N$. Chứng minh rằng tứ giác $AMBN$ là hình bình hành.
	\end{enumerate}	
	\loigiai{
		\begin{center}
			\begin{tikzpicture}[scale=1, font=\footnotesize, line join=round, line cap=round, >=stealth]
				\path 
				(2,4) coordinate (A)
				(0,0) coordinate (B)
				(4,0) coordinate (C)
				(2,0) coordinate (H)
				(1,2) coordinate (D)
				(0,4) coordinate (E)
				(2,1.33) coordinate (M)
				(0,2.67) coordinate (N)
				;
				\draw (A)--(B)--(C)--cycle (E)--(H)--(A)--(E)--(B)  (C)--(N) (B)--(M) (A)--(N);
				
				\foreach \p/\r in {A/90,B/180,C/0,H/-90,D/90,E/90,M/45,N/180}
				\fill (\p) circle (1.5pt) node[shift={(\r:3mm)}]{$\p$};
				% Style for one tick mark
				\tikzset{
					single tick/.style={
						decoration={markings,
							mark=at position 0.5 with {\draw (0,-0.2) -- (0,0.2);}},
						postaction={decorate}
					}
				}
				
				% Style for two tick marks
				\tikzset{
					double tick/.style={
						decoration={markings,
							mark=at position 0.45 with {\draw (0,-0.2) -- (0,0.2);},
							mark=at position 0.5 with {\draw (0,-0.2) -- (0,0.2);}},
						postaction={decorate}
					}
				}
				
				% Style for three tick marks
				\tikzset{
					triple tick/.style={
						decoration={markings,
							mark=at position 0.45 with {\draw (0,-0.2) -- (0,0.2);},
							mark=at position 0.5 with {\draw (0,-0.2) -- (0,0.2);},
							mark=at position 0.55 with {\draw (0,-0.2) -- (0,0.2);}},
						postaction={decorate}
					}
				}
				
				% Draw segments with marks
				\draw[double tick, color=black] (C) -- (H);
				\draw[double tick, color=black] (H) -- (B);
				\draw[single tick, color=black] (A) -- (D); 
				\draw[single tick, color=black] (D) -- (B);
				\draw[triple tick, color=black] (E) -- (D); 
				\draw[triple tick, color=black] (D) -- (H);
			\end{tikzpicture}
		\end{center}
		\begin{enumerate}
			\item Ta có $\heva{&H \text{ là trung điểm } BC\\&D \text{ là trung điểm } AB} \Rightarrow DH$ là đường trung bình của $\triangle ABC$.\\
			$\Rightarrow DH \parallel AC \Rightarrow ADHC$ là hình thang.
			\item Ta có $\heva{&DA=DB\\&DE=DH} \Rightarrow AHBE$ là hình bình hành.\\
			Trong $\triangle ABC$ cân tại $A$ có $AH$ là đường trung tuyến nên $AH$ cũng là đường cao.\\ Suy ra $\widehat{AHB}=90^\circ$.\\
			Khi đó hình bình hành $AHBE$ có $\widehat{AHB} = 90^\circ$ nên $AHBE$ là hình chữ nhật.
			\item Do $AHBE$ là hình chữ nhật nên $AH \parallel EB$. \\
			$\Rightarrow \widehat{MAD}=\widehat{NBD}$ (hai góc so le trong).\\
			Xét $\triangle ADM$ và $\triangle BDN$ có
			\begin{itemize}
			\item $\widehat{MAD}=\widehat{NDB}$
			\item $DA=DB$
			\item $\widehat{ADM}=\widehat{NDB}$ (đối đỉnh)
			\end{itemize}
			$\Rightarrow \triangle ADM=\triangle BDN $ (g.c.g).\\
			$\Rightarrow AM=BN$ (hai cạnh tương ứng).\\
			Ta có $\heva{&AM \parallel BN\\&AM=BN} \Rightarrow AMBN$ là hình bình hành.
		\end{enumerate}	
	}
\end{bt}




\subsection{Hình vuông}
\subsubsection{Kiến thức trọng tâm}
\begin{tomtat}
	\begin{dn}
		\immini{Hình vuông là tứ giác có bốn góc vuông và bốn cạnh bằng nhau.}{
				\begin{tikzpicture}[>=stealth, scale=0.7]
					% Define points
					\coordinate (A) at (0,0);
					\coordinate (B) at (3,0);
					\coordinate (C) at (3,3);
					\coordinate (D) at (0,3);
					
					% Style for one small tick mark
					\tikzset{
						single tick/.style={
							decoration={markings,
								mark=at position 0.5 with {\draw (0,-0.1) -- (0,0.1);}}, % Smaller tick (0.2 units total length)
							postaction={decorate}
						}
					}
					
					% Draw segments with one small tick mark
					\draw[single tick, color=black] (A) -- (B);
					\draw[single tick, color=black] (B) -- (C);
					\draw[single tick, color=black] (C) -- (D);
					\draw[single tick, color=black] (D) -- (A);
					
					% Label points
					\node[left] at (D) {D};
					\node[left] at (A) {A};
					\node[right] at (C) {C};
					\node[right] at (B) {B};
					
					% Draw points
					\fill[black] (A) circle (2pt);
					\fill[black] (B) circle (2pt);
					\fill[black] (C) circle (2pt);
					\fill[black] (D) circle (2pt);
					
					% Mark right angles (smaller size)
					\pic[draw, angle radius=3mm, angle eccentricity=0] {right angle=D--A--B};
					\pic[draw, angle radius=3mm, angle eccentricity=0] {right angle=A--B--C};
					\pic[draw, angle radius=3mm, angle eccentricity=0] {right angle=B--C--D};
					\pic[draw, angle radius=3mm, angle eccentricity=0] {right angle=C--D--A};
					\end{tikzpicture}}
	\end{dn}
	
	\begin{tc}
		Hình vuông có đầy đủ tính chất của hình chữ nhật và hình thoi.
	\end{tc}
	
	\begin{dl}
		\begin{itemize}
			\item Hình chữ nhật có hai cạnh kề bằng nhau là hình vuông.
			\item Hình chữ nhật có hai đường chéo vuông góc với nhau là hình vuông.
			\item Hình chữ nhật có một đường chéo là đường phân giác của một góc là hình vuông.
			\item Hình thoi có một góc vuông là hình vuông.
			\item Hình thoi có hai đường chéo bằng nhau là hình vuông.
		\end{itemize}
	\end{dl}
	
	
\end{tomtat}
\subsubsection{Ví dụ}

\begin{vd}%[Dự án EX-8-Đề Cương Toán 8]%[Dat Tien Pham]%[8H2N5-2]
	Cho tam giác $ABC$ vuông tại $A$ có đường phân giác $AD$. Qua $D$ kẻ đường thẳng vuông góc với $AB$ tại $E$. Qua $D$ kẻ đường thẳng vuông góc với $AC$ tại $F$. Chứng minh tứ giác $AEDF$ là hình vuông.
	\loigiai{
		\begin{center}
			\begin{tikzpicture}[>=stealth,line join=round,line cap=round,scale=1,declare function={r=5;a=75;}]
				\path 
				(-r,0) coordinate (B)
				(110:r) coordinate (A)
				(r,0) coordinate (C);
				\path[name path=t1] (A)--(B)--(C)--cycle;
				\path[name path=c1] (A) circle(2);
				\path[name intersections={of=c1 and t1,by={a,b}}];
				\path 
				($(a)!.5!(b)$) coordinate (d)
				(intersection of A--d and B--C) coordinate (D)
				($(A)!(D)!(B)$) coordinate (E)
				($(A)!(D)!(C)$) coordinate (F);
				\draw[thick] 
				(D)--(A)--(B)--(C)--(A)
				(E)--(D)--(F);
%				\path[fill=teal!60,opacity=.5] (A)--(E)--(D)--(F)--cycle;
				\foreach \a/\b/\c in {B/A/C,B/E/D,D/F/C}
				{\draw pic[draw=black,angle radius=10]{right angle=\a--\b--\c};}
				\draw pic["$1$",draw,angle eccentricity=1.3,angle radius=20]{angle=B--A--D};
				\draw pic["$2$",draw,angle eccentricity=1.3,angle radius=25]{angle=D--A--C};
				\foreach \x/\g in {A/90,B/-135,C/-45,D/-90,E/135,F/45}
				{\draw[fill=black] (\x) circle(1.5pt)+(\g:.4) node{$\x$};}
			\end{tikzpicture}
		\end{center}
		Xét tứ giác $AEDF$ có $\widehat{E}=\widehat{A}=\widehat{F}=90^\circ$. Suy ra $AEDF$ là hình chữ nhật.\\
		Mà $AD$ là tia phân giác của $\widehat{FAE}$, nên $AEDF$ là hình vuông.
	}
\end{vd}
\begin{vd}%[Dự án EX-8-Đề Cương Toán 8]%[Dat Tien Pham]%[8H2H5-2]
	Cho hình vuông $ABCD$. Trên các cạnh $AB$, $BC$, $CD$, $DA$ lần lượt lấy các điểm $E$, $F$, $G$, $H$ sao cho $AE=BF=CG=DH$. Chứng minh tứ giác $EFGH$ là hình vuông.
	\loigiai{
		\immini{
			Ta thấy $AB=AE+EB$, $BC=BF+FC$, $CD=CG+GD$, $DA=DH+HA$. Mà
			\[
			\begin{aligned}
				&AB=BC=CD=DA;\\
				&AE=BF=CG=DH.
			\end{aligned}
			\]
			suy ra $EB=FC=GD=HA$. Khi đó
			\[
			\triangle AEH=\triangle BFE=\triangle CGF=\triangle DHG \quad \text{(c-g-c)}
			\]
			Suy ra $EH=FE=FG=GH$ suy ra $EFGH$ là hình thoi.\\
			Lại có $\widehat{AEH}=\widehat{BFE}$. Mà $\widehat{BFE}+\widehat{BEF}=90^\circ$, nên $\widehat{AEH}+\widehat{BEF}=90^\circ$. Do đó, $\widehat{FEH}=90^\circ$.\\
			Vậy tứ giác $EFGH$ là hình vuông.
		}{
			\begin{tikzpicture}[>=stealth,line join=round,line cap=round,scale=1,declare function={r=6;a=75;}]
				\path 
				(0,0) coordinate (D)
				(0,r) coordinate (A)
				(r,0) coordinate (C)
				(r,r) coordinate (B)
				($(A)!.3!(B)$) coordinate (E)
				($(B)!.3!(C)$) coordinate (F)
				($(C)!.3!(D)$) coordinate (G)
				($(D)!.3!(A)$) coordinate (H);
				\draw[thick] 
				(A) edge node[midway,sloped,rotate=45,anchor=center]{$-$} (E)
				(B) edge node[midway,sloped,rotate=45,anchor=center]{$-$} (F)
				(C) edge node[midway,sloped,rotate=45,anchor=center]{$-$} (G)
				(D) edge node[midway,sloped,rotate=45,anchor=center]{$-$} (H)
				(E)--(F)--(G)--(H)--(E)--(B)
				(F)--(C)
				(G)--(D)
				(H)--(A);
%				\path[fill=teal!60,opacity=.5] (E)--(F)--(G)--(H)--cycle;
				\foreach \a/\b/\c in {D/A/B,A/B/C,B/C/D,C/D/A}
				{\draw pic[draw=black,angle radius=10]{right angle=\a--\b--\c};}
				\foreach \x/\g in {A/135,B/45,C/-45,D/-135,E/90,F/0,G/-90,H/180}
				{\draw[fill=black] (\x) circle(1.5pt)+(\g:.4) node{$\x$};}
			\end{tikzpicture}
	}}
\end{vd}

\begin{vd}%[Dự án EX-8-Đề Cương Toán 8]%[Dat Tien Pham]%[8H2H5-2]
	Cho hình chữ nhật $ABCD$ ($AD<AB<2AD$). Vẽ các tam giác vuông cân $ABI$, $CDK$ $\left(\widehat{I}=\widehat{K}=90^\circ\right)$, $I$ và $K$ nằm trong $ABCD$. Gọi $E$ là giao điểm của $A$ và $DK$; $F$ là giao điểm của $BI$ và $CK$. Chứng minh rằng
	\begin{enumerate}
		\item $EF$ song song với $CD$.
		\item $EKFI$ là hình vuông.
	\end{enumerate}
	\loigiai{
		\begin{enumerate}
			\item 
			\immini
			{Tam giác $CDK$ vuông cân tại $K$ nên
				$$
				\widehat{KDC}=\widehat{KCD}=\frac{180^\circ-90^\circ}{2}=45^\circ.
				$$
				Mà $\widehat{KDC}+\widehat{EDA}=90^\circ$, suy ra $\widehat{EDA}=45^\circ$;\\ Ta $\widehat{KCD}+\widehat{FCB}=90^\circ$ suy ra $\widehat{FCB}=45^\circ$.\\
				Ta có $\triangle AIB$ vuông cân tại $I$ nên
					$$
				\widehat{IAB}=\widehat{IBA}=\frac{180^\circ-90^\circ}{2}=45^\circ.\\
				$$
				Suy ra
				$
				\widehat{EAD}=90^\circ-\widehat{IAB}=45^\circ$ và
				$\widehat{FBC}=90^\circ-\widehat{IBA}=45^\circ.$
				
			}
			{
				\begin{tikzpicture}[line join = round, line cap = round,>=stealth,font=\footnotesize,scale=0.85]
					\def\x{3}
					\path (0,0) coordinate (D) (6,0)coordinate(C) (0,4)coordinate(A) (6,4)coordinate(B) ($(D)!.5!(C)+(90:\x)$)coordinate (K) ($(A)!.5!(B)+(-90:\x)$)coordinate (I) (intersection of D--K and A--I) coordinate (E) (intersection of C--K and B--I) coordinate (F);
					\draw (A)--(B)--(C)--(D)--cycle (A)--(I)--(B) (D)--(K)--(C) (E)--(F); 
					\foreach\i/\j in{A/90,B/90,D/-90,C/-90,K/90,I/-90,E/180,F/0}
					\fill[black](\i)circle(1.5pt)node[shift=(\j:2.5mm)]{$\i$};
					\pic[draw, angle radius=8pt]{right angle=A--I--B};
					\pic[draw, angle radius=8pt]{right angle=D--K--C};
				\end{tikzpicture}
			}
			Xét $\triangle EAD$ và $\triangle FBC$ có $\heva{& \widehat{ EAD}=\widehat{FBC}=45^\circ \\ & AD=BC \,(ABCD \,\text {là hình chữ nhật);}\\&\widehat{EDA}=\widehat{FCB}=45^\circ.}$\\
			Suy ra $\triangle EAD=\triangle FBC$ (g $-$ c $-$ g). \\
			Suy ra $DE=CF$.\quad $(1)$.\\
			Mặt khác, $\triangle CDK$ cân tại $K$ nên $KD=KC$.\quad $(2)$.\\
			Từ ($1$) và ($2$) suy ra $KD-DE=KC-CF$ hay $KE=KF$.\\
			Ta có $\triangle KEF$ vuông tại $K$ và có $KE = KF$ nên $\widehat{KEF}=45^\circ$.\\
			Ta lại có $\widehat{KDC}=45^\circ$. \\
			Suy ra $EF\parallel CD$ (hai góc đồng vị bằng nhau).
			\item $\triangle EAD$ có $\widehat{EDA}=\widehat{EAD}=45^\circ$ nên $\widehat{AED}=90^\circ$. \\
			Suy ra $\widehat{KEI}=\widehat{AED}=90^\circ$ (hai góc đối đỉnh).\\
			Tứ giác $EKFI$ có $\widehat{KEI}=\widehat{EKF}=\widehat{EIF}=90^\circ$ nên $EKFI$ là hình chữ nhật.\\
			Ta lại có $KE = KF$ (theo câu $a$), suy ra $EKFI$ là hình vuông.
		\end{enumerate}
	}
\end{vd}

\begin{vd}%[Dự án EX-8-Đề Cương Toán 8]%[Dat Tien Pham]%[8H2H5-2]
	Cho hình thoi $ABCD$ có hai hai đường chéo $AC$ và $BD$ cắt nhau tại $O$. Các tia phân giác của bốn góc ở đỉnh $O$ cắt các cạnh $AB$, $BC$, $CD$, $DA$ lần lượt tại $E$, $F$, $G$, $H$. Chứng minh tứ giác $EFGH$ là hình vuông.
	\loigiai{
		\begin{center}\begin{tikzpicture}[join = round, cap = round, thick, font = \small, scale = 1.4]
				\path 
				(0:0) coordinate (O)
				+(0:3) coordinate (C)
				+(90:2) coordinate (B)
				+(-90:2) coordinate (D)
				(O)+(0:-3) coordinate (A)
				;
				\coordinate[] (X) at ($(O)!1!45:(C)$); 
				\coordinate[] (Y) at ($(O)!1!-45:(C)$); 
				\coordinate[] (M) at ($(O)!1!-45:(A)$); 
				\coordinate[] (N) at ($(O)!1!-45:(D)$); 
				\path[name path=lineAB] (A) -- (B); % Đường thẳng qua A và B
				\path[name path=lineMY] (M) -- (Y); % Đường thẳng qua M và Y
				\path[name intersections={of=lineAB and lineMY, by=E}];
% Tìm giao điểm của (C,D) và (M,Y), gán vào G
\path[name path=lineCD] (C) -- (D);
\path[name path=lineMY] (M) -- (Y);
\path[name intersections={of=lineCD and lineMY, by=G}];

% Tìm giao điểm của (B,C) và (X,N), gán vào F
\path[name path=lineBC] (B) -- (C);
\path[name path=lineXN] (X) -- (N);
\path[name intersections={of=lineBC and lineXN, by=F}];

% Tìm giao điểm của (A,D) và (X,N), gán vào H
\path[name path=lineAD] (A) -- (D);
\path[name intersections={of=lineAD and lineXN, by=H}];

				\draw 
				(A)--(B)--(C)--(D)--cycle
				(B)--(D) (A)--(C) (E)--(F)--(G)--(H)--(E) (O)--(E) (O)--(F) (O)--(G) (O)--(H)
				;
				\foreach \x/\g in {O/20,D/-90,C/0,A/180,B/90,E/90,F/90,H/-90,G/-90}
				\fill (\x) circle (1.5pt)
				+(\g:3mm) node{$\x$};
		\end{tikzpicture}\end{center}
		Do $ABCD$ là hình thoi nên $AC\perp BD$, $AC$ là tia phân giác của $\widehat{BAD}$.\\
		Suy ra \\
		$\widehat{AOB}=\widehat{AOD}=90^\circ$; $\widehat{OAE}=\widehat{OAH}$. \\
		Vì $OE$, $OH$ lần lượt là tia phân giác của $\widehat{AOB},\widehat{AOD}$ nên $\widehat{AOE}=\widehat{AOH}=45^\circ$.\\
		Ta có $\widehat{HOE}=\widehat{AOE}+\widehat{AOH}=45^\circ+45^\circ=90^\circ$.\\
		Vì $OE$, $OF$ lần lượt là tia phân giác của $\widehat{AOB},\widehat{BOC}$ nên $\widehat{BOE}=\widehat{BOF}=45^\circ$.\\
		Ta có $\widehat{EOF}=\widehat{BOE}+\widehat{BOF}=45^\circ+45^\circ=90^\circ$.\\
		Suy ra $\widehat{HOF}=\widehat{HOE}+\widehat{EOF}=90^\circ+90^\circ=180^\circ$, do đó ba điểm $H$, $O$, $F$ thẳng hàng.\\
		Vì $OF$, $OG$ lần lượt là tia phân giác của $\widehat{BOC},\widehat{COD}$ nên $\widehat{FOC}=\widehat{COG}=45^\circ$.\\
		Ta có $\widehat{FOG}=\widehat{FOC}+\widehat{COG}=45^\circ+45^\circ=90^\circ$.\\
		Suy ra $\widehat{EOG}=\widehat{EOF}+\widehat{FOG}=90^\circ+90^\circ=180^\circ$, do đó ba điểm $E$, $O$, $G$ thẳng hàng.\\
		Suy ra $EG\perp FH$.\\
		Xét $\triangle OAE$ và $ \triangle OAH$ ta có
		\begin{itemize}
			\item $\widehat{OAE}=\widehat{OAH}$
			\item $OA$ là cạnh chung
			\item $\widehat{AOE}=\widehat{AOH}$
		\end{itemize}
		Suy ra $\triangle OAE=\triangle OAH$ (g.c.g)\\ 
		Suy ra $OE=OH$ (cạnh tương ứng).\\
		Xét $\triangle OBE$ và $ \triangle OBF$ ta có
		\begin{itemize}
			\item $\widehat{OBE}=\widehat{OBF}$
			\item $OB$ là cạnh chung
			\item $\widehat{BOE}=\widehat{BOF}$
		\end{itemize}
		Suy ra $\triangle OBE=\triangle OBF$ (g.c.g)\\ Suy ra $OE=OF$ (cạnh tương ứng).\\	
		Xét $\triangle OCF$ và $ \triangle OCG$ ta có
		\begin{itemize}
			\item $\widehat{OCF}=\widehat{OCG}$
			\item $OC$ là cạnh chung
			\item $\widehat{COF}=\widehat{COG}$
		\end{itemize}
		Suy ra $\triangle OCF=\triangle OCG$ (g.c.g)\\Suy ra $OF=OG$ (cạnh tương ứng).\\
		Xét $\triangle ODG$ và $ \triangle ODH$ ta có
		\begin{itemize}
			\item $\widehat{ODG}=\widehat{ODH}$
			\item $OD$ là cạnh chung
			\item $\widehat{DOG}=\widehat{DOH}$
		\end{itemize}
		Suy ra $\triangle ODG=\triangle ODH$ (g.c.g)\\ Suy ra $OG=OH$ (cạnh tương ứng).\\	
		Suy ra $OE=OF=OG=OH$. \\
		Vì vậy $OE+OG=OF+OH$ hay $EG=FH$.\\
		Tứ giác $EFGH$ có hai đường chéo $EG$, $HF$ cắt nhau tại trung điểm $O$ của mỗi đường nên $EFGH$ là hình bình hành.\\
		Hình bình hành $EFGH$ có $EG=HF$ nên $EFGH$ là hình chữ nhật.\\
		Hình chữ nhật $EFGH$ có $EG\perp FH$ nên $EFGH$ là hình vuông.
	}
\end{vd}
\subsubsection{Bài tập vận dụng}


\begin{bt}%[Dự án EX-8-Đề Cương Toán 8]%[Dat Tien Pham]%[8H2N5-2]
	Cho hình bình hành $ABCD$. Gọi $DE$, $BK$ lần lượt là đường phân giác của hai góc $\widehat{ADB}$, $\widehat{DBC}$ ($E\in AB$, $K\in CD$).
	\begin{enumerate}
		\item Chứng minh $DE\parallel BK$.
		\item Giả sử $DE\perp AB$. Chứng minh $DA=DB$.
		\item Trong trường hợp $DE\perp AB$, tìm số đo của $\widehat{ADB}$ để tứ giác $DEBK$ là hình vuông.
	\end{enumerate}
	\loigiai{
		\begin{center}
			\begin{tikzpicture}[font=\footnotesize,line join=round,line cap=round,>=stealth]
				\draw (0,0) coordinate(D)--(0:4) coordinate(C)--++(120:3) coordinate(B)--++(180:4) coordinate(A)--(D);	
				\path ($(A)!(D)!(B)$) coordinate(E) ($(C)!(B)!(D)$) coordinate(K);
				\draw (E)--(D) (B)--(K) (B)--(D);	
				\foreach \y/\g in {A/90,B/90,C/-90,D/-90,E/90,K/-90}
				{\fill (\y) circle(1pt)+(\g:.25) node{$\y$};}
				\draw pic[draw,angle eccentricity=1.3,angle radius=20]{angle=E--D--A};
				\draw pic[draw,angle eccentricity=1.3,angle radius=25]{angle=B--D--E};
				\pic [draw, angle radius=10pt] {angle=K--B--C};
				\pic [draw, angle radius=8pt] {angle=K--B--C};
				\pic [draw, angle radius=10pt] {angle=D--B--K};
				\pic [draw, angle radius=12pt] {angle=D--B--K};
			\end{tikzpicture}
		\end{center}
		\begin{enumerate}
			\item Vì $ABCD$ là hình bình hành nên $AD\parallel BC$. Suy ra $\widehat{ADB}=\widehat{DBC}$ (hai góc so le trong).\\
			 Do đó $\dfrac{\widehat{ADB}}{2}=\dfrac{\widehat{DBC}}{2}$. Suy ra $\widehat{EDB}=\widehat{KBD}$ (do $DE$, $BK$ là đường phân giác của $\widehat{ADB}$ và $\widehat{DBC}$).\\
			  Mà hai góc này ở vị trí so le trong nên $DE\parallel BK$.
			\item Xét $\triangle DAB$ có $DE$ vừa là đường cao vừa là đường phân giác.\\
			 Suy ra $\triangle DAB$ cân tại $D$. Khi đó, $DA=DB$.
			\item Xét tứ giác $DEBK$ có $DE\parallel BK$, $BE\parallel DK$.\\
			 Suy ra $DEBK$ là hình bình hành.\\
			  Mà $\widehat{E}=90^\circ$ nên $DEBK$ là hình chữ nhật.\\
			   Để tứ giác $DEBK$ là hình vuông thì $DE=EB$.\\
			   Mà $\triangle DAB$ cân tại $D$ nên $DE$ vừa là đường cao vừa là trung tuyến của $\triangle DAB$.\\
			   Suy ra $DE=EB=\dfrac{AB}{2}$ nên $\triangle DAB$ vuông tại $D$ hay $\widehat{ADB}=90^\circ$.
		\end{enumerate}
	}
\end{bt}
\begin{bt}%[Dự án EX-8-Đề Cương Toán 8]%[Dat Tien Pham]%[8H2H5-2]
	Cho hình vuông $ABCD$. Lấy điểm $E$ thuộc $CD$, $F$ thuộc tia đối của tia $BC$ sao cho $BF=DE$.
	\begin{enumerate}
		\item Chứng minh $\triangle AEF$  vuông cân.
		\item Gọi $I$ là trung điểm $EF$. Chứng minh ba điểm $B$, $I$, $D$ thẳng hàng.
		\item Lấy điểm $K$ sao cho $I$ là trung điểm của đoạn thẳng $AK$. Chứng minh $AEKF$ là hình vuông.
	\end{enumerate}
	\loigiai{
		\immini{
			\begin{enumerate}
				\item Xét $\triangle ABF$ và $\triangle$ $ADE$ có\\
				$\heva{& AD=AB \text{ ( $ABCD$ là hình vuông)}\\& \widehat{ABF}=\widehat{ADE}=90^\circ \\& DE=BF \text{  (gt).}}$\\
				Suy ra $\triangle ABF=\triangle ADE$ (c.g.c)\\
				$\Rightarrow AE=AF$ và $\widehat{BAF}=\widehat{DAE}$.\\
				Ta có $\widehat{BAF}+\widehat{BAE}=\widehat{DAE}+\widehat{BAE}=90^\circ$.\\
				Suy ra $\triangle AEF$ vuông cân tại $A$.
				\item 
				Xét $\triangle AEF$ vuông có $AI$ là trung tuyến nên $IA=IE=IF$.\\
				Lại có $\triangle CEF$ vuông tại $C$ có $CI$ là trung tuyến.\\
				Suy ra $IC=IE$ hay $IC=IA\\ \Rightarrow I$ thuộc trung trực của AC.\\
				Mà $BD$ lại là trung trực $AC$ ($ABCD$ là hình vuông).\\
				Vậy $B$ $D$, $I$ thẳng hàng.
			\end{enumerate}
		}{
			\begin{tikzpicture}[scale=1.2,line join=round, line cap=round,font=\footnotesize,>=stealth]
				\def\a{4} %cạnh
				\def\b{1.8}
				\path (0:0) coordinate (D)
				++(0:\a) coordinate (C)
				++(90:\a) coordinate (B)
				++(180:\a) coordinate (A)
				(D)++(0:\b) coordinate (E)
				(B)++(90:\b) coordinate (F)
				($(E)!0.5!(F)$) coordinate (I)
				($(A)!2!(I)$) coordinate (K);
				\draw (D)--(C)--(B)--(A)--(D) (A)--(E)--(K)--(F)--(A) (F)--(E) (A)--(K) (F)--(B) (I)--(C)--(A);
				\draw [dashed] (D) -- (B);
				\foreach \x/ \goc in {A/180,D/-90,C/-90,E/-90,K/-90,I/-90,F/90,B/0} 
				\fill (\x) circle (1pt)
				($(\x)+(\goc:3mm)$) node {$\x$};
				\draw [decoration={markings, mark=at position 0.5 with {\draw (0,-0.2) -- (0,0.2);}}, postaction={decorate}] (D) -- (E);
				\draw [decoration={markings, mark=at position 0.5 with {\draw (0,-0.2) -- (0,0.2);}}, postaction={decorate}] (B) -- (F);
				\draw [decoration={markings,
					mark=at position 0.45 with {\draw (0,-0.2) -- (0,0.2);},
					mark=at position 0.5 with {\draw (0,-0.2) -- (0,0.2);}},
				postaction={decorate}] (I) -- (F);
				\draw [decoration={markings,
					mark=at position 0.55 with {\draw (0,-0.2) -- (0,0.2);},
					mark=at position 0.5 with {\draw (0,-0.2) -- (0,0.2);}},
				postaction={decorate}] (I) -- (E);
			\end{tikzpicture}
		}	
		\begin{enumerate}
			\item[c)]  Tứ giác $AEKF$ có $I$ là trung điểm của $AK$ và $EF$ nên là hình bình hành.\\
			Mà $AE=AF$ và $\widehat{EAF}=90^\circ$ ($\triangle AEF$ vuông cân).\\
			Suy ra $AEKF$ là hình vuông.
		\end{enumerate}
	}
\end{bt}

\begin{bt}%[Dự án EX-8-Đề Cương Toán 8]%[Dat Tien Pham]%[8H2V5-2]
	Cho hình vuông $ABCD$. Trên các cạnh $AD$, $DC$ lần lượt lấy các điểm $E$, $F$ sao cho $AE=DF$. Chứng minh
	\begin{enumerate}
		\item Các tam giác $ADF$ và $BAE$ bằng nhau.
		\item $BE$ vuông góc với $AF$.
	\end{enumerate}
	\loigiai{
		\immini{\begin{enumerate}
				\item Xét $\triangle ADF$ và $\triangle BAE$ có
				\begin{itemize}
					\item $DF=AE$ (gt);
					\item $\widehat{ADF}=\widehat{BAE}=90^\circ$;
					\item $AD=AB$ ($ABCD$ là hình vuông).
				\end{itemize}
				Vậy $\triangle ADF=\triangle BAE$ (c.g.c).
				\item Gọi $I$ là giao điểm của $BE$ và $AF$.\\
				 Ta có $\widehat{AEI}=\widehat{DFA}$ (do $\triangle ADF=\triangle BAE$).\\ Do đó $\widehat{EAI}+\widehat{AEI}=\widehat{EAI}+\widehat{DFA}=90^\circ$.\\ Suy ra $BE$ vuông góc với $AF$.
			\end{enumerate}
		}{
			\begin{tikzpicture}[>=stealth,scale=1,line join=round,line cap=round]
				\def \h{4}
				\path (0,0) coordinate (D)
				(\h,\h) coordinate (B)
				(0,\h) coordinate (A)
				(\h,0) coordinate (C)
				($(A)!2/3!(D)$) coordinate (E)
				($(D)!2/3!(C)$) coordinate (F)
				(intersection of A--F and B--E) coordinate (I);
				\draw (D) rectangle (B) (B)--(E) (A)--(F);
				\foreach \x/\goc in {A/90,B/90,C/-90,D/-90,E/180,I/90,F/-90}
				{\fill[black] (\x) circle(1pt) ($(\x)+(\goc:3mm)$) node{$\x$};}
			\end{tikzpicture}
	}}
\end{bt}


\begin{bt}%[Dự án EX-8-Đề Cương Toán 8]%[Dat Tien Pham]%[8H2V5-2]
	Cho hình vuông $ABCD$. Trên tia đối của tia $BA$ lấy điểm $E$, trên tia đối của tia $CB$ lấy điểm $F$ sao cho $AE=CF$.
	\begin{enumerate}
		\item Chứng minh tam giác $EDF$ vuông cân.
		\item Gọi $I$ là trung điểm của $EF$. Chứng minh $BI=DI$.
		\item Chứng minh ba điểm $A$, $C$, $I$ thẳng hàng.
	\end{enumerate}
	\loigiai{
		\begin{center}
			\begin{tikzpicture}[>=stealth,scale=1,line join=round,line cap=round]
				\def \h{4}
				\path (0,0) coordinate (D)
				(\h,\h) coordinate (B)
				(0,\h) coordinate (A)
				(\h,0) coordinate (C)
				($(A)!3/2!(B)$) coordinate (E)
				($(C)!-3/2!(B)$) coordinate (F)
				($(E)!1/2!(F)$) coordinate (I);
				\draw (D) rectangle (B) (B)--(E) (C)--(F) (E)--(D)--(F)--(E) (B)--(I)--(D);
				\draw[dashed] (A)--(C)--(I);
				\foreach \x/\goc in {A/90,B/90,C/30,D/-120,E/90,F/-90,I/-30}
				{\fill[black] (\x) circle(1pt) ($(\x)+(\goc:3mm)$) node{$\x$};}
				\path (I)--(F) node[midway,sloped,scale=0.5]{$|$};
				\path (E)--(I) node[midway,sloped,scale=0.5]{$|$};
			\end{tikzpicture}
		\end{center}
		\begin{enumerate}
			\item Ta có $\triangle AED=\triangle CFD$ (c.g.c) do $\heva{&AE=CF\\&\widehat{EAD}=\widehat{FCD}\\&AD=CD}$.\\
			Suy ra $DE=DF$. \quad $(1)$\\
			Ta có $\widehat{ADE}=\widehat{CDF}\Rightarrow\widehat{EDF}=\widehat{EDC}+\widehat{CDF}=\widehat{EDC}+\widehat{ADE}=90^\circ$. \quad $(2)$\\
			Từ $(1)$ và $(2)$, ta có $\triangle EDF$ vuông cân.
			\item $\triangle EDF$ có $DI$ là đường trung tuyến nên $DI=\dfrac{1}{2}EF$. $\triangle EBF$ có $BI$ là đường trung tuyến nên $BI=\dfrac{1}{2}EF$. Từ đó, $DI=BI$.
			\item Ta có $ID=IB$, $CD=CB$, $AD=AB$ nên $I$, $C$, $D$ cùng thuộc đường trung trực của $BD$. \\
			Vậy $A$, $C$, $I$ thẳng hàng.
		\end{enumerate}
	}
\end{bt}

\begin{bt}%[Dự án EX-8-Đề Cương Toán 8]%[Dat Tien Pham]%[8H2V5-2]
	Cho đoạn thẳng $AB$ và điểm $M$ thuộc đoạn thẳng $AB$. Về cùng một phía của đoạn thẳng $AB$, vẽ các hình vuông $AMCD$ và $BMEF$.
	\begin{enumerate}
		\item Chứng minh $AE$ vuông góc với $BC$.
		\item Gọi $H$ là giao điểm của $AE$ và $BC$. Chứng minh ba điểm $D$, $H$, $F$ thẳng hàng.
	\end{enumerate}
	\loigiai{
		\immini{
			\begin{enumerate}
				\item
				Ta có $\heva{&\widehat{DMA}=\widehat{ABF}=45^\circ\, (\text{hình vuông})  \\& \text{mà hai góc ở vị trí đồng vị}}$\\
				$\Rightarrow DM\parallel BE$.\\
				Lại có $AC\perp DM$ nên $AC\perp BE$.\\
				Xét $\triangle AEB$ có $EM$ và $AC$ là hai đường cao cắt nhau tại $C$.\\
				Suy ra $C$ là trực tâm của $\triangle ABE$ nên $AE\perp BC$.
			\end{enumerate}
		}{
			\begin{tikzpicture}[scale=1.2,line join=round, line cap=round,font=\footnotesize,>=stealth]
				\def\a{2} %cạnh
				\path (0:0) coordinate (A)
				++(0:\a) coordinate (M)
				++(90:\a) coordinate (C)
				++(180:\a) coordinate (D);
				\def\a{3} %cạnh
				\path (M)
				++(0:\a) coordinate (B)
				++(90:\a) coordinate (F)
				++(180:\a) coordinate (E);
				\coordinate (O) at (intersection of A--C and D--M);
				\coordinate (O') at (intersection of E--B and F--M);
				\coordinate (H) at (intersection of B--C and A--E);
				\draw (A)--(M)--(C)--(D)--cycle (A)--(C)--(B) (M)--(D) (A)--(E)--(B) (C)--(H)--(O') (O)--(H)--(M);
				\draw[dashed] (D)--(F);
				\draw (M)--(B)--(F)--(E)--cycle;
				\foreach \x/ \goc in {A/-90,M/-90,B/-90,E/90,F/90,D/90,C/0,H/135,O/-90,O'/90} 
				\fill (\x) circle (1pt)
				($(\x)+(\goc:3mm)$) node {$\x$};
			\end{tikzpicture}
		}
		\begin{enumerate}
			\setcounter{enumi}{1}
			\item Gọi $O$ là giao điểm của $AC$ và $MD$, $O'$ là giao điểm của $EB$ và $MF$.\\
			Suy ra $O$ là trung điểm của $AC$ và $MD$, $O'$ là trung điểm của $EB$ và $MF$.\\
			Xét $\triangle AHC$ vuông tại $H$ có
			$HO$ là trung tuyến ứng cạnh huyền.\\
			Suy ra $HO=\dfrac{1}{2}AC$ hay $HO=\dfrac{1}{2}MD$.\\
			Vậy $DHM$ vuông tại $H$ hay $MH\perp DH$.\\
			Chứng minh tương tự ta có $\triangle MHF$ vuông tại $H$ hay $MH\perp FH$.\\
			Vậy $D$, $H$, $F$ thẳng hàng.
		\end{enumerate}
	}
\end{bt}


\begin{bt}%[Dự án EX-8-Đề Cương Toán 8]%[Dat Tien Pham]%[8H2N5-3]
	Một người thợ dùng gạch hoa hình vuông để lát nền một căn phòng. Biết viên gạch có cạnh $4$ dm. Để lát theo chiều dài căn phòng người thợ cần $10$ viên gạch, còn chiều rộng thì cần $5$ viên. Hỏi căn phòng đó có diện tích là bao nhiêu mét vuông? Biết diện tích các mạch vữa không đáng kể.
	\loigiai{
		Số viên gạch cần dùng để lát nền căn phòng là $5\cdot10=50$ (viên gạch).\\
		Diện tích một viên gạnh là $4^2=16$ (dm$^2$).\\
		Diện tích căn phòng là $50\cdot 16=800$ (dm$^2$).	
	}
\end{bt}
\begin{bt}%[Dự án EX-8-Đề Cương Toán 8]%[Dat Tien Pham]%[8H2H5-3]
	Một mảnh vườn hình vuông, ở trong người ta đào một cái ao cũng hình vuông cạnh nhỏ hơn cạnh vườn là $20$ m. Tính diện tích cái ao, biết phần diện tích thừa là $600$ m$^2$.
	\loigiai{
		Gọi cạnh của cái ao là $x$ (đơn vị: m, điều kiện: $x>0$).\\
		Khi đó cạnh của mảnh vườn là $x+20$ (m).\\
		Suy ra
		\begin{itemize}
			\item Diện tích cái ao là $x^2$ (m$^2$).
			\item Diện tích mảnh vườn là $(x+20)^2$ (m$^2$).
		\end{itemize}
		Do diện tích phần đất thừa là $600$ m$^2$ nên ta có
		\begin{eqnarray*}
			&&(x+20)^2-x^2=600\\
			&&40x+400=600\\
			&&40x=200\\
			&&x=5.
		\end{eqnarray*}
		Do đó diện tích cái ao là $5^2=25$ (m$^2$).
	}
\end{bt}
\begin{bt}%[Dự án EX-8-Đề Cương Toán 8]%[Dat Tien Pham]%[8H2V5-3]
	\immini{
		Một mặt của bánh chưng có dạng hình vuông $ABCD$ được cắt theo bốn đường thẳng $AC$, $BD$, $MP$, $NQ$, trong đó $M$, $N$, $P$, $Q$ lần lượt là trung điểm các cạnh $AB$, $BC$, $CD$, $AD$. Vì sao bốn đường cắt này đồng quy?}
	{
		\begin{tikzpicture}[scale=1, font=\footnotesize, line join=round, line cap=round, >=stealth]
			\path (0,0) coordinate (A)--+(3,0) coordinate (B)
			($ (B)!1!90:(A)$) coordinate (C)
			($(A)!1!-90:(B)$) coordinate (D)
			($(A)!.5!(B)$)  coordinate (M)
			($(A)!.5!(D)$)  coordinate (Q)
			($(C)!.5!(B)$)  coordinate (N)
			($(C)!.5!(D)$)  coordinate (P)
			;
			%			\fill[gray!10] let \p1=($  (O) -  (A) $) in  (O)  circle ({veclen(\x1,\y1)});
			\fill[gray!30] (A)--(B)--(C)--(D);
			\begin{scope}[overlay]
				\path ($ (A)!0.5!(B) $) coordinate (Nt)
				($ (Nt)!1!90:(A) $) coordinate (Xt)
				($ (B)!0.5!(C) $) coordinate (Mt)
				($ (Mt)!1!90:(B) $) coordinate (Yt)
				(intersection of Nt--Xt and Mt--Yt) coordinate (O);
			\end{scope}
			\draw let \p1=($  (O) -  (A) $) in  (O)  circle ({veclen(\x1,\y1)});
			\draw (B)--(C)--(D)--(A)--cycle (B)--(D) (A)--(C) (M)--(P) (N)--(Q);
			\foreach \t/\g in {A/180,B/0,C/0,D/180,M/90,N/0,P/-90,Q/180}{
				\draw[fill=black] (\t) circle (1pt) node[shift={(\g:7pt)},font=\scriptsize]{$ \t $};
			}		
			%			\node[below] at (current bounding box.south){Hình $3.81$};  
		\end{tikzpicture}
	}
	\loigiai{
		Vì $ABCD$ là hình vuông nên $AC$ và $BD$ cắt nhau tại trung điểm $O$ của mỗi đường.\\
		$\triangle ABC$ có $MO$ là đường trung bình suy ra $MO\parallel BC$.\\
		$\triangle BCD$ có $PO$ là đường trung bình suy ra $PO\parallel BC$.\\
		Suy ra $M$, $O$, $P$ thẳng hàng hay $MP$ qua $O$.\\
		Tương tự $NQ$ qua $O$.\\
		Vậy bốn dường thẳng $AC$, $BD$, $MP$, $NQ$ đồng quy tại $O$.
	}
\end{bt}
\begin{bt}%[Dự án EX-8-Đề Cương Toán 8]%[Dat Tien Pham]%[8H2V5-3]
	\immini{Một khu đất Hình chữ nhật $ABCD$ được chia thành bốn hình chữ nhật nhỏ như hình bên. Biết diện tích ba hình chữ nhật nhỏ lần lượt là $10$cm$^2$, $15$cm$^2$, $6 $cm$^2$. Tính diện tích $x~\left(\mathrm{cm}^2\right)$ của hình chữ nhật nhỏ còn lại.}
	{\begin{tikzpicture}[scale=1,line width=0.55pt, font={\fontsize{12pt}{0pt}}, line join=round, line cap=round, >=stealth]
			%% Khai bao diem		
			\path
			(0,0) coordinate (D)
			(3,0) coordinate (F)
			(3+1.75,0) coordinate (C)
			(0,2) coordinate (G)
			(3,2) coordinate (H)
			(3+1.75,2) coordinate (K)
			(0,3.5) coordinate (A)
			(3,3.5) coordinate (E)
			(3+1.75,3.5) coordinate (B)
			;
			
			\draw (D)--(C)--(K)--(G)--(D) (G)--(A)--(B)--(C) (E)--(F);
			\draw ($(A)!1/2!(H)$) node {$10$};
			\draw ($(G)!1/2!(F)$) node {$15$};
			\draw ($(E)!1/2!(K)$) node {$6$};
			\draw ($(H)!1/2!(C)$) node {$x$};
			
			%% vẽ điểm
			\foreach \x/\g in {A/90,E/90,B/90,K/0,C/-90,F/-90,D/-90,G/180,H/-135}
			\draw[fill=black] (\x) circle (.036)+(\g:.35)
			node{$\x$};	
			
	\end{tikzpicture}} 
	\loigiai{
		\begin{center}
			\begin{tikzpicture}[scale=1,line width=0.55pt, font={\fontsize{12pt}{0pt}}, line join=round, line cap=round, >=stealth]
				%% Khai bao diem		
				\path
				(0,0) coordinate (D)
				(3,0) coordinate (F)
				(3+1.75,0) coordinate (C)
				(0,2) coordinate (G)
				(3,2) coordinate (H)
				(3+1.75,2) coordinate (K)
				(0,3.5) coordinate (A)
				(3,3.5) coordinate (E)
				(3+1.75,3.5) coordinate (B)
				;
				
				\draw (D)--(C)--(K)--(G)--(D) (G)--(A)--(B)--(C) (E)--(F);
				\draw ($(A)!1/2!(H)$) node {$10$};
				\draw ($(G)!1/2!(F)$) node {$15$};
				\draw ($(E)!1/2!(K)$) node {$6$};
				\draw ($(H)!1/2!(C)$) node {$x$};
				\draw ($(A)!1/2!(E)$) node[above] {$a$};
				\draw ($(E)!1/2!(B)$) node[above] {$b$};
				\draw ($(A)!1/2!(G)$) node[left] {$c$};
				\draw ($(G)!1/2!(D)$) node[left] {$d$};
				
				%% vẽ điểm
				\foreach \x/\g in {A/90,E/90,B/90,K/0,C/-90,F/-90,D/-90,G/180,H/-135}
				\draw[fill=black] (\x) circle (.036)+(\g:.35)
				node{$\x$};	
				
			\end{tikzpicture}	
		\end{center}
		Đặt $AE=a$; $EB=b$; $AG=c$; $GD=d$.\\
		Ta có $ac=10$; $bc=6$; $ad=15$\\
		Suy ra $(ac)^2bd=900$ suy ra $bd=9$.\\
		Vậy diện tích hình chữ nhật còn lại là $9$ (cm$^2$).}
\end{bt}


