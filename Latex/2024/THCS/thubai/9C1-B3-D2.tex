\setcounter{section}{2}
\setcounter{subsection}{2}
% Chương 1 - Bài 3 - Giải hệ phương trình bậc nhất hai ẩn (P2)
\section{GIẢI HỆ PHƯƠNG TRÌNH BẬC NHẤT HAI ẨN} % Tên bài
\subsection{GIẢI BÀI TOÁN BẰNG CÁCH LẬP HỆ  PHƯƠNG TRÌNH}
\subsubsection{Phương pháp giải}
\begin{tomtat}
	\textbf{Các bước giải một bài toán bằng cách lập hệ phương trình}\\
	\textbf{Bước 1.} Lập hệ phương trình;
	\begin{itemize}
		\item Chọn ẩn số (thường chọn hai ẩn số) và đặt điều kiện thích hợp cho các ẩn số.
		\item Biểu diễn các đại lượng chưa biết theo ẩn và các đại lượng đã biết.
		\item Lập hệ phương trình biểu thị mối quan hệ giữa các đại lượng.
	\end{itemize}
	\textbf{Bước 2.} Giải hệ phương trình.\\
	\textbf{Bước 3.} Trả lời: Kiểm tra xem trong các nghiệm tìm được của hệ phương trình, nghiệm nào thỏa mãn, nghiệm nào không thỏa mãn điều kiện của ẩn, rồi kết luận.
\end{tomtat}
%%%==============VD1==============%%%
\begin{vd}%[Dự án EX-9-Đề Cương Toán 9]%[Lương Pho]%[9D1H3-4]
	Cho một số có hai chữ số. Nếu đổi chỗ hai chữ số của nó thì được một số lớn hơn số đã cho là $63$. Tổng của số đã cho và số mới tạo thành $99$. Tìm số đó.
	\loigiai{
		Gọi số cần tìm là $\overline{ab}$; $a\in\mathbb{N}^*$; $b\in\mathbb{N}^*$; $a$, $b\leq9$.\\
		Đổi chỗ hai chữ số của nó thì được một số mới là $\overline{ba}$.\\
		Ta có hệ phương trình
		\allowdisplaybreaks
		\begin{eqnarray*}
			&&\heva{
				&\overline{ba}-\overline{ab}=63\\
				&\overline{ba}+\overline{ab}=99}\\
			&&\heva{
				&2\overline{ab}=36\\
				&\overline{ba}+\overline{ab}=99}\\
			&&\heva{
				&\overline{ab}=18\\
				&\overline{ba}=81.}
		\end{eqnarray*}
		Vậy số cần tìm là $18$.
	}
\end{vd}
%%%==============VD1==============%%%
\begin{vd}%[Dự án EX-9-Đề Cương Toán 9]%[Lương Pho]%[9D1V3-4]
	Cân bằng các phương trình hoá học sau bằng phương pháp đại số.
	\begin{multicols}{3}
		\begin{enumerate}
			\item Fe+Cl$_2 \longrightarrow$ FeCl$_3$.
			\item SO$_2$+O$_2 \xrightarrow[t^{\circ}]{V_2 O_5}$ SO$_3$.
			\item Al+O$_2\longrightarrow$ Al$_2$O$_3$.
		\end{enumerate}
	\end{multicols}
	\loigiai{
		\begin{enumerate}
			\item Gọi $x$, $y$ lần lượt là hệ số của Fe và Cl$_2$ thỏa mãn cân bằng phương trình hóa học $x$, $y >0$.
			$$x\mathrm{Fe}+y\mathrm{Cl}_2 \longrightarrow \mathrm{FeCl}_3.$$
			Cân bằng số nguyên tử $Fe$ và số nguyên tử $Cl$ ở hai vế ta có hệ
			\begin{eqnarray*}
				&&\heva{&x=1\\&2y=3}\\
				&&\heva{&x=1\\&y=\dfrac{3}{2}.}
			\end{eqnarray*}
			Do đó $x$, $y$ thỏa điều kiện.\\
			Khi đó ta có $$\mathrm{Fe}+\dfrac{3}{2}\mathrm{Cl}_2 \longrightarrow \mathrm{FeCl}_3.$$
			Do hệ số của phương trình hóa học phải là số nguyên nên nhân hai về phương trình hóa học trên với $2$, ta được
			$$2\mathrm{Fe}+3\mathrm{Cl}_2 \longrightarrow 2\mathrm{FeCl}_3.$$
			\item Gọi $x$, $y$ lần lượt là hệ số của $SO_2$ và $O_2$ thỏa mãn cân bằng phương trình hóa học $x$, $y >0$.
			$$x\mathrm{SO}_2+y\mathrm{O}_2 \underset{V_2 O_5}{\stackrel{t^{\circ}}{\longrightarrow}} \mathrm{SO}_3.$$
			Cân bằng số nguyên tử $S$ và số nguyên tử $O$ ở hai vế ta có hệ
			\begin{eqnarray*}
				&&\heva{&x=1\\&2x+2y=3}\\
				&&\heva{&x=1\\&y=\dfrac{1}{2}.}
			\end{eqnarray*}
			Do đó $x$, $y$ thỏa điều kiện.\\
			Khi đó ta có $$\mathrm{SO}_2+\dfrac{1}{2}\mathrm{O}_2 \underset{V_2 O_5}{\stackrel{t^{\circ}}{\longrightarrow}} \mathrm{SO}_3.$$
			Do hệ số của phương trình hóa học phải là số nguyên nên nhân hai về phương trình hóa học trên với $2$, ta được
			$$2\mathrm{SO}_2+\mathrm{O}_2 \underset{V_2 O_5}{\stackrel{t^{\circ}}{\longrightarrow}} 2\mathrm{SO}_3.$$
			\item Gọi $x$, $y$ lần lượt là hệ số của $Al$ và $O_2$ thỏa mãn cân bằng phương trình hóa học $x$, $y >0$.
			$$x\mathrm{Al}+y\mathrm{O}_2 \rightarrow \mathrm{Al}_2\mathrm{O}_3.$$
			Cân bằng số nguyên tử $Al$ và số nguyên tử $O$ ở hai vế ta có hệ\\
			\begin{eqnarray*}
				&&\heva{&x=2\\&2y=3}\\
				&&\heva{&x=2\\&y=\dfrac{3}{2}.}
			\end{eqnarray*}
			Do đó $x$, $y$ thỏa điều kiện.\\
			Khi đó ta có \[2\mathrm{Al}+\dfrac{3}{2}\mathrm{O}_2 \rightarrow \mathrm{Al}_2\mathrm{O}_3.\]
			Do hệ số của phương trình hóa học phải là số nguyên nên nhân hai về phương trình hóa học trên với $2$, ta được
			\[4\mathrm{Al}+3\mathrm{O}_2 \rightarrow 2\mathrm{Al}_2\mathrm{O}_3.\]
		\end{enumerate}
	}
\end{vd}

\subsubsection{Bài tập vận dụng}
\begin{center}
	\textbf{Phần 1. CÂU HỎI TỰ LUẬN}
\end{center}
%%%=============BT_1=============%%%
\begin{bt}%[Dự án EX-9-Đề Cương Toán 9]%[Lương Pho]%[9D1H3-4]
	Nhân kỉ niệm ngày Quốc khánh $2/9$, một nhà sách giảm giá mỗi cây bút bi là $20\%$ và mỗi quyển vở là $10\%$ so với giá niêm yết. Bạn Thanh vào nhà sách mua $20$ quyển vở và $10$ cây bút bi. Khi tính tiền, bạn Thanh đưa $175\,000$ đồng và được trả lại $3\,000$ đồng. Tính giá niêm yết của mỗi quyển vở và mỗi cây bút bi, biết rằng tổng số tiền phải trả nếu không được giảm giá là $195\,000$ đồng.
	\loigiai{
		Gọi $x$, $y$ lần lượt là giá tiền niêm yết của mỗi quyển vở và mỗi cây bút bi với $x$, $y >0$.\\
		Số tiền mua $20$ quyển vở và $10$ cây bút bi khi giảm giá là 
		\[20\cdot 90\%x+10\cdot 80\%y=175\,000-3000\,\,\text{hay}\,\,9x+4y=86\,000.\qquad(1)\]
		Tổng số tiền phải trả nếu không được giảm giá là $195\,000$ đồng, nên ta có phương trình \[20x+10y=195\,000\,\,\text{hay}\,\,8x+4y=78\,000.\qquad(2)\]
		Từ $(1)$ và $(2)$ ta có  hệ phương trình
		\[\heva{&9x+4y=86\,000\\&8x+4y=78\,000.}\]
		Giải hệ ta được $x=8\,000$, $y=3\,500$. (thỏa điều kiện)\\
		Vậy giá niêm yết của mỗi quyển vở là $8\,000$ đồng, mỗi cây bút bi và $3\,500$ đồng.
	}
\end{bt}
%%%=============================%%%

%%%=============BT_2=============%%%
\begin{bt}%[Dự án EX-9-Đề Cương Toán 9]%[Lương Pho]%[9D1H3-4]
	Ông Bình có $500$ triệu đồng, ông đã dùng một phần của số tiền này để gửi ngân hàng lãi suất $7\%$ một năm. Phần còn lại, ông đầu tư vào nhà hàng của một người bạn thân để nhận lãi kinh doanh. Sau một năm, ông Bình thu về số tiền cả vốn và lãi từ cả hai nguồn trên là $574$ triệu đồng. Biết rằng, tiền lãi kinh doanh nhà hàng bằng $20\%$ số tiền đầu tư. Hỏi ông Bình đã sử dụng bao nhiêu tiền cho mỗi hình thức đầu tư?
	%\par\dapso{Ông bình đã đầu tư $200$ triệu đồng gửi ngân hàng và đầu tư nhà hàng là $300$ triệu đồng.}
	\loigiai{
		Gọi số tiền Ông Bình dùng để gửi ngân hàng là $x$ (triệu đồng),
		$(0< x < 500)$.\\
		Gọi số tiền Ông Bình dùng để đầu tư vào nhà hàng là $y$ (triệu đồng),
		$(0< y < 500)$.\\
		Số tiền lãi Ông Bình nhận được sau một năm từ gửi ngân hàng là $0{,}07x$
		(triệu đồng).\\
		Số tiền lãi Ông Bình nhận được sau một năm từ đầu tư vào nhà hàng là
		$0{,}2y$ (triệu đồng).\\
		Ông Bình có $500$ triệu đồng nên $x+y=500$.\qquad$(1)$\\
		Ông Bình thu về số tiền cả vốn và lãi từ cả hai nguồn trên là $574$ triệu đồng nên
		\[0{,}07x+0{,}2y=74.\qquad(2)\]
		Từ $(1)$ và $(2)$ ta có  hệ phương trình
		\[\heva{&x+y=500\\&0{,}07x+0{,}2y=74.}\]
		Giải hệ phương trình ta được $x=200$, $y=300$ (thỏa điều kiện).\\
		Vậy Ông Bình đã đầu tư $200$ triệu đồng gửi ngân hàng và đầu tư nhà hàng là $300$ triệu đồng.
	}
\end{bt}
%%%=============================%%%

%%%=============BT_3=============%%%
\begin{bt}%[Dự án EX-9-Đề Cương Toán 9]%[Lương Pho]%[9D1H3-4]
	Khi thực hiện xây dựng trường điển hình đổi mới $2017$, hai trường trung học cơ sở A và B có tất cả $760$ học sinh đăng ký tham gia nội dung hoạt động trải nghiệm. Đến khi tổng kết, số học sinh tham gia đạt tỷ lệ $85\%$ so với số đã đăng ký. Nếu tính riêng thì tỷ lệ học sinh tham gia của trường A và B lần lượt là $80\%$ và $89{,}5\%$. Tính số học sinh ban đầu đăng ký tham gia của mỗi trường.
	%\par\dapso{Trường A: $360$ học sinh, Trường B: $400$ học sinh}
	\loigiai{
		Gọi $x$ là số học sinh ban đầu đăng ký của trường A.\\
		Gọi $y$ là số học sinh ban đầu đăng ký của trường B.\\
		Điều kiện $x$, $y\in \mathbb{N^*}$.\\
		Hai trường trung học cơ sở A và B có tất cả $760$ học sinh đăng ký tham gia nội dung hoạt động trải nghiệm
		\[x+y=760.\qquad(1)\]
		Nếu tính riêng thì tỷ lệ học sinh tham gia của trường A và B lần lượt là $80\%$ và $89{,}5\%$ đồng thời số học sinh tham gia đạt tỷ lệ $85\%$ so với số đã đăng ký, nên
		\[80\%x+89{,}5\%y=85\%\cdot 760.\qquad(2)\]
		Từ $(1)$ và $(2)$ ta có  hệ phương trình
		\[\heva{&x+y=760 \\ &80\%x+89{,}5\%y=85\%\cdot 760.}\]
		Giải hệ ta được $x=360$, $y=400$.(thỏa mãn)\\
		Vậy trường A có $360$ học sinh đăng ký, trường B có $400$ học sinh đăng ký.
	}
\end{bt}
%%%=============================%%%

%%%=============BT_4=============%%%
\begin{bt}%[Dự án EX-9-Đề Cương Toán 9]%[Lương Pho]%[9D1H3-4]
	Phát động phong trào quyên góp sách dành tặng cho thư viện của Thành Phố. Trường A cho mỗi học sinh góp $2$ quyển và trường B cho mỗi học sinh góp $3$ quyển. Tổng số sách mà hai trường A và B quyên góp được chiếm $25\%$ tổng số sách thành phố quyên góp; Số lượng học sinh của trường A nhiều hơn số lượng học sinh trường B là $25$ học sinh. Hỏi mỗi trường A và trường B đã góp được bao nhiêu quyển sách, biết rằng toàn Thành phố đã nhận được $8\,000$ quyển sách quyên góp.
	%\par\dapso{Trường A $830$; Trường B $1\,170$}
	\loigiai{
		Gọi $x$ là số học sinh của trường A; $y$ là số học sinh của trường B.\\
		Điều kiện $x$, $y\in \mathbb{N^*}$.\\
		Ta có $25\%\cdot 8\,000=2\,000$.\\
		Số lượng học sinh của trường A nhiều hơn số lượng học sinh trường B là $25$ học sinh
		\[x-y=25.\qquad(1)\]
		Tổng số sách mà hai trường A và B quyên góp được chiếm $25\%$ tổng số sách thành phố quyên góp
		\[2x+3y=2\,000.\qquad(2)\]
		Từ $(1)$ và $(2)$ ta có  hệ phương trình
		\[\heva{&x-y=25\\&2x+3y=2\,000.}\]
		Giải hệ ta được $x=415$, $y=390$.\\
		Vậy số sách quyên góp được của mỗi trường là
		\begin{itemize}
			\item Trường A: $2\cdot 415=830$\,(quyển).
			\item Trường B: $3\cdot 390=1\,170$\,(quyển).
		\end{itemize}
	}
\end{bt}
%%%=============================%%%

%%%=============BT_5=============%%%
\begin{bt}%[Dự án EX-9-Đề Cương Toán 9]%[Lương Pho]%[9D1H3-4]
	Trong kỳ thi tuyển sinh vào lớp $10$ năm học $2019-2020$, số thí sinh vào trường THPT chuyên bằng $\dfrac{2}{3}$ số thí sinh thi vào trường PTDT Nội trú. Biết rằng tổng số phòng thi của cả hai trường là $80$ phòng thi và mỗi phòng thi có đúng $24$ thí sinh. Hỏi số thí sinh vào mỗi trường bằng bao nhiêu?
	%\par\dapso{Vậy số thí sinh vào trường THPT Chuyên và số thí sinh vào trường PTDT Nội trú lần lượt là $768$ thí sinh, $1\,152$ thí sinh.}
	\loigiai{
		Gọi số thí sinh vào trường THPT Chuyên và số thí sinh vào trường PTDT Nội trú lần lượt là $x$, $y$ (thí sinh) (điều kiện $x > 0$, $y > 0$).\\
		Vì số thí sinh vào trường THPT Chuyên bằng $\dfrac{2}{3}$ số thí sinh vào trường PTDT Nội trú nên ta có 
		\[x=\dfrac{2}{3}y \text{ hay }x-\dfrac{2}{3}y=0.\qquad(1)\]
		Vì tổng số phòng thi của cả hai trường là $80$ phòng thi và mỗi phòng thi có đúng $24$ thí sinh nên tổng số thí sinh của cả hai trường là
		$24\cdot 80=1\,920$ (thí sinh).\\
		Do đó ta có phương trình $x+y=1\,920$. \qquad$(2)$ \\
		Từ $(1)$ và $(2)$ ta có hệ phương trình
		\[\heva{&x-\dfrac{2}{3}y=0\\&x+y=1\,920.}\]
		Giải hệ ta được $x=768$; $y=1152$ (thỏa mãn).\\
		Vậy số thí sinh vào trường THPT Chuyên và số thí sinh vào trường PTDT Nội trú lần lượt là $768$ thí sinh, $1\,152$ thí sinh.
	}
\end{bt}
%%%=============================%%%

%%%=============BT_6=============%%%
\begin{bt}%[Dự án EX-9-Đề Cương Toán 9]%[Lương Pho]%[9D1H3-4]%[9D1V2-3]
	Trong Kỳ thi tuyển sinh vào lớp $10$ THPT, hai lớp $9$A và $9$B có tổng cộng $75$ học sinh dự thi. Biết rằng, lớp $9$A có $80\%$ học sinh trúng tuyển so với số học sinh dự thi của lớp, lớp $9$B có $90\%$ học sinh trúng tuyển so với số học sinh dự thi của lớp. Tổng số học sinh trúng tuyển của hai lớp $9$A và $9$B là $64$. Tính số học sinh dự thi của lớp $9$A, lớp $9$B.
	%\par\dapso{Vậy số học sinh dự thi vào lớp 10 của lớp $9$A là $35$ học sinh, của lớp $9$B là $40$ học sinh.}
	\loigiai{
		Gọi số học sinh dự thi vào lớp 10 của lớp $9$A và $9$B lần lượt là $x$ và $y$ ($x$, $y$ nguyên dương). \\
		Vì hai lớp $9$A và $9$B có tổng cộng $75$ học sinh dự thi nên $x+y=75$.\qquad$(1)$ \\
		Vì lớp $9$A có tỉ lệ trúng tuyển vào lớp 10 là $80\%$, lớp $9$B có tỉ lệ trúng tuyển vào lớp 10 là $90\%$ nên \[80\% x+90\% y=64.\qquad(2)\]
		Từ $(1)$ và $(2)$ ta có hệ phương trình $\heva{&x+y=75\\&80\% x+90\% y=64.}$ \\
		Giải hệ phương trình, ta được $\heva{&x=35\\&y=40.}$ \\
		Vậy số học sinh dự thi vào lớp $10$ của lớp $9$A là $35$ học sinh, của lớp $9$B là $40$ học sinh.
	}
\end{bt}
%%%=============================%%%

%%%=============BT_7=============%%%
\begin{bt}%[Dự án EX-9-Đề Cương Toán 9]%[Lương Pho]%[9D1H3-4]%[9D1V2-3]
	Một trường trung học cơ sở mua $500$ quyển vở bao gồm quyển vở loại thứ nhất và quyển vở loại thứ hai để làm phần thưởng cho học sinh. Giá bán của mỗi quyển vở loại thứ nhất, loại thứ hai lần lượt là $8\,000$ đồng và $9\,000$ đồng. Biết tổng số tiền nhà trường đã dùng để mua $500$ quyển vở đó là $4\,200\,000$ đồng. Mỗi học sinh xuất sắc được thưởng $2$ quyển vở loại thứ nhất và $1$ quyển vở loại thứ hai; mỗi học sinh giỏi được thưởng $1$ quyển vở loại thứ nhất và $1$ quyển vở loại thứ hai. Hỏi trường có bao nhiêu học sinh xuất sắc, bao nhiêu học sinh giỏi?
	%\par\dapso{Vậy trường có $100$ hs xuất sắc và $100$ học sinh giỏi.}
	\loigiai{
		Gọi $x$, $y$ lần lượt là số quyển vở loại thứ nhất, loại thứ hai.\\
		Điều kiện $x$, $y\in \mathbb{N}^*$.\\
		Tổng số quyển vở là $500$ nên có $x+y=500$. \qquad$(1)$\\
		Tổng số tiền nhà trường đã dùng để mua $500$ quyển vở đó là $4\,200\,000$ đồng
		\[8\,000x+9\,000y=4\,200\,000.\qquad(2)\]
		Từ $(1)$ và $(2)$ ta có hệ phương trình $\heva{& x+y=500\\&8\,000x+9\,000y=4\,200\,000.}$\\
		Giải hệ ta được $\heva{&x=300\\&y=200.}$\\
		Do đó số quyển vở loại thứ nhất có $300$ quyển; quyển vở loại thứ hai có $200$ quyển.\\
		Gọi $a$, $b$ lần lượt là số học sinh xuất sắc và học sinh giỏi ($a$, $b\in\mathbb{N^*}$).\\
		Mỗi học sinh xuất sắc được thưởng $2$ quyển vở loại thứ nhất và $1$ quyển vở loại thứ hai
		\[2a+b=300.\qquad(3)\]
		Mỗi học sinh giỏi được thưởng $1$ quyển vở loại thứ nhất và $1$ quyển vở loại thứ hai
		\[a+b=200.\qquad(4)\]
		Từ $(3)$ và $(4)$ ta có hệ phương trình $\heva{&2a+b=300\\&a+b=200.}$\\
		Giải hệ ta được $\heva{&a=100\\&b=100.}$ (nhận)\\
		Vậy trường có $100$ học sinh xuất sắc và $100$ học sinh giỏi.
	}
\end{bt}
%%%=============================%%%

%%%=============BT_8=============%%%
\begin{bt}%[Dự án EX-9-Đề Cương Toán 9]%[Lương Pho]%[9D1H3-4]%[9D1H3-4]
	Trong kỳ thi môn toán lớp $9$, một phòng thi của trường có $24$ thí sinh dự thi. Các thí sinh đều phải làm bài trên giấy thi của trường phát cho. Cuối buổi thi, sau khi thu bài, giám thị coi thi đếm được tổng số tờ là $42$ tờ giấy thi. Hỏi trong phòng thi đó có bao nhiêu thí sinh làm $1$ tờ giấy thi, bao nhiêu thí sinh làm $2$ tờ giấy thi? Biết rằng có đúng ba thí sinh làm $3$ tờ giấy thi và không ai làm hơn $3$ tờ.
	%\par\dapso{Vậy trong phòng thi đó có $9$ thí sinh làm $1$ tờ giấy thi, $12$ thí sinh làm $2$ tờ giấy thi.}
	\loigiai{
		Gọi số thí sinh làm $1$ tờ giấy thi là $x$ (thí sinh) $(x\in \mathbb{N}^*$, $x<24)$.\\
		Gọi số thí sinh làm $2$ tờ giấy thi là $y$ (thí sinh) $(y\in \mathbb{N}^*$, $y<24)$.\\
		Vì tổng số thí sinh của phòng thi là $24$ nên ta có 
		\[x+y+3=24 \text{ hay }x+y=21.\qquad(1)\]
		Lại có tổng số tờ giấy thi là $42$ nên ta có 
		\[x+2y+3\cdot 3=42 \text{ hay }x+2y=33.\qquad(2)\]
		Từ $(1)$ và $(2)$ ta có hệ phương trình\[\heva{&x+y=21\\&x+2y=33.}\]
		Giải hệ ta được $\heva{&x=9\\&y=12}$ (thỏa mãn).\\
		Vậy trong phòng thi đó có $9$ thí sinh làm $1$ tờ giấy thi, $12$ thí sinh làm $2$ tờ giấy thi.
	}
\end{bt}
%%%=============================%%%

%%%=============BT_9=============%%%
\begin{bt}%[Dự án EX-9-Đề Cương Toán 9]%[Lương Pho]%[9D1H3-4]
	Vào tháng $01/2020$, mỗi ngày mỗi bạn ở lớp $9$A tiết kiệm được $2$ ngàn đồng, trong khi đó mỗi bạn ở lớp $9$B tiết kiệm được $3$ ngàn đồng. Tổng số tiền tiết kiệm được của hai lớp trong tháng $01/2020$ là $4\,774$ ngàn đồng. Qua tháng $02/2020$ mỗi ngày mỗi bạn ở lớp $9$A tiết kiệm được $5$ ngàn đồng, trong khi đó mỗi ngày mỗi bạn ở lớp $9$B tiết kiệm được $4$ ngàn đồng. Cả tháng $02/2020$ lớp $9$A tiết kiệm được nhiều hơn lớp $9$B là $1\,160$ ngàn đồng. Tính số học sinh ở lớp $9$A, $9$B (số học sinh mỗi lớp không đổi).
	%\par\dapso{Vậy lớp $9$A có $32$ học sinh và lớp $9$B có $30$ học sinh.}
	\loigiai{
		Gọi $x$, $y$ (học sinh) lần lượt là số học sinh của lớp $9$A, $9$B ($x$, $y\in\mathbb{N}^*$). \\
		Tháng $01/2020$ có $31$ ngày, mỗi bạn lớp $9$A tiết kiếm được $2$ ngàn đồng, mỗi bạn lớp $9$B tiết kiệm được $3$ ngàn đồng và tổng số tiền tiết kiệm là $4\,774$ ngàn đồng nên ta có phương trình \[2\cdot 31x+3\cdot 31y=4\,774 \text{ hay } 62x+93y=4\,774.\qquad (1)\]
		Ta có $2020\,\vdots\, 4$ và $2020\not\vdots\,\, 100$ nên $2020$ là năm nhuận, do đó tháng $02/2020$ có $29$ ngày, mỗi bạn lớp $9$A tiết kiếm được $5$ ngàn đồng, mỗi bạn lớp $9$B tiết kiệm được $4$ ngàn đồng. Hơn nữa, lớp $9$A tiết kiệm được nhiều hơn lớp $9$B là $1\,160$ ngàn đồng nên ta có phương trình
		$$ 29\cdot 5x-29\cdot 4y=1\,160 \text{ hay } 145x-116y=1\,160. \qquad (2) $$
		Từ $(1)$ và $(2)$, ta có hệ phương trình
		$$\heva{&62x+93y=4\,774\\&145x-116y=1\,160.}$$
		Giải hệ phương trình trên, ta được $\heva{&x=32\\&y=30} \quad \text{(thoả mãn điều kiện).}$\\
		Vậy lớp $9$A có $32$ học sinh và lớp $9$B có $30$ học sinh.
	}
\end{bt}
%%%=============================%%%

%%%=============BT_10=============%%%
\begin{bt}%[Dự án EX-9-Đề Cương Toán 9]%[Lương Pho]%[9D1V2-3]
	Điểm trung bình của $100$ học sinh trong hai lớp $9$A và $9$B là $7{,}2$. Tính điểm trung bình của các học sinh mỗi lớp, biết rằng số học sinh lớp $9$A gấp rưỡi số học sinh lớp $9$B và điểm trung bình của lớp $9$B gấp rưỡi điểm trung bình lớp $9$A.
	%\par\dapso{Vậy điểm trung bình của học sinh lớp $9$A là $6$ (điểm) và điểm trung bình của học sinh lớp $9$B là $9$ (điểm).}
	\loigiai{
		Gọi $a$, $b$ lần lượt là số học sinh lớp $9$A và lớp $9$B ($a$, $b\in\mathbb{N}^*$).\\
		Tổng số học sinh hai lớp bằng $100$ nên
		\[a+b=100.\qquad(1)\]
		Số học sinh lớp $9$A gấp rưỡi số học sinh lớp $9$B
		\[a=1{,}5b \text{ hay }a-1{,}5b=0.\qquad(2)\]
		Từ $(1)$ và $(2)$ ta có hệ phương trình $\heva{&a+b=100\\&a-1{,}5b=0.}$\\
		Giải hệ ta được $\heva{&a=60\\&b=40.}$ (thỏa mãn)\\
		Do đó, số học sinh lớp $9$A là $60$ học sinh; Số học sinh lớp $9$B là $40$ học sinh.\\
		Gọi $x$ là điểm trung bình của học sinh lớp $9$A;\\
		$y$ là điểm trung bình của học sinh lớp $9$B.\\
		Điều kiện $x$, $y>0$.\\
		Vì điểm trung bình của học sinh lớp $9$B gấp rưỡi điểm trung bình lớp $9$A nên ta có 
		\[y=1{,}5x\text{ hay }1{,}5x-y=0.\qquad(3)\]
		Vì điểm trung bình của học sinh cả hai lớp là $7{,}2$ nên ta có 
		\[\dfrac{60x+40y}{100}=7{,}2\text{ hay }60x+40y=720.\qquad(4)\]
		Từ $(3)$ và $(4)$ ta có hệ phương trình$\heva{&1{,}5x-y=0\\&60x+40y=720.}$\\
		Giải hệ phương trình ta được $x=6$ và $y=9$. (nhận)\\
		Vậy điểm trung bình của học sinh lớp $9$A là $6$ điểm và điểm trung bình của học sinh lớp $9$B là $9$ điểm.
	}
\end{bt}
%%%=============================%%%

%%%=============BT_11=============%%%
\begin{bt}%[Dự án EX-9-Đề Cương Toán 9]%[Lương Pho]%[9D1V3-4]
	Tìm hai số nguyên dương biết tổng của chúng bằng $1\,006$, nếu lấy số lớn chia cho số bé được thương là $2$ và số dư là $124$.
	\loigiai{
		Gọi $x$, $y$ lần lượt là số lớn và số bé với $x$, $y \in \mathbb{N}^*$.\\
		Tổng của chúng bằng $1\,006$, nên ta có phương trình 
		\[x+y=1\,006.\qquad(1)\]
		Số lớn chia cho số bé được thương là $2$ và số dư là $124$, nên ta có phương trình \[x=2y+124\text{ hay }x-2y=124.\qquad(2)\]
		Từ $(1)$ và $(2)$ ta có hệ phương trình \[\heva{&x+y=1\,006\\&x-2y=124.}\]
		Giải hệ a được $x=712$, $y=294$. (thỏa điều kiện)\\
		Vậy hai số đó là $712$ và $294$.
	}
\end{bt}
%%%=============================%%%

%%%=============BT_12=============%%%
\begin{bt}%[Dự án EX-9-Đề Cương Toán 9]%[Lương Pho]%[9D1V3-4]
	Trong một xí nghiệp, hai tổ công nhân A và B lắp ráp cùng một loại bộ linh kiện điện tử. Nếu tổ A lắp ráp trong $5$ ngày, tổ B lắp ráp trong $4$ ngày thì xong $1\,900$ bộ linh kiện. Biết rằng mỗi ngày tổ A lắp ráp nhiều hơn tổ B là $20$ bộ linh kiện. Hỏi trong một ngày mỗi tổ ráp được bao nhiêu bộ linh kiện điện tử? (Năng suất lắp ráp của mỗi tổ trong các ngày là như nhau).
	\loigiai{
		Gọi $x$, $y$ lần lượt là bộ linh kiện điện tử của tổ A và B trong một ngày với $x$, $y \in \mathbb{N}^*$.\\
		Mỗi ngày tổ A lắp ráp nhiều hơn tổ B là $20$ bộ linh kiện, nên ta có phương trình\[x-y=20.\qquad(1)\]
		Tổ A lắp ráp trong $5$ ngày, tổ B lắp ráp trong $4$ ngày thì xong $1\,900$ bộ linh kiện, nên ta có phương trình \[5x+4y=1\,900.\qquad(2)\]
		Từ $(1)$ và $(2)$ ta có hệ phương trình \[\heva{&x-y=20\\&5x+4y=1\,900.}\]
		Giải hệ a được $x=220$, $y=200$. (thỏa điều kiện)\\
		Vậy mỗi ngày tổ A ráp được $220$ bộ linh kiện điện tử, tổ B ráp được $200$ bộ linh kiện điện tử.
	}
\end{bt}
%%%=============================%%%

%%%=============BT_13=============%%%
\begin{bt}%[Dự án EX-9-Đề Cương Toán 9]%[Lương Pho]%[9D1V3-4]
	Nhà máy luyện thép hiện có sẵn loại thép chứa $10\%$ carbon và loại thép chứa $20\%$ carbon. Giả sử trong quá trình luyện thép các nguyên liệu không bị hao hụt. Tính khối lượng thép mỗi loại cần dùng để luyện được $1\,000$ tấn thép chứa $16\%$ carbon từ hai loại thép trên.
	\loigiai{
		Gọi khối lượng thép chứa $10\%$ carbon là $x$ tấn và khối lượng thép chứa $20\%$ carbon là $y$ tấn với $x$, $y >0$.\\
		Khối lượng carbon có trong $x$ tấn thép $10\%$ carbon là $10\% x=0{,}1x$.\\
		Khối lượng carbon có trong $y$ tấn thép $20\%$ carbon là $20\% x=0{,}2y$.\\
		$1\,000$ tấn thép chứa $16\%$ carbon, nên ta có phương trình \[0{,}1x+0{,}2y=1\,000\cdot 16\%.\qquad(1)\]
		Khối lượng thép mỗi loại cần dùng để luyện được $1\,000$ tấn, nên ta có phương trình 
		\[x+y=1\,000.\qquad(2)\]
		Từ $(1)$ và $(2)$ ta có hệ phương trình \[\heva{&0{,}1x+0{,}2y=1\,000\cdot 16 \%\\&x+y=1\,000.}\]
		Giải hệ a được $x=400$, $y=600$. (thỏa điều kiện)\\
		Vậy cần $400$ tấn thép loại $10\%$ carbon và $600$ tấn thép loại $20\%$ carbon.
	}
\end{bt}
%%%=============================%%%

%%%=============BT_14=============%%%
\begin{bt}%[Dự án EX-9-Đề Cương Toán 9]%[Lương Pho]%[9D1V3-4]
	Trong tháng $11$, hai cửa hàng của một thương hiệu thời trang bán được $1\,100$ sản phẩm. Sang tháng $12$, cửa hàng thứ nhất bán vượt mức $15\%$, cửa hàng thứ hai bán vượt mức $20\%$ so với tháng $11$, do đó tháng $12$ hai cửa hàng bán bán được $1\,295$ sản phẩm. Hỏi trong tháng $12$ mỗi cửa hàng bán được bao nhiêu sản phẩm?
	%\par\dapso{$720$ sản phẩm.}
	\loigiai{
		Gọi $x$, $y$ lần lượt là số sản phẩm của cửa hàng $1$ và cửa hàng $2$ bán được trong tháng $11$ (điều kiện $x$, $y$ nguyên dương và nhỏ hơn $1\,100$).\\
		Số sản phẩm cửa hàng $1$ bán được trong tháng $12$ là $115\%\cdot x=1{,}15x$ sản phẩm.\\
		Số sản phẩm cửa hàng $2$ bán được trong tháng $12$ là $120\%\cdot x=1{,}2x$ sản phẩm.\\
		Vì trong tháng $11$ cả hai cửa hàng bán $1\,100$ sản phẩm nên ta có phương trình 
		\[x+y=1\,100.\qquad(1)\]
		Vì trong tháng $12$ cả hai cửa hàng bán $1\,295$ sản phẩm nên ta có phương trình \[1{,}15x+1{,}2y=1\,295.\qquad(2)\]
		Từ $(1)$ và $(2)$ ta có hệ phương trình \[\heva{&x+y=1\,100\\&1{,}15x+1{,}2y=1\,295.}\]
		Giải hệ phương trình trên ta được $x=500$, $y=600$ (thỏa mãn điều kiện).\\
		Vậy số sản phẩm cửa hàng $1$ bán được trong tháng $12$ là $1{,}15\cdot 500=575$ sản phẩm.\\
		Số sản phẩm cửa hàng $2$ bán được trong tháng $12$ là $1{,}2\cdot 600=720$ sản phẩm.
	}
\end{bt}
%%%=============================%%%

%%%=============BT_15=============%%%
\begin{bt}%[Dự án EX-9-Đề Cương Toán 9]%[Lương Pho]%[9D4V2-5]
	Có ba thùng dầu đựng tổng cộng $123$ lít. Nếu đổ thùng thứ nhất sang thùng thứ hai $5$ lít, rồi thùng thứ hai sang thùng thứ ba $7$ lít, tiếp tục đổ từ thùng thứ ba sang thùng thứ nhất $9$ lít thì số dầu ở thùng thứ nhất sẽ ít hơn thùng thứ hai là $4$ lít và bằng $\dfrac{2}{3}$ số dầu ở thùng thứ ba. Tính số lít dầu ở mỗi thùng lúc đầu.
	%\par\dapso{Vậy lúc đầu, thùng thứ nhất chứa $30$ lít dầu, thùng thứ hai chứa $40$ lít dầu, thùng thứ ba chứa $123-30-40=53$ lít dầu.}
	\loigiai{
		Gọi $x$ (lít), $y$ (lít) lần lượt là số lít dầu ở thùng thứ nhất, thứ hai lúc đầu.\\
		Điều kiện $x$, $y>0$; $x<123$; $y<123$.\\
		Do đó, lít dầu ở thùng thứ ba lúc đầu là $123-x-y$ (lít).\\
		Nếu đổ thùng thứ nhất sang thùng thứ hai $5$ lít, rồi thùng thứ hai sang thùng thứ ba $7$ lít, tiếp tục đổ từ thùng thứ ba sang thùng thứ nhất $9$ lít. Số lít dầu ở
		\begin{itemize}
			\item Thùng thứ nhất là $x-5+9=x+4$ (lít).
			\item Thùng thứ hai là $y+5-7=y-2$ (lít).
			\item Thùng thứ ba là $123-x-y+7-9=121-x-y$ (lít).
		\end{itemize}
		Thùng thứ nhất ít hơn thùng thứ hai $4$ lít, nên
		\[(x+4)-(y-2)=-4\text{ hay } x-y=-10.\qquad (1)\]
		Thùng thứ nhất bằng $\dfrac{2}{3}$ số dầu ở thùng thứ ba, nên
		\[x+4=\dfrac{2}{3}(121-x-y) \text{ hay } 5x+2y=230.\qquad (2)\]
		Từ $(1)$ và $(2)$ ta có hệ phương trình\[\heva{&x-y=-10\\&5x+2y=230.}\]
		Giải hệ ta được $x=30$, $y=40$. (thỏa mãn điều kiện)\\
		Vậy lúc đầu
		\begin{itemize}
			\item thùng thứ nhất chứa $30$ lít dầu;
			\item thùng thứ hai chứa $40$ lít dầu
			\item thùng thứ ba chứa $123-30-40=53$ lít dầu.
		\end{itemize}
	}
\end{bt}
%%%=============================%%%

\begin{center}
	\textbf{Phần 2. CÂU HỎI TRẮC NGHIỆM}
\end{center}
%\cauMC
\Opensolutionfile{ans}[ans/ans-9C1-B3-D2]
%%%=============EX_1=============%%%
\begin{ex}%[Dự án EX-9-Đề Cương Toán 9]%[Lương Pho]%[9D1V3-4]
	Cho một số có hai chữ số. Nếu đổi chỗ hai chữ số của nó thì được một số lớn hơn số đã cho là $18$. Tổng của số đã cho và số mới tạo thành $66$. Tổng các chữ số của số đó là
	\choice
	{$9$}
	{$8$}
	{$7$}
	{\True $6$}
	\loigiai{
		Gọi số cần tìm là $\overline{ab}$; $a\in\mathbb{N}^*$; $b\in\mathbb{N}^*$; $a$, $b\leq9$.\\
		Đổi chỗ hai chữ số của nó thì được một số mới là $\overline{ba}$.\\
		Ta có hệ phương trình
		\allowdisplaybreaks
		\begin{eqnarray*}
			&&\heva{
				&\overline{ba}-\overline{ab}=18\\
				&\overline{ba}+\overline{ab}=66} \\
			&&\heva{
				&2\overline{ab}=48\\
				&\overline{ba}+\overline{ab}=66} \\
			&&\heva{
				&\overline{ab}=24\\
				&\overline{ba}=42.}
		\end{eqnarray*}
		Vậy số cần tìm là $24$ nên tổng các chữ số là $2+4=6$.
	}
\end{ex}
%%%=============================%%%

%%%=============EX_2=============%%%
\begin{ex}%[Dự án EX-9-Đề Cương Toán 9]%[Lương Pho]%[9D1V3-4]
	Cho một số có hai chữ số. Chữ số hàng chục lớn hơn chữ số hàng đơn vị là $5$. Nếu đổi chỗ hai chữ số cho nhau ta được một số bằng $\dfrac{3}{8}$ số ban đầu. Tìm tích các chữ số của số ban đầu.
	\choice
	{$12$}
	{$16$}
	{\True $14$}
	{$6$}
	\loigiai{
		Gọi số cần tìm là $\overline{ab}$; $a\in\mathbb{N}^*$; $b\in\mathbb{N}^*$; $a$, $b\leq9$.\\
		Đổi chỗ hai chữ số của nó thì được một số mới là $\overline{ba}$.\\
		Ta có hệ phương trình
		\allowdisplaybreaks
		\begin{eqnarray*}
			&&\heva{
				&a-b=5\\
				&\overline{ba}=\dfrac{3}{8}\overline{ab}}\\
			&&\heva{
				&a=b+5\\
				&b\cdot10+a=\dfrac{3}{8}(a\cdot10+b)}\\
			&&\heva{
				&a=b+5\\
				&80b+8(b+5)=30(b+5)+3b}\\
			&&\heva{
				&a=b+5\\
				&55b=110}\\
			&&\heva{
				&b=2\\
				&a=7.}\text{ (thỏa mãn)}
		\end{eqnarray*}
		Vậy số cần tìm là $72$ nên tích các chữ số là $2\cdot7=14$.
	}
\end{ex}
%%%=============================%%%

%%%=============EX_3=============%%%
\begin{ex}%[Dự án EX-9-Đề Cương Toán 9]%[Lương Pho]%[9D1V3-4]
	Một ô tô đi quãng đường $AB$ với vận tốc $50$ km/giờ, rồi đi tiếp quãng đường $BC$ với vận tốc $45$ km/giờ. Biết quãng đường tổng cộng độ dài $165$ km và thời gian ô tô đi trên quãng đường $AB$ ít hơn thời gian đi trên quãng đường $BC$ là $30$ phút. Tính thời gian ô tô đi trên đoạn đường $AB$.
	\choice
	{$2$ giờ}
	{\True $1{,}5$ giờ}
	{$1$ giờ}
	{$3$ giờ}
	\loigiai{
		Đổi $30\text{ phút}=0{,}5\text{ giờ}$.\\
		Gọi thời gian ô tô đi trên mỗi đoạn đường $AB$ và $BC$ lần lượt là $x$, $y$ ($x>0$; $y>0{,}5$; đơn vị: giờ).\\
		Thời gian ô tô đi trên quãng đường $AB$ ít hơn thời gian đi trên quãng đường $BC$ là $30$ phút nên
		\[x-y=-0{,}5.\qquad (1)\]
		Một ô tô đi quãng đường $AB$ với vận tốc $50$ km/giờ, nên quãng đường $AB$ là
		$50x$ km.\\
		Quãng đường $BC$ đi với vận tốc $45$ km/giờ, nên quãng đường $BC$ là
		$45x$ km.\\
		Tổng quãng đường dài $165$ km nên
		\[50x+45y=165.\qquad(2)\]
		Từ $(1)$ và $(2)$ ta có hệ phương trình $\heva{&x-y=-0{,}5\\&50 x+45y=165.}$\\
		Giải hệ ta được $x=1{,}5$, $y=2$ (thỏa mãn).\\
		Vậy thời gian ô tô đi hết quãng đường $AB$ là $1{,}5$ giờ; thời gian ô tô đi hết quãng đường $BC$ là $2$ giờ.
	}
\end{ex}
%%%=============================%%%

%%%=============EX_4=============%%%
\begin{ex}%[Dự án EX-9-Đề Cương Toán 9]%[Lương Pho]%[9D1V3-4]
	Trên một cánh đồng cấy $60$ ha lúa giống mới và $40$ ha lúa giống cũ, thu hoạch được tất cả $460$ tấn thóc. Hỏi năng suất lúa mới trên $1$ ha là bao nhiêu, biết rằng $3$ ha trồng lúa mới thu hoạch được ít hơn $4$ ha trồng lúa cũ là $1$ tấn.
	\choice
	{\True $5$ tấn}
	{$4$ tấn}
	{$6$ tấn}
	{$3$ tấn}
	\loigiai{
		Gọi năng suất lúa mới và lúa cũ trên $1$ ha lần lượt là $x$; $y$ $(x,y>0)$; đơn vị: tấn/ha.\\
		Vì $3$ ha trồng lúa mới thu hoạch được ít hơn $4$ ha trồng lúa cũ là $1$ tấn nên ta có phương trình \[3x-4y=-1.\qquad(1)\]
		Vì cấy $60$ ha lúa giống mới và $40$ ha lúa giống cũ, thu hoạch được tất cả $460$ tấn thóc nên ta có \[60x+40y=460.\qquad(2)\]
		Từ $(1)$ và $(2)$ ta có hệ phương trình $\heva{&3x-4y=-1\\&60x+40y=460.}$\\
		Giải hệ ta được $x=5$, $y=4$ (thỏa mãn).\\
		Vậy năng suất lúa mới trên $1$ ha là $5$ tấn.
	}
\end{ex}
%%%=============================%%%

%%%=============EX_5=============%%%
\begin{ex}%[Dự án EX-9-Đề Cương Toán 9]%[Lương Pho]%[9D1V3-4]
	Một ô tô dự định đi từ A đến B trong một thời gian nhất định. Nếu xe chạy mỗi giờ nhanh hơn $10$ km thì đến nơi sớm hơn dự định $3$ giờ, còn nếu xe chạy chậm lại mỗi giờ $10$ km thì đến nơi chậm mất $5$ giờ. Tính vận tốc của xe lúc ban đầu.
	\choice
	{\True $40$ km/giờ}
	{$35$ km/giờ}
	{$50$ km/giờ}
	{$60$ km/giờ}
	\loigiai{
		Gọi vận tốc lúc đầu của xe là $x$ (km/giờ; $x>10$), thời gian di chuyển theo dự định là $y$ ($y>3$) (giờ).\\
		Nếu xe chạy mỗi giờ nhanh hơn $10$ km thì đến nơi sớm hơn dự định $3$ giờ nên ta có phương trình \[(x+10)(y-3)=xy\text{ hay }3x+10y=30.\qquad(1)\]
		Nếu xe chạy chậm lại mỗi giờ $10$ km thì đến nơi chậm mất $5$ giờ nên ta có phương trình \[(x-10)(y+5)=xy\text{ hay }5x-10y=50.\qquad(2)\]
		Từ $(1)$ và $(2)$ ta cóhệ phương trình
		\[\heva{&-3x+10y=30\\&5x-10y=50.}\]
		Giải hệ ta được $x=40$, $y=15$ (thoả mãn điều kiện).\\
		Vậy vận tốc ban đầu là $40$ km/giờ.
	}
\end{ex}
%%%=============================%%%

%%%=============EX_6=============%%%
\begin{ex}%[Dự án EX-9-Đề Cương Toán 9]%[Lương Pho]%[9D1V3-4]
	Một xe đạp dự định đi từ A đến B trong một thời gian nhất định. Nếu xe chạy mỗi giờ nhanh hơn $10$ km thì đến nơi sớm hơn dự định $1$ giờ, còn nếu xe chạy chậm lại mỗi giờ $5$ km thì đến nơi chậm mất $2$ giờ. Tính vận tốc của xe lúc ban đầu.
	\choice
	{$8$ km/giờ}
	{$12$ km/giờ}
	{\True $10$ km/giờ}
	{$20$ km/giờ}
	\loigiai{
		Gọi vận tốc lúc đầu của xe $x$ (km/giờ; $x>10$), thời gian theo dự định là $y$, $(y>3)$ (giờ).\\
		Nếu xe chạy mỗi giờ nhanh hơn $10$ km thì đến nơi sớm hơn dự định $10$ km/giờ nên ta có phương trình \[(x+10)(y-1)=xy\text{ hay }-x+10y=10.\qquad(1)\]
		Nếu xe chạy chậm lại mỗi giờ $5$ km thì đến nơi chậm mất $2$ giờ nên ta có phương trình \[(x-5)(y+2)=xy\text{ hay }2x-5y=10.\qquad(2)\]
		Từ $(1)$ và $(2)$ ta có hệ phương trình
		\[\heva{&-x+10y=10\\&2x-5y=10.}\]
		Giải hệ ta được $x=10$, $y=2$ (thoả mãn điều kiện).\\
		Vậy vận tốc ban đầu là $10$ km/giờ.
	}
\end{ex}
%%%=============================%%%

%%%=============EX_7=============%%%
\begin{ex}%[Dự án EX-9-Đề Cương Toán 9]%[Lương Pho]%[9D1V3-4]
	Một cano chạy trên sông trong $7$ giờ, xuôi dòng $108$ km và ngược dòng $63$ km. Một lần khác cũng trong $7$ giờ cano xuôi dòng $81$ km và ngược dòng $84$ km. Tính vận tốc nước chảy.
	\choice
	{$4$ km/giờ}
	{\True $3$ km/giờ}
	{$2$ km/giờ}
	{$2{,}5$ km/giờ}
	\loigiai{
		Gọi vận tốc thực của canô là $x$ km/giờ, ()$x>0$), vận tốc dòng nước là $y$ (km/giờ, $y>0$).\\
		Vận tốc cano khi xuôi dòng là $x+y$ km/giờ vận tốc cano khi ngược dòng là $x-y$ km/giờ.\\
		Canô chạy trên sông trong $7$ giờ, xuôi dòng $108$ km và ngược dòng $63$ km nên ta có phương trình 
		\[\dfrac{108}{x+y}+\dfrac{63}{x-y}=7.\qquad(1)\]
		Canô chạy trên sông trong $7$ giờ canô xuôi dòng $81$ km và ngược dòng $84$ km nên ta có phương trình \[\dfrac{81}{x+y}+\dfrac{84}{x-y}=7.\qquad(2)\]
		Từ $(1)$ và $(2)$ ta có hệ phương trình
		\allowdisplaybreaks
		\begin{eqnarray*}
			&&\heva{
				&\dfrac{108}{x+y}+\dfrac{63}{x-y}=7\\
				&\dfrac{81}{x+y}+\dfrac{84}{x-y}=7}\\
			&&\heva{
				&\dfrac{432}{x+y}+\dfrac{252}{x-y}=28\\
				&\dfrac{243}{x+y}+\dfrac{252}{x-y}=21}\\
			&&\heva{
				&\dfrac{1}{x+y}=\dfrac{1}{27}\\
				&\dfrac{1}{x-y}=\dfrac{1}{21}}\\
			&&\heva{
				&x+y=27\\
				&x-y=21}
		\end{eqnarray*}
		Giải hệ phương trình ta được $\heva{&x=24\\&y=3}$ (thỏa mãn).\\
		Vậy vận tốc dòng nước là $3$ km/giờ.
	}
\end{ex}
%%%=============================%%%

%%%=============EX_8=============%%%
\begin{ex}%[Dự án EX-9-Đề Cương Toán 9]%[Lương Pho]%[9D1V3-4]
	Một chiếc cano đi xuôi dòng theo một khúc sông trong $3$ giờ và đi ngược dòng trong $4$ giờ, được $380$ km. Một lần khác cano này xuôi dòng trong $1$ giờ và ngược dòng trong vòng $30$ phút được $85$ km. Hãy tính vận tốc của dòng nước (vận tốc thật của cano và vận tốc dòng nước ở hai lần là như nhau).
	\choice
	{\True $5$ km/giờ}
	{$3$ km/giờ}
	{$2$ km/giờ}
	{$2{,}5$ km/giờ}
	\loigiai{
		Đổi $30$ phút $=\dfrac{1}{2}$ giờ.\\
		Gọi vận tốc thực của canô là $x$ (km/giờ, $x>0$), vận tốc dòng nước là $y$ (km/giờ, $y>0$).\\
		Vận tốc cano khi xuôi dòng là $x+y$ km/giờ, vận tốc cano khi ngược dòng là $x-y$ km/giờ.\\
		Canô đi xuôi dòng theo một khúc sông trong $3$ giờ và đi ngược dòng trong $4$ giờ, được $380$ km nên ta có phương trình \[3(x+y)+4(x-y)=380.\qquad(1)\]
		Canô xuôi dòng trong $1$ giờ và ngược dòng trong vòng $30$ phút được $85$ km nên ta có phương trình \[(x+y)+\dfrac{1}{2}(x-y)=85.\qquad(2)\]
		Từ $(1)$ và $(2)$ ta có hệ phương trình
		\allowdisplaybreaks
		\begin{eqnarray*}
			&&\heva{
				&3(x+y)+4(x-y)=380\\
				&x+y+\dfrac{1}{2}(x-y)=85}\\
			&&\heva{
				&7x-y=380\\
				&3x+y=170}\\
			&&\heva{
				&10x=550\\
				&3x+y=170}\\
			&&\heva{
				&x=55\\
				&y=5.}
		\end{eqnarray*}
		Giải hệ phương trình ta được $x=55$, $y=5$ (thỏa mãn).\\
		Vậy vận tốc dòng nước là $5$ km/giờ.
	}
\end{ex}
%%%=============================%%%

%%%=============EX_9=============%%%
\begin{ex}%[Dự án EX-9-Đề Cương Toán 9]%[Lương Pho]%[9D1V3-4]
	Hai người đi xe đạp xuất phát đồng thời từ hai thành phố cách nhau $38$ km. Họ đi ngược chiều và gặp nhau sau $2$ giờ. Hỏi vận tốc của người thứ nhất, biết rằng đến khi gặp nhau, người thứ nhất đi được nhiều hơn người thứ hai $2$ km?
	\choice
	{$7$ km/giờ}
	{$8$ km/giờ}
	{$9$ km/giờ}
	{\True $10$ km/giờ}
	\loigiai{
		Gọi vận tốc của người thứ nhất và người thứ hai lần lượt là $x$, $y$ (km/giờ, $x$, $y>0$).\\		
		Quãng đường người thứ nhất đi được khi gặp nhau là $2x$ km.\\
		Quãng đường người thứ hai đi được đến khi gặp nhau là $2y$ km.\\
		Đến khi gặp nhau, người thứ nhất đi được nhiều hơn người thứ hai $2$ km, nên
		\[2x-2y=2.\qquad(1)\]
		Tổng quãng đường hai người đi được bằng khoảng cách hai thành phố và bằng $38$ km. Ta có phương trình \[2x+2y=38.\qquad(2)\]
		Từ $(1)$ và $(2)$ ta có hệ phương trình \[\heva{&2x-2y=2\\&2x+2y=38.}\]
		Giải hệ phương trình ta được $x=10$, $y=9$ (thỏa mãn).\\
		Vậy vận tốc của người thứ nhất là $10$ km/giờ.
	}
\end{ex}
%%%=============================%%%

%%%=============EX_10=============%%%
\begin{ex}%[Dự án EX-9-Đề Cương Toán 9]%[Lương Pho]%[9D1V3-4]
	Hai người đi xe máy xuất phát đồng thời từ hai thành phố cách nhau $225$ km. Họ đi ngược chiều và gặp nhau sau $3$ giờ. Hỏi vận tốc của người thứ nhất, biết rằng vận tốc người thứ nhất lớn hơn người thứ hai $5$ km/giờ?
	\choice
	{\True $40$ km/giờ}
	{$35$ km/giờ}
	{$45$ km/giờ}
	{$50$ km/giờ}
	\loigiai{
		Gọi vận tốc của người thứ nhất và người thứ hai lần lượt là $x$, $y$ km/giờ, ($x>5$, $y>0$).\\
		Vận tốc người thứ nhất lớn hơn người thứ hai $5$ km/giờ nên ta có
		\[x-y=5.\qquad(1)\]
		Quãng đường người thứ nhất đi được khi gặp nhau là $3x$ km.\\
		Quãng đường người thứ hai đi được đến khi gặp nhau là $3y$ km.\\
		Tổng quãng đường hai người đi được bằng khoảng cách hai thành phố và bằng $225$ km. Ta có phương trình \[3x+3y=225.\qquad(2)\]
		Từ $(1)$ và $(2)$ ta có hệ phương trình \[\heva{&3x+3y=225\\&x-y=5.}\]
		Giải hệ phương trình ta được $x=40$, $y=35$ (thỏa mãn).\\
		Vậy vận tốc của người thứ nhất là $40$ km/giờ.
	}
\end{ex}
%%%=============================%%%

%%%=============EX_11=============%%%
\begin{ex}%[Dự án EX-9-Đề Cương Toán 9]%[Lương Pho]%[9D1V3-4]
	Một khách du lịch đi trên ô tô $4$ giờ, sau đó đi tiếp bằng tàu hỏa trong $7$ giờ được quãng đường dài $640$ km. Hỏi vận tốc của tàu hỏa, biết rằng mỗi giờ tàu hỏa đi nhanh hơn ô tô $5$ km?
	\choice
	{$40$ km/giờ}
	{$50$ km/giờ}
	{\True $60$ km/giờ}
	{$65$ km/giờ}
	\loigiai{
		Gọi vận tốc của tàu hỏa và ô tô lần lượt là $x$, $y$ (km/giờ, $x>y>0$; $x>5$).\\
		Vì khách du lịch đi trên ôtô $4$ giờ, sau đó đi tiếp bằng tàu hỏa trong $7$ giờ được quãng đường dài $640$ km nên ta có phương trình $7x+4y=640$.\qquad$(1)$\\
		Và mỗi giờ tàu hỏa đi nhanh hơn ôtô $5$ km nên ta có phương trình $x-y=5$.\qquad$(2)$\\
		Từ $(1)$ và $(2)$ ta có hệ phương trình \[\heva{&x-y=5\\&7x+4y=640.}\]
		Giải hệ phương trình ta được $x=60$, $y=55$ (thỏa mãn).\\
		Vậy vận tốc tàu hỏa là $60$ km/giờ.
	}
\end{ex}
%%%=============================%%%

%%%=============EX_12=============%%%
\begin{ex}%[Dự án EX-9-Đề Cương Toán 9]%[Lương Pho]%[9D1V3-4]
	Hai vòi nước cùng chảy vào một bể thì sau $4$ giờ $48$ phút bể đầy. Nếu vòi $I$ chảy riêng trong $4$ giờ, vòi $II$ chảy riêng trong $3$ giờ thì cả hai vòi chảy được $\dfrac{3}{4}$ bể. Thời gian vòi $I$ một mình đầy bể là
	\choice
	{$6$ giờ}
	{\True $8$ giờ}
	{$10$ giờ}
	{$12$ giờ}
	\loigiai{
		Đổi $4$ giờ $48$ phút $=\dfrac{24}{5}$ giờ.\\
		Gọi thời gian vòi $I$, vòi $II$ chảy một mình đầy bể lần lượt là $x$, $y$, $\left(x,\,y>\dfrac{24}{5}\right)$ (đơn vị: giờ).\\
		Mỗi giờ vòi $I$ chảy được $\dfrac{1}{x}$bể, vòi $II$ chảy được $\dfrac{1}{y}$ bể nên cả hai vòi chảy được $\dfrac{1}{x}+\dfrac{1}{y}$ bể.\\
		Vì hai vòi nước cùng chảy vào một bể thì sau $4$ giờ $48$ phút bể đầy nên ta có phương trình \[\dfrac{1}{x}+\dfrac{1}{y}=\dfrac{5}{24}.\qquad(1)\]
		Nếu vòi $I$ chảy riêng trong $4$ giờ, vòi $II$ chảy riêng trong $3$ giờ thì cả hai vòi chảy được $\dfrac{3}{4}$ bể nên ta có phương trình \[\dfrac{4}{x}+\dfrac{3}{y}=\dfrac{3}{4}.\qquad(2)\]
		Từ $(1)$ và $(2)$ ta có hệ phương trình
		\allowdisplaybreaks
		\begin{eqnarray*}
			&& \heva{
				&\dfrac{1}{x}+\dfrac{1}{y}=\dfrac{5}{24}\\
				&\dfrac{4}{x}+\dfrac{3}{y}=\dfrac{3}{4}}\\
			&&\heva{
				&\dfrac{4}{x}+\dfrac{3}{y}=\dfrac{3}{4}\\
				&\dfrac{3}{x}+\dfrac{3}{y}=\dfrac{5}{8}}\\
			&&\heva{
				&\dfrac{1}{x}=\dfrac{1}{8}\\
				&\dfrac{1}{y}=\dfrac{1}{12.}}
		\end{eqnarray*}
		Suy ra $x=8$, $y=12$ (thỏa mãn).\\
		Vậy thời gian vòi $I$ một mình đầy bể là $8$ giờ.
	}
\end{ex}
%%%=============================%%%

%%%=============EX_13=============%%%
\begin{ex}%[Dự án EX-9-Đề Cương Toán 9]%[Lương Pho]%[9D1V3-4]
	Năm ngoái, cả $2$ cánh đồng thu hoạch được $500$ tấn thóc. Năm nay, do áp dụng khoa học kĩ thuật nên lượng lúa thu được trên cánh đồng thứ nhất tăng lên $30\%$ so với năm ngoái, trên cánh đồng thứ hai tăng $20\%$ so với năm ngoái. Do đó tổng cộng cả $2$ cánh đồng thu được $630$ tấn thóc. Hỏi trên mỗi cánh đồng năm nay thu được bao nhiêu tấn thóc.
	\choice
	{$400$ tấn và $230$ tấn}
	{\True $290$ tấn và $240$ tấn}
	{$380$ tấn và $250$ tấn}
	{$290$ tấn và $230$ tấn}
	\loigiai{
		Gọi số thóc năm ngoái thu được của cánh đồng thứ nhất là $x$ tấn ($x>0$).\\
		Gọi số thóc năm ngoái thu được của cánh đồng thứ hai là $y$ tấn ($y>0$).\\
		Năm ngoái, cả $2$ cánh đồng thu hoạch được $500$ tấn thóc nên ta có phương trình \[x+y=500.\qquad(1)\]
		Năm nay, lượng lúa thu được trên cánh đồng thứ nhất tăng lên $30\%$ so với năm ngoái, trên cánh đồng thứ hai tăng $20\%$ so với năm ngoái, nên ta có phương trình \[x+\dfrac{30}{100}x+y+\dfrac{20}{100}y=630\text{ hay }\dfrac{130}{100}x+\dfrac{120}{100}y=630.\qquad(2)\]
		Từ $(1)$ và $(2)$ ta có hệ phương trình
		\[\heva{&x+y=500\\&\dfrac{130}{100}x+\dfrac{120}{100}y=630.}\]
		Giải hệ ta được $x=300$, $y=200$ (thỏa mãn).\\
		Vậy lượng lúa thu được năm nay của
		\begin{itemize}
			\item cánh đồng thứ nhất là $300\cdot1{,}3=390$ tấn;
			\item cánh đồng thứ hai là $200\cdot1{,}2=240$ tấn.
		\end{itemize}
	}
\end{ex}
%%%=============================%%%

%%%=============EX_14=============%%%
\begin{ex}%[Dự án EX-9-Đề Cương Toán 9]%[Lương Pho]%[9D1V3-4]
	Trong tháng đầu hai tổ sản xuất được $800$ sản phẩm. Sang tháng thứ hai, tổ $1$ sản xuất vượt mức $12\%$, tổ $2$ giảm $10\%$ so với tháng đầu nên cả hai tổ làm được $786$ sản phẩm. Số sản phẩm tổ $1$ làm được trong tháng đầu là
	\choice
	{$500$ sản phẩm}
	{\True $300$ sản phẩm}
	{$200$ sản phẩm}
	{$400$ sản phẩm}
	\loigiai{
		Gọi số sản phẩm tổ $1$ và tổ $2$ làm được trong tháng đầu lần lượt là $x$, $y$ ($x$, $y\in\mathbb{N}^*$, $x$, $y<800$).\\
		Số sản phẩm tổ $1$ và tổ $2$ làm được trong tháng hai là $112\%x$ và $90\%y$ sản phẩm.\\
		Trong tháng đầu hai tổ sản xuất được $800$ sản phẩm, nên
		\[x+y=800.\qquad(1)\]
		Sang tháng thứ hai, cả hai tổ làm được $786$ sản phẩm 
		\[112\%x+90\%y=786.\qquad(2)\]
		Từ $(1)$ và $(2)$ ta có hệ phương trình
		\[\heva{&x+y=800\\&112\%x+90\%y=786.}\]
		Giải hệ ta được $x=300$, $y=500$. (thỏa mãn)\\
		Vậy số sản phẩm tổ $1$ làm được trong tháng đầu là $300$ sản phẩm.
	}
\end{ex}
%%%=============================%%%

%%%=============EX_15=============%%%
\begin{ex}%[Dự án EX-9-Đề Cương Toán 9]%[Lương Pho]%[9D1C3-4]
	Một tam giác có chiều cao bằng $\dfrac{3}{4}$ cạnh đáy. Nếu chiều cao tăng thêm $3$ dm và cạnh đáy giảm đi $3$ dm thì diện tích của nó tăng thêm $12$ dm$^2$. Diện tích ban đầu của tam giác bằng
	\choice
	{$700$ dm$^2$}
	{$678$ dm$^2$}
	{$627$ dm$^2$}
	{\True $726$ dm$^2$}
	\loigiai{
		Gọi chiều cao của tam giác là $h$, cạnh đáy tam giác là $a$, ($h$, $a\in\mathbb{N}^*$, $a>3$, đơn vị dm).\\
		Diện tích ban đầu của tam giác là $\dfrac{1}{2}ah$ dm$^2$.\\
		Vì chiều cao bằng $\dfrac{3}{4}$ cạnh đáy nên ta có phương trình $h=\dfrac{3}{4}a$.\qquad$(1)$\\
		Nếu chiều cao tăng thêm $3$ dm và cạnh đáy giảm đi $3$ dm thì diện tích của nó tăng thêm $12$ dm$^2$ nên ta có phương trình \[\dfrac{1}{2}(h+3)(a-3)-\dfrac{1}{2}ah=12.\qquad(2)\]
		Từ $(1)$ và $(2)$ ta có hệ phương trình		
		\allowdisplaybreaks
		\begin{eqnarray*}
			&&\heva{&h=\dfrac{3}{4}a\\
				&\dfrac{1}{2}(h+3)(a-3)-\dfrac{1}{2}ah=12}\\
			&&\heva{&h-\dfrac{3}{4}a=0\\
				&\dfrac{-3h}{2}+\dfrac{3a}{2}=\dfrac{33}{2}}\\
			&&\heva{&a=44\\
				&h=33} \text { (thỏa mãn).}
		\end{eqnarray*}
		Vậy chiều cao của tam giác bằng $44$ dm, cạnh đáy tam giác bằng $33$ dm.\\
		Suy ra diện tích ban đầu của tam giác là $\dfrac{1}{2}\cdot44\cdot33=726\,\left(\text{dm}^2\right)$.
	}
\end{ex}
%%%=============================%%%

%%%=============EX_16=============%%%
\begin{ex}%[Dự án EX-9-Đề Cương Toán 9]%[Lương Pho]%[9D1V3-4]
	Trong một kì thi, hai trường A, B có tổng cộng $350$ học sinh dự thi. Kết quả hai trường có $338$ học sinh trúng tuyển. Số học sinh trúng tuyển của trường A và trường B lần lượt bằng $97\%$, $96\%$ số học sinh dự thi. Hỏi trường B có bao nhiêu học sinh dự thi?
	\choice
	{$200$ học sinh}
	{\True $150$ học sinh}
	{$250$ học sinh}
	{$225$ học sinh}
	\loigiai{
		Gọi số học sinh dự thi của hai trường A, B lần lượt là $x$, $y$ ($x$, $y\in \mathbb{N^*}$).\\
		Vì hai trường A, B có tổng cộng $350$ học sinh dự thi nên ta có phương trình 
		\[x+y=350.\qquad(1)\]
		Vì trường A có $97\%$ và trường B có $96\%$ số học sinh trúng tuyển và cả hai trường có $338$ học sinh trúng tuyển nên ta có phương trình\[97\%\cdot x+96\%\cdot y=338.\qquad(2)\]
		Từ $(1)$ và $(2)$ ta có hệ phương trình \[\heva{&x+y=350\\&97\%\cdot x+96\%\cdot y=338.}\]
		Giải hệ ta được $x=200$, $y=150$ (thỏa mãn).\\
		Vậy trường B có $150$ học sinh dự thi.
	}
\end{ex}
%%%=============================%%%

%%%=============EX_17=============%%%
\begin{ex}%[Dự án EX-9-Đề Cương Toán 9]%[Lương Pho]%[9D1V3-4]
	Một khu vườn hình chữ nhật có chu vi bằng $48$ m. Nếu tăng chiều rộng lên bốn lần và tăng chiều dài lên ba lần thì chu vi của khu vườn sẽ là $162$ m. Diện tích của khu vườn ban đầu bằng
	\choice
	{$24$ m$^2$}
	{$153$ m$^2$}
	{\True $135$ m$^2$}
	{$14$ m$^2$}
	\loigiai{
		Gọi chiều dài và chiều rộng ban đầu của khu vườn hình chữ nhật lần lượt là $x$, $y$ ($24>x>y>0$; đơn vị m).\\
		Vì khu vườn hình chữ nhật có chu vi bằng $48$ m nên ta có\[(x+y)\cdot2=48\text{ hay } x+y=24.\qquad(1)\]
		Nếu tăng chiều rộng lên bốn lần và chiều dài lên ba lần thì chu vi của khu vườn sẽ là $162$ m, nên ta có phương trình \[(4y+3x)\cdot2=162\text{ hay }3x+4y=81.\qquad(2)\]
		Từ $(1)$ và $(2)$ ta có hệ phương trình \[\heva{&x+y=24\\&3x+4y=81.}\]
		Giải hệ ta được $x=15$, $y=9$ (thỏa mãn).\\
		Vậy diện tích khu vườn ban đầu là $15\cdot9=135$ (m$^2$).
	}
\end{ex}
%%%=============================%%%

%%%=============EX_18=============%%%
\begin{ex}%[Dự án EX-9-Đề Cương Toán 9]%[Lương Pho]%[9D1V3-4]
	Một hình chữ nhật có chu vi $300$ cm. Nếu tăng chiều rộng thêm $5$ cm và giảm chiều dài $5$ cm thì diện tích tăng $275$ cm$^2$. Chiều dài và chiều rộng của hình chữ nhật lần lượt bằng
	\choice
	{$120$ cm và $30$ cm}
	{\True $105$ cm và $45$ cm}
	{$70$ cm và $80$ cm}
	{$90$ cm và $60$ cm}
	\loigiai{
		Gọi chiều dài và chiều rộng của khu vườn hình chữ nhật lần lượt là $x$, $y$ ($150>x>y>0$; cm).\\
		Diện tích ban đầu của khu vườn là $x\cdot y$ (cm$^2$).\\
		Vì khu vườn hình chữ nhật có chu vi bằng $300$ cm nên ta có \[(x+y)\cdot2=300\text{ hay }x+y=150.\qquad(1)\]
		Nếu tăng chiều rộng thêm $5$ cm và giảm chiều dài $5$ cm thì diện tích tăng $275$ cm$^2$, nên ta có phương trình \[(x-5)(y+5)=xy+275\text{ hay }x-y=60.\qquad(2)\]
		Từ $(1)$ và $(2)$ ta có hệ phương trình \[\heva{&x+y=150\\&x-y=60.}\]
		Giải hệ ta được $x=105$, $y=45$ (thỏa mãn).\\
		Vậy chiều dài và chiều rộng của hình chữ nhật lần lượt bằng $105$ cm và $45$ cm.
	}
\end{ex}
%%%=============================%%%

%%%=============EX_19=============%%%
\begin{ex}%[Dự án EX-9-Đề Cương Toán 9]%[Lương Pho]%[9D1V3-4]
	Hai giá sách có $450$ cuốn. Nếu chuyển $50$ cuốn từ giá thứ nhất sang giá thứ hai thì số sách trên giá thứ hai bằng $\dfrac{4}{5}$ số sách giá thứ nhất. Số sách trên giá thứ hai bằng
	\choice
	{\True $150$ cuốn}
	{$300$ cuốn}
	{$200$ cuốn}
	{$150$ cuốn}
	\loigiai{
		Gọi số sách trên hai giá lần lượt là $x$, $y$ ($0<x$, $y<450$, cuốn).\\
		Vì hai giá sách có $450$ cuốn nên ta có phương trình $x+y=450$.\qquad $(1)$\\
		Nếu chuyển $50$ cuốn từ giá thứ nhất sang giá thứ hai thì số sách trên giá thứ hai bằng $\dfrac{4}{5}$ số sách ở giá thứ nhất nên ta có phương trình\[y+50=\dfrac{4}{5}(x-50) \text{ hay }\dfrac{4}{5}x-y=90.\]
		Từ $(1)$ và $(2)$ ta có hệ phương trình \[\heva{&x+y=450\\&\dfrac{4}{5}x-y=90.}\]
		Giải hệ ta được $x=300$, $y=150$ (thỏa mãn).\\
		Vậy số sách trên giá thứ nhất là $300$ cuốn; số sách trên giá thứ hai là $150$ cuốn.
	}
\end{ex}
%%%=============================%%%

%%%=============EX_20=============%%%
\begin{ex}%[Dự án EX-9-Đề Cương Toán 9]%[Lương Pho]%[9D1V3-4]
	Nam có $360$ viên bi trong hai hộp. Nếu Nam chuyển $30$ viên từ hộp thứ hai sang hộp thứ nhất thì số viên bi ở hộp thứ nhất bằng $\dfrac{5}{7}$ số viên bi ở hộp thứ hai. Hỏi hộp thứ hai có bao nhiêu viên bi?
	\choice
	{$250$ viên}
	{$180$ viên}
	{$120$ viên}
	{\True $240$ viên}
	\loigiai{
		Gọi số viên bi trong hộp thứ nhất và hộp thứ hai lần lượt là $x$, $y$ ($0<x$, $y<360$, viên).\\
		Vì Nam có $360$ viên bi nên ta có phương trình $x+y=360$. \qquad$(1)$\\
		Nếu Nam chuyển $30$ viên bi từ hộp thứ hai sang hộp thứ nhất thì số viên bi ở hộp thứ nhất bằng $\dfrac{5}{7}$ số viên bi ở hộp thứ hai nên ta có phương trình \[x+30=\dfrac{5}{7}(y-30) \text{ hay } 7x-5y=-360.\qquad(2)\]
		Từ $(1)$ và $(2)$ ta có hệ phương trình \[\heva{&x+y=360\\&7x-5y=-360.}\]
		Giải hệ ta được $x=120$, $y=240$ (thỏa mãn).\\
		Vậy số viên bi ở hộp thứ nhất là $120$ bi, số viên bi ở hộp thứ hai là $240$ viên bi.
	}
\end{ex}
%%%=============================%%%
\Closesolutionfile{ans}

