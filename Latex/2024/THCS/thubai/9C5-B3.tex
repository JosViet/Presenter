\section{GÓC Ở TÂM - GÓC NỘI TIẾP} % Tên bài
\subsection{Kiến thức trọng tâm}
\subsubsection{Góc ở tâm}
\begin{tomtat}
	\begin{boxdn}
		\textit{Góc ở tâm} là góc có đỉnh trùng với tâm đường tròn.
		\begin{center}
			\begin{tikzpicture}[scale=1, font=\footnotesize, line join=round, line cap=round, >=stealth]
				\coordinate (O) at (0,0);
				\coordinate (A) at ({1*cos(80)},{1*sin(80)});
				\coordinate (B) at ({1*cos(-20)},{1*sin(-20)});
				\foreach \x/\y/\z in {B/O/A}{
				\path pic[draw,fill=orange!35,angle radius=10pt]{angle= \x--\y--\z};}
				\draw (O) circle (1cm);
				\foreach \x/\y in {O/A,O/B}{
					\draw (\x)--(\y);
				}
				\foreach \x/\y in {O/180,A/80,B/-20}{
					\draw[fill=black] (\x) circle (1pt) ($(\x)+(\y:3mm)$) node {$\x$};
				}
			\end{tikzpicture}
		\end{center}
	\end{boxdn}
\end{tomtat}
%Ví dụ 1
\begin{vd}%[Dự án EX-9-Đề Cương Toán 9]%[Phạm Hoàng Việt]%[9H2N3-1]
	\immini{Cho tam giác $MNP$ có ba đỉnh nằm trên đường tròn $(I)$. Xác định các góc ở tâm của đường tròn
	\loigiai{
		Trong hình, đường tròn $(I)$ có các góc ở tâm là $\widehat{MIN}$, $\widehat{NIP}$, $\widehat{PIM}$.
	}}{\begin{tikzpicture}[scale=1, font=\footnotesize, line join=round, line cap=round, >=stealth]
		\coordinate (I) at (0,0);
		\coordinate (M) at ({2*cos(120)},{2*sin(120)});
		\coordinate (P) at ({2*cos(-30)},{2*sin(-30)});
		\coordinate (N) at ({2*cos(-150)},{2*sin(-150)});
		\draw (I) circle (2cm);
		\foreach \x/\y in{I/M,I/N,I/P,M/N,M/P,N/P}{
			\draw (\x)--(\y);
		}
		\foreach \x/\y in {I/180,M/120,N/-150,P/-30}{
			\draw[fill=black] (\x) circle (1pt) ($(\x)+(\y:3mm)$) node {$\x$};
		}
	\end{tikzpicture}}
\end{vd}

\subsubsection{Cung, số đo cung}
\begin{tomtat}
	\begin{enumerate}
		\item \textbf{Cung}\\
		\begin{boxdn}
			\begin{itemize}
			\item Mỗi phần đường tròn giới hạn bởi hai điểm $A$, $B$ trên đường tròn gọi là một \textit{cung} $AB$, kí hiệu là $\wideparen{AB}$.
				\begin{center}
					\begin{tikzpicture}[scale=1, font=\footnotesize, line join=round, line cap=round, >=stealth]
						\coordinate (O) at (0,0);
						\coordinate (A) at ({1*cos(80)},{1*sin(80)});
						\coordinate (B) at ({1*cos(-20)},{1*sin(-20)});
						\draw[domain=-20:80, blue] plot ({1*cos(\x)},{1*sin(\x)});
						\draw[domain=80:340, red] plot ({1*cos(\x)},{1*sin(\x)});
						\coordinate (n) at ({1*cos(30)},{1*sin(30)});
						\coordinate (m) at ({1*cos(210)},{1*sin(210)});
						\foreach \x/\y in {O/A,O/B}{
							\draw (\x)--(\y);
						}
						\foreach \x/\y in {O/180,A/80,B/-20}{
							\draw[fill=black] (\x) circle (1pt) ($(\x)+(\y:3mm)$) node {$\x$};
						}
						\foreach \x/\y in {n/30,m/210}{
							\draw ($(\x)+(\y:3mm)$) node {$\x$};
						}
					\end{tikzpicture}
				\end{center}
			\end{itemize}
		\end{boxdn}
		\begin{luuy}
			\begin{itemize}
				\item Cung nằm bên trong góc ở tâm $\widehat{AOB}$ được gọi là cung nhỏ, kí hiệu là $\wideparen{AnB}$. Ta còn nói $\wideparen{AnB}$ là cung bị chắn bởi góc $\widehat{AOB}$ hay $\widehat{AOB}$ chắn cung nhỏ $\wideparen{AnB}$.
				\item Cung nằm bên ngoài góc ở tâm $\widehat{AOB}$ được gọi là cung lớn, kí hiệu là $\wideparen{AmB}$.
				\item Khi nói \lq\lq góc ở tâm $\widehat{AOB}$ chắn cung $AB$\rq\rq~thì ta hiểu là góc ở tâm chắn cung nhỏ $AB$.
			\end{itemize}
		\end{luuy}
		\item \textbf{Số đo cung}\\
		\begin{boxdn}
			\begin{itemize}
				\item Số đo của cung nhỏ bằng số đo góc ở tâm chắn cung đó.
				\item Số đo của cung lớn bằng hiệu giữa $360^{\circ}$ và số đo cung nhỏ (có chung hai đầu mút với cung lớn).
				\item Số đo của nửa đường tròn bằng $180^{\circ}$.
				\item Số đo cung $\wideparen{AB}$, kí hiệu là sđ$\wideparen{AB}$.
			\end{itemize}
		\end{boxdn}
		\begin{luuy}
			\begin{itemize}
				\item Cung nhỏ có số đo nhỏ hơn $180^{\circ}$, cung lớn có số đo lớn hơn $180^{\circ}$. Cung nửa đường tròn có số đo $180^{\circ}$.
				\item Khi hai mút của cung trùng nhau, ta có \textit{cung không} với số đo $0^{\circ}$ và \textit{cung cả đường tròn} có số đo $360^{\circ}$.
				\item Một cung có số đo $n^{\circ}$ thường được gọi tắt là cung $n^{\circ}$.
				\item Trong một đường tròn, hai cung được gọi là bằng nhau nếu chúng có số đo bằng nhau.
			\end{itemize}
		\end{luuy}
	\end{enumerate}
\end{tomtat}
%Ví dụ 2
\begin{vd}%[Dự án EX-9-Đề Cương Toán 9]%[Phạm Hoàng Việt]%[9H2H3-2]
	\immini{Tính số đo các cung $\wideparen{AnB}$ và $\wideparen{AmB}$ trong hình bên.
	\loigiai{
		Ta có $\wideparen{AnB}$ bị chắn bởi góc ở tâm $\widehat{AOB}$ có số đo bằng $60^{\circ}$, suy ra sđ$\wideparen{AnB}=60^{\circ}$ và sđ$\wideparen{AmB}=360^{\circ}-60^{\circ}=300^{\circ}$.
	}}{\begin{tikzpicture}[scale=1, font=\footnotesize, line join=round, line cap=round, >=stealth]
					\coordinate (O) at (0,0);
					\coordinate (A) at ({2*cos(0)},{2*sin(0)});
					\coordinate (B) at ({2*cos(60)},{2*sin(60)});
					\draw (O) circle (2cm);
					\coordinate (n) at ({2*cos(30)},{2*sin(30)});
					\coordinate (m) at ({2*cos(210)},{2*sin(210)});
					\foreach \x/\y/\z in {A/O/B}{
					\path pic[draw,fill=orange!35,angle radius=10pt, pic text=$60^{\circ}$,
    				pic text options={shift={(10pt,5pt)}}]{angle= \x--\y--\z};}
					\foreach \x/\y in {O/A,O/B}{
						\draw (\x)--(\y);
					}
					\foreach \x/\y in {O/180,A/0,B/60}{
						\draw[fill=black] (\x) circle (1pt) ($(\x)+(\y:3mm)$) node {$\x$};
					}
					\foreach \x/\y in {n/30,m/210}{
						\draw ($(\x)+(\y:3mm)$) node {$\x$};
					}
				\end{tikzpicture}}
\end{vd}
\subsubsection{Góc nội tiếp}
\begin{tomtat}
	\begin{enumerate}
		\item \textbf{Nhận biết góc nội tiếp}\\ 
		\begin{boxdn}
			\begin{itemize}
			\item \textit{Góc nội tiếp} là góc có đỉnh nằm trên đường tròn và hai cạnh chứa hai dây cung của đường tròn đó. Cung nằm bên trong góc được gọi là \textit{cung bị chắn}.
			\begin{center}
				\begin{tikzpicture}[scale=1, font=\footnotesize, line join=round, line cap=round, >=stealth]
					\coordinate (O) at (0,0);
					\coordinate (A) at ({1*cos(-20)},{1*sin(-20)});
					\coordinate (B) at ({1*cos(60)},{1*sin(60)});
					\coordinate (C) at ({1*cos(200)},{1*sin(200)});
					\draw (O) circle (1cm);
					\foreach \x/\y/\z in {A/C/B}{
					\path pic[draw,fill=orange!35,angle radius=10pt]{angle= \x--\y--\z};}
					\foreach \x/\y in {A/C,B/C}{
						\draw (\x)--(\y);
					}
					\foreach \x/\y in {O/180,A/0,B/60,C/200}{
						\draw[fill=black] (\x) circle (1pt) ($(\x)+(\y:3mm)$) node {$\x$};
					}
				\end{tikzpicture}
			\end{center}
		\end{itemize}
		\end{boxdn}
		\item \textbf{Số đo góc nội tiếp}\\ 
		\begin{boxdn}
			\begin{itemize}
			\item Trong một đường tròn, số đo của góc nội tiếp bằng nửa số đo của cung bị chắn.
			\end{itemize}
		\end{boxdn}
		\begin{luuy}
			Trong một đường tròn
			\begin{itemize}
				\item Các góc nội tiếp bằng nhau chắn các cung bằng nhau.
				\item Các góc nội tiếp cùng chắn một cung hoặc chắn các cung bằng nhau thì bằng nhau.
				\item Góc nội tiếp nhỏ hơn hoặc bằng $90^{\circ}$ có số đo bằng nửa số đo của góc ở tâm cùng chắn một cung.
				\item Góc nội tiếp chắn nửa đường tròn là góc vuông.
			\end{itemize}
			\begin{center}
				\vspace{1cm}
				\begin{multicols}{2}
					\begin{enumerate}
						\item \begin{tikzpicture}[scale=1, font=\footnotesize, line join=round, line cap=round, >=stealth]
					\coordinate (O) at (0,0);
					\coordinate (A) at ({1*cos(-30)},{1*sin(-30)});
					\coordinate (B) at ({1*cos(60)},{1*sin(60)});
					\coordinate (C) at ({1*cos(200)},{1*sin(200)});
					\coordinate (D) at ({1*cos(-80)},{1*sin(-80)});
					\draw (O) circle (1cm);
					\foreach \x/\y/\z in {A/C/B,A/D/B}{
					\path pic[draw,fill=orange!35,angle radius=10pt]{angle= \x--\y--\z};}
					\foreach \x/\y in {A/C,B/C,D/A,D/B}{
						\draw (\x)--(\y);
					}
					\foreach \x/\y in {O/180,A/0,B/60,C/200,D/-80}{
						\draw[fill=black] (\x) circle (1pt) ($(\x)+(\y:3mm)$) node {$\x$};
					}
				\end{tikzpicture}
						\item \begin{tikzpicture}[scale=1, font=\footnotesize, line join=round, line cap=round, >=stealth]
					\coordinate (O) at (0,0);
					\coordinate (A) at ({1*cos(180)},{1*sin(180)});
					\coordinate (B) at ({1*cos(0)},{1*sin(0)});
					\coordinate (M) at ({1*cos(120)},{1*sin(120)});
					\coordinate (N) at ({1*cos(90)},{1*sin(90)});
					\coordinate (P) at ({1*cos(-80)},{1*sin(-80)});
					\draw (O) circle (1cm);
					\foreach \x/\y/\z in {A/M/B,A/N/B,A/P/B}{
					\path pic[draw,fill=orange!35,angle radius=5pt]{right angle= \x--\y--\z};}
					\foreach \x/\y in {A/M,B/M,N/A,N/B,A/P,P/B,A/B}{
						\draw (\x)--(\y);
					}
					\foreach \x/\y in {O/-90,A/180,B/0,M/120,N/90,P/-80}{
						\draw[fill=black] (\x) circle (1pt) ($(\x)+(\y:3mm)$) node {$\x$};
					}
				\end{tikzpicture}
					\end{enumerate}
				\end{multicols}
			\end{center}
		\end{luuy}
	\end{enumerate}
\end{tomtat}
%Ví dụ 3
\begin{vd}%[Dự án EX-9-Đề Cương Toán 9]%[Phạm Hoàng Việt]%[9H2H3-3]
	\immini{Tính số đo của $\widehat{AMB}$ và $\widehat{ANB}$ trong hình bên.
	\loigiai{
		Ta có $\widehat{AOB}=90^{\circ}$ và là góc ở tâm chắn $\wideparen{AB}$ nên sđ$\wideparen{AB}=\widehat{AOB}=90^{\circ}$.\\ 
		$\widehat{AMB}$ và $\widehat{ANB}$ là hai góc nội tiếp chắn $\wideparen{AB}$, suy ra $$\widehat{AMB}=\widehat{ANB}=\dfrac{1}{2}\text{sđ}\wideparen{AB}=\dfrac{1}{2}\cdot 90^{\circ}=45^{\circ}.$$
	}}{\begin{tikzpicture}[scale=1, font=\footnotesize, line join=round, line cap=round, >=stealth]
			\coordinate (O) at (0,0);
			\coordinate (A) at ({2*cos(45)},{2*sin(45)});
			\coordinate (B) at ({2*cos(-45)},{2*sin(-45)});
			\coordinate (M) at ({2*cos(100)},{2*sin(100)});
			\coordinate (N) at ({2*cos(180)},{2*sin(180)});
			\draw (O) circle (2cm);
			\foreach \x/\y/\z in {B/O/A}{
			\path pic[draw,fill=orange!35,angle radius=5pt]{right angle= \x--\y--\z};}
			\foreach \x/\y in {A/M,B/M,N/A,N/B,A/O,O/B}{
				\draw (\x)--(\y);
			}
			\foreach \x/\y in {O/180,A/45,B/-45,M/100,N/180}{
				\draw[fill=black] (\x) circle (1pt) ($(\x)+(\y:3mm)$) node {$\x$};
			}
		\end{tikzpicture}}
\end{vd}
%Ví dụ 4
\begin{vd}%[Dự án EX-9-Đề Cương Toán 9]%[Phạm Hoàng Việt]%[9H2V3-3]
	\immini{Cho $AB$ và $CD$ là hai đường kính vuông góc của đường tròn $(O)$. Gọi $M$, $N$ là hai điểm lần lượt trên hai cung nhỏ $\wideparen{AC}$, $\wideparen{BC}$ và chia mỗi cung đó thành hai cung bằng nhau (hình bên). Tìm số đo các góc sau
	\begin{enumerate}
		\item $\widehat{ACB}$, $\widehat{ADC}$.
		\item $\widehat{ADM}$, $\widehat{NCB}$.
	\end{enumerate}}{\begin{tikzpicture}[scale=1, font=\footnotesize, line join=round, line cap=round, >=stealth]
			\coordinate (O) at (0,0);
			\coordinate (A) at ({2*cos(180)},{2*sin(180)});
			\coordinate (B) at ({2*cos(0)},{2*sin(0)});
			\coordinate (C) at ({2*cos(90)},{2*sin(90)});
			\coordinate (D) at ({2*cos(-90)},{2*sin(-90)});
			\coordinate (N) at ({2*cos(45)},{2*sin(45)});
			\coordinate (M) at ({2*cos(135)},{2*sin(135)});
			\draw (O) circle (2cm);
			\foreach \x/\y/\z in {C/O/A}{
			\path pic[draw,fill=orange!35,angle radius=5pt]{right angle= \x--\y--\z};}
			\foreach \x/\y in {A/B,C/D,A/M,A/C,A/N,A/D,B/N,B/M,B/C,C/N,D/M}{
				\draw (\x)--(\y);
			}
			\foreach \x/\y in {O/-45,A/180,B/0,M/135,N/45,C/90,D/-90}{
				\draw[fill=black] (\x) circle (1pt) ($(\x)+(\y:3mm)$) node {$\x$};
			}
		\end{tikzpicture}}
		\loigiai{
			\begin{enumerate}
				\item Ta có $\widehat{ACB}$ là góc nội tiếp chắn nửa đường tròn, suy ra $\widehat{ACB}=90^{\circ}$.\\ 
				Ta có $\widehat{ADC}$ và $\widehat{AOC}$ lần lượt là góc nội tiếp và góc ở tâm cùng chắn cung $AC$, suy ra $\widehat{ADC}=\dfrac{\widehat{AOC}}{2}=\dfrac{90^{\circ}}{2}=45^{\circ}$.
				\item Do hai đường kính $AB$ và $CD$ vuông góc với nhau tại tâm $O$ của $(O)$ nên $\widehat{AOC}=\widehat{COB}=\widehat{BOD}=\widehat{DOA}=90^{\circ}$, suy ra sđ$\wideparen{AC}=\text{sđ}\wideparen{CD}=\text{sđ}\wideparen{BD}=\text{sđ}\wideparen{DA}=90^{\circ}$.\\
				Vì $M$, $N$ lần lượt chia $\wideparen{AC}$, $\wideparen{CB}$ thành hai cung có số đo bằng nhau nên\\ 
				sđ$\wideparen{AM}=\text{sđ}\wideparen{MC}=\dfrac{\text{sđ}\wideparen{AC}}{2}=45^{\circ}$.\\ 
				sđ$\wideparen{CN}=\text{sđ}\wideparen{NB}=\dfrac{\text{sđ}\wideparen{CB}}{2}=45^{\circ}$.\\ 
				Ta có $\widehat{ADM}$ là góc nội tiếp chắn cung $AM$, suy ra
				$$\widehat{ADM}=\dfrac{\text{sđ}\wideparen{AM}}{2}=\dfrac{45^{\circ}}{2}=22{,}5^{\circ}.$$
				Ta có $\widehat{NCB}$ là góc nội tiếp chắn cung $NB$, suy ra
				$$\widehat{NCB}=\dfrac{\text{sđ}\wideparen{BN}}{2}=\dfrac{45^{\circ}}{2}=22{,}5^{\circ}.$$
			\end{enumerate}
		}
\end{vd}
\subsection{Bài tập}
%Dạng 1
\begin{dang}
	{Tính số đo góc ở tâm và số đo cung bị chắn}
	\begin{enumerate}
		\item Đưa về cách tính số đo một góc của tam giác.
		\item Để tính số đo của cung nhỏ, ta tính số đo của góc ở tâm tương ứng.
		\item Để tính số đo của cung lớn, ta lấy $360^{\circ}$ trừ đi số đo của cung nhỏ.
		\item Sử dụng tỉ số lượng giác của một góc nhọn để tính góc.
	\end{enumerate}
\end{dang}
%Bài 1
\begin{bt}%[Dự án EX-9-Đề Cương Toán 9]%[Phạm Hoàng Việt]%[9H2H3-2]
	Tính số đo cung $AB$ nhỏ trong hình vẽ dưới đây, biết rằng $\widehat{AOC}=30^{\circ}$ và $\widehat{BOC}=80^{\circ}$
	\begin{center}
		\begin{tikzpicture}[scale=1, font=\footnotesize, line join=round, line cap=round, >=stealth]
			\coordinate (O) at (0,0);
			\coordinate (A) at ({2*cos(-50)},{2*sin(-50)});
			\coordinate (C) at ({2*cos(-20)},{2*sin(-20)});
			\coordinate (B) at ({2*cos(60)},{2*sin(60)});
			\draw (O) circle (2cm);
			\foreach \x/\y in {O/B,O/C,O/A}{
				\draw (\x)--(\y);
			}
			\foreach \x/\y in {O/180,A/-50,B/60,C/-20}{
				\draw[fill=black] (\x) circle (1pt) ($(\x)+(\y:3mm)$) node {$\x$};
			}
		\end{tikzpicture}
	\end{center}
	\loigiai{
		Điểm $C$ nằm trên cung nhỏ $AB$ nên sđ$\wideparen{AB}=\text{sđ}\wideparen{AC}+\text{sđ}\wideparen{BC}\quad (1)$.\\ 
		Góc ở tâm $\widehat{AOC}$ chắn cung $AC$ nên sđ$\wideparen{AC}=\widehat{AOC}=30^{\circ}$.\\ 
		Góc ở tâm $\widehat{BOC}$ chắn cung $BC$ nên sđ$\wideparen{BC}=\widehat{BOC}=80^{\circ}$.\\ 
		Thay vào $(1)$ ta được sđ$\wideparen{AB}=\text{sđ}\wideparen{AC}+\text{sđ}\wideparen{BC}=\widehat{AOC}+\widehat{BOC}=30^{\circ}+80^{\circ}=110^{\circ}$.
	}
\end{bt}
%Bài 2
\begin{bt}%[Dự án EX-9-Đề Cương Toán 9]%[Phạm Hoàng Việt]%[9H2H3-2]
	\immini{Cho hình vẽ
	\begin{enumerate}
		\item Tính số đo cung nhỏ $\wideparen{AB}$.
		\item Tính số đo cung nhỏ $\wideparen{AC}$.
		\item Tính số đo cung lớn $\wideparen{BC}$.
	\end{enumerate}}{\begin{tikzpicture}[scale=1, font=\footnotesize, line join=round, line cap=round, >=stealth]
			\coordinate (O) at (0,0);
			\coordinate (A) at ({2*cos(90)},{2*sin(90)});
			\coordinate (B) at ({2*cos(-20)},{2*sin(-20)});
			\coordinate (C) at ({2*cos(-150)},{2*sin(-150)});
			\draw (O) circle (2cm);
			\foreach \x/\y/\z in {C/A/O}{
				\path pic[draw,fill=orange!35,angle radius=10pt, pic text=$30^{\circ}$,
    				pic text options={shift={(-5pt,-20pt)}}]{angle= \x--\y--\z};}
    		\foreach \x/\y/\z in {B/O/A}{
				\path pic[draw,fill=orange!35,angle radius=10pt, pic text=$110^{\circ}$,
    				pic text options={shift={(10pt,10pt)}}]{angle= \x--\y--\z};}
			\foreach \x/\y in {O/B,O/A,A/C,O/C}{
				\draw (\x)--(\y);
			}
			\foreach \x/\y in {O/-90,A/90,B/-20,C/-150}{
				\draw[fill=black] (\x) circle (1pt) ($(\x)+(\y:3mm)$) node {$\x$};
			}
		\end{tikzpicture}}
	\loigiai{
		\begin{enumerate}
			\item Ta có sđ$\wideparen{AB}=\widehat{AOB}=110^{\circ}$ (góc ở tâm chắn cung $AB$).
			\item Xét $\triangle OAC$ có \\ 
			$OA=OC(=R)$\\ 
			$\text{Suy ra } \triangle OAC$ cân tại $O$.\\ 
			$\text{Suy ra } \widehat{ACO}=\widehat{CAO}=30^{\circ}$.
			Ta có $\widehat{AOC}+\widehat{ACO}+\widehat{CAO}=180^{\circ}$ (tổng ba góc trong một tam giác).\\ 
			$\text{Suy ra } \widehat{AOC}=180^{\circ}-30^{\circ}-30^{\circ}=120^{\circ}$.\\ 
			$\text{Suy ra }\text{sđ}\wideparen{AC}=\widehat{AOB}=120^{\circ}$.
			\item Ta có $\widehat{BOC}+\widehat{AOC}+\widehat{AOB}=360^{\circ}$.\\ 
			$\text{Suy ra } \widehat{BOC}=360^{\circ}-120^{\circ}-110^{\circ}=130^{\circ}$.
		\end{enumerate}
	}
\end{bt}
%Bài 3
\begin{bt}%[Dự án EX-9-Đề Cương Toán 9]%[Phạm Hoàng Việt]%[9H2H3-2]
	Cho đường tròn $(O;R)$. Vẽ dây $AB=R\sqrt{2}$. Tính số đo hai cung $AB$.
	\loigiai{
		\immini{Xét $\triangle OAB$ có\\ 
		$OA^2+OB^2=R^2+R^2=2R^2=AB^2$\\ 
		$\text{Suy ra } \triangle OAB$ vuông tại $O$.\\ 
		$\text{Suy ra } \text{sđ}\wideparen{AB}=90^{\circ}$.}{
			\begin{tikzpicture}[scale=1, font=\footnotesize, line join=round, line cap=round, >=stealth]
			\coordinate (O) at (0,0);
			\coordinate (A) at ({2*cos(-135)},{2*sin(-135)});
			\coordinate (B) at ({2*cos(-45)},{2*sin(-45)});
			\draw (O) circle (2cm);
			\draw (A)--(B) node[midway, above] {$R\sqrt{2}$};
			\foreach \x/\y in {O/B,O/A}{
				\draw (\x)--(\y);
			}
			\foreach \x/\y in {O/90,A/-135,B/-45}{
				\draw[fill=black] (\x) circle (1pt) ($(\x)+(\y:3mm)$) node {$\x$};
			}
		\end{tikzpicture}}
	}
\end{bt}
%Bài 4
\begin{bt}%[Dự án EX-9-Đề Cương Toán 9]%[Phạm Hoàng Việt]%[9H2V3-2]
	Cho đường tròn $(O;R)$. Vẽ dây $AB$ sao cho số đo cung nhỏ $\wideparen{AB}$ bằng nửa số đo cung lớn $\wideparen{AB}$. Tính diện tích tam giác $ABC$.
	\loigiai{
		\begin{center}
			\begin{tikzpicture}[scale=1, font=\footnotesize, line join=round, line cap=round, >=stealth]
				\coordinate (O) at (0,0);
				\coordinate (A) at ({2*cos(-150)},{2*sin(-150)});
				\coordinate (B) at ({2*cos(-30)},{2*sin(-30)});
				\coordinate (H) at ($(A)!0.5!(B)$);
				\draw (O) circle (2cm);
				\foreach \x/\y in {O/B,O/A,A/B,O/H}{
					\draw (\x)--(\y);
				}
				\foreach \x/\y in {O/90,A/-150,B/-30,H/-90}{
					\draw[fill=black] (\x) circle (1pt) ($(\x)+(\y:3mm)$) node {$\x$};
				}
			\end{tikzpicture}
		\end{center}
		Vì số đo cung nhỏ bằng nửa số đo cung lớn nên sđ$\wideparen{AB}_{\text{nhỏ}}=360^{\circ}:3=120^{\circ}$ suy ra $\widehat{AOB}=120^{\circ}$.\\ 
		$\triangle AOB$ cân tại $O$ nên $\widehat{OAB}=\widehat{OBA}=30^{\circ}$.\\
		Kẻ $OH$ vuông góc với $AB$, ta được $OH=OA\cdot \sin A=R\cdot \sin 30^{\circ}=\dfrac{1}{2}R$.\\ 
		$S_{\triangle AOB}=\dfrac{1}{2}AB\cdot OH=\dfrac{1}{2}\cdot R\sqrt{3}\cdot \dfrac{1}{2}R=\dfrac{R^2\sqrt{3}}{4}$.
	}
\end{bt}
%Bài 5
\begin{bt}%[Dự án EX-9-Đề Cương Toán 9]%[Phạm Hoàng Việt]%[9H2V3-2]
	Cho đường tròn $(O)$, hai tiếp tuyến của đường tròn tại $A$ và $B$ cắt nhau ở $M$, biết $\widehat{AMB}=65^{\circ}$.
	\begin{enumerate}
		\item Tính số đo $\widehat{AMO}$, $\widehat{AOM}$.
		\item Tính số đo góc ở tâm tạo bởi hai bán kính $OA$, $OB$.
		\item Tính số đo cung nhỏ $AB$ và số đo cung lớn $AB$.
	\end{enumerate}
	\loigiai{
		\begin{center}
			\begin{tikzpicture}[scale=1, font=\footnotesize, line join=round, line cap=round, >=stealth]
			\coordinate (O) at (0,0);
			\coordinate (A) at ({2*cos(60)},{2*sin(60)});
			\coordinate (B) at ({2*cos(-55)},{2*sin(-55)});
			\path let \p1 = (O), \p2 = (A) in coordinate (uOA) at (\x2 - \x1, \y2 - \y1);
			\path let \p1 = (uOA) in coordinate (nOA) at (-\y1, \x1);
			\path[name path=AM] ($(A)+2*(nOA)$)--($(A)-2*(nOA)$);
			\path let \p1 = (O), \p2 = (B) in coordinate (uOB) at (\x2 - \x1, \y2 - \y1);
			\path let \p1 = (uOB) in coordinate (nOB) at (-\y1, \x1);
			\path[name path=BM] ($(B)+2*(nOB)$)--($(B)-2*(nOB)$);
			\path[name intersections={of=AM and BM, by=M}];
			\foreach \x/\y/\z in {O/A/M,M/B/O}{
				\path pic[draw,fill=orange!35,angle radius=5
				pt]{right angle= \x--\y--\z};}
			\draw (O) circle (2cm);
			\foreach \x/\y in {O/B,O/A,A/M,B/M,O/M}{
				\draw (\x)--(\y);
			}
			\foreach \x/\y in {O/180,A/60,B/-55,M/0}{
				\draw[fill=black] (\x) circle (1pt) ($(\x)+(\y:3mm)$) node {$\x$};
			}
		\end{tikzpicture}
		\end{center}
		\begin{enumerate}
			\item Ta có $OM$ là tia phân giác của $\widehat{AMB}$ (định lí hai tiếp tuyến cắt nhau)\\
			$\text{Suy ra } \widehat{AMO}=32{,}5^{\circ}\text{Suy ra } \widehat{AOM}=180^{\circ}-90^{\circ}-32{,}5^{\circ}=57{,}5^{\circ}$.
			\item $\widehat{AOB}=360^{\circ}-180^{\circ}-65^{\circ}=115^{\circ}$.
			\item sđ$\wideparen{AB}_\text{nhỏ}=\widehat{AOB}=115^{\circ}$.\\ 
			sđ$\wideparen{AB}_\text{lớn}=360^{\circ}-115^{\circ}=245^{\circ}$. 
		\end{enumerate}
	}
\end{bt}
%Bài 6
\begin{bt}%[Dự án EX-9-Đề Cương Toán 9]%[Phạm Hoàng Việt]%[9H2V3-2]
	Cho hai tiếp tuyến tại $A$ và $B$ của đường tròn $(O)$ cắt nhau tại $M$, biết $\widehat{AMB}=50^{\circ}$. Tính số đo cung $\wideparen{AB}$ nhỏ và số đo cung $\wideparen{AB}$ lớn.
	\loigiai{
		\begin{center}
			\begin{tikzpicture}[scale=1, font=\footnotesize, line join=round, line cap=round, >=stealth]
			\coordinate (O) at (0,0);
			\coordinate (A) at ({2*cos(65)},{2*sin(65)});
			\coordinate (B) at ({2*cos(-65)},{2*sin(-65)});
			\path let \p1 = (O), \p2 = (A) in coordinate (uOA) at (\x2 - \x1, \y2 - \y1);
			\path let \p1 = (uOA) in coordinate (nOA) at (-\y1, \x1);
			\path[name path=AM] ($(A)+2*(nOA)$)--($(A)-2*(nOA)$);
			\path let \p1 = (O), \p2 = (B) in coordinate (uOB) at (\x2 - \x1, \y2 - \y1);
			\path let \p1 = (uOB) in coordinate (nOB) at (-\y1, \x1);
			\path[name path=BM] ($(B)+2*(nOB)$)--($(B)-2*(nOB)$);
			\path[name intersections={of=AM and BM, by=M}];
			\foreach \x/\y/\z in {O/A/M,M/B/O}{
				\path pic[draw,fill=orange!35,angle radius=5
				pt]{right angle= \x--\y--\z};}
			\draw (O) circle (2cm);
			\foreach \x/\y in {O/B,O/A,A/M,B/M,O/M,A/B}{
				\draw (\x)--(\y);
			}
			\foreach \x/\y in {O/180,A/65,B/-65,M/0}{
				\draw[fill=black] (\x) circle (1pt) ($(\x)+(\y:3mm)$) node {$\x$};
			}
		\end{tikzpicture}
		\end{center}
		Vì $MA$, $MB$ là hai tiếp tuyến của đường tròn $(O)$ nên $OM$ là tia phân giác của $\widehat{AOB}$, $MO$ là tia phân giác của $\widehat{AMB}$.\\ 
		Hay $\widehat{AMO}=\dfrac{1}{2}\widehat{AMB}=\dfrac{50^{\circ}}{2}=25^{\circ}$.\\ 
		Mà tam giác $OAM$ vuông tại $A$ (do $MA$ là tiếp tuyến) nên $\widehat{MOA}=90^{\circ}-\widehat{AMO}=65^{\circ}$.\\ 
		Mà $OM$ là tia phân giác của $\widehat{AOB}$ nên $\widehat{MOB}=\widehat{MOA}=65^{\circ}$.\\ 
		Suy ra $\widehat{AOB}=\widehat{MOB}+\widehat{MOA}=65^{\circ}+65^{\circ}=130^{\circ}$.
		Nên số đo cung nhỏ $\wideparen{AB}$ là $130^{\circ}$.\\ 
		Số đo cung lớn $\wideparen{AB}$ là $360^{\circ}-130^{\circ}=230^{\circ}$.
	}
\end{bt}
%Bài 7
\begin{bt}%[Dự án EX-9-Đề Cương Toán 9]%[Phạm Hoàng Việt]%[9H2V3-2]
	Trên cung nhỏ $\wideparen{AB}$ của $(O)$, cho hai điểm $C$ và $D$ sao cho cung $\wideparen{AB}$ được chia thành ba cung bằng nhau $\left(\wideparen{AC}=\wideparen{CD}=\wideparen{DB} \right)$. Bán kính $OC$ và $OD$ cắt dây $AB$ lần lượt tại $E$ và $F$. So sánh các đoạn thẳng $AE$ và $BF$.
	\loigiai{
		\immini{Xét $\triangle OAB$ có\\ 
		$OA=OB(=R)$\\ 
		$\text{Suy ra } \triangle OAB$ cân tại $O$ và $\widehat{OAE}=\widehat{OBF}$.\\ 
		Ta có $\wideparen{AC}=\wideparen{DB}$.\\ 
		$\text{Suy ra } \widehat{AOE}=\widehat{BOF}$.\\ 
		Xét $\triangle OAE$ và $\triangle OBF$ có\\ 
		$\widehat{OAE}=\widehat{OBF}$ (chứng minh trên)\\ 
		$OA=OB(=R)$\\
		$\widehat{AOE}=\widehat{BOF}$ (chứng minh trên)\\ 
		$\text{Suy ra } \triangle OAE=\triangle OBF$ (g-c-g).\\ 
		$\text{Suy ra } AE=BF$ (hai cạnh tương ứng).}{\begin{tikzpicture}[scale=1, font=\footnotesize, line join=round, line cap=round, >=stealth]
			\coordinate (O) at (0,0);
			\coordinate (A) at ({2*cos(-150)},{2*sin(-150)});
			\coordinate (B) at ({2*cos(-30)},{2*sin(-30)});
			\coordinate (C) at ({2*cos(-110)},{2*sin(-110)});
			\coordinate (D) at ({2*cos(-70)},{2*sin(-70)});
			\path[name path=AB] (A)--(B);
			\path[name path=OC] (O)--(C);
			\path[name path=OD] (O)--(D);
			\path[name intersections={of=AB and OC, by=E}];
			\path[name intersections={of=AB and OD, by=F}];
			\draw (O) circle (2cm);
			\foreach \x/\y in {O/B,O/A,A/B,O/C,O/D}{
				\draw (\x)--(\y);
			}
			\foreach \x/\y in {O/180,A/-150,B/-30,C/-110,D/-70,E/-135,F/-45}{
				\draw[fill=black] (\x) circle (1pt) ($(\x)+(\y:3mm)$) node {$\x$};
			}
		\end{tikzpicture}}
	}
\end{bt}
%Dạng 2
\begin{dang}
	{Tính số đo góc nội tiếp và số đo cung bị chắn}
	Trong một đường tròn
	\begin{enumerate}
		\item Các góc nội tiếp bằng nhau chắn các cung bằng nhau và ngược lại.
		\item Các góc nội tiếp cùng chắn một cung hoặc chắn các cung bằng nhau thì bằng nhau.
		\item Góc nội tiếp (nhỏ hơn hoặc bằng $90^{\circ}$) có số đo bằng nửa số đo của góc ở tâm cùng chắn một cung.
		\item Góc nội tiếp chắn nửa đường tròn là góc vuông.
	\end{enumerate}
\end{dang}
\setcounter{bt}{0}
%Bài 1
\begin{bt}%[Dự án EX-9-Đề Cương Toán 9]%[Phạm Hoàng Việt]%[9H2H3-3]
	Dựa vào hình vẽ sau
	\begin{center}
		\begin{tikzpicture}[scale=1, font=\footnotesize, line join=round, line cap=round, >=stealth]
			\coordinate (O) at (0,0);
			\coordinate (A) at ({2*cos(-15)},{2*sin(-15)});
			\coordinate (D) at ({2*cos(-165)},{2*sin(-165)});
			\coordinate (C) at ({2*cos(165)},{2*sin(165)});
			\coordinate (B) at ({2*cos(105)},{2*sin(105)});
			\foreach \x/\y/\z in {B/A/C}{
				\path pic[draw,fill=orange!35,angle radius=10pt, pic text=$30^{\circ}$,
				pic text options={shift={(-20pt,12pt)}}]{angle= \x--\y--\z};}
			\foreach \x/\y/\z in {D/O/A}{
				\path pic[pic text=$150^{\circ}$,
				pic text options={shift={(0,0pt)}}]{angle= \x--\y--\z};}
			\draw (O) circle (2cm);
			\foreach \x/\y in {A/B,A/D,A/C,O/D}{
				\draw (\x)--(\y);
			}
			\foreach \x/\y in {O/90,A/-15,B/105,C/165,D/-165}{
				\draw[fill=black] (\x) circle (1pt) ($(\x)+(\y:3mm)$) node {$\x$};
			}
		\end{tikzpicture}
	\end{center}
	\begin{enumerate}
		\item Tính số đo cung nhỏ $\wideparen{CD}$.
		\item Tính số đo cung nhỏ $\wideparen{BD}$.
	\end{enumerate}
	\loigiai{
		\begin{enumerate}
		\item Ta có $\widehat{COD}=180^{\circ}-\widehat{AOD}$ (hai góc bù nhau).\\ 
		$\widehat{COD}=180^{\circ}-\widehat{AOD}=180^{\circ}-150^{\circ}=30^{\circ}$.\\
		Số đo cung nhỏ $\wideparen{CD}$ là sđ$\wideparen{CD}=\widehat{COD}=30^{\circ}$ (góc ở tâm).
		\item Số đo cung nhỏ $\wideparen{BC}$ là sđ$\wideparen{BC}=2\widehat{CAB}=2\cdot 30^{\circ}=60^{\circ}$ (góc nội tiếp).\\ 
		Số đo cung nhỏ $\wideparen{BD}$ là sđ$\wideparen{BD}=\text{sđ}\wideparen{BC}+\text{sđ}\wideparen{CD}=60^{\circ}+30^{\circ}=90^{\circ}$.
	\end{enumerate}
	}
\end{bt}
%Bài 2
\begin{bt}%[Dự án EX-9-Đề Cương Toán 9]%[Phạm Hoàng Việt]%[9H2H3-3]
	Dựa vào hình vẽ sau, biết số đo cung nhỏ $\wideparen{XY}$ của đường tròn tâm $B$ là $70^{\circ}$. Tính $\widehat{MON}$.
	\begin{center}
		\begin{tikzpicture}[scale=1, font=\footnotesize, line join=round, line cap=round, >=stealth]
			\coordinate (O) at (0,0);
			\coordinate (M) at ({2*cos(110)},{2*sin(110)});
			\coordinate (N) at ({2*cos(250)},{2*sin(250)});
			\coordinate (B) at ({2*cos(0)},{2*sin(0)});
			\path[name path=circleB] (B) circle (1cm);
			\path[name path=BM] (B)--(M);
			\path[name path=BN] (B)--(N);
			\path[name intersections={of=BM and circleB, by=X}];
			\path[name intersections={of=BN and circleB, by=Y}];
			\foreach \x/\y/\z in {M/O/N}{
				\path pic[draw,fill=orange!35,angle radius=10pt]{angle= \x--\y--\z};}
			\foreach \x/\y/\z in {}{
				\path pic[pic text=$150^{\circ}$,
				pic text options={shift={(0,0pt)}}]{angle= \x--\y--\z};}
			\draw (O) circle (2cm);
			\draw (B) circle (1cm);
			\foreach \x/\y in {O/M,O/N,B/M,B/N}{
				\draw (\x)--(\y);
			}
			\foreach \x/\y in {O/0,M/110,B/0,N/250,Y/180,X/180}{
				\draw[fill=black] (\x) circle (1pt) ($(\x)+(\y:3mm)$) node {$\x$};
			}
		\end{tikzpicture}
	\end{center}
	\loigiai{
		Ta có $\widehat{XBY}=\text{sđ}\wideparen{XY}=70^{\circ}$ (góc ở tâm).\\ 
		Ta lại có $\widehat{MON}=\text{sđ}\wideparen{MBN}=2\cdot \widehat{MBN}=2\cdot \widehat{XBY}=2\cdot 70^{\circ}=140^{\circ}$.
	}
\end{bt}
%Bài 3
\begin{bt}%[Dự án EX-9-Đề Cương Toán 9]%[Phạm Hoàng Việt]%[9H2H3-3]
	Dựa vào hình vẽ sau, hãy tính $\widehat{BAC}$.
	\begin{center}
		\begin{tikzpicture}[scale=1, font=\footnotesize, line join=round, line cap=round, >=stealth]
			\coordinate (O) at (0,0);
			\coordinate (A) at ({2*cos(90)},{2*sin(90)});
			\coordinate (B) at ({2*cos(220)},{2*sin(220)});
			\coordinate (D) at ({2*cos(-10)},{2*sin(-10)});
			\coordinate (C) at ({2*cos(-40)},{2*sin(-40)});
			%\path (C)--(D) node[midway, below right] {$30^{\circ}$};
			\foreach \x/\y/\z in {A/O/B}{
				\path pic[pic text=$130^{\circ}$,
				pic text options={shift={(-0.1,0pt)}}]{angle= \x--\y--\z};}
			\foreach \x/\y/\z in {A/D/O}{
				\path pic[pic text=$40^{\circ}$,
				pic text options={shift={(-0.3,4pt)}}]{angle= \x--\y--\z};}
			\foreach \x/\y/\z in {C/O/D}{
				\path pic[pic text=$30^{\circ}$,
				pic text options={shift={(2,-25pt)}}]{angle= \x--\y--\z};}
			\draw (O) circle (2cm);
			\foreach \x/\y in {O/A,O/B,O/D,O/C,A/D,A/C,A/B}{
				\draw (\x)--(\y);
			}
			\foreach \x/\y in {O/-100,A/90,B/220,D/-10,C/-40}{
				\draw[fill=black] (\x) circle (1pt) ($(\x)+(\y:3mm)$) node {$\x$};
			}
		\end{tikzpicture}
	\end{center}
	\loigiai{
		Xét $\triangle AOD$ có\\ 
		$OA=OD$ (bán kính (O)).\\ 
		$\text{Suy ra } \triangle AOD$ cân tại $O$.\\ 
		$\text{Suy ra } \widehat{OAD}=\widehat{ADO}=40^{\circ}$.\\ 
		Ta có $\widehat{AOD}=180^{\circ}-\widehat{OAD}-\widehat{ADO}=180^{\circ}-40^{\circ}-40^{\circ}=100^{\circ}$.\\ 
		Mà sđ$\wideparen{AD}=\widehat{AOD}=100^{\circ}$ (góc ở tâm).\\ 
		sđ$\wideparen{AB}=\widehat{AOB}=130^{\circ}$ (góc ở tâm).\\ 
		Ta có sđ$\wideparen{BC}=360^{\circ}-\text{sđ}\wideparen{AB}-\text{sđ}\wideparen{AD}-\text{sđ}\wideparen{DC}=360^{\circ}-130^{\circ}-100^{\circ}-30^{\circ}=100^{\circ}$.
		Vì $\widehat{BAC}$ là góc nội tiếp chắn cung $BC$ nên $\widehat{BAC}=\dfrac{1}{2}\cdot\text{sđ}\wideparen{BC}=\dfrac{1}{2}\cdot 100^{\circ}=50^{\circ}$. 
	}
\end{bt}
%Dạng 3
\begin{dang}
	{Chứng minh các góc bằng nhau, các cung bằng nhau}
	Để chứng minh hai cung (của một đường tròn) bằng nhau, ta chứng minh hai cung này có cùng một số đo.
	\begin{luuy}
		Trong một đường tròn
		\begin{enumerate}
			\item Hai cung bị chắn giữa hai dây song song thì bằng nhau.
			$$AB\parallel CD \text{ suy ra } \wideparen{AC}=\wideparen{BD}.$$
			\item Các góc nội tiếp bằng nhau chắn các cung bằng nhau và ngược lại.
			\item Các góc nội tiếp cùng chắn một cung hoặc chắn các cung bằng nhau thì bằng nhau.
			\item Góc nội tiếp (nhỏ hơn hoặc bằng $90^{\circ}$) có số đo bằng nửa số đo của góc ở tâm cùng chắn một cung.
			\item Góc nội tiếp chắn nửa đường tròn là góc vuông.
		\end{enumerate}
	\end{luuy}
\end{dang}
\setcounter{bt}{0}
%Bài 1
\begin{bt}%[Dự án EX-9-Đề Cương Toán 9]%[Phạm Hoàng Việt]%[9H2V3-3]
	Cho $\triangle ACB$ cân tại $A$ $\left(\widehat{A}<90^{\circ}\right)$. Vẽ đường tròn đường kính $AB$ cắt $BC$ tại $D$, cắt $AC$ tại $E$.
	\begin{enumerate}
		\item Chứng minh $\wideparen{BD}=\wideparen{DE}$.
		\item Chứng minh $\widehat{CBE}=\dfrac{1}{2}\widehat{BAC}$.
	\end{enumerate}
	\loigiai{
		\begin{center}
			\begin{tikzpicture}[scale=1, font=\footnotesize, line join=round, line cap=round, >=stealth]
				\coordinate (B) at (0,0);
				\coordinate (A) at ($(B)+(2,3)$);
				\coordinate (C) at ($(B)+(4,0)$);
				\coordinate (I) at ($(A)!0.5!(B)$);
				%\path (C)--(D) node[midway, below right] {$30^{\circ}$};
				\path[name path=circleI] let \p1 = ($(A)-(I)$), \n1={veclen(\p1)} in (I) circle (\n1);
				\path[name path=BC] ($(B)!0.1!(C)$)--(C);
				\path[name path=AC] ($(A)!0.1!(C)$)--(C);
				\path[name intersections={of=BC and circleI, by=D}];
				\path[name intersections={of=AC and circleI, by=E}];
				\foreach \x/\y/\z in {}{
					\path pic[pic text=$130^{\circ}$,
					pic text options={shift={(-0.1,0pt)}}]{angle= \x--\y--\z};}
				\foreach \x/\y/\z in {}{
					\path pic[pic text=$40^{\circ}$,
					pic text options={shift={(-0.3,4pt)}}]{angle= \x--\y--\z};}
				\foreach \x/\y/\z in {}{
					\path pic[pic text=$30^{\circ}$,
					pic text options={shift={(2,-25pt)}}]{angle= \x--\y--\z};}
				\draw let \p1 = ($(A)-(I)$), \n1={veclen(\p1)} in (I) circle (\n1);
				\foreach \x/\y in {A/B,A/D,A/C,B/E,D/E,B/C}{
					\draw (\x)--(\y);
				}
				\foreach \x/\y in {B/-100,A/90,D/-90,C/0,E/0}{
					\draw[fill=black] (\x) circle (1pt) ($(\x)+(\y:3mm)$) node {$\x$};
				}
			\end{tikzpicture}
		\end{center}
		\begin{enumerate}
			\item Ta có $\widehat{ADB}=90^{\circ}$ (góc nội tiếp chắn nửa đường tròn).\\ 
			Suy ra $AD\perp BC$, hay $AD$ là phân giác của $\widehat{BAC}$.\\ 
			Nên $\widehat{BAD}=\widehat{DAC}$.\\ 
			Suy ra $\wideparen{BD}=\wideparen{DE}$.
			\item Ta có $\widehat{CBE}=\widehat{DAE}=\dfrac{1}{2}\widehat{BAC}$.
		\end{enumerate}
	}
\end{bt}
%Bài 2
\begin{bt}%[Dự án EX-9-Đề Cương Toán 9]%[Phạm Hoàng Việt]%[9H2V3-3]
	Cho nửa đường tròn $(O)$ đường kính $AB$ và dây $AC$ căng cung $AC$ có số đo bằng $60^{\circ}$.
	\begin{enumerate}
		\item So sánh các góc của $\triangle ABC$.
		\item Gọi $M$ và $N$ lần lượt là điểm chính giữa của các cung $AC$ và $BC$, hai dây $AN$ và $BM$ cắt nhau tại $I$. Chứng minh rằng $CI$ là tia phân giác của $\widehat{ACB}$.
	\end{enumerate}
	\loigiai{
		\begin{center}
			\begin{tikzpicture}[scale=1, font=\footnotesize, line join=round, line cap=round, >=stealth]
				\coordinate (O) at (0,0);
				\coordinate (A) at (-2,0);
				\coordinate (B) at (2,0);
				\coordinate (C) at ({2*cos(120)},{2*sin(120)});
				\coordinate (M) at ({2*cos(150)},{2*sin(150)});
				\coordinate (N) at ({2*cos(60)},{2*sin(60)});
				\draw[domain=0:180] plot ({2*cos(\x)},{2*sin(\x)});
				\path[name path=AN] (A)--(N);
				\path[name path=BM] (B)--(M);
				\path[name intersections={of=AN and BM, by=I}];
				\path[name path=CI] (C)--($(C)!2!(I)$);
				\path[name path=AB] (A)--(B);
				\path[name intersections={of=CI and AB, by=D}];
				\foreach \x/\y in {A/B,A/C,C/B,A/N,B/M,C/D}{
					\draw (\x)--(\y);
				}
				\foreach \x/\y in {A/180,B/0,O/-90,M/150,C/120,N/60,I/-120}{
					\draw[fill=black] (\x) circle (1pt) ($(\x)+(\y:3mm)$) node {$\x$};
				}
			\end{tikzpicture}
		\end{center}
		\begin{enumerate}
			\item Ta có sđ$\wideparen{AC}=60^{\circ}$ suy ra sđ$\wideparen{BC}=120^{\circ}$.\\
			Suy ra $\widehat{ABC}<\widehat{BAC}<\widehat{ACB}$.
			\item Ta có $AN$ là phân giác của góc $\widehat{BAC}$, $BM$ là phân giác của góc $\widehat{ABC}$ suy ra $CI$ là phân giác của $\widehat{ACB}$.
		\end{enumerate}
	}
\end{bt}
%Bài 3
\begin{bt}%[Dự án EX-9-Đề Cương Toán 9]%[Phạm Hoàng Việt]%[9H2V3-3]
	Cho đường tròn $\left(O;R\right)$, trên đường tròn lấy ba điểm $A$, $B$, $C$ sao cho $AB<AC$. Vẽ $AD$ là đường cao của tam giác $ABC$, $AE$ là đường kính của đường tròn $\left(O;R\right)$. Chứng minh rằng $\widehat{BAD}=\widehat{OAC}$.
	\loigiai{
		\begin{center}
			\begin{tikzpicture}[scale=1, font=\footnotesize, line join=round, line cap=round, >=stealth]
				\coordinate (O) at (0,0);
				\coordinate (A) at ({2*cos(120)},{2*sin(120)});
				\coordinate (B) at ({2*cos(-160)},{2*sin(-160)});
				\coordinate (C) at ({2*cos(-20)},{2*sin(-20)});
				\coordinate (E) at ({2*cos(-60)},{2*sin(-60)});
				\path let \p1 = (B), \p2 = (C) in coordinate (uBC) at (\x2 - \x1, \y2 - \y1);
				\path let \p1 = (uBC) in coordinate (nBC) at (-\y1, \x1);
				\path[name path=AD] (A)--($(A)-(nBC)$);
				\path[name path=BC] (B)--(C);
				\path[name intersections={of=AD and BC, by=D}];
				\draw (O) circle (2cm);
				\foreach \x/\y/\z in {A/C/E,A/D/B}{
					\path pic[draw,fill=orange!35,angle radius=5pt]{right angle= \x--\y--\z};}
				\foreach \x/\y in {A/B,A/D,A/C,A/E,B/C,C/E}{
					\draw (\x)--(\y);
				}
				\foreach \x/\y in {B/-160,A/120,C/-20,E/-60,D/-90,O/210}{
					\draw[fill=black] (\x) circle (1pt) ($(\x)+(\y:3mm)$) node {$\x$};
				}
			\end{tikzpicture}
		\end{center}
		Ta có $\widehat{ACE}=90^{\circ}$ (góc nội tiếp chắn nửa đường tròn).\\ 
		Xét $\triangle DBA$ và $\triangle CEA$ có\\ 
		$\widehat{ADB}=\widehat{ACE}(=90^{\circ})$.\\ 
		$\widehat{ABD}=\widehat{AEC}$ (cùng chắn cung $AC$).
		Suy ra $\triangle DBA\backsim \triangle CEA$.\\ 
		Từ đó suy ra $\widehat{BAD}=\widehat{OAC}$. 
	}
\end{bt}
%Dạng 4
\begin{dang}
	{Chứng minh hai đường thẳng song song, vuông góc, ba điểm thẳng hàng}
\end{dang}
\setcounter{bt}{0}
%Bài 1
\begin{bt}%[Dự án EX-9-Đề Cương Toán 9]%[Phạm Hoàng Việt]%[9H2V3-3]
	Cho đường tròn $(O)$ đường kính $AB$, điểm $D$ thuộc $(O)$. Gọi $E$ là điểm đối xứng với $A$ qua $D$.
	\begin{enumerate}
		\item $\triangle ABE$ là tam giác gì?
		\item Gọi $K$ là giao điểm của $EB$ với $(O)$. Chứng minh $OD\perp AK$.
	\end{enumerate}
	\loigiai{
		\begin{center}
			\begin{tikzpicture}[scale=1, font=\footnotesize, line join=round, line cap=round, >=stealth]
				\coordinate (O) at (0,0);
				\coordinate (A) at ({2*cos(180)},{2*sin(180)});
				\coordinate (B) at ({2*cos(0)},{2*sin(0)});
				\coordinate (D) at ({2*cos(135)},{2*sin(135)});
				\coordinate (E) at ($(A)!2!(D)$);
				\path[name path=EB] (E)--($(E)!0.9!(B)$);
				\path[name path=circleO] (O) circle (2cm);
				\path[name intersections={of=EB and circleO, by=K}];
				\draw (O) circle (2cm);
				\foreach \x/\y in {A/B,A/E,E/B,A/K,B/D,O/D}{
					\draw (\x)--(\y);
				}
				\foreach \x/\y in {A/180,B/0,O/-90,D/135,K/60,E/90}{
					\draw[fill=black] (\x) circle (1pt) ($(\x)+(\y:3mm)$) node {$\x$};
				}
			\end{tikzpicture}
		\end{center}
		\begin{enumerate}
			\item $\widehat{ADB}=90^{\circ}$ (góc nội tiếp chắn nửa đường tròn).\\ 
			Suy ra $BD\perp AE$.\\ 
			Mà $AD=DE$.\\ 
			Suy ra $\triangle ABE$ cân tại $B$.
			\item Ta có $\widehat{AKB}=90^{\circ}$ (góc nội tiếp chắn nửa đường tròn).\\ 
			Suy ra $AK\perp BE\quad (1)$.\\
			Mặt khác, $D$ là trung điểm $AE$, $O$ là trung điểm $AB$.\\ 
			Suy ra $OD$ là đường trung bình của $\triangle ABE$.\\ 
			Suy ra $OD\parallel BE\quad (2)$.\\ 
			Từ $(1)$ và $(2)$ suy ra $OD\perp AK$.
		\end{enumerate}
	}
\end{bt}
%Bài 2
\begin{bt}%[Dự án EX-9-Đề Cương Toán 9]%[Phạm Hoàng Việt]%[9H2V3-3]
	Cho nửa đường tròn $(O)$ đường kính $AB$ và điểm $C$ nằm ngoài nửa đường tròn. $AC$ cắt nửa đường tròn tại $M$, $BC$ cắt nửa đường tròn tại $N$. Gọi $H$ là giao điểm của $AN$ và $BM$. Chứng minh rằng $CH\perp AB$.
	\loigiai{
		\begin{center}
			\begin{tikzpicture}[scale=1, font=\footnotesize, line join=round, line cap=round, >=stealth]
				\coordinate (O) at (0,0);
				\coordinate (A) at ({2*cos(180)},{2*sin(180)});
				\coordinate (B) at ({2*cos(0)},{2*sin(0)});
				\coordinate (C) at ($(A)+(1.5,4)$);
				\path[name path=circleO] (O) circle (2cm);
				\path[name path=AC] ($(A)!0.1!(C)$)--(C);
				\path[name path=BC] ($(B)!0.1!(C)$)--(C);
				\path[name intersections={of=AC and circleO, by=M}];
				\path[name intersections={of=BC and circleO, by=N}];
				\path[name path=AN] (A)--(N);
				\path[name path=BM] (B)--(M);
				\path[name intersections={of=AN and BM, by=H}];
				\draw[domain=0:180] plot ({2*cos(\x)},{2*sin(\x)});
				\foreach \x/\y in {A/B,A/C,B/C,A/N,B/M,C/H}{
					\draw (\x)--(\y);
				}
				\foreach \x/\y in {A/180,B/0,O/-90,C/90,M/135,N/45,H/60}{
					\draw[fill=black] (\x) circle (1pt) ($(\x)+(\y:3mm)$) node {$\x$};
				}
			\end{tikzpicture}
		\end{center}
		Ta có $\widehat{AMB}=90^{\circ}$ (góc nội tiếp chắn nửa đường tròn).\\ 
		Suy ra $BM\perp AC$.\\ 
		Ta có $\widehat{ANB}=90^{\circ}$ (góc nội tiếp chắn nửa đường tròn).\\ 
		Suy ra $AN\perp BC$.\\ 
		Suy ra $H$ là trực tâm $\triangle ABC$.\\ 
		Suy ra $CH\perp AB$.
	}
\end{bt}
%Bài 3
\begin{bt}%[Dự án EX-9-Đề Cương Toán 9]%[Phạm Hoàng Việt]%[9H2V3-3]
	Cho đường tròn $(O)$, đường kính $AF$, điểm $B$ và $C$ thuộc đường tròn. Hai tam giác $BD$ và $CE$ của $\triangle ABC$ cắt nhau tại $H$.
	\begin{enumerate}
		\item Tứ giác $BFCH$ là hình gì?
		\item Gọi $M$ là trung điểm của $BC$. Chứng minh rằng ba điểm $H$, $M$, $F$ thẳng hàng.
		\item Chứng minh $OM=\dfrac{1}{2}AH$.
	\end{enumerate}
	\loigiai{
		\begin{center}
			\begin{tikzpicture}[scale=1, font=\footnotesize, line join=round, line cap=round, >=stealth]
				\coordinate (O) at (0,0);
				\coordinate (A) at ({2*cos(120)},{2*sin(120)});
				\coordinate (B) at ({2*cos(-160)},{2*sin(-160)});
				\coordinate (C) at ({2*cos(-20)},{2*sin(-20)});
				\coordinate (F) at ({2*cos(-60)},{2*sin(-60)});
				\coordinate (M) at ($(B)!0.5!(C)$);
				\path[name path=circleM] let \p1= ($(B)-(M)$), \n1={veclen(\p1)} in (M) circle (\n1);
				\path[name path=AB] (A)--($(B)!0.1!(A)$);
				\path[name path=AC] (A)--($(C)!0.1!(A)$);
				\path[name intersections={of=AB and circleM, by=E}];
				\path[name intersections={of=AC and circleM, by=D}];
				\path[name path=BD] (B)--(D);
				\path[name path=CE] (C)--(E);
				\path[name intersections={of=BD and CE, by=H}];
				\foreach \x/\y/\z in {B/E/C,B/D/C}{
					\path pic[draw,fill=orange!35,angle radius=5pt]{right angle= \x--\y--\z};}
				\draw (O) circle (2cm);
				\foreach \x/\y in{A/B,A/C,B/C,A/F,B/D,C/E,B/F,C/F,F/H}{
					\draw (\x)--(\y);
				}
				\foreach \x/\y in {A/120,B/-160,C/-20,F/-60,E/145,D/45,H/20,O/45,M/-100}{
					\draw[fill=black] (\x) circle (1pt) ($(\x)+(\y:3mm)$) node {$\x$};
				}
				\draw (O)--(M) (A)--(H);
			\end{tikzpicture}
		\end{center}
		\begin{enumerate}
			\item Ta có $\widehat{ABF}=90^{\circ}$ (góc nội tiếp chắn nửa đường tròn).\\ 
			Suy ra $BF\perp AB$ tại $B$.\\ 
			Mà $CH\perp AB$ tại $E$.\\ 
			Suy ra $CH\parallel BF$.\\ 
			Tương tự, ta có $\widehat{ACF}=90^{\circ}$ (góc nội tiếp chắn nửa đường tròn).\\ 
			Suy ra $CF\perp AC$ tại $C$.\\ 
			Mà $BH\perp AC$ tại $D$.\\ 
			Suy ra $BH\parallel CF$.\\ 	
			Xét tứ giác $BFCH$ có \\ 
			$CH\parallel BF$ và $BH\parallel CF$.\\ 
			Suy ra $BFCH$ là hình bình hành.
			\item Vì $BHCF$ là hình bình hành và $M$ là trung điểm $BC$ nên $M$ là trung điểm của $HF$.\\ 
			Suy ra $H$, $M$, $F$ thẳng hàng.
			\item Xét $\triangle AHF$ có\\ 
			$M$ là trung điểm của $HF$.\\ 
			$O$ là trung điểm của $AF$.\\ 
			Suy ra $OM$ là đường trung bình của $\triangle AHF$.\\ 
			Suy ra $OM=\dfrac{1}{2}AH$.
		\end{enumerate}
	}
\end{bt}