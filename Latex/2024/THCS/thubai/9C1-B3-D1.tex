\section{GIẢI HỆ HAI PHƯƠNG TRÌNH BẬC NHẤT HAI ẨN} % Tên bài
\subsection{Giải hệ phương tình bằng phương pháp thế}
\subsubsection{Kiến thức trọng tâm}
\begin{tomtat}
	Tổng quát, để giải hệ hai phương trình bậc nhất hai ẩn bằng phương pháp thế, ta thực hiện các bước như sau
	\begin{itemize}
		\item \textit{Bước $1$.} Từ một phương trình của hệ, ta biểu diễn ẩn này theo ẩn kia, rồi thế vào phương trình còn lại của hệ để nhận được một phương trình một ẩn.
		\item \textit{Bước $2$.} Giải phương trình một ẩn đó rồi suy ra nghiệm của hệ.
	\end{itemize}
\end{tomtat}
\begin{vd}%[Dự án EX-9-Đề Cương Toán 9]%[Nguyễn Cường]%[9D1H3-1]
	Giải hệ phương trình $\heva{&2x+y=3\\&x-2y=4.}$
	\loigiai{
		Ta có
		\allowdisplaybreaks
		\begin{eqnarray*}
			&&\heva{&2x+y=3\\&x-2y=4}\\
			&&\heva{&y=3-2x\\&x-2(3-2x)=4}\\
			&&\heva{&y=3-2x\\&x-6+4x=4}\\
			&&\heva{&y=3-2x\\&5x=10}\\
			&&\heva{&x=2\\&y=-1.}
		\end{eqnarray*}
		Vậy hệ phương trình có nghiệm duy nhất là $(2;-1)$.
	}
\end{vd}
\begin{vd}%[Dự án EX-9-Đề Cương Toán 9]%[Nguyễn Cường]%[9D1H3-1]
	Giải hệ phương trình $\heva{&x-3y=-5\\&2x+y=4.}$
	\loigiai{
		Ta có
		\allowdisplaybreaks
		\begin{eqnarray*}
			&&\heva{&x-3y=-5\\&2x+y=4}\\
			&&\heva{&x=3y-5\\&2(3y-5)+y=4}\\
			&&\heva{&x=3y-5\\&6y-10+y=4}\\
			&&\heva{&x=3y-5\\&7y=14}\\
			&&\heva{&x=1\\&y=2.}
		\end{eqnarray*}
		Vậy hệ phương trình có nghiệm duy nhất là $(1;2)$.
	}
\end{vd}
\begin{vd}%[Dự án EX-9-Đề Cương Toán 9]%[Nguyễn Cường]%[9D1H3-1]
	Giải các hệ phương trình:
\begin{multicols}{2}
		\begin{enumerate}
		\item $\heva{&2x+y=1\\&4x+2y=2.}$
		\item $\heva{&x-2y=4\\&2x-4y=1.}$
	\end{enumerate}
\end{multicols}
	\loigiai{
		\begin{enumerate}
			\item Ta có
			\allowdisplaybreaks
			\begin{eqnarray*}
				&&\heva{&2x+y=1\\&4x+2y=2}\\
				&&\heva{&y=1-2x\\&4x+2(1-2x)=2}\\
				&&\heva{&y=1-2x\\&0x=0.}
			\end{eqnarray*}
			Phương trình $0x=0$ nghiệm đúng với mọi $x\in\mathbb{R}$.\\
			Vậy hệ phương trình có vô số nghiệm. Các nghiệm của hệ được viết như sau: $\heva{
				&x\in\mathbb{R}\\ 
				&y=1-2x.\\}$
			\item Ta có
			\allowdisplaybreaks
			\begin{eqnarray*}
				&&\heva{&x-2y=4\\&2x-4y=1}\\
				&&\heva{&x=2y+4\\&2(2y+4)-4y=1}\\
				&&\heva{&x=2y+4\\&0y=-7.}
			\end{eqnarray*}
			Phương trình $0y=-7$ vô nghiệm.\\
			Vậy hệ phương trình vô nghiệm.
		\end{enumerate}
	}
\end{vd}

\subsubsection{Bài tập}
\begin{bt}%[Dự án EX-9-Đề Cương Toán 9]%[Nguyễn Cường]%[9D1H3-1]
	Giải các hệ phương trình sau:
	\begin{multicols}{3}
		\begin{enumerate}
			\item $\heva{&x-2y=1\\&3x+4y=13;}$
			\item$\heva{&3x+y=5\\&2x-3y=7;}$
			\item $\heva{&x+3y=11\\&4x-y=-8.}$
		\end{enumerate}
	\end{multicols}
	\loigiai{
		\begin{enumerate}
			\item 
			Ta có 
			\allowdisplaybreaks
			\begin{eqnarray*}
				&&\heva{&x-2y=1\\&3x+4y=13}\\
				&&\heva{&x=2y+1\\&3(2y+1)+4y=13}\\
				&&\heva{&x=2y+1\\&6y+3+4y=13}\\
				&&\heva{&x=2y+1\\&10y=10}\\
				&&\heva{&x=3\\&y=1.}
			\end{eqnarray*}
			Vậy hệ phương trình có nghiệm duy nhất là $(3;1)$.
			\item 
			Ta có
			\allowdisplaybreaks
			\begin{eqnarray*}
				&&\heva{&3x+y=5\\&2x-3y=7}\\
				&&\heva{&y=5-3x\\&2x-3(5-3x)=7}\\
				&&\heva{&y=5-3x\\&2x-15+9x=7}\\
				&&\heva{&y=5-3x\\&11x=22}\\
				&&\heva{&x=2\\&y=-1.}
			\end{eqnarray*}
			Vậy hệ phương trình có nghiệm duy nhất là $(2;-1)$.
			\item 
			Ta có 
			\allowdisplaybreaks
			\begin{eqnarray*}
				&&\heva{&x+3y=11\\&4x-y=-8}\\
				&&\heva{&x=11-3y\\&4(11-3y)-y=-8}\\
				&&\heva{&x=11-3y\\&44-12y-y=-8}\\
				&&\heva{&x=11-3y\\&-13y=-52}\\
				&&\heva{&x=-1\\&y=4.}
			\end{eqnarray*}
			Vậy hệ phương trình có nghiệm duy nhất là $(-1;4)$.
		\end{enumerate}
	}
\end{bt}
\begin{bt}%[Dự án EX-9-Đề Cương Toán 9]%[Nguyễn Cường]%[9D1H3-1]
	Giải các hệ phương trình sau bằng phương pháp thế.
	\begin{multicols}{2}
		\begin{enumerate}
			\item $\heva{&x-2y=1\\&3x+2y=3;}$
			\item $\heva{&2x+y=3\\&x+3y=4;}$
			\item $\heva{&-x+3y=3\\&2x-3y=-4;}$
			\item $\heva{&3x-2y=4\\&4x-3y=5.}$
		\end{enumerate}
	\end{multicols}
	\loigiai{
		\begin{enumerate}
			\item 
			Ta có
			\allowdisplaybreaks
			\begin{eqnarray*}
				&&\heva{&x-2y=1\\&3x+2y=3}\\
				&&\heva{&x=1+2y\\&3(1+2y)+2y=3}\\
				&&\heva{&x=1+2y\\&3+6y+2y=3}\\
				&&\heva{&x=1+2y\\&8y=0}\\
				&&\heva{&x=1\\&y=0.}
			\end{eqnarray*}
			Vậy hệ phương trình có nghiệm duy nhất là $(1;0)$.
			\item 
			Ta có
			\allowdisplaybreaks
			\begin{eqnarray*}
				&&\heva{&2x+y=3\\&x+3y=4}\\
				&&\heva{&y=3-2x\\&x+3(3-2x)=4}\\
				&&\heva{&y=3-2x\\&x+9-6x=4}\\
				&&\heva{&y=3-2x\\&-5x=-5}\\
				&&\heva{&x=1\\&y=1.}
			\end{eqnarray*}
			Vậy hệ phương trình có nghiệm duy nhất là $(1;1)$.
			\item 
			Ta có
			\allowdisplaybreaks
			\begin{eqnarray*}
				&&\heva{&-x+3y=3\\&2x-3y=-4}\\
				&&\heva{&x=3y-3\\&2(3y-3)-3y=-4}\\
				&&\heva{&x=3y-3\\&6y-6-3y=-4}\\
				&&\heva{&x=3y-3\\&3y=2}\\
				&&\heva{&x=-1\\&y=\dfrac{2}{3}.}
			\end{eqnarray*}
			Vậy hệ phương trình có nghiệm duy nhất là $\left(-1;\dfrac{2}{3}\right)$.
			\item 
			Ta có
			\allowdisplaybreaks
			\begin{eqnarray*}
				&&\heva{&3x-2y=4\\&4x-3y=5}\\
				&&\heva{&x=\frac{4+2y}{3}\\&4\left(\frac{4+2y}{3}\right)-3y=5}\\
				&&\heva{&x=\frac{4+2y}{3}\\&16+8y-9y=15}\\
				&&\heva{&x=\frac{4+2y}{3}\\&-y=-1}\\
				&&\heva{&x=2\\&y=1.}
			\end{eqnarray*}
			Vậy hệ phương trình có nghiệm duy nhất là $(2;1)$.
		\end{enumerate}
	}
\end{bt}
\begin{bt}%[Dự án EX-9-Đề Cương Toán 9]%[Nguyễn Cường]%[9D1H3-1]
	Giải các hệ phương trình sau bằng phương pháp thế.
	\begin{multicols}{2}
		\begin{enumerate}
			\item $\heva{&2x-3y=1\\&-4x+6y=2;}$
			\item $\heva{&2x+3y=5\\&5x-4y=1;}$
			\item $\heva{&2x-3y=2\\&-4x+6y=-2;}$
			\item $\heva{&3x-y=7\\&x+2y=0.}$
		\end{enumerate}
	\end{multicols}
	\loigiai{
		\begin{enumerate}
			\item 
			Ta có
			\allowdisplaybreaks
			\begin{eqnarray*}
				&&\heva{&2x-3y=1\\&-4x+6y=2}\\
				&&\heva{&x=\frac{1+3y}{2}\\&-4\left(\frac{1+3y}{2}\right)+6y=2}\\
				&&\heva{&x=\frac{1+3y}{2}\\&-2(1+3y)+6y=2}\\
				&&\heva{&x=\frac{1+3y}{2}\\&-2-6y+6y=2}\\
				&&\heva{&x=\frac{1+3y}{2}\\&-2=2 \quad (\text{Vô lý})}
			\end{eqnarray*}
			Vậy hệ phương trình vô nghiệm.
			\item 
			Ta có
			\allowdisplaybreaks
			\begin{eqnarray*}
				&&\heva{&2x+3y=5\\&5x-4y=1}\\
				&&\heva{&x=\frac{5-3y}{2}\\&5\left(\frac{5-3y}{2}\right)-4y=1}\\
				&&\heva{&x=\frac{5-3y}{2}\\&25-15y-8y=2}\\
				&&\heva{&x=\frac{5-3y}{2}\\&-23y=-23}\\
				&&\heva{&x=1\\&y=1.}
			\end{eqnarray*}
			Vậy hệ phương trình có nghiệm duy nhất là $(1;1)$.
			\item 
			Ta có
			\allowdisplaybreaks
			\begin{eqnarray*}
				&&\heva{&2x-3y=2\\&-4x+6y=-2}\\
				&&\heva{&x=\frac{2+3y}{2}\\&-4\left(\frac{2+3y}{2}\right)+6y=-2}\\
				&&\heva{&x=\frac{2+3y}{2}\\&-2(2+3y)+6y=-2}\\
				&&\heva{&x=\frac{2+3y}{2}\\&-4-6y+6y=-2}\\
				&&\heva{&x=\frac{2+3y}{2}\\&-4=-2 \quad (\text{Vô lý})}
			\end{eqnarray*}
			Vậy hệ phương trình vô nghiệm.
			\item 
			Ta có
			\allowdisplaybreaks
			\begin{eqnarray*}
				&&\heva{&3x-y=7\\&x+2y=0}\\
				&&\heva{&y=3x-7\\&x+2(3x-7)=0}\\
				&&\heva{&y=3x-7\\&x+6x-14=0}\\
				&&\heva{&y=3x-7\\&7x=14}\\
				&&\heva{&x=2\\&y=-1.}
			\end{eqnarray*}
			Vậy hệ phương trình có nghiệm duy nhất là $(2;-1)$.
		\end{enumerate}
	}
\end{bt}
\subsection{Giải hệ phương trình bằng phương pháp cộng đại số}
\subsubsection{Kiến thức trọng tâm}
\begin{tomtat}
	Tổng quát, để giải hệ hai phương trình bậc nhất hai ẩn bằng phương pháp cộng đại số, ta thực hiện các bước như sau:
	\begin{itemize}
		\item \textit{Bước 1:} Nhân hai vế của mỗi phương trình với một số thích hợp (nếu cần) sao cho các hệ số của một ẩn nào đó trong hai phương trình của hệ bằng nhau hoặc đối nhau.
		\item \textit{Bước 2:} Cộng hoặc trừ từng vế hai phương trình của hệ để được một phương trình một ẩn và giải phương trình đó.
		\item \textit{Bước 3:} Thế giá trị của ẩn tìm được ở Bước 2 vào một trong hai phương trình của hệ đã cho để tìm giá trị của ẩn còn lại. Kết luận nghiệm của hệ.
	\end{itemize}
\end{tomtat}
\begin{vd}%[Dự án EX-9-Đề Cương Toán 9]%[Nguyễn Cường]%[9D1H3-2]
	Giải các hệ phương trình:
	\begin{multicols}{3}
		\begin{enumerate}
			\item $\heva{&2x-3y=-5\\&x+3y=11;}$
			\item $\heva{&x-3y=5\\&2x-y=15;}$
			\item $\heva{&3x+2y=7\\&2x+3y=3.}$
		\end{enumerate}
	\end{multicols}
	\loigiai{
		\begin{enumerate}
			\item Cộng từng vế hai phương trình của hệ, ta được $3x=6$. Suy ra $x=2$.\\
			Thay $x=2$ vào phương trình thứ hai của hệ, ta được $2+3y=11$. Do đó $y=3$.\\
			Vậy hệ phương trình có nghiệm duy nhất là $(2;3)$.
			\item Nhân hai vế của phương trình thứ nhất với $2$, ta được hệ phương trình $\heva{&2x-6y=10\\&2x-y=15.}$\\
			Trừ từng vế hai phương trình của hệ ta được $-5y=-5$, suy ra $y=1$.\\
			Thay $y=1$ vào phương trình $x-3y=5$, ta có $x=8$.\\
			Vậy hệ phương trình có nghiệm duy nhất là $(8;1)$.
			\item Nhân hai vế của phương trình thứ nhất với $2$, nhân hai vế của phương trình thứ hai với $-3$, ta được $\heva{
				&6x+4y=14\\
				&-6x-9y=-9.}$\\
			Cộng từng vế hai phương trình của hệ, ta được $-5y=5$. Suy ra $y=-1$.\\
			Thay $y=-1$ vào phương trình $3x+2y=7$, ta được $3x+2\cdot(-1)=7$. Do đó $x=3$.\\
			Vậy hệ phương trình có nghiệm duy nhất là $(3;-1)$.
		\end{enumerate}
	}
\end{vd}
\begin{vd}%[Dự án EX-9-Đề Cương Toán 9]%[Nguyễn Cường]%[9D1H3-2]
Xác định $a$, $b$ để đồ thị hàm số $y=ax+b$ đi qua hai điểm $A$ và $B$ trong mỗi trường hợp sau:
	\begin{multicols}{2}
		\begin{enumerate}
			\item $A(1;2)$ và $B(3;8);$
			\item $A(2;1)$ và $B(4;-2).$
		\end{enumerate}
	\end{multicols}
	\loigiai{
		\begin{enumerate}
			\item Đồ thị hàm số $y=ax+b$ đi qua điểm $A(1;2)$ nên ta có $a+b=2$.\\
			Đồ thị hàm số tiếp tục đi qua điểm $B(3;8)$ nên $3a+b=8$.\\
			Do đó, ta có hệ phương trình $\heva{&a+b=2\\&3a+b=8.}$\\
			Giải hệ phương trình
			\allowdisplaybreaks
			\begin{eqnarray*}
				&&\heva{&a+b=2\\&3a+b=8}\\
				&&\heva{&a+b=2\\&(3a+b)-(a+b)=8-2}\\
				&&\heva{&a+b=2\\&2a=6}\\
				&&\heva{&a=3\\&3+b=2}\\
				&&\heva{&a=3\\&b=-1.}
			\end{eqnarray*}
			Vậy $a=3$, $b=-1$ nên phương trình đường thẳng là  $y=3x-1$.			
			\item Đồ thị hàm số $y=ax+b$ đi qua điểm $A(2;1)$ nên ta có $2a+b=1$.\\
			Đồ thị hàm số tiếp tục đi qua điểm $B(4;-2)$ nên $4a+b=-2$.\\
			Do đó, ta có hệ phương trình $\heva{&2a+b=1\\&4a+b=-2.}$\\
			Giải hệ phương trình
			\allowdisplaybreaks
			\begin{eqnarray*}
				&&\heva{&2a+b=1\\&4a+b=-2}\\
				&&\heva{&2a+b=1\\&(4a+b)-(2a+b)=-2-1}\\
				&&\heva{&2a+b=1\\&2a=-3}\\
				&&\heva{&a=-\dfrac{3}{2}\\&2\left(-\dfrac{3}{2}\right)+b=1}\\
				&&\heva{&a=-\dfrac{3}{2}\\&b=4.}
			\end{eqnarray*}
			Vậy $a=-\dfrac{3}{2}$, $b=4$ nên phương trình đường thẳng là $y=-\dfrac{3}{2}x+4$.
		\end{enumerate}
	}
\end{vd}
\subsubsection{Bài tập}
\begin{bt}%[Dự án EX-9-Đề Cương Toán 9]%[Nguyễn Cường]%[9D1H3-2]
	Giải các hệ phương trình sau:
	\begin{multicols}{2}
		\begin{enumerate}
			\item $\heva{& 3x+7y=7 \\& 2x+5y=-5;}$
			\item $\heva{&3x+y=0\\&x+2y=5;}$
			\item $\heva{&2x-y=-1 \\&x+2y=7;}$
			\item $\heva{&4x-5y=-2\\&-2x+y=-2;}$
			\item $\heva{&3x-4y=25\\ &5x+3y=3;}$
			\item $\heva{&-x+y=2\\&3x+y=-2;}$
			\item $\heva{&2(x+3)-5y=15\\&3x+2(y-1)=2;}$
		\item $\heva{&2(x+y)-5y=3\\&4(x-1)-2(y+1)=4.}$
		\end{enumerate}
	\end{multicols}
	\loigiai{
		\begin{enumerate}
			\item
			\allowdisplaybreaks
			\begin{eqnarray*}
				&&\heva{&-6x-14y=-14\\&	6x+15y =-15}\\
				&&\heva{&3x+y=7\\&	(-6x-14y)+(6x+15y) =-29}\\
				&&\heva{&3x+7\cdot(-29)=7\\&y=-29}\\
				&&\heva{&x=70\\&y=-29.}
			\end{eqnarray*}
			Vậy hệ phương trình có nghiệm duy nhất là $(70; -29)$.
			\item Ta có
			\allowdisplaybreaks
			\begin{eqnarray*}
				&&\heva{&3x+y=0\\&x+2y=5}\\
				&&\heva{&6x+2y=0\\&-x-2y=-5}\\
				&&\heva{&5x=-5\\&3x+y=0}\\
				&&\heva{&x=-1\\&3 \cdot (-1)+y=0}\\
				&&\heva{&x=-1\\&y=3.}
			\end{eqnarray*}
			Vậy hệ phương trình có nghiệm duy nhất là $(-1;3)$.
			\item
			Ta có
			\allowdisplaybreaks
			\begin{eqnarray*}
				&&\heva{&2x-y=-1 \\&x+2y=7}\\
				&&\heva{&4x-2y=-2\\&x+2y=7}\\
				&&\heva{&5x=5\\&x+2y=7}\\
				&&\heva{&x=1\\&y=3.}
			\end{eqnarray*}
			Vậy hệ phương trình có nghiệm duy nhất là $(x; y)=(1; 3)$.
			\item Ta có
			\allowdisplaybreaks
			\begin{eqnarray*}
				&&\heva{&4x-5y=-2\\&-2x+y=-2} \\
				&& \heva{&4x-5y=-2\\&-4x+2y=-4}\\
				&& \heva{&-3y=-6\\&-2x+y=-2}\\
				&& \heva{&y=2\\&-2x+2=-2}\\
				&& \heva{&y=2\\&-2x=-4}\\
				&& \heva{&y=2\\&x=2.}
			\end{eqnarray*}	
			Vậy hệ phương trình có nghiệm duy nhất là $(2; 2)$.
			\item
			Ta có
			\allowdisplaybreaks
			\begin{eqnarray*}
				&&\heva{&3x-4y=25\\ &5x+3y=3}\\
				&&\heva{&9x-12y=75\\ &20x+12y=12}\\
				&&\heva{&29x=87\\ &20x+12y=12}\\
				&&\heva{&x=3\\ &y=\dfrac{12-20x}{12}=\dfrac{12-20\cdot3}{12}=-4.}
			\end{eqnarray*}
			Vậy hệ phương trình có nghiệm duy nhất là $(3;-4)$.
			\item
			Ta có
			\allowdisplaybreaks
			\begin{eqnarray*}
				&&\heva{&-x+y=2\\&3x+y=-2}\\
				&&\heva{&-4x=4\\&3x+y=-2}\\
				&&\heva{&x=-1\\&y=-2-3x}\\
				&&\heva{&x=-1\\&y=1.}
			\end{eqnarray*}
			Vậy hệ phương trình có nghiệm duy nhất là $(-1;1)$.
			\item
			Ta có
			\begin{eqnarray*}
				&&\heva{&2(x+3)-5y=15\\&3x+2(y-1)=2}\\
				&&\heva{&2x-5y=9\\&3x+2y=4}\\
				&&\heva{&4x-10y=18\\&15x+10y=20}\\
				&&\heva{&19x=38\\&15x+10y=20}\\
				&&\heva{&x=2\\&15\cdot 2+10y=20}\\
				&&\heva{&x=2\\&y=-1.}
			\end{eqnarray*}	
			Vậy hệ phương trình có nghiệm duy nhất là $(2;-1)$.
			\item Ta có
			\allowdisplaybreaks
			\begin{eqnarray*}
				&&\heva{&2(x+y)-5y=3\\&4(x-1)-2(y+1)=4}\\
				&&\heva{&2x-3y=3\\&4x-2y=10} \\
				&&\heva{&4x-6y=6\\&4x-2y=10} \\
				&&\heva{&4y=4\\&2x-3y=3} \\
				&&\heva{&y=1\\&2x-3\cdot 1=3} \\
				&&\heva{&y=1\\&x=3.}			
			\end{eqnarray*}
			Vậy hệ phương trình có nghiệm duy nhất là $(3;1)$.			
		\end{enumerate}
	}
\end{bt}
\begin{bt}%[Dự án EX-9-Đề Cương Toán 9]%[Nguyễn Cường]%[9D1H3-2]
	 Giải các hệ phương trình sau
	\begin{multicols}{2}
		\begin{enumerate}
			\item $\heva{&4x+y=2 \\ &\dfrac{4}{3}x+\dfrac{1}{3}y=1;}$
			\item $\heva{&x-\sqrt{2}y=0 \\ &2x+\sqrt{2}y=3;}$
			\item $\heva{&5\sqrt{3}x+y=2\sqrt{2} \\ &\sqrt{6}x-\sqrt{2}y=2;}$
			\item $\heva{&2(x+y)+3(x-y)=4 \\ &(x+y)+2(x-y)=5.}$
		\end{enumerate}
	\end{multicols}
	\loigiai{
		\begin{enumerate}
			\item 
			Ta có
			\allowdisplaybreaks
			\begin{eqnarray*}
				&&\heva{&4x+y=2\\&\dfrac{4}{3}x+\dfrac{1}{3}y=1}\\
				&&\heva{&4x+y=2\\&4x+y=3}\\
				&&\heva{&4x+y=2\\&(4x+y)-(4x+y)=3-2}\\
				&&\heva{&4x+y=2\\&0=1 \quad (\text{Vô lý})}
			\end{eqnarray*}
			Vậy hệ phương trình vô nghiệm.
			\item 
			Ta có
			\allowdisplaybreaks
			\begin{eqnarray*}
				&&\heva{&x-\sqrt{2}y=0 \\ &2x+\sqrt{2}y=3}\\
				&&\heva{&x-\sqrt{2}y=0\\&\left(x-\sqrt{2}y\right)+\left(2x+\sqrt{2}y\right)=0+3}\\
				&&\heva{&x-\sqrt{2}y=0\\&3x=3}\\
				&&\heva{&x=1\\&1-\sqrt{2}y=0}\\
				&&\heva{&x=1\\&y=\dfrac{\sqrt{2}}{2}.}
			\end{eqnarray*}
			Vậy hệ phương trình có nghiệm duy nhất là $\left(1; \dfrac{\sqrt{2}}{2}\right)$.
			\item 
			Ta có
			\begin{eqnarray*}
				&&\heva{&5\sqrt{3}x+y=2\sqrt{2} \\ &\sqrt{6}x-\sqrt{2}y=2}\\
				&&\heva{&5\sqrt{6}x+\sqrt{2}y=4 \\ &\sqrt{6}x-\sqrt{2}y=2}\\
				&&\heva{&6\sqrt{6}x=6 \\ &\sqrt{6}x-\sqrt{2}y=2}\\
				&&\heva{&x=\dfrac{\sqrt{6}}{6} \\ &\sqrt{6}\cdot \left(\dfrac{\sqrt{6}}{6}\right)-\sqrt{2}y=2}\\
				&&\heva{&x=\dfrac{\sqrt{6}}{6} \\ &y=-\dfrac{\sqrt{2}}{2}.}
			\end{eqnarray*}
			Vậy hệ phương trình có nghiệm duy nhất là $\left(\dfrac{\sqrt{6}}{6}; -\dfrac{\sqrt{2}}{2}\right)$.
			\item 
			Ta có
			\allowdisplaybreaks
			\begin{eqnarray*}
				&&\heva{&2(x+y)+3(x-y)=4 \\ &(x+y)+2(x-y)=5}\\
				&&\heva{&2(x+y)+3(x-y)=4 \\ &2(x+y)+4(x-y)=10}\\
				&&\heva{&2(x+y)+3(x-y)=4 \\ &x-y=6}\\
				&&\heva{&x+y=-7 \\ &x-y=6}\\
				&&\heva{& (x+y)+(x-y)=-7+6 \\ &x-y=6}\\
				&&\heva{& 2x=-1 \\ &x-y=6}\\
				&&\heva{& x=-\dfrac{1}{2} \\ &-\dfrac{1}{2}-y=6}\\
				&&\heva{& x=-\dfrac{1}{2} \\ &y=-\dfrac{13}{2}.}
			\end{eqnarray*}
			Vậy hệ phương trình có nghiệm duy nhất là $\left(-\dfrac{1}{2}; -\dfrac{13}{2}\right)$.
		\end{enumerate}
	}
\end{bt}
\begin{bt}%[Dự án EX-9-Đề Cương Toán 9]%[Nguyễn Cường]%[9D1H3-2]
	Xác định $a$, $b$ để đồ thị hàm số $y=ax+b$ đi qua hai điểm $A, B$ trong mỗi trường hợp sau:
	\begin{multicols}{2}
		\begin{enumerate}
			\item $A(1;-2)$ và $B(-2;-1);$
			\item $A(2;8)$ và $B(-4;5).$
		\end{enumerate}
	\end{multicols}
	\loigiai{
		\begin{enumerate}
			\item 
			Ta có
			\allowdisplaybreaks
			\begin{eqnarray*}
				&&\heva{&a+b=-2\\&-2a+b=-1}\\
				&&\heva{&3a=-1\\&-2a+b=-1}\\
				&&\heva{&a=-\dfrac{1}{3}\\&-2\cdot \left(-\dfrac{1}{3}\right)+b=-1}\\
				&&\heva{&a=-\dfrac{1}{3}\\&b=-\dfrac{5}{3}.}
			\end{eqnarray*}
			Vậy $a=-\dfrac{1}{3}$, $b=-\dfrac{5}{3}$. Phương trình đường thẳng là $y=-\dfrac{1}{3}x - \dfrac{5}{3}$.
			
			\item 
			Ta có
			\allowdisplaybreaks
			\begin{eqnarray*}
				&&\heva{&2a+b=8\\&-4a+b=5}\\
				&&\heva{&6a=3\\&-4a+b=5}\\
				&&\heva{&a=\dfrac{1}{2}\\&-4\cdot \left(\dfrac{1}{2}\right)+b=5}\\
				&&\heva{&a=\dfrac{1}{2}\\&b=7.}
			\end{eqnarray*}
			Vậy $a=\dfrac{1}{2}$, $b=7$. Phương trình đường thẳng là $y=\dfrac{1}{2}x+7$.
		\end{enumerate}
	}
\end{bt}
\begin{bt}%[Dự án EX-9-Đề Cương Toán 9]%[Nguyễn Cường]%[9D1H3-2]
	Cho hệ phương trình $\heva{&ax+2by=5 \\& 3ax-5 by=-29}$. Với $x$, $y$ là ẩn số. Xác định cặp số $a$, $b$ biết hệ phương trình có nghiệm là $(-1;2)$.
	\loigiai{Vì $(-1;2)$ là nghiệm của hệ phương trình nên ta có
		{\allowdisplaybreaks
			\begin{eqnarray*}
				&&\heva{&-a+4b=5 \\& -3a-10b=-29}\\
				&&\heva{&-3a+12b=15\\&-3a-10b=-29}\\
				&&\heva{&22b=44\\&-3a-10b=-29}\\
				&&\heva{&a=3\\&b=2.}
		\end{eqnarray*}}
		Vậy $a=3$, $b=2$ là giá trị cần tìm.
		}
\end{bt}

