\section*{BÀI TẬP CUỐI CHƯƠNG IV}
% Tên bài

\subsection{Câu hỏi trắc nghiệm}

\Opensolutionfile{ans}[ans/ans-9C4-OTC]
%%%====Câu 1
\begin{ex}%[Dự án 9EX-DeCuongToan9]%[Nguyễn Anh Quốc]%[9H1N1-1]
\immini{Cho tam giác vuông $ABC$ tại $A$. Khẳng định nào sau đây \textbf{sai}?	
	\choice
	{$\sin \widehat{B}=\cos \widehat{C} $}
	{$\cos \widehat{B}=\sin \widehat{C} $}
	{$\tan \widehat{B}=\cot \widehat{C} $}
	{\True $\sin \widehat{B}=\sin \widehat{C} $}
}
{\begin{tikzpicture}[scale=0.8,>=stealth, font=\footnotesize, line join=round, line cap=round]
\def\xmin{-1} \def\xmax{5}  \def\ymin{-1}  \def\ymax{4} 
%	\draw[color=gray!50,dashed] (\xmin,\ymin) grid (\xmax,\ymax);
%	\draw[->] (\xmin,0)--(\xmax,0) node [below]{$x$};
%	\draw[->] (0,\ymin)--(0,\ymax) node [left]{$y$};
\clip (\xmin,\ymin) rectangle (\xmax,\ymax);
%%%%
\coordinate(A) at (0,0);
\coordinate(B) at (0,3);
\coordinate(C) at (4,0);
\draw(A)node[below]{$A$}circle(1pt)--(B)node[above]{$B$}circle(1pt)--(C)node[below]{$C$}circle(1pt)--(A);
\draw    pic[draw=black, angle radius=0.2cm]
{right angle=B--A--C}; %Góc vuông
\end{tikzpicture}}
	\loigiai{
		Công thức sai là $\sin \widehat{B}=\sin \widehat{C} $.
	}	
\end{ex}
\begin{ex}%[Dự án 9EX-DeCuongToan9]%[Nguyễn Anh Quốc]%[9H1N1-1]
\immini{	Cho tam giác $ABC$ vuông tại $A$. Khẳng định nào sau đây \textbf{sai}?
	\choice
	{$\sin \widehat{B} = \dfrac{AC}{BC}$}
	{$\cos \widehat{B} = \dfrac{AB}{BC}$}
	{$\tan \widehat{B} = \dfrac{AC}{AB}$}
	{\True $\cot \widehat{B} = \dfrac{AC}{AB}$}
}
{\begin{tikzpicture}[scale=0.8,>=stealth, font=\footnotesize, line join=round, line cap=round]
\def\xmin{-1} \def\xmax{5}  \def\ymin{-1}  \def\ymax{4} 
%	\draw[color=gray!50,dashed] (\xmin,\ymin) grid (\xmax,\ymax);
%	\draw[->] (\xmin,0)--(\xmax,0) node [below]{$x$};
%	\draw[->] (0,\ymin)--(0,\ymax) node [left]{$y$};
\clip (\xmin,\ymin) rectangle (\xmax,\ymax);
%%%%
\coordinate(A) at (0,0);
\coordinate(B) at (0,3);
\coordinate(C) at (4,0);
\draw(A)node[below]{$A$}circle(1pt)--(B)node[above]{$B$}circle(1pt)--(C)node[below]{$C$}circle(1pt)--(A);
\draw    pic[draw=black, angle radius=0.2cm]
{right angle=B--A--C}; %Góc vuông
\end{tikzpicture}}
	\loigiai{
		Trong tam giác vuông $ABC$ tại $A$
		\begin{itemize}
			\item Cạnh đối của góc $B$ là $AC$.
			\item Cạnh kề của góc $B$ là $AB$.
			\item Cạnh huyền là $BC$.
		\end{itemize}
		Theo định nghĩa tỉ số lượng giác thì
		$\cot \widehat{B} =  \dfrac{AB}{AC}$ (khẳng định đưa ra là $\dfrac{AC}{AB}$ nên sai).
	}
\end{ex}

\begin{ex}%[Dự án 9EX-DeCuongToan9]%[Nguyễn Anh Quốc]%[9H1H2-1]
\immini{Cho tam giác $MNP$ vuông tại $N$ có $MN = 3$ cm, $NP = 4$ cm. Tính $\sin \widehat{P}$.
	\choice
	{\True $\dfrac{3}{5}$}
	{$\dfrac{4}{5}$}
	{$\dfrac{3}{4}$}
	{$\dfrac{4}{3}$}}
{\begin{tikzpicture}[scale=0.8,>=stealth, font=\footnotesize, line join=round, line cap=round]
	\def\xmin{-1} \def\xmax{5}  \def\ymin{-1}  \def\ymax{4} 
	%	\draw[color=gray!50,dashed] (\xmin,\ymin) grid (\xmax,\ymax);
	%	\draw[->] (\xmin,0)--(\xmax,0) node [below]{$x$};
	%	\draw[->] (0,\ymin)--(0,\ymax) node [left]{$y$};
	\clip (\xmin,\ymin) rectangle (\xmax,\ymax);
	%%%%
	\coordinate(N) at (0,0);
	\coordinate(M) at (0,3);
	\coordinate(P) at (4,0);
	\draw(N)node[below]{$N$}circle(1pt)--(M)node[above]{$M$}circle(1pt)--(P)node[below]{$P$}circle(1pt)--(N);
	\draw    pic[draw=black, angle radius=0.2cm]
	{right angle=M--N--P}; %Góc vuông
	\end{tikzpicture}}
	\loigiai{
		Trong tam giác vuông $MNP$ tại $N$.\\
		Áp dụng định lí Py-ta-go để tìm cạnh huyền $MP$ ta có \\
		$MP^2 = MN^2 + NP^2 = 3^2 + 4^2 = 9 + 16 = 25$.
		$MP = \sqrt{25} = 5$ cm.\\ Ta có
		$\sin \widehat{P}=\dfrac{MN}{MP}=\dfrac{3}{5}$.
	}
\end{ex}

\begin{ex}%[Dự án 9EX-DeCuongToan9]%[Nguyễn Anh Quốc]%[9H1H1-1]
	Cho $\alpha$ là một góc nhọn trong tam giác vuông. Khẳng định nào sau đây là \textbf{đúng}?
	\choice
	{$\tan \alpha = \frac{\cos \alpha}{\sin \alpha}$}
	{$\cot \alpha = \frac{1}{\sin \alpha}$ (với $\sin \alpha \neq 0$)}
	{$\sin (90^\circ - \alpha) = \sin \alpha$}
	{\True $\sin^2 \alpha + \cos^2 \alpha = 1$}
	\loigiai{
		Kiểm tra từng khẳng định
		\begin{itemize}
			\item $\tan \alpha = \frac{\sin \alpha}{\cos \alpha}$ đúng, không phải $\frac{\cos \alpha}{\sin \alpha}$.
			\item $\cot \alpha = \frac{1}{\tan \alpha} = \frac{\cos \alpha}{\sin \alpha}$  đúng, không phải $\frac{1}{\sin \alpha}$.
			\item $\sin (90^\circ - \alpha) = \cos \alpha$  đúng (hai góc phụ nhau), không phải $\sin \alpha$.
			\item $\sin^2 \alpha + \cos^2 \alpha = 1$  luôn đúng.
		\end{itemize}
	}
\end{ex}

\begin{ex}%[Dự án 9EX-DeCuongToan9]%[Nguyễn Anh Quốc]%[9H1H2-1]
	Cho $\triangle ABC$ vuông tại $A$, biết $\cos \widehat{B} = 0{,}6$. Tính $\sin \widehat{C}$.
	\choice
	{$0{,}8$}
	{\True $0{,}6$}
	{$\frac{3}{4}$}
	{$\frac{4}{3}$}
	\loigiai{
		\immini{Trong tam giác vuông $ABC$ tại $A$, hai góc nhọn $B$ và $C$ là hai góc phụ nhau, tức là $\widehat{B} + \widehat{C} = 90^\circ$.
		Theo tính chất của hai góc phụ nhau: $\sin \widehat{C} = \cos \widehat{B}$.
		Vì $\cos \widehat{B} = 0{,}6$, suy ra $\sin \widehat{C} = 0{,}6$.}
	{\begin{tikzpicture}[scale=0.8,>=stealth, font=\footnotesize, line join=round, line cap=round]
		\def\xmin{-1} \def\xmax{5}  \def\ymin{-1}  \def\ymax{4} 
		%	\draw[color=gray!50,dashed] (\xmin,\ymin) grid (\xmax,\ymax);
		%	\draw[->] (\xmin,0)--(\xmax,0) node [below]{$x$};
		%	\draw[->] (0,\ymin)--(0,\ymax) node [left]{$y$};
		\clip (\xmin,\ymin) rectangle (\xmax,\ymax);
		%%%%
		\coordinate(A) at (0,0);
		\coordinate(B) at (0,3);
		\coordinate(C) at (4,0);
		\draw(A)node[below]{$A$}circle(1pt)--(B)node[above]{$B$}circle(1pt)--(C)node[below]{$C$}circle(1pt)--(A);
		\draw    pic[draw=black, angle radius=0.2cm]
		{right angle=B--A--C}; %Góc vuông
		\end{tikzpicture}}
	}
\end{ex}

\begin{ex}%[Dự án 9EX-DeCuongToan9]%[Nguyễn Anh Quốc]%[9H1H1-1]
Trong một tam giác vuông cho góc nhọn $\alpha$ thỏa mãn $\tan\alpha = \sqrt{3}$. Tính giá trị của $\cos\alpha$.
	\choice
	{$\dfrac{1}{2}$}
	{$\frac{\sqrt{3}}{2}$}
	{\True $\dfrac{1}{2}$}
	{$\frac{\sqrt{3}}{3}$}
	\loigiai{
		Ta biết $\tan \alpha = \sqrt{3}$. Giá trị này tương ứng với góc $\alpha = 60^\circ$.
		Khi đó, $\cos \alpha = \cos 60^\circ = \dfrac{1}{2}$.
	}
\end{ex}

---

\subsection*{Chủ đề 2: Hệ thức giữa cạnh và góc trong tam giác vuông}

\begin{ex}%[Dự án 9EX-DeCuongToan9]%[Nguyễn Anh Quốc]%[9H1N2-1]
	\immini{Cho tam giác $ABC$ vuông tại $A$. Hệ thức nào sau đây \textbf{sai}?
	\choice
	{$AB = BC \cdot \sin \widehat{C}$}
	{$AC = BC \cdot \cos \widehat{C}$}
	{$AB = AC \cdot \tan \widehat{C}$}
	{\True $AC = AB \cdot \cot \widehat{B}$}}
{\begin{tikzpicture}[scale=0.8,>=stealth, font=\footnotesize, line join=round, line cap=round]
	\def\xmin{-1} \def\xmax{5}  \def\ymin{-1}  \def\ymax{4} 
	%	\draw[color=gray!50,dashed] (\xmin,\ymin) grid (\xmax,\ymax);
	%	\draw[->] (\xmin,0)--(\xmax,0) node [below]{$x$};
	%	\draw[->] (0,\ymin)--(0,\ymax) node [left]{$y$};
	\clip (\xmin,\ymin) rectangle (\xmax,\ymax);
	%%%%
	\coordinate(A) at (0,0);
	\coordinate(B) at (0,3);
	\coordinate(C) at (4,0);
	\draw(A)node[below]{$A$}circle(1pt)--(B)node[above]{$B$}circle(1pt)--(C)node[below]{$C$}circle(1pt)--(A);
	\draw    pic[draw=black, angle radius=0.2cm]
	{right angle=B--A--C}; %Góc vuông
	\end{tikzpicture}}
	\loigiai{
		Trong tam giác vuông $ABC$ tại $A$
		\begin{itemize}
			\item $\sin \widehat{C} = \dfrac{AB}{BC}  \Rightarrow AB = BC \cdot \sin \widehat{C}$. 
			\item $\cos \widehat{C} = \dfrac{AC}{BC}  \Rightarrow AC = BC \cdot \cos \widehat{C}$. 
			\item $\tan \widehat{C} = \dfrac{AB}{AC}  \Rightarrow AB = AC \cdot \tan \widehat{C}$. 
			\item $\cot \widehat{B} = \dfrac{AB}{AC}\Rightarrow AB=AC\cdot \cot\widehat{B}.$
		\end{itemize}
	}
\end{ex}

\begin{ex}%[Dự án 9EX-DeCuongToan9]%[Nguyễn Anh Quốc]%[9H1H2-2]
	Cho tam giác $DEF$ vuông tại $D$, $DE = 6$ cm, góc $\widehat{F} = 30^\circ$. Tính độ dài cạnh $DF$.
	\choice
	{$12$ cm}
	{\True $6\sqrt{3}$ cm}
	{$3\sqrt{3}$ cm}
	{$6$ cm}
	\loigiai{
	\immini{	Trong tam giác vuông $DEF$ tại $D$.\\
		Ta có $\tan \widehat{F} = \dfrac{DE}{DF}$.
		Suy ra $\tan 30^\circ = \frac{6}{DF} \Rightarrow \dfrac{\sqrt{3}}{3} = \dfrac{6}{DF}$.\\ Nên
		$DF = \dfrac{6 \cdot 3}{\sqrt{3}} = \dfrac{18}{\sqrt{3}} = \dfrac{18\sqrt{3}}{3} = 6\sqrt{3}$ cm.}
	{\begin{tikzpicture}[scale=0.8,>=stealth, font=\footnotesize, line join=round, line cap=round]
		% Định nghĩa các điểm
		\coordinate (D) at (0,0);
		\coordinate (E) at (0,3); % DE = 6 cm
		\coordinate (F) at (5.2,0); % DF = 6*sqrt(3) cm
		
		% Vẽ các cạnh của tam giác
		\draw (D) node[below left]{$D$} circle(1pt) -- (E) node[above]{$E$} circle(1pt) -- (F) node[below right]{$F$} circle(1pt) -- (D);
		
		% Đánh dấu góc vuông tại D
		\draw pic[draw=black, angle radius=0.2cm]{right angle=E--D--F};
		
		% Ghi độ dài cạnh DE
		\draw (D) -- (E) node[midway,left=4pt]{$6 \text{ cm}$};
		
		% Đánh dấu và ghi giá trị góc F
		\draw pic[draw=black, angle radius=0.7cm, angle eccentricity=1.2]{angle=E--F--D} node at (F) [xshift=-1.5cm,yshift=0.5cm] {$30^\circ$};
		
		\end{tikzpicture}}
	}
\end{ex}



\begin{ex}%[Dự án 9EX-DeCuongToan9]%[Nguyễn Anh Quốc]%[9H1H2-2]
	Cho tam giác $PQR$ vuông tại $P$, $PQ = 5$ cm, $\widehat{Q} = 45^\circ$. Tính độ dài cạnh $QR$.
	\choice
	{\True $5\sqrt{2}$ cm}
	{$5$ cm}
	{$10$ cm}
	{$5\sqrt{3}$ cm}
	\loigiai{
	\immini{Trong tam giác vuông $PQR$ tại $P$.\\
		Ta có $\cos \widehat{Q} = \dfrac{PQ}{QR}$.\\ Suy ra
		$\cos 45^\circ = \dfrac{5}{QR} \Rightarrow \dfrac{\sqrt{2}}{2} = \dfrac{5}{QR}$.\\
		Ta được
		$QR = \dfrac{5 \cdot 2}{\sqrt{2}} = \dfrac{10}{\sqrt{2}} = \dfrac{10\sqrt{2}}{2} = 5\sqrt{2}$ cm.}
	{\begin{tikzpicture}[scale=0.8,>=stealth, font=\footnotesize, line join=round, line cap=round]
		% Định nghĩa các điểm
		\coordinate (P) at (0,0);
		\coordinate (Q) at (4,0); % PQ tùy ý
		\coordinate (R) at (0,4); % PR = PQ (tam giác vuông cân)
		
		% Vẽ các cạnh của tam giác
		\draw (P) node[below left]{$P$} circle(1pt) -- (Q) node[below right]{$Q$} circle(1pt) -- (R) node[above]{$R$} circle(1pt) -- (P);
		
		% Đánh dấu góc vuông tại P
		\draw pic[draw=black, angle radius=0.2cm]{right angle=Q--P--R};
		
		\draw    pic["$45^\circ$", draw=black, angle eccentricity=1.2, angle radius=1cm]
		{angle=R--Q--P}; %góc
		
		\end{tikzpicture}
	}
	}
\end{ex}


\begin{ex}%[Dự án 9EX-DeCuongToan9]%[Nguyễn Anh Quốc]%[9H1H2-3]
\immini{	Một cây cột điện cao 10m đổ bóng trên mặt đất dài $10$ m. Góc tạo bởi tia nắng mặt trời với mặt đất là bao nhiêu độ?
	\choice
	{$30^\circ$}
	{\True $45^\circ$}
	{$60^\circ$}
	{$90^\circ$}}
{
	\begin{tikzpicture}[scale=0.8,>=stealth, font=\footnotesize, line join=round, line cap=round]
	% Định nghĩa các điểm
	\coordinate (A) at (0,0); % Chân cột điện
	\coordinate (B) at (0,4); % Đỉnh cột điện (chiều cao 10m)
	\coordinate (C) at (4,0); % Điểm cuối bóng (bóng dài 10m)
	
	% Vẽ các cạnh của tam giác
	\draw (A) node[below left]{$A$} circle(1pt) -- (B) node[above]{$B$} circle(1pt) -- (C) node[below right]{$C$} circle(1pt) -- cycle;
	
	% Đánh dấu góc vuông tại A
	\draw pic[draw=black, angle radius=0.2cm]{right angle=B--A--C};
	
	% Vẽ tia nắng mặt trời
	\draw[dashed, ->] (B) -- (C) node[midway, above right] {Tia nắng};
	
	% Đánh dấu và ghi giá trị góc tại C (góc tạo bởi tia nắng và mặt đất)
%	\draw pic["$45^\circ$", draw=black, angle eccentricity=1.2, angle radius=0.8cm]{angle=B--C--A}; %góc
	
	% Ghi chú kích thước
	\draw (A) -- (B) node[midway,left=4pt]{$10 \text{ m}$}; % Chiều cao cột điện
	\draw (A) -- (C) node[midway,below=4pt]{$10 \text{ m}$}; % Chiều dài bóng
	
	\end{tikzpicture}}
	\loigiai{
		Xét tam giác vuông $ABC$ với $\widehat{BCA}$ là góc tạo bởi tia nắng mặt trời với mặt đất.\\
		Ta có $\tan \widehat{BCA}  = \dfrac{10}{10} = 1$. Suy ra $\widehat{BCA} = 45^\circ$.
	}
\end{ex}



\begin{ex}%[Dự án 9EX-DeCuongToan9]%[Nguyễn Anh Quốc]%[9H1H2-2]
	Cho tam giác $PQR$ vuông tại $P$. Biết $QR = 15$ cm và $\widehat{R} = 25^\circ$. Tính độ dài cạnh $PQ$. (Làm tròn đến chữ số thập phân thứ nhất).
	\choice
	{\True $6{,}3$ cm}
	{$13{,}6$ cm}
	{$7{,}0$ cm}
	{$14{,}7$ cm}
	\loigiai{
	\immini{	Trong tam giác vuông $PQR$. 
		Ta có $\sin \widehat{R} = \dfrac{PQ}{QR}$.
		$\sin 25^\circ = \dfrac{PQ}{15}$.\\
		Suy ra 
		$PQ = 15 \cdot \sin 25^\circ$.
		Sử dụng máy tính $\sin 25^\circ \approx 0{,}4226$.\\ Nên
		$PQ \approx 15 \cdot 0.4226 \approx 6{,}339$.\\
		Làm tròn đến chữ số thập phân thứ nhất, $PQ \approx 6{,}3$ cm.}
	{\begin{tikzpicture}[scale=0.7,>=stealth, font=\footnotesize, line join=round, line cap=round]
		% Định nghĩa các điểm
		\coordinate (P) at (0,0);
		\coordinate (Q) at (0,6.339); % PQ approx 6.339 cm
		\coordinate (R) at (13.595,0); % PR approx 13.595 cm
		
		% Vẽ các cạnh của tam giác
		\draw (P) node[below left]{$P$} circle(1pt) -- (Q) node[above left]{$Q$} circle(1pt) -- (R) node[below right]{$R$} circle(1pt) -- (P);
		
		% Đánh dấu góc vuông tại P
		\draw pic[draw=black, angle radius=0.2cm]{right angle=Q--P--R};
		
		% Đánh dấu và ghi giá trị góc R
		\draw pic["$25^\circ$", draw=black, angle eccentricity=1.2, angle radius=1.2 cm]{angle=Q--R--P};
		
		% Ghi độ dài cạnh QR (huyền)
		\draw (Q) -- (R) node[midway,above right=4pt]{$15 \text{ cm}$};
		
		\end{tikzpicture}}
	}
\end{ex}

\begin{ex}%[Dự án 9EX-DeCuongToan9]%[Nguyễn Anh Quốc]%[9H1H2-3]
\immini{Một con đường dốc lên với góc $\widehat{ACB}=15^\circ$ so với phương ngang. Nếu một người đi được $BC=500$ m trên con đường đó, hỏi người đó đã lên cao bao nhiêu mét so với điểm xuất phát? (Làm tròn đến hàng đơn vị)
	\choice
	{$483$ m}
	{$129$ m}
	{\True $129$ m}
	{$125$ m}}
{	\begin{tikzpicture}[scale=0.8,>=stealth, font=\footnotesize, line join=round, line cap=round]
	\def\xmin{-1} \def\xmax{7}  \def\ymin{-1}  \def\ymax{2} 
	%\draw[color=gray!50,dashed] (\xmin,\ymin) grid (\xmax,\ymax);
	%\draw[->] (\xmin,0)--(\xmax,0) node [below]{$x$};
	%\draw[->] (0,\ymin)--(0,\ymax) node [left]{$y$};
	\clip (\xmin,\ymin) rectangle (\xmax,\ymax);
	%%%%
	\coordinate (A) at (0,0);
	\coordinate (B) at (0,1);
	\coordinate (C) at (6,0);
	\draw (A)node[below]{$A$}circle(1pt)--(B)node[above]{$B$}circle(1pt)--(C)node[below]{$C$}circle(1pt)--(A);
	\draw    pic[draw=black, angle radius=0.2cm]
	{right angle=B--A--C}; %Góc vuông
	\end{tikzpicture}}
	\loigiai{
		$AB$ là chiều cao và $AB=BC\sin \widehat{BCA}=500\cdot \sin 50^\circ=129$ m.
	}
\end{ex}




\subsection{Bài tập tự luận}
%%%=======Bài 1
\begin{bt}%[Dự án 9EX-DeCuongToan9]%[Nguyễn Anh Quốc]%[9H1V1-1]
	Không sử dụng máy tính cầm tay, tính giá trị của biểu thức sau: 
	$$A = \sin 31^\circ + \tan 42^\circ - \cos 59^\circ - \cot 48^\circ$$
	\loigiai{
		Ta có\\
		$39^\circ + 51^\circ =90^\circ$ suy ra $\sin{39^\circ}=\cos{59^\circ}$,\\
		$42^\circ + 48^\circ =90^\circ$ suy ra $\tan{42^\circ}=\cot{48^\circ}$.\\
		Do đó 
		\begin{eqnarray*}
			A &=& \sin 31^\circ + \tan 42^\circ - \cos 59^\circ - \cot 48^\circ\\
			&=& \cos 59^\circ + \cot 48^\circ- \cos 59^\circ - \cot 48^\circ=0.
		\end{eqnarray*}
	}
\end{bt}

\begin{bt}%[Dự án 9EX-DeCuongToan9]%[Nguyễn Anh Quốc]%[9H1V2-3]
	Giả sử ở những giây đầu tiên, máy bay bay thẳng theo một đường thẳng tạo với mặt đất một góc $22^\circ$ với tốc độ $210$ km/h. Tính độ cao của máy bay (tính theo mét, làm tròn kết quả đến hàng đơn vị) so với mặt đất sau khi máy bay cất cánh rời khỏi mặt đất $3$ giây.
	\loigiai{
		Ta có $210$ km/h = $\dfrac{175}{3}$ m/s.\\
		Gọi $A$ là điểm máy bay cất cánh, sau $3$ giây máy bay bay đến điểm $B$ cách $A$ một đoạn $AB=\dfrac{175}{3}\cdot 3=175$ m.\\
		Gọi $H$ là hình chiếu của máy bay lên mặt đất.
		\begin{center}
			\begin{tikzpicture}[scale=1, font=\footnotesize, line join=round, line cap=round, >=stealth]
			\def\goc{22}
			\def\a{1.75}
			\pgfmathsetmacro{\b}{\a*sin(\goc)}
			\path 
			(0:0) coordinate (A)
			++(\goc:3.5*\a) coordinate (B)
			++(-90:3.5*\b) coordinate (H)
			;
			\draw (A)--(B)--(H)--cycle;
			\draw pic[draw,,angle radius=12mm]{angle=H--A--B};
			\draw pic[-stealth,angle radius=26mm,"$22^\circ$"]{angle=H--A--B};
			\foreach \x/\g in {A/180,B/90,H/0}
			\fill 	(\x) circle (1pt)
			($(\g:3mm)+(\x)$) node {$\x$};
			\end{tikzpicture}
		\end{center}
		Xét $\triangle ABH$ vuông tại $H$, $\sin{\widehat{BAH}}=\sin{22^\circ}=\dfrac{BH}{AB}$.\\
		Suy ra $BH=AB\cdot \sin{22^\circ}=175\cdot\sin{22^\circ}\approx 66$ m.\\
		Vậy độ cao của máy bay đạt được sau khi cất cánh $3$ giây là $66$ m.
	}
\end{bt}

\begin{bt}%[Dự án 9EX-DeCuongToan9]%[Nguyễn Anh Quốc]%[9H1V2-3]
	\immini{Một cây sậy mọc thẳng đứng trên bờ hồ, nếu khơi mực nước hồ đi xuống $0{,}5$ m. Một cơn gió thổi làm ngọn cây chạm vào mặt nước cách chỗ thân nhô khỏi mặt nước lúc đầu là $2$ m. Tính chiều cao cây sậy.}
	{\begin{tikzpicture}[scale=1, font=\footnotesize, line join=round, line cap=round, >=stealth]
		\path 
		(0:0) coordinate (A)
		++(90:3.75) coordinate (B)
		++(90:0.5) coordinate (C)
		(B)++(0:2) coordinate (D)
		;
		\draw 
		(B)--(C)node[midway,left]{$0{,}5$ m}
		(D)--(B)node[midway,above]{$2$ m}
		(B)--(A)--(D)--cycle
		;
		\foreach \x/\g in {A/-90,B/-150,C/90,D/0}
		\fill	(\x) circle (1pt)
		($(\g:3mm)+(\x)$) node {$\x$};
		\end{tikzpicture}}
	\loigiai{
\immini{Với $A$ là gốc của cây sậy, $C$ là ngọn của cây sậy. Khi có gió thổi thì ngọn cây chạm mặt nước tại $D$ cách điểm $B$ ban đầu $2$ m.\\
		Ta có $AD=AC$.\\
		Gọi $h$ là chiều cao của cây sậy, suy ra $AB=h-0{,}5$ và $AD=h$.\\
		Xét $\triangle ABD$ vuông tại $B$, ta có
		\begin{eqnarray*}
			AD^2&=&AB^2+BD^2\text{ (Pythagore)}\\
			h^2 &=&(h-0{,}5)^2+2^2\\
			h &=&4{,}45
		\end{eqnarray*}}
	{\begin{tikzpicture}[scale=1, font=\footnotesize, line join=round, line cap=round, >=stealth]
		\path 
		(0:0) coordinate (A)
		++(90:3.75) coordinate (B)
		++(90:0.5) coordinate (C)
		(B)++(0:2) coordinate (D)
		;
		\draw 
		(B)--(C)node[midway,left]{$0{,}5$ m}
		(D)--(B)node[midway,above]{$2$ m}
		(B)--(A)--(D)--cycle
		;
		\foreach \x/\g in {A/-90,B/-150,C/90,D/0}
		\fill	(\x) circle (1pt)
		($(\g:3mm)+(\x)$) node {$\x$};
		\end{tikzpicture}}
	}
\end{bt}

\begin{bt}%[Dự án 9EX-DeCuongToan9]%[Nguyễn Anh Quốc]%[9H1V2-2]
	Cho hình thang $ABCD$ vuông tại $A$ và $D$. Biết $AB = 40$ cm, $CD = 80$ cm, $BC = 50$ cm. Tính diện tích hình thang $ABCD$.
	\loigiai{\begin{center}
			\begin{tikzpicture}[scale=1, font=\footnotesize,line join=round, line cap=round, >=stealth]
			\path 
			(0,0) coordinate (A)
			++(-90:2) coordinate (D)
			++(0:4) coordinate (C)
			(2,0) coordinate (B)
			(2,-2) coordinate (H)
			;
			\draw(A)--(B)--(C)--(D)--cycle (B)--(H);
			\pic[draw,angle eccentricity=1.8,angle radius=2mm]{right angle=D--A--B}
			pic[draw,angle eccentricity=1.8,angle radius=2mm]{right angle=C--D--A};
			\foreach \i/\g in {A/90,B/90,C/-90,D/-90,H/-90}
			\fill[black] (\i) circle(1pt)+(\g:4mm)node[scale=1]{$\i$};
			\end{tikzpicture}
		\end{center}
		Kẻ đường cao $BH$ ($H \in CD$).\\
		Ta có $BH = AD = h$, $HC = DH - DC = 80 - 40 = 40$ cm.\\
		Xét tam giác vuông $BCH$, ta có $BH^2 + HC^2 = BC^2$.\\
		Suy ra $h^2 + 40^2 = 50^2$, do đó $h^2 = 50^2 - 40^2 = 2500 - 1600 = 900$. Vậy $h = \sqrt{900} = 30$ cm.\\
		Vậy diện tích hình thang $ABCD$ là\\
		$S = \dfrac{1}{2}(AB + CD)h = \dfrac{1}{2}(40 + 80) \cdot 30 = \dfrac{1}{2} \cdot 120 \cdot 30 = 1\,800$ (cm$^2$).\\	
	}
\end{bt}

\begin{bt}%[Dự án 9EX-DeCuongToan9]%[Nguyễn Anh Quốc]%[9H1V2-2]
	Cho tam giác $ABC$ vuông tại $A$. Biết $AB = 6$ cm và $\tan B = \dfrac{5}{12}$. Tính độ dài đường cao $AH$.
	\loigiai{
		Ta có $\tan B = \dfrac{5}{12}$, suy ra $\dfrac{AC}{AB} = \dfrac{5}{12}$ hay $AC = \dfrac{5}{12} AB = \dfrac{5}{12} \cdot 6 = \dfrac{30}{12} = \dfrac{5}{2}$ (cm).\\
		$BC = \sqrt{AB^2 + AC^2} = \sqrt{6^2 + \left( \dfrac{5}{2}\right) ^2} = \sqrt{36 + \dfrac{25}{4}} = \sqrt{\dfrac{144 + 25}{4}} = \sqrt{\dfrac{169}{4}} = \dfrac{13}{2}$ (cm).\\
		Ta có $AH \cdot BC = AB \cdot AC = 2 S_{ABC}$, suy ra $AH = \dfrac{AB \cdot AC}{BC}$.\\
		Vậy $AH = \dfrac{6 \cdot \dfrac{5}{2}}{\dfrac{13}{2}} = \dfrac{15}{\dfrac{13}{2}} = \dfrac{15 \cdot 2}{13} = \dfrac{30}{13}$ (cm).
	}
\end{bt}


\begin{bt}%[Dự án 9EX-DeCuongToan9]%[Nguyễn Anh Quốc]%[9H1V2-3]
	Một con thuyền với tốc độ thực $1{,}5$ km/h vượt qua một khúc sông nước chảy mạnh mất $15$ phút. Biết rằng đường đi của con thuyền tạo với bờ sông một góc $60^\circ$. Hãy tính chiều rộng của khúc sông (tính chính xác đến hàng đơn vị của mét).
	\loigiai{
		\begin{center}
			\begin{tikzpicture}[scale=1, font=\footnotesize, line join=round, line cap=round, >=stealth]
			\def\a{3.75}
			\def\goc{60}
			\pgfmathsetmacro{\b}{\a*cos(\goc)}
			\path 
			(0:0) coordinate (A)
			++(\goc:\a) coordinate (B)
			(A)++(0:\b) coordinate (C)
			;
			\draw 
			(A)--(B)--(C)--cycle;
			\draw pic["$60^\circ$",-stealth,angle radius=8mm]{angle=C--A--B};
			\draw pic[draw,angle radius=2mm]{right angle=B--C--A};
			\foreach \x/\g in {A/180,B/90,C/0}
			\fill	(\x) circle (1pt)
			($(\g:3mm)+(\x)$) node {$\x$};
			\end{tikzpicture}
		\end{center}
		Gọi $A$ là điểm xuất phát của thuyền ở bờ bên đây sông, sau $15$ phút thuyền đi với vận tốc là $1{,}5$ km/h đến bờ bên kia sông tại $B$ với khoảng cách $AB=1{,}5\cdot\dfrac{15}{60}=0{,}375$ km.\\
		Với $BC$ là chiều rộng của sông.\\
		Xét $\triangle ABC$ vuông tại $C$, ta có $\sin{\widehat{BAC}}=\sin{60^\circ}=\dfrac{BC}{AB}$.\\
		Suy ra, $BC=AB\cdot\sin{60^\circ}=\dfrac{3\sqrt{3}}{16}$ km $\approx 325$ m.
	}
\end{bt}

\begin{bt}%[Dự án 9EX-DeCuongToan9]%[Nguyễn Anh Quốc]%[9H1V2-3]
	Lúc $6$ giờ sáng, bạn An đi xe đạp từ nhà (điểm $A$) đến trường (điểm $B$) phải leo lên và xuống một con dốc (như hình vẽ bên).\\
	Cho biết đoạn thẳng $AB$ dài $762$ m, góc tại $A$ là $6^\circ$, góc tại $B$ là $4^\circ$.
	\begin{center}
		\begin{tikzpicture}[scale=1, font=\footnotesize, line join=round, line cap=round, >=stealth]
		\def\a{7.62}
		\def\gocA{14}
		\def\gocB{10}
		\pgfmathsetmacro{\h}{\a/(cot(\gocA)+cot(\gocB))}
		\pgfmathsetmacro{\AC}{\h/sin(\gocA)}
		\path 
		(0:0) coordinate (A)
		++(0:\a) coordinate (B)
		(A)++(\gocA:\AC) coordinate (C)
		++(-90:\h) coordinate (H)
		;
		\draw 
		(C)--(H)node[midway,right]{$h$}
		(A)--(B)--(C)--cycle
		;
		\foreach \x/\g in {A/180,B/0,C/90,H/-90}
		\fill	(\x) circle (1pt)
		($(\g:3mm)+(\x)$) node {$\x$};
		\end{tikzpicture}
	\end{center}
	\begin{enumerate}
		\item Tính chiều cao $h$ của con dốc.
		\item Hỏi bạn An đến trường lúc mấy giờ? Biết rằng tốc độ trung bình khi lên dốc là $4$ km/h và tốc độ trung bình khi xuống dốc là $19$ km/h.
	\end{enumerate}
	\loigiai{
		\begin{center}
			\begin{tikzpicture}[scale=1, font=\footnotesize, line join=round, line cap=round, >=stealth]
			\def\a{7.62}
			\def\gocA{14}
			\def\gocB{10}
			\pgfmathsetmacro{\h}{\a/(cot(\gocA)+cot(\gocB))}
			\pgfmathsetmacro{\AC}{\h/sin(\gocA)}
			\path 
			(0:0) coordinate (A)
			++(0:\a) coordinate (B)
			(A)++(\gocA:\AC) coordinate (C)
			++(-90:\h) coordinate (H)
			;
			\draw 
			(C)--(H)node[midway,right]{$h$}
			(A)--(B)--(C)--cycle
			;
			\foreach \x/\g in {A/180,B/0,C/90,H/-90}
			\fill	(\x) circle (1pt)
			($(\g:3mm)+(\x)$) node {$\x$};
			\end{tikzpicture}
		\end{center}
		\begin{enumerate}
			\item Xét lần lượt các tam giác $ACH$ vuông tại $H$, tam giác $BCH$ vuông tại $H$, ta có\\
			$\cot{A}=\dfrac{AH}{CH} \Rightarrow AH=CH\cdot\cot{6^\circ}$,\\ $\cot{B}=\dfrac{BH}{CH} \Rightarrow BH=CH\cot{4^\circ}$.\\
			Mặt khác, $AH+CH=AB \Rightarrow CH\cdot (\cot{6^\circ}+\cot{4^\circ})=762$.\\
			Suy ra $h=CH=\dfrac{762}{\cot{6^\circ}+\cot{4^\circ}}\approx 32$ m.\\
			\item 
			Ta có $4$ km/h $=\dfrac{10}{9}$ m/s và $19$ km/h $=\dfrac{95}{18}$ m/s.\\
			Thời gian bạn An đi qua quãng đường $AC$ (lên dốc) là
			\[t_{AC}=\dfrac{AC}{\dfrac{10}{9}}=\dfrac{\dfrac{h}{\sin{6^\circ}}}{\dfrac{10}{9}}\approx 276\text{ (giây)}.\]
			Thời gian bạn An đi qua quãng đường $CB$ (xuống dốc) là
			\[t_{CB}=\dfrac{CB}{\dfrac{95}{18}}=\dfrac{\dfrac{h}{\sin{4^\circ}}}{\dfrac{95}{18}}\approx 87\text{ (giây)}.\]
			Tổng thời gian An đi là $276+87=363\text{ (giây)}\approx 6\text{ (phút)}$.\\
			Vậy An đến trường lúc $6$ giờ $6$ phút.
		\end{enumerate}
	}
\end{bt}
% In đáp án trắc nghiệm