\section*{BÀI TẬP CUỐI CHƯƠNG 7}
\subsection{Câu hỏi trắc nghiệm}

\Opensolutionfile{ans}[ans/ans-9C7-OTC]
\begin{ex}%[Dự án EX-9-Đề Cương Toán 9]%[Phạm Văn Long]%[9D5N1-1]
	Điểm kiểm tra môn toán giữa học kì $1$ lớp $9A$ cho bởi bảng sau
	\begin{center}
		\begin{tabular}{|c|>{\centering\arraybackslash}m{.75cm}|c|>{\centering\arraybackslash}m{.75cm}|>{\centering\arraybackslash}m{.75cm}|>{\centering\arraybackslash}m{.75cm}|>{\centering\arraybackslash}m{.75cm}|>{\centering\arraybackslash}m{.75cm}|>{\centering\arraybackslash}m{.75cm}|c|}
			\hline Điểm ($x$) & $0$ & $2$ & $5$ & $6$ & $7$ & $8$ & $9$ & $10$ & Cộng \\
			\hline Tần số ($n$) & $1$ & $2$ & $5$ & $6$ & $9$ & $10$ & $4$ & $3$ & $N=40$ \\
			\hline
		\end{tabular}
	\end{center}	
	Tần số xuất hiện của điểm $7$ là
	\choice
	{$8$}
	{$10$}
	{$3$}
	{\True $9$}
	\loigiai{
		Quan sát bảng trên ta thấy điểm $7$ có số lần xuất hiện là $9$.
	}
\end{ex}
\begin{ex}%[Dự án EX-9-Đề Cương Toán 9]%[Phạm Văn Long]%[9D5H1-1]
	Thống kê điểm sau $20$ lần bắn bia của một xạ thủ như sau
	\begin{center}
		\begin{tabular}{>{\centering\arraybackslash}m{1.25cm}>{\centering\arraybackslash}m{1.25cm}>{\centering\arraybackslash}m{1.25cm}>{\centering\arraybackslash}m{1.25cm}>{\centering\arraybackslash}m{1.25cm}>{\centering\arraybackslash}m{1.25cm}>{\centering\arraybackslash}m{1.25cm}>{\centering\arraybackslash}m{1.25cm}>{\centering\arraybackslash}m{1.25cm}>{\centering\arraybackslash}m{1.25cm}}
			8 & 10 & 9 & 8 & 8 & 9 & 9 & 8 & 7 & 7 \\
			9 & 9 & 9 & 8 & 10 & 7 & 7 & 9 & 9 & 9 \\				
		\end{tabular}	
	\end{center}
	Chọn khẳng định đúng trong các khẳng định sau
	\choice
	{Tần số xuất hiện của điểm $7$ là $5$}
	{Tần số xuất hiện của điểm $8$ là $6$}
	{\True Tần số xuất hiện của điểm $9$ là $9$}
	{Tần số xuất hiện của điểm $10$ là $1$}
	\loigiai{
		Quan sát mẫu số liệu thống kê trên trên ta thấy
		\begin{itemize}
			\item Điểm $7$ có lần xuất hiện là $4$.
			\item  Điểm $8$ có số lần xuất hiện là $5$.
			\item  Điểm $9$ có số lần xuất hiện là $9$.
			\item  Điểm $10$ có số lần xuất hiện là $2$. 
		\end{itemize}
		Do đó \lq\lq Tần số xuất hiện của điểm $9$ là $9$\rq\rq\, là đúng.		 
	}
\end{ex}

\begin{ex}%[Dự án EX-9-Đề Cương Toán 9]%[Phạm Văn Long]%[9D5N2-1]
	Một cuộc khảo sát về sở thích nghe nhạc của $200$ người cho kết quả như sau
\begin{center}
	\begin{tabular}{|c|c|c|c|c|}
		\hline
		Thể loại nhạc & Pop& Rock & Jazz & Country \\
		\hline
		Tần số $(n)$ & $60$ & $30$ & $40$ & $70$ \\
		\hline
		Tần số tương đối $f$ $(\%)$ & $30$ & $15$ & $20$ & $35$ \\
		\hline
	\end{tabular}
\end{center}
Tần số tương đối của thể loại nhạc Pop là bao nhiêu?
\choice
{$15\%$}
{$20\%$}
{\True $30\%$}
{$60\%$}
\loigiai{
Tần số tương đối của thể loại nhạc Pop là $30\%$.
}	
\end{ex}
\begin{ex}%[Dự án EX-9-Đề Cương Toán 9]%[Phạm Văn Long]%[9D5H2-1]
	Điểm kiểm tra môn toán giữa học kì I lớp $9$A cho bởi bảng sau
	\begin{center}
		\begin{tabular}{|c|c|c|c|c|c|c|c|c|c|}
			\hline
			Điểm $(x)$ & $0$ & $2$ & $5$ & $6$ & $7$ & $8$ & $9$ & $10$ & Cộng\\
			\hline
			Tần số  $(n)$ & $1$ & $2$ & $5$ & $6$ & $9$ & $10$ & $4$ &$3$ &$N=40$ \\
			\hline
		\end{tabular}
	\end{center}
	Tần số tương đối của điểm $8$ là
	\choice
	{$2{,}5\%$}
	{\True $25\%$}
	{$10\%$}
	{ $40\%$}
	\loigiai{
	Tần số tương đối của điểm $8$ là $\dfrac{10}{40}\cdot 100\%=25\%$.	
	}	
\end{ex}
\begin{ex}%[Dự án EX-9-Đề Cương Toán 9]%[Phạm Văn Long]%[9D5H1-2]
	Biểu đồ cột dưới đây biểu diễn về điểm kiểm tra môn Toán cuối học kì I của khối lớp $9$ của một trường THCS. Biết rằng có $100$ bài kiểm tra được thống kê. Tỉ lệ phần trăm các em đạt điểm $8$ là?
	\begin{center}
		\begin{tikzpicture}[scale=.75, font=\footnotesize, >=stealth]
			%	\hetruc(0,0)(13,7)
			\draw [->](0,0)--(0,8);
			\draw(0,0)--(8,0);
			\foreach \a in {0,1,...,8}{
				\draw[gray!30] (0,\a)--(16,\a);
			}
			\foreach \b in {0,5,...,40}
			\draw (0,\b/5)node[left]{$\b$};
			\draw[line width =.5cm, color=gray](1,0)--(1,3)[above]node{$15$} (3,0)--(3,4)[above]node{$20$} (5,0)--(5,5)node{$25$} (7,0)--(7,6)[above]node{$30$} (9,0)--(9,2) [above]node{$10$};
			\draw (1,-0.5) node{5};
			\draw (3,-0.5)node{6};
			\draw (5,-0.5) node{7};
			\draw (7,-0.5) node{8};
			\draw (9,-0.5) node{9};		
			
			\draw (8,8.5) node{\textbf{Biểu đồ cột}};
			
			\draw (15,-0.5) node{Điểm};
			
			\draw (-2,8) node{Tần số};
		\end{tikzpicture}
	\end{center}
	\choice
	{\True $30 \%$}
	{$100 \%$}
	{$0{,}3 \%$}
	{$33 \%$}
	\loigiai{
		Từ biểu đồ ta thấy số học sinh đạt điểm $8$ là $30$ học sinh.\\
		Tỉ lệ phần trăm các em đạt điểm $8$ là $\dfrac{30}{100}=30 \%$.
	}
\end{ex}
\begin{ex}%[Dự án EX-9-Đề Cương Toán 9]%[Phạm Văn Long]%%[9D5H1-2]
	Biểu đồ ghi lại điểm kiểm tra một tiết môn toán  của học sinh lớp $7$A  như sau
	\begin{center}
		\begin{tikzpicture}[scale=0.7,font=\footnotesize, line join=round, line cap=round, >=stealth]
			
			\draw[->] (0,0) -- (20,0);
			\draw[->] (0,0) -- (0,11);
			\foreach \x/\i in {0/0,1/1,2/2,3/3,4/4,5/5,6/6,7/7,8/8,9/9,10/10}
			{
				\draw (-.05,\x) -- (.05,\x);
				\draw (0,\x) node[left]{$\i$};
			}
			\draw[fill=black!30] (2,0) rectangle (3,2) node[above left] {\color{black}$2$};
			\draw[fill=black!30](4,0) rectangle (5,1) node[above left] {\color{black}$1$};
			\draw[fill=black!30](6,0) rectangle (7,1) node[above left] {\color{black}$1$};
			\draw[fill=black!30] (8,0) rectangle (9,5) node[above left] {\color{black}$5$};
			\draw[fill=black!30] (11,0) rectangle (12,7) node[above left] {\color{black}$7$};	
			\draw[fill=black!30] (13,0) rectangle (14,8) node[above left] {\color{black}$8$};
			\draw[fill=black!30] (15,0) rectangle (16,9) node[above left] {\color{black}$9$};
			\draw[fill=black!30](17,0) rectangle (18,1) node[above left] {\color{black}$1$};	
			\node[below] at (1,-0.15) {$1$};
			\node[below] at (2.5,-0.15) {$2$};
			\node[below] at (4.5,-0.15) {$3$};
			\node[below] at (6.5,-0.15) {$4$};
			\node[below] at (8.5,-0.15) {$5$};
			\node[below] at (10,-0.15) {$6$};
			\node[below] at (11.5,-0.15) {$7$};
			\node[below] at (13.5,-0.15) {$8$};	
			\node[below] at (15.5,-0.15) {$9$};	
			\node[below] at (17.5,-0.15) {$10$};
			\node[above] at (0.1,11.25) {\textcolor{black}{\textbf{Tần số (n)}}};
			\node[below] at (19.5,-0.15) {\textcolor{black}{\textbf{Điểm $(x)$}}};
		\end{tikzpicture}
	\end{center}
	Tần số tương đối của điểm $8$ (làm tròn đến hàng phần trăm).
	\choice
	{\True $23{,}53\%$}
	{$23{,}52\%$}
	{$70\%$}
	{$0{,}24\%$}
	\loigiai{
	Tần số tương đối của điểm $8$ là $\dfrac{8}{2+1+1+5+7+8+9+1}\cdot 100\%\approx 23{,}53\%$.	
	}
\end{ex}
\begin{ex}%[Dự án EX-9-Đề Cương Toán 9]%[Phạm Văn Long]%[9D5N2-2]
	Quan sát biểu đồ tần số tương đối sau và cho biết đa số học sinh đánh giá đề thi môn Toán ở mức độ nào?
	\begin{center} 
		\begin{tikzpicture}[scale=0.85,line cap=round,line join=round,font=\footnotesize,>=stealth]
			\path
			(0,0) coordinate (O)
			(110:3) coordinate (A)
			(265:3) coordinate (B)
			(345:3) coordinate (C)
			(25:3) coordinate (D)
			;
			\draw [pattern = bricks,pattern color=blue,draw=none] (A) arc (110:265:3)--(O)--cycle;
			\draw [ pattern = horizontal lines, pattern color=red!50!yellow,draw=none] (B) arc (265:345:3)--(O)--cycle;
			\draw [ pattern =crosshatch, pattern color=green,draw=none] (D) arc (25:110:3)--(O)--cycle;
			\draw [ pattern =crosshatch dots,draw=none] (C) arc (-15:25:3)--(O)--cycle;
			\draw (O) circle (3);
			\node [fill=white] at (50:1.75){$25\%$};
			\node [fill=white] at (180:1.75){$45\%$};
			\node [fill=white] at (-55:1.75){$20\%$};
			\node [fill=white] at (-1:2.2){$10\%$};
			\draw[ pattern = horizontal lines, pattern color=red!50!yellow](4,-3) rectangle (5,-2.5);\node[right] at (5,-2.75){Dễ};
			\draw[pattern = bricks,pattern color=blue] (4,-2) rectangle (5,-1.5);\node[right] at (5,-1.75){Trung bình};
			\draw[ pattern = crosshatch, pattern color=green] (4,-1) rectangle (5,-0.5);\node[right] at (5,-0.75){Khó};
			\draw[ pattern = crosshatch dots] (4,0) rectangle (5,0.5);\node[right] at (5,0.25){Rất khó};
			%\foreach \x/\g in {A/-150, B/70, C/90, O/-140,D/90} \fill[black](\x)circle (1pt)+(\g:.3)node{$\x$};
		\end{tikzpicture}
	\end{center}
	\choice
	{Rất khó}
	{Khó}
	{Dễ}
	{\True Trung bình}
	\loigiai{
	Từ biểu đồ ta thấy mức độ trung bình chiếm $45\%$ là cao nhất nên đa số học sinh đánh giá đề thi môn Toán ở mức độ trung bình.
	}	
\end{ex}
\begin{ex}%[Dự án EX-9-Đề Cương Toán 9]%[Phạm Văn Long]%[9D5N3-1]
	Giáo viên ghi lại thời gian chạy cự li $100$ mét của các học sinh lớp $9$A và cho kết quả bởi bảng sau
	\begin{center}
		\begin{tabular}{|c|c|c|c|c|}
			\hline
			Thời gian (giây) & $[13;15)$ & $[15;17)$ & $[17;19)$ & $[19;21)$\\
			\hline
			Tần số & $5$ & $20$ & $13$ & $2$\\
			\hline
		\end{tabular}
	\end{center}
	Tần số của nhóm học sinh chạy cự li $100$ m với thời gian $[19;21)$ là
	\choice
	{$5$}
	{$20$}
	{\True $2$}
	{$13$}
	\loigiai{
	Tần số của nhóm học sinh chạy cự li $100$ m với thời gian $[19;21)$ là $2$.
	}
\end{ex}
\begin{ex}%[Dự án EX-9-Đề Cương Toán 9]%[Phạm Văn Long]%[9D5N3-3]
	Lớp $9A$ của một trường THCS thống kê về thời gian (phút) đi từ nhà đến trường của các bạn trong lớp. Số liệu được ghi lại trong biểu đồ tần số ghép nhóm ở hình dưới đây
	\begin{center}
		\begin{tikzpicture}[line join=round, line cap=round, >=stealth,font=\footnotesize, scale=1]%Hinh nón
			%\tikzset{every node/.style={scale=0.7}}% thu nhỏ phóng tỏ tex trong hình
			\def\h{5};
			\def\a{5.5};
			\draw[<->] (0,{\h})node[above left]{Tần số tương đối $(n)$}--(0,0)node[below left]{$0$}--(1,0) node[below]{$5$}--(2,0) node[below]{$10$}--(3,0) node[below]{$15$}--(4,0) node[below]{$20$}--(5,0) node[below]{$25$}--(\a,0) node[below right]{Thời gian (Phút)};
			\foreach \x/\y in {1/2,2/3.75,3/3,4/1.25}{
				\draw[fill=cyan] (\x,0) rectangle (\x+1,\y) node[above] at (\x+0.5,\y){};}
			%	\draw[dashed] (0,\y)node[left]{ }--(\x,\y);}
		\draw[fill=black] 
		(1.5,2)  node[above] {$20\%$}
		(2.5,3.75)  node[above] {$37{,}5\%$}
		(3.5,3)  node[above] {$30\%$}
		(4.5,1.25)  node[above] {$12{,}5\%$}
		
		(0,1) circle (0.02) node[left] {$10$}
		(0,2) circle (0.02)  node[left] {$20$}
		(0,3) circle (0.02)  node[left] {$30$}
		(0,4) circle (0.02)  node[left] {$40$}
		; 		
	\end{tikzpicture}
\end{center}
Tần số tương đối của nhóm học sinh có thời gian đi từ nhà đến trường $[10;15)$ là 
\choice
{$20\%$}
{\True $37{,}5\%$}
{$30\%$}
{$12{,}5\%$}
\loigiai{
Tần số tương đối của nhóm học sinh có thời gian đi từ nhà đến trường $[10;15)$ là $37{,}5\%$.	
}
\end{ex}
\begin{ex}%[Dự án EX-9-Đề Cương Toán 9]%[Phạm Văn Long]%[9D5H2-1]
	Trong môn Công nghệ, một lớp $9$ khảo sát về tần số sử dụng các thiết bị điện tử của các học sinh trong lớp. Kết quả được trình bày như sau
	$$\begin{array}{|c|c|c|c|c|}
		\hline
		\text{Thiết bị điện tử} & \text{Điện thoại} & \text{Máy tính} & \text{Máy tính bảng} & \text{Khác} \\ \hline
		\text{Tần số (m)} & 18 & 12 & 6 & 4 \\ \hline
	\end{array}$$
	Chọn khẳng định đúng trong các khẳng định sau.
	\choice
	{\True Tần số tương đối của học sinh sử dụng điện thoại là $45\%$}
	{Tần số tương đối của học sinh sử dụng máy tính là $15\%$}
	{Thiết bị điện tử ít được sử dụng nhất là máy tính bảng}
	{Tần số tương đối của học sinh sử dụng máy tính bảng là $20\%$}
	\loigiai{
		Tổng số học sinh tham gia khảo sát là $18 + 12 + 6 + 4 = 40$.\\
		Tần số tương đối của học sinh sử dụng điện thoại là $\dfrac{18}{40} \cdot 100\% = 45\%$.\\
		Tần số tương đối của học sinh sử dụng máy tính là $\dfrac{12}{40} \cdot 100\% = 30\%$.\\
		Tần số tương đối của học sinh sử dụng máy tính bảng là $\dfrac{6}{40} \cdot 100\% = 15\%$.\\
		Tần số tương đối của học sinh sử dụng các thiết bị khác là $\dfrac{4}{40} \cdot 100\% = 10\%$.
	}
\end{ex}
\begin{ex}%[Dự án EX-9-Đề Cương Toán 9]%[Phạm Văn Long]%[9D5H3-3]
	Thời gian đi từ nhà đến trường (đơn vị: phút) của các bạn học sinh lớp $9C$ được ghi lại ở bảng sau
	\begin{center}
		\begin{tabular}{|c|c|c|c|c|c|c|c|c|c|c|c|}
			\hline
			$9{,}5$ & $13{,}9$ & $5{,}6$ & $13{,}2$ & $10{,}3$ & $15{,}1$ & $19{,}5$ & $14{,}1$ & $11{,}4$ & $19{,}7$ & $15{,}1$ & $11{,}1$ \\
			\hline
			$16{,}6$ & $7{,}2$ & $18$ & $11{,}6$ & $6{,}2$ & $6{,}2$ & $16{,}7$ & $7{,}8$ & $17{,}7$ & $7{,}7$ & $7{,}7$ & $5{,}5$ \\
			\hline
			$18{,}2$ & $7{,}4$ & $19{,}8$ & $19$ & $5{,}2$ & $18{,}3$ & $14{,}7$ & $14{,}1$ & $19{,}6$ & $10{,}4$ & $7{,}2$ & $12{,}5$ \\
			\hline
		\end{tabular}
	\end{center}
	Khi ta chia số liệu thành $4$ nhóm, với nhóm thứ nhất là khoảng từ $5$ phút đến dưới $9$ phút. Tần số tương đối của nhóm $[13 ; 17)$ là
	\choice
	{$22\%$}
	{\True $25\%$}
	{$27{,}8\%$}
	{$30\%$}
	\loigiai{
		Chia số liệu thành $4$ nhóm, với nhóm thứ nhất là khoảng từ $5$ phút đến dưới $9$ phút ta được các nhóm là $[5 ; 9)$; $[9 ; 13)$; $[13 ; 17)$; $[17 ; 21)$.\\   
			Bảng tần số ghép nhóm là
			\begin{center}
				\begin{tabular}{|c|c|c|c|c|}
					\hline
					\makecell[c] {Thời gian\\ (phút)} & $[5; 9)$ & $[9 ; 13)$ & $[13 ; 17)$ & $[17 ; 21)$ \\
					\hline
					Tần số & $11$ & $7$ & $9$ & $9$\\
					\hline
				\end{tabular}
			\end{center}	
			Tần số tương đối của nhóm $[13 ; 17)$ là $\dfrac{9}{36} \cdot 100\% = 25\%$.			
 	}
\end{ex}
\begin{ex}%[Dự án EX-9-Đề Cương Toán 9]%[Phạm Văn Long]%[9D5V3-3]
	Bảng dưới đây ghi lại cự li ném tạ (đơn vị: mét) của một vận động viên trước và sau một đợt tập huấn đặc biệt.
	\begin{center}
		\begin{tabular}{|c|c|c|c|c|c|c|}
			\hline
			\begin{tabular}{c}
				Cự li (X)      \\
				$(\mathrm{m})$ \\
			\end{tabular} & $[20;20{,}2)$ & $[20{,}2;20{,}4)$ & $[20{,}4;20{,}6)$ & $[20{,}6;20{,}8)$ & $[20{,}8;21)$ & $[21;21{,}2)$ \\
			\hline
			\begin{tabular}{c}
				Tần số  \\
				trước đợt \\
				tập huấn  \\
			 \end{tabular} & $3$ & $5$ & $5$ & $2$ & $1$ & $0$ \\
			\hline
			\begin{tabular}{c}
			Tần số \\
			sau đợt \\
			tập huấn \\
			\end{tabular} & $1$ & $2$ & $4$ & $5$ & $3$ & $1$ \\ 
			\hline
		\end{tabular}
	\end{center}
	Tần số tương đối của số lần vận động viên ném từ $20{,}8 \mathrm{~m}$ trở lên sau khi tập huấn tăng thêm
		\choice
		{\True $18{,}75 \%$}
		{$30{,}5 \%$}
		{$35 \%$}
		{$37{,}5 \%$}
		\loigiai{
			Tần số tương đối của số lần vận động viên ném từ $20{,}8 \mathrm{~m}$ trở lên sau khi tập huấn tăng thêm $$\dfrac{4-1}{16}\cdot 100\%=18{,}75\%.$$
		}
\end{ex}
\begin{ex}%[Dự án EX-9-Đề Cương Toán 9]%[Phạm Văn Long]%[9D5H2-1]
	Cho bảng số liệu sau
	\begin{center}
		\begin{tabular}{|c|c|c|c|c|}
			\hline
			 Tần số & $24$ & $16$ & $6$ & $4$ \\ 
			\hline
			Tần số tương đối & $48 \%$ & $32 \%$ & $15 \%$ & $8 \%$ \\
			\hline
		\end{tabular}
	\end{center}
	Hãy tìm tần số có tần số tương đối bị \textbf{sai}.
	 \choice
	 {$24$}
	 {$16$}
	 {\True $6$}
	 {$4$}
	\loigiai{
		Theo bảng ta có cỡ mẫu $N=50$.\\
		Số liệu có tần số là $6$ và tần số tương đối $15\%$ là sai. Vì $\dfrac{6}{50}\cdot 100\%=12\%$.
	}
\end{ex}
\Closesolutionfile{ans}
\indapan{6}{ans/ans-9C7-OTC}
\subsection{Câu hỏi tự luận}
\begin{bt}%[SGK CTST Toán 9]%[Dự án EX-9-Đề Cương Toán 9]%[Phạm Văn Long]%[9D5V3-3]
	Khảo sát các học sinh lớp $6$ của một trường Trung học cơ sở về thời gian sử dụng mạng xã hội trung bình trong một ngày (đơn vị: giờ), kết quả thu được như hình bên dưới.
	\begin{enumerate}
		\item Có bao nhiêu bạn tham gia cuộc khảo sát, biết rằng có $4$ bạn sử dụng mạng xã hội từ $4{,}5$ giờ trở lên.
		\item Một người cho rằng có trên $50 \%$ học sinh tham gia khảo sát sử dụng mạng xã hội từ $3$ giờ trở lên mỗi ngày. Nhận định của người đó có hợp lí không? Tại sao?
	\end{enumerate}
	\begin{center}
		\begin{tikzpicture}[scale=1.3, font=\footnotesize, line join=round, line cap=round, >=stealth]
			\draw[->] (0,0) -- (0,6);
			\path (0,6) node[above] {Tần số tương đối $\%$};
			\draw[->] (0,0) -- (5,0);
			\path (5.2,-.5) node[above]{Thời gian (giờ)};
			\foreach \y/\z in {.5/5, 1/10, 1.5/15, 2/20, 2.5/25, 3/30, 3.5/35, 4/40, 4.5/45, 5/50, 5.5/55}{
				\draw[dashed, thin, blue!40]
				(5, \y)--(0,\y);
				\fill (0,\y) node[left]{$\z$};
			}
			\foreach \tanso/\x[count=\i from 1] in {4.67/$46{,}7\%$, 4/$40\%$, 1/$10\%$, .33/$3{,}3\%$}{
				%Định nghĩa giá trị hiển thị dạng phần trăm
				\pgfmathsetmacro{\httanso}{10*\tanso}
				\edef\roundedNumber{\fpeval{round(\httanso, 1)}}
				%Vẽ các cột hình chữ nhật
				\draw[pattern=dots]
				(\i-1,0) rectangle (\i,\tanso)
				(\i-0.5,\tanso) node[above]{$\x$}
				;}
			\foreach \x[count=\i from 1] in {0, 1.5, 3, 4.5, 6}{
				\path
				(\i-1,0) node[below]{$\x$};}
			% Thêm nhãn nằm giữa phía trên
			\node[above, align=center] at (current bounding box.north) {\textbf{Tần số tương đối của số người theo thời gian sử dụng mạng xã hội mỗi ngày}};
		\end{tikzpicture}
	\end{center}
	\loigiai{
		\begin{enumerate}
			\item Theo đề bài có $4$ bạn sử dụng mạng xã hội từ $4{,}5$ giờ trở lên, ứng với $3{,}3\%$ do đó $m=4$ và $f=3{,}3\%$.\\
			Suy ra $\dfrac{4}{N}=3{,}3\%$ nên $N=121$ (học sinh).
			\item Theo biểu đồ có $10\%+3{,}3\%=13{,}3\%$ sử dụng từ $3$ giờ trở lên. Nên nhận định có trên $50\%$ học sinh tham gia khảo sát sử dụng mạng xã hội từ $3$ giờ trở lên mỗi ngày là sai.
	\end{enumerate}}
\end{bt}
\begin{bt}%[SGK CTST Toán 9]%[Dự án EX-9-Đề Cương Toán 9]%[Phạm Văn Long]%[9D5V3-3]
	Một cửa hàng ghi lại cỡ của các đôi giày đã bán trong một ngày ở bảng sau
	\begin{center}
		\begin{tabular}{|l|l|l|l|l|l|l|l|l|l|}
			 \hline
			$42$ & $38$ & $39$ & $42$ & $39$ & $41$ & $38$ & $41$ & $41$ & $40$ \\
			\hline
			$37$ & $38$ & $37$ & $38$ & $40$ & $39$ & $38$ & $39$ & $44$ & $43$ \\
			\hline
			$42$ & $37$ & $40$ & $40$ & $44$ & $41$ & $41$ & $40$ & $42$ & $39$ \\
			\hline
			$43$ & $41$ & $37$ & $41$ & $40$ & $38$ & $40$ & $41$ & $40$ & $39$ \\
			\hline 
		\end{tabular}
	\end{center}
	\begin{enumerate}
		\item Hãy xác định cỡ mẫu, lập bảng tần số và tần số tương đối cho mẫu số liệu trên.
		\item Hãy vẽ biểu đồ dạng cột mô tả bảng số liệu trên.
		\item Cửa hàng nên nhập về để bán cỡ giày nào nhiều nhất, cỡ giày nào ít nhất?
	\end{enumerate}
	\loigiai{
		\begin{enumerate}
			\item Cỡ mẫu $N=40$.\\
			Ta có bảng tần số - tần số tương đối của mẫu số liệu trên là
			\begin{center}
				\begin{tabular}{|l|c|c|c|c|c|c|c|c|c|}
					\hline
					Cỡ giày & $37$ & $38$ & $39$ & $40$ & $41$ & $42$ & $43$ & $44$ & Tổng\\
					\hline
					Tần số & $4$ & $6$ & $6$ & $8$ & $8$ & $4$ & $2$ & $2$ & $N=40$\\ 
					\hline
					Tần số tương đối & $10\%$ & $15\%$ & $15\%$ & $20\%$ & $20\%$ & $10\%$ & $5\%$ & $5\%$ & $100\%$\\
					\hline
				\end{tabular}
			\end{center}
			\item Ta có biểu đồ dạng cột mô tả bảng số liệu trên như sau
			\begin{center}
				\begin{tikzpicture}[scale=1.3, font=\footnotesize, line join=round, line cap=round, >=stealth]
					\draw[->] (0,0) -- (0,5);
					\path (0,5.1) node[above] {Tần số tương đối \%};
					\draw[->] (0,0) -- (10,0);
					\path (10,-.6) node[above]{Cỡ giầy};
					\foreach \y/\z in {.5/5, 1/10, 1.5/15, 2/20, 2.5/25, 3/30, 3.5/35, 4/40}{
						\draw[dashed, thin, line width=.01pt]
						(10, \y)--(0,\y) node[left]{$\z$};
					}
					\foreach \tanso[count=\i from 1] in {1.0, 1.5, 1.5, 2, 2, 1, 0.5, 0.5}{
						%ĐỊnh nghĩa giá trị hiển thị dạng phần trăm
						\pgfmathsetmacro{\httanso}{int(round(10*\tanso))}
						%Vẽ các cột hình chữ nhật
						\draw[pattern=dots]
						(\i,0) rectangle (\i+1,\tanso)
						(\i+0.5,\tanso) node[above]{$\httanso\%$}
						;}
					\foreach \x[count=\i from 1] in {37, 38, 39, 40, 41, 42, 43, 44}{
						\path
						(\i+.5,0) node[below]{$\x$};}
					\fill (5.2,5.5) node[above]{\textbf{Tần số tương đối của các đôi giày đã bán trong một ngày tại cửa hàng}};
				\end{tikzpicture}
			\end{center}
			\item Dựa vào bảng tần số, ta thấy
			\begin{itemize}
				\item Cỡ giày xuất hiện nhiều nhất là $40$, $41$ (mỗi cỡ xuất hiện $8$ lần).
				\item Cỡ giày xuất hiện ít nhất là $43$, $44$ (mỗi cỡ xuất hiện $2$ lần).
			\end{itemize}
			Do đó của hàng nên nhập nhiều nhất là cỡ giầy $40$ và $41$, nhập ít nhất là cỡ giầy $43$ và $44$.
		\end{enumerate}
	}
\end{bt}
\begin{bt}%[SGK CTST Toán 9]%[Dự án EX-9-Đề Cương Toán 9]%[Phạm Văn Long]%[9D5H3-3]
	Một bác lái xe muốn ghi lại tổng độ dài quãng đường (đơn vị: km) mình lái xe mỗi ngày trong vòng $1$ tháng.
	\begin{enumerate}
		\item Hỏi bác lái xe có thể thu thập dữ liệu bằng cách nào?
		\item Dưới đây là số liệu bác lái xe đã ghi lại được.
	\end{enumerate}
	\begin{center}
		\begin{tabular}{|c|c|c|c|c|c|c|c|c|c|}
			\hline
			$23{,}9$ & $192{,}7$ & $137{,}8$ & $125{,}3$ & $147{,}5$ & $102{,}8$ & $105{,}9$ & $60{,}1$ & $186{,}7$ & $129{,}5$ \\
			\hline
			$31{,}6$ & $168{,}4$ & $97{,}4$  & $144{,}7$ & $129$ & $197{,}3$ & $113{,}7$ & $10{,}2$ & $110{,}3$ & $86{,}4$ \\
			\hline
			$77{,}9$ & $38{,}6$ & $124{,}7$ & $199{,}8$ & $22{,}8$  & $96{,}9$ & $30{,}7$ & $85{,}1$ & $188{,}1$ & $122{,}5$ \\
			\hline
		\end{tabular}
	\end{center}
	Hãy chia số liệu thành $5$ nhóm, với nhóm thứ nhất là từ $10$ km đến dưới $50$ km và lập bảng tần số ghép nhóm và tần số tương đối ghép nhóm. Vẽ biểu đồ tần số tương đối ghép nhóm dạng cột biểu diễn bảng tần số tương đối ghép nhóm.
	\loigiai{
		\begin{enumerate}
			\item Có thể ghi lại dưới dạng bảng $3$ dòng $10$ cột tương ứng với $30$ ngày.
			\item Bảng tần số ghép nhóm - tần số tương đối ghép nhóm của mẫu số liệu đã cho là
		\end{enumerate}
		\begin{center}
			\begin{tabular}{|l|c|c|c|c|c|c|}
				\hline
				Quãng đường & $[10;50)$ & $[50;90)$ & $[90;130)$ & $[130;170)$ & $[170;210)$ & Tổng\\
				\hline
				Tần số & $6$ & $4$ & $11$ & $4$ & $5$ & $N=30$ \\
				\hline
				Tần số tương đối & $20\%$ & $13\%$ & $37\%$ & $13\%$ & $17\%$ & $100\%$ \\
				\hline
			\end{tabular}
		\end{center}
		Biểu đồ tần số tương đối ghép nhóm dạng cột biểu diễn bảng tần số tương đối ghép nhóm là
		\begin{center}
			\begin{tikzpicture}[scale=1, font=\footnotesize, line join=round, line cap=round, >=stealth]
				\draw[->] (0,0) -- (0,4.5);
				\path (0,4.6) node[above, rotate=0] {Tần số tương đối \%};
				\draw[->] (0,0) -- (8,0);
				\path (9,-.8) node[above]{Quãng đường (km)};
				\foreach \y/\z in {.5/5, 1/10, 1.5/15, 2/20, 2.5/25, 3/30, 3.5/35, 4/40}{
					\draw[dashed, thin, line width=.01pt]
					(6.5, \y)--(0,\y) node[left]{$\z$};
				}
				\foreach \tanso[count=\i from 1] in {2, 1.3, 3.7, 1.3, 1.7}{
					%ĐỊnh nghĩa giá trị hiển thị dạng phần trăm
					\pgfmathsetmacro{\httanso}{int(round(10*\tanso))}
					%Vẽ các cột hình chữ nhật
					\draw[pattern=dots]
					(\i,0) rectangle (\i+1,\tanso)
					(\i+0.5,\tanso) node[above]{$\httanso\%$}
					;}
				\foreach \x[count=\i from 1] in {10, 50, 90, 130, 170, 210}{
					\path
					(\i,0) node[below]{$\x$};}
				\fill (4.2,5) node[above]{\textbf{Tần số tương đối của số ngày theo độ dài quãng đường đi được mỗi ngày}} ;
			\end{tikzpicture}
		\end{center}
	}
\end{bt}

%============ BT bổ sung 

\begin{bt}%[SGK CTST Toán 9]%[Dự án EX-9-Đề Cương Toán 9]%[9D5H2-2]
	Số bàn thắng một đội bóng ghi được trong $26$ trận đấu của Giải vô địch quốc gia được ghi lại ở bảng sau
	\begin{center}
		\begin{tabular}{|l|l|l|l|l|l|l|l|l|l|l|l|l|}
			\hline
			$1$ & $2$ & $0$ & $4$ & $0$ & $3$ & $0$ & $1$ & $0$ & $0$ & $3$ & $3$ & $0$ \\
			\hline
			$0$ & $3$ & $0$ & $2$ & $2$ & $3$ & $3$ & $4$ & $3$ & $1$ & $0$ & $0$ & $3$ \\
			\hline
		\end{tabular}
	\end{center}
	\begin{enumerate}
		\item Hãy lập bảng tần số và tần số tương đối cho bảng số liệu trên.
		\item Hãy vẽ biếu đồ hình quạt tròn mô tả tần số tương đối của bảng số liệu trên.
	\end{enumerate}
	\loigiai{
		\begin{enumerate}
			\item Bảng tần số - tần số tương đối cho bảng số liệu trên như sau.
			\begin{center}
				\begin{tabular}{|c|c|c|c|c|c|}
					\hline
					Số bàn & $0$ & $1$ & $2$ & $3$ & $4$ \\
					\hline
					Tần số & $10$ & $3$ & $3$ & $8$ & $2$ \\
					\hline
					Tần số tương đối & $38{,}5\%$ & $11{,}5\%$ & $11{,}5\%$ & $30{,}8\%$ & $8\%$ \\
					\hline
				\end{tabular}
			\end{center}
			\item Biểu đồ hình quạt tròn mô tả tần số tương đối của bảng số liệu trên là
			\begin{center}
				\begin{tikzpicture}[scale=1,font=\small,line join=round,line cap=round,>=stealth]
					\def\r{2.5}
					\def\gocxp{90}
					\coordinate (A) at (90:\r);
					\foreach \val/\freq/\col/\pattern/\rate[count=\i from 0] in{
						0/38.5/gray/grid/$38{,}5\%$,
						1/11.5/gray/fivepointed stars/$11{,}5\%$,
						2/11.5/gray/vertical lines/$11{,}5\%$,
						3/30.8/gray/dots/$30{,}8\%$, 4/8/gray/horizontal lines/$8\%$}{
						\pgfmathsetmacro\gockt{-(\freq*3.60-\gocxp)}
						\pgfmathsetmacro\gocnode{\gocxp+\gockt}
						\draw[pattern = \pattern,pattern color=\col] (0,0)--(A) arc(\gocxp:\gockt:\r) coordinate(A)--cycle;
						\filldraw[pattern = \pattern,pattern color=\col] (\r+0.5,0.8*\r-0.8*\i) --++(0:.5)--++(-90:.4) node[pos=.5,right,black]{\val}--++(180:.5)--cycle;
						\path ($(0,0)+(\gocnode/2:\r/1.5)$) 
						node[fill=white,inner sep=0.5pt,rectangle]{\color{black} \rate};
						\global\let\gocxp=\gockt
					}
					\fill (0,2.8) node[above]{\textbf{Số bàn thắng một đội bóng ghi được trong 26 trận đấu của Giải vô địch quốc gia}};
				\end{tikzpicture}
			\end{center}
		\end{enumerate}
	}
\end{bt}

\begin{bt}%[SBT CD Toán 9]%[Dự án EX-9-Đề Cương Toán 9]%[9D5H1-1]
	Tổng điểm mà các thành viên đội tuyển Olympic Toán quốc tế (IMO - hình thức thi trực tiếp) của Việt Nam đạt được trong các năm $2008$, $2009$, $2010$, $2011$, $2012$, $2013$, $2014$, $2015$, $2016$, $2017$, $2018$, $2019$, $2020$, $2022$, $2023$ được thống kê lần lượt như sau: $159$; $161$; $133$; $113$; $148$; $180$; $157$; $151$; $151$; $155$; $148$; $177$; $150$; $196$; $180$. (\textit{Nguồn: https://imo-oficial.org})
	\begin{enumerate}
		\item Nêu các đối tượng thống kê và cho biết có bao nhiêu số liệu thống kê ở trên.
		\item Trong các số liệu thống kê ở trên có bao nhiêu giá trị khác nhau?
	\end{enumerate}
	\loigiai{
		\begin{enumerate}
			\item Các đối tượng thống kê là: Tổng điểm mà các thành viên đội tuyển Olympic Toán quốc tế (IMO) của Việt Nam đạt được trong các năm $2008$, $2009$, $2010$, $2011$, $2012$, $2013$, $2014$, $2015$, $2016$, $2017$, $2018$, $2019$, $2020$, $2022$, $2023$.\\
			Vậy có $15$ số liệu thống kê.
			\item Các số liệu thống kê đó có $12$ giá trị khác nhau là $x_1 = 113$; $x_2 = 133$; $x_3 = 148$; $x_4 = 150$; $x_5 = 151$; $x_6 = 155$; $x_7 = 157$; $x_8 = 159$; $x_9 = 161$; $x_{10} = 177$; $x_{11} = 180$; $x_{12} = 196$.
		\end{enumerate}
	}
\end{bt}

\begin{bt}%[SBT CD Toán 9]%[Dự án EX-9-Đề Cương Toán 9]%[9D5H1-2]
	Thống kê lượng hàng bán được (đơn vị: chiếc) của $39$ mặt hàng ở một siêu thị điện máy như sau
	\begin{center}
		\begin{tabular}{c c c c c c c c c c c c c}
			$4$ & $7$ & $4$ & $5$ & $7$ & $9$ & $5$ & $10$ & $4$ & $8$ & $5$ & $8$ & $9$ \\
			$9$ & $5$ & $10$ & $8$ & $4$ & $5$ & $7$ & $5$ & $10$ & $9$ & $4$ & $7$ & $4$ \\
			$8$ & $9$ & $8$ & $4$ & $9$ & $8$ & $9$ & $5$ & $10$ & $5$ & $7$ & $10$ & $9$
		\end{tabular}
	\end{center}
	\begin{enumerate}
		\item Lập bảng tần số của mẫu số liệu thống kê đó.
		\item Vẽ biểu đồ tần số ở dạng biểu đồ cột của mẫu số liệu thống kê đó.
	\end{enumerate}
	\loigiai{
		\begin{enumerate}
			\item Ta có bảng tần số của mẫu số liệu thống kê đó là 
			\begin{center}
				\begin{tabular}{|>{\centering\arraybackslash}m{2.5cm}|>{\centering\arraybackslash}m{1cm}|>{\centering\arraybackslash}m{1cm}|>{\centering\arraybackslash}m{1cm}|>{\centering\arraybackslash}m{1cm}|>{\centering\arraybackslash}m{1cm}|>{\centering\arraybackslash}m{1cm}|>{\centering\arraybackslash}m{1.5cm}|}
					\hline
					\textbf{Số chiếc $(x)$} & $4$ & $5$ & $7$ & $8$ & $9$ & $10$ & Cộng \\ \hline
					\textbf{Tần số $(n)$} & $7$ & $8$ & $5$ & $6$ & $8$ & $5$ &  $N=39$ \\ \hline
				\end{tabular} 
			\end{center}
			\item Biểu đồ tần số ở dạng biểu đồ cột của mẫu số liệu thống kê đó là
			\begin{center}
				\begin{tikzpicture}[y = 7.5mm, scale = 0.7, transform shape]
					\path (0,0) coordinate (O) node[below left]{$0$}
					;
					\draw[->] (0,0)--(0,9) node[left]{Tần số $(n)$};
					\draw[->] (0,0)--(9,0) node[below]{Số chiếc};
					\foreach \giatri/\tanso [count=\i from 1] in{4/7, 5/8, 7/5, 8/6, 9/8, 10/5}{
						\draw[fill=gray!30, ->] (\i,0) rectangle (\i + 0.7, \tanso) 
						(\i + .35,0) node[below]{$\giatri$} ;
						\draw[dashed] (\i, \tanso)--(0, \tanso) ;
						\global \let\n=\i }
					\foreach \i in {1,2,...,8}{\path (0,\i) node{\tikz{\draw[shift={(2pt,0)}](0:2pt)--(180:2pt)}};}
					\foreach \x in {1,2,...,8}{\path (0,\x) node[left]{$\x$} ;}
				\end{tikzpicture} 
			\end{center}
		\end{enumerate}
	}
\end{bt}

\begin{bt}%[SBT CD Toán 9]%[Dự án EX-9-Đề Cương Toán 9]%[9D5H1-2]
	Bác An ghi lại số cuộc điện thoại bác đã gọi trong mỗi ngày của tháng $2$ năm $2023$ như sau
	\begin{center}
		\begin{tabular}{c c c c c c c c c c c c c c}
			$5$ & $5$ & $8$ & $7$ & $8$ & $7$ & $9$ & $10$ & $10$ & $7$ & $7$ & $5$ & $9$ & $5$ \\
			$10$ & $8$ & $8$ & $7$ & $9$ & $7$ & $5$ & $10$ & $8$ & $8$ & $10$ & $9$ & $7$ & $10$
		\end{tabular}
	\end{center}
	\begin{enumerate}
		\item Lập bảng tần số của mẫu số liệu thống kê đó.
		\item Vẽ biểu đồ tần số ở dạng biểu đồ cột của mẫu số liệu thống kê đó.
	\end{enumerate}
	\loigiai{
		\begin{enumerate}
			\item Ta có bảng tần số của mẫu số liệu thống kê đó là 
			\begin{center}
				\begin{tabular}{|>{\centering\arraybackslash}m{5cm}|>{\centering\arraybackslash}m{1cm}|>{\centering\arraybackslash}m{1cm}|>{\centering\arraybackslash}m{1cm}|>{\centering\arraybackslash}m{1cm}|>{\centering\arraybackslash}m{1cm}|>{\centering\arraybackslash}m{1.5cm}|}
					\hline
					\textbf{Số cuộc điện thoại $(x)$}  & $5$ & $7$ & $8$ & $9$ & $10$ & Cộng \\ \hline
					\textbf{Tần số $(n)$}  & $5$ & $7$ & $6$ & $4$ & $6$ &  $N=28$ \\ \hline
				\end{tabular} 
			\end{center}
			\item Biểu đồ tần số ở dạng biểu đồ cột của mẫu số liệu thống kê đó là
			\begin{center}
				\begin{tikzpicture}[y = 7.5mm, scale = 0.7, transform shape]
					\path (0,0) coordinate (O) node[below left]{$0$}
					;
					\draw[->] (0,0)--(0,8) node[left]{Tần số $(n)$};
					\draw[->] (0,0)--(9,0) node[below]{Số cuộc điện thoại};
					\foreach \giatri/\tanso [count=\i from 1] in{5/5, 7/7, 8/6, 9/4, 10/6}{
						\draw[fill=gray!30, ->] (\i,0) rectangle (\i + 0.7, \tanso) 
						(\i + .35,0) node[below]{$\giatri$} ;
						\draw[dashed] (\i, \tanso)--(0, \tanso) ;
						\global \let\n=\i }
					\foreach \i in {1,2,...,7}{\path (0,\i) node{\tikz{\draw[shift={(2pt,0)}](0:2pt)--(180:2pt)}};}
					\foreach \x in {1,2,...,7}{\path (0,\x) node[left]{$\x$} ;}
				\end{tikzpicture} 
			\end{center}
		\end{enumerate}
	}
\end{bt}

\begin{bt}%[SBT CD Toán 9]%[Dự án EX-9-Đề Cương Toán 9]%[9D5H1-2]
	Thống kê số quyển sách quyên góp ủng hộ thư viện nhà trường của $100$ học sinh khối $9$ như sau
	\begin{center}
		\begin{tabular}{p{1.5mm} p{1.5mm} p{1.5mm} p{1.5mm} p{1.5mm} p{1.5mm} p{1.5mm} p{1.5mm} p{1.5mm} p{1.5mm} p{1.5mm} p{1.5mm} p{1.5mm} p{1.5mm} p{1.5mm} p{1.5mm} p{1.5mm} p{1.5mm} p{1.5mm} p{1.5mm}}
			$50$ & $38$ & $35$ & $38$ & $50$ & $38$ & $27$ & $38$ & $47$ & $27$ & $27$ & $35$ & $38$ & $32$ & $38$ & $32$ & $35$ & $32$ & $35$ & $32$\\
			$38$ & $38$ & $35$ & $32$ & $35$ & $38$ & $38$ & $50$ & $32$ & $47$ & $27$ & $38$ & $35$ & $27$ & $47$ & $35$ & $38$ & $38$ & $32$ & $35$ \\
			$35$ & $35$ & $27$ & $32$ & $38$ & $35$ & $32$ & $32$ & $38$ & $32$ & $38$ & $35$ & $27$ & $38$ & $27$ & $38$ & $27$ & $32$ & $38$ & $38$ \\
			$38$ & $32$ & $38$ & $32$ & $35$ & $27$ & $35$ & $38$ & $32$ & $27$ & $50$ & $32$ & $27$ & $35$ & $47$ & $32$ & $38$ & $27$ & $32$ & $32$ \\
			$38$ & $27$ & $35$ & $38$ & $35$ & $47$ & $35$ & $38$ & $35$ & $38$ & $35$ & $35$ & $35$ & $35$ & $35$ & $27$ & $50$ & $38$ & $32$ & $38$ \\
		\end{tabular}
	\end{center}
	\begin{enumerate}
		\item Trong $100$ số liệu thống kê ở trên, có bao nhiêu giá trị khác nhau?
		\item Lập bảng tần số tương đối của mẫu số liệu thống kê đó.
		\item Vẽ biểu đồ tần số tương đối ở dạng biểu đồ cột của mẫu số liệu thống kê đó.
	\end{enumerate}
	\loigiai{
		\begin{enumerate}
			\item Trong $100$ số liệu thống kê ở trên có $6$ giá trị khác nhau là:
			\[x_1=27; x_2 = 32; x_3 = 35; x_4 = 38 ; x_5 = 47 ; x_6 = 50.\]
			\item Ta có bảng tần số tương đối của mẫu số liệu thống kê đó là 
			\begin{center}
				\begin{tabular}{|>{\centering\arraybackslash}m{5cm}|>{\centering\arraybackslash}m{1cm}|>{\centering\arraybackslash}m{1cm}|>{\centering\arraybackslash}m{1cm}|>{\centering\arraybackslash}m{1cm}|>{\centering\arraybackslash}m{1cm}|>{\centering\arraybackslash}m{1cm}|>{\centering\arraybackslash}m{1.5cm}|}
					\hline
					\textbf{Số quyển sách quyên góp $(x)$}  & $27$ & $32$ & $35$ & $38$ & $47$ & $50$ & Cộng \\ \hline
					\textbf{Tần số tương đối $(\%)$}  & $15$ & $20$ & $25$ & $30$ & $5$ & $5$ &  $100$ \\ \hline
				\end{tabular} 
			\end{center}
			\item Biểu đồ tần số tương đối ở dạng cột của mẫu số liệu thống kê đó là
			\begin{center}
				\begin{tikzpicture}[y = 7.5mm, scale = 0.7, transform shape]
					\path (0,0) coordinate (O) node[below left]{$0$}
					;
					\draw[->] (0,0)--(0,8) node[right]{Tần số tương đối $(\%)$};
					\draw[->] (0,0)--(10,0) node[below]{Số quyển sách quyên góp};
					\foreach \giatri/\tanso [count=\i from 1] in{27/15, 32/20, 35/25, 38/30, 47/5, 50/5}{
						\draw[fill=gray!30, ->] (\i,0) rectangle (\i + 0.7, \tanso /5) 
						(\i + .35,0) node[below]{\giatri} ;
					}
					\foreach \giatri/\tanso [count=\i from 1] in{27/15, 32/20, 35/25, 38/30, 47/5, 50/5}{
						\path (\i,\tanso/5)--(\i + 0.7, \tanso /5) node[midway,above]{$\tanso$} ;
					}
					\foreach \i in {1,2,...,7}{\path (0,\i) node{\tikz{\draw[shift={(2pt,0)}](0:2pt)--(180:2pt)}};}
					\foreach \x in {5,10,...,35}{\path (0,\x/5) node[left]{$\x$} ;}
				\end{tikzpicture} 
			\end{center}
		\end{enumerate}
	}
\end{bt}

\begin{bt}%[SBT CD Toán 9]%[Dự án EX-9-Đề Cương Toán 9]%[9D5H1-2]
	Điểm kiểm tra môn Toán của $200$ học sinh khối $9$ được thống kê như bảng sau
	\begin{center}
		\begin{tabular}{|>{\centering\arraybackslash}m{2.5cm}|>{\centering\arraybackslash}m{1cm}|>{\centering\arraybackslash}m{1cm}|>{\centering\arraybackslash}m{1cm}|>{\centering\arraybackslash}m{1cm}|>{\centering\arraybackslash}m{1cm}|>{\centering\arraybackslash}m{1cm}|}
			\hline
			\textbf{Điểm} & $5$ & $6$ & $7$ & $8$ & $9$ & $10$   \\ \hline
			\textbf{Số học sinh} & $30$ & $40$ & $50$ & $35$ &  $25$ & $20$ \\ \hline
		\end{tabular}
	\end{center}
	\begin{enumerate}
		\item Lập bảng tần số tương đối của mẫu số liệu thống kê đó.
		\item Vẽ biểu đồ tần số tương đối (ở dạng biểu đồ cột và biểu đồ hình quạt tròn) của mẫu số liệu thống kê đó.
	\end{enumerate}
	\loigiai{
		\begin{enumerate}
			\item Ta có bảng tần số tương đối của mẫu số liệu thống kê đó là 
			\begin{center}
				\begin{tabular}{|>{\centering\arraybackslash}m{5cm}|>{\centering\arraybackslash}m{1cm}|>{\centering\arraybackslash}m{1cm}|>{\centering\arraybackslash}m{1cm}|>{\centering\arraybackslash}m{1cm}|>{\centering\arraybackslash}m{1cm}|>{\centering\arraybackslash}m{1cm}|>{\centering\arraybackslash}m{1.5cm}|}
					\hline
					\textbf{Điểm $(x)$}  & $5$ & $6$ & $7$ & $8$ & $9$ & $10$ & Cộng \\ \hline
					\textbf{Tần số tương đối $(\%)$}  & $15$ & $20$ & $25$ & $17{,}5$ & $12{,}5$ & $10$ &  $100$ \\ \hline
				\end{tabular} 
			\end{center}
			\item Biểu đồ tần số tương đối ở dạng biểu đồ cột và biểu đồ hình quạt tròn của mẫu số liệu thống kê đó là
			\begin{center}
				\begin{tikzpicture}[scale=1, font=\footnotesize, line join=round, line cap=round, >=stealth]
					\begin{scope}[shift={(0,0)}]
						\path (0,0) coordinate (O) node[below left]{$0$}
						;
						\draw[->] (0,0)--(0,6.5) node[above]{Tần số tương đối $(\%)$};
						\draw[->] (0,0)--(8,0) node[below]{Điểm};
						\foreach \giatri/\tanso [count=\i from 1] in{5/15, 6/20, 7/25, 8/17.5, 9/12.5, 10/10}{
							\draw[fill=gray!30, ->] (\i,0) rectangle (\i + 0.7, \tanso /5) 
							(\i + .35,0) node[below]{$\giatri$};
						}
						\foreach \giatri/\tanso [count=\i from 1] in{5/15, 6/20, 7/25}{
							\path (\i,\tanso/5)--(\i + 0.7, \tanso /5) node[midway,above]{$\tanso$} ;
						}
						\path 
						(4,3.5)--(4.7,3.5) node[midway,above]{$17{,}5$}
						(5,2.5)--(5.7,2.5) node[midway,above]{$12{,}5$}
						(6,2)--(6.7,2) node[midway,above]{$10$}
						;
						\foreach \i in {1,2,...,5}{\path (0,\i) node{\tikz{\draw[shift={(2pt,0)}](0:2pt)--(180:2pt)}};}
						\foreach \x in {5,10,...,25}{\path (0,\x/5) node[left]{$\x$} ;}
					\end{scope}
					\begin{scope}[shift={(11,3)},scale=1]
						\def\r{2.5}
						\def\gocxp{90}
						\coordinate (A) at (90:\r);
						\foreach \val/\freq/\col/\pattern/\rate[count=\i from 0] in{
							5/15/gray/grid/$15\%$,
							6/20/gray/fivepointed stars/$20\%$,
							7/25/gray/vertical lines/$25\%$,
							8/17.5/gray/dots/$17{,}5\%$, 9/12.5/gray/horizontal lines/$12{,}5\%$,
							10/10/gray/north west lines/$10\%$}{
							\pgfmathsetmacro\gockt{-(\freq*3.60-\gocxp)}
							\pgfmathsetmacro\gocnode{\gocxp+\gockt}
							\draw[pattern = \pattern,pattern color=\col] (0,0)--(A) arc(\gocxp:\gockt:\r) coordinate(A)--cycle;
							\filldraw[pattern = \pattern,pattern color=\col] (\r+0.5,0.8*\r-0.8*\i) --++(0:.5)--++(-90:.4) node[pos=.5,right,black]{\val}--++(180:.5)--cycle;
							\path ($(0,0)+(\gocnode/2:\r/1.5)$) 
							node[fill=white,inner sep=0.5pt,rectangle]{\color{black} \rate};
							\global\let\gocxp=\gockt
						}
					\end{scope}
				\end{tikzpicture} 
			\end{center}
		\end{enumerate}
	}
\end{bt}

\begin{bt}%[SBT CD Toán 9]%[Dự án EX-9-Đề Cương Toán 9]%[9D5H1-2]
	Khối lượng thức ăn trung bình (đơn vị: gam) trong một ngày cho mỗi con lợn $50$ kg của một số hộ gia đình được thống kê như sau 
	\begin{center}
		\begin{tabular}{c c c c c c c c c c}
			$2\,200$ & $2\,100$ & $2\,150$ & $2\,100$ & $2\,100$ & $2\,150$ & $2\,200$ & $2\,100$ & $2\,100$ & $2\,050$ \\
			$2\,100$ & $2\,200$ & $2\,050$ & $2\,050$ & $2\,100$ & $2\,100$ & $2\,150$ & $2\,200$ & $2\,150$ & $2\,200$ \\
			$2\,200$ & $2\,050$ & $2\,150$ & $2\,100$ & $2\,200$ & $2\,200$ & $2\,150$ & $2\,100$ & $2\,150$ & $2\,100$ \\
			$2\,100$ & $2\,200$ & $2\,150$ & $2\,150$ & $2\,100$ & $2\,200$ & $2\,050$ & $2\,100$ & $2\,100$ & $2\,150$ \\
			$2\,100$ & $2\,100$ & $2\,200$ & $2\,150$ & $2\,200$ & $2\,050$ & $2\,050$ & $2\,200$ & $2\,100$ & $2\,150$ \\
		\end{tabular}
	\end{center}
	\begin{enumerate}
		\item Lập bảng tần số tương đối của mẫu số liệu thống kê đó. \\
		\item Vẽ biểu đồ tần số tương đối (ở dạng biểu đồ cột và biểu đồ hình quạt tròn) của mẫu số liệu thống kê đó.
	\end{enumerate}
	\loigiai{
		\begin{enumerate}
			\item Ta có bảng tần số tương đối của mẫu số liệu thống kê đó là 
			\begin{center}
				\begin{tabular}{|>{\centering\arraybackslash}m{5cm}|>{\centering\arraybackslash}m{1cm}|>{\centering\arraybackslash}m{1cm}|>{\centering\arraybackslash}m{1cm}|>{\centering\arraybackslash}m{1cm}|>{\centering\arraybackslash}m{1.5cm}|}
					\hline
					\textbf{Số gam thức ăn $(x)$}  & $2\,050$ & $2\,100$ & $2\,150$ & $2\,200$ & Cộng \\ \hline
					\textbf{Tần số tương đối $(\%)$}  & $14$ & $36$ & $24$ & $26$  &  $100$ \\ \hline
				\end{tabular} 
			\end{center}
			\item Biểu đồ tần số tương đối ở dạng hình cột và hình quạt tròn của mẫu số liệu thống kê đó là
			\begin{center}
				\begin{tikzpicture}[scale=1, font=\footnotesize, line join=round, line cap=round, >=stealth]
					\begin{scope}[shift={(0,0)}]
						\path (0,0) coordinate (O) node[below left]{$0$}
						;
						\draw[->] (0,0)--(0,8.5) node[above]{Tần số tương đối $(\%)$};
						\draw[->] (0,0)--(7,0) node[below]{Số gam thức ăn};
						\foreach \giatri/\tanso [count=\i from 1] in{2050/14, 2100/36, 2150/24, 2200/26}{
							\draw[fill=gray!30, ->] (\i,0) rectangle (\i + 0.7, \tanso /5) 
							(\i + .35,0) node[below]{$\giatri$};
						}
						\foreach \giatri/\tanso [count=\i from 1] in{2050/14, 2100/36, 2150/24, 2200/26}{
							\path (\i,\tanso/5)--(\i + 0.7, \tanso /5) node[midway,above]{$\tanso$} ;
						}
						\foreach \i in {1,2,...,8}{\path (0,\i) node{\tikz{\draw[shift={(2pt,0)}](0:2pt)--(180:2pt)}};}
						\foreach \x in {5,10,...,40}{\path (0,\x/5) node[left]{$\x$} ;}
					\end{scope}
					\begin{scope}[shift={(11.2,3)},scale=1]
						\def\r{2.5}
						\def\gocxp{90}
						\coordinate (A) at (90:\r);
						\foreach \val/\freq/\col/\pattern/\rate[count=\i from 0] in{
							2050/14/gray/grid/$14\%$,
							2100/36/gray/fivepointed stars/$36\%$,
							2150/24/gray/vertical lines/$24\%$,
							2200/26/gray/dots/$26\%$}{
							\pgfmathsetmacro\gockt{-(\freq*3.60-\gocxp)}
							\pgfmathsetmacro\gocnode{\gocxp+\gockt}
							\draw[pattern = \pattern,pattern color=\col] (0,0)--(A) arc(\gocxp:\gockt:\r) coordinate(A)--cycle;
							\filldraw[pattern = \pattern,pattern color=\col] (\r+0.5,0.8*\r-0.8*\i) --++(0:.5)--++(-90:.4) node[pos=.5,right,black]{\val}--++(180:.5)--cycle;
							\path ($(0,0)+(\gocnode/2:\r/1.5)$) 
							node[fill=white,inner sep=0.5pt,rectangle]{\color{black} \rate};
							\global\let\gocxp=\gockt
						}
					\end{scope}
				\end{tikzpicture} 
			\end{center}
		\end{enumerate}
	}
\end{bt}

\begin{bt}%[SBT CD Toán 9]%[Dự án EX-9-Đề Cương Toán 9]%[9D5H1-2]
	Trong bài thơ \lq\lq \textit{Lượm}\rq\rq \, nổi tiếng của nhà thơ Tố Hữu có những câu thơ:
	
	\begin{quote}
		\lq\lq \textit{Chú bé loắt choắt} \\
		\textit{Cái xắc xinh xinh} \\
		\textit{Cái chân thoăn thoắt} \\
		\textit{Cái đầu nghênh nghênh $\ldots$\rq\rq}
	\end{quote}
	Mẫu dữ liệu thống kê các chữ cái C; N; H; T; L lần lượt xuất hiện trong những câu thơ trên là: 
	\begin{center}
		\begin{tabular}{c c c c c c c c c c c}
			C; & H; & L; & T; & C; & H; & T; & C; & C; & N; & H; \\
			N; & H; & C; & C; & H; & N; & T; & H; & N; & T; & H; \\
			T; & C; & N; & H; & N; & H; & N; & H; & N; & H.
		\end{tabular}
	\end{center}
	\begin{enumerate}
		\item Lập bảng tần số tương đối của mẫu dữ liệu thống kê đó.
		\item Vẽ biểu đồ tần số tương đối ở dạng biểu đồ hình quạt tròn của mẫu dữ liệu thống kê đó.
	\end{enumerate}
	\loigiai{
		\begin{enumerate}
			\item Các chữ cái C; N; H; T; L lần lượt có tần số là: $n_1 = 7$; $n_2 = 8$; $n_3 = 11$; $n_4 = 5$; $n_5 = 1$.
			\begin{center}
				\begin{tabular}{|>{\centering\arraybackslash}m{5cm}|>{\centering\arraybackslash}m{1cm}|>{\centering\arraybackslash}m{1cm}|>{\centering\arraybackslash}m{1cm}|>{\centering\arraybackslash}m{1cm}|>{\centering\arraybackslash}m{1cm}|>{\centering\arraybackslash}m{1.5cm}|}
					\hline
					\textbf{Chữ cái $(x)$}  & C & N & H & T & L & Cộng \\ \hline
					\textbf{Tần số tương đối $(\%)$}  & $21{,}875$ & $25$ & $34{,}375$ & $15{,}625$  & $3{,}125$ &  $100$ \\ \hline
				\end{tabular} 
			\end{center}
			\item Biểu đồ tần số tương đối ở dạng hình quạt tròn của mẫu số liệu thống kê đó là
			\begin{center}
				
				\begin{tikzpicture}
					\def\r{2}
					\def\gocxp{90}
					\coordinate (A) at (90:\r);
					\foreach \ts/\col/\i/\j in {21.875/cyan!60/C/21{,}875,25/yellow!60/N/25,34.375/green!60/H/34{,}375,15.625/red!60/T/15{,}625,3.125/orange!60/L/3{,}125}{
						\pgfmathsetmacro\gockt{\gocxp - \ts *360/100}
						\pgfmathsetmacro\gocn{\gocxp - \ts *360/200}
						\draw[gray!50,fill=\col] 
						(0:0)--(A) arc (\gocxp:\gockt:\r) coordinate (A)--cycle 		
						;
						\draw[shorten <=2.5mm, thin, black] (60:\r-0.5)--+(30:8mm)
						(-40:\r-0.5)--+(-30:10mm) (-140:\r-0.5)--+(-125:10mm) (130:\r-0.5)--+(130:8mm) (95:\r-0.5)--+(95:10mm)
						($(\gocn:3)$) node[inner sep=0pt]{\normalsize\color{black} \i} 
						($(\gocn:3)$) node[inner sep=0pt,below=6pt]{\normalsize\color{black} $\j$\%};
						\global\let\gocxp=\gockt
					}
				\end{tikzpicture}
			\end{center}
		\end{enumerate}
	}
\end{bt}

\begin{bt}%[SBT KNTT Toán 9]%[Dự án EX-9-Đề Cương Toán 9]%[9D5H1-2]
	Cho biểu đồ hình quạt tròn sau
	\begin{center}
		\begin{tikzpicture}[scale=1, font=\footnotesize, line join=round, line 
			cap=round, >=stealth]
			\def\r{3}
			\def\gocxp{90}
			\coordinate (A) at (90:\r);
			\foreach \val/\freq/\col/\pattern[count=\i from 0] in{
				Vật lý/23/red/vertical lines,
				Hóa học/20/blue/north east lines,
				Kinh tế/9/magenta/checkerboard,
				Hòa bình/11/green/bricks,
				Y học/24/teal/dots,
				Văn học/13/violet/north west lines}{
				\pgfmathsetmacro\gockt{-(\freq*3.6-\gocxp)}
				\pgfmathsetmacro\gocnode{\gocxp+\gockt}
				\draw[gray!50,pattern = \pattern,pattern color=\col] (0,0)--(A) 
				arc(\gocxp:\gockt:\r) coordinate(A)--cycle;
				\fill[pattern = \pattern,pattern color=\col] (\r+1,\r-.75*\i) 
				--++(0:.5)--++(-90:.5) 
				node[pos=.5,right,black]{\val}--++(180:.5)--cycle;
				\path ($(0,0)+(\gocnode/2:{0.75*\r})$) node[fill=white,inner 
				sep=0pt,circle]{\color{black} $\freq\%$};
				\global\let\gocxp=\gockt
			}\path (current bounding box.north) node[above=2mm]{\bfseries Tỉ lệ người đạt giải Nobel theo các lĩnh vực tính đến năm $2\,020$};
		\end{tikzpicture}
	\end{center}
	\begin{enumerate}
		\item Giải thích các số liệu được biểu diễn trên biểu đồ.
		\item Lập bảng tần số tương đối cho dữ liệu được biểu diễn trên biểu đồ.
	\end{enumerate}
	\loigiai{
		\begin{enumerate}
			\item Trong số những người đạt giải Nobel tính đến năm $2020$, số người đạt giải Nobel về Vật lí, Hoá học, Kinh tế, Hoà bình, Y học, Văn học chiếm các 
			tỉ lệ tương ứng là $23\%$, $20\%$, $9\%$, $11\%$, $24\%$, $13\%$.
			\item Bảng tần số tương đối
			\begin{center}
				\begin{tabular}{|c|c|c|c|c|c|c|}
					\hline
					Lĩnh vực & Vật lí & Hoá học & Kinh tế & Hoà bình & Y học & Văn học \\
					\hline
					Tần số tương đối & $23\%$ & $20\%$ & $9\%$ & $11\%$ & $24\%$ & $13\%$ \\
					\hline
				\end{tabular}
			\end{center}
		\end{enumerate}
	}
\end{bt}

\begin{bt}%[SBT KNTT Toán 9]%[Dự án EX-9-Đề Cương Toán 9]%[9D5H1-2]
	Lập bảng thống kê sau về số lượng học sinh tại một trường tham gia các câu lạc bộ (CLB)
	\begin{center}
		\begin{tabular}{|c|c|c|c|c|}
			\hline
			& CLB tiếng Anh & CLB Toán & CLB Khoa học & Tổng \\
			\hline
			Nam & $40$ & $60$ & $50$ & $150$ \\
			\hline
			Nữ & $70$ & $30$ & $65$ & $165$ \\
			\hline
			Tổng & $110$ & $90$ & $115$ & $315$ \\
			\hline
		\end{tabular}
	\end{center}
	\begin{enumerate}
		\item Lập các bảng tần số tương đối biểu diễn tỉ lệ học sinh nam, nữ tham gia các câu lạc bộ.
		\item Vẽ các biểu đồ hình quạt tròn biểu diễn các bảng tần số tương đối thu được ở câu trên.
	\end{enumerate}
	\loigiai{
		\begin{enumerate}
			\item Bảng tần số
			\begin{center}
				\begin{tabular}{|c|c|c|c|}
					\hline
					& CLB tiếng Anh & CLB Toán & CLB Khoa học  \\
					\hline
					Nam & $26{,}7\%$ & $40\%$ & $33{,}3\%$  \\
					\hline
					Nữ & $42{,}4\%$ & $18{,}2\%$ & $39{,}4\%$ \\
					\hline
				\end{tabular}
			\end{center}
			\item Các biểu đồ tròn
			\begin{center}
				\begin{tikzpicture}[scale=1, font=\footnotesize, line join=round, line 
					cap=round, >=stealth]
					\def\r{3}
					\def\gocxp{90}
					\coordinate (A) at (90:\r);
					\foreach \val/\freq/\col/\pattern[count=\i from 0] in{
						Tiếng Anh/26.7/magenta/checkerboard,
						Toán/40/red/vertical lines,
						Văn/33.3/blue/north east lines}{
						\pgfmathsetmacro\gockt{-(\freq*3.6-\gocxp)}
						\pgfmathsetmacro\gocnode{\gocxp+\gockt}
						\draw[gray!50,pattern = \pattern,pattern color=\col] (0,0)--(A) 
						arc(\gocxp:\gockt:\r) coordinate(A)--cycle;
						\fill[pattern = \pattern,pattern color=\col] (\r+1,\r-.75*\i) 
						--++(0:.5)--++(-90:.5) 
						node[pos=.5,right,black]{\val}--++(180:.5)--cycle;
						\path ($(0,0)+(\gocnode/2:{0.75*\r})$) node[fill=white,inner 
						sep=0pt,circle]{\color{black} $\freq\%$};
						\global\let\gocxp=\gockt
					}
					\path(0,3.75) node[scale=1.2]{
						Tỉ lệ học sinh Nam tham gia các câu lạc bộ};
				\end{tikzpicture}\hspace{1cm}
				\begin{tikzpicture}[scale=1, font=\footnotesize, line join=round, line 
					cap=round, >=stealth]
					\def\r{3}
					\def\gocxp{90}
					\coordinate (A) at (90:\r);
					\foreach \val/\freq/\col/\pattern[count=\i from 0] in{
						Tiếng Anh/42.4/magenta/checkerboard,
						Toán/18.2/red/vertical lines,
						Văn/39.4/blue/north east lines}{
						\pgfmathsetmacro\gockt{-(\freq*3.6-\gocxp)}
						\pgfmathsetmacro\gocnode{\gocxp+\gockt}
						\draw[gray!50,pattern = \pattern,pattern color=\col] (0,0)--(A) 
						arc(\gocxp:\gockt:\r) coordinate(A)--cycle;
						\fill[pattern = \pattern,pattern color=\col] (\r+1,\r-.75*\i) 
						--++(0:.5)--++(-90:.5) 
						node[pos=.5,right,black]{\val}--++(180:.5)--cycle;
						\path ($(0,0)+(\gocnode/2:{0.75*\r})$) node[fill=white,inner 
						sep=0pt,circle]{\color{black} $\freq\%$};
						\global\let\gocxp=\gockt
					}
					\path(0,3.75) node[scale=1.2]{
						Tỉ lệ học sinh Nữ tham gia các câu lạc bộ};
				\end{tikzpicture}
			\end{center}
		\end{enumerate}	
	}
\end{bt}

\begin{bt}%[SBT KNTT Toán 9]%[Dự án EX-9-Đề Cương Toán 9]%[9D5H1-2]
	Người ta nghiên cứu về độ bền của hai loại ti vi màn hình phẳng $43$ inch của hai hãng sản xuất A và B. Thời gian sử dụng của một số chiếc ti vi từ khi mua về đến khi gặp sự cố hỏng hóc đầu tiên được ghi lại ở bảng tần số sau:
	\begin{center}
		\begin{tabular}{|c|c|c|c|c|c|}
			\hline Thời gian sử dụng (năm)&$\left[3;4\right)$&$\left[4;5\right)$&$\left[5;6\right)$&$\left[6;7\right)$&$\left[7;8\right)$\\
			\hline Số ti vi của hãng A&$6$&$39$&$54$&$30$&$21$\\
			\hline Số ti vi của hãng B&$15$&$75$&$90$&$40$&$30$\\
			\hline 
		\end{tabular}
	\end{center}
	\begin{enumerate}
		\item Hãy tính tần số tương đối của ti vi mỗi hãng theo thời gian sử dụng.
		\item Một chiếc ti vi được gọi là bền nếu nó có thời gian sử dụng từ $6$ năm trở lên. Hãy so sánh tần số tương đối của ti vi bền của hai hãng A và B.
	\end{enumerate}
	\loigiai{
		\begin{enumerate}
			\item Bảng tần số tương đối:
			\begin{center}
				\begin{tabular}{|c|c|c|c|c|c|}
					\hline Thời gian sử dụng (năm)&$\left[3;4\right)$&$\left[4;5\right)$&$\left[5;6\right)$&$\left[6;7\right)$&$\left[7;8\right)$\\
					\hline Số ti vi của hãng A&$4\%$&$26\%$&$36\%$&$20\%$&$14\%$\\
					\hline Số ti vi của hãng B&$6\%$&$30\%$&$36\%$&$16\%$&$12\%$\\
					\hline 
				\end{tabular}
			\end{center}
			\item Tần số tương đối của ti vi bền của hãng A là $20\% + 14\% = 34\%$.\\
			Tần số tương đối của ti vi bền của hãng B là $16\% + 12\% = 28\%$.\\
			Vậy tần số tương đối của ti vi bền do hãng A sản xuất cao hơn hãng B.
		\end{enumerate}
	}
\end{bt}

\begin{bt}%[SBT CTST Toán 9]%[Dự án EX-9-Đề Cương Toán 9]%[9D5H1-2]
	Vào đầu năm học, người ta lựa chọn ngẫu nhiên một số học sinh lớp $9$ ở khu vực A và khu vực B để kiểm tra tình trạng cân nặng. Kết quả khảo sát được ghi lại ở bảng sau
	\begin{center}
		\begin{tabular}{|c|c|c|c|c|}
			\hline Tình trạng cân nặng & Thiếu cân & Bình thường & Thừa cân & Béo phì \\
			\hline Số học sinh khu vực A & $16$ & $40$ & $16$ & $8$ \\
			\hline Số học sinh khu vực B & $6$ & $34$ & $5$ & $5$ \\
			\hline 
		\end{tabular}
	\end{center}
	\begin{enumerate}
		\item Hãy tính tần số tương đối của học sinh ở mỗi khu vực theo tình trạng cân nặng.
		\item Hãy lựa chọn, vẽ biểu đồ phù hợp và so sánh tình trạng cân nặng của học sinh ở hai khu vực.
	\end{enumerate}
	\loigiai{
		\begin{enumerate}
			\item Bảng tần số tương đối của học sinh ở mỗi khu vực theo tình trạng cân nặng
			\begin{center}
				\begin{tabular}{|c|c|c|c|c|}
					\hline Tình trạng cân nặng & Thiếu cân & Bình thường & Thừa cân & Béo phì \\
					\hline Số học sinh khu vực A & $20\%$ & $50\%$ & $20\%$ & $10\%$ \\
					\hline Số học sinh khu vực B & $12\%$ & $ 68\%$ & $10\%$ & $10\%$ \\
					\hline 
				\end{tabular}
			\end{center}
			\item Để so sánh tình trạng cân nặng của học sinh ở hai khu vực, ta sử dụng biểu đồ cột kép.
			\begin{center}
				\textbf{Tần số tương đối của số học sinh theo tình trạng cân nặng}
				\begin{tikzpicture}[>=stealth, scale=0.9,
					join = round, cap = round,
					declare function={
						x=1.5; %Co trục ngang
						y=1.5; %Co trục đứng
						kcy=1; %Khoảng cách từ trục đứng đến cột 1
						kc=1.5; %Khoảng cách giữa 2 cột
						dochia=10; %Độ chia nhỏ nhất trên trục đứng
					},
					xscale = 1/x, yscale = 1/y, 
					font = \scriptsize
					]
					\foreach \i in {1,2,...,8}{
						\draw[gray, thin, opacity=0.7] (0,\i) -- ({13+x},\i);
					}
					\foreach \x/\y/\z[count = \i from 0] in {
						Thiếu cân/20/12,
						Bình thường/50/68,
						Thừa cân/20/10,
						Béo phì/10/10
					}{
						\pgfmathsetmacro{\j}{(kc+2)*\i+kcy+1}
						\draw[pattern = north east lines, pattern color=red] (\j-1,0) rectangle (\j,\y/dochia);
						\path (\j-.5,\y/dochia) node[above]{$\y \%$};
						\draw[pattern = dots, pattern color=teal] (\j,0) rectangle (\j+1,\z/dochia);
						\path (\j+.5,\z/dochia) node[above]{$\z\%$};
						\path (\j,-.4*y) node[rotate = 0,align=center, text width=1.2cm]{\x};
						\global\let\n=\j
					}
					%Vẽ hệ trục
					\draw[<->] (\n+x+1,0)node[below ,align=center, text width=1.5cm]{Tình trạng cân nặng} -| (0,90/dochia+y/dochia)node[above,align=center, text width=2cm]{Tần số tương đối (\%)};
					\foreach \y in {0,10,...,80}{
						\draw (.05*x,\y/dochia)--(-.05*x,\y/dochia) node[left]{$\y$};
					}
					%Chú thích
					\draw[pattern = north east lines, pattern color=red] (\n+x+1,76/dochia) rectangle ++(.5*x,.5*y)++(0,-.25*y) node[right]{Khu vực A};
					\draw[pattern = dots, pattern color=teal] (\n+x+1,76/dochia-.7*y) rectangle ++(.5*x,.5*y)++(0,-.25*y) node[right]{Khu vực B};	
				\end{tikzpicture}
			\end{center}
			Tần số tương đối của học sinh thiếu cân và thừa cân ở khu vực A cao hơn khu vực B. \\
			Tần số tương đối của số học sinh có cân nặng bình thường ở khu vực A thấp hơn khu vực B. \\
			Tần số tương đối của số học sinh béo phì ở hai khu vực là như nhau.
	\end{enumerate} }
\end{bt}

\begin{bt}%[SBT CTST Toán 9]%[Dự án EX-9-Đề Cương Toán 9]%[9D5H1-2]
	Trong bảng số liệu sau có một số liệu không chính xác. Hãy tìm số liệu đó và sửa lại cho đúng.
	\begin{center}
		\begin{tabular}{|c|c|c|c|c|}
			\hline Giá trị & $2$ & $3$ & $4$ & $7$ \\
			\hline Tần số & $10$ & $20$ & $28$ & $20$ \\
			\hline Tần số tương đối & $15\%$ & $25\%$ & $35\%$ & $25\%$ \\
			\hline 
		\end{tabular}
	\end{center}
	\loigiai{
		Tổng các tần số tương đối là $15\% + 25\% + 35\% + 25\% = 100\%$ nên nếu có số liệu về tần số tương đối sai thì phải có ít nhất hai số liệu sai. \\
		Do chỉ có một số liệu sai trong bảng nên các giá trị tần số tương đối đều chính xác.\\
		Ta có $\dfrac{10}{15} \ne \dfrac{20}{25} = \dfrac{28}{35} = \dfrac{20}{25}$ do đó số liệu tần số $10$ là sai.\\
		Giá trị đúng là $\dfrac{15 \cdot 20}{25} = 12$.
	}
\end{bt}