\section*{BÀI TẬP CUỐI CHƯƠNG 1}

\subsection{Câu hỏi trắc nghiệm}
\Opensolutionfile{ans}[ans/ans-9C1-OTC-1]
	%Câu 1
	\begin{ex}%[Dự án EX-9-Đề Cương Toán 9]%[Phạm Minh Khánh]%[9D1N1-1]
		Tất cả các nghiệm của phương trình $\left(x+3\right)\left(2x-6\right)=0$ là
		\choice
		{$x=-3$}
		{$x=3$}
		{\True $x=3$ hoặc $x=-3$}
		{$x=2$}
		\loigiai{Ta có:
			$\left(x+3\right)\left(2x-6\right)=0$ nên $x+3=0$ hoặc $2x-6=0$
			\begin{itemize}
				\item $x+3=0$ suy ra $x=-3$.
				\item $2x-6=0$ hoặc $2x=6$, suy ra $x=3$.
			\end{itemize}
			Vậy phương trình có hai nghiệm $x=3$, $x=-3$.
		}
	\end{ex}
	%Câu 2
	\begin{ex}%[Dự án EX-9-Đề Cương Toán 9]%[Phạm Minh Khánh]%[9D1N1-2]
		Điều kiện xác định của phương trình $\dfrac{2x+3}{x-4}+2=\dfrac{1}{x-3}$
		\choice
		{ $x\ne 4$}
		{$x \ne 3$}
		{\True $ x \ne 4$ và $ x \ne 3$}
		{$x=4$ và $x=3$}
		\loigiai{
			Để phương trình $\dfrac{2x+3}{x-4}+2=\dfrac{1}{x-3}$ xác định thì $x-4\neq0$ và $x-3\neq0$.\\
			Suy ra $x\neq 4$ và $x\neq 3$.
		}
	\end{ex}
	%Câu 3
	\begin{ex}%[Dự án EX-9-Đề Cương Toán 9]%[Phạm Minh Khánh]%[9D1N1-2]
		Nghiệm của phương trình $\dfrac{1}{x}-\dfrac{3}{2 x}=\dfrac{1}{6}$ là
		\choice{$x=3$}
		{\True $x=-3$}
		{$x=6$}
		{$x=-6$} 
		\loigiai{
			Với điều kiện xác định $x \ne 0$, ta có:
			\begin{eqnarray*} 
				&&\dfrac{1}{x}-\dfrac{3}{2 x} =\dfrac{1}{6} \\ 
				&&\dfrac{6}{6 x}-\dfrac{9}{6 x}=\dfrac{x}{6 x} \\ 
				&&6-9 =x \\
				&&x =-3.
			\end{eqnarray*}
			Ta thấy $x=-3$ thỏa mãn điều kiện xác định.\\
			Vậy phương trình đã cho có nghiệm $x=-3$.
		}
	\end{ex} 
	%Câu 4
	\begin{ex}%[Dự án EX-9-Đề Cương Toán 9]%[Phạm Minh Khánh]%[9D1H1-2]
		Nghiệm của phương trình $\dfrac{x+2}{x-4}-1=\dfrac{30}{\left(x+3\right)\left(x-4\right)}$ là 
		\choice
		{\True $x=2$}
		{$x=-3$}
		{$x=4$}
		{$x=-2$}
		\loigiai{
			Với điều kiện $x\neq4$ và $x\neq -3$, ta có:
			\begin{eqnarray*}
				&&\dfrac{x+2}{x-4}-1=\dfrac{30}{\left(x+3\right)\left(x-4\right)}\\
				&&\dfrac{\left(x+2\right)\left(x+3\right)}{\left(x+3\right)\left(x-4\right)}-\dfrac{(x-4)(x+3)}{\left(x+3\right)\left(x-4\right)}=\dfrac{30}{\left(x+3\right)\left(x-4\right)}\\
				&&\dfrac{x^2+5x+6}{\left(x+3\right)\left(x-4\right)}-\dfrac{x^2-x-12}{\left(x+3\right)\left(x-4\right)}=\dfrac{30}{\left(x+3\right)\left(x-4\right)}\\
				&&x^2+5x+6-x^2+x+12-30=0\\
				&&6x-12=0\\
				&&x=2 \quad \left( \text{thỏa mãn}\right).
			\end{eqnarray*}
			Vậy phương trình có nghiệm là $x=2$.
		}
	\end{ex}
	%Câu 5
	\begin{ex}%[Dự án EX-9-Đề Cương Toán 9]%[Phạm Minh Khánh]%[9D1H1-2]
		Điều kiện xác định của phương trình $\dfrac{x}{2x + 1} + \dfrac{3}{x - 5} = \dfrac{x}{(2x+1)(x-5)}$ là
		\choice
		{$x \neq -\dfrac{1}{2}$}
		{$x \neq -\dfrac{1}{2}$ và $x \neq -5$}
		{$x \neq 5$}
		{\True $x \neq -\dfrac{1}{2}$ và $x \neq 5$}
		\loigiai{
			Điều kiện xác định của phương trình đã cho là $2x+1\neq0$ và $x-5\neq0$.\\
			Suy ra $x\neq\dfrac{-1}{2}$ và $x\neq5$.\\
			Vậy điều kiện xác định của phương trình là $x \neq -\dfrac{1}{2}$ và $x \neq 5$.
		}
	\end{ex}

	%Câu 6
	\begin{ex}%[Dự án EX-9-Đề Cương Toán 9]%[Phạm Minh Khánh]%[9D1N2-1]
		Phương trình nào sau đây \textbf{không} phải là phương trình bậc nhất hai ẩn?
		\choice
		{$5x-y=3$}
		{$\sqrt{5}x+0y=0$}
		{$0x-4y=\sqrt{6}$}
		{\True $0x+0y=12$}
		\loigiai{
			Phương trình \textbf{không} phải là phương trình bậc nhất hai ẩn là $0x+0y=12$.
		}
	\end{ex}
	
	%Câu 7
	\begin{ex}%[Dự án EX-9-Đề Cương Toán 9]%[Phạm Minh Khánh]%[9D1N2-1]
		Đường thẳng biểu diễn tất cả các nghiệm của phương trình $3x-y=2$
		\choice
		{vuông góc với trục tung}
		{vuông góc với trục hoành}
		{đi qua gốc tọa độ}
		{\True đi qua điểm $A(1;1)$}
		\loigiai{
			\begin{itemize}
				\item Đường thẳng vuông góc với trục tung sẽ song song với trục hoành.\\
				Đường thẳng song song với trục hoành có dạng $y=a$ ($a \in \mathbb{R}$).\\
				Mà phương trình $3x-y=2$ có dạng $y=3x-2$.\\
				Vậy đường thẳng biểu diễn tất cả các nghiệm của phương trình $3x-y=2$ không vuông góc với trục tung.
				\item Đường thẳng vuông góc với trục hoành sẽ song song với trục tung.\\
				Đường thẳng song song với trục tung có dạng $x=a$ ($a \in \mathbb{R}$).\\
				Mà phương trình $3x-y=2$ có dạng $x=\dfrac{y+2}{3}$.\\
				Vậy đường thẳng biểu diễn tất cả các nghiệm của phương trình $3x-y=2$ không vuông góc với trục hoành.
				\item Vì $3\cdot 0-1\neq 2$ nên đường thẳng biểu diễn tất cả các nghiệm của phương trình $3x-y=2$ không đi qua gốc tọa độ.
				\item Vì $3\cdot 1-1=2$ nên đường thẳng biểu diễn tất cả các nghiệm của phương trình $3x-y=2$ đi qua điểm $A(1;1)$.
			\end{itemize}
		}
	\end{ex}
	
	%Câu 8
	\begin{ex}%[Dự án EX-9-Đề Cương Toán 9]%[Phạm Minh Khánh]%[9D1N2-1]
		Trên mặt phẳng toạ độ $O x y$, cho các điểm $A(1; 2), B(5; 6), C(2; 3), D(-1;-1)$. Đường thẳng $4 x-3 y=-1$ đi qua hai điểm nào trong các điểm đã cho?
		\choice
		{$A$ và $B$}
		{$B$ và $C$}
		{\True $C$ và $D$}
		{$D$ và $A$}
		\loigiai{
			\begin{itemize}
				\item Thay tọa độ điểm $A(1;2)$ vào đường thẳng $4x-3y=-1$, ta được $4\cdot 1-3\cdot 2=-2\ne -1$.\\
				\item Thay tọa độ điểm $B(5;6)$ vào đường thẳng $4x-3y=-1$, ta được $4\cdot 5-3\cdot 6=2\neq -1$.\\
				\item Thay tọa độ điểm $C(2;3)$ vào đường thẳng $4x-3y=-1$, ta được $4\cdot 2-3\cdot 3=-1$.\\
				\item Thay tọa độ điểm $D(-1;-1)$ vào đường thẳng $4x-3y=-1$, ta được $4\cdot (-1)-3\cdot (-1)=-1$.
			\end{itemize}
			Vậy đường thẳng $4x-3y=-1$ đi qua hai điểm $C$ và $D$.
		}
	\end{ex}
	
	%Câu 9
	\begin{ex}%[Dự án EX-9-Đề Cương Toán 9]%[Phạm Minh Khánh]%[9D1N2-2]
		Cặp số $(-2;-3)$ là nghiệm của hệ phương trình nào sau đây?
		\choice
		{$\heva{&x-2y=3\\&2x+y=4}$}
		{$\heva{&2x-y=-1\\&x-3y=8}$}
		{\True $\heva{&2x-y=-1\\&x-3y=7}$}
		{$\heva{&4x-2y=0\\&x-3y=5}$}
		\loigiai{Ta có
		\begin{itemize}
			\item Vì $\heva{&(-2)-2\cdot(-3)\neq 3\\&2\cdot(-2)+(-3)\neq4}$ nên cặp số $(-2;-3)$ không là nghiệm của hệ phương trình.
			\item Vì $\heva{&2\cdot(-2)-(-3)=-1\\&-2-3\cdot(-3)\neq8}$ nên cặp số $(-2;-3)$ không là nghiệm của hệ phương trình.
			\item Vì $\heva{&2\cdot (-2)- (-3)=-1\\&(-2)-3\cdot(-3)=7}$ nên cặp số $(-2;-3)$ là nghiệm của hệ phương trình.
			\item Vì $\heva{&4\cdot(-2)-2\cdot(-3)\neq0\\&(-2)-3\cdot(-3)\neq5}$ nên cặp số $(-2;-3)$ khôgn là nghiệm của hệ phương trình.
		\end{itemize}
		Vậy cặp số $(-2;-3)$ là nghiệm của hệ phương trình $\heva{&2x-y=-1\\&x-3y=7}$.
		}
	\end{ex}

	%Câu 10
	\begin{ex}%[Dự án EX-9-Đề Cương Toán 9]%[Phạm Minh Khánh]%[9D1H2-2]
		Cặp số nào sau đây là nghiệm của hệ phương trình $\heva{& 5 x+7 y=-1 \\ & 3 x+2 y=-5}$?
		\choice
		{$(-1; 1)$}
		{\True $(-3; 2)$}
		{$(2;-3)$}
		{$(5; 5)$}
		\loigiai{
		Ta có
		\begin{itemize}
			\item Vì $\heva{&5\cdot(-1)+7\cdot1\neq-1\\&3\cdot(-1)+2\cdot1\neq-5}$ nên cặp số $(-1;1)$ không là nghiệm của hệ phương trình.
			\item Vì $\heva{&5\cdot(-3)+7\cdot2=-1\\&3\cdot(-3)+2\cdot2=-5}$ nên cặp số $(-3;2)$ là nghiệm của hệ phương trình.
			\item Vì $\heva{&5\cdot2+7\cdot(-3)\neq-1\\&3\cdot2+2\cdot(-3)\neq -5}$ nên cặp số $(2;-3)$ không là nghiệm của hệ phương trình.
			\item Vì $\heva{&5\cdot5+7\cdot5\neq-1\\&3\cdot5+2\cdot5\neq-5}$ nên cặp số $(5;5)$ không là nghiệm của hệ phương trình.
		\end{itemize}
		Vậy cặp số $(-3;2)$ là nghiệm của hệ phương trình $\heva{& 5 x+7 y=-1 \\ & 3 x+2 y=-5}$.
		}
	\end{ex}
	
	%Câu 11
	\begin{ex}%[Dự án EX-9-Đề Cương Toán 9]%[Phạm Minh Khánh]%[9D1H2-2]
		Hệ phương trình $\heva{&1{,}5 x-0{,}6 y=0{,}3 \\ &-2 x+y=-2.}$
		\choice
		{có nghiệm là $(0;-0{,}5)$}
		{có nghiệm là $(1; 0)$}
		{\True có nghiệm là $(-3;-8)$}
		{vô nghiệm}
		\loigiai{
			\[\heva{&1{,}5 x-0{,}6 y=0{,}3 \\ &-2 x+y=-2} \text{ suy ra }\heva{&5 x-2 y=1 \\ &-4x+2y=-4.}\]
			Cộng từng vế hai phương trình của hệ mới, ta được $x=-3$.\\
			Thế giá trị $x=-3$ vào phương trình thứ hai của hệ phương trình, ta có
			\begin{eqnarray*}
				&&-2.(-3)+y=-2\\
				&&y=-8.
			\end{eqnarray*}
			Vậy hệ phương trình có nghiệm là $(-3;-8)$.
		}
	\end{ex}
	
	%Câu 12
	\begin{ex}%[Dự án EX-9-Đề Cương Toán 9]%[Phạm Minh Khánh]%[9D1H2-2]
		Hệ phương trình $\heva{& 0{,}6 x+0{,}3 y=1{,}8 \\ & 2 x+y=-6.}$
		\choice
		{có một nghiệm}
		{\True vô nghiệm}
		{có vô số nghiệm}
		{có hai nghiệm}
		\loigiai{
			\begin{eqnarray*}
				&&\heva{& 0{,}6 x+0{,}3 y=1{,}8 \\ & 2 x+y=-6}\\
				&&\heva{& 2 x+ y=6 \\ & 2 x+y=-6.}
			\end{eqnarray*}
			Từ phương trình thứ nhất của hệ mới, ta có $y=6-2x$. Thế vào phương trình thứ hai của hệ ta được 
			\begin{eqnarray*}
				&&2x+(6-2x)=-6\\
				&&0x=-12.
			\end{eqnarray*}
			Do không có giá trị nào của $x$ thỏa mãn hệ thức trên nên hệ phương trình đã cho vô nghiệm.
		}
	\end{ex}
	%Câu 13
	\begin{ex}%[Dự án EX-9-Đề Cương Toán 9]%[Phạm Minh Khánh]%[9D1H2-2]
		Nghiệm của hệ phương trình $\heva{&x+y=9 \\& x-y=-1}$ là
		\choice
		{\True $(x; y)=(4; 5)$}
		{$(x; y)=(5; 4)$}
		{$(x; y)=(-5;-4)$}
		{$(x; y)=(-4;-5)$}
		\loigiai{
			Ta có
			\begin{eqnarray*}
				&&\heva{&x+y=9 \\ &x-y=-1}\\
				&&\heva{&2 x=8 \\ &x-y=-1}\\
				&&\heva{&x=4 \\ &y=4+1}\\
				&&\heva{&x=4 \\ &y=5.}
			\end{eqnarray*}
			Vậy hệ phương trình đã cho có nghiệm $(4;5)$.
		}
	\end{ex} 
	
	
\Closesolutionfile{ans}
%\indapan{6}{ans/ans-9C1-OTC-1}
\subsection{Bài tập tự luận}

\begin{bt}%[Dự án EX-9-Đề Cương Toán 9]%[Phạm Minh Khánh]%[9D1H1-1]
	Giải các phương trình:
	\begin{multicols}{3}
	\begin{enumerate}
		\item $(3 x+7)(4 x-9)=0$;
		\item $x(2 x-1)+5(2 x-1)=0$;
		\item $x^2-9-(x+3)(3 x+1)=0$;
		\item $(3x - 1)^2 - (x + 2)^2 = 0$;
		\item $x(x + 1) = 2(x^2 - 1)$;
		\item $x^2-10 x+25=5(5-x)$.
	\end{enumerate}
	\end{multicols}
	\loigiai{
		\begin{enumerate}
			\item 
			Ta có $(3 x+7)(4 x-9)=0$ nên $3 x+7=0$ hoặc $4 x-9=0$.
			\begin{itemize}
				\item $3 x+7=0$ hoặc $3 x=-7$, suy ra $x=-\dfrac{7}{3}$.
				\item $4 x-9=0$ hoặc	$4 x=9$, suy ra $x=\dfrac{9}{4}$.
			\end{itemize}
			Vậy phương trình đã cho có hai nghiệm $x=-\dfrac{7}{3}$ và $x=\dfrac{9}{4}$.
			\item 
			\begin{eqnarray*}
				&&x(2 x-1)+5(2 x-1)=0\\
				&&(2x-1)(x+5)=0
			\end{eqnarray*}
			Ta có $(2x-1)(x+5)=0$ nên $2x-1=0$ hoặc $x+5=0$.
			\begin{itemize}
				\item $2x-1=0$ hoặc $2x=1$, suy ra $x=\dfrac{1}{2}$.
				\item $x+5=0$ suy ra $x=-5$.
			\end{itemize}
			Vậy phương trình đã cho có hai nghiệm $x=\dfrac{1}{2}$ và $x=-5$.
			\item 
			\begin{eqnarray*}
				&&x^2-9-(x+3)(3 x+1)=0\\
				&&(x-3)(x+3)-(x+3)(3 x+1)=0 \\ 
				&&(x+3)(x-3-3 x-1)=0 \\ 
				&&(x+3)(-2 x-4)=0
			\end{eqnarray*}	
			Ta có $(x+3)(-2 x-4)=0$ nên $x+3=0$ hoặc $-2x-4=0$.
			\begin{itemize}
				\item $x+3=0$ suy ra $x=-3$.
				\item $-2x-4=0$ hoặc $-2x=4$, suy ra $x=-2$.
			\end{itemize}	
			Vậy phương trình đã cho có hai nghiệm $x=-3$ và $x=-2$.
			\item
			\begin{eqnarray*}
				&&(3x - 1)^2 - (x + 2)^2 = 0\\
				&&\left[(3x-1) - (x+2)\right]\left[(3x-1) + (x+2)\right] =0\\
				&&(2x-3)(4x+1)= 0
			\end{eqnarray*}
			Ta có $(2x-3)(4x+1)=0$ nên $2x-3=0$ hoặc $4x+1=0$.
			\begin{itemize}
				\item $2x-3 = 0$ hoặc $2x=3$, suy ra $x = \dfrac{3}{2}$.
				\item $4x + 1 = 0$ hoặc $4x=-1$, suy ra $x = -\dfrac{1}{4}$.
			\end{itemize}
			Vậy phương trình đã cho có hai nghiệm là $x = \dfrac{3}{2}$, $x = -\dfrac{1}{4}$.
			\item 
			\begin{eqnarray*}
				&&x(x + 1)= 2(x^2 - 1)\\
				&&x(x+1)= 2(x-1)(x+1)\\
				&&(x+1)\left[x - 2(x-1)\right]= 0\\
				&&(x+1)\left(-x + 2\right)= 0
			\end{eqnarray*}
			Ta có $(x+1)\left(-x + 2\right)= 0$ nên $x+1=0$ hoặc $-x+2=0$.
			\begin{itemize}
				\item $x + 1 = 0$ hoặc $x = -1$.
				\item $-x + 2 = 0$ hoặc $x = 2$.
			\end{itemize}
			Vậy phương trình đã cho có hai nghiệm là $x = -1$, $x = 2$.
			\item
			\begin{eqnarray*} 
				&&x^2-10 x+25=5(5-x)\\
				&&(x-5)^2+5(x-5)=0 \\ 
				&&(x-5)(x-5+5)=0 \\
				&&x(x-5)=0
			\end{eqnarray*}
			Ta có $x(x-5)=0$ nên $x=0$ hoặc $x-5=0$.\\
			Suy ra $x=0$ hoặc $x=5$.\\
			Vậy phương trình đã cho có hai nghiệm là $x=0$, $x=5$.	
	\end{enumerate}
}
\end{bt}

\begin{bt}%[Dự án EX-9-Đề Cương Toán 9]%[Phạm Minh Khánh]%[9D1H1-2]
	Giải các phương trình:
	\begin{multicols}{3}
		\begin{enumerate}
			\item $\dfrac{-6}{x+3}=\dfrac{2}{3}$;
			\item $\dfrac{x-2}{2}+\dfrac{1}{2 x}=0$;
			\item $\dfrac{8}{3 x-4}=\dfrac{1}{x+2}$;
			\item $\dfrac{x}{x-2}+\dfrac{2}{(x-2)^2}=1$;
			\item $\dfrac{3 x-2}{x+1}=4-\dfrac{x+2}{x-1}$;
			\item $\dfrac{x^2}{(x-1)(x-2)}=1-\dfrac{1}{x-1}$. 
		\end{enumerate}
	\end{multicols}
	\loigiai{
		\begin{enumerate}
			\item $\dfrac{-6}{x+3}=\dfrac{2}{3}$.\\
			Điều kiện xác định $x \ne -3$.
			\begin{eqnarray*} 
				&&\dfrac{-6}{x+3}  =\dfrac{2}{3} \\ 
				&&\dfrac{-18}{3(x+3)} =\dfrac{2(x+3)}{3(x+3)} \\ 
				&&-18  =2 x+6 \\ 
				&&2 x  =-24 \\ 
				&&x =-12.
			\end{eqnarray*}
			\\ Ta thấy $x=-12$ thỏa mãn điều kiện xác định.\\
			Vậy phương trình có nghiệm $x=-12$.
			\item $\dfrac{x-2}{2}+\dfrac{1}{2 x}=0$.\\
			Điều kiện xác định $x \ne 0$.
			\begin{eqnarray*} 
				&&\dfrac{x(x-2)}{2 x}+\dfrac{1}{2 x}=0 \\ 
				&&x^2-2 x+1=0 \\ 
				&&(x-1)^2=0 \\ 
				&&x-1=0 \\ 
				&&x=1.
			\end{eqnarray*}
			Ta thấy $x=1$ thỏa mãn điều kiện xác định.\\
			Vậy phương trình có nghiệm $x=1$.
			\item $\dfrac{8}{3 x-4}=\dfrac{1}{x+2}$.\\
			Điều kiện xác định $x \neq-2 $ và $x \neq \dfrac{4}{3}$.
			\begin{eqnarray*} 
				&&\dfrac{8}{3 x-4}=\dfrac{1}{x+2}\\
				&&\dfrac{8(x+2)}{(3 x-4)(x+2)}=\dfrac{3x-4}{(x+2)(3x-4)}\\
				&&8(x+2)  =3 x-4 \\ 
				&&8 x+16  =3 x-4 \\ 
				&&5 x  =-20 \\ 
				&&x =-4.
			\end{eqnarray*}
			Ta thấy $x=-4$ thỏa mãn điều kiện xác định.\\
			Vậy phương trình có nghiệm $x=-4$.
			\item $\dfrac{x}{x-2}+\dfrac{2}{(x-2)^2}=1$.\\
			Điều kiện xác định $x \neq 2 $.
			\begin{eqnarray*} 
				&&\dfrac{x}{x-2}+\dfrac{2}{(x-2)^2}=1\\
				&&\dfrac{x(x-2)}{(x-2)^2}+\dfrac{2}{(x-2)^2}=\dfrac{(x-2)^2}{(x-2)^2}\\
				&&x(x-2)+2  =(x-2)^2 \\ 
				&&x^2-2 x+2  =x^2-4 x+4 \\ 
				&&2 x  =2 \\ 
				&&x  =1.
			\end{eqnarray*}
			Ta thấy $x=1$ thỏa mãn điều kiện xác định.\\
			Vậy phương trình có nghiệm $x=1$.
			\item $\dfrac{3 x-2}{x+1}=4-\dfrac{x+2}{x-1}$.\\
			Điều kiện xác định $x \neq -1 $ và $x \neq 1$.
			\begin{eqnarray*}
				&&\dfrac{3 x-2}{x+1}=4-\dfrac{x+2}{x-1}\\
				&&\dfrac{(3x-2)(x-1)}{(x+1)(x-1)}=\dfrac{4(x-1)(x+1)-(x+2)(x+1)}{(x-1)(x+1)}\\
				&&(3 x-2)(x-1)  =4\left(x^2-1\right)-(x+2)(x+1) \\ 
				&&3 x^2-3 x-2 x+2  =4 x^2-4-x^2-3 x-2 \\ 
				&&-2 x  =-8 \\ 
				&&x  =4.
			\end{eqnarray*}
			Ta thấy $x=4$ thỏa mãn điều kiện xác định.\\
			Vậy phương trình có nghiệm $x=4$.	
			\item $\dfrac{x^2}{(x-1)(x-2)}=1-\dfrac{1}{x-1}$. \\
			Điều kiện xác định $x \neq 1 $ và $x \neq 2 $.
			\begin{eqnarray*} 
				&&\dfrac{x^2}{(x-1)(x-2)}=1-\dfrac{1}{x-1}\\
				&&\dfrac{x^2}{(x-1)(x-2)}=\dfrac{(x-1)(x-2)-(x-2)}{(x-1)(x-2)}\\
				&&x^2 =(x-1)(x-2)-(x-2) \\ 
				&&x^2 =x^2-3 x+2-x+2 \\ 
				&&4 x =4 \\ 
				&&x =1.
			\end{eqnarray*}
			Ta thấy $x=1$ không thỏa mãn điều kiện xác định.\\
			Vậy phương trình vô nghiệm.
		\end{enumerate}
	}
\end{bt}

\begin{bt}%[Dự án EX-9-Đề Cương Toán 9]%[Phạm Minh Khánh]%[9D1H1-2]
	Giải các phương trình:
	\begin{multicols}{2}
		\begin{enumerate}
			\item $\dfrac{5}{x+2}+\dfrac{3}{x-1}=\dfrac{3x+4}{\left(x+2\right)\left(x-1\right)}$;
			\item $\dfrac{4}{2x-3}+\dfrac{3}{x\left(2x-3\right)}=\dfrac{5}{x}$;
			\item $\dfrac{2}{x-3}+\dfrac{3}{x+3}=\dfrac{3x-5}{x^2-9}$;
			\item $\dfrac{x-1}{x+1}-\dfrac{x+1}{x-1}=\dfrac{8}{x^2-1}$;
			\item $\dfrac{1}{x+2}-\dfrac{2}{x^{2}-2 x+4}=\dfrac{x-4}{x^{3}+8}$;
			\item $\dfrac{2 x}{x-4}+\dfrac{3}{x+4}=\dfrac{x-12}{x^{2}-16}$.
		\end{enumerate}
	\end{multicols}
	\loigiai{
		\begin{enumerate}
			\item 
			Với điều kiện $x\neq -2$ và $x\neq1$, ta có
			\begin{eqnarray*}
				&&\dfrac{5}{x+2}+\dfrac{3}{x-1}=\dfrac{3x+4}{\left(x+2\right)\left(x-1\right)}\\
				&&\dfrac{5x-5}{\left(x+2\right)\left(x-1\right)}+\dfrac{3x+6}{\left(x+2\right)\left(x-1\right)}=\dfrac{3x+4}{\left(x+2\right)\left(x-1\right)}\\
				&&5x-5+3x+6-3x-4=0\\
				&&5x-3=0\\
				&&x=\dfrac{3}{5}.
			\end{eqnarray*}
			Ta thấy $x=\dfrac{3}{5}$ thỏa mãn điều kiện xác định.\\
			Vậy nghiệm của phương trình là $x=\dfrac{3}{5}$.
			\item 
			Với điều kiện $x\neq0$ và $x\neq \dfrac{3}{2}$, ta có
			\begin{eqnarray*}
				&&\dfrac{4}{2x-3}-\dfrac{3}{x\left(2x-3\right)}=\dfrac{5}{x}\\
				&&\dfrac{4x}{x\left(2x-3\right)}-\dfrac{3}{x\left(2x-3\right)}=\dfrac{10x-15}{x\left(2x-3\right)}\\
				&&4x-3-10x+15=0\\
				&&-6x+12=0\\
				&&x=2.
			\end{eqnarray*}
			Ta thấy $x=2$ thỏa mãn điều kiện xác định.\\
			Vậy nghiệm của phương trình là $x=2$.
			\item 
			Với điều kiện $x\neq 3$ và $x\neq -3$, ta có
			\begin{eqnarray*}
				&&\dfrac{2}{x-3}+\dfrac{3}{x+3}=\dfrac{3x-5}{x^2-9}\\
				&&2x+6+3x-9=3x-5\\
				&&2x+2=0\\
				&&x=-1.
			\end{eqnarray*}
			Ta thấy $x=-1$ thỏa mãn điều kiện xác định.\\
			Vậy nghiệm của phương trình là $x=-1$.
			\item
			Với điều kiện $x\neq1$ và $x\neq-1$, ta có
			\begin{eqnarray*}
				&&\dfrac{x-1}{x+1}-\dfrac{x+1}{x-1}=\dfrac{8}{x^2-1}\\
				&&\left(x-1\right)^2-\left(x+1\right)^2=8\\
				&&-4x=8\\
				&&x=-2.
			\end{eqnarray*}
			Ta thấy $x=-2$ thỏa mãn điều kiện xác định.\\
			Vậy nghiệm của phương trình là $x=-2$.
			\item 
			Với điều kiện $x\neq-2$, ta có
			Quy đồng mẫu hai vế của phương trình:
			\begin{eqnarray*}
				&&\dfrac{x^2-2x+4-2(x+2)}{(x+2)(x^2-2x+4)}=\dfrac{x-4}{x^{3}+8}\\
				&&\dfrac{x^2-2x+4-2x-4}{x^3+8}=\dfrac{x-4}{x^3+8}\\
				&&\dfrac{x^2-4x}{x^3+8}=\dfrac{x-4}{x^3+8}\\
				&&x^2-4x=x-4\\
				&&x(x-4)=x-4\\
				&&x(x-4)-(x-4)=0\\
				&&(x-4)(x-1)=0\\
				&&x-4=0~\text{hoặc}~x-1=0\\
				&&x=4~\text{hoặc}~x=1
			\end{eqnarray*}
			Ta thấy $x=4$ và $x=1$ đều thỏa mãn điều kiện xác định.\\
			Vậy phương trình có nghiệm là $x=4$ hoặc $x=1$.
			\item 
			Với điều kiện xác định $x \neq -4$ và $x \neq 4$, ta có
			\begin{eqnarray*}
				&&\dfrac{2x(x+4)+3(x-4)}{(x-4)(x+4)}=\dfrac{x-12}{x^2-16}\\
				&&\dfrac{2x^2+8x+3x-12}{x^2-16}=\dfrac{x-12}{x^2-16}\\
				&&\dfrac{2x^2+11x-12}{x^2-16}=\dfrac{x-12}{x^2-16}\\
				&&2x^2+11x-12=x-12\\
				&&2x^2+11x-x=0\\
				&&2x^2+10x=0\\
				&&2x(x+5)=0\\
				&&2x=0~\text{hoặc}~x+5=0\\
				&&x=0~\text{hoặc}~x=-5
			\end{eqnarray*}
			Ta thấy $x=0$ và $x=-5$ đều thỏa mã điều kiện xác định.\\
			Vậy phương trình có nghiệm là $x=0$ hoặc $x=-5$.
		\end{enumerate}	
	}
\end{bt}

\begin{bt}%[Dự án EX-9-Đề Cương Toán 9]%[Phạm Minh Khánh]%[9D1V1-2]
	Để loại bỏ $x\,(\%)$ một loại tảo độc khỏi một hồ nước, người ta ước tính chi phí cần bỏ ra là
	\[C(x)=\dfrac{50 x}{100-x} \text { (triệu đồng), với } 0 \leq x<100.\]
	Nếu bỏ ra $450$ triệu đồng, người ta có thể loại bỏ được bao nhiêu phần trăm loại tảo độc đó?
	\loigiai{%
		Thế $C(x)=450$ ta có
		\begin{eqnarray*}
			450&=&\dfrac{50 x}{100-x}
		\end{eqnarray*}
		Điều kiện xác định $x \neq 100$.\\
		Quy đồng mẫu hai vế của phương trình:
		\begin{eqnarray*}
			&&\dfrac{450(100-x)}{100-x}=\dfrac{50 x}{100-x}\\
			&&\dfrac{45000-450x}{100-x}=\dfrac{50 x}{100-x}\\
			&&45\,000-450x=50x\\
			&&45\,000=50x+450x\\
			&&45\,000=500x\\
			&&x=90 \text{ (thỏa mãn điều kiện xác định)}.
		\end{eqnarray*}
		Vậy nếu bỏ ra $450$ triệu đồng thì loại bỏ được $90\%$ loại tảo độc đó.
	}
\end{bt}

\begin{bt}%[Dự án EX-9-Đề Cương Toán 9]%[Phạm Minh Khánh]%[9D1V1-3]
	Một hãng viễn thông nước ngoài có hai gói cước như sau
	\begin{center}
		\begin{tabular}{|l|l|}
			\hline
			\qquad \qquad \qquad Gói cước A & \qquad \qquad \qquad Gói cước B\\
			\hline
			Cước thuê bao hàng tháng $32$ USD & Cước thuê bao hàng tháng là $44$ USD \\
			\hline
			$45$ phút miễn phí & Không có phút miễn phí\\
			\hline
			$0{,}4$ USD cho mỗi phút thêm & $0{,}25$ USD/phút\\
			\hline
		\end{tabular}
	\end{center}
	\begin{enumerate}
		\item Hãy viết một phương trình xác định thời gian gọi (phút) mà phí phải trả trong cùng một tháng của hai gói cước là như nhau và giải phương trình đó.
		\item Nếu khách hàng chỉ gọi tối đa là $180$ phút trong $1$ tháng thì nên dùng gói cước nào? Nếu khách hàng gọi $500$ phút trong $1$ tháng thì nên dùng gói cước nào?
	\end{enumerate}
	\loigiai{
		\begin{enumerate}
			\item Vì cước thuê bao hàng tháng của gói cước A và gói cước B lần lượt là $32$ USD và $44$ USD nên để phí phải trả trong cùng một tháng của hai gói cước là như nhau thì thời gian gọi phải trên $45$ phút.\\
			Gọi $x$ (phút) là thời gian gọi của một khách hàng $(x > 45)$.\\
			Phí phải trả trong một tháng khi sử dụng gói cước A là $0{,}4\cdot(x-45) + 32$ (USD).\\
			Phí phải trả trong một tháng khi sử dụng gói cước B là $0{,}25x + 44$ (USD).\\
			Để phí phải trả trong cùng một tháng của hai gói cước là như nhau thì
			\begin{eqnarray*}
				&&0{,}4\cdot(x-45) + 32 = 0{,}25x + 44\\
				&&\dfrac{2}{5}(x-45) + 32 = \dfrac{1}{4}x + 44\\
				&&\dfrac{3}{20}x= 30\\
				&&x= 200 \textrm{ (nhận).}
			\end{eqnarray*}
			Thời gian gọi thỏa mãn yêu cầu đề bài là $200$ phút.
			\item Nếu khách hàng chỉ gọi tối đa là $180$ phút trong $1$ tháng thì
			\begin{itemize}
				\item Phí tối đa sử dụng theo gói cước A là $0{,}4\cdot(180-45) + 32 = 86$ USD.
				\item Phí tối đa sử dụng theo gói cước B là $0{,}25\cdot 180 + 44 = 89$ USD.
			\end{itemize}
			Vì $86 < 89$ do đó khách hàng nên sử dụng gói cước A.\\
			Nếu khách hàng gọi $500$ phút trong $1$ tháng thì
			\begin{itemize}
				\item Phí tối đa sử dụng theo gói cước A là $0{,}4\cdot(500-45) + 32 = 214$ USD.
				\item Phí tối đa sử dụng theo gói cước B là $0{,}25\cdot 500 + 44 = 169$ USD.
			\end{itemize}
			Vì $214 > 169$ do đó khách hàng nên sử dụng gói cước B.
		\end{enumerate}
	}
\end{bt}

\begin{bt}%[Dự án EX-9-Đề Cương Toán 9]%[Phạm Minh Khánh]%[9D1V1-3]
	Hai người cùng làm chung một công việc thì xong trong $8$ giờ. Hai người cùng làm được $4$ giờ thì người thứ nhất bị điều đi làm công việc khác. Người thứ hai tiếp tục làm việc trong $12$ giờ nữa thì xong công việc. Gọi $x$ là thời gian người thứ nhất làm một mình xong công việc (đơn vị tính là giờ, $x>0$).
	\begin{enumerate}
		\item Hãy biểu thị theo $x$:
		\begin{itemize}
			\item Khối lượng công việc mà người thứ nhất làm được trong $1$ giờ;
			\item Khối lượng công việc mà người thứ hai làm được trong $1$ giờ.
		\end{itemize}
		\item Hãy lập phương trình theo $x$ và giải phương trình đó. Sau đó cho biết, nếu làm một mình thì mỗi người phải làm trong bao lâu mới xong công việc đó.
	\end{enumerate}
	\loigiai{
		\begin{enumerate}
			\item Ta có
			\begin{itemize}
				\item Khối lượng công việc mà người thứ nhất làm được trong $1$ giờ là $\dfrac1x$ (công việc).
				\item Khối lượng công việc mà người thứ hai làm được trong $1$ giờ là $\dfrac18-\dfrac1x$ (công việc).
			\end{itemize}
			\item Vì người thứ nhất chỉ làm $4$ giờ và người thứ hai làm $4+12=16$ giờ nên ta có phương trình:
			\begin{eqnarray*}
				&&4\cdot \dfrac{1}{x}+16\cdot \left(\dfrac{1}{8}-\dfrac{1}{x}\right)=1\\
				&& \dfrac{4}{x}+2-\dfrac{16}{x}=1\\
				&& \dfrac{-12}{x}=-1\\
				&& x=12 \text{ (thỏa mãn điều kiện xác định).}
			\end{eqnarray*}
			Vậy nếu làm một mình thì người thứ nhất cần $12$ giờ mới xong công việc đó.\\
			Khi làm một mình thì trong $1$ giờ, người thứ hai làm được $\dfrac{1}{8}-\dfrac{1}{12}=\dfrac1{24}$ công việc nên người thứ hai làm một mình cần $24$ giờ mới xong công việc đó.
		\end{enumerate}
	}
\end{bt}

\begin{bt}%[Dự án EX-9-Đề Cương Toán 9]%[Phạm Minh Khánh]%[9D1V1-3]
	Một ca nô đi xuôi dòng từ địa điểm $A$ đến địa điểm $B$, rồi lại đi ngược dòng từ địa điểm $B$ trở về địa điểm $A$. Thời gian cả đi và về là $3$ giờ. Tính tốc độ của dòng nước. Biết tốc độ của ca nô khi nước yên lặng là $27$ km/giờ và độ dài quãng đường $AB$ là $40$ km.
	\loigiai{
		Gọi tốc độ của dòng nước là $x$ (km/giờ) ($0 < x < 27$).\\
		Vận tốc của ca nô khi xuôi dòng là $27 + x$ (km/giờ). \\
		Vận tốc của ca nô khi ngược dòng là $27 - x$ (km/giờ). \\
		Thời gian ca nô xuôi dòng từ $A$ đến $B$ là $\dfrac{40}{27+x}$ (giờ).\\
		Thời gian ca nô ngược dòng từ $B$ về $A$ là $\dfrac{40}{27-x}$ (giờ).\\
		Vì thời gian cả đi và về là $3$ giờ nên ta có phương trình 
		\begin{eqnarray*}
			&&\dfrac{40}{27+x} + \dfrac{40}{27-x}= 3 \\
			&&\dfrac{40(27-x)}{(27+x)(27-x)} + \dfrac{40(27+x)}{(27-x)(27+x)}=\dfrac{3(27-x)(27+x)}{(27-x)(27+x)} \\
			&&40(27-x) + 40(27+x) =3(27-x)(27+x) \\
			&&40(27-x + 27+x) = 3(27^2 - x^2) \\
			&&40 \cdot 2 \cdot 27 = 3 \cdot 27^2 - 3x^2 \\
			&&3x^2 = 27 \\
			&&x^2 = 9 \\
			&&x = 3 \text{ hoặc } x = -3.
		\end{eqnarray*}
		Ta thấy $x=3$ thỏa mãn điều kiện xác định; $x=-3$ không thỏa mãn điều kiện xác định.\\
		Vậy tốc độ của dòng nước là $3$ km/h.
	}
\end{bt}

\begin{bt}%[Dự án EX-9-Đề Cương Toán 9]%[Phạm Minh Khánh]%[9D1N2-1]
	Trong các phương trình sau, phương trình nào là phương trình bậc nhất hai ẩn? Xác định các hệ số $a$, $b$, $c$ của mỗi phương trình bậc nhất hai ẩn đó.
	\begin{multicols}{4}
		\begin{enumerate}
			\item $2x + 5y = -7$;
			\item $0x - 0y = 5$;
			\item $0x - \dfrac{5}{4}y = 3$;
			\item $0{,}2x + 0y = -1{,}5$.
		\end{enumerate}
	\end{multicols}
	\loigiai{
		\begin{enumerate}
			\item $2x+5y=-7$ là phương trình bậc nhất hai ẩn với $a=2$, $b=5$, $c=-7$.
			\item $0x-0y=5$ không là phương trình bậc nhất hai ẩn vì $a$ và $b$ đồng thời bằng $0$.
			\item $0x - \dfrac{5}{4}y = 3$ là phương trình bậc nhất hai ẩn với $a=0$, $b=\dfrac{-5}{4}$, $c=3$.
			\item $0{,}2x + 0y = -1{,}5$ là phương trình bậc nhất hai ẩn với $a=0{,}2$, $b=0$, $c=-1{,}5$.
		\end{enumerate}
	}   
\end{bt}

\begin{bt}%[Dự án EX-9-Đề Cương Toán 9]%[Phạm Minh Khánh]%[9D1H2-1]
	Tìm $k$ biết:
	\begin{enumerate}
		\item Phương trình $2x - 3y = 4$ có một nghiệm là $(k;1)$.
		\item Phương trình $\dfrac{x}{2} - \dfrac{y}{3} = \dfrac{1}{6}$ có một nghiệm là $\left(-\dfrac{2}{3}; k \right)$.
		\item Phương trình $3x - 5y = -2$ có một nghiệm là $(2k+1; k-1)$.
	\end{enumerate}
	\loigiai{
		\begin{enumerate}
			\item $(k;1)$ là nghiệm của phương trình $2x-3y=4$ nên ta có
			\begin{eqnarray*}
				&&2\cdot k-3\cdot 1=4\\
				&&k=\dfrac{7}{2}.
			\end{eqnarray*}
			\item $\left(-\dfrac{2}{3};k\right)$ là nghiệm của phương trình $\dfrac{x}{2}-\dfrac{y}{3}=\dfrac{1}{6}$ nên ta có
			\begin{eqnarray*}
				&&\dfrac{\dfrac{-2}{3}}{2}-\dfrac{k}{3}=\dfrac{1}{6}\\[0.2 cm]
				&&\dfrac{-1}{3}-\dfrac{k}{3}=\dfrac{1}{6}\\[0.2 cm]
				&&\dfrac{k}{3}=\dfrac{-1}{2}\\[0.2 cm]
				&&k=\dfrac{-3}{2}.
			\end{eqnarray*}
			\item $(2k+1;k-1)$ là nghiệm của phương trình $3x-5y=-2$ nên ta có
			\begin{eqnarray*}
				&&3(2k+1)-5(k-1)=-2\\
				&&6k+3-5k+5=-2\\
				&&k=-10.
			\end{eqnarray*}
		\end{enumerate}
	}
\end{bt}

\begin{bt}%[Dự án EX-9-Đề Cương Toán 9]%[Phạm Minh Khánh]%[9D1N2-2]
	Cho hệ phương trình $\heva{& 4x - y = 2 \\& x + 3y = 7}$. Cặp số nào dưới đây là nghiệm của hệ phương trình đã cho? Giải thích.
	\begin{multicols}{3}
		\begin{enumerate}
			\item $(2;2)$
			\item $(1;2)$
			\item $(-1;-2)$
		\end{enumerate}
	\end{multicols}
	\loigiai{
		\begin{enumerate}
			\item Thế cặp số $(2;2)$ vào hệ phương trình trên, ta được
			\begin{center}
				$\heva{&4\cdot2-2=2\\&2+3\cdot2=7}$ (vô lí).
			\end{center}
			Vậy cặp số $(2;2)$ không là nghiệm của hệ phương trình.
			\item Thế cặp số $(1;2)$ vào hệ phương trình trên, ta được
			\begin{center}
				$\heva{&4\cdot1-2=2\\&1+3\cdot2=7}$ (đúng).
			\end{center}
			Cặp số $(1;2)$ là nghiệm của hệ phương trình.
			\item Thế cặp số $(-1;-2)$ vào hệ phương trình trên, ta được
			\begin{center}
				$\heva{&4\cdot(-1)-(-2)=2\\&-1+3\cdot(-1)=7}$ (vô lí).
			\end{center}
			Cặp số $(-1;-2)$ không là nghiệm của hệ phương trình.
		\end{enumerate}
	}
\end{bt}

\begin{bt}%[Dự án EX-9-Đề Cương Toán 9]%[Phạm Minh Khánh]%[9D1H2-2]
	Mẹ đưa bạn An $120~000$ đồng để mua một số quyển tập và bút. Biết rằng mỗi quyển tập giá $23~000$ đồng, mỗi cây bút giá $5~000$ đồng và số tiền thừa là $3~000$ đồng.
	
	\begin{enumerate}
		\item Gọi $x$, $y$ lần lượt là số quyển tập và cây bút. Lập phương trình bậc nhất theo hai ẩn $x$ và $y$.
		\item Bạn An nói rằng bạn mua $5$ quyển tập và $3$ cây bút. Sử dụng phương trình lập ở câu a để kiểm tra bạn An nói đúng hoặc sai?
	\end{enumerate}
	\loigiai{
		\begin{enumerate}
			\item Phương trình bậc nhất theo hai ẩn $x$ và $y$ là
			\begin{center}
				$20~000x+6~000y+3~000=120~000$ hoặc $20x+6y-117=0$.
			\end{center}
			\item Thế $x=5$ và $y=3$ vào phương trình ở câu a, ta được
			\begin{center}
				$20\cdot5+6\cdot3-117\neq0$.
			\end{center}
			Vậy bạn An nói sai.
		\end{enumerate}
	}
\end{bt}

\begin{bt}%[Dự án EX-9-Đề Cương Toán 9]%[Phạm Minh Khánh]%[9D1H2-2]
	\textit{
		\begin{center}
			Một đàn em nhỏ đứng bên sông\\
			To nhỏ bàn nhau chuyện chia hồng\\
			Mỗi người năm trái thừa năm trái\\
			Mỗi người sáu trái một người không\\
			Hỏi người bạn trẻ đang dừng bước\\
			Có mấy em thơ, mấy trái hồng?
		\end{center}
	}
	Gọi $x$ là số em nhỏ và $y$ là số quả hồng. Em hãy lập hệ phương trình bậc nhất hai ẩn theo hai ẩn $x$ và $y$.
	\loigiai{
		Mỗi người năm trái thừa năm trái.
		\begin{center}
			$y=5x+5$ hoặc $5x-y=-5$. \qquad $(1)$
		\end{center}
		Mỗi người sáu trái một người không.
		\begin{center}
			$y=6(x-1)$ hoặc $6x-y=6$. \qquad $(2)$
		\end{center}
		Từ $(1)$ và $(2)$ suy ra
		\begin{center}
			$\heva{&5x-y=-5\\&6x-y=6.}$
		\end{center}
	}
\end{bt}

\begin{bt}%[Dự án EX-9-Đề Cương Toán 9]%[Phạm Minh Khánh]%[9D1H2-3]
	Một ô tô đi từ $A$ đến $B$, cùng lúc đó một xe máy đi từ $B$ và $A$ với vận tốc nhỏ hơn tốc độ của ô tô $15$ km/giờ. Biết rằng quãng đường AB dài $210$ km và hai xe gặp nhau sau $2$ giờ.
	\begin{enumerate}
		\item Gọi $x$, $y$ lần lượt là vận tốc của xe máy và ô tô. Lập hệ phương trình bậc nhất theo hai ẩn $x$ và $y$.
		\item Bạn An khẳng định rằng tốc độ của ô tô và xe máy lần lượt là $60$ km/giờ và $45$ km/giờ. Sử dụng hệ phương trình ở câu a để kiểm tra khẳng định của bạn An là đúng hoặc sai?
	\end{enumerate}
	\loigiai{
		\begin{enumerate}
			\item Vận tốc xe máy nhỏ hơn vận tốc của ô tô $15$ km/giờ.
			\begin{center}
				$y=x+15$ hoặc $x-y=-15$. \qquad \quad $(1)$
			\end{center}
			Hai xe xuất phát cùng lúc và gặp nhau sau $2$ giờ.
			\begin{center}
				$2x+2y=210$ hoặc $x+y=105$. \qquad $(2)$
			\end{center}
			Từ $(1)$ và $(2)$, ta có hệ phương trình
			\begin{center}
				$\heva{x-y&=-15\\x+y&=105.}$
			\end{center}
			\item Tốc độ của ô tô và xe máy lần lượt là $60$ km/giờ, $45$ km/giờ.\\
			Khi đó $x=45$ và $y=60$.\\
			Thế $x=45$ và $y=60$ vào hệ phương trình của câu a, ta được
			\begin{center}
				$\heva{45-60&=-15\\45+60&=105.}$
			\end{center}
			Nên khẳng định của bạn An là đúng.
		\end{enumerate}
	}
\end{bt}

\begin{bt}%[Dự án EX-9-Đề Cương Toán 9]%[Phạm Minh Khánh]%[9D1H3-1]
	Giải các hệ phương trình sau bằng phương pháp thế.
	\begin{multicols}{4}
		\begin{enumerate}
			\item $\heva{& 2x-y=3 \\ & x+2y=4}$;
			\item $\heva{& x-3y=2 \\ & -2x+5y=1}$;
			\item $\heva{& 4x+y=-1 \\ & 7x+2y=-3}$;
			\item $\heva{& x-y=-2 \\ & 2x-2y=8}$;
			\item $\heva{& -2x+y=3 \\ & 4x-2y=-4}$;
			\item $\heva{& -x+y=-2 \\ & 3x-3y=6}$;
			\item $\heva{& x+3y=-1 \\ & 3x+9y=-3}$;
			\item $\heva{&3x+y=3\\ &-2x-3y=5.}$
		\end{enumerate}
	\end{multicols}
	\loigiai{
		\begin{enumerate}
			\item
			$\heva{& 2x-y=3 \\ & x+2y=4}$\\
			Từ phương trình thứ nhất của hệ ta có $y=2 x-3$.\\
			Thế vào phương trình thứ hai của hệ, ta được $x+2(2 x-3)=4$ hoặc $5 x-6=4$, suy ra $x=2$.\\
			Từ đó $y=2 \cdot 2-3=1$.\\
			Vậy hệ phương trình có nghiệm duy nhất là $(2;1)$.
			\item
			$\heva{& x-3y=2 \\ & -2x+5y=1}$$;$\\
			Từ phương trình thứ nhất của hệ ta có $x=2+3y$.\\
			Thế vào phương trình thứ hai của hệ, ta được: $-2(2+3y)+5y=1$ hoặc $-4-y=1$, suy ra $y=-5$.\\
			Từ đó, $x=2+3\cdot(-5)=-13$.\\
			Vậy hệ phương trình có nghiệm duy nhất là $(-13;-5)$.\\
			\item $\heva{& 4x+y=-1 \\ & 7x+2y=-3}$\\
			Từ phương trình thứ nhất của hệ ta có $y=-1-4x$.\\
			Thế vào phương trình thứ hai của hệ, ta được: $7x+2(-1-4x)=-3$ hoặc $-x=-1$, suy ra $x=1$.\\
			Từ đó, $y=-1-4\cdot1=-5$.\\
			Vậy hệ phương trình có nghiệm duy nhất là $(1;-5)$.
			\item $\heva{& x-y=-2 \\ & 2x-2y=8}$\\
			Từ phương trình thứ nhất của hệ ta có $y=x+2$.\\
			Thế vào phương trình thứ hai của hệ, ta được $2x-2(x+2)=8$ hoặc $0x-4=8$. \qquad $(1)$ \\
			Do không có giá trị nào của $x$ thỏa mãn hệ thức $(1)$ nên hệ phương trình đã cho vô nghiệm.
			\item
			$\heva{& -2x+y=3 \\ & 4x-2y=-4}$\\
			Từ phương trình thứ nhất của hệ ta có $y=2x+3$.\\
			Thế vào phương trình thứ hai của hệ, ta được $4x-2(2x+3)=-4$ hoặc $0x-6=-4$. \qquad $(1)$ \\
			Do không có giá trị nào của số $x$ thỏa mãn hệ thức $(1)$ nên hệ phương trình vô nghiệm.
			\item 
			$\heva{& -x+y=-2 \\ & 3x-3y=6}$\\
			Từ phương trình thứ nhất của hệ ta có $y=x-2$.  \qquad $(1)$\\
			Thế vào phương trình thứ hai của hệ, ta được $3x-3(x-2)=6$ hoặc $0x+6=6$\\
			hoặc $0x=0$. \qquad$(2)$\\
			Ta thấy mọi giá trị của $x$ đều thỏa mãn $(2)$.\\
			Với mỗi giá trị tùy ý của $x$ , giá trị tương ứng của $y$ được tính bởi $(1)$.\\
			Vậy hệ phương trìnhcó nghiệm là $(x;x-2)$ với $x\in\mathbb{R}$.
			\item 
			$\heva{& x+3y=-1 \\ & 3x+9y=-3}$\\
			Từ phương trình thứ nhất của hệ ta có $x=-1-3y$. \qquad $(1)$\\
			Thế vào phương trình thứ hai của hệ, ta được $3(-1-3y)+9y=-3$ hoặc $0y-3=-3$\\
			hoặc $0y=0$. \qquad (2)\\
			Ta thấy mọi giá trị của $y$ đều thỏa mãn $(2)$.\\
			Với mỗi giá trị tùy ý của $y$ , giá trị tương ứng của $x$ được tính bởi $(1)$.\\
			Vậy hệ phương trình có nghiệm là $(-1-3y;y)$ với $y\in\mathbb{R}$ tùy ý.	
			\item 
			$\heva{&3x+y=3\\ &-2x-3y=5}$\\
			Từ phương trình thức nhất của hệ ta có $y=3-3x$.\\
			Thế vào phương trình thứ hai của hệ, ta được $-2x-3(3-3x)=5$ hoặc $7x=14$, suy ra $x=2$.\\
			Từ đó, $y=3-3\cdot2=-3$.\\
			Vậy hệ phương trình có nghiệm duy nhất là $(2;-3)$.
		\end{enumerate}
	}
\end{bt}

\begin{bt}%[Dự án EX-9-Đề Cương Toán 9]%[Phạm Minh Khánh]%[9D1H3-2]
	Giải các hệ phương trình sau bằng phương pháp cộng đại số.
	\begin{multicols}{4}
		\begin{enumerate}
			\item $\heva{& -2x+5y=12 \\ & 2x+3y=4}$;
			\item $\heva{& 5x-7y=9 \\ & 5x-3y=1}$;
			\item $\heva{& -4x+3y=0 \\ & 4x-5y=-8}$;
			\item $\heva{& 4x+3y=0 \\ & x+3y=9}$;
			\item $\heva{& 3x+2y=7 \\ & 2x-3y=-4}$;
			\item $\heva{& 4x+3y=6 \\ & -5x+2y=4}$;
			\item $\heva{& 3x-5y=2 \\ & -6x+10y=-4}$;
			\item $\heva{& -0{,}5x+0{,}5y=1 \\ & -2x+2y=8.}$
		\end{enumerate}
	\end{multicols}
	\loigiai{
		\begin{enumerate}
			\item
			Cộng từng vế của hai phương trình ta được $8y=16$, suy ra $y=2$.\\
			Thế $y=2$ vào phương trình thứ hai ta được $2x+3\cdot2=4$, hoặc $2x=-2$, suy ra $x=-1$.\\
			Vậy hệ phương trình có nghiệm duy nhất là $(-1;2)$.
			\item
			Trừ từng vế của hai phương trình ta được $(5x-5x)+(-7y+3y)=9-1$, hoặc $-4y=8$, suy ra $y=-2$.\\
			Thế $y=-2$ vào phương trình thứ nhất, ta được $5x-7\cdot(-2)=9$ hoặc $5x+14=9$, suy ra $x=-1$.\\
			Vậy hệ phương trình có nghiệm duy nhất $(-1;-2)$.
			\item
			Cộng từng vế của hai phương trình ta được $-2y=-8$, suy ra $y=4$.\\
			Thế $y=4$ vào phương trình thứ nhất, ta được $-4x+3\cdot4=0$, hoặc $-4x+12=0$, suy ra $x=3$.\\
			Vậy hệ phương trình có nghiệm duy nhất $(3;4)$.
			\item
			Trừ từng vế của hai phương trình, ta được $3x=-9$, suy ra $x=-3$.\\
			Thế $x=-3$ vào phương trình thứ nhất, ta được $4\cdot(-3)+3y=0$, hoặc $-12+3y=0$, suy ra $y=4$.\\
			Vậy hệ phương trình có nghiệm duy nhất $(-3;4)$.
			\item
			Nhân hai vế của phương trình thứ nhất với $3$ và nhân hai vế của phương trình thứ $2$, ta được:\\
			\centerline{$\heva{& 9x+6y=21\\ & 2x-6y=-8}$}.\\
			Cộng từng vế hai phương trình của hệ mới, ta được $13x=13$ hoặc $x=1$.
			Thế $x=1$ vào phương trình thứ nhất của hệ đã cho, ta có $3.1+2y=7$, suy ra $y=2$. \\
			Vậy hệ phương trình có nghiệm duy nhất là $(1;2)$.
			\item
			Nhân cả hai vế của phương trình thứ nhất cho $5$ và nhân của hai vế của phương trình thứ hai cho $4$ ta được hệ $\heva{& 20x+15y=30 \\& -20x+8y=16}$.\\
			Cộng từng vế hai phương trình của hệ mới, ta được $23y=46$, suy ra $y=2$. \\
			Thế $x=2$ vào phương trình thứ nhất của hệ đã cho, ta có: $4x+3\cdot2=6$, hoặc $4x=0$, suy ra $x=0$.\\
			Vậy hệ phương trình có nghiệm duy nhất là $(0;2)$.
			\item 
			Chia cả hai vế phương trình thứ hai cho $2$, ta được hệ $\heva{& 3x-5y=2 \\& -3x+5y=-2.}$\\
			Cộng từng vế hai phương trình của hệ mới ta được $0x+0y=0$. Hệ thức này luôn thỏa mãn với các giá trị tùy ý của $x$ và $y$.\\
			Với giá trị tùy ý của $x$, giá trị của $y$ tính được nhờ hệ thức $3x-5y=2$, suy ra $y=\dfrac{3}{5}x-\dfrac{2}{5}$.\\
			Vậy hệ phương trình đã cho có nghiệm là $(x;\dfrac{3}{5}x-\dfrac{2}{5})$ với $x\in\mathbb{R}$
			\item
			Nhân cả hai vế của phương trình thứ nhất cho 4, ta được hệ $\heva{& -2x+2y=4\\& -2x+2y=8.}$\\
			Cộng từng vế của hai phương trình của hệ mới ta được $0x+0y=12$. Không tìm được bất kì giá trị $x$, $y$ nào thỏa mãn hệ thức này.\\
			Vậy hệ phương trình vô nghiệm.
		\end{enumerate}
	}
\end{bt}

\begin{bt}%[Dự án EX-9-Đề Cương Toán 9]%[Phạm Minh Khánh]%[9D1H3-4]
	Tìm hai số nguyên dương biết tổng của chúng bằng $1\,006$, nếu lấy số lớn chia cho số bé được thương là $2$ và số dư là $124$.
	\loigiai{
		Gọi $x$, $y$ lần lượt là số lớn và số bé với $x,y \in \mathbb{N^*}$.\\
		Tổng của chúng bằng $1\,006$, nên ta có phương trình $x+y=1006$. \qquad $(1)$\\
		Số lớn chia cho số bé được thương là $2$ và số dư là $124$, nên ta có phương trình $x=2y+124$ hoặc $x-2y=124$. \qquad $(2)$ \\
		Từ $(1)$ và $(2)$, ta có hệ phương trình 
		\begin{center}
			$\heva{&x+y=1006\\&x-2y=124.}$
		\end{center}
		Giải hệ phương trình, ta được
		\begin{center}
			$ \heva{&x=712\\&y=294}$ (thỏa mãn điều kiện xác định).
		\end{center} 
		Vậy hai số đó là $712$ và $294$.
	}
\end{bt}

\begin{bt}%[Dự án EX-9-Đề Cương Toán 9]%[Phạm Minh Khánh]%[9D1H3-4]
	Tìm số tự nhiên $N$ có hai chữ số, biết rằng nếu viết thêm chữ số $3$ vào giữa hai chữ số của số $N$ thì được một số lớn hơn số hai lần số $N$ là $585$ đơn vị, và nếu viết hai chữ số của số $N$ theo thứ tự ngược lại thì được một số nhỏ hơn số $N$ là $18$ đơn vị.
	\loigiai{
		Gọi $N=\overline{ab}$, với $a,b\in \mathbb{N}^{*}$.\\
		Nếu viết thêm chữ số $3$ vào giữa hai chữ số của số $N$ thì được một số lớn hơn số hai lần số $N$ là $585$ đơn vị, nên ta có phương trình $\overline{a3b}-2\overline{ab}=585$ hoặc $100a+30+b-2(10a+b)=585$. \qquad $(1)$\\
		Nếu viết hai chữ số của số $N$ theo thứ tự ngược lại thì được một số nhỏ hơn số $N$ là $18$ đơn vị, nên ta có phương trình $\overline{ab}-\overline{ba}=18$ hoặc $10a+b-(10b+a)=18$. \qquad $(2)$\\
		Từ $(1)$ và $(2)$, ta có hệ phương trình
		\begin{center}
			$\heva{&100a+30+b-2(10a+b)=585\\&10a+b-(10b+a)=18}$ hoặc $\heva{&80a-b=555\\&-a+b=-2.}$
		\end{center}
		Giải hệ phương trình, ta được
		\begin{center}
			$\heva{&a=7\\&b=5.}$
		\end{center}
		Các giá trị $a=7$ và $b=5$ đều thỏa mãn điều kiện bài toán.\\
		Vậy số $N$ cần tìm là $75$.
	}
\end{bt}

\begin{bt}%[Dự án EX-9-Đề Cương Toán 9]%[Phạm Minh Khánh]%[9D1V3-4]
	Trong một đợt khuyến mãi, siêu thị giảm giá cho mặt hàng A là $20 \%$ và mặt hàng B là $15 \%$ so với giá niêm yết. Một khách hàng mua $2$ món hàng A và $1$ món hàng B thì phải trả số tiền là $362\,000$ đồng. Nhưng nếu mua trong khung giờ vàng thì mặt hàng A được giảm giá $30 \%$ và mặt hàng B được giảm giá $25 \%$ so với giá niêm yết. Một khách hàng mua $3$ món hàng A và $2$ món hàng B trong khung giờ vàng nên phải trả số tiền là $552\,000$ đồng. Tính giá niêm yết của mỗi mặt hàng A và B. 
	\loigiai{
		Gọi $x$, $y$ (đồng) lần lượt là giá niêm yết của mỗi mặt hàng A và B ($x>0$, $y>0$).\\
		Một khách hàng mua $2$ món hàng A và $1$ món hàng B thì phải trả số tiền là $362\,000$ đồng nên ta có
		$$80\%x\cdot2+85\%y=362\,000\text{ hoặc } 1{,}6x+0{,}85y=362\,000.\qquad (1)$$
		Trong giờ vàng, khách hàng mua 3 món hàng A và 2 món hàng B phải trả số tiền là $552\,000$ đồng nên ta có
		$$70\%x\cdot3+75\%y\cdot2=552\,000\text{ hoặc } 2{,}1x+1{,}5y=552\,000.\qquad (2)$$
		Từ (1) và (2), ta có hệ phương trình 
		\begin{center}
			$\heva{&1{,}6x+0{,}85y=362\,000\\&2{,}1x+1{,}5y=552\,000.}$
		\end{center}
		Giải hệ phương trình trên, ta được 
		\begin{center}
			$\heva{&x=120\,000 \\&y=200\,000.}$ (nhận)
		\end{center}
		Vậy giá niêm yết của mặt hàng A là $120\,000$ đồng, mặt hàng B là $200\,000$ đồng.
	}
\end{bt}

\begin{bt}%[Dự án EX-9-Đề Cương Toán 9]%[Phạm Minh Khánh]%[9D1V3-4]
	Tại một buổi biểu diễn nhằm gây quỹ từ thiện, ban tổ chức đã bán được $500$ vé. Trong đó có hai loại vé: vé loại $I$ giá $100\,000$ đồng; vé loại $II$ giá $75\,000$ đồng. Tổng số tiền thu được từ bán vé là $44\,500\,000$ đồng. Tính số vé bán ra của mổi loại.
	\loigiai{
		Gọi $x$, $y$ lần lượt là số vé loại $I$, $II$ mà ban tổ chức bán ra ($x \in \mathbb{N}$, $y \in \mathbb{N^*}$).\\
		Vì ban tổ chức bán được $500$ vé nên ta có phương trình $$x+y=500.\qquad (1)$$
		Vì tổng số tiền thu được từ bán vé là $44\,500\,000$ đồng nên ta có phương trình
		\begin{center}
			$100\,000x+ 75\,000y=44\,500\,000.\qquad (2)$
		\end{center}
		Từ (1) và (2), ta có hệ phương trình
		\begin{center}
			$\heva{&x+y=500\\& 100\,000x+ 75\,000y=44\,500\,000.} $
		\end{center}
		Giải hệ phương trình trên, ta được 
		\begin{center}
			$\heva{&x=280\\& y=220.}$ (nhận)
		\end{center}
		Vậy số vé loại $I$ bán được là $280$ vé, số vé loại $II$ bán được là $220$ vé.
	}
\end{bt}


