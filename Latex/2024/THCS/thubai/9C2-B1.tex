\section{BẤT ĐẲNG THỨC} % Tên bài
\subsection{KIẾN THỨC TRỌNG TÂM}
\subsubsection{Khái niệm bất đẳng thức}
\begin{tomtat}
	Trên tập hợp số thực, khi so sánh hai số $a$ và $b$, xảy ra một trong ba trường hợp sau
	\begin{itemize}
		\item Số $a$ lớn hơn $b$, kí hiệu $a>b$.
		\item Số $a$ nhỏ hơn $b$, kí hiệu $a<b$.
		\item Số $a$ bằng $b$, kí hiệu $a=b$.
	\end{itemize}
	Ngoài ra, ta có thể ghi gọn như sau:
	\begin{itemize}
		\item Nếu $a>b$ hoặc $a=b$, ta viết $a\geq b$ (ta nói $a$ {\it lớn hơn hoặc bằng} $b$ hay $a$ {\it không nhỏ hơn} b).
		\item Nếu $a<b$ hoặc $a=b$, ta viết $a\leq b$ (ta nói $a$ {\it nhỏ hơn hoặc bằng} hay $a$ {\it không lớn hơn} $b$).
	\end{itemize}
Hệ thức dạng $a>b$ (hay $a<b$, $a\geq b$, $a\leq b$) được gọi là {\it bất đẳng thức} và $a$ được gọi là {\it vế trái}, $b$ được gọi là {\it vế phải} của bất đẳng thức.
\subsubsection{Tính chất của bất đẳng thức}
	\begin{itemize}
	\item {\bf Tính chất bắc cầu}\\
	Cho ba số $a$, $b$, $c$. Nếu $a>b$ và $b>c$ thì $a>c$.
	\item {\bf Tính chất liện hệ giữa thứ tự và phép cộng}\\
	Cho ba số $a$, $b$, $c$. Nếu $a>b$ thì $a+c>b+c$.
	\item {\bf Tính chất liên hệ giữa thứ tự và phép nhân}\\
	Cho ba số $a$, $b$, $c$ và $a>b$.
		\begin{itemize}
		\item Nếu $c>0$ thì $a\cdot c>b\cdot c$;
		\item Nếu $c<0$ thì $a\cdot c<b\cdot c$.
		\end{itemize}
	\end{itemize}
	\begin{luuy}
		Các tính chất trên vẫn đúng với các bất đẳng thức có dấu $<$, $\geq$, $\leq$.
	\end{luuy}
\end{tomtat}

\subsection{CÁC VÍ DỤ}
%Dùng bất đẳng thức diễn tả một khẳng định.
\begin{vd}%[Dự án EX-9-Đề Cương Toán 9]%[Huu Son]%[9D2N1-1]
	Dùng các kí hiệu $>$, $<$, $\geq$, $\leq$ để diễn tả
	\begin{enumerate}
		\item Số tiền $a$ mà Hùng dùng để mua sách nhỏ hơn $500$ ngàn đồng;
		\item Số $x$ là số dương;
		\item Tải trọng $M$ của thang máy không vượt quá $1\,000$ kg;
		\item Điểm tổng kết trung bình $T$ tối thiểu để đạt học sinh giỏi là $8$.
	\end{enumerate}
	\loigiai{
	\begin{multicols}{2}
		\begin{enumerate}
		\item $a<500$;
		\item $x > 0$;
		\item $M\leq 1\,000$;
		\item $T\geq 8$.
		\end{enumerate}
	\end{multicols}
	}
\end{vd}
%Dùng bất đẳng thức để so sánh.
\begin{vd}%[Dự án EX-9-Đề Cương Toán 9]%[Huu Son]%[9D2H1-2]
	So sánh hai số $m$ và $n$ trong mỗi trường hợp sau
	\begin{enumerate}
		\item $m+15<n+15$;
		\item $-17m\geq -17n$;
		\item $\dfrac{m}{7}-5\leq\dfrac{n}{7}-5$;
		\item $-0{,}7n+10>-0{,}7m+10$.
	\end{enumerate}
	\loigiai{
	\begin{enumerate}
	\item Ta có
		\begin{eqnarray*}
			m+15 &<& n+15\\
			m+15+(-15) &<& n+15+(-15) \text{ (cộng hai vế với } -15)\\
			m &<& n.
		\end{eqnarray*}
	\item Ta có
	\begin{eqnarray*}
		-17m &\geq & -17n\\
		(-17m)\cdot\left(-\dfrac{1}{17}\right) &\leq & (-17n)\cdot\left(-\dfrac{1}{17}\right)\;\left(\text{nhân hai vế với } -\dfrac{1}{17}\right)\\
		m &\leq & n.
	\end{eqnarray*}
	\item Ta có
	\begin{eqnarray*}
		\dfrac{m}{7}-5 &\leq & \dfrac{n}{7}-5\\
		\dfrac{m}{7}-5+5 &\leq & \dfrac{n}{7}-5+5\text{ (cộng hai vế với } 5)\\
		\dfrac{m}{7}\cdot 7	&\leq & \dfrac{n}{7}\cdot 7 \text{ (nhân hai vế với } 7)\\
		m &\leq & 7.
	\end{eqnarray*}
	\item Ta có
	\begin{eqnarray*}
		-0{,}7n+10	&>& -0{,}7m+10\\
		-0{,}7n+10+(-10) &>& -0{,}7m+10+(-10)\text{ (cộng hai vế với } -10)\\
		-0{,}7n\cdot\left(-\dfrac{1}{0{,}7}\right) &<& -0{,}7m\cdot\left(-\dfrac{1}{0{,}7}\right)\;\left(\text{nhân hai vế với } -\dfrac{1}{0{,}7}\right)\\
		n &<& m.
	\end{eqnarray*}
	\end{enumerate}
	}
\end{vd}
%Dùng bất đẳng thức để chứng minh.
\begin{vd}%[Dự án EX-9-Đề Cương Toán 9]%[Huu Son]%[9D2H1-2]
	Cho $a$ và $b$ là hai số dương và $a>b$. Chứng minh rằng $a^2>b^2$.
	\loigiai{
	Vì $a>0$ và $b>0$ nên
	\begin{itemize}
		\item Nhân hai vế của $a>b$ với $a$ ta được $a^2>ab. \qquad(1)$
		\item Nhân hai vế của $a>b$ với $b$ ta được $ab>b^2. \qquad(2)$
	\end{itemize}
	Từ $(1)$ và $(2)$ ta suy ra $a^2>b^2$.
	}
\end{vd}
%Bất đẳng thức Cosi đơn giản.
\begin{vd}%[Dự án EX-9-Đề Cương Toán 9]%[Huu Son]%[9D2H1-2]
	Cho $a$ và $b$ là hai số tùy ý. Chứng minh rằng $\left( \dfrac{a+b}{2} \right)^2 \geq ab$.
	\loigiai{
	Vì $a$ và $b$ là hai số không âm nên $a>0$ và $b>0$. \\
	Ta có
	\begin{eqnarray*}
		\left( a-b \right)^2 &\geq& 0 \\
		a^2 + b^2 - 2ab &\geq& 0 \\
		a^2 + b^2 + 2ab &\geq& 4ab \\
		\left( a+b \right)^2 &\geq& 4ab \\
		\left( \dfrac{a+b}{2} \right)^2 &\geq& ab.
	\end{eqnarray*}
	}
\end{vd}

\subsection{BÀI TẬP}

\begin{bt}%[Dự án EX-9-Đề Cương Toán 9]%[Huu Son]%[9D2N1-1]
	Hãy chỉ ra bất đẳng thức diễn tả mỗi khẳng định sau
\begin{enumerate}
	\item $a$ bình phương không nhỏ hơn $0$;
	\item $2$ mũ $5$ lớn hơn $5$ mũ $2$;
	\item $-x^2+4$ không lớn hơn $4$;
	\item $y$ bình phương cộng $1$ lớn hơn $0$.
\end{enumerate}
	\loigiai{
\begin{multicols}{2}
	\begin{enumerate}
		\item $a^2\geq 0$;
		\item $2^5>5^2$;
		\item $-x^2+4\leq 4$;
		\item $y^2+1>0$.
	\end{enumerate}
\end{multicols}
	}
\end{bt}

\begin{bt}%[Dự án EX-9-Đề Cương Toán 9]%[Huu Son]%[9D2N1-1]
Dùng các dấu $>$, $<$, $\geq$, $\leq$ để diễn tả
\begin{enumerate}
\item Giá bán thấp nhất $T$ của một chiếc điện thoại là $6$ triệu đồng.
\item Thời gian tối đa $t$ để hoàn thành một dự án là $12$ tháng.
\end{enumerate}
\loigiai{
\begin{multicols}{3}
\begin{enumerate}
\item $T\geq 6$ (triệu đồng);
\item $t\leq 12$ (tháng).
\end{enumerate}
\end{multicols}
}
\end{bt}

\begin{bt}%[Dự án EX-9-Đề Cương Toán 9]%[Huu Son]%[9D2N1-2]
	Điền vào chỗ chấm dấu $>$, $=$, hoặc $<$ để tạo thành một phát biểu đúng.
	\begin{enumerate}
		\item Nếu $17>10$ và $10>p$ thì $17\ldots p$.
		\item Nếu $-11>x$ và $x>y$ thì $-11\ldots y$.
		\item Nếu $a<100$ và $b>100$ thì $b\ldots a$.
		\item Nếu $x+1=y$ thì $x\ldots y$.
		\item Nếu $3x=3y$ thì $x\ldots y$.
	\end{enumerate}
	\loigiai{
	\begin{enumerate}
		\item Nếu $17>10$ và $10>p$ thì $17>p$.
		\item Nếu $-11>x$ và $x>y$ thì $-11>y$.
		\item Nếu $a<100$ và $b>100$ thì $b>a$.
		\item Nếu $x+1=y$ thì $x<y$.
		\item Nếu $3x=3y$ thì $x=y$.
	\end{enumerate}
	}
\end{bt}

\begin{bt}%[Dự án EX-9-Đề Cương Toán 9]%[Huu Son]%[9D2H1-2]
Hãy cho biết các bất đẳng thức được tạo thành khi
	\begin{enumerate}
		\item Cộng hai vế của bất đẳng thức $p+2>5$ với $-2$;
		\item Cộng hai vế của bất đẳng thức $x+10\leq y+11$ với $9$;
		\item Nhân hai vế của bất đẳng thức $\dfrac{1}{3}x<5$ với $3$, rồi tiếp tục cộng với $-15$;
		\item Cộng vào hai vế của bất đẳng thức $2m\leq -3$ với $-1$, rồi tiếp tục nhân với $-\dfrac{1}{2}$.
	\end{enumerate}
	\loigiai{
\begin{enumerate}
\item Ta có
	\begin{eqnarray*}
		p+2	&>& 5\\
		p+2+(-2)	&>& 5+(-2)\\
		p				&>& 3.
	\end{eqnarray*}
\item Ta có
	\begin{eqnarray*}
		x+10		&\leq & y+11\\
		x+10+9	&\leq & y+11+9\\
		x+19		&\leq & y+20.
	\end{eqnarray*}
\item Ta có
	\begin{eqnarray*}
		\dfrac{1}{3}x					&<& 5\\
		\dfrac{1}{3}x\cdot 3				&<& 5\cdot 3\\
		\dfrac{1}{3}x\cdot 3+(-15)		&<& 5\cdot 3+(-15)\\
		x-15					&<& 0. 
	\end{eqnarray*}
\item Ta có
	\begin{eqnarray*}
		2m	&\leq &-3\\
		2m+(-1)	&\leq & -3+(-1)\\
		\left(2m-1\right)\cdot\left(-\dfrac{1}{2}\right)	&\geq &(-4)\cdot\left(-\dfrac{1}{2}\right)\\
		-m+\dfrac{1}{2}	&\geq & 2.
	\end{eqnarray*}
\end{enumerate}
	}
\end{bt}

\begin{bt}%[Dự án EX-9-Đề Cương Toán 9]%[Huu Son]%[9D2H1-2]
	Hãy cho biết các bất đẳng thức tạo thành khi
	\begin{enumerate}
		\item Cộng hai vế của bất đẳng thức $m\leq n+1$ với $99$;
		\item Cộng hai vế của bất đẳng thức $y+2\,023>5$ với $-2\,023$;
		\item Nhân hai vế của bất đẳng thức $3x+11>1$ với $\dfrac{1}{3}$, rồi tiếp tục cộng với $-10$;
		\item Cộng hai vế của bất đẳng thức $-2m+a\leq -1$ với $-a$, rồi tiếp tục nhân với $-\dfrac{1}{2}$.
	\end{enumerate}
	\loigiai{
\begin{enumerate}
\item Ta có
	\begin{eqnarray*}
		m			&\leq & n+1\\
		m+99	&\leq & n+1+99\\
		m+99	&\leq & n+100.
	\end{eqnarray*}
\item Ta có
	\begin{eqnarray*}
		y+2\,023	&>& 5\\
		y+2\,023+(-2\,023)	&>& 5+(-2\,023)\\
		y								&>& -2\,018.
	\end{eqnarray*}
\item Ta có
	\begin{eqnarray*}
		3x+11	&>& 1\\
		(3x+11)\cdot\dfrac{1}{3}	&>& 1\cdot\dfrac{1}{3}\\
		x+\dfrac{11}{3}	&>& \dfrac{1}{3}\\
		x+\dfrac{11}{3}+(-10)	&>&\dfrac{1}{3}+(-10)\\
		x-\dfrac{19}{3}	&>&-\dfrac{29}{3}.
	\end{eqnarray*}
\item Ta có
	\begin{eqnarray*}
		-2m+a	&\leq & -1\\
		-2m+a+(-a)	&\leq &-1+(-a)\\
		-2m	&\leq & -1-a\\
		-2m\cdot\left(-\dfrac{1}{2}\right)	&\geq & (-1-a)\cdot\left(-\dfrac{1}{2}\right)\\
		m	&\geq &\dfrac{1+a}{2}.
	\end{eqnarray*}
\end{enumerate}
	}
\end{bt}

\begin{bt}%[Dự án EX-9-Đề Cương Toán 9]%[Huu Son]%[9D2H1-2]
	Tìm
	\begin{enumerate}
		\item Số nguyên $x$ lớn nhất thoả mãn $3x\leq 30$;
		\item Số nguyên $x$ nhỏ nhất thoả mãn $\dfrac{2x}{3}>7$;
		\item Số nguyên tố nhỏ nhất thoả mãn $\dfrac{7x}{8}>11$;
		\item Số nguyên tố lớn nhất thoả mãn $5x+7<29$.
	\end{enumerate}
	\loigiai{
\begin{enumerate}
\item Ta có 
	\begin{eqnarray*}
		3x		&\leq & 30\\
		3x\cdot\dfrac{1}{3}	&\leq & 30\cdot\dfrac{1}{3}\\
		x	&\leq & 10.
	\end{eqnarray*}
Vậy số nguyên $x$ lớn nhất thoả mãn $3x\leq 30$ là $10$.
\item Ta có
	\begin{eqnarray*}
		\dfrac{2x}{3}	&>& 7\\
		\dfrac{2x}{3}\cdot\dfrac{3}{2}	&>& 7\cdot\dfrac{3}{2}\\
		x	&>& \dfrac{21}{2}.
	\end{eqnarray*}
Vậy số nguyên tố $x$ nhỏ nhất thoả mãn $\dfrac{2x}{3}>7$ là $11$.
\item Ta có
	\begin{eqnarray*}
		\dfrac{7x}{8}	&>& 11 \\
		\dfrac{7x}{8}\cdot\dfrac{8}{7}	&>& 11\cdot\dfrac{8}{7}.\\
		x	&>& \dfrac{88}{7}.
	\end{eqnarray*}
Vậy số nguyên tố nhỏ nhất thoả mãn $\dfrac{7x}{8}>11$ là $13$.
\item Ta có
	\begin{eqnarray*}
		5x+7		&<& 29\\
		5x+7+(-7)		&<& 29+(-7)\\
		5x			&<& 22\\
		5x\cdot\dfrac{1}{5}	&<& 22\cdot\dfrac{1}{5}\\
		x	&<&\dfrac{22}{5}.
	\end{eqnarray*}
Vậy số nguyên tố lớn nhất thoả mãn $5x+7<29$ là $3$.
\end{enumerate}
	}
\end{bt}

\begin{bt}%[Dự án EX-9-Đề Cương Toán 9]%[Huu Son]%[9D2H1-2]
	Tìm
	\begin{enumerate}
		\item số nguyên lẻ $x$ nhỏ nhất thoả mãn $3x>27$.
		\item số nguyên $y$ lớn nhất thoả mãn $\dfrac{2y}{5}\leq 13$.
		\item số nguyên tố $x$ nhỏ nhất thoả mãn $\dfrac{8x}{15}>10$.
		\item số nguyên tố $x$ lớn nhất thoả mãn $x+2\leq 25$.
	\end{enumerate}
	\loigiai{
\begin{enumerate}
\item Ta có
	\begin{eqnarray*}
		3x		&>& 27\\
		3x\cdot\dfrac{1}{3}	&>& 27\cdot\dfrac{1}{3}\\
		x	&>& 9.
	\end{eqnarray*}
Vậy số nguyên $x$ nhỏ nhất thoả mãn $3x>27$ là $11$.
\item Ta có
	\begin{eqnarray*}
		\dfrac{2y}{5}	&\leq &13\\
		\dfrac{2y}{5}\cdot\dfrac{5}{2}	&\leq &13\cdot\dfrac{5}{2}\\
		y	&\leq &\dfrac{65}{2}.
	\end{eqnarray*}
Vậy số nguyên $y$ lớn nhất thoả mãn $\dfrac{2y}{5}\leq 13$ là $32$.
\item Ta có
	\begin{eqnarray*}
		\dfrac{8x}{15}	&>& 10\\
		\dfrac{8x}{15}\cdot\dfrac{15}{8}	&>& 10\cdot\dfrac{15}{8}\\
		x	&>& \dfrac{75}{4}.
	\end{eqnarray*}
Số nguyên tố $x$ nhỏ nhất thoả mãn $\dfrac{8x}{15}>10$ là $19$.
\item Ta có
	\begin{eqnarray*}
		x+2	&\leq & 25\\
		x+2+(-2)	&\leq & 25+(-2)\\
		x	&\leq & 23.
	\end{eqnarray*}
Số nguyên tố $x$ lớn nhất thoả mãn $x+2\leq 25$ là $23$.
\end{enumerate}
	}
\end{bt}

\begin{bt}%[Dự án EX-9-Đề Cương Toán 9]%[Huu Son]%[9D2V1-2]
	Cho $a>0$ và $b>0$. Chứng tỏ $a+b>0$.
	\loigiai{
	Cộng hai vế của $a>0$ với $b$ ta được $a+b>b$, do $b>0$ nên $a+b>0$.
	}
\end{bt}

\begin{bt}%[Dự án EX-9-Đề Cương Toán 9]%[Huu Son]%[9D2V1-2]
Cho $a$, $b$, $c$, $d$ là các số thực thoả mãn $a>b$ và $c>d$.
	\begin{enumerate}
		\item Chứng minh $a+c>b+d$.
		\item $a-c>b-d$ có luôn luôn đúng không? Nếu không, hãy cho ví dụ.
	\end{enumerate}
	\loigiai{
\begin{enumerate}
\item Cộng $c$ vào hai vế của $a>b$ ta được
	\begin{eqnarray*}
		a+c>b+c. \qquad(1)
	\end{eqnarray*}
	Cộng $b$ vào hai vế của $c>d$ ta được
	\begin{eqnarray*}
		c+b>d+b. \qquad(2)
	\end{eqnarray*}
Từ $(1)$ và $(2)$ suy ra
\begin{eqnarray*}
a+c>b+d\text{ (tính chất bắc cầu)}.
\end{eqnarray*}
\item Không phải luôn luôn đúng.\\
Lấy $a=10$, $b=9$, $c=5$, $d=1$, ta có $10>9$ và $5>1$, tuy nhiên $10-5<9-1$.
\end{enumerate}
	}
\end{bt}

\begin{bt}%[Dự án EX-9-Đề Cương Toán 9]%[Huu Son]%[9D2H1-2]
	Cho $a$ và $b$ là hai số không âm. Chứng minh rằng $\dfrac{a+b}{2} \geq \sqrt{ab}$ \textit{(bất đẳng thức Cauchy)}. Dấu ``=''xảy ra khi nào?
	\loigiai{
	Vì $a$ và $b$ là hai số không âm nên $a>0$ và $b>0$. \\
	Ta có
	\begin{eqnarray*}
		\left( \sqrt{a}-\sqrt{b} \right)^2 &\geq& 0 \\
		a + b - \sqrt{ab} &\geq& 0 \\
		a + b &\geq& \sqrt{ab}.
	\end{eqnarray*}
	Dấu ``='' xảy ra khi $a=b$.
	}
\end{bt}

\begin{bt}%[Dự án EX-9-Đề Cương Toán 9]%[Huu Son]%[9D2V1-2]
	Cho $a$ và $b$ là hai số dương. Chứng minh rằng $\dfrac{a}{b} + \dfrac{b}{a} \geq 2$.
	\loigiai{
	Với $a$ và $b$ là hai số dương, ta có
	\begin{eqnarray*}
		\left(a-b\right)^2 &\geq& 0 \\
		a^2 + b^2 - 2ab &\geq& 0 \\
		a^2 + b^2 &\geq& 2ab \\
		\dfrac{a}{b} + \dfrac{b}{a} &\geq& 2
	\end{eqnarray*}
	}
\end{bt}

\begin{bt}%[Dự án EX-9-Đề Cương Toán 9]%[Huu Son]%[9D2C1-2]
	Tìm giá trị nhỏ nhất của biểu thức $P = x + \dfrac{16}{x}$ với $x>0$.
	\loigiai{
	Với $x$ và $\dfrac{16}{x}$ là hai số dương, theo bất đẳng thức Cauchy ta có 
		$$P = x + \dfrac{16}{x} \geq 2 \sqrt{x \cdot \dfrac{16}{x}} \geq 2\sqrt{16} \geq 8.$$
	Vậy giá trị nhỏ nhất của biểu thức $P$ là $8$ khi $x = \dfrac{16}{x}$ hay $x = 8$.
	}
\end{bt}

\begin{bt}%[Dự án EX-9-Đề Cương Toán 9]%[Huu Son]%[9D2V1-2]
	Trong các hình chữ nhật có chu vi bằng $300$ m, hình chữ nhật có diện tích lớn nhất bằng bao nhiêu?
	\loigiai{
	Gọi $a$ và $b$ lần lượt chiều dài và chiều rộng của hình chữ nhật (mét; $a>0$, $b>0$). \\
	Với $a>0$, $b>0$ là hai số dương, theo bất đẳng thức Cauchy ta có 
		$$P = 2(a + b) \geq 2 \cdot 2\sqrt{ab} \geq 4\sqrt{S}.$$
	Khi đó $P^2 \geq 16S$ hay $S \leq \dfrac{P^2}{16} = 5\,625$. \\
	Vậy hình chữ nhật có diện tích lớn nhất bằng $5\,625$ m$^2$ khi $a = b = 75$ m.
	}
\end{bt}


