\section{TÍNH CHẤT PHÉP KHAI PHƯƠNG} % Tên bài
\subsection{Căn thức bậc hai một thương và trục căn thức}
\subsubsection{Kiến thức trọng tâm}
\begin{tomtat}
	\begin{itemize}
		\item Với số thực $a$ không âm và số thực $b$ dương, ta có $\sqrt{\dfrac{a}{b}}=\dfrac{\sqrt{a}}{\sqrt{b}}$.\\
		\item  Với biểu thức $A$ nhận giá trị không âm và biểu thức $B$ nhận giá trị dương, ta có $\sqrt{\dfrac{A}{B}}=\dfrac{\sqrt{A}}{\sqrt{B}}$.q
		\item Trục căn thức ở mẫu là làm cho mẫu mất căn mà không làm thay đổi giá trị của biểu thức.
		\begin{itemize}
			\item Ưu tiên 1: Phân tích cả tử và mẫu ra nhân tử rồi rút gọn.
			\item Ưu tiên 2: Nhân cả tử và mẫu cho chính căn ở mẫu.
			\item Ưu tiên 3: Nhân cả tử và mẫu cho biểu thức liên hiệp của mẫu.\\
			\textit{\textbf{Lưu ý:}} $A-B$ và $A+B$ được gọi là hai biểu thức liên hiệp của nhau.
		\end{itemize}
	\end{itemize}
\end{tomtat}


\begin{vd}%[Dự án EX-9-Đề Cương Toán 9]%[Dat Tien Pham]%[9D3N3-3]
	Tính
	\begin{multicols}{2}
		\begin{enumerate}
			\item $\sqrt{\dfrac{49}{81}}$;
			\item $\sqrt{\dfrac{0{,}169}{2{,}5}}$.
		\end{enumerate}
	\end{multicols}
	\loigiai{
		\begin{multicols}{2}
			\begin{enumerate}
				\item $\sqrt{\dfrac{49}{81}}=\dfrac{\sqrt{49}}{\sqrt{81}}=\dfrac{7}{9}$.
				\item $\sqrt{\dfrac{0{,}169}{2{,}5}}=\sqrt{\dfrac{169}{2500}}=\dfrac{\sqrt{169}}{\sqrt{2500}}=\dfrac{13}{50}$.
			\end{enumerate}
		\end{multicols}
	}
\end{vd}

\begin{vd}%[Dự án EX-9-Đề Cương Toán 9]%[Dat Tien Pham]%[9D3N4-1]
	Trục căn thức ở mẫu các biểu thức sau
	\begin{multicols}{3}
		\begin{enumerate}
			\item $\dfrac{\sqrt{10}+\sqrt{5}}{\left(\sqrt{2}+1\right)}$;
			\item $\dfrac{11}{3\sqrt{3}}$;
			\item $\dfrac{20}{\sqrt{7}-\sqrt{3}}$.
		\end{enumerate}
	\end{multicols}
	\loigiai{
		\begin{enumerate}
			\item $\dfrac{\sqrt{10}+\sqrt{5}}{\sqrt{2}+1}
			=\dfrac{\sqrt{5}(\sqrt{2}+1)}{\sqrt{2}+1}
			=\sqrt{5}.$
			\item $\dfrac{11}{3\sqrt{3}}
			=\dfrac{11\sqrt{3}}{\left(3(\sqrt{3})^2\right)}
			=\dfrac{11\sqrt{3}}{3\cdot 3}
			=\dfrac{11\sqrt{3}}{9}.$
			\item $\dfrac{20}{\sqrt{7}-\sqrt{3}}
			=\dfrac{20\left(\sqrt{7}+\sqrt{3}\right)}{\left(\sqrt{7}-\sqrt{3}\right)\left(\sqrt{7}+\sqrt{3}\right)}
			=\dfrac{20\left(\sqrt{7}+\sqrt{3}\right)}{7-3}
			=5\left(\sqrt{7}+\sqrt{3}\right)=5\sqrt{7}+5\sqrt{3}.$
		\end{enumerate}
	}
\end{vd}


\subsubsection{Bài tập}
\begin{bt}%[Dự án EX-9-Đề Cương Toán 9]%[Dat Tien Pham]%[9D3N3-3]
	Tính
	\begin{multicols}{3}
		\begin{enumerate}
			\item $\sqrt{\dfrac{9}{16}}$;
			\item $\sqrt{75} : \sqrt{3}$;
			\item $6\sqrt{2} : \sqrt{18}$;
			\item $\sqrt{2 \dfrac{28}{36}}$;
			\item $\sqrt{4 \dfrac{29}{49}}$;
			\item $\left(-16 \sqrt{18}\right) : \left(-4 \sqrt{2}\right)$;
			\item $24 \sqrt{12} : 6 \sqrt{3}$;
			\item $3 \sqrt{72} : 3 \sqrt{2}$;
			\item $-4 \sqrt{20} : \sqrt{5}$;
			\item $\left(14 \sqrt{8}\right) : \left(7 \sqrt{2}\right)$;
			\item $6 \sqrt{5} : \left(-\sqrt{20}\right)$;
			\item $10 \sqrt{6} : \sqrt{24}$;
			\item $5 \sqrt{27} : \left(-5 \sqrt{3}\right)$;
			\item $9 \sqrt{3} : \sqrt{27}$;
			\item $\sqrt{\dfrac{2}{3}} : \sqrt{\dfrac{3}{8}}$.	
		\end{enumerate}
	\end{multicols}
	\loigiai{
		\begin{enumerate}
			\item $ \sqrt{\dfrac{9}{16}} = \dfrac{3}{4}.$
			\item $ \sqrt{75} : \sqrt{3}
			= \sqrt{\dfrac{75}{3}}
			= \sqrt{25}
			= 5.$
			\item $ 6\sqrt{2} : \sqrt{18}
			= \dfrac{6\sqrt{2}}{\sqrt{18}}
			= \sqrt{\dfrac{36 \cdot 2}{18}}
			= \sqrt{4}
			= 2.$
			\item $ \sqrt{2 \dfrac{28}{36}}
			= \sqrt{\dfrac{2 \cdot 36 + 28}{36}}
			= \sqrt{\dfrac{72 + 28}{36}}
			= \sqrt{\dfrac{100}{36}}
			= \dfrac{10}{6}
			= \dfrac{5}{3}.$
			\item $ \sqrt{4 \dfrac{29}{49}}
			= \sqrt{\dfrac{4 \cdot 49 + 29}{49}}
			= \sqrt{\dfrac{196 + 29}{49}}
			= \sqrt{\dfrac{225}{49}}
			= \dfrac{15}{7}.$
			\item $ \left(-16 \sqrt{18}\right) : \left(-4 \sqrt{2}\right)
			= \dfrac{-16}{-4} \cdot \sqrt{\dfrac{18}{2}}
			= 4 \cdot \sqrt{9}
			= 4 \cdot 3
			= 12.$
			\item $ 24 \sqrt{12} : 6 \sqrt{3}
			= \dfrac{24}{6} \cdot \sqrt{\dfrac{12}{3}}
			= 4 \cdot \sqrt{4}
			= 4 \cdot 2
			= 8.$
			\item $ 3 \sqrt{72} : 3 \sqrt{2}
			= \dfrac{3}{3} \cdot \sqrt{\dfrac{72}{2}}
			= \sqrt{36}
			= 6.$
			\item $ -4 \sqrt{20} : \sqrt{5}
			= -4 \cdot \sqrt{\dfrac{20}{5}}
			= -4 \cdot \sqrt{4}
			= -4 \cdot 2
			= -8.$
			\item $ \left(14 \sqrt{8}\right) : \left(7 \sqrt{2}\right)
			= \dfrac{14}{7} \cdot \sqrt{\dfrac{8}{2}}
			= 2 \cdot \sqrt{4}
			= 2 \cdot 2
			= 4.$
			\item $ 6 \sqrt{5} : \left(-\sqrt{20}\right)
			= 6 \cdot \dfrac{\sqrt{5}}{-\sqrt{20}}
			= 6 \cdot \left(-\sqrt{\dfrac{5}{20}}\right)
			= -6 \cdot \sqrt{\dfrac{1}{4}}
			= -6 \cdot \dfrac{1}{2}
			= -3.$
			\item $ 10 \sqrt{6} : \sqrt{24}
			= 10 \cdot \sqrt{\dfrac{6}{24}}
			= 10 \cdot \sqrt{\dfrac{1}{4}}
			= 10 \cdot \dfrac{1}{2}
			= 5.$
			\item $ 5 \sqrt{27} : \left(-5 \sqrt{3}\right)
			= \dfrac{5}{-5} \cdot \sqrt{\dfrac{27}{3}}
			= -1 \cdot \sqrt{9}
			= -3.$
			\item $ 9 \sqrt{3} : \sqrt{27}
			= 9 \cdot \sqrt{\dfrac{3}{27}}
			= 9 \cdot \sqrt{\dfrac{1}{9}}
			= 9 \cdot \dfrac{1}{3}
			= 3.$
			\item $ \sqrt{\dfrac{2}{3}} : \sqrt{\dfrac{3}{8}}
			= \sqrt{\dfrac{2}{3}:\dfrac{3}{8}}
			= \sqrt{\dfrac{2}{3} \cdot \dfrac{8}{3}}
			= \sqrt{\dfrac{16}{9}}
			= \dfrac{4}{3}.$
		\end{enumerate}
	}
\end{bt}
\begin{bt}%[Dự án EX-9-Đề Cương Toán 9]%[Dat Tien Pham]%[9D3N4-1]
	Trục căn thức ở mẫu của các phân thức sau
	\begin{multicols}{4}
		\begin{enumerate}
			\item $\dfrac{1}{\sqrt{2}}$;
			\item $\dfrac{1}{\sqrt{3}}$;
			\item $\dfrac{1}{\sqrt{5}}$;
			\item $\dfrac{1}{\sqrt{7}}$;
			\item $\dfrac{\sqrt{11}}{\sqrt{3}}$;
			\item $\dfrac{3}{2 \sqrt{5}}$;
			\item $-\dfrac{20}{3 \sqrt{5}}$;
			\item $\dfrac{12}{5 \sqrt{3}}$;
			\item $\dfrac{5}{\sqrt{10}}$;
			\item $\dfrac{14}{\sqrt{7}}$;
			\item $\dfrac{\sqrt{3}}{\sqrt{2}}$;
			\item $\dfrac{4 \sqrt{2}-\sqrt{3}}{2 \sqrt{3}}$.
		\end{enumerate}
	\end{multicols}
	\loigiai{
		\begin{multicols}{3}
			\begin{enumerate}
				\item $\dfrac{1}{\sqrt{2}}=\dfrac{\sqrt{2}}{2}$.
				\item $\dfrac{1}{\sqrt{3}}=\dfrac{\sqrt{3}}{3}$.
				\item $\dfrac{1}{\sqrt{5}}=\dfrac{\sqrt{5}}{5}$.
				\item $\dfrac{1}{\sqrt{7}}=\dfrac{\sqrt{7}}{7}$.
				\item $\dfrac{\sqrt{11}}{\sqrt{3}}=\dfrac{\sqrt{33}}{3}$.
				\item $\dfrac{3}{2 \sqrt{5}}=\dfrac{3\sqrt{5}}{10}$.
				\item $-\dfrac{20}{3 \sqrt{5}}=-\dfrac{4\sqrt{5}}{3}$.
				\item $\dfrac{12}{5 \sqrt{3}}=\dfrac{4\sqrt{3}}{5}$.
				\item $\dfrac{5}{\sqrt{10}}=\dfrac{\sqrt{10}}{2}$.
				\item $\dfrac{14}{\sqrt{7}}=2\sqrt{7}$.
				\item $\dfrac{\sqrt{3}}{\sqrt{2}}=\dfrac{\sqrt{6}}{2}$.
				\item $\dfrac{4 \sqrt{2}-\sqrt{3}}{2 \sqrt{3}}=\dfrac{4\sqrt{6}-3}{6}$.
			\end{enumerate}
		\end{multicols}
	}
\end{bt}

\begin{bt}%[Dự án EX-9-Đề Cương Toán 9]%[Dat Tien Pham]%[9D3N4-1]
	Trục căn thức ở mẫu của các phân thức sau
	\begin{multicols}{3}
		\begin{enumerate}
			\item $\dfrac{3}{\sqrt{3}}$;
			\item $\dfrac{2 \sqrt{3}-\sqrt{15}}{\sqrt{3}}$;
			\item $\dfrac{2 \sqrt{2}+2}{5 \sqrt{2}}$;
			\item $\dfrac{4 \sqrt{5}+\sqrt{15}}{\sqrt{5}}$;
			\item $\dfrac{7-\sqrt{7}}{3 \sqrt{7}}$;
			\item $\dfrac{\sqrt{6}-6}{1-\sqrt{6}}$;
			\item $\dfrac{7+\sqrt{7}}{\sqrt{7}+1}$;
			\item $\dfrac{3-2 \sqrt{3}}{\sqrt{3}-2}$;
			\item $\dfrac{2 \sqrt{6}+6 \sqrt{7}}{3 \sqrt{3}}$
			\item $\dfrac{6 \sqrt{2}-4}{\sqrt{2}-3}$;
			\item $\dfrac{\sqrt{6}-\sqrt{2}}{3 \sqrt{3}-3}$;
			\item $\dfrac{5 \sqrt{2}-2 \sqrt{5}}{\sqrt{5}-\sqrt{2}}$;
			\item $\dfrac{\sqrt{14}-\sqrt{7}}{1-\sqrt{2}}$;
			\item $\dfrac{\sqrt{15}-\sqrt{5}}{1-\sqrt{3}}$;
			\item $\dfrac{3 \sqrt{2}-6}{\sqrt{2}-1}$;
			\item $\dfrac{\sqrt{15}-\sqrt{12}}{\sqrt{5}-2}$;
			\item $\dfrac{5-2 \sqrt{5}}{2-\sqrt{5}}$;
			\item $\dfrac{5+3 \sqrt{5}}{3+\sqrt{5}}$;
			\item $\dfrac{3 \sqrt{2}-2 \sqrt{3}}{\sqrt{3}-\sqrt{2}}$;
			\item $\dfrac{5 \sqrt{6}+6 \sqrt{5}}{\sqrt{5}+\sqrt{6}}$;
			\item $\dfrac{6 \sqrt{2}+3}{1+2 \sqrt{2}}$;
			\item $\dfrac{6 \sqrt{6}-27}{2 \sqrt{2}-3 \sqrt{3}}$;
			\item $\dfrac{18 \sqrt{14}-60}{2\left(3 \sqrt{7}-5 \sqrt{2}\right)}$;
			\item $\dfrac{12 \sqrt{10}-16 \sqrt{14}}{6 \sqrt{5}-8 \sqrt{7}}$.
		\end{enumerate}
	\end{multicols}
	\loigiai{
		\begin{multicols}{2}
			\begin{enumerate}
				\item $\dfrac{3}{\sqrt{3}}=\sqrt{3}$.
				\item $\dfrac{2 \sqrt{3}-\sqrt{15}}{\sqrt{3}}=\dfrac{\sqrt{3}\left(2-\sqrt{5}\right)}{\sqrt{3}}=2-\sqrt{5}$.
				\item $\dfrac{2 \sqrt{2}+2}{5 \sqrt{2}}=\dfrac{4+2\sqrt{2}}{10}=\dfrac{2+\sqrt{2}}{5}$.
				\item $\dfrac{4 \sqrt{5}+\sqrt{15}}{\sqrt{5}}=\dfrac{\sqrt{5}\left(4+\sqrt{3}\right)}{\sqrt{5}}=4+\sqrt{3}$.
				\item $\dfrac{7-\sqrt{7}}{3 \sqrt{7}}=\dfrac{\sqrt{7}\left(\sqrt{7}-1\right)}{3\sqrt{7}}=\dfrac{\sqrt{7}-1}{3}$.
				\item $\dfrac{\sqrt{6}-6}{1-\sqrt{6}}=\dfrac{\sqrt{6}\left(1-\sqrt{6}\right)}{1-\sqrt{6}}=\sqrt{6}$.
				\item $\dfrac{7+\sqrt{7}}{\sqrt{7}+1}=\dfrac{\sqrt{7}\left(\sqrt{7}+1\right)}{\sqrt{7}+1}=\sqrt{7}$.
				\item $\dfrac{3-2 \sqrt{3}}{\sqrt{3}-2}=\dfrac{\sqrt{3}\left(\sqrt{3}-2\right)}{\sqrt{3}-2}=\sqrt{3}$.
				\item $\dfrac{2 \sqrt{6}+6 \sqrt{7}}{3 \sqrt{3}}=\dfrac{2\sqrt{2}+\sqrt{21}}{3}$.
				\item $\dfrac{6 \sqrt{2}-4}{\sqrt{2}-3}=\dfrac{2\sqrt{2}\left(3-\sqrt{2}\right)}{\sqrt{2}-3}=-2\sqrt{2}$.
				\item $\dfrac{\sqrt{6}-\sqrt{2}}{3 \sqrt{3}-3}=\dfrac{\sqrt{2}\left(\sqrt{3}-1\right)}{3\left(\sqrt{3}-1\right)}=\dfrac{\sqrt{2}}{3}$.
				\item $\dfrac{5 \sqrt{2}-2 \sqrt{5}}{\sqrt{5}-\sqrt{2}}=\dfrac{\sqrt{10}\left(\sqrt{5}-\sqrt{2}\right)}{\sqrt{5}-\sqrt{2}}=\sqrt{10}$.
				\item $\dfrac{\sqrt{14}-\sqrt{7}}{1-\sqrt{2}}=\dfrac{\sqrt{7}\left(\sqrt{2}-1\right)}{1-\sqrt{2}}=-\sqrt{7}$.
				\item $\dfrac{\sqrt{15}-\sqrt{5}}{1-\sqrt{3}}=\dfrac{\sqrt{5}\left(\sqrt{3}-1\right)}{1-\sqrt{3}}=-\sqrt{5}$.
				\item $\dfrac{3 \sqrt{2}-6}{\sqrt{2}-1}=\dfrac{3\sqrt{2}\left(1-\sqrt{2}\right)}{\sqrt{2}-1}=-3\sqrt{2}$.
				\item $\dfrac{\sqrt{15}-\sqrt{12}}{\sqrt{5}-2}=\dfrac{\sqrt{3}\left(\sqrt{5}-2\right)}{\sqrt{5}-2}=\sqrt{3}$.
				\item $\dfrac{5-2 \sqrt{5}}{2-\sqrt{5}}=\dfrac{\sqrt{5}\left(\sqrt{5}-2\right)}{2-\sqrt{5}}=-\sqrt{5}$.
				\item $\dfrac{5+3 \sqrt{5}}{3+\sqrt{5}}=\dfrac{\sqrt{5}\left(\sqrt{5}+3\right)}{3+\sqrt{5}}=\sqrt{5}$.
				\item $\dfrac{3 \sqrt{2}-2 \sqrt{3}}{\sqrt{3}-\sqrt{2}}=\dfrac{\sqrt{6}\left(\sqrt{3}-\sqrt{2}\right)}{\sqrt{3}-\sqrt{2}}=\sqrt{6}$.
				\item $\dfrac{5 \sqrt{6}+6 \sqrt{5}}{\sqrt{5}+\sqrt{6}}=\dfrac{\sqrt{30}\left(\sqrt{5}+\sqrt{6}\right)}{\sqrt{5}+\sqrt{6}}=\sqrt{30}$.
				\item $\dfrac{6 \sqrt{2}+3}{1+2 \sqrt{2}}=\dfrac{3\left(2\sqrt{2}+1\right)}{1+2\sqrt{2}}=3$.
				\item $\dfrac{6 \sqrt{6}-27}{2 \sqrt{2}-3 \sqrt{3}}=\dfrac{3\sqrt{3}\left(2\sqrt{2}-3\sqrt{3}\right)}{2\sqrt{2}-3\sqrt{3}}=3\sqrt{3}$.
				\item $\dfrac{18 \sqrt{14}-60}{2\left(3 \sqrt{7}-5 \sqrt{2}\right)}=\dfrac{6\sqrt{2}\left(3\sqrt{7}-5\sqrt{2}\right)}{2\left(3 \sqrt{7}-5 \sqrt{2}\right)}=3\sqrt{2}$.
				\item $\dfrac{12 \sqrt{10}-16 \sqrt{14}}{6 \sqrt{5}-8 \sqrt{7}}=\dfrac{2\sqrt{2}\left(6\sqrt{5}-8\sqrt{7}\right)}{6 \sqrt{5}-8 \sqrt{7}}=2\sqrt{2}$.
			\end{enumerate}
		\end{multicols}
	}
\end{bt}



\begin{bt}%[Dự án EX-9-Đề Cương Toán 9]%[Dat Tien Pham]%[9D3N4-1]
	Trục căn thức ở mẫu của các phân thức sau
	\begin{multicols}{4}
		\begin{enumerate}
			\item $\dfrac{1}{\sqrt{3}+\sqrt{2}}$.
			\item $\dfrac{1}{\sqrt{5}+\sqrt{7}}$.
			\item $\dfrac{1}{5-2 \sqrt{6}}$;
			\item $\dfrac{1}{\sqrt{3}+1}$;
			\item $\dfrac{1}{2-\sqrt{6}}$;
			\item $\dfrac{1}{2+\sqrt{6}}$;
			\item $\dfrac{1}{\sqrt{5}-1}$;
			\item $\dfrac{1}{\sqrt{3}-\sqrt{2}}$;
			\item $\dfrac{\sqrt{3}}{\sqrt{5}+\sqrt{2}}$;
			\item $\dfrac{27}{\sqrt{6}-3}$;
			\item $\dfrac{-16}{\sqrt{7}+\sqrt{3}}$;
			\item $\dfrac{\sqrt{5}+3}{\sqrt{5}-3}$.
		\end{enumerate}
	\end{multicols}
	\loigiai{
		\begin{multicols}{2}
			\begin{enumerate}
				\item $\dfrac{1}{\sqrt{3}+\sqrt{2}}=\dfrac{\sqrt{3}-\sqrt{2}}{3-2}=\sqrt{3}-\sqrt{2}$.
				\item $\dfrac{1}{\sqrt{5}+\sqrt{7}}=\dfrac{\sqrt{5}-\sqrt{7}}{5-7}=-\dfrac{\sqrt{5}-\sqrt{7}}{2}$.
				\item $\dfrac{1}{5-2 \sqrt{6}}=\dfrac{5+2\sqrt{6}}{25-24}=5+2\sqrt{6}$.
				\item $\dfrac{1}{\sqrt{3}+1}=\dfrac{\sqrt{3}-1}{3-1}=\dfrac{\sqrt{3}-1}{2}$.
				\item $\dfrac{1}{2-\sqrt{6}}=\dfrac{2+\sqrt{6}}{4-6}=-\dfrac{2+\sqrt{6}}{2}$.
				\item $\dfrac{1}{2+\sqrt{6}}=\dfrac{2-\sqrt{6}}{4-6}=-\dfrac{2-\sqrt{6}}{2}$.
				\item $\dfrac{1}{\sqrt{5}-1}=\dfrac{\sqrt{5}+1}{5-1}=\dfrac{\sqrt{5}+1}{4}$.
				\item $\dfrac{1}{\sqrt{3}-\sqrt{2}}=\dfrac{\sqrt{3}+\sqrt{2}}{3-2}=\sqrt{3}+\sqrt{2}$.
				\item $\dfrac{\sqrt{3}}{\sqrt{5}+\sqrt{2}}=\dfrac{\sqrt{3}\left(\sqrt{5}-\sqrt{2}\right)}{5-2}=\dfrac{\sqrt{15}-\sqrt{6}}{3}$.
				\item $\dfrac{27}{\sqrt{6}-3}=\dfrac{27\left(\sqrt{6}+3\right)}{6-9}=-9\left(\sqrt{6}+3\right)$.
				\item $\dfrac{-16}{\sqrt{7}+\sqrt{3}}=\dfrac{-16\left(\sqrt{7}-\sqrt{3}\right)}{7-3}=-4\left(\sqrt{7}-\sqrt{3}\right)$.
				\item $\dfrac{\sqrt{5}+3}{\sqrt{5}-3}=\dfrac{\left(\sqrt{5}+3\right)^2}{5-9}=-\dfrac{7+3\sqrt{5}}{2}$.
			\end{enumerate}
		\end{multicols}
	}
\end{bt}
\begin{bt}%[Dự án EX-9-Đề Cương Toán 9]%[Dat Tien Pham]%[9D3V4-1]
	Tính
	\begin{multicols}{2}
	\begin{enumerate}
		\item  $\dfrac{\sqrt{3-2 \sqrt{2}}}{\sqrt{17-12 \sqrt{2}}}-\dfrac{\sqrt{3+2 \sqrt{2}}}{\sqrt{17+12 \sqrt{2}}}$;
		\item  $\dfrac{2}{\sqrt{3}}+\dfrac{\sqrt{2}}{3}+\dfrac{2}{\sqrt{3}} \sqrt{\dfrac{5}{12}-\dfrac{1}{\sqrt{6}}}$;
		\item  $\dfrac{1}{\sqrt{8}+\sqrt{7}}+\sqrt{175}-\dfrac{6 \sqrt{2}-4}{3-\sqrt{2}}$;
		\item  $\dfrac{2\sqrt{6-\sqrt{11}}}{\sqrt{22}-\sqrt{2}}+\dfrac{6}{\sqrt{2}}-\dfrac{3}{\sqrt{2}+1}$;
		\item  $\sqrt{\sqrt{7+\sqrt{48}}}-\dfrac{1}{\sqrt{2}}$;
		\item  $\dfrac{2+\sqrt{3}}{\sqrt{2}+\sqrt{2+\sqrt{3}}}+\dfrac{2-\sqrt{3}}{\sqrt{2}-\sqrt{2-\sqrt{3}}}$;
		\item  $\dfrac{\sqrt{2}}{2 \sqrt{2}+\sqrt{3+\sqrt{5}}}+\dfrac{\sqrt{2}}{2 \sqrt{2}-\sqrt{3-\sqrt{5}}}$;
		\item  $\dfrac{\sqrt{2}+\sqrt{3}+\sqrt{6}+\sqrt{8}+4}{\sqrt{2}+\sqrt{3}+\sqrt{4}}$
		\item  $\dfrac{(5+2 \sqrt{6})(49-20 \sqrt{6}) \sqrt{5-2 \sqrt{6}}}{9 \sqrt{3}-11 \sqrt{2}}$;
		\item  $\dfrac{\dfrac{\sqrt{2+\sqrt{3}}}{2}}{\dfrac{\sqrt{2+\sqrt{3}}}{2}-\dfrac{2}{\sqrt{6}}+\dfrac{\sqrt{2+\sqrt{3}}}{2 \sqrt{3}}}$;
		\item  $\dfrac{1+\dfrac{\sqrt{3}}{2}}{1+\sqrt{1+\dfrac{\sqrt{3}}{2}}}+\dfrac{1-\dfrac{\sqrt{3}}{2}}{1-\sqrt{1-\dfrac{\sqrt{3}}{2}}}$;
		\item  $\dfrac{1}{\sqrt{1}+\sqrt{2}}+\dfrac{1}{\sqrt{2}+\sqrt{3}}+\cdots+\dfrac{1}{\sqrt{24}+\sqrt{25}}$.
	\end{enumerate}	
	\end{multicols}
	\loigiai{
		\begin{enumerate}
			\item Ta có {\allowdisplaybreaks\begin{eqnarray*}
					&& \dfrac{\sqrt{3-2 \sqrt{2}}}{\sqrt{17-12 \sqrt{2}}}-\dfrac{\sqrt{3+2 \sqrt{2}}}{\sqrt{17+12 \sqrt{2}}}\\
					&=& \dfrac{\sqrt{2}-1}{3-2\sqrt{2}}-\dfrac{\sqrt{2}+1}{3+2\sqrt{2}}\\
					&=& \dfrac{1}{\sqrt{2}-1}-\dfrac{1}{\sqrt{2}+1}\\
					&=& 2.
			\end{eqnarray*}}
			\item Ta có  {\allowdisplaybreaks\begin{eqnarray*}
					&& \dfrac{2}{\sqrt{3}}+\dfrac{\sqrt{2}}{3}+\dfrac{2}{\sqrt{3}} \sqrt{\dfrac{5}{12}-\dfrac{1}{\sqrt{6}}}\\
					&=& \dfrac{\sqrt{2}}{\sqrt{3}}\left(\sqrt{2}+\dfrac{1}{\sqrt{3}}+\sqrt{2}\sqrt{\left(\dfrac{1}{2}-\dfrac{1}{\sqrt{6}}\right)^2}\right)\\
					&=& \dfrac{\sqrt{2}}{\sqrt{3}}\left(\sqrt{2}+\dfrac{1}{\sqrt{3}}+\dfrac{\sqrt{2}}{2}-\dfrac{\sqrt{2}}{\sqrt{6}}\right)\\
					&=& \dfrac{\sqrt{2}}{\sqrt{3}}\cdot \dfrac{3}{\sqrt{2}}\\
					&=& \sqrt{3}.
			\end{eqnarray*}}
			\item Ta có  {\allowdisplaybreaks\begin{eqnarray*}
					&& \dfrac{1}{\sqrt{8}+\sqrt{7}}+\sqrt{175}-\dfrac{6 \sqrt{2}-4}{3-\sqrt{2}}\\
					&=& \sqrt{8}-\sqrt{7}+5\sqrt{7}-2\sqrt{2}\\
					&=& 4\sqrt{7}.
			\end{eqnarray*}}
			\item Ta có  {\allowdisplaybreaks\begin{eqnarray*}
					&& \dfrac{2\sqrt{6-\sqrt{11}}}{\sqrt{22}-\sqrt{2}}+\dfrac{6}{\sqrt{2}}-\dfrac{3}{\sqrt{2}+1}\\
					&=& \dfrac{\sqrt{2}\sqrt{6-\sqrt{11}}}{\sqrt{11}-1}+3\sqrt{2}-3\sqrt{2}+3\\
					&=& \dfrac{\sqrt{12-2\sqrt{11}}}{\sqrt{11}-1}+3\\
					&=& \dfrac{\sqrt{11}-1}{\sqrt{11}-1}+3\\
					&=& 4.
			\end{eqnarray*}}
			\item Ta có  {\allowdisplaybreaks\begin{eqnarray*}
					&& \sqrt{\sqrt{7+\sqrt{48}}}-\dfrac{1}{\sqrt{2}}\\
					&=& \sqrt{\sqrt{7+4\sqrt{3}}}-\dfrac{1}{\sqrt{2}}\\
					&=& \dfrac{\sqrt{2}\sqrt{2+\sqrt{3}}-1}{\sqrt{2}}\\
					&=& \dfrac{\sqrt{4+2\sqrt{3}}-1}{\sqrt{2}}\\
					&=&\dfrac{\sqrt{3}+1-1}{\sqrt{2}}=\dfrac{\sqrt{6}}{2}.
			\end{eqnarray*}}
			\item Ta có  {\allowdisplaybreaks\begin{eqnarray*}
					&& \dfrac{2+\sqrt{3}}{\sqrt{2}+\sqrt{2+\sqrt{3}}}+\dfrac{2-\sqrt{3}}{\sqrt{2}-\sqrt{2-\sqrt{3}}}\\
					&=& \sqrt{2}\left(\dfrac{2+\sqrt{3}}{2+\sqrt{4+2\sqrt{3}}}+\dfrac{2-\sqrt{3}}{2-\sqrt{4-2\sqrt{3}}}\right)\\
					&=& \sqrt{2}\left(\dfrac{2+\sqrt{3}}{2+\sqrt{3}+1}+\dfrac{2-\sqrt{3}}{2-\sqrt{3}+1}\right)\\
					&=&\sqrt{2}\left(\dfrac{3+\sqrt{3}-1}{3+\sqrt{3}}+\dfrac{3-\sqrt{3}-1}{3-\sqrt{3}}\right)\\
					&=&\sqrt{2}\left(1-\dfrac{1}{3+\sqrt{3}}+1-\dfrac{1}{3-\sqrt{3}}\right)\\
					&=&\sqrt{2}\left(2-1\right)=\sqrt{2}.
			\end{eqnarray*}}
			\item Ta có  {\allowdisplaybreaks\begin{eqnarray*}
					&& \dfrac{\sqrt{2}}{2 \sqrt{2}+\sqrt{3+\sqrt{5}}}+\dfrac{\sqrt{2}}{2 \sqrt{2}-\sqrt{3-\sqrt{5}}}\\
					&=& \dfrac{2}{4+\sqrt{6+2\sqrt{5}}}+\dfrac{2}{4-\sqrt{6-2\sqrt{5}}}\\
					&=& \dfrac{2}{4+\sqrt{5}+1}+\dfrac{2}{4-\sqrt{5}+1}\\
					&=& \dfrac{2\left(5-\sqrt{5}\right)+2\left(5+\sqrt{5}\right)}{20}\\
					&=&1.
			\end{eqnarray*}}
			\item Ta có  {\allowdisplaybreaks\begin{eqnarray*}
					&& \dfrac{\sqrt{2}+\sqrt{3}+\sqrt{6}+\sqrt{8}+4}{\sqrt{2}+\sqrt{3}+\sqrt{4}}\\
					&=& \dfrac{\sqrt{2}+\sqrt{3}+2+2+\sqrt{6}+\sqrt{8}}{\sqrt{2}+\sqrt{3}+\sqrt{4}}\\
					&=& \dfrac{\left(\sqrt{2}+\sqrt{3}+\sqrt{4}\right)\left(1+\sqrt{2}\right)}{\sqrt{2}+\sqrt{3}+\sqrt{4}}\\
					&=&1+\sqrt{2}.
			\end{eqnarray*}}
			\item Ta có  {\allowdisplaybreaks\begin{eqnarray*}
					&& \dfrac{(5+2 \sqrt{6})(49-20 \sqrt{6}) \sqrt{5-2 \sqrt{6}}}{9 \sqrt{3}-11 \sqrt{2}}\\
					&=& \dfrac{5+2\sqrt{6}\left(5-2\sqrt{6}\right)^2\sqrt{5-2\sqrt{6}}}{9\sqrt{3}-11\sqrt{2}}\\
					&=& \dfrac{\left(5-2\sqrt{6}\right)\sqrt{5-2\sqrt{6}}}{9\sqrt{3}-11\sqrt{2}}\\
					&=& \dfrac{\left(\sqrt{3}-\sqrt{2}\right)^3}{9\sqrt{3}-11\sqrt{2}}\\
					&=&\dfrac{3\sqrt{3}-9\sqrt{2}+6\sqrt{3}-2\sqrt{2}}{9\sqrt{3}-11\sqrt{2}}\\
					&=&1.	
			\end{eqnarray*}}
			\item Ta có  {\allowdisplaybreaks\begin{eqnarray*}
					&& \dfrac{\dfrac{\sqrt{2+\sqrt{3}}}{2}}{\dfrac{\sqrt{2+\sqrt{3}}}{2}-\dfrac{2}{\sqrt{6}}+\dfrac{\sqrt{2+\sqrt{3}}}{2 \sqrt{3}}}\\
					&=& \dfrac{\dfrac{\sqrt{4+2\sqrt{3}}}{2}}{\dfrac{\sqrt{4+2\sqrt{3}}}{2}-\dfrac{2}{\sqrt{3}}+\dfrac{\sqrt{4+2\sqrt{3}}}{2 \sqrt{3}}}\\
					&=& \dfrac{\dfrac{\sqrt{3}+1}{2}}{\dfrac{\sqrt{3}+1}{2}-\dfrac{2}{\sqrt{3}}+\dfrac{\sqrt{3}+1}{2 \sqrt{3}}}\\	
					&=& \dfrac{\dfrac{\sqrt{3}+1}{2}}{\dfrac{3+\sqrt{3}-4+\sqrt{3}+1}{2\sqrt{3}}}\\
					&=& \dfrac{\sqrt{3}+1}{2}.
			\end{eqnarray*}}
			\item Ta có  {\allowdisplaybreaks\begin{eqnarray*}
					&& \dfrac{1+\dfrac{\sqrt{3}}{2}}{1+\sqrt{1+\dfrac{\sqrt{3}}{2}}}+\dfrac{1-\dfrac{\sqrt{3}}{2}}{1-\sqrt{1-\dfrac{\sqrt{3}}{2}}}\\
					&=& \dfrac{2+\sqrt{3}}{2+\sqrt{4+2\sqrt{3}}}+\dfrac{2-\sqrt{3}}{2-\sqrt{4-2\sqrt{3}}}\\
					&=& \dfrac{2+\sqrt{3}}{2+\sqrt{3}+1}+\dfrac{2-\sqrt{3}}{2-\sqrt{3}+1}\\
					&=& 2-\dfrac{1}{3+\sqrt{3}}-\dfrac{1}{3-\sqrt{3}}\\
					&=& 2-1=1. 
			\end{eqnarray*}}
			\item Ta có  {\allowdisplaybreaks\begin{eqnarray*}
					&& \dfrac{1}{\sqrt{1}+\sqrt{2}}+\dfrac{1}{\sqrt{2}+\sqrt{3}}+\cdots+\dfrac{1}{\sqrt{24}+\sqrt{25}}\\
					&=& \dfrac{\sqrt{2}-\sqrt{1}}{2-1}+\dfrac{\sqrt{3}-\sqrt{2}}{3-2}+ \cdots \dfrac{\sqrt{25}-\sqrt{24}}{25-24}\\
					&=& -\sqrt{1}+\sqrt{25}=4.
			\end{eqnarray*}} 
		\end{enumerate}
	}
\end{bt}

\begin{bt}%[Dự án EX-9-Đề Cương Toán 9]%[Dat Tien Pham]%[9D3V3-2]%[9D3V3-3]
	Rút gọn các biểu thức sau với các biểu thức đã cho có nghĩa
	\begin{multicols}{3}
		\begin{enumerate}
			\item  $\dfrac{x-\sqrt{x}}{\sqrt{x}-1}$;
			\item $\dfrac{x \sqrt{x}-2 x}{2-\sqrt{x}}$;
			\item $\dfrac{x \sqrt{y}-y \sqrt{x}}{\sqrt{x}-\sqrt{y}}$;
			\item $\dfrac{a \sqrt{b}-\sqrt{a}}{\sqrt{b}-b \sqrt{a}}$;
			\item $\dfrac{a-1}{\sqrt{a}+1}$;
			\item $\dfrac{4-x}{2 \sqrt{x}-x}$.
		\end{enumerate}
	\end{multicols}
	\loigiai {
		\begin{multicols}{2}
			\begin{enumerate}
				\item $ \dfrac{x-\sqrt{x}}{\sqrt{x}-1} =\dfrac{\sqrt{x}\left(\sqrt{x}-1\right)}{\sqrt{x}-1}=\sqrt{x}$.
				\item  $\dfrac{x \sqrt{x}-2 x}{2-\sqrt{x}} =\dfrac{x\left(\sqrt{x}-2\right)}{2-\sqrt{x}}=-x$.
				\item  $\dfrac{\sqrt{xy}\left(\sqrt{x}-\sqrt{y}\right)}{\sqrt{x}-\sqrt{y}}=\sqrt{xy}$. 
				\item  $ \dfrac{\sqrt{a}\left(\sqrt{ab}-1\right)}{\sqrt{b}\left(1-\sqrt{ab}\right)}=-\dfrac{\sqrt{a}}{\sqrt{b}}$.
				\item  $\dfrac{\left(\sqrt{a}-1\right)\left(\sqrt{a}+1\right)}{\sqrt{a}+1}=\sqrt{a}-1$.
				\item  $\dfrac{\left(2-\sqrt{x}\right)\left(2+\sqrt{x}\right)}{\sqrt{x}\left(2-\sqrt{x}\right)}=\dfrac{2+\sqrt{x}}{\sqrt{x}}$.
			\end{enumerate}
		\end{multicols}
	}
\end{bt}

\begin{bt}%[Dự án EX-9-Đề Cương Toán 9]%[Dat Tien Pham]%[9D3V3-2]%[9D3V3-3]
	Trục căn thức ở mẫu của các biểu thức sau
	\begin{multicols}{2}
		\begin{enumerate}
			\item $\dfrac{9-a}{\sqrt{a}+3}$;
			\item $\dfrac{1}{\sqrt{a}+\sqrt{a+1}}$;
			\item $\dfrac{x+y+2 \sqrt{xy}-1}{x \sqrt{y}+y \sqrt{x}+\sqrt{xy}}$;
			\item $\dfrac{x-9}{x \sqrt{x}+27}$.
		\end{enumerate}
	\end{multicols}
	\loigiai {
		\begin{enumerate}
			\item $\dfrac{9-a}{\sqrt{a}+3}=\dfrac{\left(3-\sqrt{a}\right)\left(3+\sqrt{a}\right)}{\sqrt{a}+3}=3-\sqrt{a}$.
			\item $\dfrac{1}{\sqrt{a}+\sqrt{a+1}}=\dfrac{\sqrt{a}-\sqrt{a+1}}{\left(\sqrt{a}+\sqrt{a+1}\right)\left(\sqrt{a}-\sqrt{a+1}\right)}=\dfrac{\sqrt{a}-\sqrt{a+1}}{a-a-1}=-\left(\sqrt{a}-\sqrt{a+1}\right)$.
			\item $\dfrac{x+y+2 \sqrt{xy}-1}{x \sqrt{y}+y \sqrt{x}+\sqrt{xy}}=\dfrac{\left(\sqrt{x}+\sqrt{y}\right)^2-1^2}{\sqrt{xy}\left(\sqrt{x}+\sqrt{y}+1\right)}=\dfrac{\left(\sqrt{x}+\sqrt{y}+1\right)\left(\sqrt{x}+\sqrt{y}-1\right)}{\sqrt{xy}\left(\sqrt{x}+\sqrt{y}+1\right)}=\dfrac{\sqrt{x}+\sqrt{y}-1}{\sqrt{xy}}$.
			\item $\dfrac{x-9}{x \sqrt{x}+27}$=
			$\dfrac{\left(\sqrt{x}-3\right)\left(\sqrt{x}+3\right)}{\left(\sqrt{x}+3\right)\left(x-3\sqrt{x}+9\right)}=\dfrac{\sqrt{x}-3}{x-3\sqrt{x}+9}$.
		\end{enumerate}
	}
\end{bt}

\begin{bt}%[Dự án EX-9-Đề Cương Toán 9]%[Dat Tien Pham]%[9D3V3-4]
	Biết rằng hai miếng bìa hình thang và hình chữ nhật ở hình bên dưới có diện tích bằng nhau. Tính chiều cao $ h $ của hình thang.
	\begin{center}
		\begin{tikzpicture}[>=stealth,line join=round,line cap=round]
			\path
			(0,0) coordinate (A)
			(4,0)coordinate(B)
			(3,2) coordinate (C)
			(1,2) coordinate (D)
			($(A)!(D)!(B)$)coordinate (H);
			\draw[black,scale=3] (A)--(B) node[pos=.5,below,sloped]{$\sqrt{36}$}--(C)--(D) node[pos=.5,above,sloped]{$\sqrt{18}$}--cycle;
			\draw[black,scale=3] (D)--(H) node[pos=.5, right]{$h$};
			\path   pic[angle radius=2mm,draw=blue,angle eccentricity=1.5] {right angle = B--H--D};
		\end{tikzpicture}
		\hspace{3cm}
		\begin{tikzpicture}[>=stealth,line join=round,line cap=round]
			\draw[black,scale=2] (0,0)coordinate(A)--(0:2)coordinate(B) node[pos=.5,below,sloped]{$\sqrt{27}$}--([turn]90:1)coordinate(C)--([turn]90:2)coordinate(D)--cycle node[pos=.5,left]{$\sqrt{18}$};			
		\end{tikzpicture}
	\end{center}
\loigiai{
		Diện tích hình chữ nhật là $\sqrt{27}\cdot \sqrt{18}=9\sqrt{6}$.\\
Vì hình thang và hình chữ nhật có diện tích bằng nhau, nên ta có
\allowdisplaybreaks
\begin{eqnarray*}
	\dfrac{\left(\sqrt{18}+\sqrt{36}\right)h}{2} &=& 9\sqrt{6}\\
	h &=& \dfrac{18\sqrt{6}}{\sqrt{18}+\sqrt{36}}\\
	h &=& \dfrac{18\sqrt{6}\left(3\sqrt{2}-6\right)}{18-36}\\
	h &=& 6\sqrt{6}-6\sqrt{3}.
\end{eqnarray*}
}

\end{bt}
