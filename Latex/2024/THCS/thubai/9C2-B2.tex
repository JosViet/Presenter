\section{BẤT PHƯƠNG TRÌNH BẬC NHẤT MỘT ẨN} % Tên bài
\subsection{Kiến thức cần nhớ}
\subsubsection{Bất phương trình bậc nhất một ẩn}
\begin{boxdl}
	Bất phương trình dạng $ax+b > 0$ (hoặc $ax+b < 0$, $ax+b \geqslant 0$, $ax+b \leqslant 0$), với $a$, $b$ là hai số đã cho và $a \neq 0$, được gọi là {\it bất phương trình bậc nhất một ẩn} (ẩn là $x$).
\end{boxdl}

\subsubsection{Nghiệm của bất phương trình bậc nhất một ẩn}
\begin{boxdl}
	Với bất phương nhất bậc nhất có ẩn là $x$, số $x_0$ được gọi là {\it nghiệm} của bất phương trình nếu ta thay $x=x_0$ thì nhận được một khẳng định đúng. \\
	\textit{Giải bất phương trình} là tìm tất cả các nghiệm của nó.
\end{boxdl}

\subsubsection{Giải bất phương trình bậc nhất một ẩn}
\begin{boxdl}
	Xét bất phương trình $ax+b>0$ ($a \neq 0$).
	\begin{itemize}
		\item Cộng hai vế của bất phương trình với $-b$, ta được bất phương trình
		$$ax > -b.$$
		\item Nhân hai vế của bất phương trình nhận được với $\dfrac{1}{a}$
		\begin{itemize}
			\item Nếu $a>0$ thì nhận được nghiệm của bất phương trình đã cho là $x > -\dfrac{b}{a}$.
			\item Nếu $a<0$ thì nhận được nghiệm của bất phương trình đã cho là $x<-\dfrac{b}{a}$.
		\end{itemize}
	\end{itemize}
\end{boxdl}
Với các bất phương trình dạng $ax+b<0$, $ax+b \geq 0$, $ax+b\leq 0$, ta thực hiện các bước giải tương tự.

\subsection{Bài tập}

\begin{dang}{Nhận diện bất phương trình bậc nhất một ẩn và nghiệm của bất phương trình bậc nhất một ẩn}
	\begin{itemize}
		\item 	Bất phương trình dạng $ax+b > 0$ (hoặc $ax+b < 0$, $ax+b \geqslant 0$, $ax+b \leqslant 0$), với $a$, $b$ là hai số đã cho và $a \neq 0$, được gọi là {\it bất phương trình bậc nhất một ẩn} (ẩn là $x$).
		\item 	Với bất phương trình bậc nhất có ẩn là $x$, số $x_0$ được gọi là {\it nghiệm} của bất phương trình nếu ta thay $x=x_0$ thì nhận được một khẳng định đúng.
	\end{itemize}
\end{dang}

%=====Ví dụ 1
\begin{vd}%[Dự án EX-9-Đề Cương Toán 9]%[Nguyễn Văn Cương 056]%[9D2N2-1]
	Trong các bất phương trình sau, bất phương trình nào là bất phương trình bậc nhất một ẩn?
	\begin{multicols}{4}
		\begin{itemize}
			\item $x+2\,023 > 0$;
			\item $0x - 5 <0$;
			\item $5x - 7 \leq 0$;
			\item $x^2 +1 \leq 0$.
		\end{itemize}
	\end{multicols}
	\loigiai{
		\begin{itemize}
			\item Bất phương trình $x+2\,023 > 0$ có dạng $ax+b>0$ với $a = 1 \neq 0$ và $b=2\,023$, nên nó là bất phương trình bậc nhất một ẩn.
			\item Bất phương trình $5x - 7 \leqslant 0$ có dạng $ax +b \leqslant 0$ với $a = 5$ và $b=-7$, nên nó là bất phương trình bậc nhất một ẩn.
			\item Hai bất phương trình $0x - 5 <0$ và $x^2 +1 \leqslant 0$ không phải là bất phương trình bậc nhất một ẩn.
		\end{itemize}
	}
\end{vd}

%=====Ví dụ 2
\begin{vd}%[Dự án EX-9-Đề Cương Toán 9]%[Nguyễn Văn Cương 056]%[9D2N2-1]
	Trong hai giá trị $x=1$ và $x=2$, giá trị nào là nghiệm của bất phương trình $3x-4 \leq 0$?
	\loigiai{
		\begin{itemize}
			\item Thay $x=1$ vào bất phương trình, ta được $3\cdot 1 - 4 \leq 0$ là khẳng định đúng.\\
			Vậy $x=1$ là một nghiệm của bất phương trình đã cho.
			\item Thay $x=2$ vào bất phương trình, ta được $3\cdot 2 - 4 \leq 0$ là khẳng định sai.\\
			Vậy $x=2$ không là nghiệm của bất phương trình đã cho.
		\end{itemize}
	}
\end{vd}

\begin{bt}%[Dự án EX-9-Đề Cương Toán 9]%[Nguyễn Văn Cương 056]%[9D2N2-1]
	Bất phương trình nào sau đây là bất phương trình bậc nhất một ẩn?
	\begin{multicols}{4}
		\begin{enumerate}
			\item $2x-5>0$;
			\item $3y+1 \geq 0$;
			\item $0x-3<0$;
			\item $x^2>0$.
		\end{enumerate}
	\end{multicols}
	\loigiai{
		\begin{enumerate}
			\item $2x-5>0$ là bất phương trình bậc nhất ẩn $x$.
			\item $3y+1 \geq 0$ là bất phương trình bậc nhất ẩn $y$.
			\item $0x-3<0$, vì $a=0$ nên không phải là bất phương trình bậc nhất một ẩn.
			\item $x^2>0$, vì có chứa $x^2$ nên không phải là bất phương trình bậc nhất một ẩn.
		\end{enumerate}
	}
\end{bt}

\begin{bt}%[Dự án EX-9-Đề Cương Toán 9]%[Nguyễn Văn Cương 056]%[9D2N2-1]
	Hãy chỉ ra các bất phương trình bậc nhất một ẩn trong các bất phương trình sau. Cho biết hệ số của ẩn trong mỗi bất phương trình bậc nhất một ẩn đó.
\begin{enumerate}
		\item $t-1<0$;
		\item $x^2-2 \geq 0$;
		\item $\dfrac{t+1}{t+2}<0$;
		\item $2y \geq 0$.
		\end{enumerate}
	\loigiai{
		Bất phương trình ở câu a), c) là bất phương trình bậc nhất một ẩn, trong đó 
		\begin{itemize}
			\item $t-1<0$ có ẩn là $t$, hệ số của ẩn là $a=1$.
			\item $2y\geq 0$ có ẩn là $y$,  hệ số của ẩn là $a=2$.
		\end{itemize}
	}
\end{bt}

\begin{bt}%[Dự án EX-9-Đề Cương Toán 9]%[Nguyễn Văn Cương 056]%[9D2N2-1]
	Trong hai giá trị $x=4$ và $x=3$, giá trị nào là nghiệm của bất phương trình $5x-15 \leq 0$?
	\loigiai{
		\begin{itemize}
			\item Thay $x=4$ vào bất phương trình, ta được $5\cdot 4-15 \leq 0$ là khẳng định sai.\\
			Vậy $x=4$ không là nghiệm của bất phương trình đã cho.	
			\item Thay $x=3$ vào bất phương trình, ta được $5\cdot 3-15 \leq 0$ là khẳng định đúng.\\
			Vậy $x=3$ là một nghiệm của bất phương trình đã cho.
		\end{itemize}	
	}
\end{bt}

\begin{bt}%[Dự án EX-9-Đề Cương Toán 9]%[Nguyễn Văn Cương 056]%[9D2N2-1]
	Kiểm tra xem số nào là nghiệm của mỗi bất phương trình tương ứng sau đây.
\begin{enumerate}
		\item $-3x+2>0$ với $x=-3$; $x=1{,}5$;
		\item $2-2x<3x+1$ với $x=\dfrac{2}{5}$; $x=\dfrac{1}{5}$.
\end{enumerate}
	\loigiai{
\begin{enumerate}

			\item Thay $x=-3$ vào bất phương trình, ta có $-3\cdot(-3)+2>0$ là khẳng định đúng. \\
			Vậy $x=-3$ là nghiệm của bất phương trình. \\
			Thay $x=1{,}5$ vào bất phương trình, ta có $-3\cdot(1{,}5)+2>0$ là khẳng định sai. \\
			Vậy $x=-1{,}5$ không là nghiệm của bất phương trình. 
			\item Thay $x=\dfrac{2}{5}$ vào bất phương trình, ta có $2-2 \cdot \dfrac{2}{5} <3 \cdot \dfrac{2}{5} +1$ là khẳng định đúng. \\
			Vậy $x=\dfrac{2}{5}$ là nghiệm của bất phương trình. \\
			Thay $x=\dfrac{1}{5}$ vào bất phương trình, ta có $2-2 \cdot \dfrac{1}{5} <3 \cdot \dfrac{1}{5} +1$ là khẳng định sai. \\
			Vậy $x=\dfrac{1}{5}$ không là nghiệm của bất phương trình. 
\end{enumerate}
	}
\end{bt}

\begin{bt}%[Dự án EX-9-Đề Cương Toán 9]%[Nguyễn Văn Cương 056]%[9D2N2-1]
	Tìm một số là nghiệm và một số không phải là nghiệm của bất phương trình $4x+5>0$.
	\loigiai{
		\begin{itemize}
			\item Lấy $x=0$ thay vào bất phương trình đã cho, ta được $4 \cdot 0 + 5 > 0$ là khẳng định đúng.\\
			Vậy $x=0$ là một nghiệm của bất phương trình đã cho.
			\item Lấy $x=-2$ thay vào bất phương trình đã cho, ta được $4 \cdot (-2) + 5 > 0$ là khẳng định sai.\\
			Vậy $x=-2$ không phải là nghiệm của bất phương trình đã cho.
		\end{itemize}
		
	}
\end{bt}

\begin{dang}{Giải bất phương trình bậc nhất một ẩn}
	Xét bất phương trình $ax+b>0$ ($a \neq 0$).
	\begin{itemize}
		\item Cộng hai vế của bất phương trình với $-b$, ta được bất phương trình
		$$ax > -b.$$
		\item Nhân hai vế của bất phương trình nhận được với $\dfrac{1}{a}$
		\begin{itemize}
			\item Nếu $a>0$ thì nhận được nghiệm của bất phương trình đã cho là $x > -\dfrac{b}{a}$.
			\item Nếu $a<0$ thì nhận được nghiệm của bất phương trình đã cho là $x<-\dfrac{b}{a}$.
		\end{itemize}
	\end{itemize}
	Với các bất phương trình dạng $ax+b<0$, $ax+b \geq 0$, $ax+b\leq 0$, ta thực hiện các bước giải tương tự.
\end{dang}

%=====Ví dụ 3
\begin{vd}%[Dự án EX-9-Đề Cương Toán 9]%[Nguyễn Văn Cương 056]%[9D2V2-2]
	Giải các bất phương trình sau
	\begin{multicols}{3}
		\begin{enumerate}
			\item $2x + 1 >0$;
			\item $0{,}5x - 6 \leq 0$;
			\item $-2x + 3 \leq 0$.
		\end{enumerate}
	\end{multicols}
	\loigiai{
		\begin{enumerate}
			\item Ta có 
			\allowdisplaybreaks
			\begin{eqnarray*}
				2x+1&>&0\\
				2x &>&-1\\
				(2x) \cdot \dfrac{1}{2}&>&(-1) \cdot \dfrac{1}{2} \\
				x&>&-\dfrac{1}{2}.
			\end{eqnarray*}
			Vậy nghiệm của bất phương trinh là $x>-\dfrac{1}{2}$.
			\item Ta có
			\allowdisplaybreaks
			\begin{eqnarray*}
				0{,}5x-6 &\leq& 0\\ 
				0{,}5x &\leq& 6\\
				(0{,}5x)\cdot 2 &\leq& 6\cdot 2\\
				x &\leq& 12.
			\end{eqnarray*}
			Vậy nghiệm của bất phương trình là $x \leq 12$.
			\item Ta có
			\allowdisplaybreaks
			\begin{eqnarray*}
				-2x+3 &\leq& 0\\ 
				-2x &\leq& -3\\
				(-2x)\cdot\left(-\dfrac{1}{2}\right) &\geq& (-3)\cdot\left(-\dfrac{1}{2}\right)\\
				x &\geq& \dfrac{3}{2}.
			\end{eqnarray*}
			Vậy nghiệm của bất phương trình là $x \dfrac{3}{2}$.
		\end{enumerate}
	}
\end{vd}

%==>Chú ý
\begin{luuy}
	Bằng cách sử dụng các tính chất của bất đẳng thức, ta có thể giải một số bất phương trình đưa được về bất phương trình bậc nhất một ẩn.
\end{luuy}


%=====Ví dụ 4
\begin{vd}%[Dự án EX-9-Đề Cương Toán 9]%[Nguyễn Văn Cương 056]%[9D2V2-2]
	Giải bất phương trình $2x-5 \leq 4x+3$.
	\loigiai{
		Ta có 
		\allowdisplaybreaks
		\begin{eqnarray*}
			2x-5 &\leq& 4x+3 \\
			2x - 4x	&\leq& 3 + 5 \\
			-2x	&\leq& 8 \\
			x &\geq& -4.
		\end{eqnarray*}
		Vậy nghiệm của bất phương trình là $x\geq -4$.
	}
\end{vd}

\begin{bt}%[Dự án EX-9-Đề Cương Toán 9]%[Nguyễn Văn Cương 056]%[9D2V2-2]
	Tìm $x$ sao cho
	\begin{enumerate}
		\item Giá trị của biểu thức $2x+1$ là số dương;
		\item Giá trị của biểu thức $3x-5$ là số âm.
	\end{enumerate}
	\loigiai{
		\begin{enumerate}
			\item Giá trị của biểu thức $2x+1$ là số dương nên
			\allowdisplaybreaks
			\begin{eqnarray*}
				2x + 1 &>& 0\\
				2x &>& -1\\
				x &>& \dfrac{-1}{2}.
			\end{eqnarray*}
			Vậy $x > \dfrac{-1}{2}$.
			\item Giá trị của biểu thức $3x-5$ là số âm nên
			\allowdisplaybreaks
			\begin{eqnarray*}
				3x-5 &<& 0\\
				3x &<& 5\\
				x &<&\dfrac{5}{3}. 
			\end{eqnarray*}
			Vậy $x <\dfrac{5}{3}$.
		\end{enumerate}
	}
\end{bt}

\begin{bt}%[Dự án EX-9-Đề Cương Toán 9]%[Nguyễn Văn Cương 056]%[9D2V2-2]
	Giải các bất phương trình sau
	\begin{multicols}{4}
		\begin{enumerate}
			\item $6 < x-3$;
			\item $\dfrac{1}{2}\cdot x > 5$;
			\item $-8x+1 \geq 5$;
			\item $7 < 2x+1$;
			\item $12x-4\leq 3x+12$;
			\item $1{,}5< 2{,}3-4x$;
			\item $10-0{,}5x \leq -3{,}5x$;
			\item $3{,}5x-1\geq 6$;
			\item $5+\dfrac{2}{3}x > 3$;
			\item $2x+\dfrac{4}{5} > \dfrac{9}{5}$;
			\item $5-\dfrac{1}{3}x > 2$;
			\item $3x-8 > 4x-12$.
		\end{enumerate}
	\end{multicols}
	\loigiai{
		\begin{enumerate}
			\item Ta có
			\allowdisplaybreaks
			\begin{eqnarray*}
				6 &<& x-3\\
				6+3 &< x\\
				9&<x.
			\end{eqnarray*}
			Vậy nghiệm của bất phương trình là $x>9$.
			\item Ta có
			\allowdisplaybreaks
			\begin{eqnarray*}
				\dfrac{1}{2}\cdot x	&>& 5\\
				x &>& 10.
			\end{eqnarray*}
			Vậy nghiệm của bất phương trình là $x>10$.
			\item Ta có
			\allowdisplaybreaks
			\begin{eqnarray*}
				-8x+1 &\geq& 5\\
				-8x &\geq& 4 \\
				x &\leq& \dfrac{4}{-8}\\
				x &\leq& \dfrac{-1}{2}.
			\end{eqnarray*}
			Vậy nghiệm của bất phương trình là $x\leq \dfrac{-1}{2}$.
			\item Ta có
			\allowdisplaybreaks
			\begin{eqnarray*}
				7 &<& 2x + 1 \\		
				6 &<& 2x \\
				\dfrac{6}{2} &<& x	\\
				3&<&x.
			\end{eqnarray*}
			Vậy nghiệm của bất phương trình là $x>3$.
			\item  Ta có
			\allowdisplaybreaks
			\begin{eqnarray*}
				12x-4 &\leq& 3x+12 \\		
				12x-3x&\leq& 12+4 \\
				9x&\leq& 16	\\
				x&\leq&\dfrac{16}{9}.
			\end{eqnarray*}
			Vậy nghiệm của bất phương trình là $x\leq\dfrac{16}{9}$.
			\item 
			Ta có
			\allowdisplaybreaks
			\begin{eqnarray*}
				1{,}5&<& 2{,}3-4x \\
				1{,}5 -2{,}3 &<&-4x \\
				-0{,}8 &<& -4x \\
				0{,}2 &>& x.
			\end{eqnarray*}
			Vậy nghiệm của bất phương trình là $x>0{,}2$.
			\item
			Ta có
			\allowdisplaybreaks
			\begin{eqnarray*}
				10-0{,}5x&\leq& -3{,}5x\\
				-0{,}5x+3{,}5x&\leq&-10\\
				3x&\leq&-10\\
				x&\geq&\dfrac{-10}{3}.
			\end{eqnarray*}
			Vậy nghiệm của bất phương trình là $x\geq\dfrac{-10}{3}$.
			\item
			Ta có
			\allowdisplaybreaks
			\begin{eqnarray*}
				3{,}5x-1&\geq& 6  \\
				3{,}5x&\geq 6+1&\\
				3{,}5x&\geq7&\\
				x&\geq&2.
			\end{eqnarray*}
			Vậy nghiệm của bất phương trình là $x\geq2$.
			\item
			Ta có
			\allowdisplaybreaks
			\begin{eqnarray*}
				5+\dfrac{2}{3}x &>& 3\\
				\dfrac{2}{3}x&>&3-5\\
				\dfrac{2}{3}&>&-2\\
				x&>&-3.
			\end{eqnarray*}
			Vậy nghiệm của bất phương trình là $x>-3$.
			\item
			Ta có
			\allowdisplaybreaks
			\begin{eqnarray*}
				2x+\dfrac{4}{5} &>& \dfrac{9}{5}\\
				2x&>&\dfrac{9}{5}-\dfrac{4}{5}\\
				2x&>&1\\
				x&>&\dfrac{1}{2}.
			\end{eqnarray*}	
			Vậy nghiệm của bất phương trình là $x>\dfrac{1}{2}$.
			\item
			Ta có
			\allowdisplaybreaks
			\begin{eqnarray*}
				5-\dfrac{1}{3}x &>& 2\\
				-\dfrac{1}{3}x&>&2-5\\
				-\dfrac{1}{3}x&>&-3\\
				x&<&9.
			\end{eqnarray*}
			Vậy nghiệm của bất phương trình là $x<9$.
			\item
			\allowdisplaybreaks
			\begin{eqnarray*}
				3x-8 &>& 4x-12 \\
				3x-4x&>&-12+8\\
				-x&>&-4\\
				x&<&4.
			\end{eqnarray*}
			Vậy nghiệm của bất phương trình là $x<4$.
		\end{enumerate}
	}
\end{bt}

\begin{bt}%[Dự án EX-9-Đề Cương Toán 9]%[Nguyễn Văn Cương 056]%[9D2V2-2]
	Giải các bất phương trình sau
	\begin{multicols}{4}
		\begin{enumerate}
			\item $x-7 < 2-x$;
			\item $x+2 \leq 2+3x$;
			\item $4+x > 5-3x$;
			\item $-x+7 \geq x-3$.
		\end{enumerate}
	\end{multicols}
	\loigiai{
		\begin{enumerate}
			\item Ta có
			\allowdisplaybreaks
			\begin{eqnarray*}
				x - 7 &<& 2 - x\\
				x &<& 9 - x\\
				2x &<& 9\\
				x &<& \dfrac{9}{2}.
			\end{eqnarray*}
			Nghiệm của bất phương trình là $x < \dfrac{9}{2}$.
			\item Ta có
			\allowdisplaybreaks
			\begin{eqnarray*}
				x + 2 &\leq& 2 + 3x  \\
				x &\leq& 3x \\
				0 &\leq& 2x \\
				0 &\leq& x.
			\end{eqnarray*}
			Nghiệm của bất phương trình là $x \geq 0$.
			\item Ta có
			\allowdisplaybreaks
			\begin{eqnarray*}
				4 + x &>& 5 - 3x \\
				4x &>& 1 \\
				x &>& \dfrac{1}{4}.
			\end{eqnarray*}
			Nghiệm của bất phương trình là $x > \dfrac{1}{4}$.
			\item Ta có
			\allowdisplaybreaks
			\begin{eqnarray*}
				-x + 7 &\geq& x - 3 \\
				7 &\geq& 2x - 3 \\
				10 &\geq& 2x \\
				5 &\geq& x.
			\end{eqnarray*}
			Vậy nghiệm của bất phương trình là $x \leq 5$.
		\end{enumerate}
	}
\end{bt}

\begin{bt}%[Dự án EX-9-Đề Cương Toán 9]%[Nguyễn Văn Cương 056]%[9D2V2-2]
	Giải các bất phương trình sau
	\begin{multicols}{2}
		\begin{enumerate}
			\item $\dfrac{2}{3}(2x+3) < 7-4x$;
			\item $\dfrac{1}{4}(x-3) \leq 3-2x$;
			\item $2x-4(x+2) \geq 5(-2x+1)$;
			\item $-7x+2(x-4) \leq 5-3(x-2)$;
			\item $5x-3(2-7x) > 5(x-2)+8$;
			\item $8x+3(x+1) > 5x-(2x-6)$;
			\item $6-7(x-4) \geq 3x+2(3-x)$;
			\item $10x-3(x-5) > 3x-2(x-4)$;
			\item $2x(6x-1) > (3x-2)(4x+3)$.
		\end{enumerate}
	\end{multicols}
	\loigiai{
		\begin{enumerate}
			\item Ta có
			\allowdisplaybreaks
			\begin{eqnarray*}
				\dfrac{2}{3}(2x+3) &<& 7-4x \\
				2x + 3 &<& \dfrac{21}{2} - 6x \\
				8x + 3 &<& \dfrac{21}{2} \\
				8x &<& \dfrac{21}{2} - 3 \\
				8x &<& \dfrac{15}{2} \\
				x &<& \dfrac{15}{16}.
			\end{eqnarray*}
			Vậy nghiệm của bất phương trình là $x < \dfrac{15}{16}$.
			\item Ta có
			\allowdisplaybreaks
			\begin{eqnarray*}
				\dfrac{1}{4}(x-3) &\leq& 3-2x \\
				( x-3&\leq& 4(3 - 2x) \\
				x-3 &\leq&7 12 - 8x \\
				9x - 3 &\leq& 12 \\
				9x &\leq& 15 \\
				x &\leq& \dfrac{5}{3}. 
			\end{eqnarray*}
			Vậy nghiệm của bất phương trình là $x \leq \dfrac{5}{3}$.
			\item $2x-4(x+2) \geq 5(-2x+1)$;
			Ta có
			\allowdisplaybreaks
			\begin{eqnarray*}
				2x-4(x+2) &\geq& 5(-2x+1)\\
				2x-4x-8&\geq&-10x-5\\
				2x-4x+10x&\geq&-5+8\\
				8x&\geq&3\\
				x&\geq&\dfrac{3}{8}.
			\end{eqnarray*}	
			Vậy nghiệm của bất phương trình là $x \geq \dfrac{3}{8}$.
			\item
			\allowdisplaybreaks
			\begin{eqnarray*}
				-7x+2(x-4) &\leq& 5-3(x-2)\\
				-7x+2x-8&\leq&5-3x+6\\
				-7x+2x+3x&\leq&5+6+8\\
				-2x&\leq&19\\
				x&\geq&\dfrac{-19}{2}.
			\end{eqnarray*}
			Vậy nghiệm của bất phương trình là $x \geq \dfrac{-19}{2}$.
			\item 
			Ta có
			\allowdisplaybreaks
			\begin{eqnarray*}
				5x-3(2-7x) &>& 5(x-2)+8\\
				5x-6+21x&>&5x-10+8\\
				5x+21x-5x&>&-10+8+6\\
				21x&>&4\\
				x&>&\dfrac{4}{21}.
			\end{eqnarray*}
			Vậy nghiệm của bất phương trình là $x > \dfrac{4}{21}$.
			\item
			Ta có
			\allowdisplaybreaks
			\begin{eqnarray*}
				8x+3(x+1) &>& 5x-(2x-6)\\
				8x+3x+3&>&5x-2x+6\\
				8x+3x-5x+2x&>&6+3\\
				8x&>&9\\
				x&>&\dfrac{9}{8}.
			\end{eqnarray*}
			Vậy nghiệm của bất phương trình là $x > \dfrac{9}{8}$.
			\item $6-7(x-4) \geq 3x+2(3-x)$;
			Ta có
			\allowdisplaybreaks
			\begin{eqnarray*}
				6-7(x-4) &\geq& 3x+2(3-x)\\
				6-7x+28&\geq&3x+6-2x\\
				-7x-3x+2x&\geq&6-6-28\\
				-8x&\geq&-28\\
				x&\leq&\dfrac{7}{2}
			\end{eqnarray*}
			Vậy nghiệm của bất phương trình là $x \leq \dfrac{7}{2}$.
			\item
			Ta có
			\allowdisplaybreaks
			\begin{eqnarray*}
				10x-3(x-5) &>& 3x-2(x-4)\\
				10x-3x+15&>&3x-2x+8\\
				10x-3x-3x+2x&>&8-15\\
				6x&>-7&\\
				x&>&\dfrac{-7}{6}.
			\end{eqnarray*}
			Vậy nghiệm của bất phương trình là $x >\dfrac{-7}{6}$.
			\item
			Ta có
			\allowdisplaybreaks
			\begin{eqnarray*}
				2x(6x-1) &>& (3x-2)(4x+3)\\
				12x^2-2x&>&12x^2+9x-8x-6 \\
				12x^2-2x-12x^2-9x+8x&>&-6 \\
				-3x&>&-6 \\
				x&<&2.
			\end{eqnarray*}
			Vậy nghiệm của bất phương trình là $x<2$.
		\end{enumerate}
	}
\end{bt}

\begin{bt}%[Dự án EX-9-Đề Cương Toán 9]%[Nguyễn Văn Cương 056]%[9D2V2-2]
	Giải các bất phương trình
	\begin{enumerate}
		\item $\dfrac{8-3x}{2}-x<5$;
		\item $3-2x-\dfrac{6+4x}{3}>0$;
		\item $0{,}7x+\dfrac{2x-4}{3}-\dfrac{x}{6}>1$.
	\end{enumerate}
	\loigiai{
		\begin{enumerate}
			\item
			Ta có
			\allowdisplaybreaks
			\begin{eqnarray*}
				\dfrac{8-3x}{2}-x&<&5\\
				(8-3x)-2x& <&10 \\
				8-3x-2x &<& 10 \\
				-5x &<&10-8 \\
				x &>& 2:(-5) \\
				x&>&-0{,}4.
			\end{eqnarray*}
			Vậy nghiệm của bất phương trình là $x> -0{,}4$.
			\item 
			Ta có
			\allowdisplaybreaks
			\begin{eqnarray*}
				3-2x-\dfrac{6+4 x}{3}& >&0\\
				9-6x-(6+4x) &>&0 \\
				9-6x-6-4x & >&0 \\
				-10x & >& -3 \\
				x &<&(-3):(-10) \\
				x&<& \dfrac{3}{10}
			\end{eqnarray*}	
			Vậy nghiệm của bất phương trình là $x<\dfrac{3}{10}$.
			\item
			Ta có
			\allowdisplaybreaks
			\begin{eqnarray*}
				0{,}7x+\dfrac{2 x-4}{3}-\dfrac{x}{6}& >&1\\
				4{,}2 x+2(2x-4)-x& >&6 \\
				4{,}2 x+4x-8-x& >&6 \\
				4{,}2 x+4x-x& >&6+8 \\
				7{,}2 x& >&14 \\
				x&>&14: 7{,}2 \\
				x&>&\dfrac{35}{18}. 
			\end{eqnarray*}	
			Vậy nghiệm của bất phương trình là $x>\dfrac{35}{18}$.
		\end{enumerate}
	}
\end{bt}

\begin{bt}%[Dự án EX-9-Đề Cương Toán 9]%[Nguyễn Văn Cương 056]%[9D2V2-2]
	Giải các bất phương trình sau
	\begin{multicols}{2}
		\begin{enumerate}
			\item $\dfrac{x+1}{3}+\dfrac{x}{2} \geq 4$;
			\item $4-\dfrac{x+4}{8} \geq \dfrac{x-5}{2}$;
			\item $\dfrac{7-2x}{4} \leq \dfrac{5x-1}{8}+4$;
			\item $\dfrac{3-2x}{3} > \dfrac{1-5x}{9}+2$;
			\item $\dfrac{-x+3}{6}-\dfrac{x-2}{3} \leq \dfrac{-5}{4}$;
			\item $\dfrac{3x-1}{3}-\dfrac{2x-3}{4} \geq \dfrac{4x-1}{6}$;
			\item $\dfrac{4x-1}{3}+\dfrac{2+x}{15} \geq \dfrac{2x-3}{5}$;
			\item $\dfrac{x-1}{2}+\dfrac{x-1}{4} \leq 1-\dfrac{2(x-1)}{3}$.
		\end{enumerate}
	\end{multicols}	
	\loigiai{
		\begin{enumerate}
			\item
			Ta có
			\allowdisplaybreaks
			\begin{eqnarray*}
				\dfrac{x+1}{3}+\dfrac{x}{2} &\geq& 4\\
				2(x+1)+3x&\geq&24\\
				2x+2+3x&\geq&24\\
				2x+3x&\geq&24-2\\
				5x&\geq&22\\
				x&\geq&\dfrac{22}{5}.
			\end{eqnarray*}
			Vậy nghiệm của bất phương trình là $x\geq\dfrac{22}{5}$.
			\item
			Ta có
			\allowdisplaybreaks
			\begin{eqnarray*}
				4-\dfrac{x+4}{8} &\geq& \dfrac{x-5}{2}\\
				32-x-4&\geq&4x-20\\
				-x-4x&\geq&-20-32+4\\
				-5x&\geq&-48\\
				x&\leq&\dfrac{48}{5}.
			\end{eqnarray*}
			Vậy nghiệm của bất phương trình là $x\leq\dfrac{48}{5}$.
			\item 
			Ta có
			\allowdisplaybreaks
			\begin{eqnarray*}
				\dfrac{7-2x}{4} &\leq& \dfrac{5x-1}{8}+4\\
				14-4x&\leq&5x-1+32\\
				-4x-5x&\leq&-1+32-14\\
				-9x&\leq&17\\
				x&\geq&\dfrac{-17}{9}.
			\end{eqnarray*}
			Vậy nghiệm của bất phương trình là $x\geq\dfrac{-17}{9}$.	
			\item
			Ta có
			\allowdisplaybreaks
			\begin{eqnarray*}
				\dfrac{3-2x}{3} &>& \dfrac{1-5x}{9}+2\\
				9-6x&>&1-5x+18\\
				-6x-5x&>&1+18-9\\
				-11x&>&10\\
				x&<&\dfrac{-10}{11}.
			\end{eqnarray*}
			Vậy nghiệm của bất phương trình là $x<\dfrac{-10}{11}$.	
			\item
			Ta có
			\allowdisplaybreaks
			\begin{eqnarray*}
				\dfrac{-x+3}{6}-\dfrac{x-2}{3} &\leq& \dfrac{-5}{4}\\
				-2x+6-4x+8&\leq&-15\\
				-2x-4x&\leq&-15-6-8\\
				-6x&\leq&-29\\
				x&\geq&\dfrac{29}{6}.
			\end{eqnarray*}
			Vậy nghiệm của bất phương trình là $x\geq\dfrac{29}{6}$.	
			\item 
			Ta có
			\allowdisplaybreaks
			\begin{eqnarray*}
				\dfrac{3x-1}{3}-\dfrac{2x-3}{4} &\geq& \dfrac{4x-1}{6}\\
				12x-4-6x+9&\geq&8x-2\\
				12x-6x-8x&\geq&-2+4-9\\
				-2x&\geq&-7\\
				x&\leq&\dfrac{7}{2}.
			\end{eqnarray*}
			Vậy nghiệm của bất phương trình là $x\leq\dfrac{7}{2}$.		
			\item 
			Ta có
			\allowdisplaybreaks
			\begin{eqnarray*}
				\dfrac{4x-1}{3}+\dfrac{2+x}{15} &\geq& \dfrac{2x-3}{5}\\
				20x-5+2+x&\geq&6x-9\\
				20x+x-6x&\geq&-9+5-2\\
				15x&\geq&-6\\
				x&\leq&\dfrac{-6}{15}.
			\end{eqnarray*}
			Vậy nghiệm của bất phương trình là $x\leq\dfrac{-6}{15}$.		
			\item
			\allowdisplaybreaks
			\begin{eqnarray*}
				\dfrac{x-1}{2}+\dfrac{x-1}{4} &\leq& 1-\dfrac{2(x-1)}{3}\\
				6x-6+3x-3&\leq&12-8x+8\\
				6x+3x+8x&\leq&12+8+6+3\\
				17x&\leq&29\\
				x&\leq&\dfrac{29}{17}.
			\end{eqnarray*}
			Vậy nghiệm của bất phương trình là $x\leq\dfrac{29}{17}$.
		\end{enumerate}
	}
\end{bt}

\begin{bt}%[MaT-SGK9-Moi]%[Tên Thành Viên biên soạn - Tên thành viên phản biện]
	Tìm $x>0$ sao cho ở hình bên dưới chu vi của hình tam giác luôn hơn chu vi của hình chữ nhật
	\begin{center}
		\begin{tikzpicture}[line join=round, line cap=round, >=stealth, scale=0.8]%Hinh nón
			\tikzset{every node/.style={scale=0.7}}% thu nhỏ phóng tỏ tex trong hình
			%\clip(-2,-1) rectangle (7,5);
			% set up coordinates for an easy use
			\coordinate (C) at (0,0);
			\coordinate (B) at (5,0);
			\coordinate (A) at (3,3);
			
			\coordinate (m) at ($(A)!0.5!(B)$);	
			\coordinate (n) at ($(A)!0.5!(C)$);
			\coordinate (p) at ($(B)!0.5!(C)$);
			
			\draw (A)--(B)--(C)--(A);	
			
			\draw[fill=black] 
			(A) circle (0.05) 
			(B) circle (0.05)
			(C) circle (0.05) 
			(m) node[right] {$x+2$}
			(n)  node[left] {$x+4$}
			(p)  node[below] {$x+5$}
			
			;		
			%	\draw pic[draw,angle radius=2mm] {right angle = O--J--C};
			%	\draw pic[draw,angle radius=3.5mm] {angle = P--A--C};		
		\end{tikzpicture} \, \, \, \, \, \, 
		\begin{tikzpicture}[line join=round, line cap=round, >=stealth, scale=0.8]%Hinh nón
			\tikzset{every node/.style={scale=0.7}}% thu nhỏ phóng tỏ tex trong hình
			%\clip(-2,-1) rectangle (7,5);
			% set up coordinates for an easy use
			\coordinate (A) at (0,0);
			\coordinate (B) at (5,0);
			\coordinate (C) at (5,3);
			\coordinate (D) at (0,3);
			
			\coordinate (m) at ($(A)!0.5!(B)$);	
			\coordinate (n) at ($(B)!0.5!(C)$);
			
			\draw (A)--(B)--(C)--(D)--(A);	
			
			\draw[fill=black] 
			(A) circle (0.05) 
			(B) circle (0.05)
			(C) circle (0.05) 
			(D) circle (0.05) 
			(m) node[below] {$x+3$}
			(n)  node[right] {$x+1$}	
			;		
			%	\draw pic[draw,angle radius=2mm] {right angle = O--J--C};
			%	\draw pic[draw,angle radius=3.5mm] {angle = P--A--C};		
		\end{tikzpicture}
	\end{center}
	\loigiai{
		Chu vi tam giác là $x+2+x+4+x+5= 3x+11$.\\
		Chu vi hình chữ nhật là $2(x+3+x+1)=4x+8$.\\
		Theo bài ra ta có
		\allowdisplaybreaks
		\begin{eqnarray*}
			3x+11&>& 4x+8\\
			3x-4x&>& 8-11\\
			-x & >& -3 \\
			x & <& 3.
		\end{eqnarray*}
		Vậy $0<x<3$.
	}
\end{bt}

\begin{dang}{Bài toán thực tế}
	
\end{dang}

%=====Ví dụ 5
\begin{vd}%[Dự án EX-9-Đề Cương Toán 9]%[Nguyễn Văn Cương 056]%[9D2V2-3]
	Bạn Thanh có $100$ nghìn đồng. Bạn muốn mua một cái bút giá $18$ nghìn đồng và một số quyển vở, mỗi quyển vở giá $7$ nghìn đồng. Hỏi bạn Thanh mua được nhiều nhất bao nhiêu quyển vở?
	\loigiai{
		Gọi $x$ (quyển) là số vở mà Thanh có thể mua $(x \in \mathbb{N}^{*})$.\\
		Theo đề bài, ta có bất phương trình
		\allowdisplaybreaks
		\begin{eqnarray*}
			7x + 18 &\leq& 100 \\
			7x &\leq& 100-18 \\
			7x &\leq& 82 \\
			x &\leq& \dfrac{82}{7} \, (\approx 11{,}71).
		\end{eqnarray*}
		Vì số vở là số tự nhiên nên Thanh có thể mua nhiều nhất $11$ quyển vở.
	}
\end{vd}

%=====Ví dụ 6
\begin{vd}%[Dự án EX-9-Đề Cương Toán 9]%[Nguyễn Văn Cương 056]%[9D2V2-3]
	Trong một kì thi gồm ba môn Toán, Ngữ văn và Tiếng Anh, điểm số môn Toán và Ngữ văn tính theo hệ số $2$, điểm số môn Tiếng Anh tính theo hệ số $1$. Để trúng tuyển, điểm số trung bình của ba môn ít nhất phải bằng $8$. Bạn Na đã đạt $9{,}1$ điểm môn Toán và $6{,}9$ điểm môn Ngữ văn. Hãy lập và giải bất phương trình để tìm điểm số Tiếng Anh tối thiểu mà bạn Na phải đạt để trúng tuyển.
	\loigiai{
		Gọi $x$ là điểm số môn Tiếng Anh của bạn Na.\\
		Theo đề bài, để bạn Na trúng tuyển, ta phải có
		\allowdisplaybreaks
		\begin{eqnarray*}
			\dfrac{2\cdot 9{,}1 + 2 \cdot 6{,}9 + x}{5} &\geq& 8\\
			2\cdot 9{,}1 + 2 \cdot 6{,}9 + x &\geq& 40\\
			18{,}2 + 13{,}8 + x &\geq& 40\\
			x &\geq& 8.
		\end{eqnarray*}
		Vậy để trúng tuyển, bạn Na phải đạt ít nhất $8$ điểm môn Tiếng Anh.
	}
\end{vd}

\begin{bt}%[Dự án EX-9-Đề Cương Toán 9]%[Nguyễn Văn Cương 056]%[9D2V2-3]
	Bác sĩ khuyên cô Vân mỗi ngày ăn không quá $60$ (gam) chất béo. Hôm nay, theo tính toán của cô Vân về lượng chất béo đã ăn thì bữa điểm tâm sáng là $8$ g, bữa trưa là $31$ g. Nếu tuân thủ lời khuyên của bác sĩ thì cô Vân còn có thể ăn nhiều nhất là bao nhiêu gam chất béo trong thời gian còn lại của ngày?
	\loigiai{
		Gọi $x$ là số gam chất béo mà cô Vân có thể ăn trong thời gian còn lại của ngày ($x<60$).\\
		Theo đề bài, ta có bất phương trình
		\begin{eqnarray*}
			\allowdisplaybreaks
			&&8+31+x\leq 60\\
			&&x+39\leq 60\\
			&&x\leq 60-39\\
			&&x\leq 21.
		\end{eqnarray*}
		Vậy cô Vân có thể ăn nhiều nhất $21~\mathrm{g}$ chất béo trong khoảng thời gian còn lại của ngày.
	}
\end{bt}

\begin{bt}%[Dự án EX-9-Đề Cương Toán 9]%[Nguyễn Văn Cương 056]%[9D2V2-3]
	Người ta dùng một loại xe tải để chở bia cho một nhà máy. Mỗi thùng bia $24$ lon nặng trung bình $6{,}7$ kg. Theo khuyến nghị, trọng tải của xe (tức là tổng khối lượng tối đa cho phép mà xe có thể chở) là $5{,}25$ tấn. Hỏi xe có thể chở được tối đa bao nhiêu thùng bia, biết bác lái xe nặng $65$ kg?
	\loigiai{
		Gọi $x$ (thùng) là số thùng bia mà xe có thể chở $(x\in \mathbb{N}).$\\
		Theo đề bài, ta có bất phương trình
		\allowdisplaybreaks
		\begin{eqnarray*}
			65+6{,}7x &\leq& 5{,}25 \cdot 1000 \\
			6{,}7x &\leq& 5185 \\
			x &\leq& \dfrac{5185}{6{,}7}.
		\end{eqnarray*}	
		Vậy số thùng bia tối đa mà xe có thể chở là $773$ thùng.
	}
\end{bt}

\begin{bt}%[Dự án EX-9-Đề Cương Toán 9]%[Nguyễn Văn Cương 056]%[9D2V2-3]
	Tổng chi phí của một doanh nghiệp sản xuất áo sơ mi là $410$ triệu đồng/tháng. Giá bán của mỗi chiếc áo sơ mi là $350$ nghìn đồng. Hỏi trung bình mỗi tháng doanh nghiệp phải bán được ít nhất bao nhiêu chiếc áo sơ mi để thu được lợi nhuận ít nhất là $1{,}38$ tỉ đồng sau $1$ năm?
	\loigiai{
		Gọi $x$ (chiếc áo sơ mi) là số chiếc áo trung bình mỗi tháng doanh nghiệp bán được  $\left(x \in \mathbb{N}^{*}\right)$.\\	
		Số tiền bán được áo trong mỗi tháng là $350\,000x$ (đồng).\\
		Lợi nhuận của doanh nghiệp sau $12$ tháng là	
		\[12 (350\,000x-410\,000\, 000)\ (\text{đồng}). \]	
		Do đó, để doanh nghiệp thu được lợi nhuận ít nhất là $1{,}38$ tỉ đồng thì		
		\[12 (350\,000 x-410\,000\,000) \geq 1380\,000\,000. \]		
		Giải bất phương trình trên, ta có
		\allowdisplaybreaks
		\begin{eqnarray*}
			12(350\,000 x-410\,000\,000) & \geq& 1\,380\,000\,000 \\
			350\,000 x-410\,000\,000 & \geq &115\,000\,000 \\
			350\,000 x & \geq& 115\,000\,000+410\,000\,000 \\
			350\,000 x & \geq& 525\,000\,000 \\
			x & \geq& \frac{525\,000\,000}{350\,000} \\
			x & \geq &1\,500.
		\end{eqnarray*}	
		Vậy trung bình mỗi tháng doanh nghiệp phải bán được ít nhất $1\,500$ chiếc áo sơ mi để doanh nghiệp thu được lợi nhuận ít nhất là $1{,}38$ tỉ đồng sau $1$ năm.
	}
\end{bt}

\begin{bt}%[Dự án EX-9-Đề Cương Toán 9]%[Nguyễn Văn Cương 056]%[9D2V2-3]
	Một ngân hàng đang áp dụng lãi suất gửi tiết kiệm kì hạn $12$ tháng là $7{,}4\%$/năm. Bà Mai dự kiến gửi một khoản tiền vào ngân hàng này và cần số tiền lãi hằng năm ít nhất là $60$ triệu để chi tiêu. Hỏi số tiền bà Mai cần gửi tiết kiệm ít nhất là bao nhiêu (làm tròn đến triệu đồng)?
	\loigiai{
		Gọi $x$ (triệu đồng) là số tiền bà Mai cần gửi tiết kiệm $(x > 0)$.\\
		Ta có số tiền lãi gửi tiết kiệm $x$ (triệu đồng) trong một năm là $0{,}074 \cdot x$ (triệu đồng).\\
		Để có số tiền lãi ít nhất là $60$ triệu đồng/năm thì ta phải có
		\allowdisplaybreaks
		\begin{eqnarray*}
			0{,}074x &\geq&60 \\
			x &\geq& \dfrac{60}{0{,}074} \\
			x &\geq& 810{,}81.
		\end{eqnarray*}	
		Vậy bà Mai cần gửi ngân hàng ít nhất $811$ triệu đồng.
	}
\end{bt}

\begin{bt}%[Dự án EX-9-Đề Cương Toán 9]%[Nguyễn Văn Cương 056]%[9D2V2-3]
	Một ngân hàng đang áp dụng lãi suất gửi tiết kiệm kì hạn $1$ tháng là $0{,}4\%$. Hỏi nếu muốn có số tiền lãi hằng tháng ít nhất là $3$ triệu đồng thì số tiền gửi tiết kiệm ít nhất là bao nhiêu (làm tròn đến triệu đồng)?
	\loigiai{
		Gọi $x$ (triệu đồng) là số tiền cần gửi $(x>0)$.\\
		Theo đề bài ta có bất phương trình
		\allowdisplaybreaks
		\begin{eqnarray*}
			0{,}4\%x &\geq&3 \\ 
			\dfrac{0{,}4}{100} \cdot x &\geq& 3 \\
			x &\geq&\dfrac{3\cdot 1000}{4}\\
			x &\geq& 750.
		\end{eqnarray*}	
		Vậy để có lãi ít nhất $3$ triệu/tháng thì số tiền gửi tiết kiệm ít nhất là $750$ triệu.
	}
\end{bt}

\begin{bt}%[Dự án EX-9-Đề Cương Toán 9]%[Nguyễn Văn Cương 056]%[9D2V2-3]
	Một hãng taxi có giá mở cửa là $15$ nghìn đồng và giá $12$ nghìn đồng cho mỗi kilômét tiếp theo. Hỏi với $200$ nghìn đồng thì hành khách có thể di chuyển được tối đa bao nhiêu kilômét (làm tròn đến hàng đơn vị)?
	\loigiai{
		Gọi $x$ (km) là số km mà hành khách có thể đi ($x>0$).\\
		Theo đề bài, ta có bất phương trình
		\allowdisplaybreaks
		\begin{eqnarray*}
			15+12x-12&\leq& 200\\ 
			12x &\leq&200-3 \\
			x &\leq& \dfrac{197}{12}.
		\end{eqnarray*}	
		Vậy số km tối đa khách hàng có thể đi là $16$ km.
	}
\end{bt}

\begin{bt}%[Dự án EX-9-Đề Cương Toán 9]%[Nguyễn Văn Cương 056]%[9D2V2-3]
	Trong một cuộc thi tuyển dụng việc làm, ban tổ chức quy định mỗi người ứng tuyển phải trả lời $25$ câu hỏi ở vòng sơ tuyển. Mỗi câu hỏi này có sẵn bốn đáp án, trong đó chỉ có một đáp án đúng. Người ứng tuyển chọn đáp án đúng sẽ được cộng thêm $2$ điểm, chọn đáp án sai bị trừ đi $1$ điểm. Ở vòng sơ tuyển, ban tổ chức tặng cho mỗi người dự thi $5$ điểm và theo quy định người ứng tuyển phải trả lời hết $25$ câu hỏi; người nào có số điểm từ $25$ trở lên mới được dự thi vòng tiếp theo. Hỏi người ứng tuyển phải trả lời chính xác ít nhất bao nhiêu câu hỏi ở vòng sơ tuyển thì mới được vào vòng tiếp theo?
	\loigiai{
		Gọi $x$ (câu hỏi)  là số câu trả lời đúng của người ứng tuyển  ($x \in \mathbb{N}$, $x \leq 25$).\\
		Số câu trả lời sai là $25-x$ (câu hỏi) .\\
		Số điểm của người ứng tuyển sau $25$ câu hỏi là $5+2x-(25-x)=3x-20$ điểm.\\
		Để vượt qua vòng sơ tuyển cần ít nhất $25$ điểm nên ta có bất phương trình
		\allowdisplaybreaks
		\begin{eqnarray*}
			3x-20 &\geq& 25\\ 
			3x &\geq&45\\
			x &\geq&15.\\
		\end{eqnarray*}	
		Vậy người ứng tuyển phải trả lời chính xác ít nhất $15$ câu hỏi.
	}
\end{bt}

\begin{bt}%[Dự án EX-9-Đề Cương Toán 9]%[Nguyễn Văn Cương 056]%[9D2V2-3]
	Một kho chứa $100$ tấn xi măng, mỗi ngày đều xuất đi $20$ tấn xi măng. Gọi $x$ là số ngày xuất xi măng của kho đó. Tìm $x$ sao cho khối lượng xi măng còn lại trong kho ít nhất là $10$ tấn sau $x$ ngày xuất hàng.
	\loigiai{
		Sau $x$ ngày khối lượng xi măng được xuất đi là $20x$.\\
		Khối lượng xi măng còn lại trong kho sau $x$ ngày là $100-20x$.\\
		Theo đề bài, ta có bất phương trình
		\allowdisplaybreaks
		\begin{eqnarray*}
			100-20x& \geq& 10\\
			-20x &\geq & -90 \\
			x &\leq& \dfrac{-90}{-20} \\
			x &\leq& \dfrac{9}{2}.
		\end{eqnarray*}	
		Vậy $x \in \{1;2;3;4\}$.
	}
\end{bt}

\begin{bt}%[Dự án EX-9-Đề Cương Toán 9]%[Nguyễn Văn Cương 056]%[9D2V2-3]
	Bạn Dũng tham gia học tiếng Anh ở một trung tâm ngoại ngữ. Qua hai bài kiểm tra của khoá học, bạn đã đạt lần lượt $60$ và $67$ điểm (thang điểm $100$). Bạn phấn đấu đạt điểm trung bình ít nhất là $70$ sau ba lần kiểm tra. Để có kết quả này, ở lần kiểm tra thứ ba, bạn Dũng phải được ít nhất bao nhiêu điểm?
	\loigiai{
		Gọi $x$ (điểm) là số điểm của bạn Dũng ở lần kiểm tra thứ ba ($x\leq0\leq100$).\\
		Theo đề bài, ta có bất phương trình
		\begin{eqnarray*}
			\dfrac{60+67+x}{3}&\geq& 70 \\
			127+x &\geq&210 \\
			x&\geq& 210-127 \\
			x &\geq& 83.
		\end{eqnarray*}
		Như vậy số điểm ít nhất bạn Dũng cần phải đạt được trong kì kiểm tra thứ ba là $83$ .
	}
\end{bt}

\begin{bt}%[Dự án EX-9-Đề Cương Toán 9]%[Nguyễn Văn Cương 056]%[9D2V2-3]
	Một kì thi Tiếng Anh gồm bốn kĩ năng: nghe, nói, đọc và viết. Kết quả của bài thi là điểm số trung bình của bốn kĩ nảng này. Bạn Hà đã đạt được điểm số của ba kĩ năng nghe, đọc, viết lần lượt là $6{,}5$; $6{,5}$; $5{,}5$. Hỏi bạn Hà cần đạt bao nhiêu điểm trong kĩ năng nói để kết quả đạt được của bài thi ít nhất là $6,25$?
	\loigiai{
		Gọi $x$ (điểm) là điểm số của kĩ năng nói của bạn Hà $(x>0)$.\\
		Trong trường hợp này, chúng ta muốn biết điểm số cần đạt trong kĩ năng \lq\lq nói\rq\rq\, để đạt được điểm trung bình ít nhất là $6{,}25$. Điểm số trung bình mong muốn là trung bình của $6{,}5$; $6{,}5$; $5{,}5$.\\
		Theo đề bài, ta có bất phương trình
		\allowdisplaybreaks
		\begin{eqnarray*}
			\dfrac{6{,}5 + 6{,}5 + 5{,}5 + x}{4} &\geq& 6{,}25\\
			18{,}5 + x &\geq& 25\\
			x &\geq& 25 - 18{,}5\\
			x &\geq& 6{,}5.
		\end{eqnarray*}
		Vậy để đạt được điểm trung bình ít nhất là $6{,}25$, bạn Hà cần đạt ít nhất $6{,}5$ điểm trong kĩ năng \lq\lq nói\rq\rq.
	}
\end{bt}
\begin{bt}%[Dự án EX-9-Đề Cương Toán 9]%[Nguyễn Văn Cương 056]%[9D2V2-3]
	\immini{Chủ đầu tư khu chung cư Vạn Xuân muốn quy hoạch khu đất hình chữ nhật kích thước $50$ m $\times \,75$ m giữa các toà nhà bằng cách chia nó thành ba hình chữ nhật nhỏ $A$, $B$, $C$ như hình bên. Phần $A$ dùng để làm sân tập luyện thể thao (có thể chơi bóng rổ, bóng chuyền), phần $B$ dành để trồng cây xanh và phần $C$ là nơi đặt cầu trượt, bập bênh cho trẻ em. Chủ đầu tư muốn chia khu đất sao cho diện tích hình $A$ không nhỏ hơn diện tích hình $B$. Vậy cạnh $x$ của hình $A$ phải dài tối thiểu là bao nhiêu mét?}
	{\begin{tikzpicture}[scale=0.75,font=\footnotesize,>=stealth]
			\path (0,0) coordinate (A)
			(3.5,0) coordinate (B)
			(5,0) coordinate (C)
			(0,3) coordinate (D)
			(3.5,3) coordinate (E)
			(0,7.5) coordinate (F)
			(3.5,7.5) coordinate (G)
			(5,7.5) coordinate (H);
			\fill[draw=black,fill=white] (A)--(B)--(E)--(D)--cycle;
			\fill[draw=black,fill=red!40] (B)--(C)--(H)--(G)--cycle;
			\fill[draw=black,fill=yellow!50] (D)--(E)--(G)--(F)--cycle;
			\path (E)--(F) node[midway]{$A$};
			\path (G)--(C) node[midway]{$B$};
			\path (A)--(E) node[midway,below]{$C$};
			\draw[<->] ($(E)+(-90:0.2)$)--($(D)+(-90:0.2)$)node[midway,below]{$x$};
			\draw[<->] ($(H)+(0:0.2)$)--($(C)+(0:0.2)$)node[midway,right]{$75$};
			\draw[<->] ($(F)+(90:0.2)$)--($(H)+(90:0.2)$)node[midway,above]{$50$};
			\draw[<->] ($(F)+(180:0.2)$)--($(D)+(180:0.2)$)node[midway,left]{$45$};
			%\node[below] at(current bounding box.south){\textit{Hình 2.3}};
	\end{tikzpicture}}
	\loigiai{
		Diện tích của hình chữ nhật $A$ là $45\cdot x=45x$.\\
		Diện tích của hình chữ nhật $B$ là $(50-x)\cdot 75=3750-75x$.\\
		Để diện tích hình $A$ không nhỏ hơn diện tích hình $B$ nên ta có bất phương trình
		\begin{eqnarray*}
			\allowdisplaybreaks
			45x&\geq&3750-75x\\
			45x+75x&\geq&3750\\
			120x&\geq&3750\\
			x&\geq&3750:120\\
			x&\geq&31{,}25.
		\end{eqnarray*}
		Vậy cạnh $x$ của hình $A$ phải dài tối thiểu là $31{,}25$ mét.
	}
\end{bt}

\begin{bt}%[Dự án EX-9-Đề Cương Toán 9]%[Nguyễn Văn Cương 056]%[9D2V2-3]
	Trò chơi chọn số\\
	Cô giáo đưa ra hai cách tìm số mới từ một số $x$ được chọn ngẫu nhiên.
	\begin{itemize}
		\item Cách A: Lấy $x$ nhân với 3, được bao nhiêu đem cộng thêm $50$.
		\item Cách B: Lấy $x$ trừ đi 1, được bao nhiêu đem nhân với $5$.
	\end{itemize}
	Có hai đội chơi: Đội chọn cách A gọi là đội A, đội chọn cách B gọi là đội B.\\
	\textbf{Luật chơi:}\\
	Đội đi trước được quyền chọn số. Sau khi thảo luận để chọn một giá trị của $x$, đội chọn số thông báo số đã chọn. Với số $x$ đã được thông báo, mỗi đội sử dụng cách tính số của mình và cho biết kết quả.\\
	Nếu kết quả của đội chọn số lớn hơn kết quả của đối thủ thì đội chọn số thắng và được quyền chọn tiếp một giá trị khác của $x$.\\
	Trong trường hợp kết quả của đội chọn số nhỏ hơn hoặc bằng kết quả của đối thủ thì đội chọn số thua và phải nhường lượt chơi cho đối thủ.\\
	Nếu đội $A$ được quyền chọn số thì nên chọn như thế nào để đảm bảo thắng?\\
	Nếu đội $B$ được quyền chọn số thì nên chọn như thế nào để đảm bảo thắng?
	\loigiai{
		Số mới của đội $A$ là $x\cdot 3+50=3x+50$.\\
		Số mới của đội $B$ là $(x-1)\cdot 5=5x-5$.\\
		\begin{itemize}
			\item \textbf{Trường hợp 1:} Đội $A$ thắng, khi đó ta có bất phương trình 
			\begin{eqnarray*}\allowdisplaybreaks
				3x+50&>&5x-5\\
				-2x&>&-55\\
				x&>&27{,}5.
			\end{eqnarray*}
			Vậy đội $A$ nên chọn số nhỏ hơn hoặc bằng $27$ để đảm bảo thắng.
			\item \textbf{Trường hợp 2:} Đội $B$ thắng, khi đó ta có bất phương trình 
			\begin{eqnarray*}\allowdisplaybreaks
				5x-5&>&3x+50\\
				2x&>&55\\
				x&>&27{,}5.
			\end{eqnarray*}
			Vậy đội $B$ nên chọn số lớn hơn $27$ để đảm bảo thắng.
		\end{itemize}
	}
\end{bt}

\begin{bt}%[Dự án EX-9-Đề Cương Toán 9]%[Nguyễn Văn Cương 056]%[9D2V2-3]
	Bạn Quân cao $1{,}7$ m và nặng $75$ kg, được đánh giá thể trạng béo phì độ I. Theo yêu cầu của bác sĩ, Quân bắt đầu bơi lội và chạy bộ. Quân tiêu thụ trung bình $15$ calo cho mỗi phút bơi và $10$ calo cho mỗi phút chạy bộ. Hằng ngày, Quân dành $1{,}5$ giờ cho cả hai hoạt động trên và tiêu thụ hết $1\,150$ calo.
	\begin{enumerate}
		\item Hỏi mỗi ngày, Quân mất bao nhiêu thời gian cho mỗi hoạt động?
		\item Kết hợp tập luyện như trên với chế độ ăn theo tư vấn, mỗi tuần Quân giảm $0{,}5$ kg. Hỏi Quân sẽ đạt thể trạng bình thường sau ít nhất mấy tuần tuần theo chế độ ăn và tập luyện này? Biết rằng người thể trạng bình thường có chỉ số khối BMI từ $20$ đến dưới $25$, với chỉ số khối được tính theo công thức: $BMI = \dfrac{m}{h^2}$ ($m$ là khối lượng tính theo kilôgam, $h$ là chiều cao tính theo mét).
	\end{enumerate}
	\loigiai{
		\begin{enumerate}
			\item 
			Gọi $x$ (phút) là thời gian Quân mất ở hoạt động bơi lội trong một ngày $(0<x)$.\\
			Thời gian Quân mất ở hoạt động chạy bộ trong một ngày là $1{,}5\cdot60-x=90-x$ (phút).\\
			Mỗi ngày lượng calo tiêu thụ cho bơi lội là $15x$ calo.\\
			Mỗi ngày lượng calo tiêu thụ cho chạy bộ là $10(90-x)$ calo.\\
			Hằng ngày, Quân dành tiêu thụ hết $1150$ calo nên ta có phương trình\\
			\allowdisplaybreaks
			\begin{eqnarray*}
				15x+10(90-x) &=&1\,150\\
				15x+900-10x&=&1\,150 \\
				15x-10x&=&1\,150-900\\
				5x&=&250\\
				x&=&50 \text{ (nhận)}.
			\end{eqnarray*}
			Vậy mỗi ngày Quân bơi $50$ phút và chạy $40$ phút.
			\item 
			Chỉ số $BMI$ của Quân hiện tại là $BMI=\dfrac{m}{h^2}=\dfrac{75}{1{,7}^2}\approx25{,}95$.\\
			Gọi $x$ (tuần) là số tuần Quân đạt được thể trạng bình thường $(x\in \mathbb{N})$.\\
			Sau $x$ tuần, cân nặng của Quân là $75-0{,}5x$ (kg).\\
			Theo đề bài, ta có bất phương trình
			\allowdisplaybreaks
			\begin{eqnarray*}
				\dfrac{75-0{,5}x}{1{,}7^2}&<&25\\
				75-0{,}5x&<&72{,}25\\
				-0{,}5x&<&72{,}25-75\\
				-0{,}5x&<&-2{,}75\\
				x&>&5{,5}.
			\end{eqnarray*}
			Vì $x$ là số tuần nguyên, nên Quân cần ít nhất $6$ tuần để đạt thể trạng bình thường.
		\end{enumerate}
	}
\end{bt}

\begin{bt}%[Dự án EX-9-Đề Cương Toán 9]%[Nguyễn Văn Cương 056]%[9D2V2-3]
	Nhân dịp năm học mới mẹ quyết định mua cho An một chiếc xe đạp điện mới để đi học. Cửa hàng tư vấn cho mẹ An hai mẫu xe mới như sau
	\begin{itemize}
		\item Xe hiệu Spark với giá $17$ triệu đồng và mức điện năng tiêu thụ cho $1$ lần sạc đầy là $0{,}576$ kWh.
		\item Xe hiệu Yamaha với giá $18$ triệu đồng và mức điện năng tiêu thụ cho $1$ lần sạc đầy là $0{,}276$ kWh.
	\end{itemize}
	Giả sử $1$ năm An đi khoảng $6\,000$ km, cả hai xe nếu sạc đầy thì đi được tối đa $30$ km và tiền điện cho $1$ kWh là $3\,000$ đồng.
	\begin{enumerate}
		\item Gọi $S$ là chi phí từng năm theo thời gian $t$ (năm) của mỗi xe (gồm tiền mua xe và tiền điện). Lập hàm số $S$ theo $t$.
		\item Thời gian sử dụng là bao lâu thì mua xe Yamha sẽ tiết kiệm hơn?
	\end{enumerate}
	\loigiai{
		\begin{enumerate}
			\item Số tiền điện phải trả cho 1 năm của xe Spark là $\dfrac{6\,000\cdot 0{,}576\cdot 3\,000}{30}=345\,600$ (đồng).\\
			Hàm số $S$ theo $t$ của xe Spark là $S=17\,000\,000+345\,600t$.\\
			Số tiền điện phải trả cho $1$ năm của xe Yahmaha là $\dfrac{6\,000\cdot 0{,}276\cdot 3\,000}{30}=165\,600$ (đồng).\\
			Hàm số $S$ theo $t$ của xe Yamaha là $S=18\,000\,000+165\,600t$.
			\item Để sử dụng xe Yamha tiết kiệm hơn thì
			\allowdisplaybreaks
			\begin{eqnarray*}
				17\,000\,000+345\,600 t&>&18\,000\,000+165\,600 t\\
				345\,600 t-165\,600 t&>&18\,000\,000-17\,000\,000\\
				180\,000 t&>&1\,000\,000\\
				t&>&\dfrac{50}{9}\approx 5{,}6.
			\end{eqnarray*}
			Vậy thời gian sử dụng từ $6$ năm trở lên thì đi xe Yamaha sẽ tiết kiệm hơn.
		\end{enumerate}
	}
\end{bt}

\begin{bt}%[Dự án EX-9-Đề Cương Toán 9]%[Nguyễn Văn Cương 056]%[9D2V2-3]
	Một người đi taxi sẽ phải trả chi phí gồm: phí lúc mở cửa và cứ mỗi km di chuyển sẽ trả một số tiền cố định. Nhà Ông bà ngoại của Nam cách nhà Nam $32$ km. Biết rằng một chuyến đi $10$ km thì phải trả $109\,000$ đồng và một chuyến đi $6$ km thì phải trả $69\,000$ đồng.
	\begin{center}
		\begin{tikzpicture}[line join = round, line cap=round,>=stealth,font=\footnotesize,scale=1]
			\draw 
			(0,0) coordinate (A)
			(4,-2) coordinate (B)
			(8,0) coordinate (C)
			;
			\draw (A)node[left]{$A$}--(B)node[above]{$B$}--(C) node[right]{$C$}--cycle;
			\path (A)--(C) node[pos=0.5,above,scale=1.5,red]{\faCar};
			\path (A)--(B) node[pos=0.5,above,scale=1.5,red,rotate=-25]{\faCar};
			\path (C)--(B) node[pos=0.6,above,scale=1.5,red,rotate=25]{\faBus}
			(A) node[above,scale=1.5]{\faHome}
			(C) node[above,scale=1.5]{\faHome}
			(B) node[below,scale=1.5]{\faInstitution};
			\fill ($(B)+(0,-0.65)$) node[below]{Trạm xe buýt};
		\end{tikzpicture}
	\end{center}
	\begin{enumerate}
		\item Nam muốn về thăm Ông bà ngoại bằng cách đi xe taxi từ nhà. Hỏi Nam phải trả bao nhiêu tiền cho chuyến đi?
		\item Để giảm chi phí, Nam tính toán cách di chuyển từ nhà đến nhà Ông bà ngoại như sau: Nam đi taxi đến trạm xe buýt, rồi sau đó đi xe buýt theo tuyến đường đến nhà Ông bà ngoại. Biết giá vé xe buýt là $50\,000$ đồng. Hỏi trạm xe buýt cách nhà Nam bao xa thì với cách di chuyển thứ hai sẽ ít tốn chi phí hơn?
	\end{enumerate}
	\loigiai{
		\begin{enumerate}
			\item  Gọi $x$ (nghìn đồng) là phí mở cửa $(x>0)$ và
			$y $ (nghìn đồng) là số tiền trả $1$ km di chuyển $(y>0)$.\\
			Vì một chuyến đi $10$ km thì phải trả $109\,000$ đồng nên ta có phương trình 
			$$x + 10y = 109. \qquad (1)$$
			Vì một chuyến đi $6$ km thì phải trả $69\,000$ đồng nên ta có phương trình $x + 6y = 69$. $\qquad (2)$\\
			Từ $(1)$ và $(2)$, ta có hệ phương trình $\heva{&x + 10y = 109\\ &x + 6y = 69.}$\\			
			Giải hệ phương trình, ta được $\heva{&x=9\\&y=10}$ (thỏa mãn điều kiện).\\
			Vậy khi di chuyển trên quãng đường $32$ km thì Nam phải trả số tiền là $9 + 10\cdot 32 = 329$ (nghìn đồng).
			\item Gọi $a$ (km) là số km từ nhà Nam đến trạm xe buýt ($a > 0$).\\
			Số tiền Nam phải trả khi di chuyển bằng cách thứ hai là
			$9 + 10\cdot a + 50 = 59 + 10a$ (nghìn đồng).\\
			Để di chuyển theo cách thứ hai ít tốn chi phí hơn thì
			\allowdisplaybreaks
			\begin{eqnarray*}
				59 + 10a &<& 329 \\
				10a &<& 270\\
				a &<& 27 \textrm{ (thỏa mãn điều kiện).}
			\end{eqnarray*}
			Vậy nếu trạm xe buýt cách nhà Nam ít hơn $27$ km thì với cách di chuyển thứ hai sẽ ít tốn chi phí hơn.
		\end{enumerate}		
	}
\end{bt} 

\begin{bt}%[Dự án EX-9-Đề Cương Toán 9]%[Nguyễn Văn Cương 056]%[9D2V2-3]
	Formalin là dung dịch có chứa từ $37\%$ đến $40\%$ Formaldehyde. Formaldehyde có khả năng kháng khuẩn, kháng nấm nên được dùng làm chất bảo quản trong y tế. Một nhà máy sản xuất Formaldehyde đang có một lượng dung dịch Formaldehyde nồng độ $15\%$ và một lượng Formaldehyde nồng độ $65\%$.
	\begin{enumerate}
		\item Tính thể tích mỗi loại Formaldehyde trên để điều chế được $300$ lít Formaldehyde $35\%$. Giả sử nguyên liệu không bị hao hụt trong quá trình sản xuất.
		\item Một cơ sở y tế đặt hàng nhà máy trên một đơn hàng Formalin. Nhà máy dùng $200$ lít Formaldehyde $15\%$ cùng một lượng Formaldehyde $65\%$ để sản xuất ra Formalin. Hỏi thể tích của Formaldehyde $65\%$ nằm trong khoảng nào thì có thể sản xuất được Formalin.
	\end{enumerate}
	\loigiai{
		\begin{enumerate}
			\item Gọi $x$, $y$ (lít) lần lượt là thể tích Formaldehyde nồng độ $15 \%$ và Formaldehyde nồng độ $65 \%$ để điều chế được $300$ lít Formaldehyde $35\%$ ($x, y > 0$).\\
			Vì tổng thể tích của hai dung dịch là $300$ lít nên $x+y=300$. $\qquad (1)$\\
			Vì nồng độ của dung dịch cần điều chế là $35\%$ nên 
			$$15\% x + 65\% y = 35\%\cdot 300 \text{ hay } 0{,}15x+0{,}65y=105. \qquad (2)$$
			Từ $(1)$ và $(2)$, ta có hệ phương trình $\heva{&x+y=300\\&0{,}15x+0{,}65y=105.}$\\
			Giải hệ phương trình trên, ta được $\heva{&x = 180\\&y = 120}$ (thỏa mãn điều kiện).\\
			Vậy cần dùng $180$ lít Formaldehyde nồng độ $15\%$ và $120$ lít Formaldehyde nồng độ $65\%$ để điều chế được $300$ lít Formaldehyde $35\%$.
			\item Gọi $x$ là thể tích của lượng Formaldehyde $65\%$ có thể dùng để sản xuất được Formalin ($x > 0$).\\
			Nồng độ Formaldehyde có trong dung dịch sản xuất ra là 
			\[\dfrac{200\cdot 15\%+x\cdot65\%}{200+x}\cdot 100\%\]
			Vì Formalin là dung dịch có chứa từ $37\%$ đến $40\%$ Formaldehyde. Do đó, để sản xuất ra Formalin thì ta phải có điều kiện
			\allowdisplaybreaks
			\begin{eqnarray*}
				37\% \leq \dfrac{200\cdot 15\%+x\cdot65\%}{200+x}\cdot 100\% \leq 40\%\\
				0{,}37\cdot (200+x) \leq 30+0{,}65x \leq 0{,}4\cdot (200+x)
			\end{eqnarray*}
			Suy ra 
			\allowdisplaybreaks
			\begin{eqnarray*}
				&0{,}37\cdot (200+x) \leq 30+0{,}65x \text{ và } 30+0{,}65x \leq  0{,}4\cdot (200+x)\\
				&x \geq \dfrac{1\,100}{7} \text{ và } x \leq 200.
			\end{eqnarray*}
			Vậy lượng Formaldehyde $65\%$ nằm trong khoảng từ $\dfrac{1\,100}{7}$ lít đến $200$ lít thì có thể sản xuất được Formalin.
		\end{enumerate}
	}
\end{bt}
\begin{dang}{Bài toán nâng cao}
	
\end{dang}
\begin{bt}%[Dự án EX-9-Đề Cương Toán 9]%[Nguyễn Văn Cương 056]%[9D2C2-2]
	Tìm $x$, sao cho
	\begin{multicols}{3}
		\begin{enumerate}
			\item $(x-1) x<0$;
			\item $(x-2)(x-5)>0$;
			\item $\dfrac{x+2}{x-3}>0$.
		\end{enumerate}
	\end{multicols}
	\loigiai{
		\begin{enumerate}
			\item 
			Trường hợp $1$: $x>0$ và $x-1<0 $ suy ra $ x>0$ và $x<1 $ suy ra $ 0<x<1$.\\
			Trường hợp $2$: $x<0$ và $x-1>0 $ suy ra $ x<0$ và $x>1$ (không tồn tại $x$).\\
			Vậy nghiệm của bất phương trình là $0<x<1$.\\
			\item 
			Trường hợp $1$: $x-2>0$ và $x-5>0  $ suy ra $ x>2$ và $x>5  $ suy ra $ x>5$.\\
			Truờng hợp $2$: $x-2<0$ và $x-5<0  $ suy ra $ x<2$ và $x<5  $ suy ra $ x<2$ hay $-2<x<3$.\\
			Vậy nghiệm của bất phương trình là $x>5$ hoặc $x<2$.\\
			\item 
			Trường hợp $1$: $x+2<0$ và $x-3>0 $ suy ra $ x<-2$ và $x>3$ (không tồn tại $x$).\\
			Trường hợp $2$: $x+2>0$ và $x-3<0 $ suy ra $ x>-2$ và $x<3$. \\
			Vậy nghiệm của bất phương trình là $-2<x<3$.\\
		\end{enumerate}
	}
\end{bt}

\begin{bt}%[Dự án EX-9-Đề Cương Toán 9]%[Nguyễn Văn Cương 056]%[9D2C2-2]
	\textbf{}
	\begin{enumerate}
		\item Tìm $m$ để phương trình $x-3=2 m+4$ có nghiệm dương?
		\item Tìm $m$ để phương trình $2 x-5=m+8$ có nghiệm âm?
	\end{enumerate}
	\loigiai{
		\begin{enumerate}
			\item 	Ta có $x-3=2 m+4 $ suy ra $x=2 m+7$.\\
			Xét
			\allowdisplaybreaks
			\begin{eqnarray*}
				x&>&0  \\
				2m+7&>&0 \\
				2m&>&-7\\
				m&>&\dfrac{-7}{2}.
			\end{eqnarray*}	
			\item Ta có $2 x-5=m+8 $ suy ra $ 2x=m+13 $ suy ra $ x=\dfrac{m+13}{2}$.\\
			Xét
			\allowdisplaybreaks
			\begin{eqnarray*}
				x &<&0  \\
				\dfrac{m+13}{2}&<&0 \\
				m+13&<&0\\
				m&<&-13.
			\end{eqnarray*}
		\end{enumerate}
	}
\end{bt}
\begin{bt}%[Dự án EX-9-Đề Cương Toán 9]%[Nguyễn Văn Cương 056]%[9D2C2-2]
	Giải và biện luận bất phương trình $m(x-m)>x-1$. (*)
	\loigiai{
		Ta có $m(x-m)>x-1 $ suy ra $ m x-m^{2}>x-1 $ suy ra $ m x-x>m^{2}-1$.\\
		Do đó $(m-1)x>m^2-1$. \qquad $(**)$\\
		Truờng hợp $1$: $m-1>0 $ suy ra $ m>1.$\\
		Ta có $(**) $ suy ra $ x>\frac{m^{2}-1}{m-1}  $ suy ra $ x>m+1$.\\
		Truờng hơp $2$: $m-1=0  $ suy ra $ m=1$.\\
		Ta có $(**) $ suy ra $ 0 x>0$ (không tồn tại $x$).\\
		Trường hợp $3$: $m-1<0  $ suy ra $ m<1$.\\
		Ta có $(**) $ suy ra $ x<\dfrac{m^2-1}{m-1} $ suy ra $ x<m+1$.\\
		Vậy với $ m>1$ thì  nghiệm của bất phương trình $(*)$ có dạng là $S=\{x \mid x>m+1\}$.\\
		Với $m=1$ thì  nghiệm của bất phương trình $(*)$ có dạng là $S=\varnothing$.\\
		Với $m<1$thì  nghiệm của bất phương trình $(*)$ có dạng là $S=\{x \mid x<m+1\}$.
	}
\end{bt}



