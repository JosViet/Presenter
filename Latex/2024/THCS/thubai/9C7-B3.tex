\section{BIỂU DIỄN SỐ LIỆU GHÉP NHÓM} % Tên bài
\subsection{Bảng tần số, tần số tương đối ghép nhóm}
\subsubsection{Kiến thức trọng tâm}
\begin{tomtat}
	\begin{itemize}
		\item Trong nhiều trường hợp, ta khó biết giá trị chính xác của số liệu mà chỉ biết nó thuộc một nhóm số liệu nào đó. Khi đó, ta ghép các giá trị lại thành nhóm.
		\item Nhóm số liệu thường có dạng $[a; b)$ gồm các số liệu lớn hơn hoặc bằng $a$ và nhỏ hơn $b$.
		\item Bảng tần số ghép nhóm là bảng tần số của các nhóm số liệu
		\begin{center}
			\begin{tabular}{|c|c|c|c|c|}
				\hline
				Nhóm & $[a_1; a_2)$ & $[a_2; a_3)$ & $\ldots$ & $[a_k; a_{k+1})$ \\
				\hline
				Tần số & $m_1$ & $m_2$ & $\ldots$ & $m_k$ \\
				\hline
			\end{tabular}
		\end{center}
		\item Số $m_i$ của nhóm $[a_i; a_{i+1})$ là số giá trị của mẫu số liệu lớn hơn hoặc bằng $a_i$ và nhỏ hơn $a_{i+1}$.
		\item Tần số tương đối $f_i$ của nhóm $[a_i; a_{i+1})$ là
		$$f_i = \dfrac{m_i}{n} \times 100\%$$
		với $n = m_1 + m_2 + \cdots + m_k$ là tổng số dữ liệu.
		\item Bảng tần số tương đối ghép nhóm là bảng tần số tương đối của các nhóm số liệu
		\begin{center}
			\begin{tabular}{|c|c|c|c|c|}
				\hline
				Nhóm & $[a_1; a_2)$ & $[a_2; a_3)$ & $\ldots$ & $[a_k; a_{k+1})$ \\
				\hline
				Tần số tương đối & $f_1$ & $f_2$ & $\ldots$ & $f_k$ \\
				\hline
			\end{tabular}
		\end{center}
	\end{itemize}
\end{tomtat}

\begin{vd}%[Dự án EX-9-Đề Cương Toán 9]%[Mai Sương]%[9D5H3-2]
	Bảng sau ghi lại khối lượng (đơn vị: kg) của $30$ quả trứng đà điểu ở một trang trại.
	\begin{center}
		\begin{tabular}{|c|c|c|c|c|c|c|c|c|c|}
			\hline
			$1{,}57$ & $1{,}67$ & $1{,}54$ & $1{,}64$ & $1{,}62$ & $1{,}53$ & $1{,}4$  & $1{,}78$ & $1{,}34$ & $1{,}3$ \\
			\hline
			$1{,}72$ & $1{,}46$ & $1{,}56$ & $1{,}78$ & $1{,}6$  & $1{,}45$ & $1{,}46$ & $1{,}64$ & $1{,}75$ & $1{,}33$ \\
			\hline
			$1{,}31$ & $1{,}56$ & $1{,}34$ & $1{,}49$ & $1{,}66$ & $1{,}41$ & $1{,}45$ & $1{,}36$ & $1{,}38$ & $1{,}52$ \\
			\hline
		\end{tabular}
	\end{center}
	\begin{enumerate}
		\item Hãy chia số liệu trên thành $5$ nhóm là $[1{,}3; 1{,}4)$; $[1{,}4; 1{,}5)$; $[1{,}5; 1{,}6)$; $[1{,}6; 1{,}7)$; $[1{,}7; 1{,}8)$. Tìm tần số và tần số tương đối của mỗi nhóm đó.
		\item Lập bảng tần số ghép nhóm và bảng tần số tương đối ghép nhóm mẫu số liệu ghép nhóm đó.
	\end{enumerate}
	\loigiai{
		\begin{enumerate}
			\item Tần số của các nhóm $[1{,}3; 1{,}4)$; $[1{,}4; 1{,}5)$; $[1{,}5; 1{,}6)$; $[1{,}6; 1{,}7)$; $[1{,}7; 1{,}8)$ lần lượt là $m_1 = 7$; $m_2 = 7$; $m_3 = 6$; $m_4 = 6$; $m_5 = 4$.\\
			Tần số tương đối của các nhóm $[1{,}3; 1{,}4)$; $[1{,}4; 1{,}5)$; $[1{,}5; 1{,}6)$; $[1{,}6; 1{,}7)$; $[1{,}7; 1{,}8)$ lần lượt là
			$f_1 = \dfrac{7}{30} \cdot 100\% \approx 23{,}33\%$,  
			$f_2 = \dfrac{7}{30} \cdot 100\% \approx 23{,}33\%$,  
			$f_3 = \dfrac{6}{30} \cdot 100\% = 20\%$,
			$f_4 = \dfrac{6}{30} \cdot 100\% = 20\%$,  
			$f_5 = \dfrac{4}{30} \cdot 100\% \approx 13{,}33\%.$
			\item Bảng tần số ghép nhóm
			\begin{center}
				\begin{tabular}{|c|c|c|c|c|c|}
					\hline
					\textbf{Khối lượng (kg)} & $[1{,}3; 1{,}4)$ & $[1{,}4; 1{,}5)$ & $[1{,}5; 1{,}6)$ & $[1{,}6; 1{,}7)$ & $[1{,}7; 1{,}8)$ \\
					\hline
					\textbf{Tần số} & $7$ & $7$ & $6$ & $6$ & $4$ \\
					\hline
				\end{tabular}
			\end{center}
			
			Bảng tần số tương đối ghép nhóm
			\begin{center}
				\begin{tabular}{|c|c|c|c|c|c|}
					\hline
					\textbf{Khối lượng (kg)} & $[1{,}3; 1{,}4)$ & $[1{,}4; 1{,}5)$ & $[1{,}5; 1{,}6)$ & $[1{,}6; 1{,}7)$ & $[1{,}7; 1{,}8)$ \\
					\hline
					\textbf{Tần số tương đối} & $23{,}33\%$ & $23{,}33\%$ & $20\%$ & $20\%$ & $13{,}33\%$ \\
					\hline
				\end{tabular}
			\end{center}
		\end{enumerate}
	}
\end{vd}

\begin{vd}%[Dự án EX-9-Đề Cương Toán 9]%[Mai Sương]%[9D5N3-2]
	Cho bảng tần số ghép nhóm sau về tuổi thọ của một số ong mật cái như sau
	\begin{center}
		\begin{tabular}{|c|c|c|c|}
			\hline
			Tuổi thọ (ngày) & $[30;40)$ & $[40;50)$ & $[50;60)$ \\
			\hline
			Tần số & $14$ & $24$ & $22$ \\
			\hline
		\end{tabular}
	\end{center}
	Tìm tần số ghép nhóm và tần số tương đối ghép nhóm của nhóm $[40;50)$.
	\loigiai{
		Tổng số ong mật cái được khảo sát là $14 + 24 + 22 = 60$.\\
		Tần số ghép nhóm của nhóm $[40;50)$ là $24$.\\
		Tần số tương đối của nhóm $[40;50)$ là $\dfrac{24}{60} \cdot 100\% = 40\%$.
	}
\end{vd}

\subsubsection{Bài tập}
\begin{bt}%[Dự án EX-9-Đề Cương Toán 9]%[Mai Sương]%[9D5N3-2]
	Kết quả khảo sát thời gian sử dụng liên tục (đơn vị: giờ) từ lúc sạc đầy cho đến khi hết pin của một số máy vi tính cùng loại được thống kê lại ở bảng sau
	\begin{center}
		\begin{tabular}{|c|c|c|c|c|}
			\hline
			Thời gian (giờ) & $[7{,}2;7{,}4)$ & $[7{,}4;7{,}6)$ & $[7{,}6;7{,}8)$ & $[7{,}8;8{,}0)$ \\
			\hline
			Số máy & $2$ & $4$ & $7$ & $7$ \\
			\hline
		\end{tabular}
	\end{center}
	\begin{enumerate}
		\item Tìm số lượng máy vi tính có thời gian sử dụng từ $7{,}4$ đến dưới $7{,}8$ giờ.
		\item Tìm tần số tương đối của số lượng máy tính có thời gian sử dụng từ $7{,}8$ đến $8$ giờ.
	\end{enumerate}
	\loigiai{
		\begin{enumerate}
			\item Tổng số máy được khảo sát là $2 + 4 + 7 + 7 = 20$.\\
			Số máy có thời gian sử dụng từ $7{,}4$ đến dưới $7{,}8$ giờ là $4 + 7 = 11$.
			\item Tần số tương đối của nhóm $[7{,}8;8{,}0)$ là $\dfrac{7}{20} \cdot 100\% = 35\%$.
		\end{enumerate}
	}
\end{bt}
\begin{bt}%[Dự án EX-9-Đề Cương Toán 9]%[Mai Sương]%[9D5V3-2]
	Một công ty tổ chức thi tuyển kĩ thuật viên mới. Thời gian hoàn thành một bài thực hành của các ứng cử viên được ghi lại trong bảng sau (đơn vị: giây)
	\begin{center}
		\begin{tabular}{|c|c|c|c|c|c|c|c|c|c|}
			\hline
			$507$ & $408$ & $333$ & $436$ & $483$ & $462$ & $348$ & $352$ & $433$ & $313$ \\
			\hline
			$428$ & $420$ & $327$ & $458$ & $445$ & $495$ & $481$ & $470$ & $394$ & $462$ \\
			\hline
			$390$ & $433$ & $429$ & $478$ & $491$ & $363$ & $525$ & $365$ & $514$ & $352$ \\
			\hline
		\end{tabular}
	\end{center}
	\begin{enumerate}
		\item Lập bảng tần số ghép nhóm cho mẫu số liệu trên với 5 nhóm sau: $[300;350)$; $[350;400)$; $[400;450)$; $[450;500)$; $[500;550)$.
		\item Người ta sẽ loại $40\%$ ứng cử viên có thời gian làm bài thi lâu nhất. Hỏi các thí sinh có thời gian hoàn thành bài thi trên với bao nhiêu giây sẽ bị loại?
	\end{enumerate}
	\loigiai{
		\begin{enumerate}
			\item Bảng tần số ghép nhóm
			\begin{center}
				\renewcommand{\arraystretch}{1.2}
				\begin{tabular}{|c|c|c|c|c|c|}
					\hline
					Thời gian (giây) & $[300;350)$ & $[350;400)$ & $[400;450)$ & $[450;500)$ & $[500;550)$ \\
					\hline
					Tần số & $4$ & $6$ & $8$ & $9$ & $3$ \\
					\hline
				\end{tabular}
			\end{center}
			\item Tần số tương đối của nhóm $[500;550)$ là $\dfrac{3}{30} \cdot 100\% = 10\%$.\\
			Tần số tương đối của nhóm $[450;500)$ là $\dfrac{9}{30} \cdot 100\% = 30\%$.\\
			Vì $30\% + 10\% = 40\%$\\
			Vậy các ứng cử viên có thời gian hoàn thành bài thi từ $450$ giây trở lên sẽ bị loại.
		\end{enumerate}
	}
\end{bt}

\begin{bt}%[Dự án EX-9-Đề Cương Toán 9]%[Mai Sương]%[9D5H3-2]
	Sau khi điều tra về số học sinh trong $100$ lớp học (đơn vị: học sinh), người ta có bảng tần số ghép nhóm như ở bảng sau
	\begin{center}
		\begin{tabular}{|c|c|c|c|c|c|}
			\hline
			Nhóm & $[36 ; 38)$ & $[38 ; 40)$ & $[40 ; 42)$ & $[42 ; 44)$ & $[44 ; 46)$ \\
			\hline
			Tần số & $20$ & $15$ & $25$ & $30$ & $10$ \\
			\hline
		\end{tabular}
	\end{center}
	Hãy lập bảng tần số tương đối ghép nhóm của mẫu số liệu đó.
	\loigiai{
		Tần số tương đối của các nhóm $[36;38)$, $[38;40)$, $[40;42)$, $[42;44)$, $[44;46)$ lần lượt là
		\allowdisplaybreaks
		\begin{eqnarray*}
			f_1 &=& \dfrac{20}{100} \cdot 100\% = 20\%;\\ 
			f_2 &=& \dfrac{15}{100} \cdot 100\% = 15\%;\\ 
			f_3 &=& \dfrac{25}{100} \cdot 100\% = 25\%;\\
			f_4 &=& \dfrac{30}{100} \cdot 100\% = 30\%;\\ 
			f_5 &=& \dfrac{10}{100} \cdot 100\% = 10\%.
		\end{eqnarray*}
		Bảng tần số tương đối ghép nhóm
		\begin{center}
			\begin{tabular}{|c|c|c|c|c|c|}
				\hline
				Nhóm & $[36;38)$ & $[38;40)$ & $[40;42)$ & $[42;44)$ & $[44;46)$ \\
				\hline
				Tần số tương đối & $20\%$ & $15\%$ & $25\%$ & $30\%$ & $10\%$ \\
				\hline
			\end{tabular}
		\end{center}
	}
\end{bt}

\begin{bt}%[Dự án EX-9-Đề Cương Toán 9]%[Mai Sương]%[9D5V3-2]
	Độ dài một cú nhảy ba bước (đơn vị: m) của $40$ học sinh lớp $9$ được ghi lại ở bảng tần số ghép nhóm sau
	\begin{center}
		\begin{tabular}{|c|c|c|c|c|c|}
			\hline
			\textbf{Độ dài (m)} & $[8;9)$ & $[9;10)$ & $[10;11)$ & $[11;12)$ & $[12;13)$ \\
			\hline
			\textbf{Tần số} & $18$ & $10$ & $6$ & $4$ & $2$ \\
			\hline
		\end{tabular}
	\end{center}
	\begin{enumerate}
		\item Tìm tần số tương đối của mỗi nhóm.
		\item Lập bảng tần số tương đối ghép nhóm cho mẫu số liệu trên.
		\item Một giáo viên thể dục muốn chọn ra $15\%$ học sinh có thành tích nhảy ba bước tốt nhất. Hỏi giáo viên đó nên chọn các học sinh có độ dài bước nhảy tối thiểu là bao nhiêu mét?
	\end{enumerate}
	\loigiai{
		\begin{enumerate}
			\item Tần số tương đối của các nhóm $[8; 9); [9; 10); [10; 11); [11; 12); [12; 13)$ lần lượt là 
			\allowdisplaybreaks
			\begin{eqnarray*}
				f_1 &=& \dfrac{18}{40} \cdot 100\% = 45\%;\\
				f_2 &=& \dfrac{10}{40} \cdot 100\% = 25\%;\\
				f_3 &=& \dfrac{6}{40} \cdot 100\% = 15\%;\\
				f_4 &=& \dfrac{4}{40} \cdot 100\% = 10\%;\\
				f_5 &=& \dfrac{2}{40} \cdot 100\% = 5\%.
			\end{eqnarray*}
			\item Bảng tần số tương đối ghép nhóm
			\begin{center}
				\begin{tabular}{|c|c|c|c|c|c|}
					\hline
					\textbf{Độ dài (m)} & $[8; 9)$ & $[9; 10)$ & $[10; 11)$ & $[11; 12)$ & $[12; 13)$ \\
					\hline
					\textbf{Tần số tương đối} & $45\%$ & $25\%$ & $15\%$ & $10\%$ & $5\%$ \\
					\hline
				\end{tabular}
			\end{center}
			\item Hai nhóm $[11; 12$); $[12; 13)$ có tổng tần số tương đối là $10\%+5\%=15\%$.\\
			Do đó, giáo viên nên chọn học sinh có thành tích nhảy $3$ bước tối thiểu là $11$ mét.
		\end{enumerate}
		
	}
\end{bt}

\begin{bt}%[SGK CTST Toán 9]%[Dự án EX-9-Đề Cương Toán 9]%[9D5H3-2]
	Bạn Giang ghi lại cự li nhảy xa của các bạn trong câu lạc bộ thể thao ở bảng sau (đơn vị: mét)
	\begin{center}
		\begin{tabular}{|c|c|c|c|c|c|c|c|}
			\hline
			$5{,}4$ & $3{,}6$ & $ 4{,}7 $ & $ 4{,}2 $ & $ 4{,}4 $ & $ 4{,}8 $ & $ 3{,}7 $ & $ 4{,}7 $ \\
			\hline
			$ 4{,}2 $ & $ 3{,}8 $ & $ 4{,}2 $ & $ 4{,}4 $ & $ 4{,}6 $ & $ 4{,}8 $ & $ 5{,}3 $ & $ 4{,}7 $ \\
			\hline
			$ 5{,}4 $ & $ 4{,}1 $ & $ 3{,}5 $ & $ 4{,}7 $ & $ 5{,}1 $ & $ 4{,}1 $ & $ 4{,}4 $ & $ 5{,}4 $ \\
			\hline
			$ 4{,}5 $ & $ 5{,}4 $ & $ 4{,}4 $ & $ 4{,}3 $ & $ 3{,}6 $ & $ 4{,}4 $ & $ 4{,}8 $ & $ 4{,}8 $ \\
			\hline
		\end{tabular}
	\end{center}
	\begin{enumerate}
		\item Để thu gọn bảng dữ liệu thì nên chọn bảng tần số không ghép nhóm hay bảng tần số ghép nhóm để biểu thị dữ liệu trên? Tại sao?
		\item Hãy chia số liệu thành $4$ nhóm, trong đó nhóm đầu tiên là cự li từ $3{,}5$m đến dưới $4$m; lập bảng tần số và tần số tương đối ghép nhóm.
	\end{enumerate}
	\loigiai{
		\begin{enumerate}
			\item Để thu gọn bảng dữ liệu thì nên chọn bảng tần số ghép nhóm để biểu thị dữ liệu trên vì các nhóm số liệu có độ rộng bằng nhau, thuận tiện cho việc tính toán và phù hợp với mục đích của việc thống kê.	
			\item Chia số liệu thành $4$ nhóm, trong đó nhóm đầu tiên là cự li từ $3{,}5$m đến dưới $4$m ta được các nhóm là $[3{,}5 ; 4)$, $[4 ; 4{,}5)$, $[4{,}5 ; 5)$, $[5 ; 5{,}5)$.\\
			Bảng tần số là
			\begin{center}
				\begin{tabular}{|c|c|c|c|c|}
					\hline
					\makecell [c] {Cự li \\ (m)} & $[3{,}5 ; 4)$ & $[4 ; 4{,}5)$ & $[4{,}5 ; 5)$ & $[5 ; 5{,}5)$ \\
					\hline
					Số bạn & $5$ & $11$ & $10$ & $6$ \\
					\hline
				\end{tabular}
			\end{center}
			\begin{itemize}
				\item Tần số tương đối của nhóm $[3{,}5 ; 4)$ là $\dfrac{5}{32} \cdot 100\% \approx 15{,}6\%$.
				\item Tần số tương đối của nhóm $[4 ; 4{,}5)$ là $\dfrac{11}{32} \cdot 100\% \approx 34{,}4\%$.
				\item Tần số tương đối của nhóm $[4{,}5 ; 5)$ là $\dfrac{10}{32} \cdot 100\% \approx 31{,}3\%$.
				\item Tần số tương đối của nhóm $[5 ; 5{,}5)$ là $\dfrac{6}{32} \cdot 100\% \approx 18{,}7\%$.
			\end{itemize}
			Bảng tần số tương đối ghép nhóm là
			\begin{center}
				\begin{tabular}{|c|c|c|c|c|}
					\hline
					\makecell [c] {Cự li\\ (m)} & $[3{,}5 ; 4)$ & $[4 ; 4{,}5)$ & $[4{,}5 ; 5)$ & $[5 ; 5{,}5)$ \\
					\hline
					Số bạn & $15{,}6\%$ & $34{,}4\%$ & $31{,}3\%$ & $18{,}7\%$ \\
					\hline
				\end{tabular}
			\end{center}
		\end{enumerate}
	}
\end{bt}

\begin{bt}%[SGK CTST Toán 9]%[Dự án EX-9-Đề Cương Toán 9]%[9D5H3-2]
	Kết quả đo tốc độ của $25$ xe ô tô (đơn vị: km/h) khi đi qua một trạm quan sát được ghi lại ở bảng sau:
	\begin{center}
		\begin{tabular}{|c|c|c|c|c|c|c|c|c|c|c|c|c|}
			\hline
			$48{,}6$ & $54{,}2$ & $53{,}3$ & $45{,}3$ & $48{,}2$ & $46{,}3$ & $57{,}4$ & $62{,}6$ & $61{,}4$ & $55$ & $40{,}9$ & $45{,}5$ & $54{,}3$ \\
			\hline
			$49{,}8$ & $60$ & $58{,}9$ & $53$ & $53$ & $62$ & $49{,}4$ & $48{,}4$ & $47{,}8$ & $41{,}2$ & $42{,}8$ & $48{,}8$ & {}\\
			\hline
		\end{tabular}
	\end{center}
	\begin{enumerate}
		\item Hãy lập bảng tần số tương đối ghép nhóm cho bảng số liệu trên, trong đó nhóm đầu tiên là các xe có tốc độ từ $40$km/h đến dưới $45$km/h.
		\item Hãy xác định nhóm có tần số tương đối cao nhất và nhóm có tần số tương đối thấp nhất.
	\end{enumerate}
	\loigiai{
		\begin{enumerate}
			\item Chia số liệu thành $5$ nhóm, trong đó nhóm đầu tiên là các xe có tốc độ từ $40$km/h đến dưới $45$km/h, ta được các nhóm là $[40 ; 45)$, $[45 ; 50)$, $[50 ; 55)$, $[55 ; 60)$, $[60 ; 65)$.
			\begin{itemize}
				\item Tần số tương đối của nhóm $[40 ; 45)$ là $\dfrac{3}{25} \cdot 100\%=12\%$.
				\item Tần số tương đối của nhóm $[45 ; 50)$ là $\dfrac{10}{25} \cdot 100\%=40\%$.
				\item Tần số tương đối của nhóm $[50 ; 55)$, là $\dfrac{5}{25} \cdot 100\%=20\%$.
				\item Tần số tương đối của nhóm $[55 ; 60)$ là $\dfrac{3}{25} \cdot 100\%=12\%$.
				\item Tần số tương đối của nhóm $[60 ; 65)$ là $\dfrac{4}{25} \cdot 100\%=16\%$.
			\end{itemize}		
			Bảng tần số tương đối ghép nhóm cho bảng số liệu trên là
			\begin{center}
				\begin{tabular}{|c|c|c|c|c|c|}
					\hline
					\makecell [c] {Tốc độ\\ (km/h)} & $[40 ; 45)$ & $[45 ; 50)$ & $[50 ; 55)$ & $[55 ; 60)$ & $[60 ; 65)$\\
					\hline
					Số xe & $12\%$ & $40\%$ & $20\%$ & $12\%$ & $16\%$\\
					\hline
				\end{tabular}
			\end{center}
			\item Nhóm có tần số tương đối cao nhất là $[45 ; 50)$. Nhóm có tần số tương đối thấp nhất là $[40 ; 45)$ và $[55 ; 60)$.
		\end{enumerate}
	}
\end{bt}

\begin{bt}%[SGK CTST Toán 9]%[Dự án EX-9-Đề Cương Toán 9]%[9D5H3-2]
	Thời gian hoàn thành một bài kiểm tra trực tuyến của một số học sinh được ghi lại ở bảng sau (đơn vị: phút)
	\begin{center}
		\begin{tabular}{|c|c|c|c|}
			\hline
			\makecell[c] {Thời gian\\ (phút)} & $[10; 12)$ & \quad$?$\quad\quad & $[14; 16)$ \\
			\hline
			Tần số & $25$ & $?$ & $5$ \\
			\hline
			Tần số tương đối & $?$ & $?$ & $12{,}5 \%$ \\
			\hline
		\end{tabular}
	\end{center}
	\begin{enumerate}
		\item Hãy xác định số học sinh tham gia kiểm tra.
		\item Hoàn thành bảng trên vào vở.
	\end{enumerate}
	\loigiai{
		\begin{enumerate}
			\item Số học sinh tham gia kiểm tra là $\dfrac{5}{12{,}5\%} \cdot 100\% = 40$ (học sinh).
			\item Tần số tương đối của nhóm $[10; 12)$ là $\dfrac{25}{40} \cdot 100\% = 62{,}5\%$.\\
			Tần số của nhóm $[12; 14)$ là $40 - (25+5)=10$.\\
			Tần số tương đối của nhóm $[12; 14)$ là $\dfrac{10}{40} \cdot 100\% = 25\%$.\\
			Hoàn thành bảng
			\begin{center}
				\begin{tabular}{|c|c|c|c|}
					\hline
					\makecell[c] {Thời gian\\ (phút)} & $[10; 12)$ & $[12 ; 14)$ & $[14; 16)$ \\
					\hline
					Tần số & $25$ & $10$ & $5$ \\
					\hline
					Tần số tương đối & $62{,}5\%$ & $25\%$ & $12{,}5 \%$ \\
					\hline
				\end{tabular}
			\end{center}	
		\end{enumerate}
	}
\end{bt}

\subsection{Biểu đồ tần số tương đối ghép nhóm dạng cột}
\subsubsection{Kiến thức trọng tâm}
\begin{tomtat}
	\begin{itemize}
		\item Là biểu đồ gồm các cột liền nhau, chiều cao biểu diễn tần số tương đối.
		\item Biểu đồ chỉ phù hợp khi các lớp có độ dài bằng nhau.
		\item Các bước vẽ biểu đồ cột
		\begin{itemize}
			\item Vẽ trục đứng, trục ngang. Trục đứng thể hiện đơn vị tần số tương đối. Trục ngang ghi các nhóm số liệu.
			\item Dựng các cột tương ứng với các nhóm, chiều cao là tần số tương đối.
			\item Ghi chú giải các trục, tiêu đề biểu đồ.
		\end{itemize}
	\end{itemize}
	
\end{tomtat}

\begin{vd}%[Dự án EX-9-Đề Cương Toán 9]%[Mai Sương]%[9D5N3-3]
	Một cửa hàng sách thống kê số tiền (đơn vị: nghìn đồng) mà $60$ khách hàng mua sách ở cửa hàng đó trong một ngày. Số liệu được ghi lại trong biểu đồ tần số ghép nhóm dưới đây
	\begin{center}
		\begin{tikzpicture}[font=\footnotesize, line join=round, line cap=round, >=stealth,xscale=1,yscale=0.3]
			\foreach \giatri/\tanso [count=\i from 1] in {40/3,50/6,60/19,70/23,80/9}{
				\draw[fill=gray!30]
				(\i,0) rectangle (\i+1,\tanso)
				(\i,0) node[below]{$\giatri$}
				(\i+0.5,\tanso) node[above]{$\tanso$}
				;
				%\draw[dashed](\i,\tanso)--(0,\tanso)node[left]{$\tanso$};
				\global\let\n=\i
			}
			%--Vẽ hai trục
			\draw 
			(\n+1,0) node[below]{$90$}
			(0,25)|-(\n+2,0)
			(-0.8,12.5)node[rotate=90]{\textbf{Tần số n}}
			(\n/2+1,-1) node[below]{\textbf{Số tiền (nghìn đồng)}}
			(\n/2+1,29) node[below]{\textbf{Số tiền mua sách ở một cửa hàng trong một ngày}}
			;
			%--Chia đơn vị trục đứng--
			\foreach \y in {0,5,10,15,20,25} {
				\draw[dashed, gray!70] (0,\y) -- (\n+2,\y);
				\node[left] at (0,\y) {$\y$};
			}
		\end{tikzpicture}
	\end{center}
	Tìm tần số ghép nhóm và tần số tương đối ghép nhóm của nhóm $[40;50)$.
	\loigiai{
		Quan sát biểu đồ, ta có
		\begin{itemize}
			\item Nhóm $[40;50)$ có tần số là $3$.
			\item Tổng số khách hàng là $60$.
		\end{itemize}
		Tần số tương đối của nhóm $[40;50)$ là $\dfrac{3}{60} \cdot 100\% = 5\%$.\\
		Vậy nhóm $[40;50)$ có tần số là $3$ và tần số tương đối là $5\%$.
	}
\end{vd}

\begin{vd}%[SGK CTST Toán 9]%[Dự án EX-9-Đề Cương Toán 9]%[9D5H3-3]
	Thuỷ thống kê lại độ dài quãng đường (đơn vị: km) mình đi bộ mỗi ngày trong tháng $6$ ở bảng sau
	\begin{center}
		\begin{tabular}{|c|c|c|c|c|c|}
			\hline
			\makecell [c] {Quãng đường\\ (km)} & $[4 ; 5)$ & $[5 ; 6)$ & $[6 ; 7)$ & $[7 ; 8)$ & $[8 ; 9)$ \\
			\hline
			Tần số (số ngày) & $6$ & $12$ & $8$ & $3$ & $1$ \\
			\hline
		\end{tabular}
	\end{center}
	Hãy vẽ biểu đồ tần số tương đối ghép nhóm dạng cột biểu diễn mẫu số liệu trên.
	\loigiai{
		Bảng tần số tương đối ghép nhóm
		\begin{center}
			\begin{tabular}{|c|c|c|c|c|c|}
				\hline
				\makecell [c] {Quãng đường\\ (km)} & $[4 ; 5)$ & $[5 ; 6)$ & $[6 ; 7)$ & $[7 ; 8)$ & $[8 ; 9)$ \\
				\hline
				Tần số tương đối & $20{,}0 \%$ & $40{,}0 \%$ & $26{,}7 \%$ & $10{,}0 \%$ & $3{,}3 \%$ \\
				\hline
			\end{tabular}
		\end{center}
		Biểu đồ tần số tương đối ghép nhóm dạng cột biểu diễn mẫu số liệu
		\begin{center}
			\begin{tikzpicture}[scale=0.9,font=\footnotesize,line join=round,line cap=round,>=stealth]
				\fill (3.5,10) node[above]{\textbf{Tần số tương đối của số ngày theo độ dài quãng đường đi được mỗi ngày}};
				\draw[->] (0,0) node[left]{0}--(10,0) node[below] {Quãng đường (km)};
				\draw[->] (0,0)--(0,9.5) node[left] {Tần số tương đối ($\%$)};
				\foreach \i/\j in{1/$5$, 2/$10$,3/$15$, 4/$20$, 5/$25$, 6/$30$, 7/$35$, 8/$40$, 9/$45$}
				\draw (0,\i*1) node[left]{\j}--(0,\i*1);
				\draw[dashed,color=gray!80] (0,1)--(10,1) (0,2)--(10,2) (0,3)--(10,3) (0,4)--(10,4) 
				(0,5)--(10,5) (0,6)--(10,6) (0,7)--(10,7) (0,8)--(10,8) (0,9)--(10,9)
				;
				\fill[gray!30]
				(1,0)--(1,4)--(2,4)--(2,0)
				(2,0)--(2,8)--(3,8)--(3,0)
				(3,0)--(3,5.2)--(4,5.2)--(4,0)
				(4,0)--(4,2)--(5,2)--(5,0)
				(5,0)--(5,0.5)--(6,0.5)--(6,0)
				;
				\draw (1,0)--(1,4)--(2,4)
				(2,0)--(2,8)--(3,8)--(3,0)
				(3,5.2)--(4,5.2)--(4,0)
				(4,2)--(5,2)--(5,0)
				(5,0.5)--(6,0.5)--(6,0)
				;
				\draw (1,0) node [below]{$4$};
				\draw (2,0) node [below]{$5$};
				\draw (3,0) node [below]{$6$};
				\draw (4,0) node [below]{$7$};
				\draw (5,0) node [below]{$8$};
				\draw (6,0) node [below]{$9$};
				\fill (1.5,4) node[above]{$20{,}0\%$};
				\fill (2.5,8) node[above]{$40{,}0\%$};
				\fill (3.5,5.2) node[above]{$26{,}7\%$};
				\fill (4.5,2) node[above]{$10{,}0\%$};
				\fill (5.5,0.33) node[above]{$3{,}3\%$};
			\end{tikzpicture}
		\end{center}
	}
\end{vd}

\subsubsection{Bài tập}
\begin{bt}%[Dự án EX-9-Đề Cương Toán 9]%[Mai Sương]%[9D5H3-3]
	Công ty điện lực Hải Hậu thống kê lượng điện tiêu thụ (đơn vị: kWh) của một số hộ gia đình trong tháng $4$ năm $2025$. Dữ liệu được ghi lại như sau
	\begin{center}
		\renewcommand{\arraystretch}{1.2}
		\begin{tabular}{cccccccc}
			$249$ & $150$ & $232$ & $171$ & $247$ & $170$ & $245$ & $229$ \\
			$210$ & $231$ & $190$ & $238$ & $225$ & $237$ & $209$ & $180$ \\
			$248$ & $225$ & $211$ & $232$ & $191$ & $169$ & $228$ & $249$ \\
			$211$ & $168$ & $227$ & $225$ & $199$ & $183$ & $231$ & $210$ \\
			$240$ & $208$ & $242$ & $228$ & $235$ & $229$ & $208$ & $197$
		\end{tabular}
	\end{center}
	\begin{enumerate}
		\item Lập bảng tần số tương đối ghép nhóm cho dữ liệu trên với các nhóm $[150;170)$, $[170;190)$, $[190;210)$, $[210;230)$, $[230;250)$.
		\item Vẽ biểu đồ tần số tương đối ghép nhóm dạng cột biểu diễn bảng tần số tương đối ghép nhóm ở câu trên.
		\loigiai{
			Tần số ghép nhóm của các nhóm $[150;170)$, $[170;190)$, $[190;210)$, $[210;230)$, $[230;250)$ lần lượt là $3;$ $4$; $7$; $11$; $15$.\\
			Tần số tương đối ghép nhóm của các nhóm trên lần lượt là
			$f_1=\dfrac{3}{40} \cdot 100\% = 7{,}5\%$;
			$f_2=\dfrac{4}{40} \cdot 100\% = 10\%$; 
			$f_3=\dfrac{7}{40} \cdot 100\% = 17{,}5\%$;
			$f_4=\dfrac{11}{40} \cdot 100\% = 27{,}5\%$;
			$f_5=\dfrac{15}{40} \cdot 100\% = 37{,}5\%$.\\
			Bảng tần số tương đối ghép nhóm
			\begin{center}
				\renewcommand{\arraystretch}{1.2}
				\begin{tabular}{|c|c|c|c|c|c|}
					\hline
					Lượng điện tiêu thụ (kWh) & $[150;170)$ & $[170;190)$ & $[190;210)$ & $[210;230)$ & $[230;250)$ \\
					\hline
					Tần số tương đối (\%) & $7{,}5$ & $10$ & $17{,}5$ & $27{,}5$ & $37{,}5$ \\
					\hline
				\end{tabular}
			\end{center}
			Biểu đồ tần số tương đối ghép nhóm dạng cột biểu diễn theo bảng trên được thể hiện như hình dưới.
			\begin{center}
				\begin{tikzpicture}[font=\footnotesize, line join=round, line cap=round, >=stealth,xscale=1,yscale=0.15]
					\foreach \giatri/\tanso/\nhan [count=\i from 1] in {150/7.5/7{,}5,170/10/10,190/17.5/17{,}5,210/27.5/27{,}5,230/37.5/37{,}5}{
						\draw[fill=gray!30]
						(\i,0) rectangle (\i+1,\tanso)
						(\i,0) node[below]{$\giatri$}
						(\i+0.5,\tanso) node[above]{$\nhan \%$}
						;
						%\draw[dashed](\i,\tanso)--(0,\tanso)node[left]{$\tanso$};
						\global\let\n=\i
					}
					%--Vẽ hai trục
					\draw 
					(\n+1,0) node[below]{$250$}
					(0,45)|-(\n+2,0)
					(-1.2,22.5)node[rotate=90]{\textbf{Tần số tương đối (\%)}}
					(\n/2+1,-2.5) node[below]{\textbf{Lượng điện tiêu thụ (kWh)}}
					(\n/2+1,50) node[below]{\textbf{Lượng điện tiêu thụ trong tháng 4 năm 2025}}
					;
					%--Chia đơn vị trục đứng--
					\foreach \y in {0,10,20,30,40} {
						\draw[dashed, gray!70] (0,\y) -- (\n+2,\y);
						\node[left] at (0,\y) {$\y \%$};
					}
				\end{tikzpicture}
			\end{center}
		}
	\end{enumerate}
\end{bt}
\begin{bt}%[Dự án EX-9-Đề Cương Toán 9]%[Mai Sương]%[9D5H3-3]
	Một bác thợ đóng giày thống kê lại độ dài bàn chân (đơn vị: cm) của $40$ khách hàng ở bảng tần số ghép nhóm như sau
	\begin{center}
		\begin{tabular}{|c|c|c|c|c|c|}
			\hline
			\textbf{Độ dài (cm)} & $[27; 28)$ & $[28; 29)$ & $[29; 30)$ & $[30; 31)$ & $[31; 32)$ \\
			\hline
			\textbf{Tần số} & $2$ & $8$ & $20$ & $6$ & $4$ \\
			\hline
		\end{tabular}
	\end{center}
	
	\begin{enumerate}
		\item Tìm tần số tương đối của mỗi nhóm.
		\item Lập bảng tần số tương đối ghép nhóm cho mẫu số liệu trên.
		\item Vẽ biểu đồ tần số tương đối ghép nhóm dạng cột biểu diễn mẫu số liệu trên.
	\end{enumerate}
	\loigiai{
		\begin{enumerate}
			\item Kí hiệu $f_1$, $f_2$, $f_3$, $f_4$, $f_5$ lần lượt là tần số tương đối của các nhóm $[27; 28)$; $[28; 29)$; $[29; 30)$; $[30; 31)$; $[31; 32)$.\\
			Ta có
			\allowdisplaybreaks
			\begin{eqnarray*}
				f_1 &=& \dfrac{2}{40} \cdot 100\% = 5\%; \\ 
				f_2 &=& \dfrac{8}{40} \cdot 100\% = 20\%; \\
				f_3 &=& \dfrac{20}{40} \cdot 100\% = 50\%; \\
				f_4 &=& \dfrac{6}{40} \cdot 100\% = 15\%; \\ 
				f_5 &=& \dfrac{4}{40} \cdot 100\% = 10\%.
			\end{eqnarray*}
			\item Bảng tần số tương đối ghép nhóm
			\begin{center}
				\begin{tabular}{|c|c|c|c|c|c|}
					\hline
					\textbf{Độ dài (cm)} & $[27; 28)$ & $[28; 29)$ & $[29; 30)$ & $[30; 31)$ & $[31; 32)$ \\
					\hline
					\textbf{Tần số tương đối} & $5\%$ & $20\%$ & $50\%$ & $15\%$ & $10\%$ \\
					\hline
				\end{tabular}
			\end{center}
			\item Biểu đồ tần số tương đối ghép nhóm dạng cột
			\begin{center}
				\begin{tikzpicture}[font=\footnotesize, line join=round, line cap=round, >=stealth,xscale=1,yscale=0.15]
					\foreach \giatri/\tanso [count=\i from 1] in {27/5,28/20,29/50,30/15,31/10}{
						\draw[fill=gray!30]
						(\i,0) rectangle (\i+1,\tanso)
						(\i,0) node[below]{$\giatri$}
						(\i+0.5,\tanso) node[above]{$\tanso \%$}
						;
						\global\let\n=\i
					}
					%--Vẽ hai trục
					\draw 
					(\n+1,0) node[below]{$32$}
					(0,55)|-(\n+2,0)
					(-1.2,22.5)node[rotate=90]{\textbf{Tần số tương đối (\%)}}
					(\n/2+1,-1.4) node[below]{\textbf{Độ dài (cm)}}
					(\n/2+1,60) node[below]{\textbf{Tần số tương đối của số khách hàng theo độ dài bàn chân}}
					;
					%--Chia đơn vị trục đứng--
					\foreach \y in {0,10,20,30,40,50} {
						\draw[dashed, gray!70] (0,\y) -- (\n+2,\y);
						\node[left] at (0,\y) {$\y \%$};
					}
				\end{tikzpicture}
			\end{center}
		\end{enumerate}
	}
\end{bt}
\begin{bt}%[Dự án EX-9-Đề Cương Toán 9]%[Mai Sương]%[9D5V3-3]
	Biểu đồ dưới đây biểu diễn mẫu số liệu ghép nhóm về mức lương của nhân viên một công ty (đơn vị: triệu đồng).
	\begin{center}
		\begin{tikzpicture}[font=\footnotesize, line join=round, line cap=round, >=stealth,xscale=1,yscale=0.15]
			\foreach \giatri/\tanso [count=\i from 1] in {6/8,8/24,10/40,12/16,14/12}{
				\draw[fill=gray!30]
				(\i,0) rectangle (\i+1,\tanso)
				(\i,0) node[below]{$\giatri$}
				(\i+0.5,\tanso) node[above]{$\tanso \%$}
				;
				\global\let\n=\i
			}
			%--Vẽ hai trục
			\draw 
			(\n+1,0) node[below]{$16$}
			(0,50)|-(\n+2,0)
			(-1.2,25)node[rotate=90]{\textbf{Tần số tương đối (\%)}}
			(\n/2+1,-1.4) node[below]{\textbf{Mức lương (triệu đồng)}}
			(\n/2+1,58) node[below]{\textbf{Tần số tương đối của nhân viên theo mức lương}}
			;
			%--Chia đơn vị trục đứng--
			\foreach \y in {0,10,20,30,40,50} {
				\draw[dashed, gray!70] (0,\y) -- (\n+2,\y);
				\node[left] at (0,\y) {$\y \%$};
			}
		\end{tikzpicture}
	\end{center}
	\begin{enumerate}
		\item Biết công ty có $25$ nhân viên. Hãy tìm tần số của mỗi nhóm và lập bảng tần số ghép nhóm.
		\item Có thông tin cho rằng trên $\dfrac{1}{4}$ số nhân viên của công ty có mức thu nhập từ $12$ triệu đồng trở lên. Thông tin đó có chính xác không? Tại sao?
	\end{enumerate}
	\loigiai{
		\begin{enumerate}
			\item Gọi $m_1$, $m_2$, $m_3$, $m_4$, $m_5$ lần lượt là tần số của các nhóm $[6;8)$, $[8;10)$, $[10;12)$, $[12;14)$, $[14;16)$.\\
			Ta có $m_1 = \dfrac{25 \cdot 8}{100} = 2$; $m_2 = \dfrac{25 \cdot 24}{100} = 6$; $m_3 = \dfrac{25 \cdot 40}{100} = 10$;
			$m_4 = \dfrac{25 \cdot 16}{100} = 4$; $m_5 = \dfrac{25 \cdot 12}{100} = 3$.\\
			Bảng tần số ghép nhóm của mẫu số liệu
			\begin{center}
				\begin{tabular}{|c|c|c|c|c|c|}
					\hline
					\textbf{Mức lương (triệu đồng)} & $[6;8)$ & $[8;10)$ & $[10;12)$ & $[12;14)$ & $[14;16)$ \\
					\hline
					\textbf{Tần số} & $2$ & $6$ & $10$ & $4$ & $3$ \\
					\hline
				\end{tabular}
			\end{center}
			\item Tần số tương đối của nhân viên có thu nhập từ $12$ triệu đồng trở lên là $16\% + 12\% = 28\% > \dfrac{1}{4}$.\\
			Do đó thông tin trên $\dfrac{1}{4}$ số nhân viên của công ty có mức thu nhập từ $12$ triệu đồng trở lên là chính xác.
		\end{enumerate}
	}
\end{bt}
\begin{bt}%[Dự án EX-9-Đề Cương Toán 9]%[Mai Sương]%[9D5V3-3]
	Biểu đồ dưới đây biểu diễn mẫu số liệu ghép nhóm về đường kính thân (đơn vị: cm) của $80$ cây keo trồng tại một lâm trường.
	\begin{center}
		\begin{tikzpicture}[font=\footnotesize, line join=round, line cap=round, >=stealth,xscale=1,yscale=0.15]
			\foreach \giatri/\tanso [count=\i from 1] in {60/5,64/20,68/40,72/25,76/10}{
				\draw[fill=gray!30]
				(\i,0) rectangle (\i+1,\tanso)
				(\i,0) node[below]{$\giatri$}
				(\i+0.5,\tanso) node[above]{$\tanso \%$}
				;
				%\draw[dashed](\i,\tanso)--(0,\tanso)node[left]{$\tanso$};
				\global\let\n=\i
			}
			%--Vẽ hai trục
			\draw 
			(\n+1,0) node[below]{$80$}
			(0,45)|-(\n+2,0)
			(-1.2,22.5)node[rotate=90]{\textbf{Tần số tương đối (\%)}}
			(\n/2+1,-1.7) node[below]{\textbf{Đường kính (cm)}}
			(\n/2+1,50) node[below]{\textbf{Tần số tương đối của số cây keo theo đường kính thân cây}}
			;
			%--Chia đơn vị trục đứng--
			\foreach \y in {0,10,20,30,40}{
				\draw[dashed, gray!70] (0,\y) -- (\n+2,\y);
				\node[left] at (0,\y) {$\y \%$};
			}
		\end{tikzpicture}
	\end{center}
	\begin{enumerate}
		\item Hãy tìm tần số của mỗi nhóm số liệu và lập bảng tần số tương đối ghép nhóm.
		\item Có ý kiến cho rằng $\dfrac{1}{3}$ số cây keo có đường kính thân cây từ $72$ cm trở lên.\\
		Ý kiến trên có chính xác không? Tại sao?
	\end{enumerate}
	\loigiai{
		\begin{enumerate}
			\item Gọi $m_1$, $m_2$, $m_3$, $m_4$, $m_5$ lần lượt là tần số của các nhóm $[60; 64)$, $[64; 68)$, $[68; 72)$, $[72; 76)$, $[76; 80)$.\\
			Ta có $m_1 = 80 \cdot \dfrac{5}{100} = 4$; $m_2 = 80 \cdot \dfrac{20}{100} = 16$; $m_3 = 80 \cdot \dfrac{40}{100} = 32$; $m_4 = 80 \cdot \dfrac{25}{100} = 20$; $m_5 = 80 \cdot \dfrac{10}{100} = 8$.\\
			Bảng tần số ghép nhóm của mẫu số liệu
			\begin{center}
				\begin{tabular}{|c|c|c|c|c|c|}
					\hline
					\textbf{Đường kính (cm)} & $[60; 64)$ & $[64; 68)$ & $[68; 72)$ & $[72; 76)$ & $[76; 80)$ \\
					\hline
					\textbf{Tần số} & $4$ & $16$ & $32$ & $20$ & $8$ \\
					\hline
				\end{tabular}
			\end{center}
			\item Tần số tương đối của số cây keo có đường kính thân cây từ $72$ cm trở lên là $25\% + 10\% = 35\% > \dfrac{1}{3}$.\\
			Do đó, có hơn $\dfrac{1}{3}$ số cây keo có đường kính thân cây từ $72$ cm.\\
			Vậy ý kiến đưa ra là chưa chính xác.
		\end{enumerate}
	}
\end{bt}

\subsection{Biểu đồ tần số tương đối ghép nhóm dạng đoạn thẳng}
\subsubsection{Kiến thức trọng tâm}
\begin{tomtat}
	\begin{itemize}
		\item Là biểu đồ dùng đường gấp khúc để biểu diễn bảng tần số tương đối ghép nhóm.
		\item Các bước vẽ biểu đồ đoạn thẳng.\\
		+ Chọn giá trị đại diện $x_i = \dfrac{a_i + a_{i+1}}{2}$ cho mỗi nhóm.\\
		+ Vẽ trục ngang biểu diễn các $x_i$, trục đứng biểu diễn tần số tương đối $f_i$.\\
		+ Xác định điểm $M_i(x_i; f_i)$ và nối các điểm lại thành đoạn thẳng liên tiếp.\\
		+ Ghi chú giải cho biểu đồ.
	\end{itemize}
\end{tomtat}

\begin{vd}%[Dự án EX-9-Đề Cương Toán 9]%[Mai Sương]%[9D5H3-3]
	Thời gian chờ mua vé xem bóng đá của một số cổ động viên được cho như sau
	\begin{center}
		\begin{tabular}{|c|c|c|c|c|c|c|}
			\hline
			\textbf{Thời gian (phút)} & $[0;5)$ & $[5;10)$ & $[10;15)$ & $[15;20)$ & $[20;25)$ & $[25;30)$ \\
			\hline
			\textbf{Số cổ động viên} & $15$ & $38$ & $50$ & $27$ & $20$ & $10$ \\
			\hline
		\end{tabular}
	\end{center}
	\begin{enumerate}
		\item Lập bảng tần số tương đối ghép nhóm.
		\item Vẽ biểu đồ tần số tương đối ghép nhóm dạng đoạn thẳng cho bảng thống kê thu được ở câu trên.
	\end{enumerate}
	\loigiai{
		\begin{enumerate}
			\item Tổng số cổ động viên mua vé là 
			$15 + 38 + 50 + 27 + 20 + 10 = 160$.\\
			Tần số tương đối tương ứng với các nhóm số liệu thời gian $[0;5)$; $[5;10)$; $[10;15)$; $[15;20)$; $[20;25)$; $[25;30)$ lần lượt là
			\begin{eqnarray*}
				f_1 &=&\dfrac{15}{160} = 9{,}375\%; \\
				f_2 &=&\dfrac{38}{160} = 23{,}75\%; \\ 
				f_3 &=&\dfrac{50}{160} = 31{,}25\%; \\ 
				f_4 &=&\dfrac{27}{160} = 16{,}875\%; \\ 
				f_5 &=&\dfrac{20}{160} = 12{,}5\%; \\ 
				f_6 &=&\dfrac{10}{160} = 6{,}25\%.
			\end{eqnarray*}
			Do đó, ta có bảng tần số tương đối ghép nhóm
			\begin{center}
				\begin{tabular}{|c|c|c|c|c|c|c|}
					\hline
					\textbf{Thời gian (phút)} & $[0;5)$ & $[5;10)$ & $[10;15)$ & $[15;20)$ & $[20;25)$ & $[25;30)$ \\
					\hline
					\textbf{Tần số tương đối} & $9{,}375\%$ & $23{,}75\%$ & $31{,}25\%$ & $16{,}875\%$ & $12{,}5\%$ & $6{,}25\%$ \\
					\hline
				\end{tabular}
			\end{center}  
			\item Tần số tương đối ghép nhóm dạng đoạn thẳng
			\begin{center}
				\begin{tikzpicture}[font=\footnotesize, line join=round, line cap=round, >=stealth,scale=1.5]
					\def\hsco{0.1}
					\def\ymax{35}
					\foreach \giatri/\tanso/\nhan [count=\i from 0] in {[0; 5)/9.375/9{,}375,[5; 10)/23/23{,}75,[10; 15)/31.25/31{,}25,[15; 20)/16.875/16{,}875,[20; 25)/12.5/12{,}5,[25; 30)/6.25/6{,}25}{
						\draw[fill=black]
						(\i+0.5,\hsco*\tanso) circle (0.03cm) node[above]{$\nhan \%$}
						(\i+0.5,0.03)--(\i+0.5,-0.03) node[below]{$\giatri$} 
						;
						\path (\i+0.5,\hsco*\tanso) coordinate (A\i);
						\global\let\n=\i
					}
					%--Vẽ hai trục
					\draw 
					(0,\hsco*\ymax)|-(\n+1,0)
					(-0.75,\hsco*\ymax*0.5)node[rotate=90]{\textbf{Tần số tương đối (\%)}}
					(\n/2+0.5,-0.3) node[below]{\textbf{Thời gian (phút)}}
					(\n/2+0.5,\hsco*\ymax+1) node[below]{\textbf{Tần số tương đối của số cổ động viên theo thời gian chờ mua vé}}
					;
					%--Chia đơn vị trục đứng--
					\foreach \y in {0,5,10,15,20,25,30,35} {
						\draw[dashed, gray!70] (0,\hsco*\y) -- (\n+1,\hsco*\y);
						\node[left] at (0,\hsco*\y) {$\y \%$};
					}
					\draw (A0) -- (A1) -- (A2) -- (A3) -- (A4)--(A5);
				\end{tikzpicture}
			\end{center} 
		\end{enumerate}
	}
\end{vd}

\subsubsection{Bài tập}

\begin{bt}%[SGK CTST Toán 9]%[Dự án EX-9-Đề Cương Toán 9]%[9D5H3-3]
	Bảng sau thống kê chiều cao (đơn vị: mét) của các cây keo $3$ năm tuổi ở một nông trường.
	\begin{center}
		\begin{tabular}{|c|c|c|c|c|c|}
			\hline
			\makecell[c] {Chiều cao\\ (m)} & $[8,5 ; 8,7)$ & $[8,7 ; 8,9)$ & $[8,9 ; 9,1)$ & $[9,1 ; 9,3)$ & $[9,3 ; 9,5)$ \\
			\hline
			Tần số tương đối & $15 \%$ & $25 \%$ & $25 \%$ & $20 \%$ & $15 \%$ \\
			\hline
		\end{tabular}
	\end{center}
	Hãy vẽ biểu đồ tần số tương đối ghép nhóm dạng đoạn thẳng biểu diễn số liệu trên.
	\loigiai{
		Giá trị đại diện của các nhóm dữ liệu lần lượt là $8{,}6$ ; $ 8{,}8 $ ; $ 9{,}0 $ ; $ 9{,}2 $ ; $9{,}4$.\\
		Biểu đồ tần số tương đối ghép nhóm dạng đoạn thẳng biểu diễn số liệu đã cho là
		\begin{center}
			\begin{tikzpicture}[scale=0.9,font=\footnotesize,line join=round,line cap=round,>=stealth]
				\fill (4.5,7) node[above]{\textbf{Tần số tương đối của số cây theo chiều cao}};
				\draw[->] (0,0) node[left]{0}--(10,0) node[below] {Chiều cao (m)};
				\draw[->] (0,0)--(0,6.5) node[left] {Tần số tương đối ($\%$)};
				\foreach \i/\j in{1/$5$, 2/$10$,3/$15$, 4/$20$, 5/$25$, 6/$30$}
				\draw (0,\i*1) node[left]{\j}--(0,\i*1);
				\draw[dashed,color=gray!80] (0,1)--(10,1) (0,2)--(10,2) (0,3)--(10,3) (0,4)--(10,4) (0,5)--(10,5) (0,6)--(10,6) ;
				\draw[dashed,color=gray!80] (1,0)--(1,3) (2.5,0)--(2.5,5) (4,0)--(4,5) (5.5,0)--(5.5,4) (7,0)--(7,3);
				\draw [thick, red] (1,3)--(2.5,5)--(4,5)--(5.5,4)--(7,3);		
				\draw (1,0) node [below]{$8{,}6$};
				\draw (2.5,0) node [below]{$8{,}8$};
				\draw (4,0) node [below]{$9{,}0$};
				\draw (5.5,0) node [below]{$9{,}2$};
				\draw (7,0) node [below]{$9{,}4$};
			\end{tikzpicture}
		\end{center}
	}
\end{bt}

\begin{bt}%[Dự án EX-9-Đề Cương Toán 9]%[Mai Sương]%[9D5H3-3]
	Thời gian tư vấn tuyển sinh cho $40$ học sinh của một trường THCS trên địa bàn thành phố Đà Nẵng được ghi lại ở bảng sau. (đơn vị: phút)
	\begin{center}
		\begin{tabular}{|c|c|c|c|c|c|c|c|c|c|}
			\hline
			$6{,}4$ & $6{,}6$ & $7{,}5$ & $8{,}3$ & $9{,}2$ & $8{,}4$ & $8{,}6$ & $9{,}9$ & $8{,}7$ & $7{,}9$ \\
			\hline
			$8{,}1$ & $7{,}8$ & $6{,}4$ & $6{,}7$ & $7{,}3$ & $9{,}9$ & $6{,}3$ & $6{,}7$ & $6{,}8$ & $7{,}7$ \\
			\hline
			$7{,}3$ & $6{,}2$ & $8{,}1$ & $7{,}9$ & $6{,}4$ & $6{,}5$ & $7{,}8$ & $8{,}8$ & $7{,}1$ & $9{,}8$ \\
			\hline
			$6$    & $6{,}3$ & $7{,}5$ & $6{,}8$ & $6{,}3$ & $9{,}0$ & $7{,}6$ & $8{,}4$ & $9{,}1$ & $7{,}2$ \\
			\hline
		\end{tabular}
	\end{center}
	\begin{enumerate}
		\item Chia số liệu trên thành $4$ nhóm: $[6;7)$, $[7;8)$, $[8;9)$, $[9;10)$. Hãy lập bảng tần số ghép nhóm biểu diễn thời gian tư vấn tuyển sinh của $40$ học sinh.
		\item Vẽ biểu đồ biểu thị tần số ghép nhóm dạng đoạn thẳng ở câu trên.
	\end{enumerate}
	\loigiai{
		\begin{enumerate}
			\item Bảng tần số ghép nhóm
			\begin{center}
				\begin{tabular}{|c|c|c|c|c|}
					\hline
					\textbf{Thời gian tư vấn} & $[6;7)$ & $[7;8)$ & $[8;9)$ & $[9;10)$ \\
					\hline
					\textbf{Tần số} & $14$ & $12$ & $8$ & $6$ \\
					\hline
				\end{tabular}
			\end{center}
			\item Biểu đồ tần số ghép nhóm dạng đoạn thẳng
			\begin{center}
				\begin{tikzpicture}[font=\footnotesize, line join=round, line cap=round, >=stealth,scale=1.5]
					\def\hsco{0.2}
					\foreach \giatri/\tanso [count=\i from 0] in {[6;7)/14,[7;8)/12,[8;9)/8,[9;10)/6}{
						\draw[fill=black]
						(\i+0.5,\hsco*\tanso) circle (0.03cm) node[above]{$\tanso$}
						(\i+0.5,0.03)--(\i+0.5,-0.03) node[below]{$\giatri$} 
						;
						\path (\i+0.5,\hsco*\tanso) coordinate (A\i);
						\global\let\n=\i
					}
					%--Vẽ hai trục
					\draw 
					(0,\hsco*16)|-(\n+1,0)
					(-0.75,\hsco*8)node[rotate=90]{\textbf{Tần số}}
					(\n/2+0.5,\hsco*-2.5) node[below]{\textbf{Thời gian tư vấn (phút)}}
					(\n/2+0.5,\hsco*20) node[below]{\textbf{Thời gian tư vấn tuyển sinh của một trường THCS}}
					;
					%--Chia đơn vị trục đứng--
					\foreach \y in {0,2,4,6,8,10,12,14,16} {
						\draw[dashed, gray!70] (0,\hsco*\y) -- (\n+1,\hsco*\y);
						\node[left] at (0,\hsco*\y) {$\y$};
					}
					\draw (A0) -- (A1) -- (A2) -- (A3);
				\end{tikzpicture}
			\end{center}
		\end{enumerate}
	}
\end{bt}

\begin{bt}%[Dự án EX-9-Đề Cương Toán 9]%[Mai Sương]%[9D5V3-3]
	Một quán giải khát thống kê thời gian (đơn vị: phút) mà $64$ khách hàng ở tại quán. Kết quả được ghi lại ở bảng tần số ghép nhóm như sau
	\begin{center}
		\begin{tabular}{|c|c|c|c|c|c|}
			\hline
			\textbf{Thời gian (phút)} & $[0; 10)$ & $[10; 20)$ & $[20; 30)$ & $[30; 40)$ & $[40; 50)$ \\
			\hline
			\textbf{Tần số} & $24$ & $16$ & $12$ & $8$ & $4$ \\
			\hline
		\end{tabular}
	\end{center}
	\begin{enumerate}
		\item Tìm tần số tương đối của mỗi nhóm.
		\item Lập bảng tần số tương đối ghép nhóm cho mẫu số liệu trên.
		\item Vẽ biểu đồ tần số tương đối ghép nhóm dạng đoạn thẳng biểu diễn mẫu số liệu trên.
		\item Chủ quán khẳng định rằng có trên một nửa số khách hàng ở quán nhiều hơn $20$ phút. Khẳng định của chủ quán có chính xác không? Tại sao?
	\end{enumerate}
	\loigiai{
		\begin{enumerate}
			\item Kí hiệu $f_1$, $f_2$, $f_3$, $f_4$, $f_5$ lần lượt là tần số tương đối của các nhóm $[0; 10)$; $[10; 20)$; $[20; 30)$; $[30; 40)$; $[40; 50)$.\\
			Ta có
			\allowdisplaybreaks
			\begin{eqnarray*}
				f_1 &=& \dfrac{24}{64} \cdot 100\% = 37{,}5\%, \\
				f_2 &=& \dfrac{16}{64} \cdot 100\% = 25\%, \\
				f_3 &=& \dfrac{12}{64} \cdot 100\% = 18{,}75\%,\\
				f_4 &=& \dfrac{8}{64} \cdot 100\% = 12{,}5\%, \\
				f_5 &=& \dfrac{4}{64} \cdot 100\% = 6{,}25\%.
			\end{eqnarray*}
			\item Bảng tần số tương đối ghép nhóm
			\begin{center}
				\begin{tabular}{|c|c|c|c|c|c|}
					\hline
					\textbf{Thời gian (phút)} & $[0; 10)$ & $[10; 20)$ & $[20; 30)$ & $[30; 40)$ & $[40; 50)$ \\
					\hline
					\textbf{Tần số tương đối (\%)} & $37{,}5$ & $25$ & $18{,}75$ & $12{,}5$ & $6{,}25$ \\
					\hline
				\end{tabular}
			\end{center}
			\item Giá trị đại diện của các nhóm $[0; 10)$; $[10; 20)$; $[20; 30)$; $[30; 40)$; $[40; 50)$ lần lượt là $5$; $15$; $25$; $35$; $45$.\\
			Biểu đồ tần số tương đối ghép nhóm dạng đoạn thẳng là
			\begin{center}
				\begin{tikzpicture}[font=\footnotesize, line join=round, line cap=round, >=stealth,scale=1.5]
					\def\hsco{0.1}
					\foreach \giatri/\tanso/\nhan [count=\i from 0] in {[0; 10)/37.5/37{,}50,[10; 20)/25/25{,}00,[20; 30)/18.75/18{,}75,[30; 40)/12.5/12{,}50,[40; 50)/6.25/6{,}25}{
						\draw[fill=black]
						(\i+0.5,\hsco*\tanso) circle (0.03cm) node[above right]{$\nhan \%$}
						(\i+0.5,0.03)--(\i+0.5,-0.03) node[below]{$\giatri$} 
						;
						\path (\i+0.5,\hsco*\tanso) coordinate (A\i);
						\global\let\n=\i
					}
					%--Vẽ hai trục
					\draw 
					(0,\hsco*45)|-(\n+1,0)
					(-0.75,\hsco*22.5)node[rotate=90]{\textbf{Tần số tương đối (\%)}}
					(\n/2+0.5,\hsco*-2.5) node[below]{\textbf{Thời gian (phút)}}
					(\n/2+0.5,\hsco*50) node[below]{\textbf{Tần số tương đối của số khách hàng theo thời gian ở tại quán giải khát}}
					;
					%--Chia đơn vị trục đứng--
					\foreach \y in {0,5,10,15,20,25,30,35,40} {
						\draw[dashed, gray!70] (0,\hsco*\y) -- (\n+1,\hsco*\y);
						\node[left] at (0,\hsco*\y) {$\y \%$};
					}
					\draw (A0) -- (A1) -- (A2) -- (A3) -- (A4);
				\end{tikzpicture}
			\end{center}
			\item Tổng tần số tương đối của các nhóm từ $[20; 30)$ trở đi là
			$$
			18{,}75\% + 12{,}5\% + 6{,}25\% = 37{,}5\% < 50\%.
			$$
			Vậy khẳng định của chủ quán là \textbf{không chính xác}.
		\end{enumerate}
	}
\end{bt}

\begin{bt}%[SGK CTST Toán 9]%[Dự án EX-9-Đề Cương Toán 9]%[9D5V3-3]
	Hai bạn Hà và Hồng thống kê lại chỉ số chất lượng không khí (AQI) nơi mình ở tại thời điểm $12:00$ mỗi ngày trong tháng $9/2022$ ở bảng sau
	\immini{\begin{center}
			\begin{tabular}{|c|c|c|c|c|}
				\hline
				Chỉ số & $[50 ; 100)$ & $[100 ; 150)$ & $[150 ; 200)$ & $[200 ; 250)$ \\
				\hline
				Tại nơi ở của Hà & $12$ & $8$ & $6$ & $4$ \\
				\hline
				Tại nơi ở của Hồng & $16$ & $6$ & $5$ & $3$ \\
				\hline
			\end{tabular}
	\end{center}}
	{\begin{tikzpicture}[>=stealth,font=\footnotesize,scale=1]
			\def\r{1};
			\foreach \g/\col/\va in {-45/violet/500,0/purple/300,45/red/200,90/orange/150,135/yellow/100,180/green/50}{
				\draw[\col,line width=8pt] (\g:\r) arc(\g:{\g+45}:\r);
				\node at ($(\g:\r)+(\g:0.4)$){\bfseries $ \va $};
				\draw[white,line width=1pt] (\g:{\r-0.25})--++(\g:0.4);}
			\node at ($(225:\r)+(225:0.4)$){$0$};
			\fill (0,0) circle (10pt);
			\draw[line width=5pt,-latex] (0,0)--(100:\r);
	\end{tikzpicture}}
	\begin{enumerate}
		\item Hãy vẽ trên cùng một hệ trục hai biểu đồ dạng đoạn thẳng biểu diễn tần số tương đối cho bảng chỉ số chất lượng không khí tại nơi ở của bạn Hà và tại nơi ở của bạn Hồng.
		\item Chỉ số AQI từ $150$ trở lên được coi là không lành mạnh. Dựa vào biểu đồ tần số tương đối trên, hãy so sánh tỉ lệ số ngày chất lượng không khí được coi là không lành mạnh ở mỗi khu vực.
	\end{enumerate}
	\loigiai{
		\begin{enumerate}
			\item Bảng tần số tương đối chỉ số chất lượng không khí tại nơi ở của bạn Hà và tại nơi ở của bạn Hồng.
			\begin{center}
				\begin{tabular}{|c|c|c|c|c|}
					\hline
					Chỉ số & $[50 ; 100)$ & $[100 ; 150)$ & $[150 ; 200)$ & $[200 ; 250)$ \\
					\hline
					Tại nơi ở của Hà & $40\%$ & $26{,}7\%$ & $20\%$ & $13{,}3\%$ \\
					\hline
					Tại nơi ở của Hồng & $53{,}3\%$ & $20\%$ & $16{,}7\%$ & $10\%$ \\
					\hline
				\end{tabular}
			\end{center}
			Giá trị đại diện của các nhóm dữ liệu lần lượt là $75$ ; $125$ ; $175$ ; $225$.\\
			Biểu đồ dạng đoạn thẳng biểu diễn tần số tương đối cho bảng chỉ số chất lượng không khí tại nơi ở của bạn Hà và tại nơi ở của bạn Hồng.
			\begin{center}
				\begin{tikzpicture}[scale=0.9,font=\footnotesize,line join=round,line cap=round,>=stealth]
					\path (5,12) rectangle(6,12) node[midway]{\textbf{Tần số tương đối chỉ số chất lượng không khí tại nơi ở của bạn Hà và tại nơi ở của bạn Hồng.}};
					\draw[->] (0,0) node[left]{0}--(10,0) node[below] {Chỉ số};
					\draw[->] (0,0)--(0,11.5) node[left] {Tần số tương đối ($\%$)};
					\foreach \i/\j in{1/$5$, 2/$10$,3/$15$, 4/$20$, 5/$25$, 6/$30$, 7/$35$, 8/$40$, 9/$45$, 10/$50$, 11/$55$}
					\draw (0,\i*1) node[left]{\j}--(0,\i*1);
					\draw (0,1)--(10,1) (0,2)--(10,2) (0,3)--(10,3) (0,4)--(10,4) (0,5)--(10,5) (0,6)--(10,6) (0,7)--(10,7) (0,8)--(10,8) (0,9)--(10,9) (0,10)--(10,10) (0,11)--(10,11)
					;
					\draw [dashed] (1.5,0)--(1.5,10.3) (3,0)--(3,5.2) (4.5,0)--(4.5,4) (6,0)--(6,2.3);
					\draw [thick, red, line width=0.5] (1.5,8)--(3,5.2)--(4.5,4)--(6,2.3);
					\draw [thick, blue, line width=2] (1.5,10.3)--(3,4)--(4,3.2)--(6,2);
					\draw (1.5,0) node [below]{$75$};
					\draw (3,0) node [below]{$125$};
					\draw (4.5,0) node [below]{$175$};
					\draw (6,0) node [below]{$225$};
					\draw [thick, red, line width=0.5] (0,-1)--(1,-1) node[right]{Nơi ở của Hà};
					\draw [thick, blue, line width=2] (0,-2)--(1,-2) node[right]{Nơi ở của Hồng};
				\end{tikzpicture}
			\end{center}		
			\item Tỉ lệ phần trăm số ngày chất lượng không khí được coi là không lành mạnh tại nơi ở của bạn Hà là $20\%+13{,}3\%=33{,}3\%$.\\
			Tỉ lệ phần trăm số ngày chất lượng không khí được coi là không lành mạnh tại nơi ở của bạn Hồng là $16{,}7\%+10\%=26{,}7\%$.\\
			Vậy tỉ lệ phần trăm số ngày chất lượng không khí được coi là không lành mạnh tại nơi ở của bạn Hà cao hơn nơi ở của bạn Hồng.
		\end{enumerate}		
	}
\end{bt}
