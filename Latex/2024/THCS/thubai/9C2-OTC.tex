\section*{ĐỀ CƯƠNG ÔN TẬP CHƯƠNG 2}
\subsection{Câu hỏi trắc nghiệm}

\Opensolutionfile{ans}[ans/ans-9C2-OTC]

%%%====Câu 1
\begin{ex}%[Dự án EX-9-Đề Cương Toán 9]%[Đặng Thị Thu Thảo]%[9D2N1-1]
	Khẳng định nào sau đây đúng. Cho  $a>b$ khi đó
\choice
{\True $a-b>0$}
{$a-b<0$}
{$a-b=0$}
{$a-b\leq0$}
\loigiai{
	Từ $a>b$, cộng $-b$ vào hai vế ta được $a-b>b-b$, tức là $a-b>0$.
}
\end{ex}

%%%====Câu 2
\begin{ex}%[Dự án EX-9-Đề Cương Toán 9]%[Đặng Thị Thu Thảo]%[9D2H1-2]
	Hãy chọn câu sai. Nếu $a<b$ thì
	\choice
	{$2a+1<2b+5$}
	{$7-3a<4-3b$}
	{$7a-1<7b-1$}
	{\True$2-3a<2-3b$}
	\loigiai{
		Vì $a<b$ nên $-3a>-3b$, suy ra $2-3a>2-3b$ nên $2-3a<2-3b$ \textbf{sai}.
	}
\end{ex}

%%%====Câu 3
\begin{ex}%[Dự án EX-9-Đề Cương Toán 9]%[Đặng Thị Thu Thảo]%[9D2H1-2]
		Cho $a+8<b$. So sánh $a-7$ và $b-15$.
	\choice
	{\True$a-7<b-15$}
	{$a-7>b-15$}
	{$a-7\geq b-15$}
	{$b-15\leq a-7$}
	\loigiai{
		Cộng cả hai vế của bất đẳng thức $a+8<b$ với $-15$ ta được
		\begin{eqnarray*}
			a+8&<&b\\
			a+8-15&<&b-15\\
			a-7&<&b-15.
		\end{eqnarray*}
	}
\end{ex}

%%%====Câu 4
\begin{ex}%[Dự án EX-9-Đề Cương Toán 9]%[Đặng Thị Thu Thảo]%[9D2H1-2]
		Với $x$, $y$ bất kỳ. Chọn khẳng định đúng.
	\choice
	{$(x+y)^2\leq4xy$}
	{$(x+y)^2>4xy$}
	{$(x+y)^2<4xy$}
	{\True $(x+y)^2\geq4xy$}
	\loigiai{
		Xét hiệu
		$P=(x + y)^2 - 4xy=x^2 + 2xy + y^2 - 4xy=x^2 - 2xy + y^2=(x - y)^2$.\\
		Mà $(x-y)^2\geq0$ với mọi $x$, $y$;
		do đó $P\geq0$ với mọi $x$, $y$.\\
		Vậy $(x+y)^2\geq4xy$.
	}
\end{ex}

%%%====Câu 5
\begin{ex}%[Dự án EX-9-Đề Cương Toán 9]%[Đặng Thị Thu Thảo]%[9D2H1-2]
	Khẳng định nào sau đây đúng với mọi $a$, $b$, $c$?
	\choice
	{$a^2+b^2+c^2=ab+bc+ca$}
	{\True $a^2+b^2+c^2\geq ab+bc+ca$}
	{$a^2+b^2+c^2\leq ab+bc+ca$}
	{$a^2+b^2+c^2>ab+bc+ca$}
	\loigiai{
		Xét hiệu
		\allowdisplaybreaks
		\begin{eqnarray*}
			a^2+b^2+c^2-ab-bc-ca&=&\dfrac{1}{2}\left(2a^2+2b^2+2c^2-2ab-2bc-2ca\right)\\
			&=&\dfrac{1}{2}\left[\left(a^2-2ab+b^2\right)+\left(b^2-2bc+c^2\right)+\left(c^2-2ca+a^2\right)\right]\\
			&=&\dfrac{1}{2}\left[(a-b)^2+(b-c)^2+(c-a)^2\right]\geq0.
		\end{eqnarray*}
		Vì $(a-b)^2\geq0$; $(b-c)^2\geq0$; $(c-a)^2\geq0$ với mọi $a$, $b$, $c$.\\
		Nên $a^2+b^2+c^2\geq ab+bc+ca$.\\
		Dấu \lq\lq$=$\rq\rq\,xảy ra khi $a=b=c$.
	}
\end{ex}

%%%====Câu 6
\begin{ex}%[Dự án EX-9-Đề Cương Toán 9]%[Đặng Thị Thu Thảo]%[9D2N1-2]
		Cho $-2x+3<-2y+3$. So sánh $x$ và $y$. Khẳng định nào sau đây là đúng
	\choice
	{$x<y$}
	{\True $x>y$}
	{$x\leq y$}
	{$x=y$}
	\loigiai{Theo đề bài ta có
		\allowdisplaybreaks
		\begin{eqnarray*}	
			-2x+3&<&-2y+3\\
			-2x+3-3&<&-2y+3-3\\
			-2x&<&-2y\\
			-2x\left(-\dfrac{1}{2}\right)&>&-2y\left(-\dfrac{1}{2}\right)\\
			x&>&y.
		\end{eqnarray*}	
	}
\end{ex}

%%%====Câu 7
\begin{ex}%[Dự án EX-9-Đề Cương Toán 9]%[Đặng Thị Thu Thảo]%[9D2N2-1]
		Bất phương trình nào sau đây là bất phương trình bậc nhất một ẩn $x$?
	\choice
	{\True $-x+2<0$}
	{$-x^2<0$}
	{$-x^2+2<0$}
	{$-0x-5<0$}
	\loigiai{Áp dụng định nghĩa bất phương trình bậc nhất một ẩn ta có $-x+2<0$ là bất phương trình bậc nhất một ẩn $x$.
	}
\end{ex}

%%%====Câu 8
\begin{ex}%[Dự án EX-9-Đề Cương Toán 9]%[Đặng Thị Thu Thảo]%[9D2N2-1]
		Bất phương trình nào sau đây \textbf{không phải} là bất phương trình bậc nhất một ẩn $x$?
	\choice
	{$-5x<0$}
	{$2x+2\geq 0$}
	{\True $-x^2+2<0$}
	{$x+5<0$}
	\loigiai{  Áp dụng định nghĩa bất phương trình bậc nhất một ẩn ta có $-x^2+2<0$ không phải là bất phương trình bậc nhất một ẩn $x$ vì vế trái của bất phương trình không có dạng $ax+b$.
	}
\end{ex}

%%%====Câu 9
\begin{ex}%[Dự án EX-9-Đề Cương Toán 9]%[Đặng Thị Thu Thảo]%[9D2N2-1]
		Bất phương trình bậc nhất một ẩn $3x-5\leq 0$ có các hệ số $a$, $b$ là
	\choice
	{\True $a=3$, $b=-5$}
	{$a=3$, $b=5$}
	{$a=5$, $b=-3$}
	{$a=-5$, $b=3$}
	\loigiai{  Áp dụng định nghĩa bất phương trình bậc nhất một ẩn nên $a=3$, $b=-5$.
	}
\end{ex}

%%%====Câu 10
\begin{ex}%[Dự án EX-9-Đề Cương Toán 9]%[Đặng Thị Thu Thảo]%[9D2N2-1]
		Trong các số $-3$; $0$; $3$ những số nào là nghiệm của bất phương trình $3x-1\geq x+4$?
	\choice
	{$-3$}
	{\True $3$}
	{$0$}
	{$0$; $3$}
	\loigiai{
		Thay  $x=-3$ vào bất phương trình $3x-1\geq x+4$ ta có $-10\geq 1$ (vô lí) nên $x=-3$ không là nghiệm của bất phương trình $3x-1\geq x+4$.\\
		Thay  $x=0$ vào bất phương trình $3x-1\geq x+4$ ta có $-1\geq 4$ (vô lí) nên $x=0$ không là nghiệm của bất phương trình $3x-1\geq x+4$.\\
		Thay  $x=3$ vào bất phương trình $3x-1\geq x+4$ ta có $8\geq 7$ (thỏa mãn) nên $x=3$ là nghiệm của bất phương trình $3x-1\geq x+4$.\\
	}
\end{ex}

%%%====Câu 11
\begin{ex}%[Dự án EX-9-Đề Cương Toán 9]%[Đặng Thị Thu Thảo]%[9D2V2-3]
		Bạn Nam có $25\,000$ đồng. Nam muốn mua một cái bút chì giá $4\,000$ đồng và một
	số quyển vở giá $2\,200$ đồng một quyển. Bạn Nam có thể mua được nhiều nhất số quyển vở là
	\choice
	{$7$}
	{$8$}
	{\True $9$}
	{$10$}
	\loigiai{
		Gọi số quyển vở bạn Nam có thể mua được là $x$ (quyển).\\
		Khi đó $x$ phải thỏa mãn bất phương trình $2\,200x+4\,000\leq 25\,000$.\\
		Lần lượt thay các giá trị $x=7$; $x=8$; $x=9$; $x=10$ vào bất phương trình.\\
		Ta được $x=9$ thỏa mãn.
	}
\end{ex}

%%%====Câu 12
\begin{ex}%[Dự án EX-9-Đề Cương Toán 9]%[Đặng Thị Thu Thảo]%[9D2C2-3]
		Tổng chi phí của một doanh nghiệp sản suất áo sơ mi là $410$ triệu đồng/tháng. Giá bán của mỗi chiếc áo sơ mi là $350$ nghìn đồng. Gọi $x$ (áo, $x\in\mathbb{N}^*$) là số chiếc áo sơ mi trung bình mỗi tháng doanh nghiệp bán được. Tìm $x$ để trung bình mỗi tháng doanh nghiệp thu được lợi nhuận ít nhất là $1{,}38$ tỉ đồng sau $1$ năm.
	\choice
	{$x\geq1\,500$}
	{$x\leq1\,500$}
	{\True $x=1\,500$}
	{$x>1\,500$}
	\loigiai{
		Lợi nhuận doanh nghiệp sau $12$ tháng là $12\,(350\,000x-410\,000\,000)$ (đồng).\\
		Do đó để doanh nghiệp thu được lợi nhuận ít nhất là $1{,}38$ tỉ đồng thì ta có 
		\begin{eqnarray*}
			12(350\,000x-410\,000\,000) &\geq& 1\,380\,000\,000.\\
			350\,000x-410\,000\,000 &\geq& 115\,000\,000\\
			350\,000x&\geq&525\,000\,000\\
			x&\geq&1\,500.
		\end{eqnarray*}
		Vậy trung bình mỗi tháng doanh nghiệp bán được ít nhất $1\,500$ áo sơ mi.
	}
\end{ex}

\subsection{Bài tập tự luận}
\subsubsection{Bất đẳng thức}
\begin{dang}{Viết bất đẳng thức}
	\begin{enumerate}
			\item Số $a$ bằng số $b$, kí hiệu $a=b$.
			\item Số $a$ lớn hơn số $b$, kí hiệu $a>b$.
			\item Số $a$ nhỏ hơn số $b$, kí hiệu $a<b$.
	\end{enumerate}
	Ngoài ra ta còn có: 
	\begin{itemize}
		\item Số $a$ lớn hơn hoặc bằng số $b$, kí hiệu $a\geq b$.
		\item Số $a$ nhỏ hơn hoặc bằng số $b$, kí hiệu $a\leq b$.
	\end{itemize}
	\end{dang}
%%%=======Bài 1
\begin{bt}%[Dự án EX-9-Đề Cương Toán 9]%[Đặng Thị Thu Thảo]%
	Dùng kí hiệu để viết bất đẳng thức tương ứng với mỗi trường hợp sau
\begin{multicols}{3}
	\begin{enumerate}
		\item $x$ nhỏ hơn hoặc bằng $\dfrac{-1}{2}$.
		\item $P$ lớn hơn $3{,}5$.
		\item $b$ là số dương.
		\item $m$ là số không âm.
		\item $y$ không vượt quá $12$.
		\item $q$ không nhỏ hơn $6$.
	\end{enumerate}
\end{multicols}
\loigiai{
	\begin{multicols}{3}
		\begin{enumerate}
			\item $x\leq\dfrac{-1}{2}$.
			\item $P>3{,}5$.
			\item $b>0$.
			\item $m\geq0$.
			\item $y\leq12$.
			\item $q\geq6$.
		\end{enumerate}
	\end{multicols}
}
\end{bt}

%%%=======Bài 2
\begin{bt}%[Dự án EX-9-Đề Cương Toán 9]%[Đặng Thị Thu Thảo]%[9D2V1-1]
		Viết bất đẳng thức phù hợp trong mỗi trường hợp sau
	\begin{enumerate}
		\item Xe ô tô $5$ chỗ được phép chở tối đa $6$ người bao gồm cả tài xế.
		\item Lương của bố Hùng dưới $20$ triệu.
		\item Tuổi của Lan không vượt quá  $25$.
		\item Thời gian tự học của An trong một ngày ít nhất là $2{,}5$ giờ.
		\item Tốc độ tối đa của ô tô trong khu dân cư đối với đường có dải phân cách cứng là $60$ km/h.
	\end{enumerate}
	\loigiai{
		\begin{enumerate}
			\item Gọi $x$ (người, $x>0$)  là số người mà xe ô tô chở. Khi đó $x\leq6$.
			\item Gọi $y$ (triệu đồng) là tiền lương của bố Hùng. Khi đó $y<20$.
			\item Gọi $t$ (tuổi) là tuổi của Lan. Khi đó $t\leq25$.
			\item Gọi $z$ (giờ) là thời gian tự học của An trong một ngày. Khi đó $z\geq2{,}5$ .
			\item Gọi $v$ (km/h) là tốc độ của ô tô trong khu dân cư có dải phân cách cứng. Khi đó  $v\leq60$.
		\end{enumerate}
	}
\end{bt}
\begin{dang}{So sánh và chứng minh bất đẳng thức}
	\begin{itemize}
		\item Để so sánh $2$ số ta vận dụng liên hệ giữa thứ tự và phép cộng, liên hệ giữa thứ tự và phép nhân, tính chất bắc cầu hoặc biến đổi để so sánh trực tiếp.
		\item Để chứng minh bất đẳng thức
		\begin{itemize}
			\item Muốn chứng minh $A > B$ (hay $A < B$) ta chứng minh $A - B > 0$ (hay $A - B < 0$).
			\item Dùng các tính chất của bất đẳng thức, tính chất bắc cầu.
			\item Từ các bất đẳng thức đã biết ta dùng các tính chất của bất đẳng thức để suy ra bất đẳng thức cần chứng minh.
		\end{itemize}
	\end{itemize}
\end{dang}
%%%=======Bài 3
\begin{bt}%[Dự án EX-9-Đề Cương Toán 9]%[Đặng Thị Thu Thảo]%[9D2H1-2]
		Cho $a>b$, hãy so sánh
	\begin{multicols}{2}
		\begin{enumerate}
			\item $a+5$ và $b+5$.
			\item $a-12$ và $b-12$.
			\item $7a$ và $7b$.
			\item $4-a$ và $4-b$.
			\item $3a+5$ và $3b+5$.
			\item $10-5a$ và $10-5b$.
			\item $5a+2$ và $5b-1$.
			\item $9-7a$ và $5-7b$.
		\end{enumerate}
	\end{multicols}
	\loigiai{
		\begin{enumerate}
			\item Vì $a>b$ nên $a+5>b+5$.
			\item Vì $a>b$ nên $a+(-12)>b+(-12)$ hay $a-12>b-12$.
			\item Vì $a>b$ và $7>0$ nên $7a>7b$.
			\item Vì $a>b$ nên $-a<-b$, suy ra $4-a<4-b$.
			\item Vì $a>b$ nên $3a>3b$ suy ra $3a+5>3b+5$.
			\item Vì $a>b$ nên $-5a>-5b$ suy ra $10-5a<10b-5b$.
			\item Vì $a>b$ nên $5a>5b$, suy ra $5a+2>5b+2$.\\
			Lại có $2>-1$ nên $5b+2>5b-1$.\\
			Từ đó suy ra $5a+2>5b-1$.
			\item Vì $a>b$ và $-7<0$ nên $-7a<-7b$, suy ra $5-7a<5-7b$.\\
			Lại có $5<9$ nên $5-7b<9-7b$.\\
			Vậy $5-7a<9-7b$.
		\end{enumerate}
	}
\end{bt}

%%%=======Bài 4
\begin{bt}%[Dự án EX-9-Đề Cương Toán 9]%[Đặng Thị Thu Thảo]%[9D2V1-2]
	So sánh
	\begin{multicols}{3}
		\begin{enumerate}
			\item $\dfrac{2025}{2024}$ và $\dfrac{2022}{2023}$.
			\item $\dfrac{2018}{2019}$ và $\dfrac{2019}{2020}$.
			\item $\dfrac{2023}{2025}$ và $\dfrac{2025}{2027}$.
		\end{enumerate}
	\end{multicols}
	\loigiai{
		\begin{enumerate}
			\item Ta có $\dfrac{2025}{2024}>\dfrac{2024}{2024}=1$, mà $1=\dfrac{2023}{2023}>\dfrac{2022}{2023}$.\\
			Vậy $\dfrac{2025}{2024}>\dfrac{2022}{2023}$.
			\item Ta có $\dfrac{1}{2019}>\dfrac{1}{2020}$ nên $-\dfrac{1}{2019}<-\dfrac{1}{2020}$.\\
			Suy ra $1-\dfrac{1}{2019}<1-\dfrac{1}{2020}$ hay $\dfrac{2018}{2019}<\dfrac{2019}{2020}$.
			\item Ta có $\dfrac{2}{2025}>\dfrac{2}{2027}$ nên $-\dfrac{2}{2025}<-\dfrac{2}{2027}$.\\
			Suy ra $1-\dfrac{2}{2025}<1-\dfrac{2}{2027}$.\\
			Vậy $\dfrac{2023}{2025}<\dfrac{2025}{2027}$.
		\end{enumerate}
	}
\end{bt}

%%%=======Bài 4-1
\begin{bt}%[Dự án EX-9-Đề Cương Toán 9]%[Đặng Thị Thu Thảo]%[9D2V1-2]
	Chứng minh
	\begin{multicols}{2}
		\begin{enumerate}
			\item  $3a+2b+5\geq a+3b+5$ với $2a\geq b$.
			\item $4a-b\leq 3a+2b$ với $3b\geq a$.
			\item $m-2n-\dfrac{1}{5}>3n-m-\dfrac{1}{3}$ với $5n\leq2m$.
			\item $(a-1)^2\geq 3-2a$ với $a^2\geq 2$.
		\end{enumerate}
	\end{multicols}
	\loigiai{
		\begin{enumerate}
			\item Xét hiệu $3a+2b+5-a-3b-5=2a-b$, mà $2a\geq b$ nên $2a-b\geq0$.\\
			Vậy $3a+2b+5\geq a+3b+5$.
			\item Xét hiệu $4a-b-3a-2b=a-3b$, mà $3b\geq a$ nên $a-3b\leq0$.\\
			Vậy $4a-b\leq3a+2b$.
			\item Xét hiệu $m-2n-\dfrac{1}{5}-3n+m+\dfrac{1}{3}=2m-5n+\dfrac{2}{15}$, mà $5n\leq2m$ nên $2m-5n\geq0$,\\
			suy ra $2m-5n+\dfrac{2}{15}\geq\dfrac{2}{15}>0$.\\
			Vậy $m-2n-\dfrac{1}{5}>3n-m-\dfrac{1}{3}$.
			\item Xét hiệu $(a-1)^2-(3-2a)=a^2-2a+1-3+2a=a^2-2$, mà $a^2\geq2$ nên $a^2-2\geq0$.\\
			Vậy $(a-1)^2\geq3-2a$.
		\end{enumerate}
	}
\end{bt}

%%%=======Bài 4-2
\begin{bt}%[Dự án EX-9-Đề Cương Toán 9]%[Đặng Thị Thu Thảo]%[9D2C1-2]
	Với $x$, $y$ là các số dương. Chứng minh rằng
	\begin{multicols}{2}
		\begin{enumerate}
			\item $2\left(a^2+b^2\right)\geq(a+b)^2$.
			\item $3\left(a^2+b^2+c^2\right)\geq(a+b+c)^2$.
			\item $(a+b+c)^2\geq3(ab+bc+ca)$.
			\item $a^2+b^2+c^2+3\geq2(a+b+c)$.
			\item  $(x+y)^2\geq4xy$.
			\item $(x+y)\left(\dfrac{1}{x}+\dfrac{1}{y}\right)\geq4$.
			\item $x^2+y^2+xy\geq\dfrac{3(x+y)^2}{4}$.
			\item $\dfrac{1}{x^2}+\dfrac{1}{y^2}\geq\dfrac{8}{(x+y)^2}$.
		\end{enumerate}
	\end{multicols}
	\loigiai{
		\allowdisplaybreaks
		\begin{enumerate}
			\item Xét hiệu $2\left(a^2+b^2\right)-(a+b)^2=2a^2+2b^2-a^2-2ab-b^2=a^2-2ab+b^2=(a-b)^2\geq0$.\\
			Vậy $2\left(a^2+b^2\right)\geq(a+b)^2$.
			\item Xét hiệu
			\begin{eqnarray*}
				3\left(a^2+b^2+c^2\right)-(a+b+c)^2&=&3a^2+3b^2+3c^2-a^2-b^2-c^2-2ab-2ac-2bc\\
				&=&2a^2+2b^2+2c^2-2ab-2ac-2bc\\
				&=&\left(a^2-2ab+b^2\right)+\left(a^2-2ac+c^2\right)+\left(b^2-2bc+c^2\right)\\
				&=&(a-b)^2+(a-c)^2+(b-c)^2.
			\end{eqnarray*}
			Mà $(a-b)^2\geq0$; $(a-c)^2\geq0$; $(b-c)^2\geq0$.\\
			Suy ra $(a-b)^2+(a-c)^2+(b-c)^2\geq0$.\\
			Vậy $3\left(a^2+b^2+c^2\right)\geq(a+b+c)^2$.
			\item Xét hiệu
			\begin{eqnarray*}
				(a+b+c)^2-3(ab+bc+ca)&=&a^2+b^2+c^2+2ab+2bc+2ac-3ab-3bc-3ca\\
				&=&a^2+b^2+c^2-ab-bc-ca\\
				&=&\dfrac{1}{2}\left(2a^2+2b^2+2c^2-2ab-2bc-2ca\right)\\
				&=&\dfrac{1}{2}\left[\left(a^2-2ab+b^2\right)+\left(b^2-2bc+c^2\right)+\left(c^2-2ca+a^2\right)\right]\\
				&=&\dfrac{1}{2}\left[(a-b)^2+(b-c)^2+(c-a)^2\right].
			\end{eqnarray*}
			Mà $(a-b)^2\geq0$; $(a-c)^2\geq0$; $(b-c)^2\geq0$.\\
			Suy ra $\dfrac{1}{2}\left[(a-b)^2+(b-c)^2+(c-a)^2\right]\geq0$.\\
			Vậy $(a+b+c)^2\geq3(ab+bc+ca)$.
			\item Xét hiệu
			\begin{eqnarray*}
				a^2+b^2+c^2+3-2(a+b+c)&=&a^2+b^2+c^2+3-2a-2b-2c\\
				&=&\left(a^2-2a+1\right)+\left(b^2-2b+1\right)+\left(c^2-2c+1\right)\\
				&=&(a-1)^2+(b-1)^2+(c-1)^2.
			\end{eqnarray*}
			Mà $(a-1)^2\geq0$; $(b-1)^2\geq0$; $(c-1)^2\geq0$.\\
			Vậy $a^2+b^2+c^2+3\geq2(a+b+c)$.
			\item Xét hiệu
			$(x+y)^2-4xy=x^2+2xy+y^2-4xy=x^2-2xy+y^2=(x-y)^2\geq0$.\\
			Vậy $(x+y)^2\geq4xy$.
			\item Xét hiệu 
			$$(x+y)\left(\dfrac{1}{x}+\dfrac{1}{y}\right)-4=1+\dfrac{x}{y}+\dfrac{y}{x}+1-4=\dfrac{x}{y}+\dfrac{y}{x}-2=\dfrac{x^2+y^2-2xy}{xy}=\dfrac{(x-y)^2}{xy}.$$
			Mà $(x-y)^2\geq0$; $xy>0$.
			\\
			Suy ra $\dfrac{(x-y)^2}{xy}\geq0$.\\
			Vậy $(x+y)\left(\dfrac{1}{x}+\dfrac{1}{y}\right)\geq4$.
			\item Xét hiệu
			$$x^2+y^2+xy-\dfrac{3(x+y)^2}{4}=\dfrac{4x^2+4y^2+4xy-3x^2-6xy-3y^2}{4}=\dfrac{x^2-2xy+y^2}{4}=\dfrac{(x-y)^2}{4}.$$
			Mà $(x-y)^2\geq0$, suy ra $\dfrac{(x-y)^2}{4}\geq0$.\\
			Vậy $x^2+y^2+xy\geq\dfrac{3(x+y)^2}{4}$.
			\item Xét hiệu
			\begin{eqnarray*}
				\dfrac{1}{x^2}+\dfrac{1}{y^2}-\dfrac{8}{(x+y)^2}&=&\dfrac{x^2+y^2}{(xy)^2}-\dfrac{8}{(x+y)^2}\\
				&=&\dfrac{\left(x^2+y^2\right)(x+y)^2-8x^2y^2}{\left[xy(x+y)\right]^2}\\
				&=&\dfrac{\left(x^2+y^2\right)\left(x^2+2xy+y^2\right)-8x^2y^2}{xy(x+y)]^2}\\
				&=&\dfrac{\left(x^2+y^2\right)^2+2x^3y+2xy^3-8x^2y^2}{[xy(x+y)]^2}\\
				&=&\dfrac{x^4+2x^3y+x^2y^2+x^2y^2+2xy^3+y^4-8x^2y^2}{[xy(x+y)]^2}\\
				&=&\dfrac{x^4+y^4+2x^3y+2xy^3-6x^2y^2}{[xy(x+y)]^2}\\
				&=&\dfrac{\left(x^4-2x^2y^2+y^4\right)+\left(2x^3y-2x^2y^2\right)+\left(2xy^3-2x^2y^2\right)}{[xy(x+y)]^2}\\
				&=&\dfrac{\left(x^2-y^2\right)^2+2x^2y(x-y)+2xy^2(y-x)}{[xy(x+y)]^2}\\
				&=&\dfrac{\left(x^2-y^2\right)^2+2x^2y(x-y)-2xy^2(x-y)}{[xy(x+y)]^2}\\
				&=&\dfrac{\left(x^2-y^2\right)^2+(x-y)\left(2x^2y-2xy^2\right)}{[xy(x+y)]^2}\\
				&=&\dfrac{\left(x^2-y^2\right)^2+2xy(x-y)^2}{[xy(x+y)]^2}.
			\end{eqnarray*}
			Mà $\left(x^2-y^2\right)^2\geq0$; $2xy>0$; $(x-y)^2\geq0$; $[xy(x+y)]^2>0$.\\
			Vậy $\dfrac{1}{x^2}+\dfrac{1}{y^2}\geq\dfrac{8}{(x+y)^2}$.
		\end{enumerate}
	}
\end{bt}

\begin{dang}{Áp dụng bất đẳng thức để tìm giá trị lớn nhất (GTLN), giá trị nhỏ nhất (GTNN) của một biểu thức}
	\begin{itemize}
		\item Nếu $f(x)\geq k$ ($k$ là hằng số). Dấu bằng là GTNN của $f(x)$, xảy ra khi và chỉ khi $x=a$.
		\item Nếu $f(x)\leq k$ ($k$ là hằng số). Dấu bằng là GTLN của $f(x)$, xảy ra khi và chỉ khi $x=a$.
	\end{itemize}
\end{dang}
%%%=======Bài 5
\begin{bt}%[Dự án EX-9-Đề Cương Toán 9]%[Đặng Thị Thu Thảo]%[9D2V1-2]
		Tìm giá trị nhỏ nhất (GTNN) của biểu thức
	\begin{multicols}{2}
		\begin{enumerate}
			\item $A=x^2-6x+14$.
			\item $B=7x^2-14x+3$.
			\item $C=3x^2+12x+2y^2-20y+65$.
			\item $D=\dfrac{6x+1}{x^2+8}$.
		\end{enumerate}
	\end{multicols}
	\loigiai{
		\begin{enumerate}
			\item Ta có $A=x^2-6x+14=(x-3)^2+5$.\\
			Vì $(x-3)^2\geq0$ với mọi $x$ nên $(x-3)^2+5\geq5$.\\
			Suy ra GTNN của $A=5$. Dấu bằng xảy ra khi $x=3$.
			\item Ta có $B=7x^2-14x+3=7(x-1)^2-4$.\\
			Vì $(x-1)^2\geq0$ với mọi $x$ nên $7(x-1)^2-4\geq-4$.\\
			Suy ra GTNN của $B=-4$. Dấu bằng xảy ra khi $x=1$.
			\item Có $C=3x^2+12x+2y^2-20y+65=3(x+2)^2+2(y-5)^2+3$.\\
			Vì $3(x+2)^2+2(y-5)^2+3\geq3$ với mọi $x$, $y$.\\
			Suy ra GTNN của $C=3$. Dấu bằng xảy ra khi $x=-2$; $y=5$.
			\item Có $D=\dfrac{6x+1}{x^2+8}=\dfrac{x^2+6x+9-x^2-8}{x^2+8}=\dfrac{(x+3)^2}{x^2+8}-\dfrac{x^2+8}{x^2+8}=\dfrac{(x+3)^2}{x^2+8}-1$.\\
			Vì $(x+3)^2\geq0$ và $x^2+8>0$ với mọi $x$ nên $\dfrac{(x+3)^2}{x^2+8}-1\geq-1$.\\
			Suy ra GTNN của $D=-1$. Dấu bằng xảy ra khi $x=-3$.
		\end{enumerate}
	}
\end{bt}

%%%=======Bài 6
\begin{bt}%[Dự án EX-9-Đề Cương Toán 9]%[Đặng Thị Thu Thảo]%[9D2V1-2]
		Tìm giá trị lớn nhất (GTLN) của biểu thức
	\begin{multicols}{2}
		\begin{enumerate}
			\item $A=-x^2+5x-1$.
			\item $B=-x^2+3x$.
			\item $C=\dfrac{5}{x^2-6x+13}$.
			\item $D=\dfrac{4x^2+8x+17}{x^2+2x+3}$.
		\end{enumerate}
	\end{multicols}
	\loigiai{
		\begin{enumerate}
			\item Ta có $A=-x^2+5x-1=-\left[\left(x-\dfrac{5}{2}\right)^2-\dfrac{21}{4}\right]$.\\
			Vì $\left(x-\dfrac{5}{2}\right)^2\geq0$ với mọi $x$ nên $-\left[\left(x-\dfrac{5}{2}\right)^2-\dfrac{21}{4}\right]\leq\dfrac{21}{4}$.\\
			Suy ra GTLN của $A=\dfrac{21}{4}$. Dấu bằng xảy ra khi $x=\dfrac{5}{2}$.
			\item Có $B=-x^2+3x=-\left(x^2-3x\right)=-\left[\left(x-\dfrac{3}{2}\right)^2-\dfrac{9}{4}\right]$.\\
			Vì $\left(x-\dfrac{3}{2}\right)^2\geq0$ với mọi $x$ nên $-\left[\left(x-\dfrac{3}{2}\right)^2-\dfrac{9}{4}\right]\leq\dfrac{9}{4}$.\\
			Suy ra GTLN của $B=\dfrac{9}{4}$. Dấu bằng xảy ra khi $x=\dfrac{3}{2}$.
			\item Ta có $C=\dfrac{5}{x^2-6x+13}=\dfrac{5}{(x-9)^2+4}\leq\dfrac{5}{4}$.\\
			Suy ra GTLN của $C=\dfrac{5}{4}$. Dấu bằng xảy ra khi $x=9$.
			\item Ta có $D=\dfrac{4x^2+8x+17}{x^2+2x+3}=4+\dfrac{5}{x^2+2x+3}=4+\dfrac{5}{(x+1)^2+2}\leq4+\dfrac{5}{2}=\dfrac{13}{2}$.\\
			Suy ra GTLN của $D=\dfrac{13}{2}$. Dấu bằng xảy ra khi $x=-1$.
		\end{enumerate}    
	}
\end{bt}

\begin{dang}{Bài toán thực tế}
	\begin{itemize}
		\item Gọi ẩn và biểu diễn đại lượng chưa biết theo ẩn. 
		\item Từ dữ kiện của đề bài, lập bất đẳng thức và vận dụng các tính chất của bất đẳng thức để biến đổi.
	\end{itemize}
\end{dang}
%%%=======Bài 7
\begin{bt}%[Dự án EX-9-Đề Cương Toán 9]%[Đặng Thị Thu Thảo]%[9D2V1-3]
		Một ca nô đi ngược dòng trong $2$ giờ. Biết vận tốc của ca nô khi nước yên lặng lớn hơn $15$ km/h và tốc độ của nước là $3$ km/h. Chứng minh quãng đường ca nô đi được trong thời gian trên lớn hơn $24$ km.
	\loigiai{
		Gọi vận tốc của ca nô khi nước yên lặng là $x$ (km/h; $x>15$).\\
		Vận tốc của ca nô khi ngược dòng là $x-3$ (km/h).\\
		Ta có $x>15$ nên $x-3>15-3$, tức là $x-3>12$.\\
		Gọi $s$ là quãng đường ca nô đi ngược dòng trong $2$ giờ.\\
		Ta có $s=2\cdot(x-3)$ (km).\\
		Do $x-3>12$ nên $2\cdot(x-3)>2\cdot12=24$ hay $s>24$.\\
		Vậy quãng đường ca nô đi được trong thời gian trên lớn hơn $24$ km.
	}
\end{bt}

%%%=======Bài 8
\begin{bt}%[Dự án EX-9-Đề Cương Toán 9]%[Đặng Thị Thu Thảo]%[9D2V1-3]
		Bác Phúc muốn rào xung quanh mảnh vườn hình chữ nhật có chiều rộng là $x$ (m), chiều dài hơn chiều rộng $5$ m. Bác Khang ước lượng chiều rộng nhỏ hơn $20$ m. Bác có tấm lưới dài khoảng $95$ m. Hỏi tấm lưới này có đủ dài để rào vườn không? Vì sao?
	\loigiai{
		Ta có chiều rộng của mảnh vườn hình chữ nhật là $x$ (m; $0<x<20$).\\
		Chiều dài mảnh vườn hình chữ nhật là $x+5$ (m).\\
		Chu vi mảnh vườn hình chữ nhật là $(x+x+5)\cdot2=4x+10$ (m).\\
		Vì chiều rộng nhỏ hơn $20$ m nên 
		\begin{eqnarray*}
			x&<&20\\
			4x&<&4\cdot20=80\\
			4x+10&<&80+10=90.
		\end{eqnarray*}
		Vì chu vi mảnh vườn nhỏ hơn $90$ m nên tấm lưới $95$ m đủ để rào mảnh vườn.
	}
\end{bt}
\subsubsection{Bất phương trình bậc nhất một ẩn}
\begin{dang}{Xác định bất phương trình bậc nhất một ẩn}
	Bất phương trình bậc nhất một ẩn là bất phương trình có dạng $ax + b < 0$ (hay $ax + b > 0$; $ax + b \geq 0$; $ax + b \leq 0$) trong đó $a$ và $b$ là hai số đã cho và $a \neq 0$.
\end{dang}
%%%=======Bài 9
\begin{bt}%[Dự án EX-9-Đề Cương Toán 9]%[Đặng Thị Thu Thảo]%[9D2N2-1]
	 Hãy xét xem các bất phương trình sau có là bất phương trình bậc nhất một ẩn hay không?
	\begin{multicols}{4}
		\begin{enumerate}
			\item $0x - 2024 \geq 0$;
			\item $2024x + 2025 < 0$;
			\item $-\dfrac{1}{11}x \leq 0$;
			\item $\dfrac{x^2}{2} - 1 > 0$.
		\end{enumerate}
	\end{multicols}
	\loigiai{
		\begin{enumerate}
			\item $0x + 2024 \geq 0$ không phải bất phương trình bậc nhất một ẩn. Vì hệ số của ẩn $x$ là $0$.
			\item $2024x + 2025 < 0$ là bất phương trình bậc nhất một ẩn.
			\item $-\dfrac{1}{11}x \leq 0$ là bất phương trình bậc nhất một ẩn.
			\item $\dfrac{x^2}{2} - 1 > 0$ không là bất phương trình bậc nhất một ẩn. Vì $x^2$ là ẩn bậc hai.
		\end{enumerate}
	}
\end{bt}
\begin{dang}{Giải bất phương trình bậc nhất một ẩn và các bất phương trình đưa được về dạng bất phương trình bậc nhất một ẩn}
	Bất phương trình bậc nhất một ẩn bậc nhất một ẩn $ax + b < 0$ $\left(a \neq 0 \right)$ được giải như sau
	\begin{eqnarray*}
		ax + b < 0 \text{ suy ra } ax < -b.
	\end{eqnarray*}
	Nếu $a > 0$ thì $x < \dfrac{-b}{a}$.\\
	Nếu $a < 0$ thì $x > \dfrac{-b}{a}$.
\end{dang}
%%%=======Bài 10
\begin{bt}%[Dự án EX-9-Đề Cương Toán 9]%[Đặng Thị Thu Thảo]%[9D2H2-1]
	Giải các bất phương trình sau
	\begin{multicols}{3}
		\begin{enumerate}
			\item $2x - 8 > 0$;
			\item $9 - 3x \leq 0$;
			\item $5 -\dfrac{1}{3}x < 1$;
			\item $5x - 19 > 7x + 23$;
			\item $\dfrac{2x}{5} - \dfrac{x}{4} \leq 9$;
			\item $5(x+1) < 7 - (3 - 5x)$;
			\item $\dfrac{x-5}{12} \geq \dfrac{2x-3}{8}$.   
			
		\end{enumerate}
	\end{multicols}
	\loigiai{
		\begin{enumerate}
			\item Ta có 
			\allowdisplaybreaks
			\begin{eqnarray*}
				2x - 8 &>& 0\\
				2x &>& 8\\
				x &>& \dfrac{8}{4}\\
				x &>& 4.
			\end{eqnarray*}
			Vậy bất phương trình đã cho có nghiệm là $x > 4$.
			\item Ta có 
			\allowdisplaybreaks
			\begin{eqnarray*}
				9 - 3x &\leq& 0\\
				-3x &\leq& -9\\
				x &\geq& \dfrac{-9}{-3}\\
				x &\geq& 3.
			\end{eqnarray*}
			Vậy bất phương trình đã cho có nghiệm là $x \geq 3$.
			\item Ta có 
			\allowdisplaybreaks
			\begin{eqnarray*}
				5 - \dfrac{1}{3} x &<& 1\\
				-\dfrac{1}{3}x &<& 1 - 5\\
				x &>& \left(-4 \right) : \left(-\dfrac{1}{3} \right)\\
				x &>& 12.
			\end{eqnarray*}
			Vậy nghiệm của bất phương trình đã cho là $x > 12$.
			\item Ta có
			\allowdisplaybreaks
			\begin{eqnarray*}
				5x - 19 &>& 7x + 23\\
				5x - 7x &>& 23 + 19\\
				-2x &>& 42\\
				x &<& -21.
			\end{eqnarray*}
			Vậy nghiệm của bất phương trình đã cho là $x < -21$.
			\item Ta có
			\allowdisplaybreaks
			\begin{eqnarray*}
				\dfrac{2x}{5} - \dfrac{x}{4} &\le& 9\\
				\dfrac{8x}{20} - \dfrac{5x}{20} &\le& \dfrac{180}{20}\\
				8x - 5x &\le& 180\\
				3x &\le& 180\\
				x &\le& 60.
			\end{eqnarray*}
			Vậy nghiệm của bất phương trình đã cho là $x \le 60$.
			\item Ta có
			\allowdisplaybreaks
			\begin{eqnarray*}
				5(x+1) &<& 7 - (3-5x)\\
				5x + 5 &<& 7 - 3 + 5x\\
				5x - 5x &<& 4 - 5\\
				0x &<& -1 \text{ (vô lí)}.
			\end{eqnarray*}
			Vậy bất phương trình đã cho vô nghiệm.
			\item Ta có
			\allowdisplaybreaks
			\begin{eqnarray*}
				\dfrac{x-5}{12} &\ge& \dfrac{2x-3}{8}\\
				\dfrac{2(x-5)}{24} &\ge& \dfrac{3(2x-3)}{24}\\
				2x - 10 &\ge& 6x - 9\\
				2x - 6x &\ge& 10 - 9\\
				-4x &\ge& 1\\
				x &\le& -\dfrac{1}{4}.
			\end{eqnarray*}
			Vậy nghiệm của bất phương trình đã cho là $x \le -\dfrac{1}{4}$.
		\end{enumerate}
	}
\end{bt}

\begin{dang}
{Toán thực tế}
Vận dụng các kiến thức về bất phương trình bậc nhất một ẩn để làm các bài toán liên quan trong thực tế.
\end{dang}
%\setcounter{bt}{0}
%%%=======Bài 11
\begin{bt}%[Dự án EX-9-Đề Cương Toán 9]%[Đặng Thị Thu Thảo]%[9D2V2-3]
	Để lập đội tuyển năng khiếu về bóng rổ của trường, thầy thể dục đưa ra quy định tuyển chọn như sau: mỗi bạn dự tuyển sẽ được ném $15$ quả bóng vào rổ, quả bóng vào rổ được cộng $2$ điểm; quả bóng ném ra ngoài bị trừ $1$ điểm. Nếu bạn nào có số điểm từ $15$ điểm trở lên thì sẽ được chọn vào đội tuyển. Hỏi một học sinh muốn được chọn vào đội tuyển thì phải ném ít nhất bao nhiêu quả vào rổ?
\loigiai{
	Gọi $x$ là số quả bóng học sinh cần ném vào rổ ($0 \leq x \leq 15$, $x \in \mathbb{N}^*$).\\
	Số quả bóng ném ra ngoài là $15-x$ (quả).\\
	Ném vào rổ $x$ quả bóng được cộng $2x$ (điểm).\\
	Ném ra ngoài $15-x$ quả bóng bị trừ $15-x$ (điểm).\\
	Vì vậy, sau khi ném $15$ quả bóng thì học sinh đó sẽ có số điểm là
	$$
	2x-(15-x)=2x-15+x=3x-15~(\text{điểm}).
	$$
	Theo bài, để được vào đội tuyển thì học sinh cần có số điểm từ $15$ trở lên, nên ta có bất phương trình
	\begin{eqnarray*}
		3x-15 &\ge& 15 \\
		3x &\ge& 30 \\
		x &\ge& 10.
	\end{eqnarray*}
	Mà $0 \leq x \leq 15$, $x \in \mathbb{N}^*$ nên học sinh đó cần phải ném vào rổ ít nhất là $10$ quả bóng thì mới được chọn vào đội tuyển.
}
\end{bt}

%%%=======Bài 12
\begin{bt}%[Dự án EX-9-Đề Cương Toán 9]%[Đặng Thị Thu Thảo]%[9D2V2-1]
	  Một người có số tiền không quá $70\,000$ gồm $15$ tờ giấy bạc với hai loại mệnh giá: loại $2\,000$ đồng và loại $5\,000$ đồng. Hỏi người đó có bao nhiêu tờ giấy bạc loại $5\,000$ đồng?
	\loigiai{
		Gọi số tờ giấy bạc loại $5\,000$ đồng là $x\, \left(x \in \mathbb{N}^*, x < 15 \right)$.\\
		Số tờ giấy bạc loại $2\,000$ đồng là $15 - x$.\\
		Theo đề bài ra ta có bất phương trình
		\allowdisplaybreaks
		\begin{eqnarray*}
			\left(15 - x\right) \cdot 2\,000 + x \cdot 5\,000 &\leq& 70\,000\\
			\left(15 - x\right) \cdot 2 + 5x &\leq& 70\\
			3x &\leq& 40\\
			x &\leq& \dfrac{40}{3}.
		\end{eqnarray*}
		Mà $x \in \mathrm{N}^*$, $x < 15$ suy ra $x$ là các số nguyên từ $1$ đến $13$.\\
		Vậy số tờ giấy bạc loại $5\,000$ đồng là các số nguyên từ $1$ đến $13$.
	}
\end{bt}

%%%=======Bài 13
\begin{bt}%[Dự án EX-9-Đề Cương Toán 9]%[Đặng Thị Thu Thảo]%[9D2V2-1]
	 Một người đi bộ một quãng đường dài $18$ km trong khoảng thời gian không nhiều hơn $4$ giờ. Lúc đầu người đó đi với vận tốc $5$ km/h, về sau đi với vận tốc $4$ km/h. Xác định độ dài đoạn đường mà người đó đã đi với vận tốc $5$ km/h.
	\loigiai{
		Gọi quãng đường mà người đó đã đi với vận tốc $5$ km/h là $x$ (km, $0 < x < 18$).\\
		Theo đề bài ra ta có bất phương trình
		\begin{eqnarray*}
			\dfrac{x}{5} + \dfrac{18 - x}{4} &\leq& 4\\
			4x + 90 - 5x &\leq& 80\\
			x &\geq& 10.
		\end{eqnarray*}
		Mà $0 < x < 18$ suy ra $10 \leq x < 18$.\\
		Vậy quãng đường mà người đó đã đi với vận tốc $5$ km/h là $x$ (km) thỏa mãn $10 \leq x < 18$.
	}
\end{bt}
% In đáp án trắc nghiệm
\Closesolutionfile{ans}
\indapan{6}{ans/ans-9C2-OTC}