\section{PHƯƠNG TRÌNH QUY VỀ PHƯƠNG TRÌNH BẬC NHẤT MỘT ẨN} % Tên bài
\subsection{Phương trình tích}
\subsubsection{Kiến thức trọng tâm}
\begin{tomtat}
	\begin{itemize}
		\item \textit{Phương trình tích} là phương trình có dạng $(ax+b)(cx+d)= 0$.
		\item Muốn giải phương trình $(ax+b)(cx+d)= 0$, ta giải hai phương trình $ax+b = 0$ và $cx+d=0$, rồi lấy tất cả các nghiệm của chúng.
	\end{itemize}
	\begin{luuy}
		Trong nhiều trường hợp, để giải phương trình, ta biến đổi để đưa về phương trình đó về dạng phương trình tích.\\
		Nhắc lại hằng đẳng thức số $3$: $A^2 - B^2 = (A + B)(A - B)$.
	\end{luuy}
\end{tomtat}

\begin{vd}%[Dự án EX-9-Đề Cương Toán 9]%[Hoàng Thanh Phương]%[9D1N1-1] 
	Giải các phương trình
	\begin{multicols}{2}
		\begin{enumerate}
			\item $3x(x+7)=0$;
			\item $(x-5)(2x-4)=0$.
		\end{enumerate}
	\end{multicols}
	\loigiai{
		\begin{enumerate}
			\item Ta có
			\allowdisplaybreaks
			\begin{eqnarray*}
				&& 3x(x+7)=0 \\
				&& 3x=0 \text{ hoặc } x+7=0	\\
				&& x=0 \text{ hoặc } x=-7.
			\end{eqnarray*}
			Vậy phương trình đã cho có hai nghiệm là $x=0$; $x=-7$.
			\item Ta có
			\allowdisplaybreaks
			\begin{eqnarray*}
				&& (x-5)(2x-4)=0 \\
				&& x-5=0 \text{ hoặc } 2x-4=0 \\
				&& x=5 \text{ hoặc } x=2.
			\end{eqnarray*}
			Vậy phương trình đã cho có hai nghiệm là $x=5$; $x=2$.
		\end{enumerate}
	}
\end{vd}

\begin{vd}%[Dự án EX-9-Đề Cương Toán 9]%[Hoàng Thanh Phương]%[9D1H1-1] 
	Giải các phương trình sau bằng cách đưa về phương trình tích
	\begin{multicols}{2}
		\begin{enumerate}
			\item $x^2+7x=0$;
			\item $x^2-4x+3=0$.
		\end{enumerate}
	\end{multicols}
	\loigiai{
		\begin{enumerate}
			\item Ta có
			\allowdisplaybreaks
			\begin{eqnarray*}
				&& x^2+7x=0 \\
				&& x(x+7)=0 \\
				&& x=0 \text{ hoặc } x+7=0 \\
				&& x=0 \text{ hoặc } x=-7.
			\end{eqnarray*}
			Vậy phương trình đã cho có hai nghiệm là $x=0$; $x=-7$.
			\item Ta có
			\allowdisplaybreaks
			\begin{eqnarray*}
				&&x^2-4x+3=0  \\
				&& x^2-x-3x+3=0 \\
				&&x(x-1)-3(x-1)=0 \\
				&&(x-3)(x-1)=0 \\
				&&x-3=0 \text{ hoặc } x-1=0 \\
				&&x=3 \text{ hoặc } x=1.
			\end{eqnarray*}
			Vậy phương trình đã cho có hai nghiệm là $x=3$; $x=1$.
		\end{enumerate}
	}
\end{vd}
\begin{vd}%[Dự án EX-9-Đề Cương Toán 9]%[Hoàng Thanh Phương]%[9D1H1-1] 
	Giải các phương trình sau
	\begin{multicols}{2}
		\begin{enumerate}
			\item $(3x+2)^2-4x^2=0$.
			\item $(2x-5)^2-9x^2=0$.
		\end{enumerate}
	\end{multicols}
		\loigiai{
		\begin{enumerate}
			\item Ta có
			\allowdisplaybreaks
			\begin{eqnarray*}
				&& (3x+2)^2-4x^2=0 \\
				&& (3x+2+2x)(3x+2-2x)=0 \\
				&& (5x+2)(x+2)=0 \\
				&& 5x+2=0 \text{ hoặc } x+2=0 \\
				&& x=-\dfrac{2}{5} \text{ hoặc } x=-2.
			\end{eqnarray*}
			Vậy phương trình đã cho có hai nghiệm là $x=-\dfrac{2}{5}$; $x=-2$. 
			\item Ta có
			\allowdisplaybreaks
			\begin{eqnarray*}
				&& (2x-5)^2-9x^2=0 \\
				&& (2x-5-3x)(2x-5+3x)=0 \\
				&& (-x-5)(5x-5)=0 \\
				&& -x-5=0 \text{ hoặc } 5x-5=0 \\
				&& x=-5 \text{ hoặc } x=1.
			\end{eqnarray*}
			Vậy phương trình đã cho có hai nghiệm là $x=-5$ hoặc $x=1$.
		\end{enumerate}
	}
\end{vd}
\subsubsection{Bài tập}
\begin{bt}%[Dự án EX-9-Đề Cương Toán 9]%[Hoàng Thanh Phương]%[9D1N1-1] 
	Giải các phương trình sau
	\begin{multicols}{2}
		\begin{enumerate}
			\item $2x(2x-3)=0$;
			\item $9x^2(3x-6)=0$;
			\item $(x+2)(3-3x)=0$;
			\item $(2x-7)(x+13)=0$.
		\end{enumerate}
	\end{multicols}
	\loigiai{
	\begin{enumerate}
		\item Ta có
		\allowdisplaybreaks
		\begin{eqnarray*}
			&&2x(2x-3)=0 \\
			&& 2x=0 \text{ hoặc } 2x-3=0 \\
			&& x=0 \text{ hoặc } x=\dfrac{3}{2}.
		\end{eqnarray*}
		Vậy phương trình đã cho có hai nghiệm là $x=0$; $x=\dfrac{3}{2}$.
		\item Ta có
		\allowdisplaybreaks
		\begin{eqnarray*}
			&&(x+2)(3-3x)=0 \\
			&& x+2=0 \text{ hoặc } 3-3x=0 \\
			&& x=-2 \text{ hoặc } x=1.
		\end{eqnarray*}
		Vậy phương trình đã cho có hai nghiệm là $x=-2$; $x=1$.
		\item Ta có
		\allowdisplaybreaks
		\begin{eqnarray*}
			&&9x^2(3x-6)=0 \\
			&&9x^2=0 \text{ hoặc } 3x-6=0 \\
			&&x=0 \text{ hoặc } x=2.
		\end{eqnarray*}
		Vậy phương trình đã cho có hai nghiệm là $x=0$; $x=2$.
		\item Ta có
		\allowdisplaybreaks
		\begin{eqnarray*}
			&&(2x-7)(x+13)=0 \\
			&&2x-7=0 \text{ hoặc } x+13=0 \\
			&&x=\dfrac{7}{2} \text{ hoặc } x=-13.
		\end{eqnarray*}
		Vậy phương trình đã cho có hai nghiệm là $x=\dfrac{7}{2}$; $x=-13$.
	\end{enumerate}
}
\end{bt}
\begin{bt}%[Dự án EX-9-Đề Cương Toán 9]%[Hoàng Thanh Phương]%[9D1H1-1] 
	Giải các phương trình sau
	\begin{multicols}{2}
		\begin{enumerate}
			\item  $3x^2 + 6x = 0$;
			\item $2x(x+6) + 5(x+6) = 0$;
			\item $x(2x-1) + 5(2x-1) = 0$;
			\item $(2x-5)(x+7) = x(x+7)$;
			\item $(x-1)^2 + 4x-4 = 0$;
			\item $(x-4)^2=5x-20$;
			\item $(2-3x)(x+11)=(3x-2)(2-5x)$;
			\item $(x^2-5x)^2+10(x^2-5x)+24=0$;
			\item $x(x+1)(x^2+x-18)=-72$.
		\end{enumerate}
	\end{multicols}
	\loigiai{
	\begin{enumerate}
		\item 
		Ta có 
		\allowdisplaybreaks
		\begin{eqnarray*}
			&&3x^2 + 6x = 0 \\
			&&3x(x+2)=0 \\
			&&3x=0 \text{ hoặc } x+2=0 \\
			&&x=0 \text{ hoặc } x=-2.
		\end{eqnarray*}
		Vậy phương trình đã cho có nghiệm là $x=0$; $x=-2$.
		\item 
		Ta có 
		\allowdisplaybreaks
		\begin{eqnarray*}
			&&2x(x+6)+5(x+6)=0 \\
			&&(2x+5)(x+6)=0 \\
			&&2x+5=0 \text{ hoặc } x+6=0 \\
			&&x=-\dfrac{5}{2} \text{ hoặc } x=-6.
		\end{eqnarray*}
		Vậy phương trình đã cho có nghiệm là $x=-\dfrac{5}{2}$; $x=-6$.
		\item 
		Ta có 
		\allowdisplaybreaks
		\begin{eqnarray*}
			&&x(2x-1)+5(2x-1) = 0 \\
			&&(2x-1)(x+5)=0 \\
			&&2x-1=0 \text{ hoặc } x+5=0 \\
			&&x=\dfrac{1}{2} \text{ hoặc } x=-5.
		\end{eqnarray*}
		Vậy phương trình đã cho có hai nghiệm là $x=-5$; $x=\dfrac{1}{2}$.
		\item 
		Ta có
		\allowdisplaybreaks
		\begin{eqnarray*}
			&&(2x-5)(x+7) = x(x+7) \\
			&&(2x-5)(x+7)-x(x+7)=0 \\
			&&(x-5)(x+7)=0 \\
			&&x-5=0 \text{ hoặc } x+7=0 \\
			&&x=5 \text{ hoặc } x=-7.
		\end{eqnarray*}
		Vậy phương trình đã cho có hai nghiệm là $x=5$; $x=-7$.
		\item 
		Ta có 
		\allowdisplaybreaks
		\begin{eqnarray*}
			&&(x-1)^2+4x-4=0 \\
			&&(x-1)^2+4(x-1)=0 \\
			&&(x-1)(x+3)=0 \\
			&&x-1=0 \text{ hoặc } x+3=0 \\
			&&x=1 \text{ hoặc } x=-3.
		\end{eqnarray*}
		Vậy phương trình đã cho có nghiệm $x=1$; $x=-3$.
		\item
		Ta có 
		\allowdisplaybreaks
		\begin{eqnarray*}
			&&(x-4)^2=5x-20 \\
			&&(x-4)^2-5(x-4)=0 \\
			&&(x-4)(x-9)=0 \\
			&&x-4=0 \text{ hoặc } x-9=0 \\
			&&x=4 \text{ hoặc } x=9.
		\end{eqnarray*}
		Vậy phương trình đã cho có nghiệm $x=4$; $x=9$.
		\item 
		Ta có 
		\allowdisplaybreaks
		\begin{eqnarray*}
			&&(2-3x)(x+11)=(3x-2)(2-5x) \\
			&&(3x-2)(x+11)+(3x-2)(2-5x)=0 \\
			&&(3x-2)(-4x+13)=0 \\
			&&3x-2=0 \text{ hoặc } -4x+13=0 \\
			&&x=\dfrac{2}{3} \text{ hoặc } x=\dfrac{13}{4}.
		\end{eqnarray*}
		Vậy phương trình đã cho có hai nghiệm là $x=\dfrac{2}{3}$; $x=\dfrac{13}{4}$.
		\item 
		Ta có 
		\allowdisplaybreaks
		\begin{eqnarray*}
			&&(x^2-5x)^2+10(x^2-5x)+24=0 \\
			&&(x^2-5x)^2+4(x^2-5x)+6(x^2-5x)+24=0 \\
			&&(x^2-5x)(x^2-5x+4)+6(x^2-5x+4)=0  \\
			&&(x^2-5x+6)(x^2-5x+4)=0 \\ 
			&&(x^2-3x-2x+6)(x^2-x-4x+4)=0 \\
			&&[x(x-3)-2(x-3)]\cdot [x(x-1)-4(x-1)]=0 \\
			&&(x-3)(x-2)(x-1)(x-4)=0 \\
			&&x=3 \text{ hoặc } x=2 \text{ hoặc } x=1 \text{ hoặc } x=4.
		\end{eqnarray*}
		Vậy phương trình đã cho có bốn nghiệm là $x=1$; $x=2$; $x=3$; $x=4$.
		\item 
		Ta có 
		\allowdisplaybreaks
		\begin{eqnarray*}
			&&x(x+1)(x^2+x-18)=-72 \\
			&&(x^2+x)(x^2+x-18)=-72 \\
			&&(x^2+x)^2-18(x^2+x)+72=0 \\
			&&(x^2+x)^2-6(x^2+x)-12(x^2+x)+72=0 \\
			&&(x^2+x)(x^2+x-6)-12(x^2+x-6)=0 \\
			&&(x^2+x-6)(x^2+x-12)=0 \\
			&&(x^2+3x-2x-6)\cdot (x^2+4x-3x-12)=0 \\
			&&[x(x+3)-2(x+3)]\cdot [x(x+4)-3(x+4)]=0\\
			&&(x-2)(x+3)(x-3)(x+4)=0 \\
			&&x=2 \text{ hoặc } x=-3 \text{ hoặc } x=3 \text{ hoặc } x=-4.
		\end{eqnarray*}
		Vậy phương trình đã cho có bốn nghiệm là $x=2$; $x=-3$; $x=3$; $x=-4$.
	\end{enumerate}
}
\end{bt}
\begin{bt}%[Dự án EX-9-Đề Cương Toán 9]%[Hoàng Thanh Phương]%[9D1H1-1]%[9D1V1-1] 
	Giải các phương trình sau
	\begin{multicols}{2}
		\begin{enumerate}
			\item $x^2-9=0$;
			\item $4x^2-25=0$;
			\item $25-36x^2=0$;
			\item $(x+1)^2-81=0$;
			\item $(2x+3)^2-(x-2)^2=0$;
			\item $(-x+5)^2-(2x-1)^2=0$;
			\item $(3x-2)^2-4(x+1)^2=0$;
			\item $(x^2-3x+2)^2=(x^2-6x+8)^2$;
			\item $(5x^2+3x-2)^2=(4x^2-3x-2)^2$;
			\item $(4x+3)(x^2-9)=(x+3)(16x^2-9)$.
		\end{enumerate}
	\end{multicols}
	\loigiai{
	\begin{enumerate}
		\item Ta có
		\allowdisplaybreaks
		\begin{eqnarray*}
			&&x^2-9=0 \\
			&&(x-3)(x+3)=0 \\
			&&x-3=0 \text{ hoặc } x+3=0 \\
			&&x=3 \text{ hoặc } x=-3.
		\end{eqnarray*}
		Vậy phương trình đã cho có hai nghiệm $x=3$; $x=-3$.
		\item Ta có
		\allowdisplaybreaks
		\begin{eqnarray*}
			&&4x^2-25=0 \\
			&&(2x-5)(2x+5)=0 \\
			&&2x-5=0 \text{ hoặc } 2x+5=0 \\
			&&x=\dfrac{5}{2} \text{ hoặc } x=-\dfrac{5}{2}.
		\end{eqnarray*}
		Vậy phương trình đã cho có hai nghiệm $x=\dfrac{5}{2}$; $x=-\dfrac{5}{2}$.
		\item Ta có
		\allowdisplaybreaks
		\begin{eqnarray*}
			&&25-36x^2=0 \\
			&&(5-6x)(5+6x)=0 \\
			&&5-6x=0 \text{ hoặc } 5+6x=0 \\
			&&x=\dfrac{5}{6} \text{ hoặc } x=-\dfrac{5}{6}.
		\end{eqnarray*}
		Vậy phương trình đã cho có hai nghiệm $x=-\dfrac{5}{6}$; $x=\dfrac{5}{6}$.
		\item Ta có
		\allowdisplaybreaks
		\begin{eqnarray*}
			&&(x+1)^2-81=0 \\
			&&(x+1-9)(x+1+9)=0 \\
			&&(x-8)(x+10)=0 \\
			&&x-8=0 \text{ hoặc } x+10=0 \\
			&&x=8 \text{ hoặc } x=-10.
		\end{eqnarray*}
		\item Ta có
		\allowdisplaybreaks
		\begin{eqnarray*}
			&&(2x+3)^2-(x-2)^2=0 \\
			&&(2x+3-x+2)(2x+3+x-2)=0 \\
			&&(x+5)(3x+1)=0 \\
			&&x+5=0 \text{ hoặc } 3x+1=0 \\
			&&x=-5 \text{ hoặc } x=-\dfrac{1}{3}.
		\end{eqnarray*}
		Vậy phương trình đã cho có nghiệm $x=-5$; $x=-\dfrac{1}{3}$.
		\item Ta có
		\allowdisplaybreaks
		\begin{eqnarray*}
			&&(-x+5)^2-(2x-1)^2=0 \\
			&&(-x+5-2x+1)(-x+5+2x-1)=0 \\
			&&(-3x+6)(x+4)=0 \\
			&&-3x+6=0 \text{ hoặc } x+4=0 \\
			&&x=2 \text{ hoặc } x=-4.
		\end{eqnarray*}
		Vậy phương trình đã cho có hai nghiệm $x=2$; $x=-4$.
		\item Ta có
		\allowdisplaybreaks
		\begin{eqnarray*}
			&&(3x-2)^2-4(x+1)^2=0 \\
			&&(3x-2)^2-(2x+2)^2=0 \\
			&&(3x-2-2x-2)(3x-2+2x+2)=0 \\
			&&5x(x-4)=0 \\
			&&x(x-4)=0 \\
			&&x=0 \text{ hoặc }x=4.
		\end{eqnarray*}
		Vậy phương trình đã cho có hai nghiệm $x=0$; $x=4$.
		\item Ta có
		\allowdisplaybreaks
		\begin{eqnarray*}
			&&(x^2-3x+2)^2=(x^2-6x+8)^2 \\
			&&(x^2-3x+2)^2-(x^2-6x+8)^2 =0 \\
			&&\left[(x^2-3x+2)-(x^2-6x+8)\right]\left[(x^2-3x+2)+(x^2-6x+8)\right]=0 \\
			&&(3x-6)(2x^2-9x+10)=0 \\
			&&3(x-2)(2x^2-4x-5x+10)=0 \\
			&&3(x-2)(x-2)(2x-5)=0 \\
			&&x-2=0 \text{ hoặc } 2x-5=0 \\
			&&x=2 \text{ hoặc } x=\dfrac{5}{2}. 
		\end{eqnarray*}
		Vậy phương trình đã cho có hai nghiệm $x=2$; $x=\dfrac{5}{2}$.
		\item Ta có
		\allowdisplaybreaks
		\begin{eqnarray*}
			&&(5x^2+3x-2)^2=(4x^2-3x-2)^2 \\
			&&(5x^2+3x-2)^2-(4x^2-3x-2)^2=0\\
			&&\left[(5x^2+3x-2)-(4x^2-3x-2)\right]\left[(5x^2+3x-2)+(4x^2-3x-2)\right]=0 \\
			&&(x^2+6x)(9x^2-4)=0 \\
			&&x(x+6)(3x-2)(3x+2)=0 \\
			&&x=0 \text{ hoặc } x+6=0 \text{ hoặc } 3x-2=0 \text{ hoặc } 3x+2=0 \\
			&&x=0 \text{ hoặc } x=-6 \text{ hoặc } x=\dfrac{2}{3} \text{ hoặc } x=-\dfrac{2}{3}.
		\end{eqnarray*}
		Vậy phương trình đã cho có bốn nghiệm $x=0$; $x=-6$; $x=\dfrac{2}{3}$; $x=-\dfrac{2}{3}$.
		\item Ta có
		\allowdisplaybreaks
		\begin{eqnarray*}
			&&(4x+3)(x^2-9)=(x+3)(16x^2-9) \\
			&&(4x+3)(x-3)(x+3)=(x+3)(4x-3)(4x+3) \\
			&&(4x+3)(x+3)\left[(x-3)-(4x-3)\right]=0 \\
			&&(4x+3)(x+3)(-3x)=0 \\
			&&4x+3=0 \text{ hoặc } x+3=0 \text{ hoặc } -3x=0 \\
			&&x=-\dfrac{3}{4} \text{ hoặc } x=-3 \text{ hoặc } x=0.
		\end{eqnarray*}
		Vậy phương trình đã cho có ba nghiệm $x=-\dfrac{3}{4}$; $x=-3$; $x=0$.
	\end{enumerate}
}
\end{bt}
%\begin{bt}%[Dự án EX-9-Đề Cương Toán 9]%[Tên Thầy/cô biên soạn]%[ID]

\subsection{Phương trình chứa ẩn ở mẫu quy về phương trình bậc nhất}
\subsubsection{Kiến thức trọng tâm}
\begin{tomtat}
	\begin{itemize}
		\item \textit{Phương trình chứa ẩn ở mẫu} là phương trình có chứa ẩn trong mẫu thức của phân thức.
		\item Đối với phương trình chứa ẩn ở mẫu, điều kiện của ẩn sao cho các phân thức chứa trong phương trình đều xác định gọi là \textbf{điều kiện xác định} của phương trình.  \\
		Để tìm điều kiện xác định của phương trình, ta đặt điều kiện của ẩn để tất cả các mẫu thức đều khác $0$. 
	\end{itemize}
	Ta có cách giải phương trình chứa ẩn ở mẫu như sau:
	\begin{itemize}
		\item Bước 1: Tìm điều kiện xác định của phương trình.
		\item Bước 2: Quy đồng mẫu thức hai vế của phương trình và khử mẫu.
		\item Bước 3: Giải phương trình vừa nhận được.
		\item Bước 4: Xét các giá trị tìm được ở bước $3$, giá trị nào thỏa mãn điều kiện xác định thì là nghiệm của phương trình đã cho.
	\end{itemize}
\end{tomtat}
\begin{vd}%[Dự án EX-9-Đề Cương Toán 9]%[Hoàng Thanh Phương]%[9D1N1-2] 
	Giải các phương trình sau
	\begin{multicols}{2}
		\begin{enumerate}
			\item $\dfrac{x+5}{x-3}+2=\dfrac{2}{x-3}$;
			\item $2x+\dfrac{1}{x-2}=4+\dfrac{1}{x-2}$.
		\end{enumerate}
	\end{multicols}
	\loigiai{
	\begin{enumerate}
		\item Điều kiện xác định $x\neq 3$. \\
		Ta có 
		\allowdisplaybreaks 
		\begin{eqnarray*}
			&&\dfrac{x+5}{x-3}+2=\dfrac{2}{x-3} \\
			&&\dfrac{x+5}{x-3}+\dfrac{2(x-3)}{x-3}=\dfrac{2}{x-3} \\
			&&x+5+2(x-3)=2 \\
			&&3x=3 \\
			&&x=1.
		\end{eqnarray*}
		Kiểm tra thấy $x=1$ thỏa mãn điều kiện xác định. \\
		Vậy phương trình đã cho có nghiệm là $x=1$.
		\item Điều kiện xác định $x\neq 2$.  \\
		Ta có 
		\allowdisplaybreaks
		\begin{eqnarray*}
			&&2x+\dfrac{1}{x-2}=4+\dfrac{1}{x-2} \\
			&&2x=4 \\
			&&x=2.
		\end{eqnarray*}
		Kiểm tra thấy $x=2$ không thỏa mãn điều kiện xác định. \\
		Vậy phương trình đã cho vô nghiệm.
	\end{enumerate}
}
\end{vd}
\begin{vd}%[Dự án EX-9-Đề Cương Toán 9]%[Hoàng Thanh Phương]%[9D1H1-2] 
	Giải các phương trình sau
	\begin{multicols}{2}
		\begin{enumerate}
			\item $\dfrac{x+3}{x-3}+\dfrac{x-2}{x}=2$;
			\item $\dfrac{3x+5}{x+1}+\dfrac{2}{x}=3$.
		\end{enumerate}
	\end{multicols}
	\loigiai{
	\begin{enumerate}
		\item Điều kiện xác định $x\neq 0$, $x\neq 3$. \\
		Ta có 
		\allowdisplaybreaks 
		\begin{eqnarray*}
			&&\dfrac{x+3}{x-3}+\dfrac{x-2}{x}=2 \\
			&&\dfrac{x(x+3)}{x(x-3)}+\dfrac{(x-2)(x-3)}{x(x-3)}=\dfrac{2x(x-3)}{x(x-3)} \\
			&&x(x+3)+(x-2)(x-3)=2x(x-3) \\
			&&x^2+3x+x^2-5x+6=2x^2-6x \\
			&&4x=-6 \\
			&&x=-\dfrac{3}{2}.
		\end{eqnarray*}
		Kiểm tra thấy $x=-\dfrac{3}{2}$ thỏa mãn điều kiện xác định. \\
		Vậy phương trình đã cho có nghiệm là $x=-\dfrac{3}{2}$.
		\item Điều kiện xác định $x\neq -1$, $x\neq 0$. \\
		Ta có 
		\allowdisplaybreaks
		\begin{eqnarray*}
			&&\dfrac{3x+5}{x+1}+\dfrac{2}{x}=3 \\
			&&\dfrac{x(3x+5)}{x(x+1)}+\dfrac{2(x+1)}{x(x+1)}=\dfrac{3x(x+1)}{x(x+1)} \\
			&&x(3x+5)+2(x+1)=3x(x+1) \\
			&&3x^2+5x+2x+2=3x^2+3x \\
			&&4x+2=0 \\
			&&x=-\dfrac{1}{2}.
		\end{eqnarray*}
		Kiểm tra thấy $x=-\dfrac{1}{2}$ thỏa mãn điều kiện xác định. \\
		Vậy phương trình đã cho có nghiệm là $x=-\dfrac{1}{2}$.
	\end{enumerate}
}
\end{vd}
\begin{vd}%[Dự án EX-9-Đề Cương Toán 9]%[Hoàng Thanh Phương]%[9D1H1-2] 
	Giải các phương trình sau
	\begin{multicols}{2}
		\begin{enumerate}
			\item $\dfrac{x+2}{x-2}-\dfrac{x-2}{x+2}=\dfrac{16}{x^2-4}$;
			\item $\dfrac{3}{x-2}+\dfrac{2}{x+1}=\dfrac{2x+5}{x^2-x-2}$.
		\end{enumerate}
	\end{multicols}
	\loigiai{
	\begin{enumerate}
		\item Điều kiện xác định $x\neq 2$, $x\neq -2$. \\
		Ta có 
		\allowdisplaybreaks 
		\begin{eqnarray*}
			&&\dfrac{x+2}{x-2}-\dfrac{x-2}{x+2}=\dfrac{16}{x^2-4} \\
			&&\dfrac{(x+2)^2}{(x+2)(x-2)}-\dfrac{(x-2)^2}{(x+2)(x-2)}=\dfrac{16}{(x+2)(x-2)} \\
			&&(x+2)^2-(x-2)^2=16 \\
			&&x^2+4x+4-(x^2-4x+4)=16 \\
			&&4x=16 \\
			&&x=4.
		\end{eqnarray*}
		Kiểm tra thấy $x=4$ thỏa mãn điều kiện xác định. \\
		Vậy phương trình đã cho có nghiệm $x=4$.
		\item Điều kiện xác định $x\neq -1$, $x\neq 2$. \\
		Ta có 
		\allowdisplaybreaks 
		\begin{eqnarray*}
			&&\dfrac{3}{x-2}+\dfrac{2}{x+1}=\dfrac{2x+5}{x^2-x-2} \\
			&&\dfrac{3(x+1)}{(x-2)(x+1)}+\dfrac{2(x-2)}{(x-2)(x+1)}=\dfrac{2x+5}{(x-2)(x+1)} \\
			&&3(x+1)+2(x-2)=2x+5 \\
			&&3x=6 \\
			&&x=2.
		\end{eqnarray*}
		Kiểm tra thấy $x=2$ không thỏa mãn điều kiện xác định. \\
		Vậy phương trình đã cho vô nghiệm.
	\end{enumerate}
}
\end{vd}
\subsubsection{Bài tập}
\begin{bt}%[Dự án EX-9-Đề Cương Toán 9]%[Hoàng Thanh Phương]%[9D1N1-2] 
	Giải các phương trình sau
	\begin{multicols}{2}
		\begin{enumerate}
			\item $\dfrac{2x-3}{x-3}=\dfrac{3}{x-3}$;
			\item $\dfrac{3x+1}{x-2}=\dfrac{4}{x-2}$;
			\item  $\dfrac{2x+5}{x-3}+1=\dfrac{5}{x-3}$;
			\item $\dfrac{1}{x-2}+3=\dfrac{3-x}{x-2}$.
		\end{enumerate}
	\end{multicols}
	\loigiai{
	\begin{enumerate}
		\item Điều kiện xác định $x\neq 3$. \\
		Ta có 
		\allowdisplaybreaks 
		\begin{eqnarray*}
			&&\dfrac{2x-3}{x-3}=\dfrac{3}{x-3} \\
			&&2x-3=3 \\
			&&x=3.
		\end{eqnarray*}
		Kiểm tra thấy $x=3$ không thỏa mãn điều kiện xác định. \\
		Vậy phương trình đã cho vô nghiệm.
		\item Điều kiện xác định $x\neq 2$. \\
		Ta có 
		\allowdisplaybreaks 
		\begin{eqnarray*}
			&&\dfrac{3x+1}{x-2}=\dfrac{4}{x-2} \\
			&&3x+1=4 \\
			&&x=1. 
		\end{eqnarray*}
		Kiểm tra thấy $x=1$ thỏa mãn điều kiện xác định. \\
		Vậy phương trình đã cho có nghiệm là $x=1$.
		\item Điều kiện xác định $x\neq 3$. \\
		Ta có 
		\allowdisplaybreaks
		\begin{eqnarray*}
			&&\dfrac{2x+5}{x-3}+1=\dfrac{5}{x-3} \\
			&&\dfrac{2x+5}{x-3}+\dfrac{x-3}{x-3}=\dfrac{5}{x-3} \\
			&&\dfrac{3x+2}{x-3}=\dfrac{5}{x-3} \\
			&&3x+2=5 \\
			&&x=1.
		\end{eqnarray*}
		Kiểm tra thấy $x=1$ thỏa mãn điều kiện xác định. \\
		Vậy phương trình đã cho có nghiệm là $x=1$. 
		\item Điều kiện xác định $x\neq 2$. \\
		Ta có 
		\allowdisplaybreaks
		\begin{eqnarray*}
			&&\dfrac{1}{x-2}+3=\dfrac{3-x}{x-2} \\
			&&\dfrac{1}{x-2}+\dfrac{3x-6}{x-2} =\dfrac{3-x}{x-2} \\
			&&\dfrac{3x-5}{x-2}=\dfrac{3-x}{x-2} \\
			&&3x-5=3-x \\
			&&4x=8 \\
			&&x=2.
		\end{eqnarray*}
		Kiểm tra thấy $x=2$ không thỏa mãn điều kiện xác định. \\
		Vậy phương trình đã cho vô nghiệm.
	\end{enumerate}
}
\end{bt}
\begin{bt}%[Dự án EX-9-Đề Cương Toán 9]%[Hoàng Thanh Phương]%[9D1H1-2] 
	Giải các phương trình sau
	\begin{multicols}{2}
		\begin{enumerate}
			\item $\dfrac{1}{x}+\dfrac{2}{x-2}=0$;
			\item $\dfrac{2}{x+1}=\dfrac{1}{3-7x}$;
			\item $\dfrac{4}{x-1}=\dfrac{x}{x-2}$;
			\item $\dfrac{x+3}{x+1}+\dfrac{x-2}{x}=2$;
			\item $\dfrac{3x-2}{x+7}=\dfrac{6x+1}{2x-3}$;
			\item  $\dfrac{4}{x-1}-\dfrac{5}{x-2}=-3$;
			\item $\dfrac{2x+5}{2x}-\dfrac{x}{x+5}=0$;
			\item $\dfrac{x-5}{2x^2-5x+3}=\dfrac{x-5}{2x^2-9x+7}$;
			\item $\dfrac{x+10}{2x^2+x-21}=\dfrac{6}{x^2-9}$.
		\end{enumerate}
	\end{multicols}
	\loigiai{
	\begin{enumerate}
		\item Điều kiện xác định  $x\neq 0$, $x\neq 2$.    \\
		Ta có 
		\allowdisplaybreaks
		\begin{eqnarray*}
			&&\dfrac{1}{x}+\dfrac{2}{x-2}=0 \\
			&&\dfrac{x-2}{x(x-2)}+\dfrac{2x}{x(x-2)}=0 \\
			&&\dfrac{x-2+2x}{x(x-2)}=0 \\
			&&3x-2=0 \\
			&&x=\dfrac{2}{3}.
		\end{eqnarray*}
		Kiểm tra thấy $x=\dfrac{2}{3}$ thỏa mãn điều kiện xác định. \\
		Vậy phương trình đã cho có nghiệm là $x=\dfrac{2}{3}$.
		\item Điều kiện xác định  $x\neq -1$, $x\neq \dfrac{3}{7}$.   \\
		Ta có 
		\allowdisplaybreaks
		\begin{eqnarray*}
			&&\dfrac{2}{x+1}=\dfrac{1}{3-7x} \\
			&&\dfrac{2(3-7x)}{(x+1)(3-7x)}=\dfrac{x+1}{(x+1)(3-7x)} \\
			&&2(3-7x)=x+1 \\
			&&15x=5 \\
			&&x=\dfrac{1}{3}.
		\end{eqnarray*}
		Kiểm tra thấy $x=\dfrac{1}{3}$ thỏa mãn điều kiện xác định. \\
		Vậy phương trình đã cho có nghiệm là $x=\dfrac{1}{3}$.
		\item Điều kiện xác định  $x\neq 1$, $x\neq 2$.    \\
		Ta có 
		\allowdisplaybreaks
		\begin{eqnarray*}
			&&\dfrac{4}{x-1}=\dfrac{x-1}{x-2} \\
			&&\dfrac{4(x-2)}{(x-1)(x-2)}=\dfrac{(x-1)^2}{(x-2)(x-1)} \\
			&&4(x-2)=(x-1)^2 \\
			&&x^2-6x+9=0 \\
			&&(x-3)^2=0 \\
			&&x=3.
 		\end{eqnarray*}
 		Kiểm tra thấy $x=3$ thỏa mãn điều kiện xác định. \\
 		Vậy phương trình đã cho có nghiệm là $x=3$.
		\item Điều kiện xác định  $x\neq -1$, $x\neq 0$.  \\
		Ta có 
		\allowdisplaybreaks 
		\begin{eqnarray*}
			&&\dfrac{x+3}{x+1}+\dfrac{x-2}{x}=2 \\
			&&\dfrac{x(x+3)}{x(x+1)}+\dfrac{(x-2)(x+1)}{x(x+1)}=\dfrac{2x(x+1)}{x(x+1)} \\
			&&x(x+3)+(x-2)(x+1)=2x(x+1) \\
			&&x^2+3x+x^2-x-2=2x^2+2x \\
			&&0x=2 ~(\text{vô nghiệm}).
		\end{eqnarray*}
		Vậy phương trình đã cho vô nghiệm.
		\item Điều kiện xác định $x\neq -7$, $x\neq \dfrac{3}{2}$.     \\
		Ta có 
		\allowdisplaybreaks 
		\begin{eqnarray*}
			&&\dfrac{3x-2}{x+7}=\dfrac{6x+1}{2x-3} \\
			&&\dfrac{(3x-2)(2x-3)}{(x+7)(2x-3)}=\dfrac{(6x+1)(x+7)}{(2x-3)(x+7)} \\
			&&(3x-2)(2x-3)=(6x+1)(x+7) \\
			&&6x^2-13x+6=6x^2+43x+7 \\
			&&-56x=1 \\
			&&x=-\dfrac{1}{56}.
		\end{eqnarray*}
		Kiểm tra thấy $x=-\dfrac{1}{56}$ thỏa mãn điều kiện xác định. \\
		Vậy phương trình đã cho có nghiệm là $x=-\dfrac{1}{56}$.
		\item Điều kiện xác định  $x\neq 1$, $x\neq 2$.   \\
		Ta có 
		\allowdisplaybreaks
		\begin{eqnarray*}
			&&\dfrac{4}{x-1}-\dfrac{5}{x-2}=-3 \\
			&&\dfrac{4(x-2)}{(x-1)(x-2)}-\dfrac{5(x-1)}{(x-2)(x-1)}=\dfrac{-3(x-1)(x-2)}{(x-1)(x-2)} \\
			&&4(x-2)-5(x-1)=-3(x-1)(x-2) \\
			&&4x-8-5x+5=-3(x^2-3x+2) \\
			&&-x-3=-3x^2+9x-6 \\
			&&3x^2-10x+3=0 \\
			&&3x^2-9x-x+3=0 \\
			&&3x(x-3)-(x-3)=0 \\
			&&(3x-1)(x-3)=0 \\
			&&3x-1=0 \text{ hoặc } x-3=0 \\
			&&x=\dfrac{1}{3} \text{ hoặc } x=3.
		\end{eqnarray*}
		Kiểm tra thấy $x=\dfrac{1}{3}$ và $x=3$ đều thỏa mãn điều kiện xác định. \\
		Vậy phương trình đã cho có hai nghiệm $x=\dfrac{1}{3}$; $x=3$.
		\item Điều kiện xác định  $x\neq 0$, $x\neq -5$.    \\
		Ta có 
		\begin{eqnarray*}
			&&\dfrac{2x+5}{2x}-\dfrac{x}{x+5}=0 \\
			&&\dfrac{(2x+5)(x+5)}{2x(x+5)}-\dfrac{2x^2}{2x(x+5)}=0 \\
			&&(2x+5)(x+5)-2x^2=0 \\
			&&2x^2+15x+25-2x^2=0 \\
			&&15x+25=0 \\
			&&x=-\dfrac{5}{3}.
			\end{eqnarray*}
			Kiểm tra thấy $x=-\dfrac{5}{3}$ thỏa mãn điều kiện xác định. \\
			Vậy phương trình đã cho có nghiệm $x=-\dfrac{5}{3}$.
		\allowdisplaybreaks
		\item Điều kiện xác định  $2x^2-5x+3\neq 0$, $2x^2-9x+7\neq 0$.   \\
		Ta có 
		\allowdisplaybreaks
		\begin{eqnarray*}
			&&\dfrac{x-5}{2x^2-5x+3}=\dfrac{x-5}{2x^2-9x+7} \\
			&&\dfrac{x-5}{(x-1)(2x-3)}=\dfrac{x-5}{(x-1)(2x-7)}  \\
			&&\dfrac{x-5}{x-1}\left(\dfrac{1}{2x-3}-\dfrac{1}{2x-7}\right)=0 \\
			&&\dfrac{x-5}{x-1}\cdot \dfrac{-4}{(2x-3)(2x-7)}=0 \\
			&&\dfrac{-4(x-5)}{(x-1)(2x-3)(2x-7)} \\
			&&-4(x-5)=0 \\
			&&x-5=0 \\
			&&x=5.
		\end{eqnarray*}
		Kiểm tra thấy $x=5$ thỏa mãn điều kiện xác định. \\
		Vậy phương trình đã cho có nghiệm $x=5$.
		\item Điều kiện xác định  $2x^2+x-21\neq 0$, $x^2-9\neq 0$.    \\
		Ta có 
		\allowdisplaybreaks 
		\begin{eqnarray*}
			&&\dfrac{x+10}{2x^2+x-21}=\dfrac{6}{x^2-9} \\
			&&\dfrac{x+10}{(2x+7)(x-3)}=\dfrac{6}{(x-3)(x+3)} \\
			&&\dfrac{(x+10)(x+3)}{(2x+7)(x+3)(x-3)}=\dfrac{6(2x+7)}{(2x+7)(x+3)(x-3)} \\
			&&(x+10)(x+3)=6(2x+7) \\
			&&x^2+13x+30=12x+42 \\
			&&x^2+x-12=0  \\
			&&x^2+4x-3x-12=0 \\
			&&x(x+4)-3(x+4)=0 \\
			&&(x+4)(x-3)=0 \\
			&&x+4=0 \text{ hoặc } x-3=0 \\
			&&x=-4 \text{ hoặc } x=3.
		\end{eqnarray*}
		Kiểm tra thấy $x=-4$ thỏa mãn điều kiện xác định, còn $x=3$ không thỏa mãn. \\
		Vậy phương trình đã cho có nghiệm $x=-4$.
	\end{enumerate}
}
\end{bt}
\begin{bt}%[Dự án EX-9-Đề Cương Toán 9]%[Hoàng Thanh Phương]%[9D1H1-2]%[9D1V1-2]
	Giải các phương trình sau
	\begin{multicols}{2}
		\begin{enumerate}
			\item $\dfrac{1}{x-2}+\dfrac{2x}{2x-4}=\dfrac{-x+2}{-3x+6}$;
			\item $3x-\dfrac{1}{x-2}=\dfrac{x-1}{2-x}$;
			\item $\dfrac{14}{3x-12}-\dfrac{2+x}{x-4}=\dfrac{3}{8-2x}-\dfrac{5}{6}$;
			\item $\dfrac{3}{x+2}-\dfrac{2}{3x+6}=\dfrac{4x}{5x+10}$;
			\item $\dfrac{x+1}{x-1}-\dfrac{x-1}{x+1}=\dfrac{4}{x^2-1}$;
			\item $\dfrac{3}{x+1}-\dfrac{1}{x-2}=\dfrac{9}{(x+1)(x-2)}$;
			\item $\dfrac{x+1}{x-2}+\dfrac{x-1}{x+2}=\dfrac{2(x^2+2)}{x^2-4}$;
			\item $\dfrac{1}{2x-3}-\dfrac{3}{x(2x-3)}=\dfrac{5}{x}$;
			\item $\dfrac{13}{2x^2+x-21}+\dfrac{1}{2x+7}=\dfrac{6}{x^2-9}$;
			\item $\dfrac{x^3-(x-1)^3}{(4x+3)(x-5)}=\dfrac{7x-1}{4x+3}-\dfrac{x}{x-5}$.
		\end{enumerate}
	\end{multicols}
	\loigiai{
	\begin{enumerate}
		\item Điều kiện xác định $x\neq 2$. \\
		Ta có
		\allowdisplaybreaks
		\begin{eqnarray*}
			&&\dfrac{1}{x-2}+\dfrac{2x}{2x-4}=\dfrac{-x+2}{-3x+6} \\
			&&\dfrac{3}{3x-6}+\dfrac{3x}{3x-6}=\dfrac{x-2}{3x-6} \\
			&&3+3x=x-2 \\
			&&2x=-5 \\
			&&x=-\dfrac{5}{2}.
		\end{eqnarray*}
		Kiểm tra thấy $x=-\dfrac{5}{2}$ thỏa mãn điều kiện xác định. \\
		Vậy phương trình đã cho có nghiệm là $x=-\dfrac{5}{2}$.
		\item Điều kiện xác định $x\neq 2$. \\ 
		Ta có
		\allowdisplaybreaks
		\begin{eqnarray*}
			&&3x-\dfrac{1}{x-2}=\dfrac{x-1}{2-x} \\
			&&\dfrac{3x(x-2)}{x-2}-\dfrac{1}{x-2}=\dfrac{1-x}{x-2} \\
			&&3x(x-2)-1=1-x \\
			&&3x^2-5x-2=0 \\
			&&3x^2-6x+x-2=0 \\
			&&3x(x-2)+(x-2)=0 \\
			&&(3x+1)(x-2)=0 \\
			&&3x+1=0 \text{ hoặc  } x-2=0 \\
			&&x=-\dfrac{1}{3} \text{ hoặc } x=2.
		\end{eqnarray*}
		Kiểm tra thấy $x=-\dfrac{1}{3}$ thỏa mãn điều kiện xác định. \\
		Vậy phương trình đã cho có nghiệm $x=-\dfrac{1}{3}$.
		\item Điều kiện xác định $x\neq 4$. \\
		Ta có
		\allowdisplaybreaks
		\begin{eqnarray*}
			&&\dfrac{14}{3x-12}-\dfrac{2+x}{x-4}=\dfrac{3}{8-2x}-\dfrac{5}{6} \\
			&&\dfrac{14}{3x-12}-\dfrac{3x+6}{3x-12}=\dfrac{-\dfrac{9}{2}}{3x-12}-\dfrac{\dfrac{5}{2}(x-4)}{3x-12} \\
			&&14-(3x+6)=-\dfrac{9}{2}-\dfrac{5}{2}(x-4) \\
			&&8-3x=-\dfrac{5}{2}x+\dfrac{11}{2} \\
			&&x=5.
		\end{eqnarray*}
		Kiểm tra thấy $x=5$ thỏa mãn điều kiện xác định. \\
		Vậy phương trình đã cho có nghiệm $x=5$.
		\item Điều kiện xác định  $x\neq -2$. \\
		Ta có
		\allowdisplaybreaks
		\begin{eqnarray*}
			&&\dfrac{3}{x+2}-\dfrac{2}{3x+6}=\dfrac{4x}{5x+10} \\
			&&\dfrac{15}{5x+10}-\dfrac{\dfrac{10}{3}}{5x+10}=\dfrac{4x}{5x+10} \\
			&&15-\dfrac{10}{3}=4x \\
			&&x=\dfrac{35}{12}.
		\end{eqnarray*}
		Kiểm tra thấy $x=\dfrac{35}{12}$ thỏa mãn điều kiện xác định. \\
		Vậy phương trình đã cho có nghiệm $x=\dfrac{35}{12}$.
		\item Điều kiện xác định $x\neq 1$, $x\neq -1$.  \\
		Ta có
		\allowdisplaybreaks
		\begin{eqnarray*}
			&&\dfrac{x+1}{x-1}-\dfrac{x-1}{x+1}=\dfrac{4}{x^2-1}  \\
			&&\dfrac{(x+1)^2}{(x+1)(x-1)}-\dfrac{(x-1)^2}{(x+1)(x-1)}=\dfrac{4}{(x+1)(x-1)} \\
			&&(x+1)^2-(x-1)^2=4 \\
			&&x^2+2x+1-(x^2-2x+1)=4 \\
			&&4x=4 \\
			&&x=1.
		\end{eqnarray*}
		Kiểm tra thấy $x=1$ không thỏa mãn điều kiện xác định. \\
		Vậy phương trình đã cho vô nghiệm.
		\item Điều kiện xác định $x\neq -1$, $x\neq 2$. \\
		Ta có
		\allowdisplaybreaks
		\begin{eqnarray*}
			&&\dfrac{3}{x+1}-\dfrac{1}{x-2}=\dfrac{9}{(x+1)(x-2)} \\
			&&\dfrac{3(x-2)}{(x+1)(x-2)}-\dfrac{x+1}{(x+1)(x-2)}=\dfrac{9}{(x+1)(x-2)} \\
			&&3(x-2)-(x+1)=9 \\
			&&3x-6-x-1=9 \\
			&&2x-7=9 \\
			&&x=8.
		\end{eqnarray*}
		Kiểm tra thấy $x=8$ thỏa mãn điều kiện xác định. \\
		Vậy phương trình đã cho có nghiệm $x=8$.
		\item Điều kiện xác định $x\neq 2$, $x\neq -2$. \\
		Ta có
		\allowdisplaybreaks
		\begin{eqnarray*}
			&&\dfrac{x+1}{x-2}+\dfrac{x-1}{x+2}=\dfrac{2(x^2+2)}{x^2-4} \\
			&&\dfrac{(x+1)(x+2)}{(x+2)(x-2)}+\dfrac{(x-1)(x-2)}{(x+2)(x-2)}=\dfrac{2(x^2+2)}{(x-2)(x+2)} \\
			&&(x+1)(x+2)+(x-1)(x-2)=2(x^2+2) \\
			&&2x^2+4=2x^2+4 \\
			&&0x=0 ~(\text{vô số nghiệm}).
		\end{eqnarray*}
		Vậy phương trình đã cho có vô số nghiệm $x$ thỏa mãn $x\neq 2$ và $x\neq -2$.
		\item Điều kiện xác định $x\neq 0$ và $x\neq \dfrac{3}{2}$.  \\
		Ta có
		\allowdisplaybreaks
		\begin{eqnarray*}
			&&\dfrac{1}{2x-3}-\dfrac{3}{x(2x-3)}=\dfrac{5}{x} \\
			&&\dfrac{x}{x(2x-3)}-\dfrac{3}{x(2x-3)}=\dfrac{5(2x-3)}{x(2x-3)} \\
			&&x-3=5(2x-3) \\
			&&-9x=-12 \\
			&&x=\dfrac{4}{3}.
		\end{eqnarray*}
		Kiểm tra thấy $x=\dfrac{4}{3}$ thỏa mãn điều kiện xác định. \\
		Vậy phương trình đã cho có nghiệm $x=\dfrac{4}{3}$.
		\item Điều kiện xác định $x\neq -3$, $x\neq 3$, $x\neq -\dfrac{7}{2}$.  \\
		Ta có
		\allowdisplaybreaks
		\begin{eqnarray*}
			&&\dfrac{13}{2x^2+x-21}+\dfrac{1}{2x+7}=\dfrac{6}{x^2-9} \\
			&&\dfrac{13}{2x^2+6x-7x-21}+\dfrac{1}{2x+7}=\dfrac{6}{(x-3)(x+3)} \\
			&&\dfrac{13}{(x-3)(2x+7)}+\dfrac{1}{2x+7}=\dfrac{6}{(x-3)(x+3)} \\
			&&\dfrac{13(x+3)}{(x-3)(x+3)(2x+7)}+\dfrac{(x-3)(x+3)}{(x-3)(x+3)(2x+7)}=\dfrac{6(2x+7)}{(x-3)(x+3)(2x+7)} \\
			&&13(x+3)+(x-3)(x+3)=6(2x+7) \\
			&&13x+39+x^2-9=12x+42 \\
			&&x^2+x-12=0 \\
			&&x^2+4x-3x-12=0 \\
			&&(x+4)(x-3)=0 \\
			&&x+4=0 \text{ hoặc } x-3=0 \\
			&&x=-4 \text{ hoặc } x=3.
		\end{eqnarray*}
		Kiểm tra thấy $x=-4$ thỏa mãn điều kiện xác định. \\
		Vậy phương trình đã cho có nghiệm $x=-4$.
		\item Điều kiện xác định $x\neq -\dfrac{3}{4}$, $x\neq 5$. \\
		Ta có
		\allowdisplaybreaks
		\begin{eqnarray*}
			&&\dfrac{x^3-(x-1)^3}{(4x+3)(x-5)}=\dfrac{7x-1}{4x+3}-\dfrac{x}{x-5} \\
			&&\dfrac{x^3-(x^3-3x^2+3x-1)}{(4x+3)(x-5)}=\dfrac{(7x-1)(x-5)}{(4x+3)(x-5)}-\dfrac{x(4x+3)}{(x-5)(4x+3)} \\
			&&\dfrac{3x^2-3x+1}{(4x+3)(x-5)}=\dfrac{(7x-1)(x-5)}{(4x+3)(x-5)}-\dfrac{x(4x+3)}{(x-5)(4x+3)} \\ 
			&&3x^2-3x+1=(7x-1)(x-5)-x(4x+3) \\
			&&3x^2-3x+1=7x^2-36x+5-4x^2-3x \\
			&&3x^2-3x+1=3x^2-39x+5 \\
			&&36x=4 \\
			&&x=\dfrac{1}{9}.
		\end{eqnarray*}
		Kiểm tra thấy $x=\dfrac{1}{9}$ thỏa mãn điều kiện xác định. \\
		Vậy phương trình đã cho có nghiệm $x=\dfrac{1}{9}$.
	\end{enumerate}
}
\end{bt}
