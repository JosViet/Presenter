\section{ĐỊNH LÝ VIÈTE} % Tên bài
\subsection{Định lý Viète}
\subsubsection{Kiến thức trọng tâm}
\begin{tomtat}
	Nếu phương trình bậc hai $ax^2+bx+c=0$, $(a\ne0)$ có hai nghiệm $x_1$, $x_2$ thì tổng và tích của hai nghiệm đó là
	$$ S=x_1+x_2=-\dfrac{b}{a};\qquad P=x_1\cdot x_2=\dfrac{c}{a} $$
	\begin{luuy}
		\begin{itemize}
			\item Nếu phương trình $ax^2+bx+c=0$ $(a\ne0)$ có $a+b+c=0$ thì phương trình có một nghiệm là $x_1=1$, nghiệm còn lại là $x_2=\dfrac{c}{a}$.
			\item Nếu phương trình $ax^2+bx+c=0$ $(a\ne0)$ có $a-b+c=0$ thì phương trình có một nghiệm là $x_1=-1$, nghiệm còn lại là $x_2=-\dfrac{c}{a}$.
		\end{itemize}
	\end{luuy}
	\begin{chuy}
	{CÁC KẾT QUẢ QUAN TRỌNG CẦN NHỚ}
	\begin{itemize}
	\item $x_1^2+x_2^2=(x_1+x_2)^2-2x_1x_2=S^2-2P$.
	\item $(x_1-x_2)^2=x_1^2-2x_1x_2+x_2^2=(x_1^2+2x_1x_2+x_2^2)-4x_1x_2=(x_1+x_2)^2-4x_1x_2=S^2-4P$.
	\item $x_1^3+x_2^3=(x_1^3+x_2^3+3x_1^2x_2+3x_1x_2^2)-(3x_1^2x_2+3x_1x_2^2)=(x_1+x_2)^3-3x_1x_2(x_1+x_2)=S^3-3PS$.
	\item $x_1-x_2=\pm |x_1-x_2|=\pm\sqrt{(x_1-x_2)^2}=\pm\sqrt{(x_1+x_2)^2-4x_1x_2}=\pm\sqrt{S^2-4P}$.
	\item $\dfrac{1}{x_1}+\dfrac{1}{x_2}=\dfrac{x_1+x_2}{x_1x_2}=\dfrac{S}{P}$.
	\item $\dfrac{x_1}{x_2}+\dfrac{x_2}{x_1}=\dfrac{x_1^2+x_2^2}{x_1x_2}=\dfrac{S^2-2P}{P}$.
	\item
	$\dfrac{1}{x_1^2}+\dfrac{1}{x_2^2}=\dfrac{x_1^2+x_2^2}{x_1^2x_2^2}=\dfrac{S^2-2P}{P^2}$.
	\end{itemize}
	\end{chuy}
\end{tomtat}
\begin{vd}%[Dự án EX-9-Đề Cương Toán 9]%[GVSB: Hoàng Minh Nhân Mã - GVPB1: Phan Minh Huế - GVPB2: Nguyễn Trần Anh Tuấn]%[9D4H3-1]
	Không giải phương trình, hãy tính tổng và tích các nghiệm (nếu có) của các phương trình:
	\begin{multicols}{2}
		\begin{enumerate}
			\item $x^2-7x+5=0$.
			\item $5x^2-2x+7=0$.
		\end{enumerate}
	\end{multicols}
	\loigiai{
		\begin{enumerate}
			\item Xét phương trình $x^2-7x+5=0$ với $a=1$, $b=-7$, $c=5$.\\
			Ta có $\Delta=b^2-4ac=(-7)^2-4\cdot 1\cdot 5=29>0$ nên phương trình có hai nghiệm phân biệt $x_1$; $x_2$.\\
			Do đó, theo định lý Viète, ta được $x_1+x_2=-\dfrac{b}{a}=-\dfrac{-7}{1}=7$ và $x_1\cdot x_2=\dfrac{c}{a}=\dfrac{5}{1}=5$.
			\item Xét phương trình $5x^2-2x+7=0$ với $a=5$, $b=-2$, $c=7$.\\
			Ta có $\Delta=b^2-4ac=(-2)^2-4\cdot 5\cdot 7=-136<0$ nên phương trình vô nghiệm.
		\end{enumerate}
	}
\end{vd}
\begin{vd}%[Dự án EX-9-Đề Cương Toán 9]%[GVSB: Hoàng Minh Nhân Mã - GVPB1: Phan Minh Huế - GVPB2: Nguyễn Trần Anh Tuấn]%[9D4V3-1]
	Gọi $x_1$; $x_2$ là hai nghiệm của phương trình $x^2-5x+3=0$. Không giải phương trình, hãy tính giá trị của các biểu thức
	\begin{multicols}{2}
		\begin{enumerate}
			\item $\dfrac{1}{x_1}+\dfrac{1}{x_2}$.
			\item $x_1^2+x_2^2$.
		\end{enumerate}
	\end{multicols}
	\loigiai{
	Xét phương trình $x^2-5x+3=0$ với $a=1$, $b=-5$, $c=3$.\\
		Ta có $\Delta=b^2-4ac=(-5)^2-4\cdot 1\cdot 3=13>0$ nên phương trình có hai nghiệm phân biệt $x_1$; $x_2$.\\
		Do đó, theo định lý Viète, ta được $x_1+x_2=-\dfrac{b}{a}=-\dfrac{-5}{1}=5$ và $x_1\cdot x_2=\dfrac{c}{a}=\dfrac{3}{1}=3$.
		\begin{enumerate}
			\item Ta có $\dfrac{1}{x_1}+\dfrac{1}{x_2}=\dfrac{x_1+x_2}{x_1\cdot x_2}=\dfrac{5}{3}$.
			\item Ta có $x_1^2+x_2^2=(x_1+x_2)^2-2x_1x_2=5^2-2\cdot 3=19$. 
		\end{enumerate}
	}
\end{vd}
\begin{vd}%[Dự án EX-9-Đề Cương Toán 9]%[GVSB: Hoàng Minh Nhân Mã - GVPB1: Phan Minh Huế - GVPB2: Nguyễn Trần Anh Tuấn]%[9D4V3-1]
	Tính nhẩm nghiệm của các phương trình
	\begin{multicols}{2}
		\begin{enumerate}
			\item $15x^2+7x-22=0$.
			\item $18x^2-7x-25=0$.
		\end{enumerate}
	\end{multicols}
	\loigiai{
		\begin{enumerate}
			\item Xét phương trình $15x^2+7x-22=0$ với $a=15$, $b=7$, $c=-22$.\\
			Ta có $a+b+c=15+7+(-22)=0$ nên phương trình có hai nghiệm là $x_1=1$; $x_2=\dfrac{c}{a}=\dfrac{-22}{15}$.
			\item Xét phương trình $18x^2-7x-25=0$ với $a=18$, $b=-7$, $c=-25$.\\
			Ta có $a-b+c=18-(-7)+(-25)=0$ nên phương trình có hai nghiệm là $x_1=-1$; $x_2=-\dfrac{c}{a}=\dfrac{25}{18}$.
		\end{enumerate}
	}
\end{vd}
\subsubsection{Bài tập}
\begin{bt}%[Dự án EX-9-Đề Cương Toán 9]%[GVSB: Hoàng Minh Nhân Mã - GVPB1: Phan Minh Huế - GVPB2: Nguyễn Trần Anh Tuấn]%[9D4H3-1]
	Kiểm tra sự tồn tại nghiệm của mỗi phương trình sau rồi tính tổng và tích hai nghiệm mà không giải phương trình:
	\begin{multicols}{3}
	\begin{enumerate}
		\item $x^2-4x-2=0$;
		\item $x^2-5x+6=0$;
		\item $4x^2+5x-1=0$;
		\item $x^2-x-1=0$;
		\item $2x^2+3x-1=0$;
		\item $4x^2-4x+1=0$;
		\item $2x^2+3x+11=0$;
		\item $x^2-8x-7=0$;
		\item $x^2-2\sqrt{5}x+5=0$;
		\item $3x^2+2\sqrt{5}x+\sqrt{15}=0$;
		\item $x^2-5x+2=0$;
		\item $x^2+2x-17=0$;
		\item $2x^2-x+1=0$;
		\item $-2x^2+5x-4=0$;
		\item $16x^2+13x-48=0$;
		\item $x^2-2x-35=0$;
		\item $5x^2-2\sqrt{3}x-4=0$;
		\item $2x^2+3\sqrt{2}x-5=0$;
		\item $-4x^2+2x+5=0$;
		\item $\sqrt{3}x^2+(2-\sqrt{3})x-2=0$;
		\item $4x^2-(4+\sqrt{3})x+\sqrt{3}=0$;
		\item $x^2-\sqrt{3}x-2-\sqrt{6}=0$;
		\item $x^2-(\sqrt{2}+1)x+\sqrt{2}=0$;
		\item $(\sqrt{5}+1)x^2-5x+\sqrt{5}-1=0$.
	\end{enumerate}
	\end{multicols}
\loigiai{
	\begin{enumerate}
	\item $x^2-4x-2=0$ với $a=1$; $b=-4$ và $c=-2$.\\
		Ta có $\Delta=(-4)^2-4\cdot 1\cdot (-2)=24>0$ nên phương trình có hai nghiệm phân biệt $x_1$; $x_2$.\\
		Do đó, theo định lý Viète, ta được $x_1+x_2=-\dfrac{b}{a}=-\dfrac{-4}{1}=4$ và $x_1\cdot x_2=\dfrac{c}{a}=\dfrac{-2}{1}=-2$.
	\item $x^2-5x+6=0$ với $a=1$; $b=-5$ và $c=6$.\\
		Ta có $\Delta=(-5)^2-4\cdot 1\cdot 6=1>0$ nên phương trình có hai nghiệm phân biệt $x_1$; $x_2$.\\
		Do đó, theo định lý Viète, ta được $x_1+x_2=-\dfrac{b}{a}=-\dfrac{-5}{1}=5$ và $x_1\cdot x_2=\dfrac{c}{a}=\dfrac{6}{1}=6$.
	\item $4x^2+5x-1=0$ với $a=4$; $b=5$ và $c=-1$.\\
		Ta có $\Delta=5^2-4\cdot 4\cdot (-1)=41>0$ nên phương trình có hai nghiệm phân biệt $x_1$; $x_2$.\\
		Do đó, theo định lý Viète, ta được $x_1+x_2=-\dfrac{b}{a}=-\dfrac{5}{4}$ và $x_1\cdot x_2=\dfrac{c}{a}=\dfrac{-1}{4}$.
	\item $x^2-x-1=0$ với $a=1$; $b=-1$ và $c=-1$.\\
		Ta có $\Delta=(-1)^2-4\cdot 1\cdot (-1)=5>0$ nên phương trình có hai nghiệm phân biệt $x_1$; $x_2$.\\
		Do đó, theo định lý Viète, ta được $x_1+x_2=-\dfrac{b}{a}=-\dfrac{-1}{1}=1$ và $x_1\cdot x_2=\dfrac{c}{a}=\dfrac{-1}{1}=-1$.
	\item $2x^2+3x-1=0$ với $a=2$; $b=3$ và $c=-1$.\\
		Ta có $\Delta=3^2-4\cdot 2\cdot (-1)=17>0$ nên phương trình có hai nghiệm phân biệt $x_1$; $x_2$.\\
		Do đó, theo định lý Viète, ta được $x_1+x_2=-\dfrac{b}{a}=-\dfrac{3}{2}$ và $x_1\cdot x_2=\dfrac{c}{a}=\dfrac{-1}{2}$.
	\item $4x^2-4x+1=0$ với $a=4$; $b=-4$ và $c=1$.\\
		Ta có $\Delta=(-4)^2-4\cdot 4\cdot 1=0$ nên phương trình có nghiệm kép $x_1=x_2$.\\
		Do đó, theo định lý Viète, ta được $x_1+x_2=-\dfrac{b}{a}=-\dfrac{-4}{4}=1$ và $x_1\cdot x_2=\dfrac{c}{a}=\dfrac{1}{4}$.
	\item $2x^2+3x+11=0$ với $a=2$; $b=3$ và $c=11$.\\
		Ta có $\Delta=3^2-4\cdot 2\cdot 11=-79<0$ nên phương trình vô nghiệm.
	\item $x^2-8x-7=0$ với $a=1$; $b=-8$ và $c=-7$.\\
		Ta có $\Delta=(-8)^2-4\cdot 1\cdot (-7)=92>0$ nên phương trình có hai nghiệm phân biệt $x_1$; $x_2$.\\
		Do đó, theo định lý Viète, ta được $x_1+x_2=-\dfrac{b}{a}=-\dfrac{-8}{1}=8$ và $x_1\cdot x_2=\dfrac{c}{a}=\dfrac{-7}{1}=-7$.
	\item $x^2-2\sqrt{5}x+5=0$ với $a=1$; $b=-2\sqrt{5}$ và $c=5$.\\
		Ta có $\Delta=(-2\sqrt{5})^2-4\cdot 1\cdot 5=0$ nên phương trình có nghiệm kép $x_1=x_2$.\\
		Do đó, theo định lý Viète, ta được $x_1+x_2=-\dfrac{b}{a}=-\dfrac{-2\sqrt{5}}{1}=2\sqrt{5}$ và $x_1\cdot x_2=\dfrac{c}{a}=\dfrac{5}{1}=5$.
	\item $3x^2+2\sqrt{5}x+\sqrt{15}=0$ với $a=3$; $b=2\sqrt{5}$ và $c=\sqrt{15}$.\\
		Ta có $\Delta=(2\sqrt{5})^2-4\cdot 3\cdot \sqrt{15}=20-12\sqrt{15}<0$ nên phương trình vô nghiệm.
	\item $x^2-5x+2=0$ với $a=1$; $b=-5$ và $c=2$.\\
		Ta có $\Delta=(-5)^2-4\cdot 1\cdot 2=17>0$ nên phương trình có hai nghiệm phân biệt $x_1$; $x_2$.\\
		Do đó, theo định lý Viète, ta được $x_1+x_2=-\dfrac{b}{a}=-\dfrac{-5}{1}=5$ và $x_1\cdot x_2=\dfrac{c}{a}=\dfrac{2}{1}=2$.
	\item $x^2+2x-17=0$ với $a=1$; $b=2$ và $c=-17$.\\
		Ta có $\Delta=2^2-4\cdot 1\cdot (-17)=4+68=72>0$ nên phương trình có hai nghiệm phân biệt $x_1$; $x_2$.\\
		Do đó, theo định lý Viète, ta được $x_1+x_2=-\dfrac{b}{a}=-\dfrac{2}{1}=-2$ và $x_1\cdot x_2=\dfrac{c}{a}=\dfrac{-17}{1}=-17$.
	\item $2x^2-x+1=0$ với $a=2$; $b=-1$ và $c=1$.\\
		Ta có $\Delta=(-1)^2-4\cdot 2\cdot 1=-7<0$ nên phương trình vô nghiệm.
	\item $-2x^2+5x-4=0$ với $a=-2$; $b=5$ và $c=-4$.\\
		Ta có $\Delta=5^2-4\cdot (-2)\cdot (-4)=-7<0$ nên phương trình vô nghiệm.
	\item $16x^2+13x-48=0$ với $a=16$; $b=13$ và $c=-48$.\\
		Ta có $\Delta=13^2-4\cdot 16\cdot (-48)=3241>0$ nên phương trình có hai nghiệm phân biệt $x_1$; $x_2$.\\
		Do đó, theo định lý Viète, ta được $x_1+x_2=-\dfrac{b}{a}=-\dfrac{13}{16}$ và $x_1\cdot x_2=\dfrac{c}{a}=\dfrac{-48}{16}=-3$.
	\item $x^2-2x-35=0$ với $a=1$; $b=-2$ và $c=-35$.\\
		Ta có $\Delta=(-2)^2-4\cdot 1\cdot (-35)=144>0$ nên phương trình có hai nghiệm phân biệt $x_1$; $x_2$.\\
		Do đó, theo định lý Viète, ta được $x_1+x_2=-\dfrac{b}{a}=-\dfrac{-2}{1}=2$ và $x_1\cdot x_2=\dfrac{c}{a}=\dfrac{-35}{1}=-35$.
	\item $5x^2-2\sqrt{3}x-4=0$ với $a=5$; $b=-2\sqrt{3}$ và $c=-4$.\\
		Ta có $\Delta=(-2\sqrt{3})^2-4\cdot 5\cdot (-4)=92>0$ nên phương trình có hai nghiệm phân biệt $x_1$; $x_2$.\\
		Do đó, theo định lý Viète, ta được $x_1+x_2=-\dfrac{b}{a}=-\dfrac{-2\sqrt{3}}{5}=\dfrac{2\sqrt{3}}{5}$ và $x_1\cdot x_2=\dfrac{c}{a}=\dfrac{-4}{5}$.
	\item $2x^2+3\sqrt{2}x-5=0$ với $a=2$; $b=3\sqrt{2}$ và $c=-5$.\\
		Ta có $\Delta=(3\sqrt{2})^2-4\cdot 2\cdot (-5)=58>0$ nên phương trình có hai nghiệm phân biệt $x_1$; $x_2$.\\
		Do đó, theo định lý Viète, ta được $x_1+x_2=-\dfrac{b}{a}=-\dfrac{3\sqrt{2}}{2}$ và $x_1\cdot x_2=\dfrac{c}{a}=\dfrac{-5}{2}$.
	\item $-4x^2+2x+5=0$ với $a=-4$; $b=2$ và $c=5$.\\
		Ta có $\Delta=(2)^2-4\cdot (-4)\cdot 5=84>0$ nên phương trình có hai nghiệm phân biệt $x_1$; $x_2$.\\
		Do đó, theo định lý Viète, ta được $x_1+x_2=-\dfrac{b}{a}=-\dfrac{2}{-4}=\dfrac{1}{2}$ và $x_1\cdot x_2=\dfrac{c}{a}=\dfrac{5}{-4}=-\dfrac{5}{4}$.
	\item $\sqrt{3}x^2+(2-\sqrt{3})x-2=0$ với $a=\sqrt{3}$; $b=2-\sqrt{3}$ và $c=-2$.\\
		Ta có $\Delta=(2-\sqrt{3})^2-4\cdot \sqrt{3}\cdot (-2)=7+4\sqrt{3}>0$.\\
		Nên phương trình có hai nghiệm phân biệt $x_1$; $x_2$.\\
		Do đó, theo định lý Viète, ta được\\ $x_1+x_2=-\dfrac{b}{a}=-\dfrac{2-\sqrt{3}}{\sqrt{3}}=\dfrac{3-2\sqrt{3}}{3}$ và $x_1\cdot x_2=\dfrac{c}{a}=\dfrac{-2}{\sqrt{3}}=-\dfrac{2\sqrt{3}}{3}$.
	\item $4x^2-(4+\sqrt{3})x+\sqrt{3}=0$ với $a=4$; $b=-(4+\sqrt{3})$ và $c=\sqrt{3}$.\\
		Ta có $\Delta=[-(4+\sqrt{3})]^2-4\cdot 4\cdot (\sqrt{3})=19-8\sqrt{3}>0$.\\
		Nên phương trình có hai nghiệm phân biệt $x_1$; $x_2$.\\
		Do đó, theo định lý Viète, ta được\\ $x_1+x_2=-\dfrac{b}{a}=-\dfrac{-(4+\sqrt{3})}{4}=\dfrac{4+\sqrt{3}}{4}$ và $x_1\cdot x_2=\dfrac{c}{a}=\dfrac{\sqrt{3}}{4}$.
	\item $x^2-\sqrt{3}x-2-\sqrt{6}=0$ với $a=1$; $b=-\sqrt{3}$ và $c=-(2+\sqrt{6})$.\\
		Ta có $\Delta=(-\sqrt{3})^2-4\cdot 1\cdot (-(2+\sqrt{6}))=11+4\sqrt{6}>0$.\\
		Nên phương trình có hai nghiệm phân biệt $x_1$; $x_2$.\\
		Do đó, theo định lý Viète, ta được\\ $x_1+x_2=-\dfrac{b}{a}=-\dfrac{-\sqrt{3}}{1}=\sqrt{3}$ và $x_1\cdot x_2=\dfrac{c}{a}=\dfrac{-(2+\sqrt{6})}{1}=-2-\sqrt{6}$.
	\item $x^2-(\sqrt{2}+1)x+\sqrt{2}=0$ với $a=1$; $b=-(\sqrt{2}+1)$ và $c=\sqrt{2}$.\\
		Ta có $\Delta=[-(\sqrt{2}+1)]^2-4\cdot 1\cdot (\sqrt{2})=3-2\sqrt{2}>0$.\\
		Nên phương trình có hai nghiệm phân biệt $x_1$; $x_2$.\\
		Do đó, theo định lý Viète, ta được\\ $x_1+x_2=-\dfrac{b}{a}=-\dfrac{-(\sqrt{2}+1)}{1}=\sqrt{2}+1$ và $x_1\cdot x_2=\dfrac{c}{a}=\dfrac{\sqrt{2}}{1}=\sqrt{2}$.
	\item $(\sqrt{5}+1)x^2-5x+\sqrt{5}-1=0$ với $a=\sqrt{5}+1$; $b=-5$ và $c=\sqrt{5}-1$.\\
		Ta có $\Delta=(-5)^2-4\cdot (\sqrt{5}+1)\cdot (\sqrt{5}-1)=9>0$.\\
		Nên phương trình có hai nghiệm phân biệt $x_1$; $x_2$.\\
		Do đó, theo định lý Viète, ta được\\ $x_1+x_2=-\dfrac{b}{a}=-\dfrac{-5}{\sqrt{5}+1}=\dfrac{5(\sqrt{5}-1)}{4}$ và $x_1\cdot x_2=\dfrac{c}{a}=\dfrac{\sqrt{5}-1}{\sqrt{5}+1}=\dfrac{3-\sqrt{5}}{2}$.
	\end{enumerate}
}
\end{bt}
\begin{bt}%[Dự án EX-9-Đề Cương Toán 9]%[GVSB: Hoàng Minh Nhân Mã - GVPB1: Phan Minh Huế - GVPB2: Nguyễn Trần Anh Tuấn]%[9D4V3-1]
	Cho phương trình $x^2+2x-3=0$.
	\begin{enumerate}
		\item Chứng minh rằng phương trình có hai nghiệm $x_1$; $x_2$.
		\item Tính tổng và tích hai nghiệm trên.
		\item Tính $x_1^2+x_2^2$.
	\end{enumerate}
	\loigiai{
		\begin{enumerate}
			\item Ta có $\Delta=2^2-4\cdot 1\cdot(-3)=16>0$ nên phương trình có hai nghiệm $x_1$; $x_2$.
			\item Áp dụng định lý Viète, ta được $x_1+x_2=-\dfrac{b}{a}=-\dfrac{2}{1}=-2$ và $x_1x_2=\dfrac{c}{a}=\dfrac{-3}{1}=-3$.
			\item $x_1^2+x_2^2=(x_1+x_2)^2-2x_1x_2=(-2)^2-2\cdot (-3)=10$.
		\end{enumerate}
	}
\end{bt}
\begin{bt}%[Dự án EX-9-Đề Cương Toán 9]%[GVSB: Hoàng Minh Nhân Mã - GVPB1: Phan Minh Huế - GVPB2: Nguyễn Trần Anh Tuấn]%[9D4V3-1]
	Cho phương trình $5x^2-x-10=0$ có hai nghiệm $x_1$; $x_2$. Không giải phương trình hãy tính giá trị của biểu thức $A=x_1^2+x_2^2$.
	\loigiai{
		Phương trình $5x^2-x-10=0$ có $a=5$; $b=-1$; $c=-10$.\\
		Ta có $\Delta=b^2-4ac=(-1)^2-4\cdot 5\cdot (-10)=201>0$ nên phương trình có hai nghiệm phân biệt $x_1$; $x_2$.\\
		Theo định lý Viète, ta có
		$x_1+x_2=-\dfrac{b}{a}=-\dfrac{-1}{5}=\dfrac{1}{5}$ và
		$x_1x_2=\dfrac{c}{a}=\dfrac{-10}{5}=-2$.\\
		$A=x_1^2+x_2^2=(x_1+x_2)^2-2x_1x_2=\left(\dfrac{1}{5}\right)^2-2(-2)=\dfrac{101}{25}$.
	}
\end{bt}
\begin{bt}%[Dự án EX-9-Đề Cương Toán 9]%[GVSB: Hoàng Minh Nhân Mã - GVPB1: Phan Minh Huế - GVPB2: Nguyễn Trần Anh Tuấn]%[9D4V3-1]
	Cho phương trình $3x^2+x-2=0$. Không giải phương trình hãy tính tổng bình phương hai nghiệm.
	\loigiai{
		Phương trình $3x^2+x-2=0$ có $a=3$; $b=1$; $c=-2$.\\
		Ta có $\Delta=b^2-4ac=(1)^2-4\cdot 3\cdot (-2)=25>0$ nên phương trình có hai nghiệm phân biệt $x_1$; $x_2$.\\
		Theo định lý Viète, ta có
		$x_1+x_2=-\dfrac{b}{a}=-\dfrac{1}{3}$ và
		$x_1x_2=\dfrac{c}{a}=\dfrac{-2}{3}$.\\
		$x_1^2+x_2^2=(x_1+x_2)^2-2x_1x_2=\left(-\dfrac{1}{3}\right)^2-2\left(-\dfrac{2}{3}\right)=\dfrac{13}{9}$
	}
\end{bt}
\begin{bt}%[Dự án EX-9-Đề Cương Toán 9]%[GVSB: Hoàng Minh Nhân Mã - GVPB1: Phan Minh Huế - GVPB2: Nguyễn Trần Anh Tuấn]%[9D4V3-1]
	Nếu các phương trình sau có hai nghiệm $x_1$, $x_2$ thì hãy tính giá trị của các biểu thức $x_1^2+x_2^2$, $(x_1-x_2)^2$, $x_1-x_2$, $x_1^2-x_2^2$, $x_1^3+x_2^3$ mà không được giải phương trình
	\begin{multicols}{3}
		\begin{enumerate}
		\item $x^2-4x-12=0$;
		\item $x^2+5x+6=0$;
		\item $x^2+3x-2=0$;
		\item $x^2+2x-3=0$;
		\item $x^2+x-6=0$;
		\item $x^2+3x-2=0$;
		\item $2x^2-5x-7=0$;
		\item $3x^2-4x+1=0$;
		\item $-x^2-3x+10=0$;
		\item $4x^2+5x-1=0$;
		\item $-3x^2-8x+11=0$;
		\item $-5x^2+2x+8=0$.
		\end{enumerate}
	\end{multicols}
\loigiai{
	Ta có \begin{itemize}
		\item $x_1^3+x_2^3=(x_1^3+x_2^3+3x_1^2x_2+3x_1x_2^2)-(3x_1^2x_2+3x_1x_2^2)=(x_1+x_2)^3-3x_1x_2(x_1+x_2)$.
		\item $x_1-x_2=\pm |x_1-x_2|=\pm\sqrt{(x_1-x_2)^2}=\pm\sqrt{(x_1+x_2)^2-4x_1x_2}$.
	\end{itemize}
	\begin{enumerate}
	%Câu 1
	\item $x^2-4x-12=0$ với $a=1$; $b=-4$; $c=-12$.\\
	Ta có $\Delta=(-4)^2-4\cdot 1\cdot (-12)=64>0$ nên phương trình có hai nghiệm phân biệt $x_1$; $x_2$.\\
	Do đó, theo định lý Viète ta được $x_1+x_2=-\dfrac{b}{a}=-\dfrac{-4}{1}=4$ và $x_1\cdot x_2=\dfrac{c}{a}=\dfrac{-12}{1}=-12$.
	\begin{itemize}
	\item $x_1^2+x_2^2=(x_1+x_2)^2-2x_1x_2=4^2-2\cdot (-12)=40$.
	\item $(x_1-x_2)^2=(x_1+x_2)^2-4x_1x_2=4^2-4\cdot (-12)=64$.
	\item $x_1-x_2=\pm\sqrt{(x_1+x_2)^2-4x_1x_2}=\pm\sqrt{4^2-4\cdot (-12)}=\pm8$.
	\item $x_1^2-x_2^2=(x_1+x_2)(x_1-x_2)=4\cdot(\pm8)=\pm32$.
	\item $x_1^3+x_2^3=(x_1+x_2)^3-3x_1x_2(x_1+x_2)=4^3-3\cdot (-12)\cdot 4=208$.
	\end{itemize}
	%Câu 2
	\item $x^2+5x+6=0$ với $a=1$; $b=5$; $c=6$.\\
	Ta có $\Delta=(5)^2-4\cdot 1\cdot 6=1>0$ nên phương trình có hai nghiệm phân biệt $x_1$; $x_2$.\\
	Do đó, theo định lý Viète ta được $x_1+x_2=-\dfrac{b}{a}=-\dfrac{5}{1}=-5$ và $x_1\cdot x_2=\dfrac{c}{a}=\dfrac{6}{1}=6$.
	\begin{itemize}
	\item $x_1^2+x_2^2=(x_1+x_2)^2-2x_1x_2=(-5)^2-2\cdot 6=13$.
	\item $(x_1-x_2)^2=(x_1+x_2)^2-4x_1x_2=(-5)^2-4\cdot 6=1$.
	\item $x_1-x_2=\pm\sqrt{(x_1+x_2)^2-4x_1x_2}=\pm\sqrt{(-5)^2-4\cdot 6}=\pm1$.
	\item $x_1^2-x_2^2=(x_1+x_2)(x_1-x_2)=(-5)\cdot(\pm1)=\mp5$.
	\item $x_1^3+x_2^3=(x_1+x_2)^3-3x_1x_2(x_1+x_2)=(-5)^3-3\cdot 6\cdot (-5)=-35$.
	\end{itemize}
	%Câu 3
	\item $x^2+3x-2=0$ với $a=1$; $b=3$; $c=-2$.\\
	Ta có $\Delta=(3)^2-4\cdot 1\cdot (-2)=17>0$ nên phương trình có hai nghiệm phân biệt $x_1$; $x_2$.\\
	Do đó, theo định lý Viète ta được $x_1+x_2=-\dfrac{b}{a}=-\dfrac{3}{1}=-3$ và $x_1\cdot x_2=\dfrac{c}{a}=\dfrac{-2}{1}=-2$.
	\begin{itemize}
	\item $x_1^2+x_2^2=(x_1+x_2)^2-2x_1x_2=(-3)^2-2\cdot (-2)=13$.
	\item $(x_1-x_2)^2=(x_1+x_2)^2-4x_1x_2=(-3)^2-4\cdot (-2)=17$.
	\item $x_1-x_2=\pm\sqrt{(x_1+x_2)^2-4x_1x_2}=\pm\sqrt{(-3)^2-4\cdot (-2)}=\pm\sqrt{17}$.
	\item $x_1^2-x_2^2=(x_1+x_2)(x_1-x_2)=(-3)\cdot(\pm\sqrt{17})=\mp3\sqrt{17}$.
	\item $x_1^3+x_2^3=(x_1+x_2)^3-3x_1x_2(x_1+x_2)=(-3)^3-3\cdot (-2)\cdot (-3)=-45$.
	\end{itemize}
	%Câu 4
	\item $x^2+2x-3=0$ với $a=1$; $b=2$; $c=-3$.\\
	Ta có $\Delta=2^2-4\cdot 1\cdot (-3)=16>0$ nên phương trình có hai nghiệm phân biệt $x_1$; $x_2$.\\
	Do đó, theo định lý Viète ta được $x_1+x_2=-\dfrac{b}{a}=-\dfrac{2}{1}=-2$ và $x_1\cdot x_2=\dfrac{c}{a}=\dfrac{-3}{1}=-3$.
	\begin{itemize}
	\item $x_1^2+x_2^2=(x_1+x_2)^2-2x_1x_2=(-2)^2-2\cdot (-3)=10$.
	\item $(x_1-x_2)^2=(x_1+x_2)^2-4x_1x_2=(-2)^2-4\cdot (-3)=16$.
	\item $x_1-x_2=\pm\sqrt{(x_1+x_2)^2-4x_1x_2}=\pm\sqrt{(-2)^2-4\cdot (-3)}=\pm4$.
	\item $x_1^2-x_2^2=(x_1+x_2)(x_1-x_2)=(-2)\cdot(\pm4)=\mp8$.
	\item $x_1^3+x_2^3=(x_1+x_2)^3-3x_1x_2(x_1+x_2)=(-2)^3-3\cdot (-3)\cdot (-2)=-26$.
	\end{itemize}
	%Câu 5
	\item $x^2+x-6=0$ với $a=1$; $b=1$; $c=-6$.\\
	Ta có $\Delta=1^2-4\cdot 1\cdot (-6)=25>0$ nên phương trình có hai nghiệm phân biệt $x_1$; $x_2$.\\
	Do đó, theo định lý Viète ta được $x_1+x_2=-\dfrac{b}{a}=-\dfrac{1}{1}=-1$ và $x_1\cdot x_2=\dfrac{c}{a}=\dfrac{-6}{1}=-6$.
	\begin{itemize}
	\item $x_1^2+x_2^2=(x_1+x_2)^2-2x_1x_2=(-1)^2-2\cdot (-6)=13$.
	\item $(x_1-x_2)^2=(x_1+x_2)^2-4x_1x_2=(-1)^2-4\cdot (-6)=25$.
	\item $x_1-x_2=\pm\sqrt{(x_1+x_2)^2-4x_1x_2}=\pm\sqrt{(-1)^2-4\cdot (-6)}=\pm5$.
	\item $x_1^2-x_2^2=(x_1+x_2)(x_1-x_2)=(-1)\cdot(\pm5)=\mp5$.
	\item $x_1^3+x_2^3=(x_1+x_2)^3-3x_1x_2(x_1+x_2)=(-1)^3-3\cdot (-6)\cdot (-1)=-19$.
	\end{itemize}
	%Câu 6
	\item $x^2-3x-4=0$ với $a=1$; $b=-3$; $c=-4$.\\
	Ta có $\Delta=(-3)^2-4\cdot 1\cdot (-4)=25>0$ nên phương trình có hai nghiệm phân biệt $x_1$; $x_2$.\\
	Do đó, theo định lý Viète ta được $x_1+x_2=-\dfrac{b}{a}=-\dfrac{-3}{1}=3$ và $x_1\cdot x_2=\dfrac{c}{a}=\dfrac{-4}{1}=-4$.
	\begin{itemize}
	\item $x_1^2+x_2^2=(x_1+x_2)^2-2x_1x_2=3^2-2\cdot (-4)=17$.
	\item $(x_1-x_2)^2=(x_1+x_2)^2-4x_1x_2=3^2-4\cdot (-4)=25$.
	\item $x_1-x_2=\pm\sqrt{(x_1+x_2)^2-4x_1x_2}=\pm\sqrt{3^2-4\cdot (-4)}=\pm5$.
	\item $x_1^2-x_2^2=(x_1+x_2)(x_1-x_2)=3\cdot(\pm5)=\pm15$.
	\item $x_1^3+x_2^3=(x_1+x_2)^3-3x_1x_2(x_1+x_2)=3^3-3\cdot (-4)\cdot 3=63$.
	\end{itemize}
	%Câu 7
	\item $2x^2-5x-7=0$ với $a=2$; $b=-5$; $c=-7$.\\
	Ta có $\Delta=(-5)^2-4\cdot 2\cdot (-7)=81>0$ nên phương trình có hai nghiệm phân biệt $x_1$; $x_2$.\\
	Do đó, theo định lý Viète ta được $x_1+x_2=-\dfrac{b}{a}=-\dfrac{-5}{2}=\dfrac{5}{2}$ và $x_1\cdot x_2=\dfrac{c}{a}=\dfrac{-7}{2}$.
	\begin{itemize}
	\item $x_1^2+x_2^2=(x_1+x_2)^2-2x_1x_2=\left(\dfrac{5}{2}\right)^2-2\cdot \left(\dfrac{-7}{2}\right)=\dfrac{53}{4}$.
	\item $(x_1-x_2)^2=(x_1+x_2)^2-4x_1x_2=\left(\dfrac{5}{2}\right)^2-4\cdot \left(\dfrac{-7}{2}\right)=\dfrac{81}{4}$.
	\item $x_1-x_2=\pm\sqrt{(x_1+x_2)^2-4x_1x_2}=\pm\sqrt{\left(\dfrac{5}{2}\right)^2-4\cdot \left(\dfrac{-7}{2}\right)}=\pm\dfrac{9}{2}$.
	\item $x_1^2-x_2^2=(x_1+x_2)(x_1-x_2)=\dfrac{5}{2}\cdot\left(\pm\dfrac{9}{2}\right)=\pm\dfrac{45}{4}$.
	\item $x_1^3+x_2^3=(x_1+x_2)^3-3x_1x_2(x_1+x_2)=\left(\dfrac{5}{2}\right)^3-3\cdot \left(\dfrac{-7}{2}\right)\cdot \dfrac{5}{2}=\dfrac{335}{8}$.
	\end{itemize}
	%Câu 8
	\item $3x^2-4x+1=0$ với $a=3$; $b=-4$; $c=1$.\\
	Ta có $\Delta=(-4)^2-4\cdot 3\cdot 1=4>0$ nên phương trình có hai nghiệm phân biệt $x_1$; $x_2$.\\
	Do đó, theo định lý Viète ta được $x_1+x_2=-\dfrac{b}{a}=-\dfrac{-4}{3}=\dfrac{4}{3}$ và $x_1\cdot x_2=\dfrac{c}{a}=\dfrac{1}{3}$.
	\begin{itemize}
	\item $x_1^2+x_2^2=(x_1+x_2)^2-2x_1x_2=\left(\dfrac{4}{3}\right)^2-2\cdot \left(\dfrac{1}{3}\right)=\dfrac{10}{9}$.
	\item $(x_1-x_2)^2=(x_1+x_2)^2-4x_1x_2=\left(\dfrac{4}{3}\right)^2-4\cdot \left(\dfrac{1}{3}\right)=\dfrac{4}{9}$.
	\item $x_1-x_2=\pm\sqrt{(x_1+x_2)^2-4x_1x_2}=\pm\sqrt{\left(\dfrac{4}{3}\right)^2-4\cdot \left(\dfrac{1}{3}\right)}=\pm\dfrac{2}{3}$.
	\item $x_1^2-x_2^2=(x_1+x_2)(x_1-x_2)=\dfrac{4}{3}\cdot\left(\pm\dfrac{2}{3}\right)=\pm\dfrac{8}{9}$.
	\item $x_1^3+x_2^3=(x_1+x_2)^3-3x_1x_2(x_1+x_2)=\left(\dfrac{4}{3}\right)^3-3\cdot \left(\dfrac{1}{3}\right)\cdot \dfrac{4}{3}=\dfrac{28}{27}$.
	\end{itemize}
	%Câu 9
	\item $-x^2-3x+10=0$ với $a=-1$; $b=-3$; $c=10$.\\
	Ta có $\Delta=(-3)^2-4\cdot (-1)\cdot 10=49>0$ nên phương trình có hai nghiệm phân biệt $x_1$; $x_2$.\\
	Do đó, theo định lý Viète ta được $x_1+x_2=-\dfrac{b}{a}=-\dfrac{-3}{-1}=-3$ và $x_1\cdot x_2=\dfrac{c}{a}=\dfrac{10}{-1}=-10$.
	\begin{itemize}
	\item $x_1^2+x_2^2=(x_1+x_2)^2-2x_1x_2=(-3)^2-2\cdot (-10)=29$.
	\item $(x_1-x_2)^2=(x_1+x_2)^2-4x_1x_2=(-3)^2-4\cdot (-10)=49$.
	\item $x_1-x_2=\pm\sqrt{(x_1+x_2)^2-4x_1x_2}=\pm\sqrt{(-3)^2-4\cdot (-10)}=\pm7$.
	\item $x_1^2-x_2^2=(x_1+x_2)(x_1-x_2)=(-3)\cdot(\pm7)=\mp21$.
	\item $x_1^3+x_2^3=(x_1+x_2)^3-3x_1x_2(x_1+x_2)=(-3)^3-3\cdot (-10)\cdot (-3)=-117$.
	\end{itemize}
	%Câu 10
	\item $4x^2+5x-1=0$ với $a=4$; $b=5$; $c=-1$.\\
	Ta có $\Delta=(5)^2-4\cdot 4\cdot (-1)=41>0$ nên phương trình có hai nghiệm phân biệt $x_1$; $x_2$.\\
	Do đó, theo định lý Viète ta được $x_1+x_2=-\dfrac{b}{a}=-\dfrac{5}{4}$ và $x_1\cdot x_2=\dfrac{c}{a}=\dfrac{-1}{4}$.
	\begin{itemize}
	\item $x_1^2+x_2^2=(x_1+x_2)^2-2x_1x_2=\left(-\dfrac{5}{4}\right)^2-2\cdot \left(-\dfrac{1}{4}\right)=\dfrac{33}{16}$.
	\item $(x_1-x_2)^2=(x_1+x_2)^2-4x_1x_2=\left(-\dfrac{5}{4}\right)^2-4\cdot \left(-\dfrac{1}{4}\right)=\dfrac{41}{16}$.
	\item $x_1-x_2=\pm\sqrt{(x_1+x_2)^2-4x_1x_2}=\pm\sqrt{\left(-\dfrac{5}{4}\right)^2-4\cdot \left(-\dfrac{1}{4}\right)}=\pm\dfrac{\sqrt{41}}{4}$.
	\item $x_1^2-x_2^2=(x_1+x_2)(x_1-x_2)=\left(-\dfrac{5}{4}\right)\cdot\left(\pm\dfrac{\sqrt{41}}{4}\right)=\mp\dfrac{5\sqrt{41}}{16}$.
	\item $x_1^3+x_2^3=(x_1+x_2)^3-3x_1x_2(x_1+x_2)=\left(-\dfrac{5}{4}\right)^3-3\cdot \left(-\dfrac{1}{4}\right)\cdot \left(-\dfrac{5}{4}\right)=-\dfrac{185}{64}$.
	\end{itemize}
	%Câu 11
	\item $-3x^2-8x+11=0$ với $a=-3$; $b=-8$; $c=11$.\\
	Ta có $\Delta=(-8)^2-4\cdot (-3)\cdot 11=196>0$ nên phương trình có hai nghiệm phân biệt $x_1$; $x_2$.\\
	Do đó, theo định lý Viète ta được $x_1+x_2=-\dfrac{b}{a}=-\dfrac{-8}{-3}=-\dfrac{8}{3}$ và $x_1\cdot x_2=\dfrac{c}{a}=\dfrac{11}{-3}=-\dfrac{11}{3}$.
	\begin{itemize}
	\item $x_1^2+x_2^2=(x_1+x_2)^2-2x_1x_2=\left(-\dfrac{8}{3}\right)^2-2\cdot \left(-\dfrac{11}{3}\right)=\dfrac{130}{9}$.
	\item $(x_1-x_2)^2=(x_1+x_2)^2-4x_1x_2=\left(-\dfrac{8}{3}\right)^2-4\cdot \left(-\dfrac{11}{3}\right)=\dfrac{196}{9}$.
	\item $x_1-x_2=\pm\sqrt{(x_1+x_2)^2-4x_1x_2}=\pm\sqrt{\left(-\dfrac{8}{3}\right)^2-4\cdot \left(-\dfrac{11}{3}\right)}=\pm\dfrac{14}{3}$.
	\item $x_1^2-x_2^2=(x_1+x_2)(x_1-x_2)=\left(-\dfrac{8}{3}\right)\cdot\left(\pm\dfrac{14}{3}\right)=\mp\dfrac{112}{9}$.
	\item $x_1^3+x_2^3=(x_1+x_2)^3-3x_1x_2(x_1+x_2)=\left(-\dfrac{8}{3}\right)^3-3\cdot \left(-\dfrac{11}{3}\right)\cdot \left(-\dfrac{8}{3}\right)=-\dfrac{1304}{27}$.
	\end{itemize}
	%Câu 12
	\item $-5x^2+2x+8=0$ với $a=-5$; $b=2$; $c=8$.\\
	Ta có $\Delta=2^2-4\cdot (-5)\cdot 8=164>0$ nên phương trình có hai nghiệm phân biệt $x_1$; $x_2$.\\
	Do đó, theo định lý Viète ta được $x_1+x_2=-\dfrac{b}{a}=-\dfrac{2}{-5}=\dfrac{2}{5}$ và $x_1\cdot x_2=\dfrac{c}{a}=\dfrac{8}{-5}=-\dfrac{8}{5}$.
	\begin{itemize}
	\item $x_1^2+x_2^2=(x_1+x_2)^2-2x_1x_2=\left(\dfrac{2}{5}\right)^2-2\cdot \left(-\dfrac{8}{5}\right)=\dfrac{84}{25}$.
	\item $(x_1-x_2)^2=(x_1+x_2)^2-4x_1x_2=\left(\dfrac{2}{5}\right)^2-4\cdot \left(-\dfrac{8}{5}\right)=\dfrac{164}{25}$.
	\item $x_1-x_2=\pm\sqrt{(x_1+x_2)^2-4x_1x_2}=\pm\sqrt{\left(\dfrac{2}{5}\right)^2-4\cdot \left(-\dfrac{8}{5}\right)}=\pm\dfrac{\sqrt{164}}{5}=\pm\dfrac{2\sqrt{41}}{5}$.
	\item $x_1^2-x_2^2=(x_1+x_2)(x_1-x_2)=\dfrac{2}{5}\cdot\left(\pm\dfrac{2\sqrt{41}}{5}\right)=\pm\dfrac{4\sqrt{41}}{25}$.
	\item $x_1^3+x_2^3=(x_1+x_2)^3-3x_1x_2(x_1+x_2)=\left(\dfrac{2}{5}\right)^3-3\cdot \left(-\dfrac{8}{5}\right)\cdot \dfrac{2}{5}=\dfrac{248}{125}$.
	\end{itemize}
	\end{enumerate}
}
\end{bt}
\begin{bt}%[Dự án EX-9-Đề Cương Toán 9]%[GVSB: Hoàng Minh Nhân Mã - GVPB1: Phan Minh Huế - GVPB2: Nguyễn Trần Anh Tuấn]%[9D4V3-1]
	Cho phương trình $x^2-2x-1=0$ có hai nghiệm $x_1$; $x_2$. Không giải phương trình, hãy tính giá trị của các biểu thức sau.
	\begin{multicols}{3}
	\begin{enumerate}
		\item $5x_1x_2^2+5x_2x_1^2$;
		\item $x_1^2+x_2^2$;
		\item $2x_2x_1^3+2x_1x_2^3$;
		\item $\dfrac{x_1}{x_2}+\dfrac{x_2}{x_1}$;
		\item $\dfrac{3}{x_1}+\dfrac{3}{x_2}$;
		\item $\dfrac{x_1+1}{x_2}+\dfrac{x_2+1}{x_1}$;
		\item $\dfrac{x_1}{x_2+2}+\dfrac{x_2}{x_1+2}$;
		\item $\dfrac{x_1-1}{x_2}+\dfrac{x_2-1}{x_1}$;
		\item $\dfrac{x_1^2+2}{x_2}+\dfrac{x_2^2+2}{x_1}$.
	\end{enumerate}
	\end{multicols}
\loigiai{
	Ta có $\Delta=(-2)^2-4\cdot 2\cdot (-1)=12>0$ nên phương trình có hai nghiệm phân biệt $x_1$; $x_2$.\\
	Do đó, theo định lý Viète ta được $x_1+x_2=-\dfrac{b}{a}=-\dfrac{-2}{1}=2$ và $x_1\cdot x_2=\dfrac{c}{a}=\dfrac{-1}{1}=-1$.
	\begin{enumerate}
	\item $5x_1x_2^2+5x_2x_1^2=5x_1x_2(x_1+x_2)=5\cdot (-1)\cdot 2=-10$.
	\item $x_1^2+x_2^2=(x_1+x_2)^2-2x_1x_2=(2)^2-2\cdot (-1)=4+2=6$.
	\item $2x_2x_1^3+2x_1x_2^3=2x_1x_2(x_1^2+x_2^2)=2\cdot (-1)\cdot (6)=-12$.
	\item $\dfrac{x_1}{x_2}+\dfrac{x_2}{x_1}=\dfrac{x_1^2+x_2^2}{x_1x_2}=\dfrac{6}{-1}=-6$.
	\item $\dfrac{3}{x_1}+\dfrac{3}{x_2}=3\left(\dfrac{x_1+x_2}{x_1x_2}\right)=3\left(\dfrac{2}{-1}\right)=3\cdot (-2)=-6$.
	\item \begin{eqnarray*}
		\dfrac{x_1+1}{x_2}+\dfrac{x_2+1}{x_1} & = & \dfrac{x_1(x_1+1)+x_2(x_2+1)}{x_1x_2}\\
		& = & \dfrac{x_1^2+x_1+x_2^2+x_2}{x_1x_2}\\
		& = & \dfrac{(x_1^2+x_2^2)+(x_1+x_2)}{x_1x_2}\\
		& = & \dfrac{6+2}{-1}\\
		& = & -8.
	\end{eqnarray*}
	\item \begin{eqnarray*}
		\dfrac{x_1}{x_2+2}+\dfrac{x_2}{x_1+2} & = & \dfrac{x_1(x_1+2)+x_2(x_2+2)}{(x_2+2)(x_1+2)}\\
		& = & \dfrac{x_1^2+2x_1+x_2^2+2x_2}{x_1x_2+2x_1+2x_2+4}\\
		& = & \dfrac{(x_1^2+x_2^2)+2(x_1+x_2)}{x_1x_2+2(x_1+x_2)+4}\\
		& = & \dfrac{6+2\cdot 2}{-1+2\cdot 2+4}\\
		& = & \dfrac{10}{7}.
	\end{eqnarray*}
	\item \begin{eqnarray*}
		\dfrac{x_1-1}{x_2}+\dfrac{x_2-1}{x_1} & = & \dfrac{x_1(x_1-1)+x_2(x_2-1)}{x_1x_2}\\
		& = & \dfrac{x_1^2-x_1+x_2^2-x_2}{x_1x_2}\\
		& = & \dfrac{(x_1^2+x_2^2)-(x_1+x_2)}{x_1x_2}\\
		& = & \dfrac{6-2}{-1}\\
		& = & -4.
	\end{eqnarray*}
	\item \begin{eqnarray*}
		\dfrac{x_1^2+2}{x_2}+\dfrac{x_2^2+2}{x_1} & = & \dfrac{x_1(x_1^2+2)+x_2(x_2^2+2)}{x_1x_2}\\
		& = & \dfrac{x_1^3+2x_1+x_2^3+2x_2}{x_1x_2}\\
		& = & \dfrac{(x_1^3+x_2^3)+2(x_1+x_2)}{x_1x_2}\\
		& = & \dfrac{(x_1+x_2)^3-3x_1x_2(x_1+x_2)+2(x_1+x_2)}{x_1x_2}\\
		& = & \dfrac{2^3-3(-1)\cdot 2+2\cdot 2}{-1}\\
		& = & -18.
	\end{eqnarray*}
	\end{enumerate}
}
\end{bt}
\begin{bt}%[Dự án EX-9-Đề Cương Toán 9]%[GVSB: Hoàng Minh Nhân Mã - GVPB1: Phan Minh Huế - GVPB2: Nguyễn Trần Anh Tuấn]%[9D4V3-1]
	Cho phương trình: $2x^2-4x-3=0$. Không giải phương trình, hãy:
	\begin{enumerate}
		\item Tính tổng và tích các nghiệm $x_1$; $x_2$ của phương trình trên.
		\item Tính giá trị biểu thức: $M=x_1^2+x_2^2-4x_1x_2$.
	\end{enumerate}
\loigiai{
	\begin{enumerate}
		\item Phương trình $2x^2-4x-3=0$ có $a=2$; $b=-4$; $c=-3$.\\
		Ta có $\Delta=b^2-4ac=(-4)^2-4\cdot 2\cdot (-3)=40>0$.\\
		Nên phương trình có hai nghiệm phân biệt $x_1$; $x_2$.\\
		Theo định lý Viète, ta có
		$x_1+x_2=-\dfrac{b}{a}=-\dfrac{-4}{2}=2$. và
		$x_1x_2=\dfrac{c}{a}=\dfrac{-3}{2}$.
		\item $M=(x_1^2+x_2^2)-4x_1x_2=(x_1+x_2)^2-2x_1x_2-4x_1x_2=2^2-6\cdot\dfrac{-3}{2}=13$.
	\end{enumerate}
}
\end{bt}
\begin{bt}%[Dự án EX-9-Đề Cương Toán 9]%[GVSB: Hoàng Minh Nhân Mã - GVPB1: Phan Minh Huế - GVPB2: Nguyễn Trần Anh Tuấn]%[9D4V3-1]
	Cho phương trình: $x^2+3x-4=0$.
	\begin{enumerate}
		\item Chứng minh rằng phương trình có hai nghiệm phân biệt $x_1$; $x_2$.
		\item Tính tổng và tích hai nghiệm đó.
		\item Tính $\dfrac{2x_1+2x_2}{x_1x_2+1}$; $\dfrac{1}{x_1}+\dfrac{1}{x_2}$; $(x_1+x_2)^2-3x_1x_2$; $x_1^2+x_2^2$.
	\end{enumerate}
\loigiai{
	\begin{enumerate}
	\item Phương trình $x^2+3x-4=0$ có $a=1$; $b=3$; $c=-4$.\\
		Ta có $\Delta=b^2-4ac=3^2-4\cdot 1\cdot (-4)=25>0$.\\
		Nên phương trình có hai nghiệm phân biệt $x_1$; $x_2$.
	\item Theo định lý Viète, ta có
		$x_1+x_2=-\dfrac{b}{a}=-\dfrac{3}{1}=-3$ và
		$x_1x_2=\dfrac{c}{a}=\dfrac{-4}{1}=-4$.
	\item \begin{itemize}
		\item $\dfrac{2x_1+2x_2}{x_1x_2+1}=\dfrac{2(x_1+x_2)}{x_1x_2+1}=\dfrac{2\cdot (-3)}{-4+1}=2$.
		\item $\dfrac{1}{x_1}+\dfrac{1}{x_2}=\dfrac{x_1+x_2}{x_1x_2}=\dfrac{-3}{-4}=\dfrac{3}{4}$.
		\item $(x_1+x_2)^2-3x_1x_2=(-3)^2-3\cdot (-4)=21$.
		\item $x_1^2+x_2^2=(x_1+x_2)^2-2x_1x_2=(-3)^2-2\cdot (-4)=17$. 
	\end{itemize}
	\end{enumerate}
}
\end{bt}
\begin{bt}%[Dự án EX-9-Đề Cương Toán 9]%[GVSB: Hoàng Minh Nhân Mã - GVPB1: Phan Minh Huế - GVPB2: Nguyễn Trần Anh Tuấn]%[9D4V3-1]
	Cho phương trình: $3x^2-5x-4=0$.
	\begin{enumerate}
		\item Chứng minh rằng phương trình có hai nghiệm $x_1$; $x_2$.
		\item Tính tổng và tích hai nghiệm đó.
		\item Tính $x_1^2x_2+x_1x_2^2$; $x_1^2+x_2^2$; $x_1^3x_2+x_1x_2^3$.
	\end{enumerate}
\loigiai{
	\begin{enumerate}
		\item
		Phương trình $3x^2-5x-4=0$ có $a=3$; $b=-5$; $c=-4$.\\
		Ta có $\Delta=b^2-4ac=(-5)^2-4\cdot 3\cdot (-4)=73>0$\\
		Nên phương trình có hai nghiệm phân biệt $x_1$; $x_2$.
		\item Theo định lý Viète, ta có
		$x_1+x_2=-\dfrac{b}{a}=-\dfrac{-5}{3}=\dfrac{5}{3}$ và
		$x_1x_2=\dfrac{c}{a}=\dfrac{-4}{3}$.
		\item
		\begin{itemize}
			\item $x_1^2x_2+x_1x_2^2=x_1x_2(x_1+x_2)=\left(-\dfrac{4}{3}\right)\cdot\left(\dfrac{5}{3}\right)=-\dfrac{20}{9}$.
			\item $x_1^2+x_2^2=(x_1+x_2)^2-2x_1x_2=\left(\dfrac{5}{3}\right)^2-2\cdot\left(-\dfrac{4}{3}\right)=\dfrac{25}{9}+\dfrac{8}{3}=\dfrac{25+24}{9}=\dfrac{49}{9}$.
			\item $x_1^3x_2+x_1x_2^3=x_1x_2(x_1^2+x_2^2)=\left(-\dfrac{4}{3}\right)\cdot\left(\dfrac{49}{9}\right)=-\dfrac{196}{27}$.
		\end{itemize}
	\end{enumerate}
}
\end{bt}
\begin{bt}%[Dự án EX-9-Đề Cương Toán 9]%[GVSB: Hoàng Minh Nhân Mã - GVPB1: Phan Minh Huế - GVPB2: Nguyễn Trần Anh Tuấn]%[9D4V3-1]
	Cho phương trình $x^2+9x-8=0$.
	\begin{enumerate}
		\item Chứng minh rằng phương trình có hai nghiệm phân biệt $x_1, x_2$.
		\item Tính $\dfrac{1}{x_1}+\dfrac{1}{x_2}$; $x_1^2+x_2^2$; $\dfrac{x_1}{x_2}+\dfrac{x_2}{x_1}$; $x_1^2x_2+x_1x_2^2$.
	\end{enumerate}
\loigiai{
	\begin{enumerate}
		\item
		Phương trình $x^2+9x-8=0$ có $a=1$; $b=9$; $c=-8$.\\
		Ta có $\Delta=b^2-4ac=(9)^2-4\cdot 1\cdot (-8)=113>0$\\
		Nên phương trình có hai nghiệm phân biệt $x_1$; $x_2$.
		\item Theo định lý Viète, ta có
		$x_1+x_2=-\dfrac{b}{a}=-\dfrac{9}{1}=-9$ và
		$x_1x_2=\dfrac{c}{a}=\dfrac{-8}{1}=-8$.
		\begin{itemize}
			\item $\dfrac{1}{x_1}+\dfrac{1}{x_2}=\dfrac{x_1+x_2}{x_1x_2}=\dfrac{-9}{-8}=\dfrac{9}{8}$.
			\item $x_1^2+x_2^2=(x_1+x_2)^2-2x_1x_2=(-9)^2-2\cdot(-8)=97$.
			\item $\dfrac{x_1}{x_2}+\dfrac{x_2}{x_1}=\dfrac{x_1^2+x_2^2}{x_1x_2}=\dfrac{97}{-8}=-\dfrac{97}{8}$.
			\item $x_1^2x_2+x_1x_2^2=x_1x_2(x_1+x_2)=(-8)\cdot(-9)=72$.
		\end{itemize}
	\end{enumerate}
}
\end{bt}
\begin{bt}%[Dự án EX-9-Đề Cương Toán 9]%[GVSB: Hoàng Minh Nhân Mã - GVPB1: Phan Minh Huế - GVPB2: Nguyễn Trần Anh Tuấn]%[9D4V3-1]
	Cho phương trình: $3x^2-5x+2=0$.
	\begin{enumerate}
		\item Chứng minh rằng phương trình có hai nghiệm phân biệt $x_1, x_2$.
		\item Tính tổng và tích hai nghiệm trên.
		\item Tính $x_1^2+x_2^2$; $x_1^3+x_2^3$.
	\end{enumerate}
\loigiai{
	\begin{enumerate}
		\item
		Phương trình $3x^2-5x+2=0$ có $a=3$; $b=-5$; $c=2$.\\
		Ta có $\Delta=b^2-4ac=(-5)^2-4\cdot 3\cdot 2=1>0$\\
		Nên phương trình có hai nghiệm phân biệt $x_1$; $x_2$.
		\item Theo định lý Viète, ta có
		$x_1+x_2=-\dfrac{b}{a}=-\dfrac{-5}{3}=\dfrac{5}{3}$ và
		$x_1x_2=\dfrac{c}{a}=\dfrac{2}{3}$.
		\item
		\begin{itemize}
			\item $x_1^2+x_2^2=(x_1+x_2)^2-2x_1x_2=\left(\dfrac{5}{3}\right)^2-2\cdot\dfrac{2}{3}=\dfrac{13}{9}$.
			\item $x_1^3+x_2^3=(x_1+x_2)^3-3x_1x_2(x_1+x_2)=\left(\dfrac{5}{3}\right)^3-3\cdot\dfrac{2}{3}\cdot\dfrac{5}{3}=\dfrac{35}{27}$.
		\end{itemize}
	\end{enumerate}
}
\end{bt}
\begin{bt}%[Dự án EX-9-Đề Cương Toán 9]%[GVSB: Hoàng Minh Nhân Mã - GVPB1: Phan Minh Huế - GVPB2: Nguyễn Trần Anh Tuấn]%[9D4V3-1]
	Cho phương trình: $x^2+x-11=0$.
	\begin{enumerate}
		\item Không giải phương trình, hãy chứng minh rằng phương trình có hai nghiệm phân biệt $x_1$; $x_2$.
		\item Tính $x_1^2+x_2^2$; $\dfrac{1}{x_1}+\dfrac{1}{x_2}$; $\dfrac{x_1}{x_2}+\dfrac{x_2}{x_1}$.
	\end{enumerate}
\loigiai{
	\begin{enumerate}
		\item
		Phương trình $x^2+x-11=0$ có $a=1$; $b=1$; $c=-11$.\\
		Ta có $\Delta=b^2-4ac=(1)^2-4\cdot 1\cdot (-11)=1+44=45>0$\\
		Nên phương trình có hai nghiệm phân biệt $x_1$; $x_2$.
		\item Theo định lý Viète, ta có
		$x_1+x_2=-\dfrac{b}{a}=-\dfrac{1}{1}=-1$ và
		$x_1x_2=\dfrac{c}{a}=\dfrac{-11}{1}=-11$.
		\begin{itemize}
			\item $x_1^2+x_2^2=(x_1+x_2)^2-2x_1x_2=(-1)^2-2\cdot(-11)=1+22=23$.
			\item $\dfrac{1}{x_1}+\dfrac{1}{x_2}=\dfrac{x_1+x_2}{x_1x_2}=\dfrac{-1}{-11}=\dfrac{1}{11}$.
			\item $\dfrac{x_1}{x_2}+\dfrac{x_2}{x_1}=\dfrac{x_1^2+x_2^2}{x_1x_2}=\dfrac{23}{-11}=-\dfrac{23}{11}$.
		\end{itemize}
	\end{enumerate}
}
\end{bt}
\begin{bt}%[Dự án EX-9-Đề Cương Toán 9]%[GVSB: Hoàng Minh Nhân Mã - GVPB1: Phan Minh Huế - GVPB2: Nguyễn Trần Anh Tuấn]%[9D4V3-1]
Cho phương trình: $3x^2-7x+4=0$. Gọi $x_1$; $x_2$ là nghiệm của phương trình. Không giải phương trình, hãy tính giá trị của biểu thức $A=(x_1+2x_2)(x_2+2x_1)-x_1^2x_2^2$.
\loigiai{
	Phương trình $3x^2-7x+4=0$ có $a=3$; $b=-7$; $c=4$.\\
	Ta có $\Delta=b^2-4ac=(-7)^2-4\cdot 3\cdot 4=49-48=1>0$ nên phương trình có hai nghiệm phân biệt $x_1$; $x_2$.\\
	Theo định lý Viète, ta có $x_1+x_2=-\dfrac{b}{a}=-\dfrac{-7}{3}=\dfrac{7}{3}$ và $x_1x_2=\dfrac{c}{a}=\dfrac{4}{3}$.
	\begin{eqnarray*}
		A  & = & (x_1+2x_2)(x_2+2x_1)-x_1^2x_2^2\\
		& = & x_1x_2+2x_1^2+2x_2^2+4x_1x_2-(x_1x_2)^2\\
		& = & (2x_1^2+4x_1x_2+2x_2^2)+x_1x_2-(x_1x_2)^2\\
		& = & 2(x_1+x_2)^2+x_1x_2-(x_1x_2)^2\\
		& = & 2\cdot\left(\dfrac{7}{3}\right)^2+\dfrac{4}{3}-\left(\dfrac{4}{3}\right)^2\\
		& = & \dfrac{94}{9}.
	\end{eqnarray*}
}
\end{bt}
\begin{bt}%[Dự án EX-9-Đề Cương Toán 9]%[GVSB: Hoàng Minh Nhân Mã - GVPB1: Phan Minh Huế - GVPB2: Nguyễn Trần Anh Tuấn]%[9D4V3-1]
	Cho phương trình: $3x^2-7x+2=0$.
	\begin{enumerate}
		\item Không giải phương trình, hãy chứng minh rằng phương trình có hai nghiệm phân biệt $x_1$; $x_2$.
		\item Tính $x_1+x_2$; $x_1x_2$; $x_1^2+x_2^2$; $(x_1+1)(x_2+1)$; $x_1^2+x_2^2+(x_1+1)(x_2+1)$.
	\end{enumerate}
\loigiai{
	\begin{enumerate}
		\item
		Phương trình $3x^2-7x+2=0$ có $a=3$; $b=-7$; $c=2$.\\
		Ta có $\Delta=b^2-4ac=(-7)^2-4\cdot 3\cdot 2=25>0$\\
		Nên phương trình có hai nghiệm phân biệt $x_1$; $x_2$.
		\item Theo định lý Viète, ta có
		$x_1+x_2=-\dfrac{b}{a}=-\dfrac{-7}{3}=\dfrac{7}{3}$ và
		$x_1x_2=\dfrac{c}{a}=\dfrac{2}{3}$.
		\begin{itemize}
			\item $x_1^2+x_2^2=(x_1+x_2)^2-2x_1x_2=\left(\dfrac{7}{3}\right)^2-2\cdot\dfrac{2}{3}=\dfrac{37}{9}$.
			\item $(x_1+1)(x_2+1)=x_1x_2+x_1+x_2+1=\dfrac{2}{3}+\dfrac{7}{3}+1=4$.
			\item $x_1^2+x_2^2+(x_1+1)(x_2+1)=\dfrac{37}{9}+4=\dfrac{73}{9}$.
		\end{itemize}
	\end{enumerate}
}
\end{bt}
\begin{bt}%[Dự án EX-9-Đề Cương Toán 9]%[GVSB: Hoàng Minh Nhân Mã - GVPB1: Phan Minh Huế - GVPB2: Nguyễn Trần Anh Tuấn]%[9D4V3-1]
	Cho phương trình: $3x^2+6x-7=0$.
	\begin{enumerate}
		\item Chứng minh rằng phương trình có hai nghiệm phân biệt $x_1$; $x_2$.
		\item Tính $x_1+x_2$; $x_1x_2$; $x_1^2+x_2^2$; $2x_1^2+3x_1x_2+2x_2^2$.
	\end{enumerate}
\loigiai{
	\begin{enumerate}
		\item
		Phương trình $3x^2+6x-7=0$ có $a=3$; $b=6$; $c=-7$.\\
		Ta có $\Delta=b^2-4ac=(6)^2-4\cdot 3\cdot (-7)=120>0$\\
		Nên phương trình có hai nghiệm phân biệt $x_1$; $x_2$.
		\item Theo định lý Viète, ta có
		$x_1+x_2=-\dfrac{b}{a}=-\dfrac{6}{3}=-2$ và
		$x_1x_2=\dfrac{c}{a}=\dfrac{-7}{3}$.
		\begin{itemize}
			\item $x_1^2+x_2^2=(x_1+x_2)^2-2x_1x_2=(-2)^2-2\cdot\left(-\dfrac{7}{3}\right)=\dfrac{26}{3}$.
			\item $2x_1^2+3x_1x_2+2x_2^2=2(x_1^2+x_2^2)+3x_1x_2=2\cdot\dfrac{26}{3}+3\cdot\left(-\dfrac{7}{3}\right)=\dfrac{31}{3}$.
		\end{itemize}
	\end{enumerate}
}
\end{bt}
\begin{bt}%[Dự án EX-9-Đề Cương Toán 9]%[GVSB: Hoàng Minh Nhân Mã - GVPB1: Phan Minh Huế - GVPB2: Nguyễn Trần Anh Tuấn]%[9D4V3-1]
	Cho phương trình: $x^2-3x-5=0$.
	\begin{enumerate}
		\item Không giải phương trình, hãy chứng minh rằng phương trình có hai nghiệm phân biệt $x_1$; $x_2$.
		\item Tính $x_1+x_2$; $x_1x_2$; $x_1^2+x_2^2$; $\dfrac{1}{x_1}+\dfrac{1}{x_2}$; $\dfrac{x_1}{x_2}+\dfrac{x_2}{x_1}$; $(x_1-x_2)^2$; $(x_1-x_2)^2-5x_1x_2$.
	\end{enumerate}
\loigiai{
	\begin{enumerate}
		\item
		Phương trình $x^2-3x-5=0$ có $a=1$; $b=-3$; $c=-5$.\\
		Ta có $\Delta=b^2-4ac=(-3)^2-4\cdot 1\cdot (-5)=29>0$\\
		Nên phương trình có hai nghiệm phân biệt $x_1$; $x_2$.
		\item Theo định lý Viète, ta có
		$x_1+x_2=-\dfrac{b}{a}=-\dfrac{-3}{1}=3$ và
		$x_1x_2=\dfrac{c}{a}=\dfrac{-5}{1}=-5$.
		\begin{itemize}
			\item $x_1^2+x_2^2=(x_1+x_2)^2-2x_1x_2=(3)^2-2\cdot(-5)=19$.
			\item $\dfrac{1}{x_1}+\dfrac{1}{x_2}=\dfrac{x_1+x_2}{x_1x_2}=\dfrac{3}{-5}=-\dfrac{3}{5}$.
			\item $\dfrac{x_1}{x_2}+\dfrac{x_2}{x_1}=\dfrac{x_1^2+x_2^2}{x_1x_2}=\dfrac{19}{-5}=-\dfrac{19}{5}$.
			\item $(x_1-x_2)^2=(x_1+x_2)^2-4x_1x_2=3^2-4\cdot(-5)=29$.
			\item $(x_1-x_2)^2-5x_1x_2=29-5\cdot(-5)=54$.
		\end{itemize}
	\end{enumerate}
}
\end{bt}
\begin{bt}%[Dự án EX-9-Đề Cương Toán 9]%[GVSB: Hoàng Minh Nhân Mã - GVPB1: Phan Minh Huế - GVPB2: Nguyễn Trần Anh Tuấn]%[9D4V3-1]
	Cho phương trình: $x^2-2x-6=0$.
	\begin{enumerate}
		\item Chứng minh rằng phương trình có hai nghiệm phân biệt $x_1$; $x_2$.
		\item Tính giá trị của các biểu thức sau: $x_1+x_2$; $x_1x_2$; $x_1^2+x_2^2$; $(x_1-x_2)^2$; $x_1^2+x_2^2+7x_1x_2$.
	\end{enumerate}
\loigiai{
	\begin{enumerate}
		\item
		Phương trình $x^2-2x-6=0$ có $a=1$; $b=-2$; $c=-6$.\\
		Ta có $\Delta=b^2-4ac=(-2)^2-4\cdot 1\cdot (-6)=28>0$\\
		Nên phương trình có hai nghiệm phân biệt $x_1$; $x_2$.
		\item Theo định lý Viète, ta có
		$x_1+x_2=-\dfrac{b}{a}=-\dfrac{-2}{1}=2$ và
		$x_1x_2=\dfrac{c}{a}=\dfrac{-6}{1}=-6$.
		\begin{itemize}
			\item $x_1^2+x_2^2=(x_1+x_2)^2-2x_1x_2=(2)^2-2\cdot(-6)=16$.
			\item $(x_1-x_2)^2=(x_1+x_2)^2-4x_1x_2=(2)^2-4\cdot(-6)=28$.
			\item $x_1^2+x_2^2+7x_1x_2=16+7\cdot(-6)=-26$.
		\end{itemize}
	\end{enumerate}
}
\end{bt}
\begin{bt}%[Dự án EX-9-Đề Cương Toán 9]%[GVSB: Hoàng Minh Nhân Mã - GVPB1: Phan Minh Huế - GVPB2: Nguyễn Trần Anh Tuấn]%[9D4V3-1]
	Cho phương trình: $-x^2+8x-7=0$.
	\begin{enumerate}
		\item Chứng minh rằng phương trình có hai nghiệm phân biệt $x_1$; $x_2$.
		\item Tính giá trị các biểu thức sau: $x_1+x_2$; $x_1x_2$; $x_1^2x_2^2-x_1-x_2$; $x_1^2+x_2^2$; $(x_1-x_2)^2$.
	\end{enumerate}
\loigiai{
	\begin{enumerate}
		\item
		Phương trình $-x^2+8x-7=0$ có $a=-1$; $b=8$; $c=-7$.\\
		Ta có $\Delta=b^2-4ac=(8)^2-4\cdot(-1)\cdot(-7)=36>0$\\
		Nên phương trình có hai nghiệm phân biệt $x_1$; $x_2$.
		\item Theo định lý Viète, ta có
		$x_1+x_2=-\dfrac{b}{a}=-\dfrac{8}{-1}=8$ và
		$x_1x_2=\dfrac{c}{a}=\dfrac{-7}{-1}=7$.
		\begin{itemize}
			\item $x_1^2x_2^2-x_1-x_2=(x_1x_2)^2-(x_1+x_2)=7^2-8=41$.
			\item $x_1^2+x_2^2=(x_1+x_2)^2-2x_1x_2=(8)^2-2\cdot 7=50$.
			\item $(x_1-x_2)^2=(x_1+x_2)^2-4x_1x_2=(8)^2-4\cdot 7=36$.
		\end{itemize}
	\end{enumerate}
}
\end{bt}
\begin{bt}%[Dự án EX-9-Đề Cương Toán 9]%[GVSB: Hoàng Minh Nhân Mã - GVPB1: Phan Minh Huế - GVPB2: Nguyễn Trần Anh Tuấn]%[9D4V3-1]
	Cho phương trình $4x^2-5x-3=0$ có nghiệm là $x_1$; $x_2$. Không giải phương trình, hãy tính giá trị của biểu thức $S=x_1+x_2$; $P=x_1x_2$; $F=(x_1+1)(x_2+1)-(x_1-x_2)^2$.
\loigiai{
	Phương trình $4x^2-5x-3=0$ có $a=4$; $b=-5$; $c=-3$.\\
	Ta có $\Delta=b^2-4ac=(-5)^2-4\cdot 4\cdot (-3)=73>0$ nên phương trình có hai nghiệm phân biệt $x_1$; $x_2$.\\
	Theo định lý Viète, ta có
		$S=x_1+x_2=-\dfrac{b}{a}=-\dfrac{-5}{4}=\dfrac{5}{4}$ và
		$P=x_1x_2=\dfrac{c}{a}=\dfrac{-3}{4}$.
	\begin{eqnarray*}
		F & = & (x_1+1)(x_2+1)-(x_1-x_2)^2\\
		& = & x_1x_2+(x_1+x_2)+1-[(x_1+x_2)^2-4x_1x_2]\\
		& = & \dfrac{-3}{4}+\dfrac{5}{4}+1-\left[\left( \dfrac{5}{4} \right)^2-4\cdot\dfrac{-3}{4}\right]\\
		& = & \dfrac{-49}{16}.
	\end{eqnarray*}
}
\end{bt}
\begin{bt}%[Dự án EX-9-Đề Cương Toán 9]%[GVSB: Hoàng Minh Nhân Mã - GVPB1: Phan Minh Huế - GVPB2: Nguyễn Trần Anh Tuấn]%[9D4V3-1]
	Cho phương trình $2x^2-6x+3=0$ có nghiệm là $x_1$; $x_2$. Không giải phương trình, hãy tính giá trị của biểu thức $H=x_1^2+x_2^2-5x_1-5x_2$.
\loigiai{
	Phương trình $2x^2-6x+3=0$ có $a=2$; $b=-6$; $c=3$.\\
	Ta có $\Delta=b^2-4ac=(-6)^2-4\cdot 2\cdot 3=12>0$ nên phương trình có hai nghiệm phân biệt $x_1$; $x_2$.\\
	Theo định lý Viète, ta có
	$x_1+x_2=-\dfrac{b}{a}=-\dfrac{-6}{2}=3$ và
	$x_1x_2=\dfrac{c}{a}=\dfrac{3}{2}$.
	\begin{eqnarray*}
		H & = & x_1^2+x_2^2-5x_1-5x_2\\
		& = & (x_1+x_2)^2-2x_1x_2-5(x_1+x_2)\\
		& = & 3^2-2\cdot\dfrac{3}{2}-5\cdot 3\\
		& = & -9.
	\end{eqnarray*}
}
\end{bt}
\begin{bt}%[Dự án EX-9-Đề Cương Toán 9]%[GVSB: Hoàng Minh Nhân Mã - GVPB1: Phan Minh Huế - GVPB2: Nguyễn Trần Anh Tuấn]%[9D4V3-1]
	Cho phương trình $2x^2+5x-3=0$ có nghiệm là $x_1$; $x_2$. Không giải phương trình, hãy tính giá trị của biểu thức $Q=(x_1+3x_2)(x_2+3x_1)$.
\loigiai{
	Phương trình $2x^2+5x-3=0$ có $a=2$; $b=5$; $c=-3$.\\
	Ta có $\Delta=b^2-4ac=5^2-4\cdot 2\cdot (-3)=49>0$ nên phương trình có hai nghiệm phân biệt $x_1$; $x_2$.\\
	Theo định lý Viète, ta có
	$x_1+x_2=-\dfrac{b}{a}=-\dfrac{5}{2}$ và
	$x_1x_2=\dfrac{c}{a}=\dfrac{-3}{2}$.
	\begin{eqnarray*}
		Q & = & (x_1+3x_2)(x_2+3x_1)\\
		& = & x_1x_2+3x_1^2+3x_2^2+9x_1x_2\\
		& = & 10x_1x_2+3[(x_1+x_2)^2-2x_1x_2]\\
		& = & 10\cdot\left(-\dfrac{3}{2}\right)+3\left[\left(-\dfrac{5}{2}\right)^2-2\cdot\left(-\dfrac{3}{2}\right)\right]\\
		& = & \dfrac{51}{4}.
	\end{eqnarray*}
}
\end{bt}
\begin{bt}%[Dự án EX-9-Đề Cương Toán 9]%[GVSB: Hoàng Minh Nhân Mã - GVPB1: Phan Minh Huế - GVPB2: Nguyễn Trần Anh Tuấn]%[9D4V3-1]
	Cho phương trình $3x^2-4x-2=0$ có nghiệm là $x_1$; $x_2$. Không giải phương trình, hãy tính giá trị của biểu thức $T=(x_1-x_2)^2$.
\loigiai{
	Phương trình $3x^2-4x-2=0$ có $a=3$; $b=-4$; $c=-2$.\\
	Ta có $\Delta=b^2-4ac=(-4)^2-4\cdot 3\cdot (-2)=40>0$ nên phương trình có hai nghiệm phân biệt $x_1$; $x_2$.\\
	Theo định lý Viète, ta có
	$x_1+x_2=-\dfrac{b}{a}=-\dfrac{-4}{3}=\dfrac{4}{3}$ và
	$x_1x_2=\dfrac{c}{a}=\dfrac{-2}{3}$.\\
	Khi đó, $T=(x_1-x_2)^2=(x_1+x_2)^2-4x_1x_2=\left(\dfrac{4}{3}\right)^2-4\cdot\left(-\dfrac{2}{3}\right)=\dfrac{40}{9}$.
}
\end{bt}
\begin{bt}%[Dự án EX-9-Đề Cương Toán 9]%[GVSB: Hoàng Minh Nhân Mã - GVPB1: Phan Minh Huế - GVPB2: Nguyễn Trần Anh Tuấn]%[9D4V3-1]
	Cho phương trình $2x^2-13x-4=0$ có nghiệm là $x_1$; $x_2$. Không giải phương trình, hãy tính giá trị của biểu thức $G=(x_1+x_2)(x_1+2x_2)-x_2^2$.
\loigiai{
	Phương trình $2x^2-13x-4=0$ có $a=2$; $b=-13$; $c=-4$.\\
	Ta có $\Delta=b^2-4ac=(-13)^2-4\cdot 2\cdot (-4)=201>0$ nên phương trình có hai nghiệm phân biệt $x_1$; $x_2$.\\
	Theo định lý Viète, ta có
	$x_1+x_2=-\dfrac{b}{a}=-\dfrac{-13}{2}=\dfrac{13}{2}$ và
	$x_1x_2=\dfrac{c}{a}=\dfrac{-4}{2}=-2$.
	\begin{eqnarray*}
		G & = & (x_1+x_2)(x_1+2x_2)-x_2^2\\
		& = & x_1^2+3x_1x_2+x_2^2\\
		& = & (x_1+x_2)^2+x_1x_2\\
		& = & \left(\dfrac{13}{2}\right)^2+(-2)\\
		& = & \dfrac{161}{4}.
	\end{eqnarray*}
}
\end{bt}
\begin{bt}%[Dự án EX-9-Đề Cương Toán 9]%[GVSB: Hoàng Minh Nhân Mã - GVPB1: Phan Minh Huế - GVPB2: Nguyễn Trần Anh Tuấn]%[9D4V3-1]
	Cho phương trình $x^2-3x-2=0$ có 2 nghiệm $x_1$; $x_2$. Không giải phương trình hãy tính giá trị của biểu thức $K=\dfrac{x_1-1}{x_2+1}+\dfrac{x_2-1}{x_1+1}$.
\loigiai{
	Phương trình $x^2-3x-2=0$ có $a=1$; $b=-3$; $c=-2$.\\
	Ta có $\Delta=b^2-4ac=(-3)^2-4\cdot 1\cdot (-2)=17>0$ nên phương trình có hai nghiệm phân biệt $x_1$; $x_2$.\\
	Theo định lý Viète, ta có
	$x_1+x_2=-\dfrac{b}{a}=-\dfrac{-3}{1}=3$ và
	$x_1x_2=\dfrac{c}{a}=\dfrac{-2}{1}=-2$.
	\begin{eqnarray*}
		K & = & \dfrac{x_1-1}{x_2+1}+\dfrac{x_2-1}{x_1+1}\\
		& = & \dfrac{(x_1-1)(x_1+1)+(x_2-1)(x_2+1)}{(x_2+1)(x_1+1)}\\
		& = & \dfrac{x_1^2+x_2^2-2}{x_1x_2+(x_1+x_2)+1}\\
		& = & \dfrac{(x_1+x_2)^2-2x_1x_2-2}{x_1x_2+(x_1+x_2)+1}\\
		& = & \dfrac{3^2-2\cdot(-2)-2}{-2+3+1}\\
		& = & \dfrac{11}{2}.
	\end{eqnarray*}
}
\end{bt}
\begin{bt}%[Dự án EX-9-Đề Cương Toán 9]%[GVSB: Hoàng Minh Nhân Mã - GVPB1: Phan Minh Huế - GVPB2: Nguyễn Trần Anh Tuấn]%[9D4V3-1]
	Tính nhẩm nghiệm của các phương trình
	\begin{multicols}{2}
		\begin{enumerate}
			\item $2x^2+5x-7=0$;
			\item $3x^2+7x-10=0$;
			\item $4x^2-5x-9=0$;
			\item $-2x^2-11x-9=0$;
			\item $7x^2-5x-12=0$;
			\item $-3x^2+7x-4=0$;
			\item $10x^2-7x-17=0$;
			\item $13x^2-2x-11=0$;
			\item $\dfrac{7}{2}x^2+\dfrac{3}{2}x-5=0$;
			\item $\dfrac{8}{3}x^2+6x+\dfrac{10}{3}=0$;
			\item $\dfrac{9}{4}x^2-4x+\dfrac{7}{4}=0$;
			\item $3{,}5x^2+8{,}4x+4{,}9=0$;
			\item $-2{,}3x^2-4{,}7x+7=0$;
			\item $415x^2-27x-388=0$;
			\item $-671x^2-45x+626=0$;
			\item $x^2-(1+\sqrt{3})x+\sqrt{3}=0$;
			\item $x^2-(1+\sqrt{2})x+\sqrt{2}=0$;
			\item $x^2-\sqrt{5}=(1-\sqrt{5})x$;
			\item $x^2-(\sqrt{6}-1)x-\sqrt{6}=0$;
			\item $\sqrt{3}x^2-(1-\sqrt{3})x-1=0$;
			\item $-x^2+(3+\sqrt{11})x+4+\sqrt{11}=0$;
			\item $(3+\sqrt{2})x^2-(1+\sqrt{2})x-2=0$;
			\item $3x^2-x\sqrt{3}+\sqrt{3}-3=0$;
			\item $(\sqrt{3}-1)x^2+(\sqrt{3}-2)x+3-2\sqrt{3}=0$;
			\item $(1-\sqrt{2})x^2-2x-2\sqrt{2}x=-1-3\sqrt{2}$.
		\end{enumerate}
	\end{multicols}
	\loigiai{
		\begin{enumerate}
			%Câu 1
			\item $2x^2+5x-7=0$ với $a=2$; $b=5$; $c=-7$.\\
			Ta có $a+b+c=2+5+(-7)=0$.\\ 
			Nên phương trình có hai nghiệm là $x_1=1$; $x_2=\dfrac{c}{a}=\dfrac{-7}{2}$.
			%Câu 2
			\item $3x^2+7x-10=0$ với $a=3$; $b=7$; $c=-10$.\\
			Ta có $a+b+c=3+7+(-10)=0$.\\
			Nên phương trình có hai nghiệm là $x_1=1$; $x_2=\dfrac{c}{a}=\dfrac{-10}{3}$.
			%Câu 3
			\item $4x^2-5x-9=0$ với $a=4$; $b=-5$; $c=-9$.\\
			Ta có $a-b+c=4-(-5)+(-9)=4+5-9=0$.\\
			Nên phương trình có hai nghiệm là $x_1=-1$; $x_2=-\dfrac{c}{a}=-\dfrac{-9}{4}=\dfrac{9}{4}$.
			%Câu 4
			\item $-2x^2-11x-9=0$ với $a=-2$; $b=-11$; $c=-9$.\\
			Ta có $a-b+c=(-2)-(-11)+(-9)=-2+11-9=0$.\\
			Nên phương trình có hai nghiệm là $x_1=-1$; $x_2=-\dfrac{c}{a}=-\dfrac{-9}{-2}=-\dfrac{9}{2}$.
			%Câu 5
			\item $7x^2-5x-12=0$ với $a=7$; $b=-5$; $c=-12$.\\
			Ta có $a-b+c=7-(-5)+(-12)=7+5-12=0$.\\
			Nên phương trình có hai nghiệm là $x_1=-1$; $x_2=-\dfrac{c}{a}=-\dfrac{-12}{7}=\dfrac{12}{7}$.
			%Câu 6
			\item $-3x^2+7x-4=0$ với $a=-3$; $b=7$; $c=-4$.\\
			Ta có $a+b+c=(-3)+7+(-4)=0$.\\
			Nên phương trình có hai nghiệm là $x_1=1$; $x_2=\dfrac{c}{a}=\dfrac{-4}{-3}=\dfrac{4}{3}$.
			%Câu 7
			\item $10x^2-7x-17=0$ với $a=10$; $b=-7$; $c=-17$.\\
			Ta có $a-b+c=10-(-7)+(-17)=10+7-17=0$.\\
			Nên phương trình có hai nghiệm là $x_1=-1$; $x_2=-\dfrac{c}{a}=-\dfrac{-17}{10}=\dfrac{17}{10}$.
			%Câu 8
			\item $13x^2-2x-11=0$ với $a=13$; $b=-2$; $c=-11$.\\
			Ta có $a+b+c=13+(-2)+(-11)=0$.\\
			Nên phương trình có hai nghiệm là $x_1=1$; $x_2=\dfrac{c}{a}=\dfrac{-11}{13}$.
			%Câu 9
			\item $\dfrac{7}{2}x^2+\dfrac{3}{2}x-5=0$ với $a=\dfrac{7}{2}$; $b=\dfrac{3}{2}$; $c=-5$.\\
			Ta có $a+b+c=\dfrac{7}{2}+\dfrac{3}{2}+(-5)=0$.\\
			Nên phương trình có hai nghiệm là $x_1=1$; $x_2=\dfrac{c}{a}=-5\colon\dfrac{7}{2}=-\dfrac{10}{7}$.
			%Câu 10
			\item $\dfrac{8}{3}x^2+6x+\dfrac{10}{3}=0$ với $a=\dfrac{8}{3}$; $b=6$; $c=\dfrac{10}{3}$.\\
			Ta có $a-b+c=\dfrac{8}{3}-6+\dfrac{10}{3}=0$.\\
			Nên phương trình có hai nghiệm là $x_1=-1$; $x_2=-\dfrac{c}{a}=-\dfrac{10}{3}\colon\dfrac{8}{3}=-\dfrac{5}{4}$.
			%Câu 11
			\item $\dfrac{9}{4}x^2-4x+\dfrac{7}{4}=0$ với $a=\dfrac{9}{4}$; $b=-4$; $c=\dfrac{7}{4}$.\\
			Ta có $a+b+c=\dfrac{9}{4}+(-4)+\dfrac{7}{4}==0$.\\
			Nên phương trình có hai nghiệm là $x_1=1$; $x_2=\dfrac{c}{a}=\dfrac{7}{4}\colon\dfrac{9}{4}=\dfrac{7}{9}$.
			%Câu 12
			\item $3{,}5x^2+8{,}4x+4{,}9=0$ với $a=3{,}5$; $b=8{,}4$; $c=-4{,}9$.\\
			Ta có $a+b+c=3{,}5-8{,}4+4{,}9=0$.\\
			Nên phương trình có hai nghiệm là  $x_1=-1$; $x_2=-\dfrac{c}{a}=-\dfrac{4{,}9}{3{,}5}=-\dfrac{7}{5}$.
			%Câu 13
			\item $-2{,}3x^2-4{,}7x+7=0$ với $a=-2{,}3$; $b=-4{,}7$; $c=7$.\\
			Ta có $a+b+c=(-2{,}3)+(-4{,}7)+7=0$.\\
			Nên phương trình có hai nghiệm là $x_1=1$; $x_2=\dfrac{c}{a}=\dfrac{7}{-2{,}3}=-\dfrac{70}{23}$.
			%Câu 14
			\item $415x^2-27x-388=0$ với $a=415$; $b=-27$; $c=-388$.\\
			Ta có $a+b+c=415+(-27)+(-388)=0$.\\
			Nên phương trình có hai nghiệm là $x_1=1$; $x_2=\dfrac{c}{a}=\dfrac{-388}{415}$.
			%Câu 15
			\item $-671x^2-45x+626=0$ với $a=-671$; $b=-45$; $c=626$.\\
			Ta có $a-b+c=(-671)-(-45)+626=-671+45+626=0$.\\
			Nên phương trình có hai nghiệm là $x_1=-1$; $x_2=-\dfrac{c}{a}=-\dfrac{626}{-671}=\dfrac{626}{671}$.
			%Câu 16
			\item $x^2-(1+\sqrt{3})x+\sqrt{3}=0$ với $a=1$; $b=-(1+\sqrt{3})$; $c=\sqrt{3}$.\\
			Ta có $a+b+c=1-(1+\sqrt{3})+\sqrt{3}=0$.\\
			Nên phương trình có hai nghiệm là $x_1=1$; $x_2=\dfrac{c}{a}=\dfrac{\sqrt{3}}{1}=\sqrt{3}$.
			%Câu 17
			\item $x^2-(1+\sqrt{2})x+\sqrt{2}=0$ với $a=1$; $b=-(1+\sqrt{2})$; $c=\sqrt{2}$.\\
			Ta có $a+b+c=1-(1+\sqrt{2})+\sqrt{2}=0$.\\
			Nên phương trình có hai nghiệm là $x_1=1$; $x_2=\dfrac{c}{a}=\dfrac{\sqrt{2}}{1}=\sqrt{2}$.
			%Câu 18
			\item $x^2-\sqrt{5}=(1-\sqrt{5})x$.\\
			Chuyển về dạng tổng quát: $x^2-(1-\sqrt{5})x-\sqrt{5}=0$ với $a=1$; $b=-(1-\sqrt{5})$; $c=-\sqrt{5}$.\\
			Ta có $a+b+c=1-(1-\sqrt{5})+(-\sqrt{5})=0$.\\
			Nên phương trình có hai nghiệm là $x_1=1$; $x_2=\dfrac{c}{a}=\dfrac{-\sqrt{5}}{1}=-\sqrt{5}$.
			%Câu 19
			\item $x^2-(\sqrt{6}-1)x-\sqrt{6}=0$ với $a=1$; $b=-(\sqrt{6}-1)$; $c=-\sqrt{6}$.\\
			Ta có $a-b+c=1-(\sqrt{6}-1)+(-\sqrt{6})=0$.\\
			Nên phương trình có hai nghiệm là $x_1=-1$; $x_2=-\dfrac{c}{a}=-\dfrac{-\sqrt{6}}{1}=\sqrt{6}$.
			%Câu 20
			\item $\sqrt{3}x^2-(1-\sqrt{3})x-1=0$ với $a=\sqrt{3}$; $b=-(1-\sqrt{3})$; $c=-1$.\\
			Ta có $a-b+c=\sqrt{3}-(1-\sqrt{3})+(-1)=\sqrt{3}+1-\sqrt{3}-1=0$.\\
			Nên phương trình có hai nghiệm là $x_1=-1$; $x_2=-\dfrac{c}{a}=-\dfrac{-1}{\sqrt{3}}=\dfrac{1}{\sqrt{3}}=\dfrac{\sqrt{3}}{3}$.
			%Câu 21
			\item $-x^2+(3+\sqrt{11})x+4+\sqrt{11}=0$ với $a=-1$; $b=3+\sqrt{11}$; $c=4+\sqrt{11}$.\\
			Ta có $a-b+c=(-1)-(3+\sqrt{11})+(4+\sqrt{11})=0$.\\
			Nên phương trình có hai nghiệm là $x_1=-1$; $x_2=-\dfrac{c}{a}=-\dfrac{4+\sqrt{11}}{-1}=4+\sqrt{11}$.
			%Câu 22
			\item $(3+\sqrt{2})x^2-(1+\sqrt{2})x-2=0$ với $a=3+\sqrt{2}$; $b=-(1+\sqrt{2})$; $c=-2$.\\
			Ta có $a+b+c=(3+\sqrt{2})-(1+\sqrt{2})-2=0$.\\
			Nên phương trình có hai nghiệm là $x_1=1$; $x_2=\dfrac{c}{a}=\dfrac{-2}{3+\sqrt{2}}=\dfrac{-6+2\sqrt{2}}{7}$.
			%Câu 23
			\item $3x^2-x\sqrt{3}+\sqrt{3}-3=0$ với $a=3$; $b=-\sqrt{3}$; $c=\sqrt{3}-3$.\\
			Ta có $a+b+c=3-\sqrt{3}+\sqrt{3}-3=0$.\\
			Nên phương trình có hai nghiệm là $x_1=1$; $x_2=\dfrac{c}{a}=\dfrac{\sqrt{3}-3}{3}$.
			%Câu 24
			\item $(\sqrt{3}-1)x^2+(\sqrt{3}-2)x+3-2\sqrt{3}=0$ với $a=\sqrt{3}-1$; $b=\sqrt{3}-2$; $c=3-2\sqrt{3}$.\\
			Ta có $a+b+c=(\sqrt{3}-1)+(\sqrt{3}-2)+(3-2\sqrt{3})=0$.\\
			Nên phương trình có hai nghiệm là $x_1=1$; $x_2=\dfrac{c}{a}=\dfrac{3-2\sqrt{3}}{\sqrt{3}-1}=\dfrac{\sqrt{3}-3}{2}$.
			%Câu 25
			\item $(1-\sqrt{2})x^2-2x-2\sqrt{2}x=-1-3\sqrt{2}$.\\
			Nghĩa là: $(1-\sqrt{2})x^2+(-2-2\sqrt{2})x+(1+3\sqrt{2})=0$ với $a=1-\sqrt{2}$; $b=-(2+2\sqrt{2})$; $c=1+3\sqrt{2}$.\\
			Ta có $a+b+c=(1-\sqrt{2})-(2+2\sqrt{2})+(1+3\sqrt{2})=0$.\\
			Nên phương trình có hai nghiệm là $x_1=1$; $x_2=\dfrac{c}{a}=\dfrac{1+3\sqrt{2}}{1-\sqrt{2}}=-7-4\sqrt{2}$.
		\end{enumerate}
	}
\end{bt}
\subsection{Tìm hai số khi biết tổng và tích của chúng}
\subsubsection{Kiến thức trọng tâm}
\begin{tomtat}
	Nếu hai số có \textit{tổng} bằng $S$ và \textit{tích} bằng $P$ thì hai số đó là nghiệm của phương trình:
	$$ x^2-Sx+P=0.$$
	Điều kiện để có hai số đó là $S^2-4P\ge0$.
\end{tomtat}
\begin{vd}%[Dự án EX-9-Đề Cương Toán 9]%[GVSB: Hoàng Minh Nhân Mã - GVPB1: Phan Minh Huế - GVPB2: Nguyễn Trần Anh Tuấn]%[9D4H3-2]
	Tìm hai số (nếu có) trong mỗi trường hợp sau
	\begin{enumerate}
			\item Tổng của chúng bằng $23$ và tích của chúng bằng $120$.
			\item Tổng của chúng bằng $10$ và tích của chúng bằng $30$.
	\end{enumerate}
	\loigiai{
		\begin{enumerate}
			\item Ta xét $S^2-4P=23^2-4\cdot 120=49\ge0$.\\
			Nên hai số cần tìm là nghiệm của phương trình $x^2-23x+120=0$.\\
			Ta có $\Delta=(-23)^2-4\cdot 1\cdot 120=49>0$.
			Nên phương trình có hai nghiệm phân biệt
			$$ x_1=\dfrac{-b+\sqrt{\Delta}}{2a}=\dfrac{-(-23)+\sqrt{49}}{2\cdot 1}=15;\quad x_2=\dfrac{-b-\sqrt{\Delta}}{2a}=\dfrac{-(-23)-\sqrt{49}}{2\cdot 1}=8. $$
			Vậy hai số cần tìm là $15$ và $8$.
			\item Ta xét $S^2-4P=10^2-4\cdot 30=-20<0$.\\
			Vậy không có hai số thỏa mãn điều kiện đã cho.
		\end{enumerate}
	}
\end{vd}
\subsubsection{Bài tập}
\begin{bt}%[Dự án EX-9-Đề Cương Toán 9]%[GVSB: Hoàng Minh Nhân Mã - GVPB1: Phan Minh Huế - GVPB2: Nguyễn Trần Anh Tuấn]%[9D4H3-2]
	Cho $S=x_1+x_2$ và $P=x_1x_2$. Tìm $x_1$; $x_2$ biết
	\begin{multicols}{3}
	\begin{enumerate}
		\item $S=3$, $P=2$;
		\item $S=1$, $P=-2$;
		\item $S=5$, $P=6$;
		\item $S=1$, $P=-6$;
		\item $S=7$, $P=12$;
		\item $S=-1$, $P=-12$;
		\item $S=-2$, $P=-8$;
		\item $S=5$, $P=4$;
		\item $S=-3$, $P=-4$;
		\item $S=5$, $P=-6$;
		\item $S=-2$, $P=-15$;
		\item $S=3$, $P=-10$.
	\end{enumerate}
	\end{multicols}
\loigiai{
	\begin{enumerate}
	\item Ta xét $S^2-4P=3^2-4\cdot 2=1\ge0$.\\
	Nên hai số $x_1$; $x_2$ cần tìm là nghiệm của phương trình $x^2-3x+2=0$.\\
	Ta có $\Delta=(-3)^2-4\cdot 1\cdot 2=1>0$.\\
	Nên phương trình có hai nghiệm phân biệt
	$$ x_1=\dfrac{-(-3)+\sqrt{1}}{2\cdot 1}=2;\quad x_2=\dfrac{-(-3)-\sqrt{1}}{2\cdot 1}=1. $$
	Vậy $x_1=2$ và $x_2=1$ (hoặc ngược lại).
	\item Ta xét $S^2-4P=1^2-4\cdot (-2)=9\ge0$.\\
	Nên hai số $x_1$; $x_2$ cần tìm là nghiệm của phương trình $x^2-x-2=0$.\\
	Ta có $\Delta=(-1)^2-4\cdot 1\cdot (-2)=9>0$.\\
	Nên phương trình có hai nghiệm phân biệt
	$$ x_1=\dfrac{-(-1)+\sqrt{9}}{2\cdot 1}=2;\quad x_2=\dfrac{-(-1)-\sqrt{9}}{2\cdot 1}=-1. $$
	Vậy $x_1=2$ và $x_2=-1$ (hoặc ngược lại).
	\item Ta xét $S^2-4P=5^2-4\cdot 6=1\ge0$.\\
	Nên hai số $x_1$; $x_2$ cần tìm là nghiệm của phương trình $x^2-5x+6=0$.\\
	Ta có $\Delta=(-5)^2-4\cdot 1\cdot 6=1>0$.\\
	Nên phương trình có hai nghiệm phân biệt
	$$ x_1=\dfrac{-(-5)+\sqrt{1}}{2\cdot 1}=3;\quad x_2=\dfrac{-(-5)-\sqrt{1}}{2\cdot 1}=2. $$
	Vậy $x_1=3$ và $x_2=2$ (hoặc ngược lại).
	\item Ta xét $S^2-4P=1^2-4\cdot (-6)=25\ge0$.\\
	Nên hai số $x_1$; $x_2$ cần tìm là nghiệm của phương trình $x^2-x-6=0$.\\
	Ta có $\Delta=(-1)^2-4\cdot 1\cdot (-6)=25>0$.\\
	Nên phương trình có hai nghiệm phân biệt
	$$ x_1=\dfrac{-(-1)+\sqrt{25}}{2\cdot 1}=3;\quad x_2=\dfrac{-(-1)-\sqrt{25}}{2\cdot 1}=-2. $$
	Vậy $x_1=3$ và $x_2=-2$ (hoặc ngược lại).
	\item Ta xét $S^2-4P=7^2-4\cdot 12=1\ge0$.\\
	Nên hai số $x_1$; $x_2$ cần tìm là nghiệm của phương trình $x^2-7x+12=0$.\\
	Ta có $\Delta=(-7)^2-4\cdot 1\cdot 12=1>0$.\\
	Nên phương trình có hai nghiệm phân biệt
	$$ x_1=\dfrac{-(-7)+\sqrt{1}}{2\cdot 1}=4;\quad x_2=\dfrac{-(-7)-\sqrt{1}}{2\cdot 1}=3. $$
	Vậy $x_1=4$ và $x_2=3$ (hoặc ngược lại).
	\item Ta xét $S^2-4P=(-1)^2-4\cdot (-12)=49\ge0$.\\
	Nên hai số $x_1$; $x_2$ cần tìm là nghiệm của phương trình $x^2+x-12=0$.\\
	Ta có $\Delta=1^2-4\cdot 1\cdot (-12)=49>0$.\\
	Nên phương trình có hai nghiệm phân biệt
	$$ x_1=\dfrac{-1+\sqrt{49}}{2\cdot 1}=3;\quad x_2=\dfrac{-1-\sqrt{49}}{2\cdot 1}=-4. $$
	Vậy $x_1=3$ và $x_2=-4$ (hoặc ngược lại).
	\item Ta xét $S^2-4P=(-2)^2-4\cdot (-8)=36\ge0$.\\
	Nên hai số $x_1$; $x_2$ cần tìm là nghiệm của phương trình $x^2+2x-8=0$.\\
	Ta có $\Delta=2^2-4\cdot 1\cdot (-8)=36>0$.\\
	Nên phương trình có hai nghiệm phân biệt
	$$ x_1=\dfrac{-2+\sqrt{36}}{2\cdot 1}=2;\quad x_2=\dfrac{-2-\sqrt{36}}{2\cdot 1}=-4. $$
	Vậy $x_1=2$ và $x_2=-4$ (hoặc ngược lại).
	\item Ta xét $S^2-4P=5^2-4\cdot 4=9\ge0$.\\
	Nên hai số $x_1$; $x_2$ cần tìm là nghiệm của phương trình $x^2-5x+4=0$.\\
	Ta có $\Delta=(-5)^2-4\cdot 1\cdot 4=9>0$.\\
	Nên phương trình có hai nghiệm phân biệt
	$$ x_1=\dfrac{-(-5)+\sqrt{9}}{2\cdot 1}=4;\quad x_2=\dfrac{-(-5)-\sqrt{9}}{2\cdot 1}=1. $$
	Vậy $x_1=4$ và $x_2=1$ (hoặc ngược lại).
	\item Ta xét $S^2-4P=(-3)^2-4\cdot (-4)=25\ge0$.\\
	Nên hai số $x_1$; $x_2$ cần tìm là nghiệm của phương trình $x^2+3x-4=0$.\\
	Ta có $\Delta=3^2-4\cdot 1\cdot (-4)=25>0$.\\
	Nên phương trình có hai nghiệm phân biệt
	$$ x_1=\dfrac{-3+\sqrt{25}}{2\cdot 1}=1;\quad x_2=\dfrac{-3-\sqrt{25}}{2\cdot 1}=-4. $$
	Vậy $x_1=1$ và $x_2=-4$ (hoặc ngược lại).
	\item Ta xét $S^2-4P=5^2-4\cdot (-6)=49\ge0$.\\
	Nên hai số $x_1$; $x_2$ cần tìm là nghiệm của phương trình $x^2-5x-6=0$.\\
	Ta có $\Delta=(-5)^2-4\cdot 1\cdot (-6)=49>0$.\\
	Nên phương trình có hai nghiệm phân biệt
	$$ x_1=\dfrac{-(-5)+\sqrt{49}}{2\cdot 1}=6;\quad x_2=\dfrac{-(-5)-\sqrt{49}}{2\cdot 1}=-1. $$
	Vậy $x_1=6$ và $x_2=-1$ (hoặc ngược lại).
	\item Ta xét $S^2-4P=(-2)^2-4\cdot (-15)=64\ge0$.\\
	Nên hai số $x_1$; $x_2$ cần tìm là nghiệm của phương trình $x^2+2x-15=0$.\\
	Ta có $\Delta=2^2-4\cdot 1\cdot (-15)=64>0$.\\
	Nên phương trình có hai nghiệm phân biệt
	$$ x_1=\dfrac{-2+\sqrt{64}}{2\cdot 1}=3;\quad x_2=\dfrac{-2-\sqrt{64}}{2\cdot 1}=-5. $$
	Vậy $x_1=3$ và $x_2=-5$ (hoặc ngược lại).
	\item Ta xét $S^2-4P=3^2-4\cdot (-10)=49\ge0$.\\
	Nên hai số $x_1$; $x_2$ cần tìm là nghiệm của phương trình $x^2-3x-10=0$.\\
	Ta có $\Delta=(-3)^2-4\cdot 1\cdot (-10)=49>0$.\\
	Nên phương trình có hai nghiệm phân biệt
	$$ x_1=\dfrac{-(-3)+\sqrt{49}}{2\cdot 1}=5;\quad x_2=\dfrac{-(-3)-\sqrt{49}}{2\cdot 1}=-2. $$
	Vậy $x_1=5$ và $x_2=-2$ (hoặc ngược lại).
	\end{enumerate}
}
\end{bt}
\begin{bt}%[Dự án EX-9-Đề Cương Toán 9]%[GVSB: Hoàng Minh Nhân Mã - GVPB1: Phan Minh Huế - GVPB2: Nguyễn Trần Anh Tuấn]%[9D4H3-2]
	\begin{enumerate}
		\item Tìm hai số, biết tổng của chúng bằng $-11$ và tích của chúng bằng $24$.
		\item Tìm hai số, biết tổng của chúng bằng $-5$ và tích của chúng bằng $-14$.
		\item Tìm hai số, biết tổng của chúng bằng $8$ và tích của chúng bằng $-48$.
		\item Tìm hai số, biết tổng của chúng bằng $-16$ và tích của chúng bằng $63$.
		\item Tìm hai số, biết tổng của chúng bằng $-8$ và tích của chúng bằng $-65$.
		\item Tìm hai số, biết tổng của chúng bằng $7$ và tích của chúng bằng $-98$.
		\item Tìm hai số, biết tổng của chúng bằng $-15$ và tích của chúng bằng $50$.
		\item Tìm hai số, biết tổng của chúng bằng $13$ và tích của chúng bằng $-48$.
	\end{enumerate}
\loigiai{
	\begin{enumerate}
	\item Gọi hai số cần tìm là $x_1$; $x_2$. Khi đó $S=x_1+x_2=-11$ và $P=x_1x_2=24$.\\
	Ta xét $S^2-4P=(-11)^2-4\cdot 24=25\ge0$.\\
	Nên hai số cần tìm là nghiệm của phương trình $x^2+11x+24=0$.\\
	Ta có $\Delta=11^2-4\cdot 1\cdot 24=25>0$.\\
	Nên phương trình có hai nghiệm phân biệt
	$$ x_1=\dfrac{-11+\sqrt{25}}{2\cdot 1}=-3;\quad x_2=\dfrac{-11-\sqrt{25}}{2\cdot 1}=-8. $$
	Vậy hai số cần tìm là $-3$ và $-8$.
	\item Gọi hai số cần tìm là $x_1$; $x_2$. Khi đó $S=x_1+x_2=-5$ và $P=x_1x_2=-14$.\\
	Ta xét $S^2-4P=(-5)^2-4\cdot (-14)=81\ge0$.\\
	Nên hai số cần tìm là nghiệm của phương trình $x^2+5x-14=0$.\\
	Ta có $\Delta=5^2-4\cdot 1\cdot (-14)=81>0$.\\
	Nên phương trình có hai nghiệm phân biệt
	$$ x_1=\dfrac{-5+\sqrt{81}}{2\cdot 1}=2;\quad x_2=\dfrac{-5-\sqrt{81}}{2\cdot 1}=-7. $$
	Vậy hai số cần tìm là $2$ và $-7$.
	\item Ta xét $S^2-4P=8^2-4\cdot (-48)=256\ge0$.\\
	Nên hai số cần tìm là nghiệm của phương trình $x^2-8x-48=0$.\\
	Ta có $\Delta=(-8)^2-4\cdot 1\cdot (-48)=256>0$.\\
	Nên phương trình có hai nghiệm phân biệt
	$$ x_1=\dfrac{-(-8)+\sqrt{256}}{2\cdot 1}=12;\quad x_2=\dfrac{-(-8)-\sqrt{256}}{2\cdot 1}=-4. $$
	Vậy hai số cần tìm là $12$ và $-4$.
	\item Gọi hai số cần tìm là $x_1$; $x_2$. Khi đó $S=x_1+x_2=16$ và $P=x_1x_2=63$.\\
	Ta xét $S^2-4P=16^2-4\cdot 63=4\ge0$.\\
	Nên hai số cần tìm là nghiệm của phương trình $x^2-16x+63=0$.\\
	Ta có $\Delta=(-16)^2-4\cdot 1\cdot 63=4>0$.\\
	Nên phương trình có hai nghiệm phân biệt
	$$ x_1=\dfrac{-(-16)+\sqrt{4}}{2\cdot 1}=9;\quad x_2=\dfrac{-(-16)-\sqrt{4}}{2\cdot 1}=7. $$
	Vậy hai số cần tìm là $9$ và $7$.
	\item Gọi hai số cần tìm là $x_1$; $x_2$. Khi đó $S=x_1+x_2=-8$ và $P=x_1x_2=-65$.\\
	Ta xét $S^2-4P=(-8)^2-4\cdot (-65)=324\ge0$.\\
	Nên hai số cần tìm là nghiệm của phương trình $x^2+8x-65=0$.\\
	Ta có $\Delta=8^2-4\cdot 1\cdot (-65)=324>0$.\\
	Nên phương trình có hai nghiệm phân biệt
	$$ x_1=\dfrac{-8+\sqrt{324}}{2\cdot 1}=5;\quad x_2=\dfrac{-8-\sqrt{324}}{2\cdot 1}=-13. $$
	Vậy hai số cần tìm là $5$ và $-13$.
	\item Gọi hai số cần tìm là $x_1$; $x_2$. Khi đó $S=x_1+x_2=7$ và $P=x_1x_2=-98$.\\
	Ta xét $S^2-4P=7^2-4\cdot (-98)=49+392=441\ge0$.\\
	Nên hai số cần tìm là nghiệm của phương trình $x^2-7x-98=0$.\\
	Ta có $\Delta=(-7)^2-4\cdot 1\cdot (-98)=49+392=441>0$.\\
	Nên phương trình có hai nghiệm phân biệt
	$$ x_1=\dfrac{-(-7)+\sqrt{441}}{2\cdot 1}=14;\quad x_2=\dfrac{-(-7)-\sqrt{441}}{2\cdot 1}=-7. $$
	Vậy hai số cần tìm là $14$ và $-7$.
	\item Gọi hai số cần tìm là $x_1$; $x_2$. Khi đó $S=x_1+x_2=-15$ và $P=x_1x_2=50$.\\
	Ta xét $S^2-4P=(-15)^2-4\cdot 50=25\ge0$.\\
	Nên hai số cần tìm là nghiệm của phương trình $x^2+15x+50=0$.\\
	Ta có $\Delta=15^2-4\cdot 1\cdot 50=25>0$.\\
	Nên phương trình có hai nghiệm phân biệt
	$$ x_1=\dfrac{-15+\sqrt{25}}{2\cdot 1}=-5;\quad x_2=\dfrac{-15-\sqrt{25}}{2\cdot 1}=-10. $$
	Vậy hai số cần tìm là $-5$ và $-10$.
	\item Gọi hai số cần tìm là $x_1$; $x_2$. Khi đó $S=x_1+x_2=13$ và $P=x_1x_2=-48$.\\
	Ta xét $S^2-4P=13^2-4\cdot (-48)=361\ge0$.\\
	Nên hai số cần tìm là nghiệm của phương trình $x^2-13x-48=0$.\\
	Ta có $\Delta=(-13)^2-4\cdot 1\cdot (-48)=361>0$.\\
	Nên phương trình có hai nghiệm phân biệt
	$$ x_1=\dfrac{-(-13)+\sqrt{361}}{2\cdot 1}=16;\quad x_2=\dfrac{-(-13)-\sqrt{361}}{2\cdot 1}=-3. $$
	Vậy hai số cần tìm là $16$ và $-3$.
	\end{enumerate}
}
\end{bt}
\begin{bt}%[Dự án EX-9-Đề Cương Toán 9]%[GVSB: Hoàng Minh Nhân Mã - GVPB1: Phan Minh Huế - GVPB2: Nguyễn Trần Anh Tuấn]%[9D4H3-2]
	Tìm hai số $u$ và $v$ (nếu có) trong mỗi trường hợp sau
	\begin{multicols}{3}
	\begin{enumerate}
		\item $u+v=-8$, $uv=12$.
		\item $u+v=-19$, $uv=-42$.
		\item $u+v=10$, $uv=-24$.
		\item $u+v=-7$, $uv=-18$.
		\item $u+v=11$, $uv=33$.
		\item $u+v=1$, $uv=-56$.
	\end{enumerate}
	\end{multicols}
\loigiai{
	\begin{enumerate}
	\item Ta xét $\Delta=(u+v)^2-4uv=(-8)^2-4\cdot 12=64-48=16\ge0$.\\
		Hai số $u$; $v$ là nghiệm của phương trình $x^2+8x+12=0$.\\
		Ta có $\Delta=8^2-4\cdot 1\cdot 12=16>0$. Nên phương trình có hai nghiệm phân biệt
		$$ x_1=\dfrac{-8+\sqrt{16}}{2\cdot 1}=-2;\quad x_2=\dfrac{-8-\sqrt{16}}{2\cdot 1}=-6. $$
		Vậy $u=-2$ và $v=-6$ (hoặc ngược lại).
	\item Ta xét $\Delta=(u+v)^2-4uv=(-19)^2-4\cdot (-42)=529\ge0$.\\
		Hai số $u$; $v$ là nghiệm của phương trình $x^2+19x-42=0$.\\
		Ta có $\Delta=19^2-4\cdot 1\cdot (-42)=529>0$. Nên phương trình có hai nghiệm phân biệt
		$$ x_1=\dfrac{-19+\sqrt{529}}{2\cdot 1}=2;\quad x_2=\dfrac{-19-\sqrt{529}}{2\cdot 1}=-21. $$
		Vậy $u=2$ và $v=-21$ (hoặc ngược lại).
	\item Ta xét $\Delta=(u+v)^2-4uv=10^2-4\cdot (-24)=196\ge0$.\\
		Hai số $u$; $v$ là nghiệm của phương trình $x^2-10x-24=0$.\\
		Ta có $\Delta=(-10)^2-4\cdot 1\cdot (-24)=196>0$. Nên phương trình có hai nghiệm phân biệt
		$$ x_1=\dfrac{-(-10)+\sqrt{196}}{2\cdot 1}=12;\quad 	x_2=\dfrac{-(-10)-\sqrt{196}}{2\cdot 1}=-2. $$
		Vậy $u=12$ và $v=-2$ (hoặc ngược lại).
	\item Ta xét $\Delta=(u+v)^2-4uv=(-7)^2-4\cdot (-18)=121\ge0$.\\
		Hai số $u$; $v$ là nghiệm của phương trình $x^2+7x-18=0$.\\
		Ta có $\Delta=7^2-4\cdot 1\cdot (-18)=121>0$. Nên phương trình có hai nghiệm phân biệt
		$$ x_1=\dfrac{-7+\sqrt{121}}{2\cdot 1}=2;\quad x_2=\dfrac{-7-\sqrt{121}}{2\cdot 1}=-9. $$
		Vậy $u=2$ và $v=-9$ (hoặc ngược lại).
	\item Ta xét $\Delta=(u+v)^2-4uv=(-11)^2-4\cdot 33=121-132=-11<0$.\\
		Vậy không tồn tại hai số $u$ và $v$ thỏa mãn điều kiện đã cho.
	\item Ta xét $\Delta=(u+v)^2-4uv=1^2-4\cdot (-56)=225\ge0$.\\
		Hai số $u$; $v$ là nghiệm của phương trình $x^2-x-56=0$.\\
		Ta có $\Delta=(-1)^2-4\cdot 1\cdot (-56)=225>0$. Nên phương trình có hai nghiệm phân biệt
		$$ x_1=\dfrac{-(-1)+\sqrt{225}}{2\cdot 1}=8;\quad x_2=\dfrac{-(-1)-\sqrt{225}}{2\cdot 1}=-7. $$
		Vậy $u=8$ và $v=-7$ (hoặc ngược lại).
	\end{enumerate}
}
\end{bt}
\begin{bt}%[Dự án EX-9-Đề Cương Toán 9]%[GVSB: Hoàng Minh Nhân Mã - GVPB1: Phan Minh Huế - GVPB2: Nguyễn Trần Anh Tuấn]%[9D4V3-2]
	\begin{enumerate}
		\item Có tồn tại hai số $a$ và $b$ có tổng bằng $5$ và tích bằng $7$ hay không?
		\item Có tồn tại hai số $a$ và $b$ có tổng bằng $-8$ và tích bằng $-84$ không?
		\item Có tồn tại hai số $a$ và $b$ có tổng bằng $-7$ và tích bằng $15$ không?
		\item Có tồn tại hai số $a$ và $b$ có tổng bằng $-9$ và tích bằng $18$ không?
	\end{enumerate}
\loigiai{
	\begin{enumerate}
	\item Theo đề bài, ta có $a+b=5$ và $ab=7$.\\
	Ta xét $\Delta=(a+b)^2-4ab=5^2-4\cdot 7=-3<0$.\\
	Vậy không tồn tại hai số $a$ và $b$ thỏa mãn yêu cầu đề bài.
	\item Theo đề bài, ta có $a+b=-8$ và $ab=-84$.\\
	Ta xét $\Delta=(a+b)^2-4ab=(-8)^2-4\cdot (-84)=400\ge0$.\\
	Vậy có tồn tại hai số $a$ và $b$ thỏa mãn yêu cầu đề bài.\\
	Dẫn đến, $a$ và $b$ là nghiệm của phương trình $x^2+8x-84=0$.\\
	Ta có $\Delta=8^2-4\cdot 1\cdot 84=400>0$ nên phương trình có hai nghiệm phân biệt
	$$ x_1=\dfrac{-8+\sqrt{400}}{2\cdot 1}=6;\quad x_2=\dfrac{-8-\sqrt{400}}{2\cdot 1}=-14. $$
	Vậy $a=6$ và $b=14$ (hoặc ngược lại).
	\item Theo đề bài, ta có $a+b=-7$ và $ab=15$.\\
	Ta xét $\Delta=(a+b)^2-4ab=(-7)^2-4\cdot 15=-11<0$.\\
	Vậy không tồn tại hai số $a$ và $b$ thỏa mãn yêu cầu đề bài.
	\item Theo đề bài, ta có $a+b=-9$ và $ab=18$.\\
	Ta xét $\Delta=(a+b)^2-4ab=(-9)^2-4\cdot 18=9\ge0$.\\
	Vậy có tồn tại hai số $a$ và $b$ thỏa mãn yêu cầu đề bài.\\
	Dẫn đến, $a$ và $b$ là nghiệm của phương trình $x^2+9x+18=0$.\\
	Ta có $\Delta=9^2-4\cdot 1\cdot 18=9>0$ nên phương trình có hai nghiệm phân biệt
	$$ x_1=\dfrac{-9+\sqrt{9}}{2\cdot 1}=-3;\quad x_2=\dfrac{-9-\sqrt{9}}{2\cdot 1}=-6. $$
	Vậy $a=-3$ và $b=-6$ (hoặc ngược lại).
	\end{enumerate}
}
\end{bt}
\begin{bt}%[Dự án EX-9-Đề Cương Toán 9]%[GVSB: Hoàng Minh Nhân Mã - GVPB1: Phan Minh Huế - GVPB2: Nguyễn Trần Anh Tuấn]%[9D4V3-3]
	Khu vườn hình chữ nhật của bác Lâm có chu vi bằng $80$ m, diện tích $336$ m$^2$. Hãy tính chiều dài và chiều rộng của khu vườn?
\loigiai{
Gọi chiều dài và chiều rộng của khu vườn lần lượt là $a$ (m) và $b$ (m) với điều kiện: $a>0$, $b>0$.\\
Theo đề bài, ta có 
\begin{itemize}
	\item Chu vi khu vườn là $80$ m nên ta có $2(a+b)=80$.
	\item Diện tích khu vườn là $336$ m$^2$ nên ta có $ab=336$.
\end{itemize}
Khi đó, $a$ và $b$ là hai nghiệm của phương trình $x^2-(a+b)x+ab=0$, hay $x^2-40x+336=0$.\\
Ta có $\Delta=(-40)^2-4\cdot 1\cdot 336=256>0$ nên phương trình có hai nghiệm phân biệt
$$ x_1=\dfrac{-(-40)+\sqrt{256}}{2\cdot 1}=\dfrac{40+16}{2}=28;\quad x_2=\dfrac{-(-40)-\sqrt{256}}{2\cdot 1}=\dfrac{40-16}{2}=12. $$
Vì chiều dài lớn hơn chiều rộng, nên chiều dài và chiều rộng của khu vườn lần lượt là $28$ m và $12$ m.
}
\end{bt}
\begin{bt}%[Dự án EX-9-Đề Cương Toán 9]%[GVSB: Hoàng Minh Nhân Mã - GVPB1: Phan Minh Huế - GVPB2: Nguyễn Trần Anh Tuấn]%[9D4C3-3]
	Khu vườn hình chữ nhật có đường chéo bằng $120$ m, diện tích là $450$ m$^2$. Hãy tính chiều dài và chiều rộng của khu vườn?
\loigiai{
Gọi chiều dài và chiều rộng của khu vườn là $a$ (m) và $b$ (m) với điều kiện: $a>0$, $b>0$.\\
Theo đề bài, ta có
\begin{itemize}
	\item Diện tích khu vườn là $450$ m$^2$ nên ta có $ab=450$.
	\item Đường chéo khu vườn là $120$ m nên ta có $a^2+b^2=120^2=14\;400$.\\
		Suy ra, $a+b=\sqrt{(a+b)^2}=\sqrt{a^2+b^2+2ab}=\sqrt{14\;400+2\cdot 450}=30\sqrt{17}$.
\end{itemize}
Khi đó, $a$ và $b$ là hai nghiệm của phương trình $x^2-(a+b)x+ab=0$, hay $x^2-30\sqrt{17}x+450=0$.\\
Ta có $\Delta=(-30\sqrt{17})^2-4\cdot 1\cdot 450=13\;500>0$.\\
Nên phương trình có hai nghiệm phân biệt
$$ x_1=\dfrac{-(-30\sqrt{17})+\sqrt{13\;500}}{2\cdot 1}=15\sqrt{17}+15\sqrt{15};\quad
x_2=\dfrac{-(-30\sqrt{17})-\sqrt{13\;500}}{2\cdot 1}=15\sqrt{17}-15\sqrt{15}. $$
Vì chiều dài lớn hơn chiều rộng, nên $\heva{
&\text{chiều dài của khu vườn là}\;15\sqrt{17}+15\sqrt{15}\;\text{(m)}\\
&\text{chiều rộng của khu vườn là}\;15\sqrt{17}-15\sqrt{15}\;\text{(m)}.
}$
}
\end{bt}
\begin{bt}%[Dự án EX-9-Đề Cương Toán 9]%[GVSB: Hoàng Minh Nhân Mã - GVPB1: Phan Minh Huế - GVPB2: Nguyễn Trần Anh Tuấn]%[9D4V3-3]
	Một công ty sản xuất các khay có dạng hình hộp chữ nhật để trồng rau trong chung cư ở các thành phố. Biết diện tích mặt đáy của khay đó là $2\;856$ cm$^2$ và chu vi mặt đáy của khay đó là $220$ cm. Tìm các kích thước mặt đáy của khay đó.
\loigiai{
Gọi chiều dài và chiều rộng mặt đáy của khay lần lượt là $a$ (m) và $b$ (m) với điều kiện: $a>0$, $b>0$.\\
Theo đề bài, ta có 
\begin{itemize}
	\item Chu vi mặt đáy là $220$ cm nên ta có $2(a+b)=220$.
	\item Diện tích mặt đáy là $2856$ cm$^2$ nên ta có $ab=2\;856$.
\end{itemize}
Khi đó, $a$ và $b$ là hai nghiệm của phương trình $x^2-(a+b)x+ab=0$, hay $x^2-110x+2\;856=0$.\\
Ta có $\Delta=(-110)^2-4\cdot 1\cdot 2\;856=676>0$.\\
Nên phương trình có hai nghiệm phân biệt
$$ x_1=\dfrac{-(-110)+\sqrt{676}}{2\cdot 1}=68;\quad x_2=\dfrac{-(-110)-\sqrt{676}}{2\cdot 1}=42. $$
Vậy kích thước mặt đáy của khay lần lượt là $68$ cm và $42$ cm.
}
\end{bt}
\begin{bt}%[Dự án EX-9-Đề Cương Toán 9]%[GVSB: Hoàng Minh Nhân Mã - GVPB1: Phan Minh Huế - GVPB2: Nguyễn Trần Anh Tuấn]%[9D4V3-3]
	Hiện nay, tổng số tuổi của hai anh em Nhân và Duy là $14$ và tích số tuổi của hai em hiện nay là $45$. Tính số tuổi của Nhân và Duy (biết Nhân lớn hơn Duy).
\loigiai{
Gọi tuổi của Nhân và Duy lần lượt là $a$ (tuổi) và $b$ (tuổi) với điều kiện: $a>0$, $b>0$.\\
Theo đề bài, ta có 
\begin{itemize}
	\item Tổng số tuổi của hai anh em Nhân và Duy là $14$ nên ta có $a+b=14$.
	\item Tích số tuổi của hai anh em Nhân và Duy là $45$ nên ta có $ab=45$.
\end{itemize}
Khi đó, $a$ và $b$ là hai nghiệm của phương trình $x^2-(a+b)x+ab=0$, hay $x^2-14x+45=0$.\\
Ta có $\Delta=(-14)^2-4\cdot 1\cdot 45=16>0$ nên phương trình có hai nghiệm phân biệt
$$ x_1=\dfrac{-(-14)+\sqrt{16}}{2\cdot 1}=9;\quad x_2=\dfrac{-(-14)-\sqrt{16}}{2\cdot 1}=5. $$
Vì Nhân lớn hơn Duy, nên tuổi của Nhân và Duy lần lượt là $9$ tuổi và $5$ tuổi.
}
\end{bt}
\begin{bt}%[Dự án EX-9-Đề Cương Toán 9]%[GVSB: Hoàng Minh Nhân Mã - GVPB1: Phan Minh Huế - GVPB2: Nguyễn Trần Anh Tuấn]%[9D4V3-3]
	Hiện nay, tổng số tuổi của hai chị em Thảo và Linh là $27$ và tích số tuổi của hai em hiện nay là $176$. Tính số tuổi của Thảo và Linh (biết Thảo là chị, Linh là em).
\loigiai{
Gọi tuổi của Thảo và Linh lần lượt là $a$ (tuổi) và $b$ (tuổi) với điều kiện: $a>0$, $b>0$.\\
Theo đề bài, ta có 
\begin{itemize}
	\item Tổng số tuổi của hai chị em Thảo và Linh là $27$ nên ta có $a+b=27$.
	\item Tích số tuổi của hai chị em hiện nay là $176$ nên ta có $ab=176$.
\end{itemize}
Khi đó, $a$ và $b$ là hai nghiệm của phương trình $x^2-(a+b)x+ab=0$, hay $x^2-27x+176=0$.\\
Ta có $\Delta=(-27)^2-4\cdot 1\cdot 176=25>0$ nên phương trình có hai nghiệm phân biệt.
$$ x_1=\dfrac{-(-27)+\sqrt{25}}{2\cdot 1}=16; \quad
x_2=\dfrac{-(-27)-\sqrt{25}}{2\cdot 1}=11. $$
Vì Thảo là chị và Linh là em, nên tuổi của Thảo và Linh lần lượt là $16$ tuổi và $11$ tuổi.
}
\end{bt}
\begin{bt}%[Dự án EX-9-Đề Cương Toán 9]%[GVSB: Hoàng Minh Nhân Mã - GVPB1: Phan Minh Huế - GVPB2: Nguyễn Trần Anh Tuấn]%[9D4C3-3]
	Vào cuối năm tài chính 2\;023, một công ty khởi nghiệp công nghệ đã báo cáo tổng lợi nhuận từ hai dự án chính (Dự án $A$ về Trí tuệ nhân tạo và Dự án $B$ về Ứng dụng di động) là $500$ triệu đồng. Biết rằng tích của hai khoản lợi nhuận này là $60\;000$ (đơn vị: triệu đồng$^2$). Hỏi mỗi dự án đã mang lại bao nhiêu triệu đồng lợi nhuận cho công ty?
\loigiai{
	Gọi lợi nhuận từ Dự án $A$ và dự án $B$ lần lượt là $L_A$ và $L_B$ (triệu đồng) với $L_A>0$, $L_B>0$.\\
	Theo đề bài, ta có 
	\begin{itemize}
		\item Tổng lợi nhuận từ hai dự án là $500$ triệu đồng nên ta có $L_A+L_B=500$.
		\item Tích của hai khoản lợi nhuận này là $60\;000$ triệu đồng$^2$ nên ta có $L_A\cdot L_B=60\;000$.
	\end{itemize}
	Khi đó, $L_A$ và $L_B$ là hai nghiệm của phương trình $x^2-(L_A+L_B)x+L_A L_B=0$, hay $x^2-500x+60\;000=0$.\\
	Ta có $\Delta=(-500)^2-4\cdot 1\cdot 60\;000=10\;000>0$ nên phương trình có hai nghiệm phân biệt
	$$ x_1=\dfrac{-(-500)+\sqrt{10\;000}}{2\cdot 1}=300; \quad
	x_2=\dfrac{-(-500)-\sqrt{10\;000}}{2\cdot 1}=200. $$
	Vậy lợi nhuận của dự án $A$ và $B$ lần luọt là $200$ triệu đồng và $300$ triệu đồng.
}
\end{bt}
\begin{bt}%[Dự án EX-9-Đề Cương Toán 9]%[GVSB: Hoàng Minh Nhân Mã - GVPB1: Phan Minh Huế - GVPB2: Nguyễn Trần Anh Tuấn]%[9D4V3-3]
	Tại Khu bảo tồn thiên nhiên X, các nhà sinh vật học đang tiến hành thống kê số lượng hai loài động vật quý hiếm: Sếu đầu đỏ và Vọoc ngũ sắc. Trong đợt khảo sát gần đây nhất (tháng 11/2\;024), tổng số lượng cá thể của cả hai loài được ghi nhận là $80$ con. Đồng thời, các nhà khoa học tính toán rằng tích của số lượng cá thể hai loài này là $1\;500$. Hãy xác định số lượng cá thể của mỗi loài Sếu đầu đỏ và Vọoc ngũ sắc trong khu bảo tồn.
\loigiai{
	Gọi số lượng cá thể loài Sếu đầu đỏ và Vọoc ngũ sắc lần lượt là $N_S$ và $N_V$ (con) với $N_S\in\mathbb{N}^*$, $N_V\in\mathbb{N}^*$.\\
	Theo đề bài, ta có
	\begin{itemize}
		\item Tổng số lượng cá thể của cả hai loài là $80$ con nên ta có $N_S+N_V=80$.
		\item Tích của số lượng cá thể hai loài này là $1\;500$ nên ta có $N_S\cdot N_V=1\;500$
	\end{itemize}
	Khi đó, $N_S$ và $N_V$ là hai nghiệm của phương trình $x^2-(N_S+N_V)x+N_S N_V=0$, hay $x^2-80x+1\;500=0$.\\
	Ta có $\Delta=(-80)^2-4\cdot 1\cdot 1\;500=400>0$ nên phương trình có hai nghiệm phân biệt:
	$$ x_1=\dfrac{-(-80)+\sqrt{400}}{2\cdot 1}=50\quad
	x_2=\dfrac{-(-80)-\sqrt{400}}{2\cdot 1}=30. $$
	Vậy số lượng cá thể Sếu đầu đỏ là $50$ con và Vọoc ngũ sắc là $30$ con (hoặc ngược lại).
}
\end{bt}
\begin{bt}%[Dự án EX-9-Đề Cương Toán 9]%[GVSB: Hoàng Minh Nhân Mã - GVPB1: Phan Minh Huế - GVPB2: Nguyễn Trần Anh Tuấn]%[9D4C3-3]
	Trong một dự án thiết kế mạch điện tử tại Đại học Bách Khoa, hai điện trở $R_1$ và $R_2$ được sử dụng. Khi mắc nối tiếp, tổng trở của chúng là $150$ Ohm. Khi mắc song song, điện trở tương đương của chúng là $36$ Ohm. Hãy tính giá trị của mỗi điện trở $R_1$ và $R_2$.
\loigiai{
	Gọi giá trị của điện trở $R_1$ và $R_2$ lần lượt là $R_1$ (Ohm) và $R_2$ (Ohm) với $R_1>0$, $R_2>0$.\\
	Theo đề bài, ta có
	\begin{itemize}
		\item Khi mắc nối tiếp, tổng trở là $150$ Ohm nên ta có $R_1+R_2=150$.
		\item Khi mắc song song, điện trở tương đương là $36$ Ohm nên ta có $\dfrac{1}{R_1}+\dfrac{1}{R_2}=\dfrac{1}{36}$.\\
		Lúc đó, $\dfrac{1}{R_1}+\dfrac{1}{R_2}= \dfrac{R_1+R_2}{R_1R_2}=\dfrac{150}{R_1R_2}=\dfrac{1}{36}$ nên $R_1R_2=150\cdot 36=5\;400$.
	\end{itemize}
	Khi đó, $R_1$ và $R_2$ là hai nghiệm của phương trình $x^2-(R_1+R_2)x+R_1 R_2=0$, hay $x^2-150x+5\;400=0$.\\
	Ta có $\Delta=(-150)^2-4\cdot 1\cdot 5\;400=900>0$ nên phương trình có hai nghiệm phân biệt:
	$$ x_1=\dfrac{-(-150)+\sqrt{900}}{2\cdot 1}=\dfrac{150+30}{2}=90; \quad
	x_2=\dfrac{-(-150)-\sqrt{900}}{2\cdot 1}=\dfrac{150-30}{2}=60. $$
	Vậy giá trị của hai điện trở là $60$ Ohm và $90$ Ohm.
}
\end{bt}
\begin{bt}%[Dự án EX-9-Đề Cương Toán 9]%[GVSB: Hoàng Minh Nhân Mã - GVPB1: Phan Minh Huế - GVPB2: Nguyễn Trần Anh Tuấn]%[9D4V3-3]
	Trong nghiên cứu về thành phần nguyên tử, một nguyên tố X được xác định có hai đồng vị ổn định. Tổng khối lượng nguyên tử tương đối của hai đồng vị này là $40$ đvC. Biết rằng tích của khối lượng nguyên tử tương đối của chúng là $384$ (đơn vị: đvC$^2$). Hãy xác định khối lượng nguyên tử tương đối của mỗi đồng vị.
\loigiai{
	Gọi khối lượng nguyên tử tương đối của đồng vị thứ nhất và thứ hai lần lượt là $m_1$ và $m_2$ (đvC) với $m_1>0$, $m_2>0$.\\
	Theo đề bài, ta có 
	\begin{itemize}
		\item Tổng khối lượng nguyên tử tương đối là $40$ đvC nên ta có $m_1+m_2=40$.
		\item Tích của khối lượng nguyên tử tương đối là $384$ đvc$^2$ nên ta có $m_1\cdot m_2=384$.
	\end{itemize}
	Khi đó, $m_1$ và $m_2$ là hai nghiệm của phương trình $x^2-(m_1+m_2)x+m_1 m_2=0$, hay $x^2-40x+384=0$.\\
	Ta có $\Delta=(-40)^2-4\cdot 1\cdot 384=64>0$ nên phương trình có hai nghiệm phân biệt:
	$$ x_1=\dfrac{-(-40)+\sqrt{64}}{2\cdot 1}=\dfrac{40+8}{2}=24; \quad
	x_2=\dfrac{-(-40)-\sqrt{64}}{2\cdot 1}=\dfrac{40-8}{2}=16. $$
	Vậy khối lượng nguyên tử tương đối của hai đồng vị là $16$ đvC và $24$ đvC.
}
\end{bt}
\begin{bt}%[Dự án EX-9-Đề Cương Toán 9]%[GVSB: Hoàng Minh Nhân Mã - GVPB1: Phan Minh Huế - GVPB2: Nguyễn Trần Anh Tuấn]%[9D4V3-3]
	Một vận động viên marathon đang chuẩn bị cho giải chạy lớn. Trong kế hoạch tập luyện hàng tuần, anh ấy có hai loại buổi tập chính: chạy dài và chạy biến tốc. Tổng quãng đường chạy của hai loại buổi tập này trong một tuần là $20$ km. Các huấn luyện viên tính toán rằng tích của hai quãng đường chạy này là $96$ (đơn vị: km$^2$). Hỏi quãng đường chạy của mỗi loại buổi tập là bao nhiêu km?
\loigiai{
	Gọi quãng đường chạy dài và quãng đường chạy biến tốc lần lượt là $D$ (km) và $B$ (km) với $D>0$, $B>0$.\\
	Theo đề bài, ta có
	\begin{itemize}
		\item Tổng quãng đường chạy của hai loại buổi tập là $20$ km nên ta có $D+B=20$.
		\item Tích của hai quãng đường chạy này là $96$ km$^2$ mêm ta có $D\cdot B=96$.
	\end{itemize}
	Khi đó, $D$ và $B$ là hai nghiệm của phương trình $x^2-(D+B)x+D\cdot B=0$, hay $x^2-20x+96=0$.\\
	Ta có $\Delta=(-20)^2-4\cdot 1\cdot 96=16>0$ nên phương trình có hai nghiệm phân biệt:
	$$ x_1=\dfrac{-(-20)+\sqrt{16}}{2\cdot 1}=\dfrac{20+4}{2}=12; \quad
	x_2=\dfrac{-(-20)-\sqrt{16}}{2\cdot 1}=\dfrac{20-4}{2}=8. $$
	Vậy quãng đường chạy dài là $12$ km và quãng đường chạy biến tốc là $8$ km (hoặc ngược lại).
}
\end{bt}

\begin{bt}%[Dự án EX-9-Đề Cương Toán 9]%[GVSB: Hoàng Minh Nhân Mã - GVPB1: Phan Minh Huế - GVPB2: Nguyễn Trần Anh Tuấn]%[9D4V3-3]
	Theo số liệu thống kê từ Cục Thống kê Quốc gia (năm 2\;024), hai thành phố vệ tinh lớn của TP.HCM là Biên Hòa và Dĩ An có tổng dân số ước tính là $2$ triệu người. Đồng thời, các nhà quy hoạch đô thị ước tính rằng tích của dân số hai thành phố này là $0{,}75$ (đơn vị: triệu người$^2$). Hãy tính dân số ước tính của mỗi thành phố Biên Hòa và Dĩ An.
\loigiai{
	Gọi dân số của thành phố Biên Hòa và thành phố Dĩ An lần lượt là $P_B$ (triệu người) và  $P_D$ (triệu người) với $P_B>0$, $P_D>0$.\\
	Theo đề bài, ta có
	\begin{itemize}
		\item Tổng dân số ước tính là $2$ triệu người nên ta có $P_B+P_D=2$,
		\item Tích của dân số hai thành phố này là $0{,}75$ triệu người$^2$ nên ta có $P_B\cdot P_D=0{,}75$.
	\end{itemize}
	Khi đó, $P_B$ và $P_D$ là hai nghiệm của phương trình $x^2-(P_B+P_D)x+P_B\cdot P_D=0$, hay $x^2-2x+0{,}75=0$.\\
	Ta có $\Delta=(-2)^2-4\cdot 1\cdot 0{,}75=1>0$ nên phương trình có hai nghiệm phân biệt:
	$$ x_1=\dfrac{-(-2)+\sqrt{1}}{2\cdot 1}=1{,}5; \quad
	x_2=\dfrac{-(-2)-\sqrt{1}}{2\cdot 4}=0{,}5. $$
	Vậy dân số của thành phố Biên Hòa là $1{,}5$ triệu người và Dĩ An là $0{,}5$ triệu người (hoặc ngược lại).
}
\end{bt}