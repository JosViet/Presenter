\section*{BÀI TẬP CUỐI CHƯƠNG 3}
\subsection{Câu hỏi trắc nghiệm}
\Opensolutionfile{ans}[ans/ans-9C3-OTC]
%%%====Câu 1
\begin{ex}%[Dự án EX-9-Đề Cương Toán 9]%[Thầy Nguyễn Đức Tuấn Anh]%9D3H1-1]
	Khẳng định nào sau đây là \textbf{sai}?
	\choice
	{Số âm không có căn bậc hai}
	{Số $0$ chỉ có đúng một căn bậc hai là chính nó}
	{\True Mỗi số thực $a$ có đúng hai căn bậc hai là hai số đối nhau}
	{Với hai số $a$, $b$ không âm, nếu $a<b$ thì $\sqrt{a}<\sqrt{b}$}
	\loigiai{
	Nếu $a<0$ thì không có căn bậc hai.
	}
\end{ex}

%%%====Câu 2
\begin{ex}%[Dự án EX-9-Đề Cương Toán 9]%[Thầy Nguyễn Đức Tuấn Anh]%[9D3N1-2]
	Giá trị của biểu thức $\sqrt{0{,}09}+7\cdot\sqrt{0{,}36}-3\cdot\sqrt{2{,}25}$ bằng
	\choice
	{\True $0$}
	{$1$}
	{$2$}
	{$3$}
	\loigiai{
	Sử dụng MTCT, tính được $\sqrt{0{,}09}+7\cdot\sqrt{0{,}36}-3\cdot\sqrt{2{,}25}=0$.
	}
\end{ex}

%%%====Câu 3
\begin{ex}%[Dự án EX-9-Đề Cương Toán 9]%[Thầy Nguyễn Đức Tuấn Anh]%[9D3H1-3]
	Tìm điều kiện xác định của $\sqrt{5 - 2x}$
	\choice
	{$x\leq\dfrac{2}{5}$}
	{$x>\dfrac{5}{2}$}
	{\True $x\leq\dfrac{5}{2}$}
	{$x\geq\dfrac{5}{2}$}
	\loigiai{
	Điều kiện xác định của căn thức là $5-2x\geq 0$ hay $x\leq\dfrac{5}{2}$.
	}
\end{ex}

%%%====Câu 4
\begin{ex}%[Dự án EX-9-Đề Cương Toán 9]%[Thầy Nguyễn Đức Tuấn Anh]%[9D3H1-3]
	Tìm $x$ biết $3x^2 = 0{,}75$
	\choice
	{$0{,}25$}
	{$0{,}5$}
	{$x=0{,}25$ hoặc $x=-0{,}25$}
	{\True $x=0{,}5$ hoặc $x=-0{,}5$}
	\loigiai{
	Ta có $3x^2 = 0{,}75$ suy ra $x^2=0{,}25$. Do đó $|x| =0{,}5$.\\
	Vậy $x=0{,}5$ hoặc $x=-0{,}5$.
	}
\end{ex}


%%%====Câu 5
\begin{ex}%[Dự án EX-9-Đề Cương Toán 9]%[Thầy Nguyễn Đức Tuấn Anh]%[9D3H1-3]
	\immini{
		Biết rằng hình $A$ và hình vuông $B$ trong hình bên có diện tích bằng nhau. Tính độ dài cạnh $x$ của hình vuông $B$.
		\choice
		{$x=\sqrt{2}$ cm}
		{\True $x=\sqrt{7}$ cm}
		{$x=\sqrt{7}$ cm hoặc $x=-\sqrt{7}$ cm}
		{$x=\sqrt{3}$ cm}
	}
	{
		\begin{tikzpicture}[line join=round,line cap=round,>=stealth,font=\footnotesize,scale=0.8]
			\def\canh{3}
			\coordinate (D) at (0,\canh);
			\coordinate (B) at (\canh,\canh);
			\coordinate (A) at (0,0);
			\coordinate (C) at ($(B)+(A)-(D)$);
			\coordinate (x) at ($(B)!1/3!(D)$);
			\coordinate (y) at ($(C)!2/3!(B)$);
			\coordinate (z) at ($(x)+(y)-(B)$);
			\draw (A)--(D)--(x)--(z)--(y)--(C)--(A);
			\draw [dashed](x)--(B)--(y);
			\node[below] at ($($(A)+(0,-0.25)$)!0.5!($(C)+(0,-0.25)$)$){$3$ cm};
			\node[below] at ($($(A)+(0,-0.25)$)!0.5!($(C)+(0,-0.25)$)+(0,-0.6)$){Hình $A$};
			\draw[<->] ($(A)+(0,-0.2)$)--($(C)+(0,-0.2)$);
			\node[left=0.3cm] at ($($(A)$)!0.5!($(D)$)$){$3$ cm};
			\draw[<->] ($(A)+(-0.2,0)$)--($(D)+(-0.2,0)$);
			\node[above=0.4cm] at ($($(x)+(0,-0.25)$)!0.5!($(B)+(0,-0.25)$)$){$\sqrt{2}$ cm};
			\draw[<->] ($(x)+(0,0.2)$)--($(B)+(0,0.2)$);
			\node[right=0.2cm] at ($($(B)$)!0.5!($(y)$)$){$\sqrt{2}$ cm};
			\draw[<->] ($(B)+(0.2,0)$)--($(y)+(0.2,0)$);
			\begin{scope}[shift={(5.5,0)},rotate=0]
				\def\canh{2.5}
				\coordinate (A) at (0,\canh);
				\coordinate (D) at (\canh,\canh);
				\coordinate (B) at (0,0);
				\coordinate (C) at ($(B)+(D)-(A)$);
				\draw(A)--(B)--(C)--(D)--cycle;
				\node[below] at ($($(B)+(0,-0.25)$)!0.5!($(C)+(0,-0.25)$)$){$x$ cm};
				\node[below] at ($($(B)+(0,-0.25)$)!0.5!($(C)+(0,-1.5)$)$){Hình $B$};
				\draw[<->] ($(B)+(0,-0.2)$)--($(C)+(0,-0.2)$);
			\end{scope}
		\end{tikzpicture}
	}\loigiai{
		Diện tích hình $A$ là $3^2-\sqrt{2}^2=7$ cm$^2$.\\
		Theo đề bài, ta có diện tích hình $B$ cũng là $7$ cm$^2$. Mà diện tích hình $B$ là $x^2$ nên $x^2 = 7$.\\
		Suy ra $x=\sqrt{7}$ cm (do độ dài cạnh luôn dương).
	}
\end{ex}

%%%====Câu 6
\begin{ex}%[Dự án EX-9-Đề Cương Toán 9]%[Thầy Nguyễn Đức Tuấn Anh]%[9D3N1-2]
	Tính giá trị của biểu thức $\sqrt[3]{\dfrac{-8}{125}}$
	\choice
	{$\dfrac{2}{5}$}
	{\True $-\dfrac{2}{5}$}
	{$\dfrac{-8}{125}$}
	{$-\dfrac{5}{2}$}
	\loigiai{
		Sử dụng MTCT, tính được $\sqrt[3]{\dfrac{-8}{125}}=-\dfrac{2}{5}$.
	}
\end{ex}

%%%====Câu 7
\begin{ex}%[Dự án EX-9-Đề Cương Toán 9]%[Thầy Nguyễn Đức Tuấn Anh]%[9D3N2-3]
	Giá trị của căn thức $\sqrt[3]{5x-1}$ tại $x=0$ là
	\choice
	{\True $-1$}
	{$1$}
	{$2$}
	{$-2$}
	\loigiai{
	Với $x=0$ ta có $\sqrt[3]{5\cdot 0-1}=\sqrt[3]{-1}=\sqrt[3]{(-1)^3}=-1$.
	}
\end{ex}

%%%====Câu 8
\begin{ex}%[Dự án EX-9-Đề Cương Toán 9]%[Thầy Nguyễn Đức Tuấn Anh]%[9D3H2-3]
	Rút gọn biểu thức $A=\sqrt[3]{x^3 + 1 + 3x(x + 1)}$, ta được 
	\choice
	{\True $x+1$}
	{$x-1$}
	{$-1$}
	{$x$}
	\loigiai{
	Ta có $\sqrt[3]{x^3 + 1 + 3x(x + 1)}=\sqrt[3]{(x + 1)^3}=x + 1$.
	}
\end{ex}

%%%====Câu 9
\begin{ex}%[Dự án EX-9-Đề Cương Toán 9]%[Thầy Nguyễn Đức Tuấn Anh]%[9D3H2-3]
	Tìm $x$ biết $\sqrt[3]{x}=8$.
	\choice
	{\True $512$}
	{$8$}
	{$2$}
	{$1$}
	\loigiai{
	Ta có $\sqrt[3]{x}=8$, suy ra $x=8^3=512$.
	}
\end{ex}

%%%====Câu 10
\begin{ex}%[Dự án EX-9-Đề Cương Toán 9]%[Thầy Nguyễn Đức Tuấn Anh]%[9D3H2-4]
	\immini{Cho khối bê tông hình lập phương có thể tích là $8$ dm$^3$. Tính độ dài cạnh của khối bê tông đó.
	\choice
	{$2$ cm}
	{$4$ dm}
	{\True $20$ cm}
	{$8$ dm}
	}{\begin{tikzpicture}
			\def\a{2}
			\path 	(0:0) coordinate (A)
			++(0:\a) coordinate (D)
			++(-130:\a/2) coordinate (C)
			($(A)+(C)-(D)$) coordinate (B)
			($(A)+(90:\a)$) coordinate (A')
			($(B)+(90:\a)$) coordinate (B')
			($(C)+(90:\a)$) coordinate (C')
			($(D)+(90:\a)$) coordinate (D');
			\draw[dashed,thick] 	(B)--(A)--(D)	(A)--(A');
			\draw[thick] 	(C)--(C') 	(D)--(D') 	(B)--(B') 	
			(B)--(C)--(D) (A')--(B')--(C')--(D')--cycle;	
	\end{tikzpicture}}
	\loigiai{
	Gọi $a$ (cm) là độ dài cạnh của khối bê tông, với $a>0$.\\
	Ta có $a^3=8$ hay $a^3=2^3$ suy ra $a=2$ (dm). \\
	Vậy độ dài cạnh của khối bê tông là $a=2$ (dm).
	}
\end{ex}

%%%====Câu 11
\begin{ex}%[Dự án EX-9-Đề Cương Toán 9]%[Thầy Nguyễn Đức Tuấn Anh]%[9D3H4-2]
	Rút gọn biểu thức  $\sqrt{1-2 x+x^2}$ với $x>1$, ta được
	\choice
	{$x-2$}
	{$2-x$}
	{$1-x$}
	{\True $x-1$}
	\loigiai{
	Ta có $\sqrt{1-2 x+x^2}=\sqrt{(1-x)^2}=|1-x|$.
	Vì  $x>2$ nên $|1-x|=x-1$.\\
	Vậy $\sqrt{1-2 x+x^2}=x-1$.
	}
\end{ex}

%%%====Câu 12
\begin{ex}%[Dự án EX-9-Đề Cương Toán 9]%[Thầy Nguyễn Đức Tuấn Anh]%[9D3H4-1]
	Rút gọn biểu thức  $\sqrt{\frac{4 a^{2}}{25}}$, ta được
	\choice
	{$\frac{5a}{b}$}
	{$\frac{4a}{25b}$}
	{\True $\frac{2|a|}{5}$}
	{$\frac{2a}{5}$}
	\loigiai{
	Ta có $\sqrt{\frac{4 a^{2}}{25}}=\frac{\sqrt{4 a^{2}}}{\sqrt{25}}=\frac{\sqrt{4} \cdot \sqrt{a^{2}}}{5}=\frac{2|a|}{5}$.
	}
\end{ex}


%%%====Câu 13
\begin{ex}%[Dự án EX-9-Đề Cương Toán 9]%[Thầy Nguyễn Đức Tuấn Anh]%[9D3N4-2]
	Tìm khẳng định \textbf{sai} trong các khẳng định sau
	\choice
	{\True Với số thực $a$ bất kì và $b$ không âm, ta có $\sqrt{a^2b}=a\sqrt{b}$}
	{Nếu $a\geq 0$ thì $a\sqrt{b}=\sqrt{a^2b}$}
	{Nếu $a< 0$ thì $a\sqrt{b}=-\sqrt{a^2b}$}
	{Với hai số thực $a$ và $b$ không âm, ta có $\sqrt{a\cdot b}=\sqrt{a}\cdot \sqrt{b}$}
	\loigiai{
	Với số thực $a$ bất kì và $b$ không âm, ta có $\sqrt{a^2b}=|a|\sqrt{b}$.
	}
\end{ex}

%%%====Câu 14
\begin{ex}%[Dự án EX-9-Đề Cương Toán 9]%[Thầy Nguyễn Đức Tuấn Anh]%[9D3V4-2]
	Tìm $x$ biết $\sqrt{9 x^2-90 x+225}=6$. 
	\choice
	{\True $x=3$ hoặc $x=7$}
	{$x=7$}
	{$x=6$ hoặc $x=7$}
	{$x=6$}
	\loigiai{
		Ta có \allowdisplaybreaks 
		\begin{eqnarray*} 
			 \sqrt{9 x^2-90 x+225}&=&6 \\ 
			 \sqrt{9 \left(x^2-10 x+25 \right)}&=&6\\
			 \sqrt{9(x-5)^2}&=&6 \\
			 3 |x-5|&=&6 \\
			 |x-5|&=&2\\
			 x-5=2\text{ hoặc }x-5&=&-2\\
			 x=7 \text{ hoặc } x&=&3.
		\end{eqnarray*}
	} 
\end{ex}

%%%====Câu 15
\begin{ex}%[Dự án EX-9-Đề Cương Toán 9]%[Thầy Nguyễn Đức Tuấn Anh]%[9D3V4-3]
		\immini{Bạn Lan cắt một hình chữ nhật $A B C D$ thành những hình tam giác như hình bên (đơn vị tính theo centimét). Tính độ dài cạnh $CD$.
		\choice[2]
		{$\sqrt{2}$ cm}
		{$\sqrt{3}$ cm}
		{\True $\sqrt{8}$ cm}
		{\True $2$ cm}
		}{\begin{tikzpicture}[scale=1, font=\footnotesize, line join=round, line cap=round, >=stealth]
				\path
				(0,0) coordinate (K)
				(3,0) coordinate (D)
				(0,-3) coordinate (C)
				($(K)!0.5!(D)$) coordinate (M)
				($(K)!0.5!(C)$) coordinate (I)
				($(M)!1!90:(D)$) coordinate (A)
				($(A)!2!(K)$) coordinate (B)
				;
				\draw 
				(A)--(B)--(C)--(D)--(A)--(M)node[midway,right]{$1$} (D)--(M)node[midway,above]{$1$}--(K)node[midway,above]{$1$}--(I)node[midway,left]{$1$}--(C)node[midway,left]{$1$} (B)--(I)node[midway,above]{$1$}
				;
				\foreach \p/\g in {A/90, B/180, C/-90, D/0, M/-90, I/0, K/120}
				\draw[fill=black] (\p) circle (1pt) node[shift=(\g:3mm)] {$\p$};
				\pic[draw,angle radius=2mm]{right angle=A--M--D};
				\pic[draw,angle radius=2mm]{right angle=D--K--C};
				\pic[draw,angle radius=2mm]{right angle=B--I--C};
		\end{tikzpicture}}
		\loigiai{
		Ta có $CK=1+1=2$ cm; $DK=1+1=2$ cm.\\
		Trong tam giác vuông cân $CKD$, ta có $CD^2=CK^2+DK^2$ (định lí Pythagore).\\
		Suy ra $C D^2=2^2+2^2=8$. Do đó $CD=\sqrt{8}$ cm.
		}
\end{ex}
\subsection{Bài tập tự luận}

%%%====Bài 1
\begin{bt}%[Dự án EX-9-Đề Cương Toán 9]%[Thầy Nguyễn Đức Tuấn Anh]%[9D3H1-1]
	Không sử dụng MTCT, hãy so sánh
	\begin{multicols}{3}
		\begin{enumerate}
			\item $\sqrt{3}$ và $\sqrt{5}$;
			\item $3$ và $\sqrt{10}$;
			\item $8$ và $\sqrt{65}$.
		\end{enumerate}
	\end{multicols}
	\loigiai{
		\begin{enumerate}
			\item Ta có $3<5$ nên $\sqrt{3}<\sqrt{5}$.
			\item Ta có $10>9$ nên $\sqrt{10}>\sqrt{9}$ hay $\sqrt{10}>3$.
			\item Ta có $64<65$ nên $\sqrt {64} < \sqrt {65}$ hay $8 < \sqrt { 65 }$.
		\end{enumerate}
	}
\end{bt}

%%%====Bài 2
\begin{bt}%[Dự án EX-9-Đề Cương Toán 9]%[Thầy Nguyễn Đức Tuấn Anh]%[9D3H1-3]
	Tìm điều kiện xác định của mỗi căn thức sau
	\begin{multicols}{2}
		\begin{enumerate}
			\item $\sqrt{1 - 2x}$;
			\item $\sqrt{\dfrac{1}{x^2 - 4x + 4}}$;
			\item $\sqrt{25-x^2}$;
			\item $\sqrt{\dfrac{1}{x^2 - 100}}$.
		\end{enumerate}
	\end{multicols}
	\loigiai{
		\begin{enumerate}
			\item
			Điều kiện xác định của căn thức là $1-2x\geq 0$ hay $x\leq\dfrac{1}{2}$.
			\item 	Điều kiện xác định của căn thức là $x^2 - 4x + 4$ hay $(x - 2)^2> 0$. Suy ra $x\neq 2$.
			\item 
			Điều kiện xác định của căn thức là 
			\begin{eqnarray*}
				25-x^2&\geq& 0\\
				x^2&\leq& 25\\
				|x|&\leq& 5\\
				-5\leq x&\leq& 5.
			\end{eqnarray*}
			\item
			Điều kiện xác định của căn thức là 
			\begin{eqnarray*}
				&x^2-100>0\\
				&x^2>100\\
				&|x|>10\\
				&x>10\text{ hoặc } x<-10.
			\end{eqnarray*}
		\end{enumerate}	
	}
\end{bt}

%%%====Bài 3
\begin{bt}%[Dự án EX-9-Đề Cương Toán 9]%[Thầy Nguyễn Đức Tuấn Anh]%[9D3H1-3]%
	Tìm $x$, biết
	\begin{multicols}{2}
		\begin{enumerate}
			\item $2\sqrt{3x}= 12$;
			\item $\dfrac{1}{2}\sqrt{5x}= 10$.
			\item $x^3=\dfrac{64}{125}$;
			\item $\sqrt[3]{x}=-0{,}9$.
		\end{enumerate}
	\end{multicols}
	\loigiai{
	\begin{enumerate}
		\item Điều kiện xác định $x\geq 0$.\\Ta có
		\begin{eqnarray*}
			2\sqrt{3x}&=& 12\\
			\sqrt{3x}&=& 6\\
			3x &=& 36\\
			x &=& 12\text{ (thỏa mãn)}.
		\end{eqnarray*}
		Vậy $x=12$.
		\item Điều kiện xác định $x\geq 0$.\\ Ta có 
		\begin{eqnarray*}
			\dfrac{1}{2}\sqrt{5x}&=& 10\\
			\sqrt{5x}&=& 20\\
			5x &=& 400\\
			x &=& 80\text{ (thỏa mãn)}.
		\end{eqnarray*}
		Vậy $x=80$.
		\item Ta có
		\begin{eqnarray*}
			x^3&=&\dfrac{64}{125}\\
			x&=&\sqrt[3]{\dfrac{64}{125}}\\
			x &=& \dfrac{4}{5}.
		\end{eqnarray*}
		Vậy $x=\dfrac{4}{5}$.
		\item Ta có
		\begin{eqnarray*}
			\sqrt[3]{x}&=&-0{,}9\\
			x&=&(-0{,}9)^3 \\
			x &=& -0{,}729.
		\end{eqnarray*}
		Vậy $x=-0{,}729$.
	\end{enumerate}
	}
\end{bt}

%%%====Bài 4
\begin{bt}%[Dự án EX-9-Đề Cương Toán 9]%[Thầy Nguyễn Đức Tuấn Anh]%[9D3V1-3]
	Có bao nhiêu giá trị nguyên của $x$ để biểu thức $M=\sqrt{x+4}+\sqrt{2-x}$ xác định?
	\loigiai{$M$ xác định khi $\heva{&x+4\geq 0\\&2-x\geq 0}$ hay $\heva{&x\geq -4\\&x\leq 2.}$\\Vì $x\in\mathbb{Z}$ nên $x\in\left\lbrace -4; -3; -2; -1; 0; 1; 2\right\rbrace $.\\
	Vậy có $7$ giá trị nguyên của $x$ để biểu thức $M=\sqrt{x+4}+\sqrt{2-x}$ có nghĩa.}
\end{bt}

%%%====Bài 5
\begin{bt}%[Dự án EX-9-Đề Cương Toán 9]%[Thầy Nguyễn Đức Tuấn Anh]%[9D3H1-4]
	Trong một thí nghiệm, một vật rơi tự do từ độ cao $80$ m so với mặt đất. Biết quãng đường dịch chuyển được của vật đó tính theo đơn vị mét được cho bởi công thức $s=5t^2$ với $t$ là thời gian vật đó rơi, tính theo đơn vị giây ($t>0$). Hỏi sau bao nhiêu lâu kể từ lúc rơi thì vật đó chạm đất?
	\loigiai{
		Khi vật chạm đất thì quãng đường dịch chuyển được của vật đó là $80$ m.\\
		Ta có $80=5t^2$ hay $t^2=16$. Do đó $t=\sqrt{16}=4$ hoặc $t=-\sqrt{16}=-4$.\\
		Vì $t>0$ nên $t=4$. Vậy sau $4$ giây kể từ lúc rơi thì vật đó chạm đất.
	}
\end{bt}

%%%====Bài 6
\begin{bt}%[Dự án EX-9-Đề Cương Toán 9]%[Thầy Nguyễn Đức Tuấn Anh]%[9D3H2-1]
	Cho $a<0$, so sánh $\sqrt[3]{2a}$ và $\sqrt[3]{3a}$?
	\loigiai
	{Ta có $2<3$ suy ra $2a>3a$ (vì $a<0$).\\ Do đó $\sqrt[3]{2a}>\sqrt[3]{3a}$.
	}
\end{bt}

%%%====Bài 7
\begin{bt}%[Dự án EX-9-Đề Cương Toán 9]%[Thầy Nguyễn Đức Tuấn Anh]%[9D3H2-3]
	Cho biểu thức $P=\sqrt[3]{3x-2}$. Tính giá trị của $P$ khi $x=3$ và khi $x=-2$ (kết quả làm tròn đến chữ số thập phân thứ ba).
	\loigiai{
		\begin{itemize}
			\item Với $x=3$, ta có $P=\sqrt[3]{3\cdot 3-2}=\sqrt[3]{7}\approx 1{,}913$;
			\item Với $x=-2$, ta có $P=\sqrt[3]{3\cdot (-2)-2}=\sqrt[3]{-8}=-2$.
		\end{itemize}
	}
\end{bt}


%%%====Bài 8
\begin{bt}%[Dự án EX-9-Đề Cương Toán 9]%[Thầy Nguyễn Đức Tuấn Anh]%[9D3V2-3]
	Rút gọn biểu thức 
	\begin{multicols}{3}
		\begin{enumerate}
			\item $\dfrac{x + 1}{\sqrt[3]{x^2} - \sqrt[3]{x} + 1}$;
			\item $\sqrt[3]{x^3-3x^2+3x-1}$;
			\item $-x+5+\sqrt[3]{x^3+3x^2+3x+1}$.
		\end{enumerate}
	\end{multicols}
	\loigiai
	{
		\begin{enumerate}
			\item $\dfrac{x + 1}{\sqrt[3]{x^2} - \sqrt[3]{x} + 1}=\dfrac{(\sqrt[3]{x} + 1)\left(\sqrt[3]{x^2} - \sqrt[3]{x} + 1\right)}{\sqrt[3]{x^2} - \sqrt[3]{x} + 1}=\sqrt[3]{x} + 1$.
			\item $\sqrt[3]{x^3-3x^2+3x-1}=\sqrt[3]{(x-1)^3}=x-1$.
			\item $-x+5+\sqrt[3]{x^3+3x^2+3x+1}=-x+5+\sqrt[3]{(x+1)^3}=-x+5+x+1=6$.
		\end{enumerate}
	}
\end{bt}

%%%====Bài 9
\begin{bt}%[Dự án EX-9-Đề Cương Toán 9]%[Thầy Nguyễn Đức Tuấn Anh]%[9D3H2-4]
	Chiều cao ngang vai của một con voi đực ở châu Phi là $h$ (cm) có thể được tính xấp xỉ bằng công thức $h=62{,}5\sqrt[3]{t}+75{,}8$ với $t$ là tuổi con voi tính theo năm.
	\begin{enumerate}
		\item Một con voi đực $8$ tuổi thì có chiều cao ngang vai là bao nhiêu cm?
		\item Nếu một con voi đực có chiều cao ngang vai là $205$ cm thì con voi đó bao nhiêu tuổi (làm tròn kết quả đến hàng đơn vị)?
	\end{enumerate}
	\loigiai{
	\begin{enumerate}
		\item Thay $t = 8$ vào biểu thức $h = 62{,}5 \sqrt[3]{t} + 75{,}8$, ta được
		$$h = 62{,}5  \sqrt[3]{8} + 75{,}8 = 62{,}5 \cdot 2 + 75{,}8 = 200{,}8 \text{ (cm)}.$$
		Vậy nếu con voi được $8$ tuổi ở châu Phi thì có chiều cao ngang vai là $200{,}8$ cm.
		\item Theo bài, $h = 205$ (cm), khi đó ta có
		\begin{eqnarray*} 
			205 &=& 62{,}5 \cdot \sqrt[3]{t} + 75{,}8 \\
			62{,}5 \sqrt[3]{t} &=& 129{,}2\\
			\sqrt[3]{t} &=& 2{,}0672\\
			t &=& 2{,}0672^3 \approx 8{,}83.
		\end{eqnarray*}
		Vậy nếu con voi được ở châu Phi có chiều cao ngang vai là $205$ cm thì con voi đó khoảng $9$ tuổi.
	\end{enumerate}
	}
\end{bt}

%%%====Bài 10
\begin{bt}%[Dự án EX-9-Đề Cương Toán 9]%[Thầy Nguyễn Đức Tuấn Anh]%[9D3H4-1]
	Khử mẫu của biểu thức lấy căn
	\begin{multicols}{3}
	\begin{enumerate}
		\item $\sqrt{\dfrac{2a}{3b}}$ với $a > 0$, $b<0$;
		\item $a \sqrt{\dfrac{2}{5a}}$ với $a > 0$;
		\item $4x \sqrt{\dfrac{3}{4xy}}$ với $x > 0$, $y > 0$.
	\end{enumerate}
	\end{multicols}
	\loigiai{
	\begin{enumerate}
		\item $\sqrt{\dfrac{2a}{3b}} = \sqrt{\dfrac{2a \cdot 3b}{3b \cdot 3b}} = \sqrt{\dfrac{6ab}{(3b)^2}} = \dfrac{\sqrt{6ab}}{3|b|}=-\dfrac{\sqrt{6ab}}{3b}$.
		\item $a\sqrt{\dfrac{2}{5a}} = a \cdot \dfrac{\sqrt{2\cdot 5a}}{|5a|} =  a \cdot \dfrac{\sqrt{10a}}{5a} =\dfrac{\sqrt{10a}}{5}$.
		\item $4x\sqrt{\dfrac{3}{4xy}} = 4x\sqrt{\dfrac{3 \cdot 4xy}{4xy \cdot 4xy}} = 4x\sqrt{\dfrac{12xy}{(4xy)^2}} = 4x \cdot \dfrac{\sqrt{12xy}}{4xy} = \dfrac{\sqrt{12xy}}{y}$.
	\end{enumerate}
	}
\end{bt}

%%%====Bài 11
\begin{bt}%[Dự án EX-9-Đề Cương Toán 9]%[Thầy Nguyễn Đức Tuấn Anh]%%[9D3H4-1]
	Trục căn thức ở mẫu các biểu thức sau
	\begin{multicols}{2}
		\begin{enumerate}
			\item $\dfrac{2}{\sqrt{a}}$ với $a > 0$;
			\item $\dfrac{5}{\sqrt{x} + 3}$ với $x \geq 0, x \neq 9$.
			\item $\dfrac{1}{\sqrt{x} - \sqrt{3}}$ với $x \geq 0, x \neq 3$.
			\item $\dfrac{1 - \sqrt{a}}{1 + \sqrt{a}}$ với $a \geq 0, a \neq 1$.
			\item $\dfrac{1}{\sqrt{2x} + \sqrt{2}}$ với $x > 0;$
			\item $\dfrac{1}{\sqrt{x} - 3}$ với $x \geq 0$ và $x \neq 9$.
		\end{enumerate}
	\end{multicols}
	\loigiai{
	\begin{enumerate}
		\item $\dfrac{2}{\sqrt{a}} = \dfrac{2 \cdot \sqrt{a}}{\sqrt{a} \cdot \sqrt{a}} = \dfrac{2 \sqrt{a}}{a}$.
		\item $\dfrac{5}{\sqrt{x} + 3} = \dfrac{5 \cdot (\sqrt{x} - 3)}{(\sqrt{x} - 3) \cdot (\sqrt{x} + 3)} = \dfrac{5(\sqrt{x} - 3)}{x - 9}$.
		\item $\dfrac{1}{\sqrt{x} - \sqrt{3}} = \dfrac{\sqrt{x} + \sqrt{3}}{(\sqrt{x} - \sqrt{3}) \cdot (\sqrt{x} + \sqrt{3})} = \dfrac{\sqrt{x} + \sqrt{3}}{x - 3}$.
		\item $\dfrac{1 - \sqrt{a}}{1 + \sqrt{a}} = \dfrac{(1 - \sqrt{a})^2}{(1 + \sqrt{a})(1 - \sqrt{a})} = \dfrac{1 - 2\sqrt{a} + a}{1 - a}$.
		\item $\dfrac{1}{\sqrt{x} + 1} = \dfrac{\sqrt{x} - 1}{\left( \sqrt{x} + 1 \right)\left( \sqrt{x} - 1 \right)} = \dfrac{\sqrt{x} - 1}{\left(\sqrt{x}\right)^2-1^2 } = \dfrac{\sqrt{x} - 1}{x - 1}$.
		\item $\dfrac{1}{\sqrt{x} - 3} = \dfrac{\sqrt{x} + 3}{(\sqrt{x} - 3)(\sqrt{x} + 3)} = \dfrac{\sqrt{x} + 3}{x - 9}$.
	\end{enumerate}
	}
\end{bt}

%%%====Bài 11
\begin{bt}%[Dự án EX-9-Đề Cương Toán 9]%[Thầy Nguyễn Đức Tuấn Anh]%%[9D3V4-2]
	Rút gọn biểu thức
	\begin{multicols}{2}
	\begin{enumerate}
		\item $\sqrt{\dfrac{(3 - a)^2}{9}}$ với $a > 3$;
		\item $\dfrac{\sqrt{75x^5}}{\sqrt{5x^3}}$ với $x > 0$;
		\item $\sqrt{\dfrac{9}{x^2 - 2x + 1}}$ với $x > 1$;
		\item $\sqrt{\dfrac{x^2 - 4x + 4}{x^2 + 6x + 9}}$ với $x \geq 2$.
	\end{enumerate}
	\end{multicols}
	\loigiai{
	\begin{enumerate}
		\item $\sqrt{\dfrac{(3 - a)^2}{9}} = \dfrac{\sqrt{(3 - a)^2}}{\sqrt{9}} = \dfrac{|3 - a|}{3} = \dfrac{-(3 - a)}{3} = \dfrac{a - 3}{3}$ (vì $a > 3$ nên $3 - a < 0$).
		\item $\dfrac{\sqrt{75x^5}}{\sqrt{5x^3}} = \sqrt{\dfrac{75x^5}{5x^3}} = \sqrt{25x^2} = \sqrt{25} \cdot \sqrt{x^2} = 5 \cdot |x| = 5x$ (vì $x > 0$).
		\item $\sqrt{\dfrac{9}{x^2 - 2x + 1}} = \dfrac{\sqrt{9}}{\sqrt{(x - 1)^2}} = \dfrac{3}{|x - 1|} = \dfrac{3}{x - 1}$ (vì $x > 1$ nên $x - 1 > 0$). 
		\item $ \sqrt{\dfrac{x^2 - 4x + 4}{x^2 + 6x + 9}} = \sqrt{\dfrac{(x - 2)^2}{(x + 3)^2}} = \dfrac{|x - 2|}{|x + 3|} = \dfrac{x - 2}{x + 3}$ (vì $x \geq 2$ nên $x - 2 \geq 0$ và $x + 3 > 0$).
	\end{enumerate}
	}
\end{bt}

\begin{bt}%[Dự án EX-9-Đề Cương Toán 9]%[Thầy Nguyễn Đức Tuấn Anh]%%[9D3V4-2]
	Rút gọn biểu thức $A = \sqrt{x}  \sqrt{x}  \Bigg( \dfrac{1}{\sqrt{x} + 3} - \dfrac{1}{3 - \sqrt{x}} \Bigg)$ với $x \geq 0$ và $x \neq 3$.
	\loigiai{
		\begin{eqnarray*}
			A &=& \sqrt{x}  \Bigg( \dfrac{1}{\sqrt{x} + 3} - \dfrac{1}{3 - \sqrt{x}} \Bigg) \\
			&=& \sqrt{x} \Bigg(\dfrac{3 - \sqrt{x}}{(3 - \sqrt{x})(\sqrt{x} + 3)} - \dfrac{\sqrt{x} + 3}{(3 - \sqrt{x})(\sqrt{x} + 3)}\Bigg) \\
			&=& \sqrt{x} \cdot \dfrac{3 - \sqrt{x} - (\sqrt{x} + 3)}{(3 - \sqrt{x})(\sqrt{x} + 3)} \\
			&=& \sqrt{x} \cdot \dfrac{-2\sqrt{x}}{(3 - \sqrt{x})(\sqrt{x} + 3)} \\
			&=& \dfrac{-2x}{9 - x}.
		\end{eqnarray*}
	}
\end{bt}

\begin{bt}%[Dự án EX-9-Đề Cương Toán 9]%[Thầy Nguyễn Đức Tuấn Anh]%%[9D3C4-2]
	Cho biểu thức $A = \left( \dfrac{1}{x-\sqrt{x}}+\dfrac{\sqrt{x}}{\sqrt{x}-1}\right):\left(\dfrac{2}{x-1}+\dfrac{1}{\sqrt{x}+1}\right)$ với $ x>0$; $x \neq 1$.
	\begin{multicols}{2}
	\begin{enumerate}
		\item Rút gọn biểu thức $A$.
		\item Tính giá trị của biểu thức $A$ khi $x=4$.
		\item Tìm $x$ để $A=2$.
		\item So sánh biểu thức $A$ với $2$.
	\end{enumerate}
	\end{multicols}
	\loigiai{
		\begin{enumerate}
			\item Với $x$ thỏa mãn điều kiện, ta có
			\begin{eqnarray*}
				A &=& \dfrac{1 + x}{\sqrt{x} \left( \sqrt{x} - 1 \right)} : \dfrac{2 + \sqrt{x} - 1}{\left( \sqrt{x} - 1 \right) \left( \sqrt{x} + 1 \right)} \\
				&=& \dfrac{1 + x}{\sqrt{x} \left( \sqrt{x} - 1 \right)} : \dfrac{1}{\left( \sqrt{x} - 1 \right)} \\
				&=& \dfrac{x + 1}{\sqrt{x}}.
			\end{eqnarray*}
			\item Thay $x=4$ vào $A$ ta được $A = \dfrac{4+1}{\sqrt{4}}=\dfrac{5}{2}$.
			\item Ta có
			\begin{eqnarray*}
				A &=& 2 \\
				\dfrac{x+1}{\sqrt{x}}&=&2 \\
				x+1 &=& 2\sqrt{x} \\
				\left(\sqrt{x}-1 \right)^2&=&0 \\
				\sqrt{x} &=& 1 \\
				x&=&1 \text{ (không thỏa mãn điều kiện)}.
			\end{eqnarray*}
			Vậy không tồn tại $x$ thỏa mãn điều kiện.
			\item Ta có $A - 2 = \dfrac{x+1}{\sqrt{x}} - 2 = \dfrac{(\sqrt{x} - 1)^2}{\sqrt{x}}$.\\
			Với $x$ thỏa mãn điều kiện thì $\sqrt{x}\neq 1$ hay $(\sqrt{x}-1)^2 >0$ và $\sqrt{x} >0$. Suy ra $A - 2>0$.\\
			Vậy $A>2$.
		\end{enumerate}
	}
\end{bt}

\begin{bt}%[Dự án EX-9-Đề Cương Toán 9]%[Thầy Nguyễn Đức Tuấn Anh]%%[9D3H4-3]
	Tốc độ gần đúng của một ô tô ngay trước khi đạp phanh được tính theo công thức $v=\sqrt{2 \lambda g d}$, trong đó $v$ (m/s) là tốc độ của ô tô, $d$ (m) là chiều dài của vết trượt tính từ thời điểm đạp phanh cho đến khi ô tô dừng lại trên đường, $\lambda$ là hệ số ma sát của mặt đường, $g=9,8$ m/s$^2$. Nếu một chiếc ô tô để lại vết trượt dài khoảng $20$ m trên đường nhựa thì tốc độ của ô tô trước khi đạp phanh là khoảng bao nhiêu mét trên giây (làm tròn kết quả đến hàng đơn vị)? Biết rằng hệ số ma sát của đường nhựa là $\lambda=0{,}7$.
	\loigiai{
	Theo đề bài, ta có $\lambda = 0{,}7$; $d = 20$ m; $g = 9{,}8$ m/s.\\
	Do đó tốc độ của ô tô đó trước khi đạp phanh là:
	$v = \sqrt{2\lambda gd} = \sqrt{2 \cdot 0{,}7 \cdot 9{,}8 \cdot 20} = \sqrt{274{,}4} \approx 17\ \left( \mathrm{m/s} \right).$
	Vậy tốc độ của ô tô trước khi đạp phanh là khoảng $17\ \mathrm{m/s}$.
	}
\end{bt}
\Closesolutionfile{ans}
\indapan{6}{ans/ans-9C3-OTC}
