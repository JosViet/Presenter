\def\thoigian{90}%--Thời gian
\de{ĐỀ THI GIỮA HỌC KỲ I NĂM HỌC 2024-2025}{THPT TÂN BÌNH  - Tp Hồ Chí Minh}
\begin{center}
	\textbf{PHẦN 1 - Câu trắc nghiệm nhiều phương án lựa chọn.}
\end{center}

\textbf{I. PHẦN TRẮC NGHIỆM: (5,0 điểm) (Học sinh làm bài trên giấy chấm trắc nghiệm)}
%Câu 1
\begin{ex}%[0H4N1-1]%[Dự án đề kiểm tra Toán 10 GHKI NH24-25-ĐÀO HOÀNG VŨ]%[THPT TÂN BÌNH - Tp HCM]
	Cho góc $\alpha$ tù. Điều khẳng định nào sau đây là đúng?
	\choice
	{\True $\cot \alpha < 0$}
	{$\cos \alpha > 0$}
	{$\sin \alpha < 0$}
	{$\tan \alpha > 0$} 
	\loigiai{
		Với $\alpha$ là góc tù thì $\sin \alpha > 0$, $\cos \alpha < 0$, $\tan \alpha < 0$, $\cot \alpha < 0$.
	}
\end{ex}
%%%%%%%%%%%%%%55
\begin{ex}%[0D1N3-4]%[Dự án đề kiểm tra Toán 10 GHKI NH24-25-ĐÀO HOÀNG VŨ]%[THPT TÂN BÌNH - Tp HCM]
	Cho $A$ và $B$ là hai tập hợp bất kì. Phần gạch sọc trong hình vẽ dưới đây là tập hợp nào sau đây
	\begin{center}
		\begin{tikzpicture}[line join=round, line cap=round,>=stealth,thick, scale=.5,rotate=30]
			
			\path (0,0) coordinate (O)
			;
			\draw[pattern=north west lines] (O) ellipse (5 cm and 3 cm);
			\draw[fill=white] (O) ellipse (3 cm and 2 cm);
			\path (-1,.5) node {$B$}
			(-4,-.5) node [fill=white] {$A$}
			;
			
		\end{tikzpicture}
	\end{center}
	\choice
	{$\mathrm{C}_B A$}
	{\True$\mathrm{C}_A B $}
	{$A\cup B$}
	{$A\cap B$}
	\loigiai{
		Phần gạch sọc là phần bù của $B$ trong $A$, kí hiệu $\mathrm{C}_A B$.
	}
\end{ex}
%%%%%%%%%%%%%%%%55
\begin{ex}%[0D1N2-1]%[Dự án đề kiểm tra Toán 10 GHKI NH24-25-ĐÀO HOÀNG VŨ]%[THPT TÂN BÌNH - Tp HCM]
	Hãy liệt kê các phần tử của tập hợp $A=\left\{x\in\mathbb{Z}\mid 2x^2-5x+2=0\right\}$
	\choice
	{$A=\left\{\dfrac{1}{2}\right\}$}
	{$A=\left\{2; \dfrac{1}{2}\right\}$}
	{\True$A=\left\{2\right\}$}
	{$A=\{0\}$}
	\loigiai{
		Giải phương trình $2x^2-5x+2=0$ ta được $x=\dfrac{1}{2}$; $x=2$.\\
		Vì $x\in\mathbb{Z}$ nên $A=\{2\}$.
	}
\end{ex}
%%%%%%%%%%%%%55
\begin{ex}%[0H4N1-1]%[Dự án đề kiểm tra Toán 10 GHKI NH24-25-ĐÀO HOÀNG VŨ]%[THPT TÂN BÌNH - Tp HCM]
	Với mỗi góc $\alpha \left(0^\circ\leq\alpha\leq180^\circ\right)$, xác định môt điểm $M\left(x_0;y_0\right)$ duy nhất nằm trên nửa đường tròn đơn vị sao cho $\widehat{xOM}=\alpha$. Khẳng định nào sau đây là \textbf{sai}?	
	\choice
	{$\sin\alpha=y_0$}
	{$\cos\alpha=x_0$}
	{$\tan\alpha=\dfrac{y_0}{x_0}$ $\left(x_0\neq 0\right)$}
	{\True$\cot\alpha=\dfrac{x_0}{y_0}$}
	\loigiai{
		Khẳng định sai $\cot\alpha=\dfrac{x_0}{y_0}$ do thiếu điều kiện $y_0\neq0$.
	}
\end{ex}
%%%%%%%%%%%5
\begin{ex}%[0D2N2-1]%[Dự án đề kiểm tra Toán 10 GHKI NH24-25-ĐÀO HOÀNG VŨ]%[THPT TÂN BÌNH - Tp HCM]
	Hệ nào sau đây là hệ bất phương trình bậc nhất hai ẩn số?
	\choice
	{$\heva{&2x-y\geq 1\\&3y-x-x^2>0}$}
	{\True$\heva{&2x-y\leq1\\&x+y<2}$}
	{$\heva{&2x+y+z<3\\& y\geq-2}$}
	{$\heva{&x+y\leq2\\&y^2\geq4}$}
	\loigiai{
		Hệ bất phương trình bậc nhất hai ẩn là $\heva{&2x-y\leq1\\&x+y<2}$.
	}
\end{ex}
%%%%%%%%%%%%%%%%%%%%%
\begin{ex}%[0D1N3-1]%[Dự án đề kiểm tra Toán 10 GHKI NH24-25-ĐÀO HOÀNG VŨ]%[THPT TÂN BÌNH - Tp HCM]
	Cho tập hợp $X=\{1;5\}$, $Y=\{1;3;5\}$. Tập $X\cap Y$ là tập hợp nào sau đây?
	\choice
	{$\{1\}$}
	{\True$\{1;5\}$}
	{$\{1;3;5\}$}
	{$\{1;3\}$}
	\loigiai{
		$X\cap Y=\{1;5\}$
	}
\end{ex}
%%%%%%%%%%%%%%%
\begin{ex}%[0D1H1-1]%[Dự án đề kiểm tra Toán 10 GHKI NH24-25-ĐÀO HOÀNG VŨ]%[THPT TÂN BÌNH - Tp HCM]
	Trong các câu sau có bao nhiêu câu là mệnh đề?\\
	(1)$\colon$ Số $3$ là một số chẵn.\\
	(2)$\colon$ $2x+1=3$.\\
	(3)$\colon$ Các em hãy cố gắng làm bài thi cho tốt nhé!\\
	(4)$\colon$ $1<5\Rightarrow 8<6$.
	\choice
	{\True$2$}
	{$3$}
	{$1$}
	{$4$}
	\loigiai{
		Các ý (1) và (4) là mệnh đề.}
\end{ex}
%%%%%%%%%%%%%%%%%5
\begin{ex}%[0D2N1-1]%[Dự án đề kiểm tra Toán 10 GHKI NH24-25-ĐÀO HOÀNG VŨ]%[THPT TÂN BÌNH - Tp HCM]
	Cặp số nào sau đây \textbf{không} là nghiệm của bất phương trình $x+2y-3>0$?
	\choice
	{$(-2;3)$}
	{$(-1;4)$}
	{\True$(-2;0)$}
	{$(4;0)$}
	\loigiai{
		Thay $x=-2$, $y=0$ vào bất phương trình $x+2y-3>0$ ta được $-2+2\cdot0-3=-5>0$ (vô lý).\\
		Do đó cặp số $(-2;0)$ không là nghiệm của bất phương trình trên.
	}
\end{ex}
%%%%%%%%%%%%%%
\begin{ex}%[0D1H1-2]%[Dự án đề kiểm tra Toán 10 GHKI NH24-25-ĐÀO HOÀNG VŨ]%[THPT TÂN BÌNH - Tp HCM]
	Mệnh đề nào sau đây \textbf{sai}?\\
	(1)$\colon$ $\varnothing\in\{0\}$.\\
	(2)$\colon$ $\{1\}\subset \{0;1;3\}$.\\
	(3)$\colon$ $\{0\}=\varnothing$.\\
	(4)$\colon$ $\{0\}\subset\left\{x\mid x^2-x=0\right\}$.
	\choice
	{(1), (4)}
	{(2), (3)}
	{\True (1), (3)}
	{(2), (4)}
	\loigiai{
		Các ý sai là (1), (3).
	}
\end{ex}
%%%%%%%%%%%%%%%%%
\begin{ex}%[0D1N2-2]%[Dự án đề kiểm tra Toán 10 GHKI NH24-25-ĐÀO HOÀNG VŨ]%[THPT TÂN BÌNH - Tp HCM]
	Cho tập hợp $A=\{0;1\}$. Tập hợp $A$ có bao nhiêu tập hợp con?
	\choice
	{$6$}
	{\True$4$}
	{$3$}
	{$2$}
	\loigiai{Số tập hợp con của tập hợp $A$ bằng $2^2=4$.}
\end{ex}




\begin{ex}%[0H4N2-2][Dự án tex hóa đề kiểm tra GHKI NH24-25- Ngô Tất Thành]%[THPT - Tân Bình HCM]
	Cho $\triangle ABC$, gọi $r$, $R$ lần lượt là bán kính đường tròn nội tiếp và ngoại tiếp $\triangle ABC$, $p$ là nửa chu vi. Đặt $AB = c$, $AC = b$, $BC = a$. Diện tích $\triangle ABC$ được tính theo công thức nào?
	\choice
	{\True $S = p \cdot r$}
	{$S = \dfrac{abc}{4r}$}
	{$S = \dfrac{1}{2} a c \sin A$}
	{$S = \sqrt{(p-a)(p-b)(p-c)}$}
	\loigiai{ 
		Công thức diện tích tam giác bằng tích của nửa chu vi và bán kính đường tròn nội tiếp $S = p \cdot r$.
	}
\end{ex}

% Câu 12
\begin{ex}%[0D1N1-1][Dự án tex hóa đề kiểm tra GHKI NH24-25- Ngô Tất Thành]%[THPT - Tân Bình HCM]
	Cho mệnh đề chứa biến \lq\lq$P(n) \colon 3 - n > 0$\rq\rq \, với $n$ là số tự nhiên. Mệnh đề nào sau đây đúng?
	\choice
	{\True $P(2)$}
	{$P(4)$}
	{$P(5)$}
	{$P(3)$}
	\loigiai{ 
		Thay $n = 2$ vào $P(n)\colon 3 - 2 > 0 \Rightarrow$ $P(2)$ đúng.
	}
\end{ex}

% Câu 13
\begin{ex}%[0H4N2-1][Dự án tex hóa đề kiểm tra GHKI NH24-25- Ngô Tất Thành]%[THPT - Tân Bình HCM]
	Cho tam giác $ABC$. Gọi $R$ là bán kính đường tròn ngoại tiếp tam giác $ABC$. Đặt $AB = c$, $AC = b$, $BC = a$. Mệnh đề nào sau đây đúng?
	\choice
	{$a = R \sin A$}
	{$b = R \sin B$}
	{\True $c = 2R \sin C$}
	{$c = R \sin C$}
	\loigiai{ 
		Công thức $c = 2R \sin C$ là công thức đúng.
	}
\end{ex}

% Câu 14
\begin{ex}%[0H4N2-1][Dự án tex hóa đề kiểm tra GHKI NH24-25- Ngô Tất Thành]%[THPT - Tân Bình HCM]
	Cho tam giác $ABC$ có $BC = a$, $AC = b$, $AB = c$. Mệnh đề nào sau đây đúng?
	\choice
	{$\cos A = \dfrac{b^2 + c^2 + a^2}{bc}$}
	{\True $\cos A = \dfrac{b^2 + c^2 - a^2}{2bc}$}
	{$\cos A = \dfrac{b^2 + c^2 + a^2}{2bc}$}
	{$\cos A = \dfrac{b^2 + c^2 - a^2}{bc}$}
	\loigiai{ 
		Công thức đúng của $\cos A$ trong tam giác là $\cos A = \dfrac{b^2 + c^2 - a^2}{2bc}$.
	}
\end{ex}

% Câu 15
\begin{ex}%[0D1H3-4][Dự án tex hóa đề kiểm tra GHKI NH24-25- Ngô Tất Thành]%[THPT - Tân Bình HCM]
	Cho $A = [-7;7]$, $B = (-7;7)$. Trong các mệnh đề sau, mệnh đề nào \textbf{sai}?
	\choice
	{$A \cup B = [-7;7]$}
	{$A \cap B = (-7;7)$}
	{$B \subset A$}
	{\True $A \setminus B = \{7\}$}
	\loigiai{ 
		Mệnh đề $A \setminus B = \{-7;7\}$ nên $A \setminus B = \{7\}$ sai.
	}
\end{ex}

% Câu 16
\begin{ex}%[0D1H1-4][Dự án tex hóa đề kiểm tra GHKI NH24-25- Ngô Tất Thành]%[THPT - Tân Bình HCM]
	Biết rằng $P \Rightarrow Q$ là mệnh đề đúng. Mệnh đề nào sau đây đúng?
	\choice
	{\True $Q$ là điều kiện cần để có $P$}
	{$Q$ là điều kiện đủ để có $P$}
	{$P$ là điều kiện cần để có $Q$}
	{$Q$ là điều kiện cần và đủ để có $P$}
	\loigiai{ 
		Mệnh đề $Q$ là điều kiện cần và đủ để có $P$ là phát biểu đúng nhất trong trường hợp $P \Rightarrow Q$ đúng.
	}
\end{ex}

% Câu 17
\begin{ex}%[0H4N2-1][Dự án tex hóa đề kiểm tra GHKI NH24-25- Ngô Tất Thành]%[THPT - Tân Bình HCM]
	Cho tam giác $ABC$ có $AB = 3$, $AC = 5$, $BC = 7$. Số đo góc $\widehat A$ bằng
	\choice
	{$60^\circ$}
	{$90^\circ$}
	{$150^\circ$}
	{\True $120^\circ$}
	\loigiai{ 
		Sử dụng định lý cosin ta có $\cos A = \dfrac{AC^2 + AB^2 - BC^2}{2 AC \cdot AB}=-\dfrac{1}{2} \Rightarrow \widehat A = 120^\circ$.
	}
\end{ex}

% Câu 18
\begin{ex}%[0H4N1-3][Dự án tex hóa đề kiểm tra GHKI NH24-25- Ngô Tất Thành]%[THPT - Tân Bình HCM]
	Cho tam giác $ABC$. Khẳng định nào sau đây là đúng?
	\choice
	{$\tan A = \tan (B + C)$}
	{$\cos A = \cos (B + C)$}
	{$\cot A = \cot (B + C)$}
	{\True $\sin A = \sin (B + C)$}
	\loigiai{ 
		Tổng ba góc trong tam giác bằng $180^\circ$, do đó $\sin A = \sin (B + C)$ là mệnh đề đúng.
	}
\end{ex}

% Câu 19
\begin{ex}%[0D1H2-2][Dự án tex hóa đề kiểm tra GHKI NH24-25- Ngô Tất Thành]%[THPT - Tân Bình HCM]
	Cho hai tập hợp $A = \{1;2;3;4;5\}$, $B = \{1;3;5;7;9\}$. Có tất cả bao nhiêu tập hợp $X$ thỏa mãn $X \subset A$, $X \subset B$?
	\choice
	{$5$}
	{$7$}
	{\True $8$}
	{$6$}
	\loigiai{ 
		$X$ là giao của $A$ và $B$, do đó $X = \{1;3;5\}$. Số tập con của tập hợp này là $2^3 = 8$.
	}
\end{ex}

% Câu 20
\begin{ex}%[0H4H2-1][Dự án tex hóa đề kiểm tra GHKI NH24-25- Ngô Tất Thành]%[THPT - Tân Bình HCM]
	Cho tam giác $ABC$ có $\widehat B = 30^\circ$, $\widehat C = 45^\circ$ và $AB = 5$. Tính độ dài $AC$.
	\choice
	{$AC = 5\sqrt{2}$}
	{$AC = \dfrac{5\sqrt{3}}{2}$}
	{$AC = \dfrac{5\sqrt{6}}{2}$}
	{\True $AC = \dfrac{5\sqrt{2}}{2}$}
	\loigiai{ 
		Sử dụng định lý sin trong tam giác ta có\\
		$AC = \dfrac{AB \cdot \sin B}{\sin C} = \dfrac{5 \cdot \sin 30^\circ}{\sin 45^\circ} \Rightarrow AC = \dfrac{5\sqrt{2}}{2}$.
	}
\end{ex}





\textbf{PHẦN II. (2 Điểm) Câu trắc nghiệm đúng sai. Trong mỗi ý a,b,c,d ở mỗi câu, thí sinh chọn đúng hoặc sai}

\setcounter{ex}{0} 

\Opensolutionfile{ans}[ans-DS]

\begin{ex}%[0D1H3-4]%[KNTT - Lớp 10 - Ôn tập giữa học kì 1 - Đề 1]%[Dương Văn Đức]
	Cho hai tập hợp $A=\left\lbrace-1;0;1;2 \right\rbrace $ và $B=\left\lbrace x\in \mathbb{R}|x-1\geq 0\right\rbrace $. Các phát biểu sau đây \textbf{đúng} hay \textbf{sai}?
	\choiceTF
	{Tập hợp $A=\left\lbrace x\in\mathbb{N}|-1\leq x\leq 2 \right\rbrace $}
	{Tập hợp $B=\left(1;+\infty \right)$ và $B\subset \mathbb{R}$}
	{\True $A\cap B=\left\lbrace 1;2\right\rbrace $}
	{\True Tập hợp $C_{\mathbb{R}}\left(B\setminus A \right)=\left(-\infty;1 \right]\cup \left\lbrace 2 \right\rbrace$}
	\loigiai{
		\begin{itemchoice}
			\itemch \textbf{Sai}. Tập hợp $A=\left\lbrace-1;0;1;2 \right\rbrace$ hay  $A=\left\lbrace x\in\mathbb{Z}|-1\leq x\leq 2 \right\rbrace$.
			\itemch \textbf{Sai}. Tập hợp $B=\left\lbrace x\in \mathbb{R}|x-1\geq 0\right\rbrace$ hay $B=\left[1;+\infty \right)$ và $B\subset \mathbb{R}$.
			\itemch \textbf{Đúng}. Ta có $A=\left\lbrace-1;0;1;2 \right\rbrace$ và $B=\left[1;+\infty \right)$ nên $A\cap B=\left\lbrace 1;2\right\rbrace$.
			\itemch \textbf{Đúng}. Ta có $A=\left\lbrace-1;0;1;2 \right\rbrace$ và $B=\left[1;+\infty \right)$ nên $B \setminus A=\left(1;+\infty\right)\setminus \left\lbrace 2\right\rbrace $.\\
			Khi đó, $C_{\mathbb{R}}\left(B\setminus A \right)=\left(-\infty;1\right]\cup \left\lbrace 2\right\rbrace $.
		\end{itemchoice}
	}
\end{ex}

\begin{ex}%[0D2V2-2]%[Dự án đề kiểm tra Toán khối 10 GHKI NH24-25-Dot4-Khắc Thiên]%[THPT Tân Bình - TP HCM]
	Bác Minh dự định quy hoạch $x$ sào đất trồng cà tím và $y$ sào đất trồng cà chua. Bác chỉ có không quá $9$ triệu đồng để mua hạt giống. Cho biết tiền mua hạt giống cà tím là $200\,000$ đồng/sào và $100\,000$ đồng/sào. Các mệnh đề sau \textbf{đúng} hay \textbf{sai}?
	\choiceTF
	{\True Gọi $x$ là số sào đất trồng cà tím và $y$ là số sào đất trồng cà chua. Ta có hệ bất phương trình
		$\heva{&2x+y\leq 90\\&x\geq 0\\&y\geq0}$}
	{\True $\left(2;1\right)$ là một nghiệm của bất phương trình trên}
	{Miền nghiệm của hệ bất phương trình trên là một tứ giác}
	{\True Biểu thức $L=3x-2y$ đạt giá trị lớn nhất là $M$ và giá trị nhỏ nhất là $m$. Khi đó $M-m=315$}
	\loigiai{
		\begin{itemchoice}
			\itemch \textbf{Đúng}. Gọi $x$ là số sào đất trồng cà tím và $y$ là số sào đất trồng cà chua. Ta có hệ bất phương trình
			$\heva{&200\,000x+100\,000y\leq 9\,000\,000\\&x\geq 0\\&y\geq0}$ hay $\heva{&2x+y\leq 90\\&x\geq 0\\&y\geq0}$.
			\itemch \textbf{Đúng}. Thay $\left(2;1\right)$ vào hệ bất phương trình $\heva{&2x+y\leq 90\\&x\geq 0\\&y\geq0}$ ta thấy thoả mãn nên $\left(2;1\right)$ là một nghiệm của hệ bất phương trình trên.
			\itemch \textbf{Sai}.
			\begin{center}
				\begin{tikzpicture}[scale=1, font=\footnotesize, line join=round, line cap=round, >=stealth]
					\draw[->] (-1,0) -- (3.3,0)node[above]{$x$};
					\draw[->,color=black] (0,-1) -- (0,5.3)node[right]{$y$};
					\clip(-1,-1) rectangle (3,5);
					\fill[pattern=north east lines] (-5,-4) -- (-5,5) -- (5,5) -- (5,-4) -- cycle;
					\fill[white] (0,0) -- (2,0) -- (0,4)--cycle;
					\draw[violet,line width=2pt,samples=100] (0,0) -- (2,0) -- (0,4)--cycle;
					
					%\foreach \x in {-2,2,4}
					\draw[shift={(0,2)},color=black] (0pt,2pt) -- (0pt,-2pt) node[left] {$45$};
					\draw[shift={(0,4)},color=black] (0pt,2pt) -- (0pt,-2pt) node[left] {$90$};
					\draw[shift={(2,0)},color=black] (0pt,2pt) -- (0pt,-2pt) node[below] {$45$};
					

					\node[below left] at (0,0){$O$};

					

				\end{tikzpicture}
			\end{center} Miền nghiệm của hệ bất phương trình $\heva{&2x+y\leq 90\\&x\geq 0\\&y\geq0}$ là một tam giác.
			\itemch \textbf{Đúng}. Ta có $L(x;y)=3x-2y$.\\Khi đó, ta có
			\begin{itemize}
				\item $L(0;0)=3\cdot0-2\cdot0=0$.
				\item $L(45;0)=3\cdot45-2\cdot0=135$.
				\item $L(0;90)=3\cdot0-2\cdot90=-180$.
			\end{itemize}
			Do đó $M=135$ và $m=-180$.\\ Vậy $M-m=135-(-180)=135+180=315$.
		\end{itemchoice}
	}
\end{ex}
\Closesolutionfile{ans}



\textbf{PHẦN III. (3 Điểm) Học sinh làm trên giấy tự luận}
\setcounter{ex}{0}
\begin{ex}%[0H4H2-2]%[Dự án đề kiểm tra giữa kì khối 10 HKI NH24-25-Dot 4-TheHung Nguyen]%[THPT Tân Bình-TpHCM]
	Cho $\triangle ABC$ có $AB=8$, $AC=5$, $\widehat{BAC}=60^{\circ}$. Tính chiều cao $AH$ của tam giác $ABC$ (kết quả làm tròn đến $1$ chữ số thập phân).
	\loigiai{
		\immini{Theo định lý hàm số cos ta có
		\begin{eqnarray*}
			BC&= & \sqrt{AB^2+AC^2-2AB\cdot AC\cos \widehat{BAC}}\\
			&= & \sqrt{8^2+5^2-2\cdot 8\cdot 5\cos 60^\circ}\\
			&= & \sqrt{64+25-40} = \sqrt{49} = 7.
		\end{eqnarray*}
		Ta có $S_{\triangle ABC}=\dfrac{1}{2}AB\cdot AC\sin \widehat{BAC}=\dfrac{1}{2}\cdot 8\cdot5\sin 60^\circ=10\sqrt{3}$.\\
		Ta có $ S_{\triangle ABC}=\dfrac{1}{2}AH\cdot BC\Rightarrow AH=\dfrac{2\cdot S_{\triangle ABC}}{BC}=\dfrac{2\cdot 10\sqrt{3}}{7}\approx 4{,}9$.
	}{\begin{tikzpicture}[line join = round, line cap=round,>=stealth,font=\footnotesize,scale=.65]
				\path  (0,0)coordinate (A)--(0:5)coordinate (C)
				(60:8)coordinate (B)
				($(B)!(A)!(C)$)coordinate (H);
				\draw (A)--(B)--(C)--(A)--(H);
				\foreach \y/\g in {A/180,B/90,C/0,H/10}  \fill (\y) circle (1pt)+(\g:.5)node {$\y$};
				\draw pic [draw,angle radius=.35cm]  {right angle = B--H--A}
				pic [draw,angle radius=.35cm]  {angle = C--A--B}
				;
		\end{tikzpicture}}
	}
\end{ex}

\begin{ex}%[0D1H3-5] %[Dự án đề kiểm tra giữa kì khối 10 HKI NH24-25-Dot 4-TheHung Nguyen]%[THPT Tân Bình-TpHCM]
	Trong một lớp học có $40$ học sinh, trong đó có $30$ học sinh đạt học sinh giỏi môn Toán, $25$ học sinh đạt học sinh giỏi môn Văn. Biết rằng chỉ có $5$ học sinh không đạt học sinh giỏi môn nào trong cả hai môn Toán và Văn. Hỏi có bao nhiêu học sinh đạt học sinh giỏi cả hai môn Văn và Toán?
	\loigiai{
	Gọi $A$ là tập hợp các học sinh đạt học sinh giỏi môn Toán, suy ra $n(A)=30$.\\
	$B$ là tập hợp các học sinh đạt học sinh giỏi môn Văn, suy ra $n(B)=25$.\\
	$A\cup B$ là tập hợp học sinh giỏi môn Toán, Văn của lớp.\\ Do đó
	$n(A\cup B)=40-5=35$ (học sinh).\\
	$A\cap B$ là tập hợp các học sinh đạt học sinh giỏi cả hai môn Toán và Văn.\\
	Do đó  $n(A\cap B)=n(A)+n(B)-n(A\cup B)=30+25-35=20$ (học sinh).
	}
\end{ex}

\begin{ex}%[0H4H3-2] %[Dự án đề kiểm tra giữa kì khối 10 HKI NH24-25-Dot 4-TheHung Nguyen]%[THPT Tân Bình-TpHCM]
	\immini{Người ta cần lắp đặt một thiết bị chiếu sáng gắn trên tường cho một phòng triển lãm. Thiết bị này có góc chiếu sáng là $30^{\circ}$ và cần đặt cao hơn mặt đất là $3{,}5$ m. Người ta đặt thiết bị này sát tường và canh chỉnh sao cho trên mặt đất dải ánh sáng bắt đầu từ vị trí cách tường $3$ m (tham khảo hình vẽ). Độ dài vùng được chiếu sáng trên mặt đất bằng bao nhiêu m? (Làm tròn kết quả đến hàng phần mười)}{\begin{tikzpicture}[line join = round, line cap=round,>=stealth,font=\footnotesize,scale=.65]
			\pgfmathsetmacro{\g}{atan(6/7)}
			\pgfmathsetmacro{\d}{90-30-\g}
			\pgfmathsetmacro{\c}{90-\d}
			\pgfmathsetmacro{\bd}{3.5/tan(\d)}
			\pgfmathsetmacro{\bc}{\bd-3}

			\path  (0,0)coordinate (D)--(0:\bc)coordinate (C)
			(0:\bd) coordinate (B)--++(90:3.5) coordinate (A);
			\foreach \y/\g in {A/90,B/-90,C/-90,D/-90}  \fill (\y) circle (1pt)+(\g:.5)node {$\y$};

			\draw (C)--node[midway,above,sloped,scale=.65]{$3$ m}(B) (D)--(B)--node[midway,above,sloped,scale=.65]{$3{,}5$ m}(A)--(D)--node[midway,below,sloped]{$?$ m}(C)--(A);
			\draw pic [draw,angle radius=.3cm]  {right angle = C--B--A}
			pic [draw,angle radius=.75cm]  {angle = D--A--C}
			;

			\fill[violet!20!yellow ] (A)--(C)--(D);
			\draw ($(B)+(1,-.5)$) node[right,rotate=90]{$\text{Đèn chiếu sáng}$};
			\draw ($(D)+(5,-.1)$) node[above]{$\text{Dải ánh sáng}$};
			\draw ($(A)+(-.55,-.45)$) node[scale=.65]{$30^\circ$};
	\end{tikzpicture}}
	 \loigiai{Xét $\triangle ABC$ vuông tại $B$, ta có
	 	$\tan \widehat{ACB}=\dfrac{3{,}5}{3}\Rightarrow \widehat{ACB}=\arctan \dfrac{7}{6}$.\\
	 	Ta có $\widehat{D}=\widehat{ACB}-\widehat{DAC}\approx  19{,}3987^\circ$.\\
	 	Xét $\triangle ABD$ vuông tại $B$, có \\
	 	$\tan \widehat{ADB}=\dfrac{AB}{BD}\Rightarrow BD=\dfrac{AB}{\tan \widehat{ADB}}=\dfrac{3{,}5}{\tan 19{,}3987^\circ }\approx 9{,}9395$.\\
	 	Do đó độ dài vùng chiếu sáng trên mặt đất là $DC=DB-CD=9{,}9395-3\approx 6{,}9$ m.
	 }
\end{ex}








