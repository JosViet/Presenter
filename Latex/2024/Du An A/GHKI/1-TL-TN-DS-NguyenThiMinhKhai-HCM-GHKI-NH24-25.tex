\def\thoigian{90}%--Thời gian
\de{ĐỀ THI GIỮA HỌC KỲ I NĂM HỌC 2024-2025}{THPT Nguyễn Thị Minh Khai - Tp Hồ Chí Minh}



\begin{center}
	\textbf{PHẦN 1 - TRẮC NGHIỆM}
\end{center}
\Opensolutionfile{ans}[ans/ans]

%Câu 1
\begin{ex}%[1D1N1-3]%[Dự án đề kiểm tra Toán 11 GHKI NH24-25-Dot3- Thanh Phong]%[THPT Nguyễn Thị Minh Khai - Tp HCM]
	\immini{Cho một góc lượng giác $\left(OM,ON\right)$ có số đo được cho bởi hình bên. Công thức tổng quát của số đo góc lượng giác $\left(OM,ON\right)$ là}
	{\begin{tikzpicture}[scale=0.77, font=\footnotesize, line join=round, line cap=round, >=stealth]
		\draw (0,0) -- (4,0)node[below] {$M$};
		\draw[red] (0,0) -- (60:4)node[below right] {$N$};
		\draw (0:.6) arc (0:60:.6);
		\path (30:1) node{$60^\circ$};
		\path (0:0) node[below left]{$O$};
		\end{tikzpicture}}
\choice
	{$\left(OM,ON\right)=60^{\circ}$}
	{$\left(OM,ON\right)=60^{\circ}+k\cdot 180^{\circ}$\,$(k \in \mathbb{Z})$}
	{\True $\left(OM,ON\right)=60^{\circ}+k\cdot 360^{\circ}$\,$(k \in \mathbb{Z})$}
	{$\left(OM,ON\right)=-60^{\circ}+k\cdot360^{\circ}$\,$(k \in \mathbb{Z})$}
\loigiai{
	Góc lượng giác tia đầu $OM$, tia cuối $ON$ có số đo là $\left(OM,ON\right)=60^{\circ}+k\cdot 360^{\circ}$\,$(k \in \mathbb{Z})$.
}
\end{ex}
\begin{ex}%[1D1N1-5]%[Dự án đề kiểm tra Toán 11 GHKI NH24-25-Dot3- Thanh Phong]%[THPT Nguyễn Thị Minh Khai - Tp HCM]
	Trên đường tròn lượng giác, gọi $M\left(x_0 ; y_0\right)$ là điểm biểu diễn của góc lượng giác có số đo $\alpha$. Khẳng định nào sau đây là đúng?
	\choice
	{\True $\cos\alpha=x_0$}
	{$\cos\alpha=y_0$}
	{$\cos\alpha=\dfrac{x_0}{y_0}$, với $y_0 \neq 0$}
	{$\cos\alpha=\dfrac{y_0}{x_0}$, với $x_0 \neq 0$}
	\loigiai{
	\immini{Dựa vào đường tròn lượng giác, ta có $\cos\alpha=x_0$.}
	{\begin{tikzpicture}[line join = round, line cap = round, >=stealth, font=\footnotesize, scale=0.77]
	\tikzset{label style/.style={font=\footnotesize}}
	\path (0,0) coordinate (O)
	(3,0) coordinate (A)
	(0,3) coordinate (B)
	(0,-3) coordinate (B')
	(-3,0) coordinate (A')
	(0:0)++(40:3) coordinate (M)
	($(O)!(M)!(A')$) coordinate (H)
	($(O)!(M)!(B)$) coordinate (K)
	;
	\draw[->] (-4,0) -- (4,0) node[above,blue]{$x$};
	\draw[->] (0,-4) -- (0.,4) node[left,blue]{$y$};
	\draw[orange] (O) circle (3cm);
	\draw[rotate=0,->,green!50!black] (0.7,0) arc (0:40:0.7cm);
	\draw (1,0) node[above,blue] {$\alpha$};
	\draw[dashed] (H)--(M)--(K);
	\draw[green!50!black] (M)--(O);
	\draw[blue,fill=black] (0,2) node[left]{$\sin\alpha$}(2,0) circle(1pt) node[below]{$\cos\alpha$}(3,2.3) node{$M(x_0;y_0)$};
	\foreach \p/\r in {A/-45,O/-135,A'/-135,B'/-45,B/45}
	\fill (\p) circle (1pt) node[shift={(\r:3mm)},blue]{$\p$};
	\end{tikzpicture}}	
	}
\end{ex}
\begin{ex}%[1D1N3-4]%[Dự án đề kiểm tra Toán 11 GHKI NH24-25-Dot3- Thanh Phong]%[THPT Nguyễn Thị Minh Khai - Tp HCM]
	Với mọi góc lượng giác $a$ và $b$, khẳng định nào sau đây là \textbf{sai}?
	\choice
	{$\cos a+\cos b=2\cos \dfrac{a+b}{2}\cdot\cos \dfrac{a-b}{2}$}
	{\True $\cos a-\cos b=2\sin \dfrac{a+b}{2}\cdot\sin \dfrac{a-b}{2}$}
	{$\sin a+\sin b=2\sin \dfrac{a+b}{2}\cdot\cos \dfrac{a-b}{2}$}
	{$\sin a-\sin b=2\cos \dfrac{a+b}{2}\cdot\sin \dfrac{a-b}{2}$}
	\loigiai{
	Ta có $\cos a-\cos b=-2\sin \dfrac{a+b}{2}\cdot\sin \dfrac{a-b}{2}$	
	}
\end{ex}
\begin{ex}%[1D1H2-4]%[Dự án đề kiểm tra Toán 11 GHKI NH24-25-Dot3- Thanh Phong]%[THPT Nguyễn Thị Minh Khai - Tp HCM]
	Cho biết $\cos \alpha=-\dfrac{1}{4}$ và $\dfrac{\pi}{2}<\alpha<\pi$, khi đó $\cos \left(\dfrac{\pi}{2}-2\alpha\right)$ có giá trị bằng
	\choice
	{$-\dfrac{\sqrt{15}}{16}$}
	{\True $-\dfrac{\sqrt{15}}{8}$}
	{$\dfrac{\sqrt{15}}{16}$}
	{$\dfrac{\sqrt{15}}{8}$}
	\loigiai{
	Ta có $\sin^2\alpha =1-\cos^2\alpha=\dfrac{15}{16}$.\\
	Vì $\dfrac{\pi}{2}<\alpha<\pi$ nên $\sin\alpha=\dfrac{\sqrt{15}}{4}$.\\
	Mà $\cos \left(\dfrac{\pi}{2}-2\alpha\right)=\sin 2\alpha=2\cdot \sin\alpha\cdot \cos\alpha=2\cdot\dfrac{\sqrt{15}}{4} \left(-\dfrac{1}{4}\right)=-\dfrac{\sqrt{15}}{8}$.	
	}
\end{ex}
\begin{ex}%[1D1N4-5]%[Dự án đề kiểm tra Toán 11 GHKI NH24-25-Dot3- Thanh Phong]%[THPT Nguyễn Thị Minh Khai - Tp HCM]
	Hàm số $y=\tan x$ tuần hoàn với chu kì $T$ nào sau đây?
	\choice
	{$T=4\pi$}
	{$T=3\pi$}
	{$T=2\pi$}
	{\True $T=\pi$}
	\loigiai{
		Hàm số $y=\tan\left(ax+b\right)$ tuần hoàn với chu kì $T=\dfrac{\pi}{|a|}$.\\
		Vậy hàm số $y=\tan x$ tuần hoàn với chu kì $T=\pi$.	
	}
\end{ex}
\begin{ex}%[1D1N4-2]%[Dự án đề kiểm tra Toán 11 GHKI NH24-25-Dot3- Thanh Phong]%[THPT Nguyễn Thị Minh Khai - Tp HCM]
	Tập xác định của hàm số $y=\dfrac{1}{\sin 3 x}$ là
	\choice
	{$\left\{x \in \mathbb{R}\ \middle|\ x \neq k\pi,k \in \mathbb{Z}\right\}$}
	{\True $\left\{x \in \mathbb{R}\ \middle|\ x \neq \dfrac{k\pi}{3},k \in \mathbb{Z}\right\}$}
	{$\left\{x \in \mathbb{R}\ \middle|\ x \neq \dfrac{k2\pi}{3},k \in \mathbb{Z}\right\}$}
	{$\left\{x\in\mathbb{R}\ \middle|\ x\neq k2\pi, k\in\mathbb{Z} \right\}$}
	\loigiai{
	Hàm số xác định khi $\sin 3x\neq 0\Leftrightarrow 3x\neq k\pi\Leftrightarrow x\neq \dfrac{k\pi}{3}$, $k\in\mathbb{Z}$.	
	}
\end{ex}
\begin{ex}%[1D2H1-3]%[Dự án đề kiểm tra Toán 11 GHKI NH24-25-Dot3- Thanh Phong]%[THPT Nguyễn Thị Minh Khai - Tp HCM]
	Cho dãy số $\left(u_n\right)$ được xác định bởi $\heva{&u_1=-2\\&u_{n+1}=3u_n-1\,(n \geq 1)}$. Khi đó $u_2$ có giá trị bằng
	\choice
	{$-4$}
	{$-5$}
	{$-6$}
	{\True $-7$}
	\loigiai{
	Ta có $u_2=3\cdot u_1-1=3\cdot (-2)-1=-7$.	
	}
\end{ex}
\begin{ex}%[1D2H1-4]%[Dự án đề kiểm tra Toán 11 GHKI NH24-25-Dot3- Thanh Phong]%[THPT Nguyễn Thị Minh Khai - Tp HCM]
	Dãy số nào sau đây là dãy số tăng?
	\choice
	{Dãy số $\left(a_n\right)$ với $a_n=\dfrac{1}{n}$}
	{\True Dãy số $\left(b_n\right)$ với $b_n=2n+1$}
	{Dãy số $\left(c_n\right)$ với $c_n=-n^2$}
	{Dãy số $\left(d_n\right)$ với $d_n=\dfrac{1}{2^n}$}
	\loigiai{
	Ta có $b_{n+1}-b_n=2\left(n+1\right)+1-2n-1=2>0$.\\
	Vậy $\left(b_n\right)$ là dãy số tăng.	
	}
\end{ex}
\Closesolutionfile{ans}
%\begin{center}
%	\textbf{ĐÁP ÁN}
%	\inputansbox{10}{ans/ans}	
%\end{center}



\begin{center}
	\textbf{PHẦN 2 - ĐÚNG SAI}
\end{center}

\Opensolutionfile{ans}[ans/answer]
\begin{ex}%[1D1V5-6]%[Dự án đề kiểm tra Toán 11 GHKI NH24-25-Dot3-Hieu Hieu Minh Minh]%[THPT Nguyễn Thị Minh Khai - Tp HCM]
	Huyết áp là áp lực máu cần thiết tác động lên thành động mạch nhằm đưa máu đi nuôi dưỡng các mô trong cơ thể. Nhờ lực co bóp của tim và sức cản của động mạch mà huyết áp được tạo ra. Giả sử huyết áp của một người thay đổi theo thời gian được cho bởi công thức: $p(t)=120+15 \cos 150 \pi t$, trong đó $p(t)$ là huyết áp tính theo đơn vị mmHg (milimét thuỷ ngân) và thời gian $t$ tính theo đơn vị phút. Huyết áp cao nhất được gọi là huyết áp tâm thu và huyết áp thấp nhất được gọi là huyết áp tâm trương.
	\choiceTF
	{\True $-1 \leq \cos 150 \pi t \leq 1, \forall t \geq 0$}
	{$p(2)=120 \mathrm{mmHg}$}
	{\True Huyết áp tâm thu của người đó là $135$ mmHg và huyết áp tâm trương của người đó là $105$ mmHg}
	{\True Trong thời gian từ $0$ giây đến $30$ giây, huyết áp của một người có $75$ lần đạt $120$ mmHg}
	\loigiai{
		\begin{itemchoice}
			\itemch \textbf{Đúng}. Vì $-1 \leq \cos 150 \pi t \leq 1, \forall t \geq 0$.
			\itemch \textbf{Sai}. \\
			Thay $t=2$ vào biểu thức $p(t)=120+15 \cos 150 \pi t$, \\
			ta được $p(2)=120+15 \cos (150 \cdot \pi \cdot 2)=135$ mmHg.
			\itemch \textbf{Đúng}. \\
			Ta có
			\begin{eqnarray*}
				&&-1 \leq \cos 150 \pi t \leq 1 \\&\Leftrightarrow& -15 \leq 15 \cos 150 \pi t \leq 15\\ &\Leftrightarrow& 105 \leq 120 + 15 \cos 150 \pi t \leq 135
			\end{eqnarray*}
			Vậy huyết áp tâm thu của người đó là $135$ mmHg, huyết áp tâm trương là $105$ mmHg.
			\itemch \textbf{Đúng}. Vì huyết áp của một người đạt $120$ mmHg nên \begin{eqnarray*}
				&& 120 + 15 \cos 150 \pi t = 120 \\&\Leftrightarrow& \cos 150 \pi t =0 \\&\Leftrightarrow & 150 \pi t=\dfrac{\pi}{2}+k \pi\\
				&\Leftrightarrow& t=\dfrac{1}{300}+\dfrac{1}{150}k.
			\end{eqnarray*}
			Đổi thời gian từ $0$ giây đến $30$ giây tương ứng với $0 \le t \le 0{,}5$ (phút).\\
			Xét điều kiện $0\leq t \leq 0{,}5$, ta có
			\begin{eqnarray*}
				&&0\leq t \leq 0{,}5 \\
				&\Leftrightarrow& 0 \leq \dfrac{1}{300}+\dfrac{1}{150}k \leq 0{,}5\\ &\Leftrightarrow& -\dfrac{1}{300} \leq \dfrac{1}{150}k \leq \dfrac{149}{300}\\ &\Leftrightarrow& -\dfrac{1}{2} \leq k \leq \dfrac{149}{2}.	
			\end{eqnarray*}
			Vì $k \in \mathbb{Z}$ nên $k \in \{0;1;2; \ldots ;74\}$.\\
			Vậy có $75$ lần huyết áp đạt $120$ mmHg.
		\end{itemchoice}
	}
\end{ex}

\begin{ex}%[1D2H2-7]%[Dự án đề kiểm tra Toán 11 GHKI NH24-25-Dot3-Hieu Hieu Minh Minh]%[THPT Nguyễn Thị Minh Khai - Tp HCM]
	Bạn Lam muốn mua một máy tính Casio Fx-580VNX tại một nhà sách có giá là $840$ ngàn đồng. Bạn Lam lập kế hoạch mỗi ngày sẽ tiết kiệm $20$ ngàn đồng tiền tiêu vặt mà ba mẹ cho. Gọi $u_n$ ( $n \in \mathbb{N}^*$ ) là số tiền bạn Lam có được vào ngày thứ $n$, biết rằng ngày đầu tiên bạn Lam có được số tiền là $70$ ngàn đồng.
	\choiceTF
	{\True Dãy số $\left(u_n\right)$ là một cấp số cộng có số hạng đầu $u_1=70$ (ngàn đồng) và công sai $d=20$ (ngàn đồng)}
	{Công thức số hạng tổng quát của dãy số $\left(u_n\right)$ là $u_n=90-20 n$ (ngàn đồng), $n \in \mathbb{N}^*$}
	{Bạn Lam có được số tiền là $270$ (ngàn đồng) vào ngày thứ $10$}
	{\True Vào ngày thứ $40$, bạn Lam đã đủ tiền mua máy tính Casio Fx580 VNX tại nhà sách trên}
	\loigiai{
		
		Ta có $u_1=70$ và công sai $d=20$ nên $u_n=70+(n-1)\cdot 20=20n+50$ (ngàn đồng).
		\begin{itemchoice}
			\itemch \textbf{Đúng}. Theo chứng minh trên thì $\left(u_n\right)$ là một cấp số cộng với số hạng đầu $u_1=70$ (ngàn đồng) và công sai $d=20$ (ngàn đồng)  
			\itemch \textbf{Sai}. Theo chứng minh trên thì $u_n=70+(n-1)\cdot 20=20n+50$ (ngàn đồng).
			\itemch \textbf{Sai}. Ngày thứ $10$, bạn Lam có số tiền là $u_{10}=20\cdot 10+50=250$ ngàn đồng.
			\itemch \textbf{Đúng}. \\
			Ngày thứ $40$, bạn Lam có số tiền là $u_{40}=20\cdot 40+50=850$ ngàn đồng.\\
			Vì $850$ ngàn đồng > $840$ ngàn đồng nên bạn Lam đã đủ tiền mua máy tính Casio Fx580 VNX tại nhà sách trên.
		\end{itemchoice}
	}
\end{ex}


\Closesolutionfile{ans}
%\inputansbox[2]{2}{ans/answer.tex}



\begin{center}
	\textbf{PHẦN 3 - TỰ LUẬN}
\end{center}

%Câu 1...........................
\begin{ex}%[1D1N5-4]%[1D1H5-5]
	Giải các phương trình sau:
	\begin{enumerate}
		\item $\tan \left(2x-55^{\circ}\right)=1$.
		\item $1-2\sin^2x=\cos (x-12)$.
		\item $\sin x-\sqrt{3}\cos x=-2\cos \left(3x+\dfrac{\pi}{2}\right)$.
	\end{enumerate}
\loigiai{
	\begin{enumerate}
		\item $\tan \left(2x-55^{\circ}\right)=1\Leftrightarrow 2x-55^{\circ}=45^{\circ}+k\cdot 180^{\circ}\Leftrightarrow x=50^{\circ}+k\cdot 90^{\circ}$, $k\in\mathbb{Z}$.
		\item Ta có \allowdisplaybreaks
		\begin{eqnarray*}
		& &1-2\sin^2x=\cos (x-12)\\
		&\Leftrightarrow& \cos 2x=\cos (x-12)\\
		&\Leftrightarrow&\hoac{&2x=x-12+k\cdot 2\pi\\&2x=12-x+k\cdot 2\pi}\\
		&\Leftrightarrow&\hoac{&x=-12+k\cdot 2\pi\\&x=4+k\cdot\dfrac{2\pi}{3}}, k\in\mathbb{Z}.
		\end{eqnarray*}
		\item Ta có \allowdisplaybreaks
		\begin{eqnarray*}
			& &\sin x-\sqrt{3}\cos x=-2\cos \left(3x+\dfrac{\pi}{2}\right)\\
			&\Leftrightarrow& \dfrac{1}{2}\cdot\sin x-\dfrac{\sqrt{3}}{2}\cdot\cos x=\sin 3x\\
			&\Leftrightarrow& \sin x\cdot\cos\dfrac{\pi}{3}-\sin\dfrac{\pi}{3}\cdot\cos x=\sin 3x\\
			&\Leftrightarrow& \sin\left(x-\dfrac{\pi}{3}\right)=\sin 3x\\ 
			&\Leftrightarrow&\hoac{&x-\dfrac{\pi}{3}=3x+k\cdot 2\pi\\&x-\dfrac{\pi}{3}=\pi-3x+k\cdot 2\pi}\\
			&\Leftrightarrow&\hoac{&x=-\dfrac{\pi}{6}-k\cdot \pi\\&x=\dfrac{\pi}{3}+k\cdot\dfrac{\pi}{2}}, k\in\mathbb{Z}.
		\end{eqnarray*}
	\end{enumerate}
}
\end{ex}
%Câu 2...........................
\begin{ex}%[1D2H2-2]
	Chứng minh dãy số $\left(u_n\right)$ với $u_n=2024n-2025$ là cấp số cộng. Xác định công sai, số hạng đầu của cấp số cộng đó.
	( 1 điểm).
	\loigiai{
		Ta có $u_n=2024n-2025\Rightarrow u_{n+1}=2024(n+1)-2025=2024n-1$.\\
		Vì $u_{n+1}-u_n=2024$ nên dãy số $\left(u_n\right)$ là một cấp số cộng với $\heva{&d=2024\\&u_1=-1.}$
	}
\end{ex}


