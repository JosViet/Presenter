\def\thoigian{60}%--Thời gian
\de{ĐỀ THI GIỮA KỲ I NĂM HỌC 2024-2025}{THPT Tây Thạnh - Tp Hồ Chí Minh}

\begin{center}
	\textbf{PHẦN 1 - Câu trắc nghiệm nhiều phương án lựa chọn.}
\end{center}
\setcounter{ex}{0}

\Opensolutionfile{ans}[ans-ABCD]
\begin{ex}%[0D1N1-1]%[Dự án đề kiểm tra Toán 10 GHKI NH24-25- Nguyễn Sĩ Đạt]%[THPT Tây Thạnh]
	Trong các câu sau, câu nào \textbf{không} phải là mệnh đề?
	\choice
	{$2017$ là số lẻ}
	{Hôm nay là thứ hai}
	{\True Hôm nay trời đẹp quá!}
	{Hình bình hành là đa giác có $3$ cạnh}
	\loigiai{
		Phát biểu \lq\lq Hôm nay trời đẹp quá!\rq\rq\ không phải là một mệnh đề.
	}
\end{ex}
\begin{ex}%[0H4N1-1]%[Dự án đề kiểm tra Toán 10 GHKI NH24-25- Nguyễn Sĩ Đạt]%[THPT Tây Thạnh
	Cho $\alpha$ là góc tù. Mệnh đề nào đúng trong các mệnh đề sau?
	\choice
	{$\cot \alpha>0$}
	{\True $\tan \alpha<0$}
	{$\cos \alpha>0$}
	{$\sin \alpha<0$}
	\loigiai{
		Do $\alpha$ là góc tù nên $\sin \alpha> 0$ và $\cos \alpha <0$. Suy ra $\tan\alpha=\dfrac{\sin \alpha}{\cos \alpha}<0$.
	}
\end{ex}

\begin{ex}%[0D1N2-3]%[Dự án đề kiểm tra Toán 10 GHKI NH24-25- Nguyễn Sĩ Đạt]%[THPT Tây Thạnh
	Cho hai tập hợp $A=[-2 ; 5]$, $ B=(0 ; 6)$. Tìm $A \cap B$.
	\choice
	{$A \cap B=[-2 ; 6)$}
	{\True $A \cap B=(0 ; 5]$}
	{$A \cap B=(0 ; 5)$}
	{$A \cap B=[0 ; 5]$}
	\loigiai{
		Ta có $A \cap B=(0 ; 5]$.
	}
\end{ex}
\begin{ex}%[0H4N2-1]%[Dự án đề kiểm tra Toán 10 GHKI NH24-25- Nguyễn Sĩ Đạt]%[THPT Tây Thạnh
	Cho tam giác $A B C$ có $A B=6$, $A C=\sqrt{20}$, $B C=\sqrt{32}$. Tính góc $\widehat{B}$ của tam giác $A B C$.
	\choice
	{$\widehat{B}=90^{\circ}$}
	{$\widehat{B}=120^{\circ}$}
	{$\widehat{B}=60^{\circ}$}
	{\True $\widehat{B}=45^{\circ}$}
	\loigiai{
		Áp dụng định lí côsin cho tam giác $ABC$, ta có:\\
		$\cos B=\dfrac{AB^2+BC^2-AC^2}{2AB\cdot BC}=\dfrac{6^2+(\sqrt{32})^2-(\sqrt{20})^2}{2\cdot6\cdot\sqrt{32}}=\dfrac{\sqrt{2}}{2}$.\\
		Suy ra $\widehat{B}=45^\circ$.
	}
\end{ex}

\begin{ex}%[0D1N2-3]%[Dự án đề kiểm tra Toán 10 GHKI NH24-25- Nguyễn Sĩ Đạt]%[THPT Tây Thạnh
	Cho tập hơp $A=\{2 ; 3 ; 4\}$ và $B=\{2 ; 4 ; 6 ; 7 ; 8\}$. Khi đó $A \cup B$ là:
	\choice
	{$\{2 ; 4 ; 6 ; 7\}$}
	{$\{2 ; 3 ; 4 ; 6 ; 7 ; 8\}$}
	{\True $\{2 ; 3 ; 4 ; 5 ; 6 ; 7 ; 8\}$}
	{$\{2 ; 4\}$}
	\loigiai{
		Ta có $A \cup B=\{2 ; 3 ; 4 ; 6 ; 7 ; 8\}$.
	}
\end{ex}
\begin{ex}%[0D2N2-1]%[Dự án đề kiểm tra Toán 10 GHKI NH24-25- Nguyễn Sĩ Đạt]%[THPT Tây Thạnh
	Hệ bất phương trình nào sau đây \textbf{không} là hệ bất phương trình bậc nhất hai ẩn?
	\choice
	{$\heva{&2(x+9)+y \leq 13\\	&3(x+6)>y-2}$}
	{\True $\heva{&x^2<y+2\\&3x-5y \leq 10}$}
	{$\heva{&x+y-3\leq 0\\	&x-y>4}$}
	{$\heva{&x \leq 0\\	&x+y-2>0}$}
	\loigiai{
		Ta có $\heva{&x^2<y+2\\&3x-5y \leq 10}$ không là hệ bất phương trình bậc nhất hai ẩn.
	}
\end{ex}
\begin{ex}%[0H4H3-1]%[Dự án đề kiểm tra Toán 10 GHKI NH24-25- Nguyễn Sĩ Đạt]%[THPT Tây Thạnh
	Một tam giác có ba cạnh là $52$, $56$, $60$. Bán kính đường tròn ngoại tiếp là
	\choice
	{$\dfrac{65}{8}$}
	{$40$ }
	{\True $32{,}5$}
	{$\dfrac{65}{4}$}
	\loigiai{
		Nửa chu vi của tam giác là $p=\dfrac{52+56+60}{2}=84$.\\
		Diện tích của tam giác là $S=\sqrt{84(84-52)(84-56)(84-60)}=1344$.\\
		Bán kính đường tròn ngoại tiếp của tam giác là $R=\dfrac{52\cdot56\cdot60}{4\cdot1344}=32{,}5$.
	}
\end{ex}
\begin{ex}%[0D2N2-2]%[Dự án đề kiểm tra Toán 10 GHKI NH24-25- Nguyễn Sĩ Đạt]%[THPT Tây Thạnh
	Miền nghiệm của hệ bất phương trình 
	$\heva{&x-2y<0 \\& x+3y >-2}$ không chứa điểm nào sau đây?
	\choice
	{\True $B(2 ; 1)$}
	{$D(0 ; 3)$}
	{$C(-3 ; 4)$}
	{$A(-1 ; 0)$}
	\loigiai{
		Thay $x=2$, $y=1$ vào
		$\heva{&x-2y<0 \\& x+3y >-2}$ ta được
		$\heva{&2-2\cdot 1<0 \\& 2+3\cdot 1 >-2}$ (mâu thuẫn).
	}
\end{ex}

\begin{ex}%[0D2H2-2]%[Dự án đề kiểm tra Toán 10 GHKI NH24-25- Nguyễn Sĩ Đạt]%[THPT Tây Thạnh
	Phần nửa mặt phẳng không bị gạch (kể cả bờ đường thẳng) trong hình sau là miền nghiệm của bất phương trình nào?
	\begin{center}
		\begin{tikzpicture}[scale=1, font=\footnotesize, line join=round, line cap=round, >=stealth]
			\def\xt{-4} \def\xp{4} \def\yd{-2}\def\yt{4} 
			\def\a{-3} \def\b{3} \def\c{-1} \def\d{3}
			\draw[step=1,gray, very thin,opacity=0.2] (\xt+0.1,\yd+0.1) grid (\xp-0.1,\yt-0.1);
			\clip (\xt+0.05,\yd+0.05) rectangle (\xp-0.05,\yt-0.05);
			%%-----
			\draw[smooth, domain=\xt-.2:\xp+.2] plot(\x,{2.5*\x-0.5}) ;
			\fill[pattern=north west lines,pattern color =gray,opacity=0.4] (5,5)--(-5,5)--plot[domain=\xt:\xp](\x,{2.5*\x-.5})--cycle;
			%%-----
			\foreach \x in {\a,...,-1,1,2,...,\b} {\draw[ shift={(\x,0)}] (0pt,3pt) -- (0pt,-3pt) node[below] { $\x$};}
			\foreach \y in {\c,...,-1,1,2,...,\d} {\draw[ shift={(0,\y)}] (3pt,0pt) -- (-3pt,0pt) node[left] {$\y$};}
			\draw[->] (\xt-0.2,0)--(\xp-0.05,0);
			\draw (\xp-0.2,-0.1) node [below]{\normalsize$x$};
			\draw[->] (0,\yd-0.2)--(0,\yt-0.05);
			\draw (0,\yt-0.2) node [left]{\normalsize$y$};
			\fill (0,0) circle (1.2pt);
			\draw (0.05,0.05) node[below left]{$O$};
			%%-----
			\draw (1.8,3.1)node {\normalsize$d$};
		\end{tikzpicture}
	\end{center}
	\choice
	{\True $5 x-2 y \geq 1$}
	{$5 x-2 y<1$}
	{$5 x-2 y>1$}
	{$5 x-2 y \leq 1$}
	\loigiai{
		Miền nghiệm chứa điểm $(1;0)$ và kể cả bờ đường thẳng nên chỉ có đáp án $5 x-2 y \geq 1$ thỏa mãn. 
	}
\end{ex}

\begin{ex}%[0D1N2-1]%[Dự án đề kiểm tra Toán 10 GHKI NH24-25- Nguyễn Sĩ Đạt]%[THPT Tây Thạnh
	Cho tập hơp $A=\{x \in \mathbb{N} \mid x \leq 6\}$. Tập $A$ được viết dưới dạng liệt kê các phần tử là
	\choice
	{$A=\{1 ; 2 ; 3 ; 4 ; 5 ; 6\}$}
	{$A=\{0 ; 1 ; 2 ; 4 ; 5 ; 6\}$}
	{\True $A=\{0 ; 1 ; 2 ; 3 ; 4 ; 5 ; 6\}$}
	{$A=\{0 ; 1 ; 2 ; 3 ; 4 ; 5\}$}
	\loigiai{
		Ta có $A=\{0 ; 1 ; 2 ; 3 ; 4 ; 5 ; 6\}$.
	}
\end{ex}

\begin{ex}%[0H4H1-2]%[Dự án đề kiểm tra Toán 10 GHKI NH24-25- Nguyễn Sĩ Đạt]%[THPT Tây Thạnh
	Cho biết $\sin \alpha=\dfrac{4}{5}$, $\left(90^{\circ}<\alpha<180^{\circ}\right)$. Khi đó giá trị $\cos \alpha$ bằng
	\choice
	{$\dfrac{3}{5}$}
	{\True $-\dfrac{3}{5}$}
	{$\dfrac{1}{5}$}
	{$-\dfrac{1}{5}$}
	\loigiai{
		Vì $90^{\circ}<\alpha<180^{\circ}$ nên $\cos \alpha <0$. Ta có $\sin ^2 \alpha + \cos ^2 \alpha =1 \Rightarrow \cos\alpha=-\dfrac{3}{5}$.
	}
\end{ex}

\begin{ex}%[0D2N1-1]%[Dự án đề kiểm tra Toán 10 GHKI NH24-25- Nguyễn Sĩ Đạt]%[THPT Tây Thạnh]
	Với giá trị nào của $b$ để bất phương trình $2x+by<7$ là bất phương trình bậc nhất hai ẩn?
	\choice
	{$b<0$}
	{$b \neq 0$}
	{\True $b \in \mathbb{R}$}
	{$b>0$}
	\loigiai{
		Bất phương trình $ax+by<c$ là bất phương trình bậc nhất hai ẩn khi và chỉ khi $a$ và $b$ không đồng thời bằng $0$. Vì $a=2 \neq 0$ nên $b \in \mathbb{R}$.
	}
\end{ex}

\Closesolutionfile{ans}

%\indapan{6}{ans-ABCD}

%\cauds

\begin{center}
	\textbf{PHẦN 2 - Câu trắc nghiệm đúng sai. Trong mỗi ý a,b,c,d ở mỗi câu, thí sinh chọn đúng hoặc sai}
\end{center}
\setcounter{ex}{0}
\Opensolutionfile{ans}[ans-DS]
\begin{ex}%[0D1H3-4]%[Dự án đề kiểm tra Toán 10 GHKI NH24-25- Vô Văn Tự]%[THPT Tây Thạnh]
	Cho các tập hợp $A=[-2;5)$, $B=[0;7]$ và $C=(-2;1)$.
	\choiceTF
	{\True Hợp của hai tập hợp $B$ và $C$ là $B\cup C=(-2;7]$}
	{Giao của hai tập hợp $A$ và $B$ là $A\cap B=[0;4]$}
	{\True Phần bù của tập hợp $B$ trong tập hợp số thực là $\mathrm{C}_{\mathbb{R}}B=(-\infty;0)\cup(7;+\infty)$}
	{Tập hợp $A\setminus C$ có $4$ phần tử là số nguyên}
	\loigiai{
		\begin{itemchoice}
			\itemch \textbf{Đúng}.\\
			Ta có $B\cup C=[0;7]\cup (-2;1)=(-2;7]$.
			\itemch \textbf{Sai}.\\
			Ta có $A\cap B=[-2;5)\cap[0;7]= [0;5)$.
			\itemch \textbf{Đúng}.\\
			Ta có $\mathrm{C}_{\mathbb{R}}B=\mathbb{R}\setminus B=(-\infty;0)\cup(7;+\infty)$.
			\itemch \textbf{Sai}.\\Ta có $A\setminus C=[-2;5)\setminus(-2;1)=\{-2\}\cup[1;5)$.\\
			Các số nguyên thuộc $A\setminus C$ là $-2$; $1$; $2$; $3$; $4$.\\
			Vậy có $5$ số nguyên.
		\end{itemchoice}
	}
\end{ex}
\begin{ex}%[0H4C3-1]%[Dự án đề kiểm tra Toán 10 GHKI NH24-25- Vô Văn Tự]%[THPT Tây Thạnh]
	Cho tam giác $ABC$ có $AB=9$, $AC=5$, $\cos A=\dfrac{3}{5}$.
	\choiceTF
	{\True Góc $\widehat{ACB}$ là góc tù}
	{$\sin A=-\dfrac{4}{5}$}
	{\True Độ dài cạnh $BC=2\sqrt{13}$}
	{Tồn tại duy nhất một điểm $M$ nằm trên cạnh $AB$ để $\sin\widehat{AMC}=\dfrac{4\sqrt{17}}{17}$}
	\loigiai{
		\begin{itemchoice}
			\itemch \textbf{Đúng}.\\
			Áp dụng định lý côsin ta có $$BC=\sqrt{AB^2+AC^2-2\cdot AB\cdot AC\cdot\cos A}=\sqrt{9^2+5^2-2\cdot9\cdot5\cdot \dfrac{3}{5}}=2\sqrt{13}.$$
			Khi đó
			$$\cos\widehat{ACB}=\dfrac{AC^2+BC^2-AB^2}{2\cdot AC\cdot BC}=\dfrac{5^2+52-9^2}{2\cdot5\cdot 2\sqrt{13}}=-\dfrac{\sqrt{13}}{65}<0.$$
			Suy ra góc $\widehat{ACB}$ tù.
			\itemch \textbf{Sai}.\\
			Vì $\cos A=\dfrac{3}{5}$ nên góc $\widehat{A}$ nhọn, suy ra $\sin A>0$.
			\itemch \textbf{Đúng}.\\
			Áp dụng định lý côsin ta có $$BC=\sqrt{AB^2+AC^2-2\cdot AB\cdot AC\cdot\cos A}=\sqrt{9^2+5^2-2\cdot9\cdot5\cdot \dfrac{3}{5}}=2\sqrt{13}.$$
			\itemch \textbf{Sai}.
			\begin{center}
				\begin{tikzpicture}[scale=0.7, font=\footnotesize, line join=round, line cap=round, >=stealth]
					\def\r{3}
					\path
					(0,0) coordinate (A)
					(9,0) coordinate (B)
					(5,0) coordinate (i)
					($(i)!1!-53:(A)$) coordinate (C)%phep quay
					($(A)!0.55!(B)$) coordinate (M)
					;
					\draw
					(A)--(M)node[midway,below]{$x$}--(B)--(C)--cycle
					(C)--(M)
					;
					\foreach \x/\g in{A/180, B/0, C/120, M/-90} \draw[fill=black](\x) circle (1pt)
					($(\x)+(\g:3mm)$) node{\x};
				\end{tikzpicture}
			\end{center}
			Gọi điểm $M$ thuộc cạnh $AB$ sao cho $AM=x$ $(0<x\le9)$ thoả mãn $\sin\widehat{AMC}=\dfrac{4\sqrt{17}}{17}$.\\
			Áp dụng định lý côsin vào tam giác $AMC$ ta được\\
			\allowdisplaybreaks
			$\begin{aligned} CM^2&=AC^2+AM^2-2\cdot AC\cdot AM\cdot\cos A\\
				&=25+x^2-2\cdot5\cdot x\cdot\dfrac{3}{5}\\
				&=x^2-6x+25\\
				\Rightarrow CM&=\sqrt{x^2-6x+25}.\end{aligned}$\\
			Vì $\cos A=\dfrac{3}{5}$ nên $\widehat{A}$ là góc nhọn và $\sin A=\sqrt{1-\left(\dfrac{3}{5}\right)^2}=\dfrac{4}{5}$.\\
			Áp dụng định lý sin vào tam giác $AMC$ ta được
			\allowdisplaybreaks
			\begin{eqnarray*}
				& & \dfrac{AC}{\sin \widehat{AMC}}=\dfrac{MC}{\sin A}\\
				&\Leftrightarrow& \dfrac{5}{\dfrac{4\sqrt{17}}{17}}=\dfrac{\sqrt{x^2-6x+25}}{\dfrac{4}{5}}\\
				&\Leftrightarrow& \sqrt{x^2-6x+25}=\sqrt{17}\\
				&\Leftrightarrow& x^2-6x+25=17\\
				&\Leftrightarrow& x^2-6x+8=0\\
				&\Leftrightarrow& \hoac{&x=2\quad\text{(thoả mãn)}\\&x=4\quad\text{(thoả mãn)}.}				
			\end{eqnarray*}
			Vậy tồn tại hai điểm $M$ thuộc cạnh $AB$ thoả mãn $\sin\widehat{AMC}=\dfrac{4\sqrt{17}}{17}$.
		\end{itemchoice}
	}
\end{ex}

\begin{ex}%[0D2H2-3]%[Dự án đề kiểm tra Toán 10 GHKI NH24-25- Vô Văn Tự]%[THPT Tây Thạnh]
	Cho hệ bất phương trình có miền nghiệm là miền tam giác không gạch chéo như hình.
	\begin{center}
		\begin{tikzpicture}[scale=1, font=\footnotesize, line join=round, line cap=round, >=stealth]
			\draw[->] (-1,0)--(0,0)node[below left]{$O$}--(6.05,0)node[right]{$x$};
			\draw[->] (0,-1)--(0,5.05)node[right]{$y$};
			\clip (-0.98,-0.98) rectangle (5.98,4.98);
			\filldraw[pattern = north east lines, pattern color = black] (-1,8/3)--(6,1/3)--(6,-1)--(-1,-1)--cycle;%Tô nền
			\filldraw[pattern = north west lines, pattern color = black] (5.34,-1)--(4/3,5)--(6,5)--(6,-1)--cycle;
			\filldraw[pattern = north east lines, pattern color = black] (-0.5,-1)--(2.5,5)--(-1,5)--(-1,-1)--cycle;
			\draw[dashed] (1,0)--(1,2)--(0,2)
			(4,0)--(4,1)--(0,1) (2,0)--(2,4)--(0,4)
			;
			\draw[thick,blue] (1,2)--(2,4)--(4,1)--cycle;
			\foreach \x in {1,2,4} \fill[black] (\x,0)node[below]{$\x$} circle(1pt);
			\foreach \y in {1,2,4} \fill[black] (0,\y)node[left]{$\y$} circle(1pt);
			\fill[black] (1,2)circle(1pt) (2,4)circle(1pt) (4,1)circle(1pt);
		\end{tikzpicture}
	\end{center}
	\choiceTF
	{\True Điểm $A(2;2)$ thuộc miền nghiệm của hệ bất phương trình đã cho}
	{Hệ bất phương trình đã cho $3$ cặp số $\left(x_0;y_0\right)$ (với $x_0$, $y_0\in\mathbb{Z}$) thoả mãn}
	{\True Điểm $B(3;3)$ không thuộc miền nghiệm của hệ bất phương trình đã cho}
	{Biểu thức $T=29x+5y$ đạt giá trị bé nhất trên miền nghiệm của hệ bất phương trình cho cho bằng $34$}
	\loigiai{
		\begin{center}
			\begin{tikzpicture}[scale=1, font=\footnotesize, line join=round, line cap=round, >=stealth]
				\draw[->] (-1,0)--(0,0)node[below left]{$O$}--(6.05,0)node[right]{$x$};
				\draw[->] (0,-1)--(0,5.05)node[right]{$y$};
				\clip (-0.98,-0.98) rectangle (5.98,4.98);
				\filldraw[pattern = north east lines, pattern color = black] (-1,8/3)--(6,1/3)--(6,-1)--(-1,-1)--cycle;%Tô nền
				\filldraw[pattern = north west lines, pattern color = black] (5.34,-1)--(4/3,5)--(6,5)--(6,-1)--cycle;
				\filldraw[pattern = north east lines, pattern color = black] (-0.5,-1)--(2.5,5)--(-1,5)--(-1,-1)--cycle;
				\draw[dashed] (1,0)--(1,2)--(0,2)
				(4,0)--(4,1)--(0,1) (2,0)--(2,4)--(0,4)
				;
				\draw[thick,blue] (1,2)--(2,4)--(4,1)--cycle;
				\foreach \x in {1,2,4} \fill[black] (\x,0)node[below]{$\x$} circle(1pt);
				\foreach \y in {1,2,4} \fill[black] (0,\y)node[left]{$\y$} circle(1pt);
				\fill[black] (1,2)node[above]{$M$}circle(1pt) (2,4)node[right]{$N$}circle(1pt) (4,1)node[above right]{$P$}circle(1pt);
			\end{tikzpicture}
		\end{center}
		Gọi ba đỉnh của miền nghiệm lần lượt là $M(1;2)$, $N(2;4)$ và $P(4;1)$.\\
		Đường thẳng $MN$ đi qua $M$ và $N$ nên có phương trình $2x-y=0$.\\
		Đường thẳng $MP$ đi qua $M$ và $P$ nên có phương trình $x+3y=7$.\\
		Đường thẳng $PN$ đi qua $P$ và $N$ nên có phương trình $3x+2y=14$.\\
		Miền tam giác là miền nghiệm của hệ bất phương trình $\heva{&2x-y\ge0\\&3x+2y\le14\\&x+3y\ge7.}$
		\begin{itemchoice}
			\itemch \textbf{Đúng}.\\
			Điểm $A(2;2)$ là một đỉnh của miền tam giác nên điểm $A$  thuộc miền nghiệm.
			\itemch \textbf{Sai}.\\
			Các điểm có toạ độ $\left(x_0;y_0\right)$ (với $x_0$, $y_0\in\mathbb{Z}$) thuộc miền tam giác $MNP$ là $(1;2)$, $(2;4)$, $(2;2)$, $(4;1)$.\\
			Thay $x=3;\,y=2$ vào hệ bất phương trình ta được  $\heva{&2\cdot3-2\ge0\text{ (thoả mãn)}\\&3\cdot3+2\cdot2\le14\text{ (thoả mãn)}\\&3+3\cdot2\ge7\text{ (thoả mãn)}}$.\\
			Suy ra có $5$ cặp số $\left(x_0;y_0\right)$ (với $x_0$, $y_0\in\mathbb{Z}$) thoả mãn.
			\itemch \textbf{Đúng}.\\
			Ta có toạ độ điểm $B(3;3)$ không thoả mãn bất phương trình $3x+2y\le14$ nên không thuộc miền nghiệm của hệ bất phương trình đã cho.
			\itemch \textbf{Sai}.\\
			Tại điểm $M(1;2)$, ta có $T=29\cdot1+5\cdot2=39$.\\
			Tại điểm $N(2;4)$, ta có $T=29\cdot2+5\cdot4=78$.\\
			Tại điểm $P(4;1)$, ta có $T=29\cdot4+5\cdot1=121$.\\
			Suy ra $T_{\min}=39$.
		\end{itemchoice}	
	}
\end{ex}

\begin{ex}%[0H4C3-2]%[Dự án đề kiểm tra Toán 10 GHKI NH24-25- Vô Văn Tự]%[THPT Tây Thạnh]
	Một đoạn đường đi từ điểm $A$ đến điểm $B$ nhưng bị vướng một ngọn núi. Nhà đầu tư thực hiện phương án xây dựng đường tránh từ $A$ đến $M$, từ $M$ đến $N$ và sau đó mới đến $B$. Biết rằng $AM=3{,}2$ km, $MN=5{,}5$ km, $NB=3{,}6$ km, $\alpha=140^\circ$ và $\beta=145^\circ$.
	\begin{center}
		\begin{tikzpicture}[scale=1, font=\footnotesize, line join=round, line cap=round, >=stealth]
			\path
			(0,0.3) coordinate (A)
			(4.5,0.3) coordinate (B)
			(0.7,-1.2) coordinate (M)
			(2.8,-1.4) coordinate (N)
			;
			\draw
			(A)--(B)--(N)--(A)--(M)--(B)
			(M)--(N)
			;
			\draw 
			pic[draw, angle radius=2mm]{angle=N--M--A}
			pic[draw,double, angle radius=2mm]{angle=B--N--M}
			($(M)+(0.,0.1)$)node[above right]{$\alpha$}
			($(N)+(0.,0.1)$)node[above]{$\beta$}
			;
			\fill[gray!90] (0.5,-0.2)--(1,1.0)--(1.8,1.5)--(2.4,0.5)--(2.6,1.3)--(3,1.5)--(3.4,1)--(3.9,-0.2)--(2.3,-0.5)--(1,-0.6)--cycle ;
			\foreach \x/\g in{A/180, B/0, N/-90, M/-90} \draw[fill=black](\x) circle (1pt)
			($(\x)+(\g:3mm)$) node{\x};
		\end{tikzpicture}
	\end{center}
	Các phát biểu sau đúng hay sai?
	\choiceTF
	{\True Số đo của góc $\widehat{ANM}$ bé hơn $15^\circ$}
	{\True Độ dài cạnh $MB\approx 8{,}7$ km (làm tròn đến hàng phần mười)}
	{Độ dài cạnh $AN\approx 3{,}68$ km (kết quả làm tròn đến hàng phần trăm)}
	{\True Giả sử nhà đầu tư dự định phương án đường hầm xuyên qua núi đi thẳng từ $A$ đến $B$ (chi phí cao và sẽ có thu phí). Một người đi xe với vận tốc trung bình $80$ km/h trong đường hầm và vận tốc trung bình $40$ km/h trên đường tránh. Thời gian đi từ $A$ đến $B$ bằng đường hầm sẽ tiết kiệm một khoảng thời gian khoảng $10$ phút (làm tròn hàng đơn vị phút)}
	\loigiai{
		\begin{center}
			\begin{tikzpicture}[scale=1, font=\footnotesize, line join=round, line cap=round, >=stealth]
				\path
				(0,0) coordinate (A)
				(4.5,0) coordinate (B)
				(0.7,-1.2) coordinate (M)
				(2.8,-1.4) coordinate (N)
				;
				\draw
				(A)--(B)--(N)--(A)--(M)--(B)
				(M)--(N)
				;
				\draw 
				pic[draw, angle radius=2mm]{angle=N--M--A}
				pic[draw,double, angle radius=2mm]{angle=B--N--M}
				($(M)+(0.,0.1)$)node[above right]{$\alpha$}
				($(N)+(0.,0.1)$)node[above]{$\beta$}
				;
				\foreach \x/\g in{A/180, B/0, N/-90, M/-90} \draw[fill=black](\x) circle (1pt)
				($(\x)+(\g:3mm)$) node{\x};
			\end{tikzpicture}
		\end{center}
		\begin{itemchoice}
			\itemch \textbf{Đúng}.\\
			Áp dụng định lý côsin cho tam giác $AMN$ ta được\\
			$\begin{aligned} AN^2&=AM^2+MN^2-2\cdot AM\cdot MN\cdot\cos\widehat{AMN}\\
				&=3{,}2^2+5{,}5^2-2\cdot 3{,}2\cdot5{,}5\cdot\cos140^\circ\approx67{,}455.\\
				\Rightarrow AN&\approx 8{,}213 ~\mathrm{(km)}.\end{aligned}$\\
			Suy ra $\cos\widehat{ANM}=\dfrac{AN^2+MN^2-AM^2}{2\cdot AN\cdot MN}=\dfrac{8{,}213^2+5{,}5^2-3{,}2^2}{2\cdot8{,}213\cdot5{,}5}\approx0{,}9681$.\\
			$\Rightarrow \widehat{ANM}\approx 14^\circ30'$.
			\itemch \textbf{Đúng}.\\
			Áp dụng định lý côsin cho tam giác $BMN$ ta được\\
			$\begin{aligned} MB^2&=NM^2+NB^2-2\cdot NM\cdot NB\cdot\cos\widehat{MNB}\\
				&=5{,}5^2+3{,}6^2-2\cdot 5{,}5\cdot3{,}6\cdot\cos145^\circ\approx75{,}648.\\
				\Rightarrow MB&\approx 8{,}7 ~\mathrm{(km)}.\end{aligned}$
			\itemch \textbf{Sai}.\\
			Ta có $AN \approx 8{,}21$ (km).
			\itemch \textbf{Đúng}.\\
			Ta có $\widehat{ANB}=145^\circ-\widehat{ANM}\approx 145^\circ-14^\circ30'\approx 130^\circ30'$.\\
			Áp dụng định lý côsin cho tam giác $ANB$ ta được\\
			$\begin{aligned} AB^2&=AN^2+NB^2-2\cdot AN\cdot NB\cdot\cos\widehat{ANB}\\
				&\approx 8{,}213^2+3{,}6^2-2\cdot 8{,}213\cdot3{,}6\cdot\cos\left(130^\circ30'\right)\approx118{,}82.\\
				\Rightarrow AB&\approx 10{,}9 ~\mathrm{(km)}.\end{aligned}$\\
			Thời gian đi xe trong đường hầm là $\dfrac{10{,}9}{80}=\dfrac{109}{800}$ (giờ) $= \dfrac{327}{40}$ (phút).\\
			Thời gian đi xe trên đường tránh là $\dfrac{3{,}2+5{,}5+3{,}6}{40}=\dfrac{123}{400}$ (giờ) $=\dfrac{369}{20}$ (phút).\\
			Suy ra thời gian từ $A$ đến $B$ sẽ tiết kiệm một khoảng thời gian khoảng\break  $\dfrac{369}{20}-\dfrac{327}{40}=\dfrac{411}{40}\approx 10$ phút.
		\end{itemchoice}
	}
\end{ex}
\Closesolutionfile{ans}

\begin{center}
	\textbf{PHẦN 3 - Câu trắc nghiệm trả lời ngắn}
\end{center}
\setcounter{ex}{0}
\Opensolutionfile{ans}[ans-KQ]
	%%%==============Bai_BT1==============%%%
%%%==============Bai_BT1==============%%%
	\begin{ex}%[0D1V3-5]%[Thi giữa học kì 1 -Tây Thạnh-TPHCM]%[Mui Doan]
		Lớp $10A26$ có $45$ học sinh được thực hiện bài khảo sát về ý kiến về việc tổ chức giải bóng chuyền và giải bóng rổ. Kết quả cho thấy có $30$ học sinh đồng ý tổ chức giải bóng chuyền và $25$ học sinh đồng ý tổ chức giải bóng rổ. Hỏi có bao nhiêu học sinh đồng ý tổ chức cả hai giải thể thao biết rằng học sinh nào cũng có thực hiện khảo sát.
		
		\shortans[oly]{10}
		\loigiai{
			Gọi $A$ là tập hợp số học sinh đồng ý tổ chức giải bóng chuyền, $B$ là tập hợp số học sinh đồng ý tổ chức giải bóng rổ.\\
			Ta có $n(A)=30$, $n(B)=25$, $n(A\cup B)=45 $.\\
			Suy ra $n(A\cap B)=n(A)+n(B)-n(A\cup B)=30+25-45=10$.\\
			Vậy có $10$ học sinh đồng ý tổ chức cả hai giải thể thao.
		}
	\end{ex}
	%%%==============HetBai_BT1==============%%%
	
	%%%==============Bai_BT2==============%%%
	\begin{ex}%[0D2V2-3]%[Thi giữa học kì 1 -Tây Thạnh-TPHCM]%[Mui Doan]
		Bạn Linh dự định làm tối đa $9$ sản phẩm trang trí để bày bán tại gian hàng hội chợ của trường. Nếu làm một sản phẩm loại $A$ thì cần $40$ phút và thu được $15$ nghìn đồng. Nếu làm một sản phẩm loại $B$ thì cần $60$ phút và thu được $20$ nghìn đồng. Hãy tính số tiền nhiều nhất mà Linh có thể thu được (đơn vị nghìn đồng)? Biết bạn Linh chỉ có tối đa $8$ giờ cho việc làm các sản phẩm trang trí.
		
		\shortans[oly]{165}
		\loigiai{
			Gọi $x$ là số sản phẩm loại \( A \) mà Linh làm.\\
			Gọi $y$ là số sản phẩm loại \( B \) mà Linh làm (điều kiện $x\geq 0$, $y\geq 0$).\\
			Linh chỉ làm tối đa $9$ sản phẩm suy ra $x + y \leq 9$.\\
			Tổng thời gian làm sản phẩm không vượt quá $8$ giờ $=480$ phút.\\
			Suy ra $40x + 60y \leq 480\Leftrightarrow 2x + 3y \leq 24$.\\
			Lợi nhuận thu được (đơn vị: nghìn đồng) là $P = 15x + 20y$.\\
			Ta có hệ bất phương trình \[\heva{&x+y \leq 9\\&	2x+3y \leq 24\\&		x \geq 0\\&y \geq 0.}\]
			Biểu diễn các bất phương trình trên hệ trục tọa độ ta được miền nghiệm là miền trong của tứ giác $OABC$ kể cả các cạnh của tứ giác.
			\begin{center}
				\begin{tikzpicture}[line join=round, line cap=round,>=stealth,thick]
					\tikzset{label style/.style={font=\footnotesize}}
					\begin{scope}
						\clip (-2,-1) rectangle (13,10);
						\fill[pattern=north east lines] (-2,11)--(13,11)--(13,-1)--(10,-1);
						\fill[pattern=north east lines] (-2,9.33)--(13,9.33)--(13,-1)--(13.5,-1);
						\fill[pattern=north east lines] (0,-1)--(-2,-1)--(-2,10)--(0,10)--cycle;
						\fill[pattern=north east lines] (-1,0)--(-1,-1)--(13,-1)--(13,0)--cycle;
						\draw (-2,11)--(10,-1) node [pos=0.3, above, sloped] {$x+y=9$};
						\draw (-2,9.33)--(13.5,-1) node [pos=0.7, above, sloped] {$2x+3y=24$};
					\end{scope}
					\draw[->] (-3,0)--(13,0) node[below]{$x$};
					\draw[->] (0,-1)--(0,10) node[left]{$y$};
					\draw (0,0) node[below left]{$O$} (9,0)node[above right]{$A$} (3,6)node[above right]{$B$} (0,8)node[above right]{$C$};
					\draw[dashed](3,0)--(3,6)--(0,6);
					\foreach \x in {-2,-1,1,2,3,4,9,12}
					\draw[thin] (\x,1pt)--(\x,-1pt) node [below] {$\x$};
					\foreach \y in {1,2,3,4,5,6,8,9}
					\draw[thin] (1pt,\y)--(-1pt,\y) node [left] {$\y$};
				\end{tikzpicture}
			\end{center}
			\begin{itemize}
				\item Tại $ O(0;0) $: $ P = 15\cdot 0 + 20\cdot 0 = 0$.
				\item Tại $A(9; 0) $: $ P = 15\cdot 9 + 20\cdot 0 = 135 $.
				\item Tại $B(3; 6) $: $ P = 15\cdot 3 + 20\cdot 6  = 165 $.
				\item Tại $C(0; 8) $: $ P = 15\cdot 0 + 20\cdot 8 = 160 $.
			\end{itemize}
			Số tiền lớn nhất Linh thu được là $165$ nghìn đồng, khi làm $3$ sản phẩm loại $ A $ và $6$ sản phẩm loại $ B $.
		}
	\end{ex}
	%%%==============HetBai_BT2==============%%%
	
	%%%==============Bai_BT3==============%%%
	\begin{ex}%[0H4V3-2]%[Thi giữa học kì 1 -Tây Thạnh-TPHCM]%[Mui Doan]
		Tại một ngã ba đường ($B$, $H$, $C$ thẳng hàng) có ba tòa nhà nằm ở ba vị trí $A$, $B$, $C$ như hình vẽ. Người ta đặt một trạm phát tín hiệu tại điểm $I$ cách đều cả ba tòa nhà. Biết rằng $BH=2{,}5$ km, $AH=6{,}5$ km và $CH=6{,}5$ km, hãy tính giá trị $100 AI^2$ (làm tròn đến hàng đơn vị).
		\begin{center}		
			\begin{tikzpicture}[declare function={r=3;}]
				\path (0:0) coordinate (I)
				(120:r) coordinate (A)
				(210:r) coordinate (B)
				(-30:r) coordinate (C)
				($(B)!(A)!(C)$) coordinate (H)
				;
				% Thay pic right angle bằng lệnh draw thủ công
				\draw ($(H)!5pt!(A)$)--($($(H)!5pt!(A)$)!5pt!90:(A)$)--($(H)!5pt!(C)$);
				
				\draw[dashed] (B)--(I)--(C) (I)--(A);
				\draw (B)--(C) (A)--(H);
				\foreach \t/\g in {A/90,B/-90,C/-90,H/-90,I/30}{
					\draw[fill=black] (\t) circle (1pt) node[shift={(\g:7pt)},font=\scriptsize]{$ \t $};
				}
			\end{tikzpicture}
		\end{center}
		
		\shortans[oly]{2425}
		\loigiai{
			Vì $IA=IB=IC$ nên $I$ là tâm đường tròn ngoại tiếp $\triangle ABC$.\\
			Ta có $BC=BH+HC=2{,}5+6{,}5=9$ km.\\
			$AB=\sqrt{AH^2+BH^2}=\sqrt{6{,}5^2+2{,}5^2}=\sqrt{48{,}5}$.\\
			$AC=\sqrt{AH^2+CH^2}=\sqrt{6{,}5^2+6{,}5^2}=6{,}5\sqrt{2}$.\\
			Diện tích tam giác $ABC$ là $S=\dfrac{1}{2}BC\cdot AH=\dfrac{1}{2}\cdot 9\cdot 6{,}5=29{,}25$ km$^2$.\\
			Bán kính đường tròn ngoại tiếp tam giác $ABC$ là \[R=\dfrac{AB\cdot AC\cdot BC}{4S}=\dfrac{\sqrt{48{,}5}\cdot 6{,}5\sqrt{2}\cdot 9}{4\cdot 29{,}25}=\dfrac{\sqrt{97}}{2}.\]
			Vậy $100AI^2= 100\cdot \dfrac{97}{4}=2\,425$ km$^2$.
		}
	\end{ex}
	%%%==============HetBai_BT3==============%%%
	\Closesolutionfile{ans}


